\documentclass[usenames,dvipsnames,pdftex,unicode]{beamer}
\usepackage{ucs}
\usepackage[utf8x]{inputenc}
\usepackage[english,russian]{babel}
\usepackage{algorithmic}
\usepackage{ulem}

\usepackage{tikz}
\usetikzlibrary{positioning,arrows,shapes}

\usetheme{Warsaw}
\usecolortheme{seahorse}

\setbeamercovered{transparent}

\title[Анализ указателей и синонимов]{
  Анализ указателей и синонимов для многопоточных программ
}
\author[Владимир Парфиненко]{
  Владимир Парфиненко
  \texorpdfstring{%
    \\ \small Научный руководитель: м.\,н.\,с.~ИСИ~СО~РАН, Павлов\,П.\,Е.
  }{}% short title in pdf
}
\institute{
  Новосибирский Государственный Университет
}
\date{
  Новосибирск, 2011
}


% добавить номер фрейма в footline
\newcommand*\oldmacro{}
\let\oldmacro\insertshorttitle
\renewcommand*\insertshorttitle{
  \oldmacro\hfill
  \insertframenumber\,/\,\inserttotalframenumber
}

\newcommand{\algorithmictitle}[1]{\hspace{8mm}\textbf{#1}}

\newcommand{\remark}[1]{\textcolor{Green}{#1}}
\newcommand{\todo}[1]{\textcolor{red}{TODO: #1}}
\newcommand{\todocite}{[\todo{cite}]}

\newcommand{\eng}[1]{{\English#1}}
\newcommand{\engdef}[1]{(англ.~\eng{#1})}

\newcommand{\notmatter}[1]{\only<1>{#1}\only<2->{\textcolor{Gray}{#1}}}
\newcommand{\pts}[1]{\overline{#1}}
\newcommand{\cupe}{\,\cup\!\!=}

\begin{document}

\begin{frame}
  \titlepage
\end{frame}

\begin{frame}{Статический анализ}

  \begin{itemize}
    \item \textbf{Оптимизирующие компиляторы}\\
        GCC, MSVC, ICC, Borland C++, MinGW, \ldots
    \item \textbf{Инструменты статического анализа}\\
        FindBugs, SourceAnalyzer, ReSharper, Static~Security~Analyzer, \ldots
  \end{itemize}

\end{frame}

\begin{frame}{Вынесение инварианта из цикла}

  \begin{visibleenv}<2>
    \begin{center}
      \large А что если $x$ и $y$ указывают на один объект?
    \end{center}
  \end{visibleenv}

  \begin{columns}[t]
    \begin{column}{5cm}
      \begin{block}{Исходный код}
        \begin{algorithmic}
          \FOR{$i = 0 \textbf{ to } n$}
          \STATE $x.f = i$
          \STATE $t = y.f$
          \STATE $a[i] = t$
          \ENDFOR
        \end{algorithmic}
      \end{block}
    \end{column}
    \begin{column}{5cm}
      \begin{block}{Преобразованный код}
        \begin{algorithmic}
          \STATE $t = y.f$
          \FOR{$i = 0 \textbf{ to } n$}
          \STATE $x.f = i$
          \STATE $a[i] = t$
          \ENDFOR
        \end{algorithmic}
      \end{block}
    \end{column}
  \end{columns}

\end{frame}

\begin{frame}{Оптимизаторы и анализаторы}

  \tikzstyle{optimizer} = [rectangle, draw, thin, fill=blue!10,%
                         minimum height=2em, rounded corners=1mm]
  \tikzstyle{analyzer} =  [rectangle, draw, thin, fill=blue!20,%
                         minimum height=2em, rounded corners=1mm]

  \begin{figure}
    \tikzstyle{every node}=[font=\small]

    \tikzstyle{optimizer} = [rectangle, draw, thin, fill=red!15,
                        minimum height=2em, rounded corners=1mm,
                        text width=2.9cm, text centered]
    \tikzstyle{analyzer} =  [rectangle, draw, thin, fill=yellow!20,
                        minimum height=2em, rounded corners=1mm,
                        text width=2.9cm, text centered]

    \begin{tikzpicture}[node distance=0.5cm, auto,>=latex', thick]
      \path[->]
                node[optimizer] (o1) {Выделение общих подвыражений}
                node[optimizer] (o2) [right=of o1] {Протяжка констант}
                node[optimizer] (o3) [right=of o2] {Вынесение инвариантов из цикла}
                node[analyzer]  (a1) [below=2cm of o1] {Escape-анализ}
                node[analyzer]  (a2) [right=of a1, line width=0.5mm] {Анализ синонимов}
                node[analyzer]  (a3) [right=of a2] {Анализ потоков управления};
      \path[->]
                (o1) edge (a2)
                (o2) edge (a1) edge (a3)
                (o3) edge (a2) edge (a3)
                (a1) edge (a2);
    \end{tikzpicture}
  \end{figure}

\end{frame}

\begin{frame}{Анализ синонимов и указателей}

  \begin{block}{Анализ синонимов \engdef{Alias Analysis}}
    Могут ли два разных выражения ссылочного типа указывать на одно и то же
    место в памяти?
  \end{block}

  \begin{block}{Анализ указателей \engdef{Points-To Analysis}}
    На какие объекты в памяти могут указывать выражения ссылочного типа?
  \end{block}

\end{frame}

\begin{frame}{Многопоточность}

  \begin{itemize}
    \item<1-> Алгоритму анализа указателей необходимо знать, какие значения
          могут быть прочитаны из полей объектов
    \item<1-> Проблемы с получением значений поля в многопоточных программах:
          \begin{itemize}
            \item<2-> поле может изменяться другими потоками
            \item<3-> изменение поля может быть не видно другим потокам
          \end{itemize}
  \end{itemize}

\end{frame}

\begin{frame}{Цель и задачи}

  \begin{itemize}
    \item \textbf{Цель:}
          разработка внутрипроцедурного алгоритма анализа указателей и
          внутреннего представления для использования в статическом
          оптимизирующем Java компиляторе Excelsior RVM
    \item \textbf{Основные задачи:}
      \begin{itemize}
        \item изучить существующие алгоритмы и выбрать подходящий
        \item адаптировать для анализа Java программ
        \item адаптировать для анализа многопоточных программ
        \item реализовать алгоритм в компиляторе Excelsior RVM
      \end{itemize}
  \end{itemize}

\end{frame}

\begin{frame}{Существующие алгоритмы}

  \begin{itemize}
    \item \textbf{Алгоритм Андерсена}\\более точный, временная сложность $O(N^3)$
    \item \textbf{Алгоритм Стинсгарда}\\менее точный, временная сложность $O(N
          \alpha(N))$,\\где $\alpha(N)$~--- обратная функция Аккермана, очень
          медленно растущая, можно принять за константу.
  \end{itemize}

\end{frame}

\begin{frame}{Идея алгоритма}

  $\pts{expr}$~--- множество объектов, на которые может указывать выражение
  $expr$

  \[
    \text{Присваивание }lhs = rhs
  \]
  \[
    \text{До: }
    \pts{lhs} = \{\textcolor{blue}{O_1, O_2}\}, \pts{rhs} =
    \{\textcolor{blue}{O_2, O_3}\}
  \]
  \[
    \text{После: }
    \pts{lhs} = \{\textcolor{blue}{O_1, O_2, O_3}\}, \pts{rhs} =
    \{\textcolor{blue}{O_2, O_3}\}
  \]

\end{frame}

\begin{frame}{C $\neq$ Java}

  \begin{columns}[t]
    \begin{column}{5cm}
      {\Large Язык C}

      $\textbf{int } *x = (\textbf{int } *)37 + 5;$

      \begin{itemize}
        \item есть адресная арифметика
        \item указатели на примитивы, объекты, \ldots
        \item объекты на стеке, в куче
        \item нестрогая типизация
      \end{itemize}
    \end{column}
    \begin{column}{5cm}
      {\Large Язык Java}

      $\textbf{Object } x = \textbf{new Object}();$

      \begin{itemize}
        \item нет адресной арифметики
        \item указатели только на объекты
        \item объекты только в куче
        \item строгая типизация
      \end{itemize}
    \end{column}
  \end{columns}

\end{frame}

\begin{frame}{Java Memory Model}
  \begin{columns}[t]
    \begin{column}{5cm}
      \begin{block}{Чтение из \eng{volatile} поля}
        \begin{algorithmic}
          \STATE $a.x = 42$
          \STATE $\ldots$
          \STATE $t = b.volatile\_field$
          \STATE $\ldots$
          \STATE $x = a.x$
        \end{algorithmic}
      \end{block}
    \end{column}
    \begin{column}{5cm}
      \begin{block}{Чтение из обычного поля}
        \begin{algorithmic}
          \STATE $a.x = 42$
          \STATE $\ldots$
          \STATE $t = b.normal\_field$
          \STATE $\ldots$
          \STATE \only<1>{$x = a.x$}
                 \only<2>{\sout{$x = a.x$} $\Rightarrow$ $x = 42$}
        \end{algorithmic}
      \end{block}
    \end{column}
  \end{columns}
\end{frame}

\begin{frame}{Внутреннее представление}
  \begin{columns}[t]
    \begin{column}{5cm}
      \begin{block}{Отрывок Java-программы}
        \begin{algorithmic}
          \STATE \notmatter{$\textbf{int } x = 42, y = 0;$}
          \STATE $\textbf{Object } a = \textbf{new A}();$
          \STATE $\notmatter{y} = foo(\notmatter{x}, a);$
          \STATE \notmatter{$\textbf{if }(y == 37) \{$}
            \STATE $\text{\quad}a.f = \textbf{null};$
          \STATE \notmatter{$\}\textbf{ else }\{$}
            \STATE $\text{\quad}o = \textbf{new B}();$
            \STATE $\text{\quad}a.f = o;$
          \STATE \notmatter{$\}$}
          \STATE \notmatter{$\textbf{return } a;$}
        \end{algorithmic}
      \end{block}
    \end{column}
    \begin{column}{5cm}
      \begin{visibleenv}<3->
      \begin{block}{Внутреннее представление}
        \small
        \def\lhs{\textbf{lhs}\colon}
        \def\rhs{\textbf{rhs}\colon}
        \begin{algorithmic}
          \STATE $assignments = [$
          \STATE $\{\lhs Var(a), \rhs Alloc(A)\},$
          \STATE $\{\lhs null, \rhs Call(foo, \ldots)\},$
          \STATE $\{\lhs Field(a, f), \rhs Null()\},$
          \STATE $\{\lhs Var(o), \rhs Alloc(B)\},$
          \STATE $\{\lhs Field(a, f), \rhs Var(o)\},$
          \STATE $]$
        \end{algorithmic}
      \end{block}
      \end{visibleenv}
    \end{column}
  \end{columns}
\end{frame}

\begin{frame}{Практические результаты}

  Алгоритм реализован в статическом Java компиляторе и будет проведено
  сравнение следующих его модификаций:
  \begin{itemize}
      \item алгоритм анализа не адаптированный для многопоточных программ
      \item алгоритм анализа адаптированный для многопоточных программ
      \item алгоритм анализа адаптированный для многопоточных программ с
            учетом семантики final полей
  \end{itemize}
\end{frame}

\begin{frame}{Заключение}

  Сделано следующее:
  \begin{itemize}
    \item проведен анализ существующих алгоритмов
    \item изучены особенности языка Java
    \item изучена спецификация модели памяти языка Java
    \item разработан алгоритм анализа указателей и внутреннее представление,
          адаптированные для анализа многопоточных Java программ
    \item предложенный алгоритм реализован в рамках статического
          оптимизирующего Java компилятор Excelsior RVM
    \item проведены практические замеры производительности и точности алгоритма
  \end{itemize}

\end{frame}

\begin{frame}

  \centering\huge

  Спасибо!

  Вопросы?

\end{frame}

\end{document}
