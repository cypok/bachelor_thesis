\documentclass[12pt]{article}
\usepackage{ucs}
\usepackage[utf8x]{inputenc}
\usepackage[english]{babel}
\usepackage{indentfirst}

\usepackage[left=3cm,right=3cm,top=3cm,bottom=3cm,bindingoffset=0cm]{geometry}
\linespread{1.3}

\newcommand{\eng}[1]{{\English#1}}

\title{
  Points-to analysis and alias analysis for multithreaded programs.
  Vladimir Parfinenko.
}
\author{
  Vladimir Parfinenko
}

\begin{document}

  \thispagestyle{empty}

  \begin{center}
    \bfseries
    Points-to analysis and alias analysis\\
    of multithreaded programs.\\
    Vladimir Parfinenko.
  \end{center}
  \vspace{0.5cm}

  Hello. My name is Vladimir Parfinenko, I am the first year master student of
  the Department of Mechanics and Mathematics of Novosibirsk State University.
  My scientific advisor is Pavel Pavlov, researcher in the Institute of
  Informatics Systems. And the theme of my work is "Point-to analysis and alias
  analysis of multithreaded programs".

  Static code analysis is widely used in optimizing compilers and in a software
  model checking tools. Points-to analysis is a static code analysis technique
  that establishes which pointers, or heap references, can point to which
  variables or storage locations. Alias analysis is similar to points-to
  analysis, it computes pairs of expressions that may be aliased (point to the
  same location).

  Classical points-to analysis algorithms were developed in early 1990s for
  C-based languages. I was given a task to develop a points-to analysis
  algorithm for Java language which appreciably differs from C language.
  Also analysis algorithm should be adapted for analysing multithreaded
  programs because nowadays such programs become increasingly popular
  due to omnipresence of multicore processors.

  During my work classical points-to analysis algorithms for analysis were
  reviewed and compared. Taking the algorithm proposed by Lars Ole Andersen, a
  new algorithm suitable for the efficient analysis of multithreaded programs
  in Java was developed. This required a careful study of the Java Language
  Specification and the Java Memory Model.

  The developed algorithm was implemented within the optimizing ahead-of-time
  Java compiler Excelsior Research Virtual Machine. In practical experiments
  time complexity and space complexity were measured. Also the comparison of
  precision with other analysis algorithms was performed. The developed
  algorithm has demostrated satisfactory performance in practice and a
  significant increase in accuracy compared to the algorithms not adapted for
  analysing of multithreaded programs in Java.

\end{document}

