\documentclass[12pt]{article}
\usepackage{ucs}
\usepackage[utf8x]{inputenc}
\usepackage[english,russian]{babel}
\usepackage{indentfirst}

\usepackage[left=3cm,right=3cm,top=3cm,bottom=3cm,bindingoffset=0cm]{geometry}
\linespread{1.3}

\newcommand{\eng}[1]{{\English#1}}

\title{
  Аннотация к квалификационной работе на соискание степени бакалавра студента 4
  курса физического факультета Парфиненко Владимира Владимировича.
  Тема~--- «Анализ указателей и синонимов для многопоточных программ»
}
\author{
  В.\,В.\,Парфиненко, кафедра АФТИ~ФФ~НГУ, гр.\,7305\\
  Руководитель П.\,Е.\,Павлов, м.\,н.\,с.~ИСИ~СО~РАН
}

\begin{document}

  \thispagestyle{empty}

  \begin{center}
    \bfseries
    Аннотация к квалификационной работе\\
    на соискание степени бакалавра\\
    студента 4 курса физического факультета\\
    Парфиненко Владимира Владимировича.\\
    Тема~--- «Анализ указателей и синонимов\\
    для многопоточных программ».
  \end{center}
  \vspace{0.5cm}

  Статический анализ программ активно применяется в оптимизирующих компиляторах
  и в инструментах статического анализа. Одним из видов статического анализа и
  является анализ указателей, классические алгоритмы для проведения которого
  были разработаны еще в 1990-х годах для \eng{C}-подобных языков. В данной
  работе ставилась задача разработать алгоритм анализа указателей для
  языка \eng{Java}, который имеет существенные отличия от языка \eng{C}.
  Также алгоритм анализа должен быть адаптирован для анализа многопоточных
  программ, что актуально в данное время, в связи с повсеместным
  распространением многоядерных процессоров.

  В работе проведен анализ и сравнение классических алгоритмов анализа
  указателей и синонимов. Взяв за основу алгоритм, предложенный \eng{Lars Ole
  Andersen}, был разработан новый алгоритм, пригодный для эффективного анализа
  многопоточных программ на языке \eng{Java}, что потребовало тщательного
  изучения спецификации языка \eng{Java} и его модели памяти.

  Алгоритм реализован в рамках оптимизирующего статического \eng{Java}
  компилятора \eng{Excelsior Research Virtual Machine}. В ходе экспериментов
  было сделано следующее: измерена практическая сложность алгоритма
  (временная и емкостная) и проведено сравнение точности с другими алгоритмами
  анализа. Разработанный алгоритм на практике продемонстрировал
  удовлетворительную производительность и существенный прирост точности по
  сравнению с алгоритмами, не адаптированными для анализа многопоточных
  \eng{Java}-программ.

  \vspace{0.5cm}

  \begin{flushright}

    студент 4 курса Физического Факультета

    Парфиненко\,В.\,В.

  \end{flushright}

\end{document}

