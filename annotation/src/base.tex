\documentclass[12pt]{article}
\usepackage{ucs}
\usepackage[utf8x]{inputenc}
\usepackage[russian]{babel}
\usepackage{indentfirst}

\title{
  «Анализ указателей и синонимов для многопоточных программ»\\
  Аннотация
}
\author{
  В.\,В.\,Парфиненко, кафедра АФТИ~ФФ~НГУ, гр.\,7305\\
  Руководитель П.\,Е.\,Павлов, м.\,н.\,с.~ИСИ~СО~РАН
}

\begin{document}
  \maketitle

  \thispagestyle{empty}

    Современные оптимизирующие компиляторы для доказательства корректности
    каких-либо преобразований программ используют результаты
    статического анализа. Одним из таких анализов является анализ
    указателей и синонимов, который позволяет определить, какие указатели
    указывают на какие объекты в памяти, и могут ли два выражения указывать
    на один объект.  

    Данный анализ требует адаптации для применения к многопоточным
    программам, которые в наше время получают широкое распространение среди
    прикладных программ из-за распространения многоядерных систем.

    Целью данной работы является реализация анализа программ,
    способного предоставлять следующую информацию о двух выражениях
    ссылочного типа:
    \begin{itemize}
      \item указывают ли они на один объект;
      \item могут ли они указывать на один объект;
      \item указывают ли они на разные объекты.
    \end{itemize}
    
    Для решения поставленной задачи необходимо спроектировать алгоритм
    анализа и внутреннее представление и реализовать их в виде компоненты
    для оптимизирующего статического компилятора Java программ.

\end{document}

