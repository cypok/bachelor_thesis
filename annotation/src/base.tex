\documentclass[10pt]{article}
\usepackage{ucs}
\usepackage[utf8x]{inputenc}
\usepackage[russian]{babel}
\usepackage{indentfirst}

\title{
  «Анализ указателей и синонимов для многопоточных программ»\\
  Аннотация
}
\author{
  В.\,В.\,Парфиненко, кафедра АФТИ~ФФ~НГУ, гр.\,7305\\
  Руководитель П.\,Е.\,Павлов, м.\,н.\,с.~ИСИ~СО~РАН
}

\begin{document}
  \maketitle

  \thispagestyle{empty}

    Современные оптимизирующие компиляторы для доказательства корректности
    каких-либо преобразований программ используют результаты
    статического анализа.
    Одним из видов анализа является анализ указателей и синонимов, который
    позволяет определить, какие указатели указывают на какие объекты в памяти,
    и, в частности, предоставляет следующую информацию о двух выражениях
    ссылочного типа:
    \begin{itemize}
      \item указывают ли они на один объект;
      \item могут ли они указывать на один объект;
      \item указывают ли они на разные объекты.
    \end{itemize}

    Целью данной работы является изучение существующих алгоритмов анализа
    указателей, последующее проектирование алгоритма и
    внутренного представления для использования в оптимизирующем
    статическом компиляторе Java программ
    Excelsior Research Virtual Machine (Excelsior~RVM)
    с учетом двух следующих требований.

    Во-первых,
    проектирование алгоритма анализа программ исполняемых в управляемых средах
    (таких как Java Virtual Machine (JVM) и платформа Microsoft .NET)
    накладывает дополнительные ограничения на алгоритм в связи с наличием
    строгой типизации и размещением большинства объектов в куче.

    Во-вторых, необходима адаптация алгоритма анализа
    для применения к многопоточным программам,
    которые в данный момент получают широкое распространение
    из-за появления многоядерных процессоров практически
    на каждом персональном компьютере.

    Для решения поставленной в работе задачи необходимо спроектировать алгоритм
    анализа и внутреннее представление и реализовать их в виде компоненты
    для оптимизирующего статического компилятора Java программ.

\end{document}

