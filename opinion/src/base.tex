\documentclass[12pt]{article}
\usepackage{ucs}
\usepackage[utf8x]{inputenc}
\usepackage[english,russian]{babel}
\usepackage{indentfirst}
\usepackage{xspace}

\usepackage[left=3cm,right=3cm,top=2.5cm,bottom=2.5cm,bindingoffset=0cm]{geometry}
\linespread{1.3}

\newcommand{\eng}[1]{{\English#1}}

\def\parf{В.\,В.\,Парфиненко\xspace}

\title{
  Отзыв научного руководителя о квалификационной работе на соискание степени
  бакалавра студента 4 курса физического факультета Парфиненко Владимира
  Владимировича.
  Тема~--- «Анализ указателей и синонимов для многопоточных программ»
}
\author{
  Руководитель П.\,Е.\,Павлов, м.\,н.\,с.~ИСИ~СО~РАН
}

\begin{document}

  \thispagestyle{empty}

  \begin{center}
    \bfseries
    Отзыв научного руководителя\\
    о квалификационной работе на соискание степени бакалавра\\
    студента 4 курса физического факультета\\
    Парфиненко Владимира Владимировича.\\
    Тема~--- «Анализ указателей и синонимов\\
    для многопоточных программ».
  \end{center}
  \vspace{0.5cm}

  Анализ указателей является важным видом статического анализа программ и
  используется при проведении большого числа оптимизаций. Задача
  квалификационной работы \parf заключалась в разработке внутрипроцедурного
  алгоритма анализа указателей для языка \eng{Java} и последующей его
  реализации в рамках оптимизирующего статического компилятора \eng{Excelsior
  Research Virtual Machine}.

  Для решения поставленной задачи \parf провел анализ существующих алгоритмов
  анализа указателей, выявив их сильные и слабые стороны. Также была подробно
  изучена спецификация языка \eng{Java} и, в частности, его модели памяти. Как
  результат, дипломник разработал эффективный алгоритм анализа и
  внутреннее представление, учитывающие особенности, возникающие при анализе
  многопоточных \eng{Java}-программ.

  \parf выполнил реализацию предложенного алгоритма в рамках оптимизирующего
  статического компилятора. Это позволило провести измерения производительности
  алгоритма на большом наборе тестов, в ходе которых была продемонстрирована
  его эффективность.
  Также были проведены сравнительные измерения точности, которые убедительно
  свидетельствуют о превосходстве разработанного алгоритма над алгоритмами,
  разработанными для анализа \eng{C}-подобных языков.

  В процессе работы дипломник проявил отличные знания теории компиляции и
  продемонстрировал личные качества, такие как целеустремленность и хорошую
  работоспособность.

  Квалификационная работа безусловно заслуживает оценки <<отлично>>, а студент
  присуждения степени бакалавра. Также \parf рекомендуется для обучения в
  магистратуре.

  \vspace{0.2cm}

  \begin{flushright}

    м.\,н.\,с.~ИСИ~СО~РАН

    Павлов\,П.\,Е.

  \end{flushright}

\end{document}

