\documentclass[12pt]{article}
\usepackage{ucs}
\usepackage[utf8x]{inputenc}
\usepackage[english,russian]{babel}
\usepackage{indentfirst}

\textwidth=16cm
\oddsidemargin=0pt
\topmargin=0pt
\headheight=0pt
\headsep=0pt
\textheight=24cm
\linespread{1.3}

\title{
  Обзор литературы на тему \\
  «Анализ указателей и синонимов\\ для многопоточных программ»
}
\author{
  Владимир Парфиненко
}

\begin{document}
  \maketitle

    \paragraph{Lars Ole Andersen.}
      {\English
        {\it Program Analysis and Specialization for
          the C Programming Language.}
        PhD thesis, Department of Computer Science, University of Copenhagen,
        May 1994.
      }

      Довольно большая работа, из которой я изучил Главу 4 «Pointer Analysis»
      (с.~111--152). В ней описываются основные характеристики, которыми
      может обладать алгоритм анализа, влияющие на его скорость и точность.

      Суммарный ({\English summary, flow-insensitive}) анализ предоставляет
      результат верный для любой точки программы, в то время как точечный
      ({\English program-point specific, flow-sensitive}) вычисляет результат
      для каждой отдельной точки программы, что значительно повышает точность
      анализа, но понижает производительность.

      Также в работе показывается, что меж-процедурные алгоритмы повышают
      точность анализа, но сами по себе сложнее и требуют больше ресурсов по
      сравнению с внутри-процедурными алгоритмами.

      Затем автор переходит к описанию своего меж-процедурного
      алгоритма анализа, вводя понятие абстракного места и
      абстрактного указателя~--- отношение из абстрактных мест в множество
      абстракных мест. В этих терминах он формулирует алгоритм анализа, который
      состоит в построении системы ограничений и дальнейшем ее решении.

      {\it Вот как именно строится эта система, и как она решается я
      не понял :(}


    \paragraph{Bjarne Steensgaard.}
      {\English
        {\it Points-to analysis in almost linear time.}
        Microsoft Research,
        March 1995.
      }

      Автор описывает другой алгоритм, характерной чертой которого является
      практически линейное время работы. Также формулируется система типов
      и ограничений на нее, с описанием алгоритма решения.

      Особенность алгоритма заключается в том, что присваивание $x = y$
      накладывает ограничение на полное равенство типов,
      $\tau_x = \tau_y$ ({\English equality-based}),
      в то время как алгоритм представленный в работе [Andersen] требует лишь,
      чтобы один из типов был подмножеством другого,
      $\tau_x \supset \tau_y$ ({\English subset-based}).
      За счет такого ограничения алгоритм становится менее точным, но более
      быстрым с точки зрения производительности.

    \paragraph{Derek Rayside.}
      {\English
        {\it Points-To Analysis.}
        MIT CSAIL 6.883, Prof Ernst,
        November 14, 2005.
      }

      Очень небольшая статья, в которой автор разделяет разные алгоритмы
      анализа на группы, и на примерах показывает сильные и слабые стороны
      {\English equality-based} и {\English subset-based} алгоритмов.
      {\it Довольно проблематично привести эти примеры из-за
      их большого размера.}


\end{document}

