\documentclass[a4,14pt,titlepage]{extarticle}
\usepackage{ucs}
\usepackage[utf8x]{inputenc}
\usepackage[english,russian]{babel}
\usepackage{indentfirst}
\usepackage{color}

\title{
  Анализ указателей и синонимов для многопоточных программ
}
\author{
  Владимир Парфиненко
}

% поля и размер текста
\textwidth=17cm
\oddsidemargin=-8pt
\topmargin=0pt
\headheight=0pt
\headsep=0pt
\textheight=23cm
\linespread{1.3}

%\newcommand{\remark}[1]{}
\newcommand{\remark}[1]{\textcolor{blue}{#1}}
\newcommand{\todo}[1]{\textcolor{red}{TODO: #1}}

\newcommand{\eng}[1]{{\English#1}}

\begin{document}

  %%%%%%%%%%%%%%%%%%%%%%%%%%%%%%%%%%%%%%%%%%%%%%%%
  %           Титульная страница
  %%%%%%%%%%%%%%%%%%%%%%%%%%%%%%%%%%%%%%%%%%%%%%%%
  \thispagestyle{empty}
  \begin {center}
    Министерство образования и науки

    Российской Федерации

    Федеральное агентство по образованию

    \vspace{0.3cm}

    Новосибирский государственный университет

    \vspace{0.3cm}

    Физический факультет

    Кафедра Автоматизации Физико-Технических Исследований

    \vspace {40mm}

    %Выпускная квалификационная работа бакалавра
    Октябрьский отчет

    \vspace {10mm}

    Парфиненко Владимир Владимирович

    \vspace {5mm}

    \textbf{АНАЛИЗ УКАЗАТЕЛЕЙ И СИНОНИМОВ\\ ДЛЯ МНОГОПОТОЧНЫХ ПРОГРАММ}

    \vspace {20mm}

    {\raggedleft

    Научный руководитель

    м.\,н.\,с.~ИСИ~СО~РАН, Павлов\,П.\,Е.

    \vspace {50mm}

    Новосибирск 2010}
  \end {center}


  \tableofcontents

  %%%%%%%%%%%%%%%%%%%%%%%%%%%%%%%%%%%%%%%%%%%%%%%%
  %                  Введение
  %%%%%%%%%%%%%%%%%%%%%%%%%%%%%%%%%%%%%%%%%%%%%%%%
  \newpage
  \section{Введение}

    Средства оптимизации программ нужны прежде всего для получения высокой
    скорости исполнения программ или улучшения других характеристик программы.

    Под оптимизацией программ понимается модификация программы в семантически
    эквивалентную, но более эффективную относительно некоторого заданного
    критерия.
    Преобразование программы $A$ в программу $B$ эквивалентно (или корректно),
    если из того, что программа $A$ выполнима на некотором наборе данных,
    следует, что и $B$ также выполнима на этом наборе и дает тот же результат,
    что и $A$.
    В общем случае задача проверки эквивалентности двух программ неразрешима,
    и не существует алгоритма, который по данной программе находил бы
    эквивалентную ей и оптимальную относительно заданного критерия [6].
    Но существуют набор оптимизирующих преобразований, и эквивалентность
    каждого из них гарантирует эквивалентность их последовательного применения.

    Для доказательства корректности конкретного преобразования данной
    программы необходимо выполнение некоторых условий. Например, чтобы иметь
    возможность убрать генерацию некоторой части программы, мы должны быть
    уверены, что управление никогда не попадет в эту часть программы.
    Для получения подобной информации используют результаты статического
    анализа. Анализируя код программы, деляют выводы о тех или иных свойствах
    программы, необходимых для доказательства корректности преобразования.

    Также статический анализ применяется не только для доказательства
    корректности преобразований, но и в инструментах статического анализа,
    которые позволяют находить в коде программы потенциальные ошибки.

    Существует множество видов анализа, и одним из них является анализ
    указателей и синонимов.

  %%%%%%%%%%%%%%%%%%%%%%%%%%%%%%%%%%%%%%%%%%%%%%%%%
  %%        Описание предметной области
  %%%%%%%%%%%%%%%%%%%%%%%%%%%%%%%%%%%%%%%%%%%%%%%%%
  \newpage
  \section{Описание предметной области}
    \todo{описать}

    Одним из видов анализа является анализ указателей и синонимов, который
    позволяет определить, какие указатели указывают на какие объекты в памяти,
    и, в частности, предоставляет следующую информацию о двух выражениях
    ссылочного типа:
    \begin{itemize}
      \item указывают ли они на один объект;
      \item могут ли они указывать на один объект;
      \item указывают ли они на разные объекты.
    \end{itemize}


  %%%%%%%%%%%%%%%%%%%%%%%%%%%%%%%%%%%%%%%%%%%%%%%%
  %             Постановка задачи
  %%%%%%%%%%%%%%%%%%%%%%%%%%%%%%%%%%%%%%%%%%%%%%%%
  \newpage
  \section{Постановка задачи}
    \todo{поставить}

    Целью данной работы является изучение существующих алгоритмов анализа
    указателей, последующее проектирование алгоритма и
    внутренного представления для использования в оптимизирующем
    статическом компиляторе Java программ
    \eng{Excelsior Research Virtual Machine (Excelsior~RVM)}
    с учетом двух следующих требований.

    Во-первых,
    проектирование алгоритма анализа программ исполняемых в управляемых средах
    (таких как \eng{Java Virtual Machine (JVM)} и платформа
    \eng{Microsoft .NET}) накладывает дополнительные ограничения на алгоритм в
    связи с наличием строгой типизации и размещением большинства объектов в
    куче.

    Во-вторых, необходима адаптация алгоритма анализа
    для применения к многопоточным программам,
    которые в данный момент получают широкое распространение
    из-за появления многоядерных процессоров практически
    на каждом персональном компьютере.

    Для решения поставленной в работе задачи необходимо спроектировать алгоритм
    анализа и внутреннее представление и реализовать их в виде компоненты
    для оптимизирующего статического компилятора Java программ.


  %%%%%%%%%%%%%%%%%%%%%%%%%%%%%%%%%%%%%%%%%%%%%%%%
  %             Список литературы
  %%%%%%%%%%%%%%%%%%%%%%%%%%%%%%%%%%%%%%%%%%%%%%%%
  \newpage
  \section{Список литературы}
    \todo{use bibtex}
    \paragraph{[1] Lars Ole Andersen.}
      \eng{
        \textit{Program Analysis and Specialization for
          the C Programming Language.}
        PhD thesis, Department of Computer Science, University of Copenhagen,
        May 1994.
      }

    \paragraph{[2] Bjarne Steensgaard.}
      \eng{
        \textit{Points-to analysis in almost linear time.}
        Microsoft Research,
        March 1995.
      }

    \paragraph{[3] Derek Rayside.}
      \eng{
        \textit{Points-To Analysis.}
        MIT CSAIL 6.883, Prof Ernst,
        November 14, 2005.
      }

    \paragraph{[4] Marc Shapiro, Susan Horwitz.}
      \eng{
        \textit{Fast and Accurate Flow-Insensitive Points-To Analysis.}
        Computer Sciences Department, University of Wisconsin-Madison,
        1997.
      }

    \paragraph{[5] John Whaley, Martin Rinard.}
      \eng{
        \textit{Compositional Pointer and Escape Analysis for Java Programs.}
        Laboratory for Computer Science, MIT,
        1999.
      }
    \paragraph{[6] Касьянов В.\,Н., Поттосин И.\,В.}
      \textit{Методы построения трансляторов.}
      Новосибирск: Наука,
      1986.

\end{document}

