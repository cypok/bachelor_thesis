\documentclass[14pt,titlepage]{extarticle}
\usepackage[pdftex,unicode,hidelinks]{hyperref}
\usepackage{ucs}
\usepackage[utf8x]{inputenc}
\usepackage[english,russian]{babel}
\usepackage{indentfirst}
\usepackage[usenames,dvipsnames]{color}
\usepackage{algorithm}
\usepackage{algorithmic}
\usepackage{amsmath}
\usepackage{amssymb}
\usepackage{mathtools}
\usepackage{multicol}
\usepackage{rotating}
\usepackage{array}
\usepackage{multirow}
\usepackage{xspace}
\usepackage{gnuplot-lua-tikz}
\usepackage{subfig}
\usepackage{amsthm}

% todo : Change geometry: left=3cm,right=2cm !
\usepackage[left=2.5cm,right=2.5cm,top=2cm,bottom=2cm,bindingoffset=0cm]{geometry}
\linespread{1.3}

\usepackage{numprint}
\newcommand{\num}[1]{\numprint{#1}}
  \npthousandsep{\,}
  \npthousandthpartsep{}
  \npdecimalsign{,}

\usepackage{tikz}
\usetikzlibrary{positioning,arrows,shapes}

\usepackage{footmisc}
\renewcommand\footnotelayout{\small}

\usepackage{etoolbox}
\apptocmd{\sloppy}{\hbadness 10000\relax}{}{}

% used by bibliography:
\usepackage{datetime}
\newcommand{\usedate}[3]{({\Russian дата обращения: \formatdate{#1}{#2}{#3}})}
\bibliographystyle{gost780u}

%\setcounter{tocdepth}{2} % глубина оглавления

\floatname{algorithm}{Пример}
\newcommand{\algorithmictitle}[1]{\hspace{8mm}\textbf{#1}}
\renewcommand{\algorithmicrequire}{\textbf{Дано:}}
\renewcommand{\algorithmiccomment}[1]{// #1}
\newcommand{\NEW}{\textbf{new }}
\newcommand{\NEWi}[1]{\textbf{new}_{#1}\textbf{ }}
\newcommand{\NULL}{\textbf{null }}
\newcommand{\BOOLTRUE}{\textbf{true }}
\newcommand{\FUNCTION}{\textbf{function }}

\newcommand{\Type}[1]{\textrm{Type}(#1)}
\newcommand{\IsAssignable}[2]{\textrm{IsAssignable}(#1, #2)}
\newcommand{\Pts}[1]{\textrm{Pts}(#1)}
\newcommand{\VPts}[1]{\textrm{VarPts}(#1)}
\newcommand{\OFPts}[2]{\textrm{ObjectFieldPts}(#1, #2)}
\newcommand{\SharedOFPts}[2]{\textrm{SharedObjectFieldPts}(#1, #2)}
\newcommand{\Filter}[2]{\textrm{Filter}_{#1}(#2)}
\newcommand{\cupe}{\,\cup\!\!=}

\let\oldphi\phi
\renewcommand{\phi}{\ensuremath{\oldphi}}

\renewcommand{\leq}{\leqslant}
\renewcommand{\geq}{\geqslant}
\newcommand{\incomp}{\not\lessgtr}

%\newcommand{\remark}[1]{}
%\newcommand{\todo}[1]{}
%\newcommand{\todocite}{}
\newcommand{\remark}[1]{\textcolor{Green}{(#1)}}
\newcommand{\todo}[1]{\textcolor{red}{(\eng{TODO}: #1)}}
\newcommand{\todocite}{[\textcolor{red}{\eng{cite}}]}

\newcommand{\eng}[1]{{\English#1}}
\newcommand{\engdef}[1]{(англ.~\eng{#1})}

\addto\captionsrussian{
  \let\oldrefname\refname
  \renewcommand\refname{\addcontentsline{toc}{section}{\oldrefname}\oldrefname}
}

% русские нумераторы
\renewcommand{\theenumii}{(\asbuk{enumii})}
\renewcommand{\labelenumii}{\asbuk{enumii})}
\renewcommand{\thesubfigure}{\asbuk{subfigure}}


% обёртка с моими настройками поверх figure:
% \begin{myfigure}{подпись}{label} ... \end{myfigure}
\newenvironment{myfigure}[2]%
  {\pushQED{\caption{#1} \label{#2}} % push caption & label
   \begin{figure}[!htb]\centering } %
  {  \popQED % pop caption & label
   \end{figure}}
\newenvironment{myplot}[2]%
  {\pushQED{\caption{#1} \label{#2}} % push caption & label
   \begin{figure}[p]\centering\small } %
  {  \popQED % pop caption & label
   \end{figure}}

% вставка subfigure внутри myfigure:
% \subfigure[params]{подпись}{file}
\newcommand{\mysubfigure}[3][]{
  \subfloat[#2]{\label{fig:#3}\includegraphics[#1]{#3}}
}

\let\oldsection\section
\renewcommand{\section}{\newpage\oldsection}

\newcommand{\sectionwithoutnumber}[1]{
  \section*{#1}
  \addcontentsline{toc}{section}{#1}
}

\newcommand{\java}{\eng{Java}\xspace}

\title{
  Анализ указателей и синонимов для многопоточных программ
}
\author{
  Владимир Парфиненко
}

\begin{document}

  \thispagestyle{empty}
  \begin{center}

    Министерство образования и науки\\
    Российской Федерации

    \vspace{0.7cm}

    Государственное образовательное учреждение\\
    высшего профессионального образования\\
    «Новосибирский государственный университет» (НГУ)

    \vspace{0.7cm}

    Физический факультет

    \vspace{0.2cm}

    Кафедра автоматизации физико-технических исследований

    \vspace{1.2cm}

    Квалификационная работа на соискание\\
    степени бакалавра

    \vspace{0.2cm}

    Парфиненко Владимир Владимирович

    \vspace{1.5cm}

    \textbf{АНАЛИЗ УКАЗАТЕЛЕЙ И СИНОНИМОВ\\ ДЛЯ МНОГОПОТОЧНЫХ ПРОГРАММ}

    \vspace{2.5cm}

    \begin{flushright}

      Научный руководитель

      м.\,н.\,с.~ИСИ~СО~РАН, Павлов\,П.\,Е.

    \end{flushright}

    \vspace {4cm}

    Новосибирск~--- 2011~год
  \end{center}

  \tableofcontents

  \sectionwithoutnumber{Введение}

    Средства оптимизации программ нужны для получения высокой
    скорости их исполнения или улучшения других характеристик.

    Под оптимизацией программы понимается ее преобразование в
    семантически эквивалентную, но более эффективную относительно некоторого
    заданного критерия.
    Преобразование программы $A$ в программу $B$ эквивалентно (или корректно),
    если из того, что программа $A$ выполнима на некотором наборе данных,
    следует, что и $B$ также выполнима на этом наборе и дает тот же результат,
    что и $A$.
    В общем случае задача проверки эквивалентности двух программ неразрешима,
    и не существует алгоритма, который по данной программе находил бы
    эквивалентную ей и оптимальную относительно заданного
    критерия~\cite{kasjanov_translators}.

    Тем не менее существует набор известных оптимизирующих преобразований,
    таких, что корректность каждого из них гарантирует корректность их
    последовательного применения.
    И чтобы конкретное преобразование данной программы было корректным,
    необходимо выполнение некоторых условий. Например, чтобы иметь
    возможность убрать генерацию некоторой части программы, нужно быть
    уверенным, что управление никогда не попадет в эту часть программы.
    Для получения подобной информации используют результаты статического
    анализа. Анализируя код программы, делаются выводы о тех или иных свойствах
    программы, необходимых для проведения преобразования.

    Статический анализ применяется не только для проверки
    корректности преобразований, но и в инструментах статического анализа
    кода, которые могут находить потенциальные ошибки и определять
    другие свойства программы без ее непосредственного исполнения.

    Существует множество видов статического анализа, в данной работе
    рассматриваются анализ указателей и анализ синонимов.

  \section{Анализ указателей и синонимов}

    Анализ указателей~--- это один из видов статического анализа, который
    позволяет определить, на какие объекты в памяти могут указывать выражения
    ссылочного типа в программе (такие объекты называются целями выражения
    ссылочного типа). Анализ синонимов похож на анализ указателей: его целью
    является определение, могут ли два разных выражения ссылаться на одно и
    то же место в памяти (такие выражения называют синонимами)~\cite{andersen}.

    \subsection{Классификация алгоритмов анализа указателей}
    \label{section:analysis_classification}

      Разные алгоритмы анализа указателей отличаются по точности, по скорости
      работы, по количеству памяти, необходимой для проведения анализа.
      Далее будут рассмотрены основные параметры алгоритмов, влияющие на эти
      характеристики.

      Для сравнения точности алгоритмов анализа указателей необходимо ввести
      некую меру точности. В качестве простой меры точности алгоритма часто
      используется усредненное количество синонимов для переменных ссылочного
      типа, появляющихся в
      программе~\cite[раздел~3.2]{hind_pointer_analysis_not_solved_yet}
      (также существуют и более изощренные
      меры~\cite{hind_pointer_analysis_not_solved_yet,diwan_tbaa}).

      Понятно, что для «идеального» алгоритма анализа это число будет
      минимальным, а для самого консервативного алгоритма максимальным.

      \subsubsection{Чувствительные к потоку данных алгоритмы}
        \label{section:analysis_classification_data_flow}

        Примером алгоритма, не учитывающего потоки данных в программе, может
        служить алгоритм анализа синонимов, основанный на типах, применимый
        для языков со строгой типизацией (подробнее про строгую систему типов в
        разделе~\ref{section:type_system})
        Простейшая реализация такого алгоритма дает следующий результат:
        независимо от контекста и потоков данных в программе, два выражения
        могут быть синонимами, если они имеют совместимые формальные
        типы~\cite[раздел~2.2]{diwan_tbaa}.
        Такой алгоритм работает за константное время, но обладает очень низкой
        точностью.

        Большей точностью обладают алгоритмы анализа, которые учитывают потоки
        данных в программе.
        Например, если существует только одно присваивание вида $v = \NEW T()$,
        то можно гарантировать, что переменная $v$ может указывать только на
        объект, созданный этим оператором \eng{new}.
        С присваиванием вида $v_1 = v_2$ ситуация сложнее. Рассмотрим
        пример~\ref{code:data_flow}.
        \begin{algorithm}
          \caption{Сравнение алгоритмов \eng{subset-based} и
                                        \eng{equality-based} типов}
          \label{code:data_flow}
          \begin{algorithmic}[1]
            \STATE $b$ = \NEW T()
            \STATE $c$ = \NEW T()
            \STATE $a$ = $b$
            \STATE $a$ = $c$
          \end{algorithmic}
        \end{algorithm}

        Цели указателя, то есть множество объектов, на которые может указывать
        переменная $p$ (или любое другое выражение ссылочного типа), для
        удобства обозначим как $Pts(p)$ \engdef{points-to set}.
        Учитывая строки 1 и 2, для переменных $b$ и $c$ мы можем точно
        определить множество объектов, на которые они указывают:
        \[Pts(b) = \{O_1\}, Pts(c) = \{O_2\},\]
        где $O_1$ и $O_2$~--- уникальные объекты в куче, соответствующие
        операторам \eng{new} в строках 1 и 2.  То есть мы учли поток данных от
        оператора \eng{new}, создавшего новый объект, в переменную.

        Интерпретировать присваивание $a = b$ можно двумя способами,
        и алгоритмы анализа разбиваются на два типа по этому признаку:
        \begin{itemize}
          \item алгоритмы первого типа накладывают ограничение
                $Pts(a) \supset Pts(b)$
                (\eng{subset-based} алгоритмы)~\cite{andersen}
          \item алгоритмы второго типа накладывают ограничение
                $Pts(a) = Pts(b)$
                (\eng{equality-based} алгоритмы)~\cite{steensgaard}
        \end{itemize}
        В нашем примере \eng{subset-based} алгоритм получит, что
        \[Pts(a) = \{O_1, O_2\}, Pts(b) = \{O_1\}, Pts(c) = \{O_2\},\]
        а \eng{equality-based}
        \[Pts(a) = Pts(b) = Pts(c) = \{O_1, O_2\}.\]
        Первый тип алгоритмов более точен, хотя его время работы в худшем
        случае кубически зависит от размера анализируемой программы.
        Второй тип алгоритмов работает за практически линейное время\footnote{
          Временная сложность \eng{equality-based} алгоритма
          $O(N \alpha(N,N))$~\cite{steensgaard}, где $\alpha(N,N)$~--- обратная
          функция Аккермана. Она является очень медленно растущей, и при
          анализе асимптотики алгоритмов можно принять ее за константу.
        },
        предоставляя менее точные результаты.
        Каждый из этих алгоритмов имеет свою область применения, например,
        \eng{subset-based} алгоритм имеет смысл использовать при анализе
        отдельного метода программы, в то время как \eng{equality-based} лучше
        подходит для анализа всей программы целиком.


      \subsubsection{Чувствительные к потоку управления алгоритмы}
        \label{section:analysis_classification_control_flow}

        Также для повышения точности алгоритм анализа указателей может
        учитывать поток управления в программе.
        Рассмотрим пример~\ref{code:control_flow}.
        \begin{algorithm}
          \caption{Сравнение чувствительного и нечувствительного к потоку
                   управления алгоритмов}
          \label{code:control_flow}
          \begin{algorithmic}[1]
            \STATE $a$ = \NEW T()
            \STATE $b$ = \NEW T()
            \STATE $c$ = \NEW T()
            \STATE $a$ = $b$
            \STATE $b$ = $c$
            \STATE $c$ = $a$
          \end{algorithmic}
        \end{algorithm}

        Нечувствительный к потоку управления алгоритм анализа воспринимает
        программу как неупорядоченный набор операций.
        Для указанного примера такой алгоритм получит следующее
        \[Pts(a) = Pts(b) = Pts(c) = \{O_a, O_b, O_c\}.\]
        Этот результат не является особо точным, но зато он верен
        для любой точки программы.

        Чувствительный к потоку управления алгоритм получит следующие данные
        после 3-ей строки
        \[\textrm{строка 3}:
            Pts(a) = \{O_a\}, Pts(b) = \{O_b\}, Pts(c) = \{O_c\}.\]
        Далее, при анализе 4-ой строки, присваивание будет интерпретировано
        как строгое присваивание (подробнее в
        разделе~\ref{section:flow_sensetive_analysis}), и алгоритм получит
        результат, что $Pts(a) = \{O_b\}$. То есть присваивание нового значения
        в переменную уничтожило информацию о том, что $a$ может указывать на
        $O_a$. Продолжая подобные рассуждения до последней строки, получаются
        следующие результаты:
        \[\textrm{строка 6}:
            Pts(a) = \{O_b\}, Pts(b) = \{O_c\}, Pts(c) = \{O_b\}.\]
        Хотя такой алгоритм дает более точный результат, приходится хранить
        информацию о целях каждого указателя для каждой операции отдельно,
        что значительно увеличивает суммарный объем памяти, занимаемый
        результатами анализа.

      \subsubsection{Межпроцедурные алгоритмы}

        Алгоритм анализа может учитывать потоки данных не только внутри
        единственной анализируемой функции, но и между отдельными функциями,
        которые могут быть вызваны из анализируемой.
        Рассмотрим пример~\ref{code:interprocedural}.
        \begin{algorithm}
          \caption{Демонстрация работы межпроцедурного алгоритма}
          \label{code:interprocedural}
          \begin{algorithmic}[1]
            \STATE \FUNCTION foo($x$, $y$) \{
            \RETURN $x$
            \STATE \}
            \STATE
            \STATE $b$ = \NEW T()
            \STATE $c$ = \NEW T()
            \STATE $a$ = foo($b$, $c$)
          \end{algorithmic}
        \end{algorithm}

        Наша задача понять, чему равно $Pts(a)$.
        Алгоритмы анализа можно разделить на две категории, в зависимости от
        того, как они обрабатывают вызовы функций:
        \begin{itemize}
          \item межпроцедурные алгоритмы анализа могут сначала проанализировать
                тело вызываемой функции, и затем учесть результат при обработке
                вызова,
          \item внутрипроцедурные алгоритмы рассматривают вызов функции в
                наиболее консервативном предположении: может быть возвращен
                либо один из параметров, либо один из глобальных объектов.
        \end{itemize}
        Понятно, что первый тип алгоритмов дает более точные результаты,
        а второй потребляет меньше памяти и работает
        быстрее~\cite[с.~117]{andersen}.
        В нашем примере межпроцедурный алгоритм анализа, проанализировав
        функцию \eng{foo}, запоминает, что для нее выполнено следующее условие
        на возвращаемое значение
        \[Pts(retval) = Pts(x),\]
        и тогда может сделать вывод, что \[Pts(a) = Pts(b) = \{O_1\}.\]
        В такой же ситуации внутрипроцедурный алгоритм анализа обязан сделать
        консервативное предположение
        \[Pts(a) = Pts(b) \cup Pts(c) = \{O_1, O_2\}.\]

  \section{Описание предметной области}

    \subsection{Место алгоритмов анализа в компиляторе}

      Рассмотрим общую архитектуру оптимизирующего статического компилятора.

      \begin{myfigure}%
        {Архитектура оптимизирующего статического компилятора (оранжевые~---
         преобразователи, зеленые~--- представления, синие~--- оптимизаторы)}%
        {fig:arch}

        \tikzstyle{converter} = [rectangle, draw, thin, fill=orange!20,
                            minimum height=2em, rounded corners=3mm,
                            text width=1.2cm, text centered]

        \tikzstyle{representation} = [circle, draw, thin, fill=green!20,
                            minimum height=2em,
                            text width=1.2cm, text centered]

        \tikzstyle{optimizer} = [rectangle, draw, thin, fill=blue!10,
                            minimum height=2em,
                            text width=2.6cm, text centered]

        \begin{tikzpicture}[auto,>=latex', thick]
          \path[->] node[representation] (src) {SRC}
                    node                 (d1)  [right=0.75cm of src] {}
                    node[representation] (ast) [right=1.8cm of src] {AST}
                    node                 (d2)  [right=0.75cm of ast] {}
                    node[representation] (cfg) [right=1.8cm of ast] {CFG}
                    node                 (d3)  [right=0.75cm of cfg] {}
                    node[representation] (asm) [right=1.8cm of cfg] {ASM}
                    node                 (d4)  [right=0.75cm of asm] {}
                    node[representation] (obj) [right=1.8cm of asm] {OBJ}
                    node[optimizer] (high)   [below=1cm of ast] {High-level Optimizer}
                    node[optimizer] (middle) [below=1cm of cfg] {Middle-level Optimizer}
                    node[optimizer] (low)    [below=1cm of asm] {Low-level Optimizer}
                    node[converter] (c1)  [above=1.5cm of d1,text width=2.2cm] {Front-end}
                    node[converter] (c2)  [above=1.5cm of d2] {}
                    node[converter] (c3)  [above=1.5cm of d3,text width=2.2cm] {Code generator}
                    node[converter] (c4)  [above=1.5cm of d4] {}
                    (src) edge (c1)
                    (c1) edge (ast)
                    (ast) edge (c2)
                    (c2) edge (cfg)
                    (cfg) edge (c3)
                    (c3) edge (asm)
                    (asm) edge (c4)
                    (c4) edge (obj)
                    (ast) edge [<->] (high)
                    (cfg) edge [<->] (middle)
                    (asm) edge [<->] (low)
                    ;
        \end{tikzpicture}
      \end{myfigure}

      Такой компилятор представляет из себя конвейер (см.~рис.~\ref{fig:arch}):
      на входе он получает исходный код программы, на выходе отдает код целевой
      машины.
      Также существуют промежуточные представления, пригодные для работы
      оптимизаторов и необходимых им алгоритмов анализа.

      Для перевода программы из одного представления в другое существуют
      преобразователи.
      Самый первый преобразователь переводит исходный код программы
      или байт-код программы (в случае \java)
      во внутреннее представление, с которым удобно работать другим
      компонентам компилятора.
      Последний преобразователь выполняет перевод в самое низкоуровневое
      представление, содержащее данные и код, пригодный для исполнения
      на целевой машине.

      Оптимизации проводятся на одном из внутренних представлений, они
      модифицируют его с целью повышения эффективности исполнения программы
      относительно заданного критерия. Однако, эти преобразования должны быть
      корректными. Информацию о том, возможно ли проведение данного
      преобразования, преобразователи получают от анализаторов, вспомогательных
      компонент, которые проводят статический анализ программы и могут
      предоставлять ту или иную информацию о ее свойствах
      (см.~рис.~\ref{fig:optimizators}).

      \begin{myfigure}%
        {Оптимизаторы и анализаторы}%
        {fig:optimizators}

        \tikzstyle{optimizer} = [rectangle, draw, thin, fill=red!15,
                            minimum height=2em, rounded corners=1mm,
                            text width=3cm, text centered]
        \tikzstyle{analyzer} =  [rectangle, draw, thin, fill=yellow!20,
                            minimum height=2em, rounded corners=1mm,
                            text width=3cm, text centered]

        \begin{tikzpicture}[node distance=2.5cm, auto,>=latex', thick]
          \path[->] node[optimizer] (o1) {Common Subexpression Elimination}
                    node[optimizer] (o2) [right=of o1] {Another Optimization}
                    node[optimizer] (o3) [right=of o2] {Loop Invariant Code Motion};
          \path[->] node[analyzer]  (a1) [below=of o1] {Another Analysis}
                    node[analyzer]  (a2) [right=of a1, line width=0.5mm] {Alias Analysis}
                    node[analyzer]  (a3) [right=of a2] {Another Analysis}
                    (o1) edge (a1) edge (a2)
                    (o2) edge (a2) edge (a3)
                    (o3) edge (a2)
                    (a1) edge (a2);
        \end{tikzpicture}
      \end{myfigure}

      Рассмотрим оптимизацию выноса инвариантов цикла \engdef{loop invariant
      code motion} на примере~\ref{code:licm}.
      Для выноса какого-либо выражения из цикла, требуется убедиться, что оно
      не будет изменяться (является инвариантом цикла).
      В данном примере, кандидатом на вынесение из цикла является выражение
      $x.field$. Оптимизатору необходимо убедиться, что поле объекта, на
      который ссылается переменная $x$, не изменяется при выполнении других
      операций в теле цикла. В данном примере оптимизатору необходимо знать,
      могут ли переменные $x$ и $y$ быть синонимами. Если могут, то
      проводить преобразование нельзя, так как запись в $y.field$ может
      изменить значение $x.field$. Если же переменные точно не являются
      синонимами, то вынос $x.field$ из цикла является корректным
      преобразованием.

      \begin{algorithm}
        \caption{Вынесение инвариантов цикла}
        \label{code:licm}
        \begin{multicols}{2}
          \algorithmictitle{Исходный код}
          \begin{algorithmic}[1]
            \FOR{$i = 1$ to \textbf{length}($a$)}
            \STATE $y.field = func(i)$
            \STATE $a[i] = x.field$
            \ENDFOR
          \end{algorithmic}
          \columnbreak
          \algorithmictitle{Преобразованный код}
          \begin{algorithmic}[1]
            \STATE $temp = x.field$
            \FOR{$i = 1$ to \textbf{length}($a$)}
            \STATE $y.field = func(i)$
            \STATE $a[i] = temp$
            \ENDFOR
          \end{algorithmic}
        \end{multicols}
      \end{algorithm}

      Анализ синонимов, описанный в работе, позволяет отвечать на
      вопрос оптимизатора: «А могут ли две переменные ссылочного типа быть
      синонимами?»

    \subsection{Анализ многопоточных программ}
      \label{section:intro_to_multithreading}

      Компьютеры, в которых несколько процессоров взаимодействуют с
      использованием разделяемой памяти, разрабатываются с 1960-х годов, а
      сейчас они уже распространены повсеместно.

      С одной стороны, изначально доступ к общей памяти был строго
      последовательным для всех потоков, исполнявшихся на разных процессорах.
      Это приводило к тому, что два последовательных чтения в одном потоке из
      одного и того же места памяти могли давать разные результаты, так как в
      параллельном потоке могла быть совершена запись в эту же самую память в
      момент времени между последовательными чтениями в первом потоке.

      С другой стороны, со временем появились весьма изощренные техники
      повышения производительности работы с общей памятью: многоуровневые кэши,
      внеочередное исполнение \engdef{out-of-order execution} и другие.
      Это приводит к тому, что запись в память, произведенная одним потоком,
      может быть не видна другим. Или наоборот: запись может произойти раньше
      либо позже по сравнению со строго последовательным порядком исполнения.

      Рассмотрим пример~\ref{code:out_of_order_exec}, исполнение которого на
      современном процессоре может привести к неожиданному результату.
      Если изначально $obj.x = obj.y = 0$, то по окончанию работы примера
      значения переменных $x$ и $y$ могут равняться также нулю. Это связано с
      тем, что процессор при исполнении первого потока мог сначала выполнить
      вторую операцию, так как она не зависит от первой, аналогично при
      исполнении второго потока.

      \begin{algorithm}
        \caption{Нарушение логики программы при внеочередном исполнении}
        \label{code:out_of_order_exec}
        \begin{multicols}{2}
          \algorithmictitle{Поток 1}
          \begin{algorithmic}[1]
            \STATE $obj.x = 1$
            \STATE $y = obj.y$
          \end{algorithmic}
          \columnbreak
          \algorithmictitle{Поток 2}
          \begin{algorithmic}[1]
            \STATE $obj.y = 1$
            \STATE $x = obj.x$
          \end{algorithmic}
        \end{multicols}
      \end{algorithm}

      Получается, что если нет четкой семантики, определяющей, какие значения
      могут быть получены при чтении переменных из памяти, анализ указателей
      для них получится очень неточным\footnote{
        В такой ситуации алгоритму анализа придется делать очень консервативное
        предположение: считать, что все указатели, прочитанные из памяти, могут
        указывать на одни и те же данные и могут быть синонимами.
      } и не будет давать хоть сколько-нибудь полезной информации, что приведет
      к невозможности проведения большого количества оптимизирующих
      преобразований программы.

      На самом деле в большинстве систем есть определенные правила,
      регулирующие работу с памятью. Такие правила есть на уровне процессора,
      виртуальной машины и языка. Эти правила называют моделью памяти.
      Модель памяти для многопоточной системы определяет в каком
      порядке могут происходить доступы к памяти в программе и, как следствие,
      какие значения может возвращать конкретное чтение памяти. Соответственно,
      наличие четко определенной модели памяти позволяет разработать алгоритм
      анализа, пригодный для анализа многопоточных программ.

  \section{Постановка задачи}

    Целью данной работы является разработка внутрипроцедурного алгоритма
    анализа указателей и внутреннего представления для использования в
    оптимизирующем статическом компиляторе \java программ \eng{Excelsior
    Research Virtual Machine (Excelsior~RVM)}~\cite{excelsior_jet} с учетом
    приведенных ниже требований.

    Алгоритм анализа должен учитывать следующие особенности языка \java:
    \begin{itemize}
      \item наличие строгой типизации,
      \item указатели только на объекты в куче,
      \item отсутствие адресной арифметики.
    \end{itemize}
    Именно эти особенности отличают язык \java от языка C, для которого были
    разработаны классические алгоритмы анализа указателей
    Стинсгарда~\cite{steensgaard} и Андерсена~\cite{andersen}.

    В связи с широким распространением многопоточных программ, алгоритм анализа
    необходимо адаптировать согласно спецификации языка \java, которая имеет
    строгое и подробное описание модели памяти~\cite{manson_jmm}.

    Кроме того, внутреннее представление программы должно быть подходящим для
    эффективного проведения анализа и хранения его результатов с возможностью
    быстрого доступа к ним.

    Для достижения поставленной цели необходимо:
    \begin{itemize}
      \item изучить существующие алгоритмы анализа указателей,
      \item выбрать один из существующих алгоритмов анализа и, изучив
            спецификацию языка \java и его виртуальной машины, адаптировать
            алгоритм для анализа многопоточных программ на языке \java,
      \item разработать внутреннее представление программы для эффективной
            работы алгоритма анализа и хранения его результатов,
      \item реализовать алгоритм и внутреннее представление в рамках
            оптимизирующего статического компилятора \java программ
            \eng{Excelsior~RVM}.
    \end{itemize}

  \section{Внутреннее представление и SSA-форма}
    \label{section:ir_and_ssa}

    В данном разделе будут введены понятия внутреннего представления программы
    и SSA-формы программы, которые понадобятся в дальнейшем при описании
    алгоритма анализа указателей.

    В процессе компиляции программа на исходном языке программирования
    переводится в так называемое внутреннее представление
    \engdef{internal representation, IR}, c которым работают
    алгоритмы анализа и над котором проводятся оптимизации.
    Рассмотрим внутреннее представление тела функции, заданное в виде
    управляющего графа \engdef{control flow graph, CFG}, именно это
    внутреннее представление используется для проведения большинства
    оптимизаций в современных компиляторах~\cite{muchnick}.
    CFG~--- это ориентированный граф, в котором вершинам соответствуют
    последовательности операторов программы, а дугам~--- переходы из конца
    одной последовательности операторов в начало другой. Такие
    последовательности операторов, являющиеся вершинами, назовем линейными
    участками. В конце каждого линейного участка присутствует оператор
    перехода, который передает управление по одной из дуг, выходящих из
    данной вершины CFG.

    Будем говорить, что программа находится в SSA-форме (\eng{Static Single
    Assignment}), если существует не более одного присваивания в любую из
    переменных~\cite{ssa}.
    Для представления программ в SSA-форме переменные версионируются и
    вводится дополнительная операция слияния значений переменных, так
    называемая \phi-функция.

    Версии вводятся для переменных, которые имеют более одного присваивания.
    Участок программы вида
    \[ v = 1; \textrm{use}(v); v = 2; \textrm{use}(v); \]
    после версионирования будет выглядеть следующим образом
    \[ v_1 = 1; \textrm{use}(v_1); v_2 = 2; \textrm{use}(v_2); \]

    \phi-функции вводятся для слияния версий переменных, определяемых
    на разных путях исполнения. Пусть в CFG программы существует
    вершина $N$, такая что в нее входят дуги из вершин $N_1, \ldots, N_k$, в
    которых использовались версии $v_1, \ldots, v_k$ переменной $v$. Тогда в
    $N$ будет располагаться \phi-функция
    \[ v_x = \phi(v_1, \ldots, v_k). \]
    Семантика данной операции заключается в присваивании переменной $v_x$
    значения переменной $v_i$, соответствующего вершине $N_i$, из которой
    управление пришло в $N$. В случае программы, приведенной в
    примере~\ref{code:ssa_with_phi}, переменной $a_3$ будет присвоено значение
    5 или 7, в зависимости от того, из какой ветки условного блока придет
    исполнение.

    \begin{algorithm}
      \caption{Пример перевода программы в SSA-форму с \phi-функцией}
      \label{code:ssa_with_phi}
      \begin{multicols}{2}
        \begin{algorithmic}[1]
          \IF{\ldots}
            \STATE $a = 5$
          \ELSE
            \STATE $a = 7$
          \ENDIF
          \STATE $\textrm{use}(a)$
        \end{algorithmic}
        \columnbreak
        \begin{algorithmic}[1]
          \IF{\ldots}
            \STATE $a_1 = 5$
          \ELSE
            \STATE $a_2 = 7$
          \ENDIF
          \STATE $a_3 = \phi(a_1, a_2)$
          \STATE $\textrm{use}(a_3)$
        \end{algorithmic}
      \end{multicols}
    \end{algorithm}

    Любую программу можно перевести в SSA-форму посредством
    версионирования переменных и расстановки \phi-функций~\cite{ssa}.
    Для перевода в SSA-форму и вывода из нее существуют эффективные
    алгоритмы~\cite{bilardi_ssa, briggs_ssa}.

  \section{Алгоритм анализа указателей}
    \label{section:algorithm}

    В данном разделе будет описан разработанный алгоритм анализа указателей.
    Сначала будет определен тип алгоритма в соответствии с классификацией,
    представленной в разделе~\ref{section:analysis_classification}.
    Затем будет дано описание основной идеи алгоритма.
    После этого будут описаны особенности алгоритма, связанные с
    многопоточностью и системой типов языка.
    И наконец, будет описано использование алгоритма в рамках набора операций
    языка \java.

    \subsection{Тип алгоритма}

      Необходимо определиться с характеристиками алгоритма анализа,
      подходящего для нашей задачи. Описание основных характеристик уже было
      приведено в разделе~\ref{section:analysis_classification}.

      \subsubsection{Внутрипроцедурный алгоритм анализа}

        Хотя межпроцедурные алгоритмы анализа указателей и дают более точные
        результаты, они сложны для реализации, тем более для таких
        языков как \java, где большинство методов являются виртуальными.
        Реализация такого алгоритма анализа выходит за рамки моей
        квалификационной работы, в ней рассматривается внутрипроцедурный
        алгоритм. Однако отсутствие межпроцедурного анализа смягчается тем, что
        до вызова алгоритма анализа указателей может быть проведена
        оптимизация открытой подстановки методов.

      \subsubsection{\texorpdfstring{\eng{Subset-based} алгоритм анализа}
                                    {Subset-based алгоритм анализа}}

        Как уже было показано в
        разделе~\ref{section:analysis_classification_data_flow}
        алгоритмы анализа, не учитывающие потоки данных в анализируемой
        программе, обладают очень низкой точностью, что не является
        удовлетворительным для данной работы. Поэтому далее будем рассматривать
        только алгоритмы, учитывающие потоки данных.

        Среди алгоритмов, учитывающих потоки данных, также существует выбор
        между типом алгоритма: \eng{subset-based} или \eng{equality-based}.
        При таком выборе необходимо решить, что важнее в конкретном
        случае: точность или скорость работы, соответственно.
        Хотя время работы \eng{subset-based} алгоритма в худшем случае
        кубически зависит от размеров программы, а для \eng{equality-based}
        практически линейно, проведенные
        эксперименты~\cite{shapiro_fast_and_accurate} показывают, что на
        небольших программах (до \num{3000} строк), время работы обоих
        алгоритмов анализа примерно одинаково, но точность у \eng{subset-based}
        алгоритма значительно выше.

        Заметим, что отдельный метод программы разумно отнести к <<небольшим
        программам>>, поэтому при использовании внутрипроцедурного алгоритма
        \eng{subset-based} типа, мы получим весомый выигрыш в точности анализа
        при незначительных потерях во времени. Соответственно, в данной работе
        будет использоваться именно этот тип алгоритма.

      \subsubsection{Нечувствительный к потоку алгоритм анализа}
        \label{section:flow_sensetive_analysis}

        Рассмотрим другую характеристику алгоритма анализа указателей:
        чувствительность к потоку управления в программе.
        Алгоритм анализа, чувствительный к потоку управления, дает более точные
        результаты, при этом потребляя значительно больше ресурсов, таких как
        память и
        время~\cite[раздел.~4.4]{hind_pointer_analysis_not_solved_yet}.

        Напомним, что точность чувствительного к потоку алгоритма во многом
        обусловлена эффектом строгих присваиваний \engdef{strong update}.
        Строгое присваивание, по возможности заменяет текущий набор целей
        указателя на новый, а не просто расширяет его.
        Однако известно, что эффекта строгого присваивания для переменных
        верхнего уровня можно добиться, переведя программу в
        SSA-форму~\cite{points_to_with_efficient_strong_updates}.

        Продемонстрируем работу нечувствительного к потоку алгоритма на примере
        \ref{code:ssa_precision} (это пример \ref{code:control_flow},
        переведенный в SSA-форму).
        \begin{algorithm}
          \caption{Повышение точности за счет использования SSA-формы}
          \label{code:ssa_precision}
          \begin{algorithmic}[1]
            \REQUIRE $x_0, x_1, \ldots, x_i, \ldots$~--- версии одной переменной $x$
            \STATE $a_0$ = \NEW T()
            \STATE $b_0$ = \NEW T()
            \STATE $c_0$ = \NEW T()
            \STATE $a_1$ = $b_0$
            \STATE $b_1$ = $c_0$
            \STATE $c_1$ = $a_1$
          \end{algorithmic}
        \end{algorithm}

        Так как присутствует только одно присваивание каждой переменной, легко
        получается следующий результат:
        \[Pts(a_0) = \{O_a\}, Pts(b_0) = \{O_b\}, Pts(c_0) = \{O_c\},\]
        \[Pts(a_1) = \{O_b\}, Pts(b_1) = \{O_c\}, Pts(c_1) = \{O_b\}.\]
        Как было показано в разделе
        \ref{section:analysis_classification_control_flow},
        чувствительный к
        потоку алгоритм анализа дает идентичный результат для конечной точки
        примера \ref{code:control_flow} (учитывая, что для конечной точки
        программы $a = a_1$, $b = b_1$, $c = c_1$):
        \[Pts(a) = \{O_b\}, Pts(b) = \{O_c\}, Pts(c) = \{O_b\}.\]

        В итоге, в данной работе будет использован нечувствительный к потоку
        управления алгоритм. Этот выбор обусловлен, во-первых,
        неудовлетворительным потреблением ресурсов чувствительными к потоку
        алгоритмами и, во-вторых, сложностью реализации подобных алгоритмов.
        Недостаток точности будет отчасти нивелирован использованием SSA-формы
        программы, которая уже используется в большинстве современных
        компиляторов и в \eng{Excelsior~RVM} в том числе.

    \subsection{Основа алгоритма анализа}
      \label{section:algorithm_basis}

      В этом разделе описана основная идея внутрипроцедурного,
      нечувствительного к потоку управления алгоритма анализа указателей
      \eng{subset-based} типа, работающего с программой в SSA-форме. Основа
      алгоритма идентична внутрипроцедурному алгоритму анализа Андерсена для
      языка C, представленному в работе~\cite{andersen}.

      Алгоритм анализа работает с выражениями ссылочного типа, каждое из
      которых имеет множество целей. Это множество состоит из абстрактных
      объектов, которые соответствуют объектам кучи в анализируемой программе.

      Введем следующее обозначение для множества целей выражения $expr$:
      $\Pts{expr} = \{O_1, O_2, \ldots, O_n\}$, где $O_i$~--- абстрактные объекты.

      В рамках анализа указателей достаточно рассматривать только те операции
      анализируемой программы, которые модифицируют множества целей выражений.
      Такими операциями являются присваивания выражений ссылочного типа,
      которые имеют в общем случае вид:
      \[lhs\_expr = rhs\_expr.\]
      Подобное выражение накладывает следующее ограничение на множества целей
      выражений:
      \[\Pts{lhs\_expr} \cupe \Pts{rhs\_expr},\]
      где $a \cupe b$~--- расширение множества $a$ множеством $b$.
      Для анализируемого метода получается набор ограничений на цели выражений,
      и цель алгоритма анализа определить для каждого выражения набор целей,
      удовлетворяющий ограничениям.

      В данной работе используется следующий алгоритм: имея некоторые
      начальные значения, множества целей выражений итеративно расширяются в
      соответствии с каждым отдельным ограничением до тех пор, пока все
      ограничения не будут удовлетворены. Доказательство сходимости и
      корректности данного алгоритма анализа приведено Андерсеном в
      работе~\cite{andersen}.

      \subsubsection{Основные анализируемые выражения}
        \label{section:pts_providers}

        Далее представлены основные выражения ссылочного типа, с которыми будет
        работать разработанный алгоритм.

        \paragraph{Локальные переменные~---} это простейшие выражения
        ссылочного типа, которые хранят непосредственно множество абстрактных
        объектов с тривиальными операциями для доступа к этому множеству и его
        расширению. Введем обозначение $\VPts{v}$ для обозначения доступа к
        множеству целей переменной.

        \paragraph{Доступ к полям переменных} есть более сложная операция.
        Чтение и расширение множества целей этого выражения являются
        нетривиальными операциями.
        Нужно понимать, что множество целей поля переменной~--- это объединение
        множества целей полей объектов, на которые может указывать данная
        переменная:
        \[\Pts{v.field} = \bigcup\limits_{O \in \VPts{v}} \OFPts{O}{field},\]
        где $\OFPts{O}{field}$~--- множество целей поля абстрактного объекта,
        с тривиальными операциями для чтения и расширения.
        При расширении множества целей поля переменной необходимо аналогично
        расширить множество целей поля каждого объекта, на который может
        указывать переменная:
        \begin{gather*}
          \Pts{v.field} \cupe pts
          \Leftrightarrow \\ \Leftrightarrow
          \forall O \in \VPts{v}\quad \OFPts{O}{field} \cupe pts.
        \end{gather*}

        Заметим, что часть выражений может находиться лишь в правой части
        присваивания, и следовательно, их множество целей не может быть
        расширено другим выражением. Такие выражения перечислены далее.

        \paragraph{Операция создания объекта} является выражением, множество
        целей которого всегда содержит один уникальный абстрактный объект для
        каждой отдельной операции создания объекта:
        \[\Pts{\NEWi{i} \textrm{T}} = \{O_i\}.\]

        \paragraph{\phi-функция} является выражением, доступ к множеству целей
        которого возвращает объединение множества целей ее аргументов:
        \[\Pts{\phi(a_0, \ldots, a_n)} = \bigcup\limits_{i} \Pts{a_i}.\]

    \subsection{\texorpdfstring{Модель памяти языка \java}
                               {Модель памяти языка Java}}

      В этом разделе описана модель памяти, представленная в спецификации языка
      \java версии 5.0.
      Далее описаны особенности алгоритма анализа указателей для
      эффективного и корректного анализа программ, взаимодействующих через
      разделяемую память.

      Модели памяти различаются по тому, насколько сильные ограничения
      накладываются на последовательность исполнения операций чтения и
      записи.
      Модель памяти может быть очень строгой и требовать последовательного
      исполнения всех операций чтения и записи~\cite{lamport}.
      Такая модель сильно ограничивает набор используемых оптимизаций,
      так как многим из них требуется менять отдельные операции чтения и
      записи местами, а строгая модель памяти не позволит это сделать, даже
      если между операциями нет зависимости по управлению и данным.
      Слабая модель памяти может не определять какого-либо жесткого порядка
      исполнения операций. Основываясь на этой модели компилятор может
      довольно сильно преобразовывать программу с целью ее оптимизации,
      однако разработчику этой программы придется уделять очень много времени
      для написания корректной программы в рамках такой слабой модели памяти.

      Модель памяти, представленная в спецификации языка \java версии 5.0
      является компромиссом между возможностью проведения широкого класса
      оптимизаций и удобством разработки программ на языке \java. Подробное
      описание можно найти в спецификации JSR-133~\cite{jsr133}. Модель памяти
      является достаточно строгой и однозначно определяет, как будут
      исполняться корректно синхронизированные программы (программы, в которых
      отсутствуют состояния гонки\footnote{
        Состояние гонки \engdef{data-race} образуют две операции доступа к
        разделяемой памяти из разных потоков, если хотя бы одна из них~---
        запись, и эти операции не упорядочены в рамках модели памяти (подробнее
        см.~\cite[раздел~2.1]{manson_jmm}).
      } \engdef{data-race free}). Однако она является и достаточно слабой,
      позволяя проводить многие оптимизации.

      Рассмотрим семантику чтения и записи полей с учетом данной
      модели памяти (подробнее в~\cite{jsr133_cookbook}).

      Сначала определим, какие операции чтения или записи памяти разрешено
      переставлять компилятору и/или процессору
      (см.~табл.~\ref{tabular:can_reorder}).
      Допустимо переставлять две операции чтения или записи обычных полей, если
      между ними нет зависимости по данным. Но не позволяется переставлять две
      операции чтения или записи \eng{volatile} полей. Так же не позволяется
      переставлять две операции, где первая~--- чтение \eng{volatile} поля или
      где вторая~--- запись в \eng{volatile} поле.

      \begin{table}[!htb]
        \centering

        \def\yes{\multirow{2}{*}{}}
        \def\no{\multirow{2}{*}{Нет}}

        \begin{tabular}{ |c|c|c|c|c| }
          \cline{3-5}
          \multicolumn{2}{c|}{} & \multicolumn{3}{c|}{2-ая операция} \\
          \cline{3-5}
          \multicolumn{2}{c|}{} & Чтение/запись & Чтение              & Запись \\
          \multicolumn{2}{c|}{} & обычного поля & \eng{volatile} поля & \eng{volatile} поля \\
          \hline
          \multirow{6}{*}{\begin{sideways}1-ая операция\end{sideways}}
          & Чтение/запись       & \yes & \yes & \no \\
          & обычного поля       &      &      &     \\ \cline{2-5}
          & Чтение              & \no  & \no  & \no \\
          & \eng{volatile} поля &      &      &     \\ \cline{2-5}
          & Запись              & \yes & \no  & \no \\
          & \eng{volatile} поля &      &      &     \\ \hline
        \end{tabular}
        \caption{Возможность перестановки двух операций чтения или записи поля
                 объекта}
        \label{tabular:can_reorder}
      \end{table}

      Далее определим, какие значения могут возвращать операции чтения
      разделяемой памяти. Модель памяти гарантирует, что после записи значения
      в \eng{volatile} поле одним из потоков все последующие чтения этого поля
      вернут новое значение. Но подобное поведение не гарантируется при работе
      с обычными полями. Запись в обычное поле одним из потоков может быть не
      видна другим потокам при чтении этого поля. Это может происходить из-за
      наличия у каждого потока локальных копий обычных полей.  Однако модель
      памяти определяет, когда эти локальные копии обязаны быть
      синхронизированы с реальной разделяемой памятью: при записи в
      \eng{volatile} поле все локальные записи памяти должны быть отражены в
      разделяемой памяти, а при чтении \eng{volatile} поля все локальные данные
      должны быть заново прочитаны из разделяемой памяти.

      Операции языка \java для входа в и выхода из блока синхронизации
      \engdef{synchronized block} влияют на возможность перестановки операций и
      синхронизированность локальной памяти с разделяемой так же, как чтение и
      запись \eng{volatile} полей, соответственно.

      Данные правила можно пояснить на примере \ref{code:volatile_synch}.
      В первом потоке запись данных произойдет до установки флага готовности,
      так он является \eng{volatile}, и следовательно не может быть переставлен
      с предыдущей операцией. Также при установке флага готовности данные
      гарантированно будут записаны в разделяемую память.
      Во втором потоке после чтения \eng{volatile} флага данные будут прочитаны
      именно из разделяемой памяти, причем это произойдет после чтения флага,
      так как эти две операции не могут быть переставлены.
      Следовательно, если данные будут выставлены первым потоком, то именно
      они и будут прочитаны вторым потоком.
      \begin{algorithm}
        \caption{Синхронизация через \eng{volatile} переменную
          ($data$~--- обычное поле, $ready$~--- \eng{volatile} поле)}
        \label{code:volatile_synch}
        \begin{multicols}{2}
          \algorithmictitle{Поток 1}
          \begin{algorithmic}[1]
            \STATE $data$ = get\_data()
            \STATE $ready$ = \BOOLTRUE
          \end{algorithmic}
          \columnbreak
          \algorithmictitle{Поток 2}
          \begin{algorithmic}[1]
            \WHILE{ \textbf{not } $ready$ }
            \STATE \COMMENT{waiting}
            \ENDWHILE
            \STATE use\_data($data$)
          \end{algorithmic}
        \end{multicols}
      \end{algorithm}

      \subsubsection{Особенности анализа многопоточных программ}

        Для описания взаимодействия с разделяемой памятью введем специальное
        выражение ссылочного типа ${<}shared{>}$, которое представляет
        множество всех абстрактных объектов, разделяемых между потоками. Одной
        из его особенностей является то, что множество целей этого выражения
        изначально не пусто, а содержит в себе специальный объект
        $O_{global}$, соответствующий всем абстрактным объектам, созданным
        вне анализируемого метода (введение такого объекта соответствует
        консервативному предположению о том, что все объекты, созданные вне
        анализируемого метода, могут быть синонимами).

        Определим вспомогательное свойство анализируемого метода: присутствие
        операций, приводящих к перечитыванию полей объектов (т.\,е. операций
        чтения \eng{volatile} поля и входа в блок синхронизации). Если такие
        операции присутствуют, то согласно модели памяти языка \java, поля
        разделяемых объектов обязаны быть перечитаны из разделяемой памяти
        после каждой такой операции. Так как мы используем нечувствительный к
        потоку исполнения анализ, можно считать, что поля разделяемых объектов
        должны перечитываться при каждом доступе к ним. А это в свою очередь
        означает, что при чтении поля разделяемого объекта может быть получен
        любой из разделяемых объектов, так как другие потоки могли его туда
        записать.
        Для имитации такого поведения при доступе к множеству целей поля
        разделяемого объекта его собственное множество целей дополнительно
        расширяется множеством разделяемых объектов:
        \[\OFPts{O}{f} \cup \Pts{{<}shared{>}}.\]

        Так же необходимо помнить, что если объект является разделяемым, то и
        все объекты, на которые ссылаются его поля, тоже разделяемые.
        Соответственно, объект становится разделяемым при присваивании в поле
        другого разделяемого объекта.

        Если операций, приводящих к перечитыванию полей объектов, нет,
        анализируемой программе разрешено хранить локальные копии полей
        объектов, а алгоритму, соответственно, нет необходимости расширять их
        множество целей всеми разделяемыми объектами.

    \subsection{Система типов языка}
      \label{section:type_system}

      В начале описана строгая система типов на примере языка \java, затем
      представлены изменения в алгоритме с целью повышения его точности.

      В языке \java все типы делятся на примитивные и ссылочные.
      Примитивные типы можно не рассматривать в контексте анализа указателей,
      так как они не могут переносить информацию о целях указателей.
      Ссылочные типы~--- это классы, интерфейсы и массивы.

      На множестве ссылочных типов можно определить отношение частичного
      порядка, соответствующее совместимости по присваиванию:
      $A \leq B$ тогда и только тогда, когда значение типа $B$ можно присвоить
      в выражение с формальным типом $A$~\cite{nastia_type_analysis}.
      Также можно ввести отношение несовместимости двух типов, выполняющееся,
      если не существует типа, совместимого по присваиванию с ними обоими:
      \begin{equation}\label{eq:incomp}
        A \incomp B
        \Leftrightarrow
        \lnot \exists C\colon A \leq C \land B \leq C.
      \end{equation}

      Согласно спецификации языка \java, во время исполнения программы
      выражение ссылочного типа может указывать только на те объекты, тип
      которых совместим по присваиванию с формальным типом этого выражения:
      \begin{gather}
        O \in \Pts{expr}
        \Rightarrow
        \Type{expr} \leq \Type{O}, \\
        \label{eq:strict_typing}
        \Type{expr} \not\leq \Type{O}
        \Rightarrow
        O \not\in \Pts{expr}.
      \end{gather}

      Отсюда получается, что два выражения, имеющие несовместимые типы, не
      могут указывать на один и тот же объект:
      \begin{align*}
        \quad& \Type{a} \incomp \Type{b}
        \xRightarrow{(\ref{eq:incomp})} \\ \xRightarrow{(\ref{eq:incomp})}
        \lnot \exists C \quad& \Type{a} \leq C \land \Type{b} \leq C
        \Rightarrow \\ \Rightarrow
        \lnot \exists O \quad& \Type{a} \leq \Type{O}
                         \land \Type{b} \leq \Type{O}
        \Rightarrow \\ \Rightarrow
        \forall O \quad& \Type{a} \not\leq \Type{O}
                    \lor \Type{b} \not\leq \Type{O}
        \xRightarrow{(\ref{eq:strict_typing})} \\ \xRightarrow{(\ref{eq:strict_typing})}
        \forall O \quad& O \not\in \Pts{a}
                    \lor O \not\in \Pts{b}
        \Rightarrow \\ \Rightarrow
        \lnot \exists O \quad& O \in \Pts{a}
                         \land O \in \Pts{b}.
      \end{align*}

      \subsubsection{Использование информации о типах}

        Рассмотрим, какой тип имеют абстрактные объекты. Абстрактные объекты,
        соответствующие операциям создания объектов, имеют точный тип,
        определенный этой операцией.
        Специальный объект $O_{global}$ имеет искусственный тип, такой, что
        этот объект совместим по присваиванию с любым другим ссылочным типом.

        Все выражения ссылочного типа имеют формальный тип. Причем все цели
        конкретного выражения должны быть совместимы по присваиванию с его
        формальным типом.
        Отсюда следует, что при расширении множества целей выражения
        достаточно расширять его лишь теми объектами, которые совместимы по
        присваиванию с формальным типом выражения.
        Введем операцию $\Filter{T}{pts}$, которая фильтрует множество
        абстрактных объектов, оставляя только совместимые по присваиванию с
        типом $T$:
        \[ \Filter{T}{pts} =
           \bigl\{ O \bigm| O \in pts \land T \leq \Type{O} \bigr\}. \]
        Тогда операция расширения множества целей выражения $lhs$ множеством
        целей выражения $rhs$ может быть записана следующим образом:
        \[ \Pts{lhs} \cupe \Filter{\Type{lhs}}{\Pts{rhs}}. \]

        Также информация о типах может быть использована при проверке, могут ли
        два выражения быть синонимами. В первую очередь нужно проверить,
        совместимы ли их типы: если нет, то они точно не могут содержать ссылку
        на один и тот же объект и быть синонимами.

        Со строгой типизацией языка также связана операция преобразования типа
        $a = (T)b$.
        При таком преобразовании, если значение переменной $b$ не совместимо
        по присваиванию с типом $T$, будет выброшено исключение во время
        исполнения, поэтому в рамках алгоритма анализа можно считать, что
        выражение $(T)b$ имеет формальный тип $T$ и может указывать только на
        объекты, совместимые по присваиванию с типом $T$:
        \[\Pts{(T)b} = \Filter{T}{\Pts{b}}.\]

    \subsection{\texorpdfstring{Операции языка \java}
                               {Операции языка Java}}
      \label{section:operations}

      В этом разделе представлен список операций языка \java, которые прямо или
      косвенно влияют на цели указателей, и их интерпретация в рамках
      выражений, с которыми работает разработанный алгоритм анализа.

      Сначала определим точно, что может быть указателем в языке \java и
      подобных ему.

      В отличие от C-подобных языков, где допустимы указатели на:
      \begin{itemize}
        \item объекты в куче,
        \item объекты на стеке,
        \item локальные и глобальные переменные произвольного типа,
        \item поля объектов,
        \item элементы массивов,
        \item функции;
      \end{itemize}
      в \java-подобных языках целью указателя может быть только объект,
      находящийся в куче. Также отсутствует адресная арифметика (операции
      взятия адреса, чтения и записи по адресу), за счет чего ссылка на объект
      может появиться только по цепочке присваиваний, начиная с создания этого
      объекта в куче.

      При проведении анализа указателей достаточно рассматривать лишь
      операции, приведенные в таблице \ref{tabular:operations}.

      \begin{table}
        \centering

        \begin{tabular}{|l|p{0.65\textwidth}|}\hline
          \textbf{Операция} & \textbf{Описание}\\ \hline

          $a = b$
          & присваивание значения $b$ переменной $a$ \\ \hline

          $a = \NULL$
          & указание, что $a$ не указывает ни на один объект \\ \hline

          $a_0 = \phi(a_1, a_2, \ldots)$
          & \phi-функция, которая появляется в связи с переводом программы в
            SSA-форму \\ \hline

          $a = \NEW \textrm{T}$
          & создание нового объекта типа $\textrm{T}$ в куче \\ \hline

          $a = b.f$
          & чтение поля объекта, на который ссылается $b$ \\ \hline

          $b.f = a$
          & запись в поле объекта, на который ссылается $b$ \\ \hline

          $a = \textrm{T}.f$
          & чтение статического поля класса $\textrm{T}$ \\ \hline

          $\textrm{T}.f = a$
          & запись в статическое поле класса $\textrm{T}$ \\ \hline

          $a = b$[\ldots]
          & чтение элемента массива $b$ \\ \hline

          $b[\ldots] = a$
          & запись в элемент массива $b$ \\ \hline

          $a = (\textrm{T})b$
          & преобразование значения переменной $b$ к типу $\textrm{T}$ \\ \hline

          $a = \textrm{foo}(b, \ldots)$
          & вызов функции foo, возвращающей значение ссылочного типа \\ \hline

          $\textrm{foo}(b, \ldots)$
          & вызов функции foo, либо не возвращающей значение, либо возвращающей
            значение примитивного типа \\ \hline

          $\textrm{synchronized}(a) \{\ldots\}$
          & вход и выход в блок синхронизации $a$ \\ \hline
        \end{tabular}
        \caption{Операции языка \java, влияющие на цели указателей
                 ($a$, $b$, $c$~--- переменные или формальные параметры
                 ссылочного типа)}
        \label{tabular:operations}
      \end{table}

      Рассмотрим подробнее чтение и запись элементов массива. Зачастую
      алгоритм анализа не может определить реальных значений индексов
      элементов, по которым происходит доступ к элементам массива, так как их
      значения будут доступны лишь во время исполнения программы.
      По этой причине алгоритм анализа не может различить доступ к отдельным
      элементам массива без привлечения дополнительного анализа
      диапазонов~\cite[раздел~8.12]{muchnick}, который выходит за рамки данной
      работы.
      Соответственно будем интерпретировать доступ к любому элементу массива,
      как доступ к одному единственному элементу, что является корректным
      поведением с точки зрения анализа, хотя и понижает его точность.

      \subsubsection{\texorpdfstring{Интерпретация операций языка \java}
                                    {Интерпретация операций языка Java}}

        В этом разделе для всех операций языка \java, достаточных для
        проведения анализа, приведена интерпретация с использованием выражений,
        описанных в предыдущих разделах.
        \begin{itemize}
          \item Присваивание вида $a = \NULL$ никак не влияет на $\Pts{a}$.
          \item Чтение \eng{volatile} поля объекта или вход в
                блок синхронизации задает, что в анализируемом методе
                присутствует операция, приводящая к перечитыванию полей
                объектов.
          \item Доступ к статическим полям класса $T$ осуществляется через
                синтетическую переменную $klass\_T$, которая определяется с
                помощью присваивания $klass\_T = {<}shared{>}$.
          \item Доступ к формальным параметрам анализируемого метода
                осуществляется через синтетические переменные $param\_i$ с
                соответствующим формальным типом.
                Определяются они с помощью набора присваиваний вида
                $param\_i = {<}shared{>}$.
          \item Доступ к элементам массива $a[\ldots]$ превращается в доступ к
                единственному полю $elements$, синтетически добавляемому к
                каждому типу-массиву.
          \item Вызов функции с параметрами $p_0, p_1, \ldots, p_n$
                преобразуется в серию присваиваний вида ${<}shared{>} = p_i$ и
                задает, что в анализируемом методе присутствуют операции,
                приводящие к перечитыванию полей объектов. Это соответствует
                консервативному предположению, что вызываемая функция может
                исполнять совершенно произвольный код.
          \item Возврат значения из функции $lhs = \textrm{func}(\ldots)$
                интерпретируется как присваивание $lhs = {<}shared{>}$.
        \end{itemize}
        После такого преобразования все операции будут иметь вид $lhs = rhs$,
        где в левой и правой части стоит одно из выражений, описанных в
        предыдущих разделах.

  \section{Вспомогательное внутреннее представление}
    \label{section:analysis_aux_ir}

    Большинство оптимизаций, которым требуются результаты анализа указателей,
    работают с внутренним представлением программы в виде CFG (это верно для
    большинства современных оптимизирующих компиляторов, для
    \eng{Excelsior~RVM} в том числе). Соответственно, компонента для проведения
    анализа указателей должна использовать то же внутреннее представление.

    Однако для эффективного проведения какого-либо анализа нередко требуется
    введение вспомогательного внутреннего представления, которое сохраняет в
    себе лишь ту часть семантики программы, которая ему необходима.
    В данной работе было решено использовать вспомогательное внутреннее
    представление для проведения анализа указателей и хранения его результатов.

    Далее будет описано построение объектно-ориентированной модели
    вспомогательного внутреннего представления.

    \subsection{Абстрактные объекты и их множества}

      В первую очередь, необходимо ввести структуру данных, соответствующую
      объекту, на который может указывать выражение ссылочного типа. Назовем
      такой тип данных абстрактным объектом (\eng{AbstractObject}). Так как
      любой объект анализируемой программы является экземпляром некоторого
      класса, структура данных \eng{AbstractObject} также будет иметь ссылку
      \eng{type} на класс данного объекта.

      Введем структуру данных, соответствующую множеству целей указателя
      (\eng{Points\-To\-Set}). Эта структура данных инкапсулирует
      множество абстрактных объектов, предоставляя операции для объединения двух
      таких множеств и проверки наличия абстрактных объектов, общих для двух
      множеств.

      Добавим, что у объектов в анализируемой программе присутствуют поля
      ссылочного типа, которые также имеют множество целей. Для сохранения
      этой семантики структура данных \eng{AbstractObject} хранит в себе
      ассоциативный массив, в котором каждому нестатическому полю класса объекта
      сопоставляется его множество целей (\eng{PointsToSet}).

      Отдельно необходимо ввести структуру данных, которая соответствует
      специальному объекту $O_{global}$. Такая структура данных~--- это
      \eng{Everything\-Abstract\-Object}, наследуемая от \eng{AbstractObject}.
      У нее есть две особенности:
      \begin{enumerate}
        \item Ссылка \eng{type} указывает на специальный тип \eng{Everything},
              совместимый по присваиванию с любым другим ссылочным типом.
        \item Ассоциативный массив с множеством целей полей является
              динамическим. Будучи изначально пустым, он может быть расширен
              по мере необходимости, за счет чего может хранить множество целей
              полей произвольных классов анализируемой программы.
      \end{enumerate}

    \subsection{Поставщики целей указателя}

      В рамках разработанного алгоритма анализа указателей все выражения
      ссылочного типа рассматриваются как сущности, которые имеют некоторое
      множество целей и формальный тип.
      Данная идея реализуется во внутреннем представлении посредством введения
      интерфейса, назовем его поставщиком целей указателя
      (\eng{PointsToSetProvider}).
      Интерфейс определяет методы для доступа непосредственно к множеству целей
      ($getPointsToSet$) и типу соответствующего выражения ($getType$).

      Все структуры данных внутреннего представления, соответствующие выражениям
      ссылочного типа анализируемой программы, реализуют интерфейс поставщика
      целей указателей.
      Локальной переменной в анализируемой программе соответствует структура
      данных \eng{Variable}, результату создания объекта~--- \eng{Allocation} и
      так далее.
      Все структуры данных хранят в себе тип соответствующего выражения и набор
      целей, который может быть изменяемым набором, например, у локальной
      переменной, или вычислимым из других наборов, например, у \phi-функции
      (подробнее про правила вычисления наборов целей выражений, использованных
      в алгоритме анализа, в разделе \ref{section:pts_providers}).

    \subsection{Набор присваиваний}

      В разделе \ref{section:algorithm_basis} уже говорилось, что разработанный
      алгоритм работает с анализируемой программой, как с набором присваиваний,
      в которых в левой и правой части стоят выражения ссылочного типа. Во
      внутреннем представлении эта идея реализуется посредством хранения
      массива присваиваний (\eng{Assignment}), которые содержат в себе ссылки
      на два поставщика целей указателей, соответствующих выражениям в левой и
      правой части присваивания.

    \subsection{Использование разработанного представления}

      В примере \ref{code:aux_ir} показан фрагмент \java программы и
      соответствующее вспомогательное внутреннее представление (выраженное в
      псевдокоде).
      Можно увидеть, как именно операции присваивания языка \java преобразуются
      в структуры данных, используемые в данной работе.

      \begin{algorithm}
        \caption{Вспомогательное внутреннее представление}
        \label{code:aux_ir}
        \begin{multicols}{2}
          \algorithmictitle{\java код}
          \begin{algorithmic}
            \STATE $\textbf{class }\textmd{A }%
                    \textbf{extends }\textmd{Object }\{$
            \STATE $\textmd{\quad Object }f;$
            \STATE $\}$
            \STATE $\ldots$
            \STATE $\{$
            \STATE $\textmd{\quad Object }x = \textbf{new }\textmd{A}();$
            \STATE $\textmd{\quad A }y = (\textmd{A})x;$
            \STATE $\textmd{\quad}y.f = \textbf{new }\textmd{Object}();$
            \STATE $\}$
          \end{algorithmic}
          \columnbreak
          \algorithmictitle{Его представление}
          \begin{algorithmic}
            \STATE $vx = \textmd{Var}\{\textmd{var}: \textmd{x},%
                              \textmd{type}: \textbf{Object}\}$
            \STATE $vy = \textmd{Var}\{\textmd{var}: \textmd{y},%
                              \textmd{type}: \textbf{A}\}$
            \STATE $assignments = [$
            \STATE $\textmd{\ }\{ \textmd{lhs}: vx,$
            \STATE $\textmd{\ \, rhs}: \textmd{Alloc}\{\textmd{type}: \textbf{A}\}\}$
            \STATE $\textmd{\ }\{ \textmd{lhs}: vy,$
            \STATE $\textmd{\ \, rhs}: \textmd{Cast}\{\textmd{target}: \textbf{A}, \textmd{var}: vx\}\}$
            \STATE $\textmd{\ }\{ \textmd{lhs}: \textmd{FieldAcc}\{\textmd{var}: vy, \textmd{field}: \textbf{f}\},$
            \STATE $\textmd{\ \, rhs}: \textmd{Alloc}\{\textmd{type}: \textbf{Object}\}\}$
            \STATE $]$
          \end{algorithmic}
        \end{multicols}
      \end{algorithm}

      Как можно заметить, при введении такого вспомогательного внутреннего
      представления, становится удобно проводить анализ указателей.
      Все операции, достаточные для проведения анализа, представляют из себя
      присваивание выражения, стоящего справа, выражению, стоящему слева.
      Причем этим выражениям анализируемой программы соответствуют структуры
      данных, реализующие интерфейс поставщика целей указателя. Это позволяет
      проводить анализ указателей, абстрагируясь от реального вида выражений,
      работая лишь с их типами и множествами целей.

      Также, имея такое вспомогательное внутреннее представление, можно легко
      определять, являются ли два выражения синонимами: достаточно получить
      экземпляры структур данных, соответствующие этим выражениями, проверить
      совместимость их типов и наличие общих абстрактных объектов в их
      множествах целей.

  \section{Реализация}

    Изначально был реализован тестовый стенд, изолированное окружение для
    проведения экспериментов. С его помощью можно проверять работу
    разных алгоритмов анализа указателей на характерных примерах, сравнивать их
    характеристики. Изолированность позволяет быстро разработать такую систему
    и легко вносить в нее изменения, так как не требуется интеграция с
    существующими интерфейсами оптимизирующего компилятора.

    Затем уже алгоритм анализа был реализован в рамках статического
    оптимизирующего \java компилятора.

    \subsection{Реализация тестового стенда}

      Реализована система, являющаяся простой моделью компилятора. Она включает
      в себя парсер простого \java-подобного языка, структуру для хранения
      внутреннего представления и компоненту, выполняющую анализ указателей.
      Также присутствуют методы просмотра разнообразной статистики: множества
      целей всех указателей, присутствующих в программе, точности результатов,
      времени работы алгоритма.

      Система принимает на входе программу в специальном формате через
      стандартный поток ввода. В начале программы идут определения классов в
      стиле \java с поддержкой наследования и полей. Затем задается тело метода,
      для которого и будет проводится анализ указателей.
      Тело метода задается в SSA-форме и состоит из набора операций,
      достаточных для проведения анализа указателей (подробнее в разделе
      \ref{section:operations}).

      Парсер на выходе получает AST (\eng{Abstract Syntax Tree}), из которого
      инициализируется иерархия классов и внутреннее представление
      единственного метода, которое передается в анализатор для проведения
      анализа указателей. В результате проведения анализа указателей
      предоставляются результаты в виде наборов целей указателей и другая
      статистика работы алгоритма.

      Система реализована на языке \eng{Ruby}, так как он позволяет очень
      быстро создать работающее приложение, для которого скорость работы не
      является критической характеристикой. Парсер осуществляет разбор входной
      программы с использованием регулярных выражений, что так же является
      простым и быстрым в исполнении решением. Внутреннее представление метода
      реализовано в соответствии с вспомогательным внутренним представлением,
      описанным в разделе \ref{section:analysis_aux_ir}.
      При инициализации этого внутреннего представления также проверяется
      корректность всех операций.

      Парсер и модуль инициализации внутреннего представления имеют хорошее
      покрытие модульными тестами \engdef{unit test}. Разные варианты
      алгоритма анализа указателей также имеют модульные тесты для тестирования
      особенностей конкретного алгоритма.

    \subsection{Реализация в рамках компилятора}

      Разработанный алгоритм анализа был реализован в оптимизирующем
      компиляторе в виде отдельной компоненты на языке \eng{Oberon} 2.

      Работа компоненты разбита на два этапа: инициализация и предоставление
      результатов. Так как пользователю чаще всего необходимо проведение
      анализа синонимов для нескольких пар выражений одного метода, такое
      разбиение является естественным: при таком подходе инициализация
      внутреннего представления и анализ указателей будут проведены один раз
      для всего метода, а затем может следовать сколь угодно много запросов о
      синонимичности выражений.

      Для инициализации компоненты требуется внутреннее представление
      анализируемого метода программы в виде CFG. На его основе строится
      вспомогательное внутреннее представление в соответствии с разделом
      \ref{section:analysis_aux_ir}. Затем проводится анализ указателей для
      всего метода программы (описание алгоритма приведено в разделе
      \ref{section:algorithm}) и его результаты сохраняются во вспомогательном
      внутреннем представлении.

      После такой инициализации компонента может предоставлять информацию о
      синонимичности пар выражений ссылочного типа. Для этого необходимо
      определить элементы вспомогательного внутреннего представления,
      соответствующие данным выражениям, и проверить их синонимичность.
      Данная процедура может быть повторена многократно для
      различных пар выражений, если внутреннее представление данного метода при
      этом не изменялось. Если же по каким-либо причинам внутреннее
      представление метода было преобразовано, необходимо заново проводить
      инициализацию всей компоненты анализа.

  \section{Практические результаты}

    Представленный алгоритм анализа указателей был реализован в рамках
    статического \java компилятора \eng{Excelsior~RVM} и протестирован на
    стандартных тестах \java программ.

    Тестирование проходило следующим образом. Каждая программа, входящая в
    тестовый набор, компилируется статическим компилятором, и для каждого
    метода этой программы проводится анализ указателей с помощью разработанного
    алгоритма.
    После проведения анализа вычисляется мера точности результатов и
    предоставляется информация о времени работы и потреблении памяти
    алгоритмом.

    \subsection{Входные данные}

      В рамках тестирования использовались следующие приложения:
      \begin{enumerate}
        \item \eng{SPEC~JVM98}~\cite{spec_jvm98}~--- это набор, состоящий из 8
              тестов, 5 из которых являются реальными приложениями. Общее
              количество методов в этом тесте: \num{3701}.
        \item \eng{SPECjvm2008}~\cite{spec_jvm2008}~--- это набор, состоящий из
              нескольких реальных приложений и некоторого количества
              искусственных бенчмарков, приложений для измерения
              производительности.
              Общее количество методов в данном наборе: \num{37931}.
        \item \eng{Eclipse~IDE}~\cite{eclipse} (версия 3.6)~--- среда
              разработки кроссплатформенных приложений.
              Общее количество методов в приложении: \num{263802}.
      \end{enumerate}
      Данные о размерах методов, скорости работы алгоритма и потреблении памяти
      представлены для приложения \eng{Eclipse~IDE}.

      Для каждого анализируемого метода можно получить следующую информацию:
      \begin{itemize}
        \item количество операций присваивания выражений ссылочного типа;
        \item количество операций создания новых объектов;
        \item количество переменных ссылочного типа.
      \end{itemize}

      Определим, с какими входными данными приходится работать алгоритму
      анализа. Как видно из графиков, представленных на
      рис.~\ref{plot:nassig_distribution} и \ref{plot:nobj_distribution},
      абсолютное большинство методов имеет меньше \num{20} присваиваний и в
      этих методах находится меньше \num{6} операций создания новых объектов.
      Можно заметить, что предположение о том, что большинство методов является
      небольшими по размеру, использованное при выборе типа алгоритма анализа,
      оправдало себя на практике.

      \begin{myplot}%
        {Распределение количества методов от числа присваиваний в них}%
        {plot:nassig_distribution}
        \begin{tikzpicture}[gnuplot]
%% generated with GNUPLOT 4.5p0 (Lua 5.1; terminal rev. 99, script rev. 98)
%% 27.05.2011 10:51:01
\path (0.000,0.000) rectangle (12.500,8.750);
\gpcolor{color=gp lt color border}
\gpsetlinetype{gp lt border}
\gpsetlinewidth{1.00}
\draw[gp path] (1.504,0.985)--(1.684,0.985);
\draw[gp path] (11.947,0.985)--(11.767,0.985);
\node[gp node right] at (1.320,0.985) {\num{0}};
\draw[gp path] (1.504,1.725)--(1.684,1.725);
\draw[gp path] (11.947,1.725)--(11.767,1.725);
\node[gp node right] at (1.320,1.725) {\num{0.1}};
\draw[gp path] (1.504,2.464)--(1.684,2.464);
\draw[gp path] (11.947,2.464)--(11.767,2.464);
\node[gp node right] at (1.320,2.464) {\num{0.2}};
\draw[gp path] (1.504,3.204)--(1.684,3.204);
\draw[gp path] (11.947,3.204)--(11.767,3.204);
\node[gp node right] at (1.320,3.204) {\num{0.3}};
\draw[gp path] (1.504,3.943)--(1.684,3.943);
\draw[gp path] (11.947,3.943)--(11.767,3.943);
\node[gp node right] at (1.320,3.943) {\num{0.4}};
\draw[gp path] (1.504,4.683)--(1.684,4.683);
\draw[gp path] (11.947,4.683)--(11.767,4.683);
\node[gp node right] at (1.320,4.683) {\num{0.5}};
\draw[gp path] (1.504,5.423)--(1.684,5.423);
\draw[gp path] (11.947,5.423)--(11.767,5.423);
\node[gp node right] at (1.320,5.423) {\num{0.6}};
\draw[gp path] (1.504,6.162)--(1.684,6.162);
\draw[gp path] (11.947,6.162)--(11.767,6.162);
\node[gp node right] at (1.320,6.162) {\num{0.7}};
\draw[gp path] (1.504,6.902)--(1.684,6.902);
\draw[gp path] (11.947,6.902)--(11.767,6.902);
\node[gp node right] at (1.320,6.902) {\num{0.8}};
\draw[gp path] (1.504,7.641)--(1.684,7.641);
\draw[gp path] (11.947,7.641)--(11.767,7.641);
\node[gp node right] at (1.320,7.641) {\num{0.9}};
\draw[gp path] (1.504,8.381)--(1.684,8.381);
\draw[gp path] (11.947,8.381)--(11.767,8.381);
\node[gp node right] at (1.320,8.381) {\num{1}};
\draw[gp path] (2.453,0.985)--(2.453,1.165);
\draw[gp path] (2.453,8.381)--(2.453,8.201);
\node[gp node center] at (2.453,0.677) {\num{10}};
\draw[gp path] (3.508,0.985)--(3.508,1.165);
\draw[gp path] (3.508,8.381)--(3.508,8.201);
\node[gp node center] at (3.508,0.677) {\num{20}};
\draw[gp path] (4.563,0.985)--(4.563,1.165);
\draw[gp path] (4.563,8.381)--(4.563,8.201);
\node[gp node center] at (4.563,0.677) {\num{30}};
\draw[gp path] (5.618,0.985)--(5.618,1.165);
\draw[gp path] (5.618,8.381)--(5.618,8.201);
\node[gp node center] at (5.618,0.677) {\num{40}};
\draw[gp path] (6.673,0.985)--(6.673,1.165);
\draw[gp path] (6.673,8.381)--(6.673,8.201);
\node[gp node center] at (6.673,0.677) {\num{50}};
\draw[gp path] (7.728,0.985)--(7.728,1.165);
\draw[gp path] (7.728,8.381)--(7.728,8.201);
\node[gp node center] at (7.728,0.677) {\num{60}};
\draw[gp path] (8.782,0.985)--(8.782,1.165);
\draw[gp path] (8.782,8.381)--(8.782,8.201);
\node[gp node center] at (8.782,0.677) {\num{70}};
\draw[gp path] (9.837,0.985)--(9.837,1.165);
\draw[gp path] (9.837,8.381)--(9.837,8.201);
\node[gp node center] at (9.837,0.677) {\num{80}};
\draw[gp path] (10.892,0.985)--(10.892,1.165);
\draw[gp path] (10.892,8.381)--(10.892,8.201);
\node[gp node center] at (10.892,0.677) {\num{90}};
\draw[gp path] (11.947,0.985)--(11.947,1.165);
\draw[gp path] (11.947,8.381)--(11.947,8.201);
\node[gp node center] at (11.947,0.677) {\num{100}};
\draw[gp path] (1.504,8.381)--(1.504,0.985)--(11.947,0.985)--(11.947,8.381)--cycle;
\node[gp node center,rotate=-270] at (0.246,4.683) {Количество методов, \%};
\node[gp node center] at (6.725,0.215) {Число присваиваний};
\gpsetlinetype{gp lt plot 0}
\draw[gp path] (1.504,5.076)--(2.453,5.076)--(2.453,2.097)--(3.508,2.097)--(3.508,1.476)%
  --(4.563,1.476)--(4.563,1.407)--(5.618,1.407)--(5.618,1.273)--(6.673,1.273)--(6.673,1.143)%
  --(7.728,1.143)--(7.728,1.125)--(8.782,1.125)--(8.782,1.097)--(9.837,1.097)--(9.837,1.073)%
  --(10.892,1.073)--(10.892,1.045)--(11.947,1.045)--(11.947,1.027);
\gpsetlinetype{gp lt border}
\draw[gp path] (1.504,8.381)--(1.504,0.985)--(11.947,0.985)--(11.947,8.381)--cycle;
%% coordinates of the plot area
\gpdefrectangularnode{gp plot 1}{\pgfpoint{1.504cm}{0.985cm}}{\pgfpoint{11.947cm}{8.381cm}}
\end{tikzpicture}
%% gnuplot variables

      \end{myplot}

      \begin{myplot}%
        {Распределение количества методов от числа операций создания новых
        объектов в них}%
        {plot:nobj_distribution}
        \begin{tikzpicture}[gnuplot]
%% generated with GNUPLOT 4.5p0 (Lua 5.1; terminal rev. 99, script rev. 98)
%% 26.05.2011 20:30:16
\path (0.000,0.000) rectangle (12.500,8.750);
\gpcolor{color=gp lt color border}
\gpsetlinetype{gp lt border}
\gpsetlinewidth{1.00}
\draw[gp path] (1.504,0.985)--(1.684,0.985);
\draw[gp path] (11.947,0.985)--(11.767,0.985);
\node[gp node right] at (1.320,0.985) {\num{0}};
\draw[gp path] (1.504,1.725)--(1.684,1.725);
\draw[gp path] (11.947,1.725)--(11.767,1.725);
\node[gp node right] at (1.320,1.725) {\num{0.1}};
\draw[gp path] (1.504,2.464)--(1.684,2.464);
\draw[gp path] (11.947,2.464)--(11.767,2.464);
\node[gp node right] at (1.320,2.464) {\num{0.2}};
\draw[gp path] (1.504,3.204)--(1.684,3.204);
\draw[gp path] (11.947,3.204)--(11.767,3.204);
\node[gp node right] at (1.320,3.204) {\num{0.3}};
\draw[gp path] (1.504,3.943)--(1.684,3.943);
\draw[gp path] (11.947,3.943)--(11.767,3.943);
\node[gp node right] at (1.320,3.943) {\num{0.4}};
\draw[gp path] (1.504,4.683)--(1.684,4.683);
\draw[gp path] (11.947,4.683)--(11.767,4.683);
\node[gp node right] at (1.320,4.683) {\num{0.5}};
\draw[gp path] (1.504,5.423)--(1.684,5.423);
\draw[gp path] (11.947,5.423)--(11.767,5.423);
\node[gp node right] at (1.320,5.423) {\num{0.6}};
\draw[gp path] (1.504,6.162)--(1.684,6.162);
\draw[gp path] (11.947,6.162)--(11.767,6.162);
\node[gp node right] at (1.320,6.162) {\num{0.7}};
\draw[gp path] (1.504,6.902)--(1.684,6.902);
\draw[gp path] (11.947,6.902)--(11.767,6.902);
\node[gp node right] at (1.320,6.902) {\num{0.8}};
\draw[gp path] (1.504,7.641)--(1.684,7.641);
\draw[gp path] (11.947,7.641)--(11.767,7.641);
\node[gp node right] at (1.320,7.641) {\num{0.9}};
\draw[gp path] (1.504,8.381)--(1.684,8.381);
\draw[gp path] (11.947,8.381)--(11.767,8.381);
\node[gp node right] at (1.320,8.381) {\num{1}};
\draw[gp path] (1.504,0.985)--(1.504,1.165);
\draw[gp path] (1.504,8.381)--(1.504,8.201);
\node[gp node center] at (1.504,0.677) {$10^{0}$};
\draw[gp path] (2.552,0.985)--(2.552,1.075);
\draw[gp path] (2.552,8.381)--(2.552,8.291);
\draw[gp path] (3.165,0.985)--(3.165,1.075);
\draw[gp path] (3.165,8.381)--(3.165,8.291);
\draw[gp path] (3.600,0.985)--(3.600,1.075);
\draw[gp path] (3.600,8.381)--(3.600,8.291);
\draw[gp path] (3.937,0.985)--(3.937,1.075);
\draw[gp path] (3.937,8.381)--(3.937,8.291);
\draw[gp path] (4.213,0.985)--(4.213,1.075);
\draw[gp path] (4.213,8.381)--(4.213,8.291);
\draw[gp path] (4.446,0.985)--(4.446,1.075);
\draw[gp path] (4.446,8.381)--(4.446,8.291);
\draw[gp path] (4.648,0.985)--(4.648,1.075);
\draw[gp path] (4.648,8.381)--(4.648,8.291);
\draw[gp path] (4.826,0.985)--(4.826,1.075);
\draw[gp path] (4.826,8.381)--(4.826,8.291);
\draw[gp path] (4.985,0.985)--(4.985,1.165);
\draw[gp path] (4.985,8.381)--(4.985,8.201);
\node[gp node center] at (4.985,0.677) {$10^{1}$};
\draw[gp path] (6.033,0.985)--(6.033,1.075);
\draw[gp path] (6.033,8.381)--(6.033,8.291);
\draw[gp path] (6.646,0.985)--(6.646,1.075);
\draw[gp path] (6.646,8.381)--(6.646,8.291);
\draw[gp path] (7.081,0.985)--(7.081,1.075);
\draw[gp path] (7.081,8.381)--(7.081,8.291);
\draw[gp path] (7.418,0.985)--(7.418,1.075);
\draw[gp path] (7.418,8.381)--(7.418,8.291);
\draw[gp path] (7.694,0.985)--(7.694,1.075);
\draw[gp path] (7.694,8.381)--(7.694,8.291);
\draw[gp path] (7.927,0.985)--(7.927,1.075);
\draw[gp path] (7.927,8.381)--(7.927,8.291);
\draw[gp path] (8.129,0.985)--(8.129,1.075);
\draw[gp path] (8.129,8.381)--(8.129,8.291);
\draw[gp path] (8.307,0.985)--(8.307,1.075);
\draw[gp path] (8.307,8.381)--(8.307,8.291);
\draw[gp path] (8.466,0.985)--(8.466,1.165);
\draw[gp path] (8.466,8.381)--(8.466,8.201);
\node[gp node center] at (8.466,0.677) {$10^{2}$};
\draw[gp path] (9.514,0.985)--(9.514,1.075);
\draw[gp path] (9.514,8.381)--(9.514,8.291);
\draw[gp path] (10.127,0.985)--(10.127,1.075);
\draw[gp path] (10.127,8.381)--(10.127,8.291);
\draw[gp path] (10.562,0.985)--(10.562,1.075);
\draw[gp path] (10.562,8.381)--(10.562,8.291);
\draw[gp path] (10.899,0.985)--(10.899,1.075);
\draw[gp path] (10.899,8.381)--(10.899,8.291);
\draw[gp path] (11.175,0.985)--(11.175,1.075);
\draw[gp path] (11.175,8.381)--(11.175,8.291);
\draw[gp path] (11.408,0.985)--(11.408,1.075);
\draw[gp path] (11.408,8.381)--(11.408,8.291);
\draw[gp path] (11.610,0.985)--(11.610,1.075);
\draw[gp path] (11.610,8.381)--(11.610,8.291);
\draw[gp path] (11.788,0.985)--(11.788,1.075);
\draw[gp path] (11.788,8.381)--(11.788,8.291);
\draw[gp path] (11.947,0.985)--(11.947,1.165);
\draw[gp path] (11.947,8.381)--(11.947,8.201);
\node[gp node center] at (11.947,0.677) {$10^{3}$};
\draw[gp path] (1.504,8.381)--(1.504,0.985)--(11.947,0.985)--(11.947,8.381)--cycle;
\node[gp node center,rotate=-270] at (0.246,4.683) {Количество методов, \%};
\node[gp node center] at (6.725,0.215) {Число операций создания объектов};
\gpcolor{color=gp lt color 0}
\gpsetlinetype{gp lt plot 0}
\draw[gp path] (1.504,7.504)--(3.165,7.504)--(3.165,1.438)--(4.213,1.438)--(4.213,1.143)%
  --(4.826,1.143)--(4.826,1.087)--(5.261,1.087)--(5.261,1.053)--(5.598,1.053)--(5.598,1.003)%
  --(5.874,1.003)--(5.874,1.009)--(6.107,1.009)--(6.107,0.997)--(6.309,0.997)--(6.309,0.995)%
  --(6.487,0.995)--(6.487,0.993)--(6.646,0.993)--(6.646,0.991)--(6.790,0.991)--(6.790,0.993)%
  --(6.921,0.993)--(6.921,0.989)--(7.969,0.989)--(7.969,0.987)--(8.886,0.987)--(11.444,0.987);
\gpcolor{color=gp lt color border}
\gpsetlinetype{gp lt border}
\draw[gp path] (1.504,8.381)--(1.504,0.985)--(11.947,0.985)--(11.947,8.381)--cycle;
%% coordinates of the plot area
\gpdefrectangularnode{gp plot 1}{\pgfpoint{1.504cm}{0.985cm}}{\pgfpoint{11.947cm}{8.381cm}}
\end{tikzpicture}
%% gnuplot variables

      \end{myplot}

    \subsection{Скорость работы}

      Как было сказано в разделе~\ref{section:algorithm}, алгоритм осуществляет
      многочисленные проходы по присваиваниям метода до тех пор, пока
      множества целей всех выражений не стабилизируются. В ходе тестов было
      получено, что число таких проходов обычно невелико: в нашем случае оно не
      превышает \num{10} и не увеличивается при увеличении размеров метода
      (см.~рис.~\ref{plot:iter_count}).

      \begin{myplot}%
        {Зависимость числа проходов алгоритма анализа от размера метода
        (радиус кругов пропорционален количеству методов)}%
        {plot:iter_count}
        \begin{tikzpicture}[gnuplot]
%% generated with GNUPLOT 4.5p0 (Lua 5.1; terminal rev. 99, script rev. 98)
%% 27.05.2011 12:41:07
\path (0.000,0.000) rectangle (12.500,8.750);
\gpcolor{color=gp lt color border}
\gpsetlinetype{gp lt border}
\gpsetlinewidth{1.00}
\draw[gp path] (1.136,0.985)--(1.316,0.985);
\draw[gp path] (11.947,0.985)--(11.767,0.985);
\node[gp node right] at (0.952,0.985) {\num{0}};
\draw[gp path] (1.136,2.042)--(1.316,2.042);
\draw[gp path] (11.947,2.042)--(11.767,2.042);
\node[gp node right] at (0.952,2.042) {\num{1}};
\draw[gp path] (1.136,3.098)--(1.316,3.098);
\draw[gp path] (11.947,3.098)--(11.767,3.098);
\node[gp node right] at (0.952,3.098) {\num{2}};
\draw[gp path] (1.136,4.155)--(1.316,4.155);
\draw[gp path] (11.947,4.155)--(11.767,4.155);
\node[gp node right] at (0.952,4.155) {\num{3}};
\draw[gp path] (1.136,5.211)--(1.316,5.211);
\draw[gp path] (11.947,5.211)--(11.767,5.211);
\node[gp node right] at (0.952,5.211) {\num{4}};
\draw[gp path] (1.136,6.268)--(1.316,6.268);
\draw[gp path] (11.947,6.268)--(11.767,6.268);
\node[gp node right] at (0.952,6.268) {\num{5}};
\draw[gp path] (1.136,7.324)--(1.316,7.324);
\draw[gp path] (11.947,7.324)--(11.767,7.324);
\node[gp node right] at (0.952,7.324) {\num{6}};
\draw[gp path] (1.136,8.381)--(1.316,8.381);
\draw[gp path] (11.947,8.381)--(11.767,8.381);
\node[gp node right] at (0.952,8.381) {\num{7}};
\draw[gp path] (1.136,0.985)--(1.136,1.165);
\draw[gp path] (1.136,8.381)--(1.136,8.201);
\node[gp node center] at (1.136,0.677) {\num{0}};
\draw[gp path] (2.487,0.985)--(2.487,1.165);
\draw[gp path] (2.487,8.381)--(2.487,8.201);
\node[gp node center] at (2.487,0.677) {\num{50}};
\draw[gp path] (3.839,0.985)--(3.839,1.165);
\draw[gp path] (3.839,8.381)--(3.839,8.201);
\node[gp node center] at (3.839,0.677) {\num{100}};
\draw[gp path] (5.190,0.985)--(5.190,1.165);
\draw[gp path] (5.190,8.381)--(5.190,8.201);
\node[gp node center] at (5.190,0.677) {\num{150}};
\draw[gp path] (6.541,0.985)--(6.541,1.165);
\draw[gp path] (6.541,8.381)--(6.541,8.201);
\node[gp node center] at (6.541,0.677) {\num{200}};
\draw[gp path] (7.893,0.985)--(7.893,1.165);
\draw[gp path] (7.893,8.381)--(7.893,8.201);
\node[gp node center] at (7.893,0.677) {\num{250}};
\draw[gp path] (9.244,0.985)--(9.244,1.165);
\draw[gp path] (9.244,8.381)--(9.244,8.201);
\node[gp node center] at (9.244,0.677) {\num{300}};
\draw[gp path] (10.596,0.985)--(10.596,1.165);
\draw[gp path] (10.596,8.381)--(10.596,8.201);
\node[gp node center] at (10.596,0.677) {\num{350}};
\draw[gp path] (11.947,0.985)--(11.947,1.165);
\draw[gp path] (11.947,8.381)--(11.947,8.201);
\node[gp node center] at (11.947,0.677) {\num{400}};
\draw[gp path] (1.136,8.381)--(1.136,0.985)--(11.947,0.985)--(11.947,8.381)--cycle;
\node[gp node center,rotate=-270] at (0.246,4.683) {Количество проходов};
\node[gp node center] at (6.541,0.215) {Количество присваиваний};
\gpfill{color=gp lt color border,opacity=0.50} (2.649,6.268)--(2.648,6.276)--(2.648,6.284)--(2.647,6.293)%
    --(2.645,6.301)--(2.643,6.309)--(2.641,6.318)--(2.638,6.326)--(2.634,6.333)%
    --(2.631,6.341)--(2.627,6.348)--(2.622,6.356)--(2.618,6.363)--(2.612,6.369)%
    --(2.607,6.376)--(2.601,6.382)--(2.595,6.388)--(2.588,6.393)--(2.582,6.399)%
    --(2.575,6.403)--(2.568,6.408)--(2.560,6.412)--(2.552,6.415)--(2.545,6.419)%
    --(2.537,6.422)--(2.528,6.424)--(2.520,6.426)--(2.512,6.428)--(2.503,6.429)%
    --(2.495,6.429)--(2.487,6.430)--(2.478,6.429)--(2.470,6.429)--(2.461,6.428)%
    --(2.453,6.426)--(2.445,6.424)--(2.436,6.422)--(2.428,6.419)--(2.421,6.415)%
    --(2.413,6.412)--(2.406,6.408)--(2.398,6.403)--(2.391,6.399)--(2.385,6.393)%
    --(2.378,6.388)--(2.372,6.382)--(2.366,6.376)--(2.361,6.369)--(2.355,6.363)%
    --(2.351,6.356)--(2.346,6.348)--(2.342,6.341)--(2.339,6.333)--(2.335,6.326)%
    --(2.332,6.318)--(2.330,6.309)--(2.328,6.301)--(2.326,6.293)--(2.325,6.284)%
    --(2.325,6.276)--(2.325,6.268)--(2.325,6.259)--(2.325,6.251)--(2.326,6.242)%
    --(2.328,6.234)--(2.330,6.226)--(2.332,6.217)--(2.335,6.209)--(2.339,6.202)%
    --(2.342,6.194)--(2.346,6.186)--(2.351,6.179)--(2.355,6.172)--(2.361,6.166)%
    --(2.366,6.159)--(2.372,6.153)--(2.378,6.147)--(2.385,6.142)--(2.391,6.136)%
    --(2.398,6.132)--(2.405,6.127)--(2.413,6.123)--(2.421,6.120)--(2.428,6.116)%
    --(2.436,6.113)--(2.445,6.111)--(2.453,6.109)--(2.461,6.107)--(2.470,6.106)%
    --(2.478,6.106)--(2.486,6.106)--(2.495,6.106)--(2.503,6.106)--(2.512,6.107)%
    --(2.520,6.109)--(2.528,6.111)--(2.537,6.113)--(2.545,6.116)--(2.552,6.120)%
    --(2.560,6.123)--(2.568,6.127)--(2.575,6.132)--(2.582,6.136)--(2.588,6.142)%
    --(2.595,6.147)--(2.601,6.153)--(2.607,6.159)--(2.612,6.166)--(2.618,6.172)%
    --(2.622,6.179)--(2.627,6.186)--(2.631,6.194)--(2.634,6.202)--(2.638,6.209)%
    --(2.641,6.217)--(2.643,6.226)--(2.645,6.234)--(2.647,6.242)--(2.648,6.251)%
    --(2.648,6.259)--(2.649,6.267)--cycle;
\gpfill{color=gp lt color border,opacity=0.50} (5.433,5.211)--(5.432,5.223)--(5.431,5.236)--(5.430,5.249)%
    --(5.427,5.261)--(5.424,5.273)--(5.421,5.286)--(5.416,5.298)--(5.411,5.309)%
    --(5.406,5.321)--(5.400,5.332)--(5.393,5.343)--(5.386,5.353)--(5.378,5.363)%
    --(5.370,5.373)--(5.361,5.382)--(5.352,5.391)--(5.342,5.399)--(5.332,5.407)%
    --(5.322,5.414)--(5.311,5.421)--(5.300,5.427)--(5.288,5.432)--(5.277,5.437)%
    --(5.265,5.442)--(5.252,5.445)--(5.240,5.448)--(5.228,5.451)--(5.215,5.452)%
    --(5.202,5.453)--(5.190,5.454)--(5.177,5.453)--(5.164,5.452)--(5.151,5.451)%
    --(5.139,5.448)--(5.127,5.445)--(5.114,5.442)--(5.102,5.437)--(5.091,5.432)%
    --(5.079,5.427)--(5.068,5.421)--(5.057,5.414)--(5.047,5.407)--(5.037,5.399)%
    --(5.027,5.391)--(5.018,5.382)--(5.009,5.373)--(5.001,5.363)--(4.993,5.353)%
    --(4.986,5.343)--(4.979,5.332)--(4.973,5.321)--(4.968,5.309)--(4.963,5.298)%
    --(4.958,5.286)--(4.955,5.273)--(4.952,5.261)--(4.949,5.249)--(4.948,5.236)%
    --(4.947,5.223)--(4.947,5.211)--(4.947,5.198)--(4.948,5.185)--(4.949,5.172)%
    --(4.952,5.160)--(4.955,5.148)--(4.958,5.135)--(4.963,5.123)--(4.968,5.112)%
    --(4.973,5.100)--(4.979,5.089)--(4.986,5.078)--(4.993,5.068)--(5.001,5.058)%
    --(5.009,5.048)--(5.018,5.039)--(5.027,5.030)--(5.037,5.022)--(5.047,5.014)%
    --(5.057,5.007)--(5.068,5.000)--(5.079,4.994)--(5.091,4.989)--(5.102,4.984)%
    --(5.114,4.979)--(5.127,4.976)--(5.139,4.973)--(5.151,4.970)--(5.164,4.969)%
    --(5.177,4.968)--(5.189,4.968)--(5.202,4.968)--(5.215,4.969)--(5.228,4.970)%
    --(5.240,4.973)--(5.252,4.976)--(5.265,4.979)--(5.277,4.984)--(5.288,4.989)%
    --(5.300,4.994)--(5.311,5.000)--(5.322,5.007)--(5.332,5.014)--(5.342,5.022)%
    --(5.352,5.030)--(5.361,5.039)--(5.370,5.048)--(5.378,5.058)--(5.386,5.068)%
    --(5.393,5.078)--(5.400,5.089)--(5.406,5.100)--(5.411,5.112)--(5.416,5.123)%
    --(5.421,5.135)--(5.424,5.148)--(5.427,5.160)--(5.430,5.172)--(5.431,5.185)%
    --(5.432,5.198)--(5.433,5.210)--cycle;
\gpfill{color=gp lt color border,opacity=0.50} (3.541,3.098)--(3.540,3.110)--(3.539,3.123)--(3.538,3.136)%
    --(3.535,3.148)--(3.532,3.160)--(3.529,3.173)--(3.524,3.185)--(3.519,3.196)%
    --(3.514,3.208)--(3.508,3.219)--(3.501,3.230)--(3.494,3.240)--(3.486,3.250)%
    --(3.478,3.260)--(3.469,3.269)--(3.460,3.278)--(3.450,3.286)--(3.440,3.294)%
    --(3.430,3.301)--(3.419,3.308)--(3.408,3.314)--(3.396,3.319)--(3.385,3.324)%
    --(3.373,3.329)--(3.360,3.332)--(3.348,3.335)--(3.336,3.338)--(3.323,3.339)%
    --(3.310,3.340)--(3.298,3.341)--(3.285,3.340)--(3.272,3.339)--(3.259,3.338)%
    --(3.247,3.335)--(3.235,3.332)--(3.222,3.329)--(3.210,3.324)--(3.199,3.319)%
    --(3.187,3.314)--(3.176,3.308)--(3.165,3.301)--(3.155,3.294)--(3.145,3.286)%
    --(3.135,3.278)--(3.126,3.269)--(3.117,3.260)--(3.109,3.250)--(3.101,3.240)%
    --(3.094,3.230)--(3.087,3.219)--(3.081,3.208)--(3.076,3.196)--(3.071,3.185)%
    --(3.066,3.173)--(3.063,3.160)--(3.060,3.148)--(3.057,3.136)--(3.056,3.123)%
    --(3.055,3.110)--(3.055,3.098)--(3.055,3.085)--(3.056,3.072)--(3.057,3.059)%
    --(3.060,3.047)--(3.063,3.035)--(3.066,3.022)--(3.071,3.010)--(3.076,2.999)%
    --(3.081,2.987)--(3.087,2.976)--(3.094,2.965)--(3.101,2.955)--(3.109,2.945)%
    --(3.117,2.935)--(3.126,2.926)--(3.135,2.917)--(3.145,2.909)--(3.155,2.901)%
    --(3.165,2.894)--(3.176,2.887)--(3.187,2.881)--(3.199,2.876)--(3.210,2.871)%
    --(3.222,2.866)--(3.235,2.863)--(3.247,2.860)--(3.259,2.857)--(3.272,2.856)%
    --(3.285,2.855)--(3.297,2.855)--(3.310,2.855)--(3.323,2.856)--(3.336,2.857)%
    --(3.348,2.860)--(3.360,2.863)--(3.373,2.866)--(3.385,2.871)--(3.396,2.876)%
    --(3.408,2.881)--(3.419,2.887)--(3.430,2.894)--(3.440,2.901)--(3.450,2.909)%
    --(3.460,2.917)--(3.469,2.926)--(3.478,2.935)--(3.486,2.945)--(3.494,2.955)%
    --(3.501,2.965)--(3.508,2.976)--(3.514,2.987)--(3.519,2.999)--(3.524,3.010)%
    --(3.529,3.022)--(3.532,3.035)--(3.535,3.047)--(3.538,3.059)--(3.539,3.072)%
    --(3.540,3.085)--(3.541,3.097)--cycle;
\gpfill{color=gp lt color border,opacity=0.50} (10.596,7.324)--(10.596,7.324)--(10.596,7.324)--(10.596,7.324)%
    --(10.596,7.324)--(10.596,7.324)--(10.596,7.324)--(10.596,7.324)--(10.596,7.324)%
    --(10.596,7.324)--(10.596,7.324)--(10.596,7.324)--(10.596,7.324)--(10.596,7.324)%
    --(10.596,7.324)--(10.596,7.324)--(10.596,7.324)--(10.596,7.324)--(10.596,7.324)%
    --(10.596,7.324)--(10.596,7.324)--(10.596,7.324)--(10.596,7.324)--(10.596,7.324)%
    --(10.596,7.324)--(10.596,7.324)--(10.596,7.324)--(10.596,7.324)--(10.596,7.324)%
    --(10.596,7.324)--(10.596,7.324)--(10.596,7.324)--(10.596,7.324)--(10.596,7.324)%
    --(10.596,7.324)--(10.596,7.324)--(10.596,7.324)--(10.596,7.324)--(10.596,7.324)%
    --(10.596,7.324)--(10.596,7.324)--(10.596,7.324)--(10.596,7.324)--(10.596,7.324)%
    --(10.596,7.324)--(10.596,7.324)--(10.596,7.324)--(10.596,7.324)--(10.596,7.324)%
    --(10.596,7.324)--(10.596,7.324)--(10.596,7.324)--(10.596,7.324)--(10.596,7.324)%
    --(10.596,7.324)--(10.596,7.324)--(10.596,7.324)--(10.596,7.324)--(10.596,7.324)%
    --(10.596,7.324)--(10.596,7.324)--(10.596,7.324)--(10.596,7.324)--(10.596,7.324)%
    --(10.596,7.324)--(10.596,7.324)--(10.596,7.324)--(10.596,7.324)--(10.596,7.324)%
    --(10.596,7.324)--(10.596,7.324)--(10.596,7.324)--(10.596,7.324)--(10.596,7.324)%
    --(10.596,7.324)--(10.596,7.324)--(10.596,7.324)--(10.596,7.324)--(10.596,7.324)%
    --(10.596,7.324)--(10.596,7.324)--(10.596,7.324)--(10.596,7.324)--(10.596,7.324)%
    --(10.596,7.324)--(10.596,7.324)--(10.596,7.324)--(10.596,7.324)--(10.596,7.324)%
    --(10.596,7.324)--(10.596,7.324)--(10.596,7.324)--(10.596,7.324)--(10.596,7.324)%
    --(10.596,7.324)--(10.596,7.324)--(10.596,7.324)--(10.596,7.324)--(10.596,7.324)%
    --(10.596,7.324)--(10.596,7.324)--(10.596,7.324)--(10.596,7.324)--(10.596,7.324)%
    --(10.596,7.324)--(10.596,7.324)--(10.596,7.324)--(10.596,7.324)--(10.596,7.324)%
    --(10.596,7.324)--(10.596,7.324)--(10.596,7.324)--(10.596,7.324)--(10.596,7.324)%
    --(10.596,7.324)--(10.596,7.324)--(10.596,7.324)--(10.596,7.324)--(10.596,7.324)%
    --(10.596,7.324)--cycle;
\gpfill{color=gp lt color border,opacity=0.50} (4.110,5.211)--(4.109,5.225)--(4.108,5.239)--(4.106,5.253)%
    --(4.104,5.267)--(4.100,5.281)--(4.096,5.294)--(4.092,5.308)--(4.086,5.321)%
    --(4.080,5.334)--(4.073,5.346)--(4.066,5.358)--(4.058,5.370)--(4.049,5.381)%
    --(4.040,5.392)--(4.030,5.402)--(4.020,5.412)--(4.009,5.421)--(3.998,5.430)%
    --(3.986,5.438)--(3.974,5.445)--(3.962,5.452)--(3.949,5.458)--(3.936,5.464)%
    --(3.922,5.468)--(3.909,5.472)--(3.895,5.476)--(3.881,5.478)--(3.867,5.480)%
    --(3.853,5.481)--(3.839,5.482)--(3.824,5.481)--(3.810,5.480)--(3.796,5.478)%
    --(3.782,5.476)--(3.768,5.472)--(3.755,5.468)--(3.741,5.464)--(3.728,5.458)%
    --(3.715,5.452)--(3.703,5.445)--(3.691,5.438)--(3.679,5.430)--(3.668,5.421)%
    --(3.657,5.412)--(3.647,5.402)--(3.637,5.392)--(3.628,5.381)--(3.619,5.370)%
    --(3.611,5.358)--(3.604,5.346)--(3.597,5.334)--(3.591,5.321)--(3.585,5.308)%
    --(3.581,5.294)--(3.577,5.281)--(3.573,5.267)--(3.571,5.253)--(3.569,5.239)%
    --(3.568,5.225)--(3.568,5.211)--(3.568,5.196)--(3.569,5.182)--(3.571,5.168)%
    --(3.573,5.154)--(3.577,5.140)--(3.581,5.127)--(3.585,5.113)--(3.591,5.100)%
    --(3.597,5.087)--(3.604,5.075)--(3.611,5.063)--(3.619,5.051)--(3.628,5.040)%
    --(3.637,5.029)--(3.647,5.019)--(3.657,5.009)--(3.668,5.000)--(3.679,4.991)%
    --(3.691,4.983)--(3.703,4.976)--(3.715,4.969)--(3.728,4.963)--(3.741,4.957)%
    --(3.755,4.953)--(3.768,4.949)--(3.782,4.945)--(3.796,4.943)--(3.810,4.941)%
    --(3.824,4.940)--(3.838,4.940)--(3.853,4.940)--(3.867,4.941)--(3.881,4.943)%
    --(3.895,4.945)--(3.909,4.949)--(3.922,4.953)--(3.936,4.957)--(3.949,4.963)%
    --(3.962,4.969)--(3.974,4.976)--(3.986,4.983)--(3.998,4.991)--(4.009,5.000)%
    --(4.020,5.009)--(4.030,5.019)--(4.040,5.029)--(4.049,5.040)--(4.058,5.051)%
    --(4.066,5.063)--(4.073,5.075)--(4.080,5.087)--(4.086,5.100)--(4.092,5.113)%
    --(4.096,5.127)--(4.100,5.140)--(4.104,5.154)--(4.106,5.168)--(4.108,5.182)%
    --(4.109,5.196)--(4.110,5.210)--cycle;
\gpfill{color=gp lt color border,opacity=0.50} (6.541,8.381)--(6.541,8.381)--(6.541,8.381)--(6.541,8.381)%
    --(6.541,8.381)--(6.541,8.381)--(6.541,8.381)--(6.541,8.381)--(6.541,8.381)%
    --(6.541,8.381)--(6.541,8.381)--(6.541,8.381)--(6.541,8.381)--(6.541,8.381)%
    --(6.541,8.381)--(6.541,8.381)--(6.541,8.381)--(6.541,8.381)--(6.541,8.381)%
    --(6.541,8.381)--(6.541,8.381)--(6.541,8.381)--(6.541,8.381)--(6.541,8.381)%
    --(6.541,8.381)--(6.541,8.381)--(6.541,8.381)--(6.541,8.381)--(6.541,8.381)%
    --(6.541,8.381)--(6.541,8.381)--(6.541,8.381)--(6.541,8.381)--(6.541,8.381)%
    --(6.541,8.381)--(6.541,8.381)--(6.541,8.381)--(6.541,8.381)--(6.541,8.381)%
    --(6.541,8.381)--(6.541,8.381)--(6.541,8.381)--(6.541,8.381)--(6.541,8.381)%
    --(6.541,8.381)--(6.541,8.381)--(6.541,8.381)--(6.541,8.381)--(6.541,8.381)%
    --(6.541,8.381)--(6.541,8.381)--(6.541,8.381)--(6.541,8.381)--(6.541,8.381)%
    --(6.541,8.381)--(6.541,8.381)--(6.541,8.381)--(6.541,8.381)--(6.541,8.381)%
    --(6.541,8.381)--(6.541,8.381)--(6.541,8.381)--(6.541,8.381)--(6.541,8.381)%
    --(6.541,8.381)--(6.541,8.381)--(6.541,8.381)--(6.541,8.381)--(6.541,8.381)%
    --(6.541,8.381)--(6.541,8.381)--(6.541,8.381)--(6.541,8.381)--(6.541,8.381)%
    --(6.541,8.381)--(6.541,8.381)--(6.541,8.381)--(6.541,8.381)--(6.541,8.381)%
    --(6.541,8.381)--(6.541,8.381)--(6.541,8.381)--(6.541,8.381)--(6.541,8.381)%
    --(6.541,8.381)--(6.541,8.381)--(6.541,8.381)--(6.541,8.381)--(6.541,8.381)%
    --(6.541,8.381)--(6.541,8.381)--(6.541,8.381)--(6.541,8.381)--(6.541,8.381)%
    --(6.541,8.381)--(6.541,8.381)--(6.541,8.381)--(6.541,8.381)--(6.541,8.381)%
    --(6.541,8.381)--(6.541,8.381)--(6.541,8.381)--(6.541,8.381)--(6.541,8.381)%
    --(6.541,8.381)--(6.541,8.381)--(6.541,8.381)--(6.541,8.381)--(6.541,8.381)%
    --(6.541,8.381)--(6.541,8.381)--(6.541,8.381)--(6.541,8.381)--(6.541,8.381)%
    --(6.541,8.381)--(6.541,8.381)--(6.541,8.381)--(6.541,8.381)--(6.541,8.381)%
    --(6.541,8.381)--cycle;
\gpfill{color=gp lt color border,opacity=0.50} (2.784,5.211)--(2.783,5.226)--(2.782,5.242)--(2.780,5.257)%
    --(2.777,5.272)--(2.773,5.287)--(2.769,5.302)--(2.764,5.317)--(2.758,5.331)%
    --(2.751,5.345)--(2.744,5.359)--(2.736,5.372)--(2.727,5.385)--(2.717,5.397)%
    --(2.707,5.409)--(2.697,5.421)--(2.685,5.431)--(2.673,5.441)--(2.661,5.451)%
    --(2.648,5.460)--(2.635,5.468)--(2.621,5.475)--(2.607,5.482)--(2.593,5.488)%
    --(2.578,5.493)--(2.563,5.497)--(2.548,5.501)--(2.533,5.504)--(2.518,5.506)%
    --(2.502,5.507)--(2.487,5.508)--(2.471,5.507)--(2.455,5.506)--(2.440,5.504)%
    --(2.425,5.501)--(2.410,5.497)--(2.395,5.493)--(2.380,5.488)--(2.366,5.482)%
    --(2.352,5.475)--(2.338,5.468)--(2.325,5.460)--(2.312,5.451)--(2.300,5.441)%
    --(2.288,5.431)--(2.276,5.421)--(2.266,5.409)--(2.256,5.397)--(2.246,5.385)%
    --(2.237,5.372)--(2.229,5.359)--(2.222,5.345)--(2.215,5.331)--(2.209,5.317)%
    --(2.204,5.302)--(2.200,5.287)--(2.196,5.272)--(2.193,5.257)--(2.191,5.242)%
    --(2.190,5.226)--(2.190,5.211)--(2.190,5.195)--(2.191,5.179)--(2.193,5.164)%
    --(2.196,5.149)--(2.200,5.134)--(2.204,5.119)--(2.209,5.104)--(2.215,5.090)%
    --(2.222,5.076)--(2.229,5.062)--(2.237,5.049)--(2.246,5.036)--(2.256,5.024)%
    --(2.266,5.012)--(2.276,5.000)--(2.288,4.990)--(2.300,4.980)--(2.312,4.970)%
    --(2.325,4.961)--(2.338,4.953)--(2.352,4.946)--(2.366,4.939)--(2.380,4.933)%
    --(2.395,4.928)--(2.410,4.924)--(2.425,4.920)--(2.440,4.917)--(2.455,4.915)%
    --(2.471,4.914)--(2.486,4.914)--(2.502,4.914)--(2.518,4.915)--(2.533,4.917)%
    --(2.548,4.920)--(2.563,4.924)--(2.578,4.928)--(2.593,4.933)--(2.607,4.939)%
    --(2.621,4.946)--(2.635,4.953)--(2.648,4.961)--(2.661,4.970)--(2.673,4.980)%
    --(2.685,4.990)--(2.697,5.000)--(2.707,5.012)--(2.717,5.024)--(2.727,5.036)%
    --(2.736,5.049)--(2.744,5.062)--(2.751,5.076)--(2.758,5.090)--(2.764,5.104)%
    --(2.769,5.119)--(2.773,5.134)--(2.777,5.149)--(2.780,5.164)--(2.782,5.179)%
    --(2.783,5.195)--(2.784,5.210)--cycle;
\gpfill{color=gp lt color border,opacity=0.50} (5.190,8.381)--(5.190,8.381)--(5.190,8.381)--(5.190,8.381)%
    --(5.190,8.381)--(5.190,8.381)--(5.190,8.381)--(5.190,8.381)--(5.190,8.381)%
    --(5.190,8.381)--(5.190,8.381)--(5.190,8.381)--(5.190,8.381)--(5.190,8.381)%
    --(5.190,8.381)--(5.190,8.381)--(5.190,8.381)--(5.190,8.381)--(5.190,8.381)%
    --(5.190,8.381)--(5.190,8.381)--(5.190,8.381)--(5.190,8.381)--(5.190,8.381)%
    --(5.190,8.381)--(5.190,8.381)--(5.190,8.381)--(5.190,8.381)--(5.190,8.381)%
    --(5.190,8.381)--(5.190,8.381)--(5.190,8.381)--(5.190,8.381)--(5.190,8.381)%
    --(5.190,8.381)--(5.190,8.381)--(5.190,8.381)--(5.190,8.381)--(5.190,8.381)%
    --(5.190,8.381)--(5.190,8.381)--(5.190,8.381)--(5.190,8.381)--(5.190,8.381)%
    --(5.190,8.381)--(5.190,8.381)--(5.190,8.381)--(5.190,8.381)--(5.190,8.381)%
    --(5.190,8.381)--(5.190,8.381)--(5.190,8.381)--(5.190,8.381)--(5.190,8.381)%
    --(5.190,8.381)--(5.190,8.381)--(5.190,8.381)--(5.190,8.381)--(5.190,8.381)%
    --(5.190,8.381)--(5.190,8.381)--(5.190,8.381)--(5.190,8.381)--(5.190,8.381)%
    --(5.190,8.381)--(5.190,8.381)--(5.190,8.381)--(5.190,8.381)--(5.190,8.381)%
    --(5.190,8.381)--(5.190,8.381)--(5.190,8.381)--(5.190,8.381)--(5.190,8.381)%
    --(5.190,8.381)--(5.190,8.381)--(5.190,8.381)--(5.190,8.381)--(5.190,8.381)%
    --(5.190,8.381)--(5.190,8.381)--(5.190,8.381)--(5.190,8.381)--(5.190,8.381)%
    --(5.190,8.381)--(5.190,8.381)--(5.190,8.381)--(5.190,8.381)--(5.190,8.381)%
    --(5.190,8.381)--(5.190,8.381)--(5.190,8.381)--(5.190,8.381)--(5.190,8.381)%
    --(5.190,8.381)--(5.190,8.381)--(5.190,8.381)--(5.190,8.381)--(5.190,8.381)%
    --(5.190,8.381)--(5.190,8.381)--(5.190,8.381)--(5.190,8.381)--(5.190,8.381)%
    --(5.190,8.381)--(5.190,8.381)--(5.190,8.381)--(5.190,8.381)--(5.190,8.381)%
    --(5.190,8.381)--(5.190,8.381)--(5.190,8.381)--(5.190,8.381)--(5.190,8.381)%
    --(5.190,8.381)--(5.190,8.381)--(5.190,8.381)--(5.190,8.381)--(5.190,8.381)%
    --(5.190,8.381)--cycle;
\gpfill{color=gp lt color border,opacity=0.50} (3.839,8.381)--(3.839,8.381)--(3.839,8.381)--(3.839,8.381)%
    --(3.839,8.381)--(3.839,8.381)--(3.839,8.381)--(3.839,8.381)--(3.839,8.381)%
    --(3.839,8.381)--(3.839,8.381)--(3.839,8.381)--(3.839,8.381)--(3.839,8.381)%
    --(3.839,8.381)--(3.839,8.381)--(3.839,8.381)--(3.839,8.381)--(3.839,8.381)%
    --(3.839,8.381)--(3.839,8.381)--(3.839,8.381)--(3.839,8.381)--(3.839,8.381)%
    --(3.839,8.381)--(3.839,8.381)--(3.839,8.381)--(3.839,8.381)--(3.839,8.381)%
    --(3.839,8.381)--(3.839,8.381)--(3.839,8.381)--(3.839,8.381)--(3.839,8.381)%
    --(3.839,8.381)--(3.839,8.381)--(3.839,8.381)--(3.839,8.381)--(3.839,8.381)%
    --(3.839,8.381)--(3.839,8.381)--(3.839,8.381)--(3.839,8.381)--(3.839,8.381)%
    --(3.839,8.381)--(3.839,8.381)--(3.839,8.381)--(3.839,8.381)--(3.839,8.381)%
    --(3.839,8.381)--(3.839,8.381)--(3.839,8.381)--(3.839,8.381)--(3.839,8.381)%
    --(3.839,8.381)--(3.839,8.381)--(3.839,8.381)--(3.839,8.381)--(3.839,8.381)%
    --(3.839,8.381)--(3.839,8.381)--(3.839,8.381)--(3.839,8.381)--(3.839,8.381)%
    --(3.839,8.381)--(3.839,8.381)--(3.839,8.381)--(3.839,8.381)--(3.839,8.381)%
    --(3.839,8.381)--(3.839,8.381)--(3.839,8.381)--(3.839,8.381)--(3.839,8.381)%
    --(3.839,8.381)--(3.839,8.381)--(3.839,8.381)--(3.839,8.381)--(3.839,8.381)%
    --(3.839,8.381)--(3.839,8.381)--(3.839,8.381)--(3.839,8.381)--(3.839,8.381)%
    --(3.839,8.381)--(3.839,8.381)--(3.839,8.381)--(3.839,8.381)--(3.839,8.381)%
    --(3.839,8.381)--(3.839,8.381)--(3.839,8.381)--(3.839,8.381)--(3.839,8.381)%
    --(3.839,8.381)--(3.839,8.381)--(3.839,8.381)--(3.839,8.381)--(3.839,8.381)%
    --(3.839,8.381)--(3.839,8.381)--(3.839,8.381)--(3.839,8.381)--(3.839,8.381)%
    --(3.839,8.381)--(3.839,8.381)--(3.839,8.381)--(3.839,8.381)--(3.839,8.381)%
    --(3.839,8.381)--(3.839,8.381)--(3.839,8.381)--(3.839,8.381)--(3.839,8.381)%
    --(3.839,8.381)--(3.839,8.381)--(3.839,8.381)--(3.839,8.381)--(3.839,8.381)%
    --(3.839,8.381)--cycle;
\gpfill{color=gp lt color border,opacity=0.50} (11.217,4.155)--(11.216,4.159)--(11.216,4.163)--(11.216,4.167)%
    --(11.215,4.171)--(11.214,4.175)--(11.213,4.180)--(11.211,4.184)--(11.209,4.187)%
    --(11.208,4.191)--(11.206,4.195)--(11.203,4.199)--(11.201,4.202)--(11.198,4.205)%
    --(11.196,4.209)--(11.193,4.212)--(11.190,4.215)--(11.186,4.217)--(11.183,4.220)%
    --(11.180,4.222)--(11.176,4.225)--(11.172,4.227)--(11.168,4.228)--(11.165,4.230)%
    --(11.161,4.232)--(11.156,4.233)--(11.152,4.234)--(11.148,4.235)--(11.144,4.235)%
    --(11.140,4.235)--(11.136,4.236)--(11.131,4.235)--(11.127,4.235)--(11.123,4.235)%
    --(11.119,4.234)--(11.115,4.233)--(11.110,4.232)--(11.106,4.230)--(11.103,4.228)%
    --(11.099,4.227)--(11.095,4.225)--(11.091,4.222)--(11.088,4.220)--(11.085,4.217)%
    --(11.081,4.215)--(11.078,4.212)--(11.075,4.209)--(11.073,4.205)--(11.070,4.202)%
    --(11.068,4.199)--(11.065,4.195)--(11.063,4.191)--(11.062,4.187)--(11.060,4.184)%
    --(11.058,4.180)--(11.057,4.175)--(11.056,4.171)--(11.055,4.167)--(11.055,4.163)%
    --(11.055,4.159)--(11.055,4.155)--(11.055,4.150)--(11.055,4.146)--(11.055,4.142)%
    --(11.056,4.138)--(11.057,4.134)--(11.058,4.129)--(11.060,4.125)--(11.062,4.122)%
    --(11.063,4.118)--(11.065,4.114)--(11.068,4.110)--(11.070,4.107)--(11.073,4.104)%
    --(11.075,4.100)--(11.078,4.097)--(11.081,4.094)--(11.085,4.092)--(11.088,4.089)%
    --(11.091,4.087)--(11.095,4.084)--(11.099,4.082)--(11.103,4.081)--(11.106,4.079)%
    --(11.110,4.077)--(11.115,4.076)--(11.119,4.075)--(11.123,4.074)--(11.127,4.074)%
    --(11.131,4.074)--(11.135,4.074)--(11.140,4.074)--(11.144,4.074)--(11.148,4.074)%
    --(11.152,4.075)--(11.156,4.076)--(11.161,4.077)--(11.165,4.079)--(11.168,4.081)%
    --(11.172,4.082)--(11.176,4.084)--(11.180,4.087)--(11.183,4.089)--(11.186,4.092)%
    --(11.190,4.094)--(11.193,4.097)--(11.196,4.100)--(11.198,4.104)--(11.201,4.107)%
    --(11.203,4.110)--(11.206,4.114)--(11.208,4.118)--(11.209,4.122)--(11.211,4.125)%
    --(11.213,4.129)--(11.214,4.134)--(11.215,4.138)--(11.216,4.142)--(11.216,4.146)%
    --(11.216,4.150)--(11.217,4.154)--cycle;
\gpfill{color=gp lt color border,opacity=0.50} (6.649,7.324)--(6.648,7.329)--(6.648,7.335)--(6.647,7.340)%
    --(6.646,7.346)--(6.645,7.351)--(6.643,7.357)--(6.641,7.362)--(6.639,7.367)%
    --(6.637,7.373)--(6.634,7.377)--(6.631,7.382)--(6.628,7.387)--(6.624,7.391)%
    --(6.621,7.396)--(6.617,7.400)--(6.613,7.404)--(6.608,7.407)--(6.604,7.411)%
    --(6.599,7.414)--(6.595,7.417)--(6.590,7.420)--(6.584,7.422)--(6.579,7.424)%
    --(6.574,7.426)--(6.568,7.428)--(6.563,7.429)--(6.557,7.430)--(6.552,7.431)%
    --(6.546,7.431)--(6.541,7.432)--(6.535,7.431)--(6.529,7.431)--(6.524,7.430)%
    --(6.518,7.429)--(6.513,7.428)--(6.507,7.426)--(6.502,7.424)--(6.497,7.422)%
    --(6.491,7.420)--(6.487,7.417)--(6.482,7.414)--(6.477,7.411)--(6.473,7.407)%
    --(6.468,7.404)--(6.464,7.400)--(6.460,7.396)--(6.457,7.391)--(6.453,7.387)%
    --(6.450,7.382)--(6.447,7.377)--(6.444,7.373)--(6.442,7.367)--(6.440,7.362)%
    --(6.438,7.357)--(6.436,7.351)--(6.435,7.346)--(6.434,7.340)--(6.433,7.335)%
    --(6.433,7.329)--(6.433,7.324)--(6.433,7.318)--(6.433,7.312)--(6.434,7.307)%
    --(6.435,7.301)--(6.436,7.296)--(6.438,7.290)--(6.440,7.285)--(6.442,7.280)%
    --(6.444,7.274)--(6.447,7.269)--(6.450,7.265)--(6.453,7.260)--(6.457,7.256)%
    --(6.460,7.251)--(6.464,7.247)--(6.468,7.243)--(6.473,7.240)--(6.477,7.236)%
    --(6.482,7.233)--(6.486,7.230)--(6.491,7.227)--(6.497,7.225)--(6.502,7.223)%
    --(6.507,7.221)--(6.513,7.219)--(6.518,7.218)--(6.524,7.217)--(6.529,7.216)%
    --(6.535,7.216)--(6.540,7.216)--(6.546,7.216)--(6.552,7.216)--(6.557,7.217)%
    --(6.563,7.218)--(6.568,7.219)--(6.574,7.221)--(6.579,7.223)--(6.584,7.225)%
    --(6.590,7.227)--(6.595,7.230)--(6.599,7.233)--(6.604,7.236)--(6.608,7.240)%
    --(6.613,7.243)--(6.617,7.247)--(6.621,7.251)--(6.624,7.256)--(6.628,7.260)%
    --(6.631,7.265)--(6.634,7.269)--(6.637,7.274)--(6.639,7.280)--(6.641,7.285)%
    --(6.643,7.290)--(6.645,7.296)--(6.646,7.301)--(6.647,7.307)--(6.648,7.312)%
    --(6.648,7.318)--(6.649,7.323)--cycle;
\gpfill{color=gp lt color border,opacity=0.50} (1.136,5.211)--(1.136,5.211)--(1.136,5.211)--(1.136,5.211)%
    --(1.136,5.211)--(1.136,5.211)--(1.136,5.211)--(1.136,5.211)--(1.136,5.211)%
    --(1.136,5.211)--(1.136,5.211)--(1.136,5.211)--(1.136,5.211)--(1.136,5.211)%
    --(1.136,5.211)--(1.136,5.211)--(1.136,5.211)--(1.136,5.211)--(1.136,5.211)%
    --(1.136,5.211)--(1.136,5.211)--(1.136,5.211)--(1.136,5.211)--(1.136,5.211)%
    --(1.136,5.211)--(1.136,5.211)--(1.136,5.211)--(1.136,5.211)--(1.136,5.211)%
    --(1.136,5.211)--(1.136,5.211)--(1.136,5.211)--(1.136,5.211)--(1.136,5.211)%
    --(1.136,5.211)--(1.136,5.211)--(1.136,5.211)--(1.136,5.211)--(1.136,5.211)%
    --(1.136,5.211)--(1.136,5.211)--(1.136,5.211)--(1.136,5.211)--(1.136,5.211)%
    --(1.136,5.211)--(1.136,5.211)--(1.136,5.211)--(1.136,5.211)--(1.136,5.211)%
    --(1.136,5.211)--(1.136,5.211)--(1.136,5.211)--(1.136,5.211)--(1.136,5.211)%
    --(1.136,5.211)--(1.136,5.211)--(1.136,5.211)--(1.136,5.211)--(1.136,5.211)%
    --(1.136,5.211)--(1.136,5.211)--(1.136,5.211)--(1.136,5.211)--(1.136,5.211)%
    --(1.136,5.211)--(1.136,5.211)--(1.136,5.211)--(1.136,5.211)--(1.136,5.211)%
    --(1.136,5.211)--(1.136,5.211)--(1.136,5.211)--(1.136,5.211)--(1.136,5.211)%
    --(1.136,5.211)--(1.136,5.211)--(1.136,5.211)--(1.136,5.211)--(1.136,5.211)%
    --(1.136,5.211)--(1.136,5.211)--(1.136,5.211)--(1.136,5.211)--(1.136,5.211)%
    --(1.136,5.211)--(1.136,5.211)--(1.136,5.211)--(1.136,5.211)--(1.136,5.211)%
    --(1.136,5.211)--(1.136,5.211)--(1.136,5.211)--(1.136,5.211)--(1.136,5.211)%
    --(1.136,5.211)--(1.136,5.211)--(1.136,5.211)--(1.136,5.211)--(1.136,5.211)%
    --(1.136,5.211)--(1.136,5.211)--(1.136,5.211)--(1.136,5.211)--(1.136,5.211)%
    --(1.136,5.211)--(1.136,5.211)--(1.136,5.211)--(1.136,5.211)--(1.136,5.211)%
    --(1.136,5.211)--(1.136,5.211)--(1.136,5.211)--(1.136,5.211)--(1.136,5.211)%
    --(1.136,5.211)--(1.136,5.211)--(1.136,5.211)--(1.136,5.211)--(1.136,5.211)%
    --(1.136,5.211)--cycle;
\gpfill{color=gp lt color border,opacity=0.50} (2.487,8.381)--(2.487,8.381)--(2.487,8.381)--(2.487,8.381)%
    --(2.487,8.381)--(2.487,8.381)--(2.487,8.381)--(2.487,8.381)--(2.487,8.381)%
    --(2.487,8.381)--(2.487,8.381)--(2.487,8.381)--(2.487,8.381)--(2.487,8.381)%
    --(2.487,8.381)--(2.487,8.381)--(2.487,8.381)--(2.487,8.381)--(2.487,8.381)%
    --(2.487,8.381)--(2.487,8.381)--(2.487,8.381)--(2.487,8.381)--(2.487,8.381)%
    --(2.487,8.381)--(2.487,8.381)--(2.487,8.381)--(2.487,8.381)--(2.487,8.381)%
    --(2.487,8.381)--(2.487,8.381)--(2.487,8.381)--(2.487,8.381)--(2.487,8.381)%
    --(2.487,8.381)--(2.487,8.381)--(2.487,8.381)--(2.487,8.381)--(2.487,8.381)%
    --(2.487,8.381)--(2.487,8.381)--(2.487,8.381)--(2.487,8.381)--(2.487,8.381)%
    --(2.487,8.381)--(2.487,8.381)--(2.487,8.381)--(2.487,8.381)--(2.487,8.381)%
    --(2.487,8.381)--(2.487,8.381)--(2.487,8.381)--(2.487,8.381)--(2.487,8.381)%
    --(2.487,8.381)--(2.487,8.381)--(2.487,8.381)--(2.487,8.381)--(2.487,8.381)%
    --(2.487,8.381)--(2.487,8.381)--(2.487,8.381)--(2.487,8.381)--(2.487,8.381)%
    --(2.487,8.381)--(2.487,8.381)--(2.487,8.381)--(2.487,8.381)--(2.487,8.381)%
    --(2.487,8.381)--(2.487,8.381)--(2.487,8.381)--(2.487,8.381)--(2.487,8.381)%
    --(2.487,8.381)--(2.487,8.381)--(2.487,8.381)--(2.487,8.381)--(2.487,8.381)%
    --(2.487,8.381)--(2.487,8.381)--(2.487,8.381)--(2.487,8.381)--(2.487,8.381)%
    --(2.487,8.381)--(2.487,8.381)--(2.487,8.381)--(2.487,8.381)--(2.487,8.381)%
    --(2.487,8.381)--(2.487,8.381)--(2.487,8.381)--(2.487,8.381)--(2.487,8.381)%
    --(2.487,8.381)--(2.487,8.381)--(2.487,8.381)--(2.487,8.381)--(2.487,8.381)%
    --(2.487,8.381)--(2.487,8.381)--(2.487,8.381)--(2.487,8.381)--(2.487,8.381)%
    --(2.487,8.381)--(2.487,8.381)--(2.487,8.381)--(2.487,8.381)--(2.487,8.381)%
    --(2.487,8.381)--(2.487,8.381)--(2.487,8.381)--(2.487,8.381)--(2.487,8.381)%
    --(2.487,8.381)--(2.487,8.381)--(2.487,8.381)--(2.487,8.381)--(2.487,8.381)%
    --(2.487,8.381)--cycle;
\gpfill{color=gp lt color border,opacity=0.50} (5.217,7.324)--(5.216,7.325)--(5.216,7.326)--(5.216,7.328)%
    --(5.216,7.329)--(5.216,7.330)--(5.215,7.332)--(5.215,7.333)--(5.214,7.334)%
    --(5.214,7.336)--(5.213,7.337)--(5.212,7.338)--(5.211,7.339)--(5.210,7.340)%
    --(5.210,7.342)--(5.209,7.343)--(5.208,7.344)--(5.206,7.344)--(5.205,7.345)%
    --(5.204,7.346)--(5.203,7.347)--(5.202,7.348)--(5.200,7.348)--(5.199,7.349)%
    --(5.198,7.349)--(5.196,7.350)--(5.195,7.350)--(5.194,7.350)--(5.192,7.350)%
    --(5.191,7.350)--(5.190,7.351)--(5.188,7.350)--(5.187,7.350)--(5.185,7.350)%
    --(5.184,7.350)--(5.183,7.350)--(5.181,7.349)--(5.180,7.349)--(5.179,7.348)%
    --(5.177,7.348)--(5.176,7.347)--(5.175,7.346)--(5.174,7.345)--(5.173,7.344)%
    --(5.171,7.344)--(5.170,7.343)--(5.169,7.342)--(5.169,7.340)--(5.168,7.339)%
    --(5.167,7.338)--(5.166,7.337)--(5.165,7.336)--(5.165,7.334)--(5.164,7.333)%
    --(5.164,7.332)--(5.163,7.330)--(5.163,7.329)--(5.163,7.328)--(5.163,7.326)%
    --(5.163,7.325)--(5.163,7.324)--(5.163,7.322)--(5.163,7.321)--(5.163,7.319)%
    --(5.163,7.318)--(5.163,7.317)--(5.164,7.315)--(5.164,7.314)--(5.165,7.313)%
    --(5.165,7.311)--(5.166,7.310)--(5.167,7.309)--(5.168,7.308)--(5.169,7.307)%
    --(5.169,7.305)--(5.170,7.304)--(5.171,7.303)--(5.173,7.303)--(5.174,7.302)%
    --(5.175,7.301)--(5.176,7.300)--(5.177,7.299)--(5.179,7.299)--(5.180,7.298)%
    --(5.181,7.298)--(5.183,7.297)--(5.184,7.297)--(5.185,7.297)--(5.187,7.297)%
    --(5.188,7.297)--(5.189,7.297)--(5.191,7.297)--(5.192,7.297)--(5.194,7.297)%
    --(5.195,7.297)--(5.196,7.297)--(5.198,7.298)--(5.199,7.298)--(5.200,7.299)%
    --(5.202,7.299)--(5.203,7.300)--(5.204,7.301)--(5.205,7.302)--(5.206,7.303)%
    --(5.208,7.303)--(5.209,7.304)--(5.210,7.305)--(5.210,7.307)--(5.211,7.308)%
    --(5.212,7.309)--(5.213,7.310)--(5.214,7.311)--(5.214,7.313)--(5.215,7.314)%
    --(5.215,7.315)--(5.216,7.317)--(5.216,7.318)--(5.216,7.319)--(5.216,7.321)%
    --(5.216,7.322)--(5.217,7.323)--cycle;
\gpfill{color=gp lt color border,opacity=0.50} (11.136,3.098)--(11.136,3.098)--(11.136,3.098)--(11.136,3.098)%
    --(11.136,3.098)--(11.136,3.098)--(11.136,3.098)--(11.136,3.098)--(11.136,3.098)%
    --(11.136,3.098)--(11.136,3.098)--(11.136,3.098)--(11.136,3.098)--(11.136,3.098)%
    --(11.136,3.098)--(11.136,3.098)--(11.136,3.098)--(11.136,3.098)--(11.136,3.098)%
    --(11.136,3.098)--(11.136,3.098)--(11.136,3.098)--(11.136,3.098)--(11.136,3.098)%
    --(11.136,3.098)--(11.136,3.098)--(11.136,3.098)--(11.136,3.098)--(11.136,3.098)%
    --(11.136,3.098)--(11.136,3.098)--(11.136,3.098)--(11.136,3.098)--(11.136,3.098)%
    --(11.136,3.098)--(11.136,3.098)--(11.136,3.098)--(11.136,3.098)--(11.136,3.098)%
    --(11.136,3.098)--(11.136,3.098)--(11.136,3.098)--(11.136,3.098)--(11.136,3.098)%
    --(11.136,3.098)--(11.136,3.098)--(11.136,3.098)--(11.136,3.098)--(11.136,3.098)%
    --(11.136,3.098)--(11.136,3.098)--(11.136,3.098)--(11.136,3.098)--(11.136,3.098)%
    --(11.136,3.098)--(11.136,3.098)--(11.136,3.098)--(11.136,3.098)--(11.136,3.098)%
    --(11.136,3.098)--(11.136,3.098)--(11.136,3.098)--(11.136,3.098)--(11.136,3.098)%
    --(11.136,3.098)--(11.136,3.098)--(11.136,3.098)--(11.136,3.098)--(11.136,3.098)%
    --(11.136,3.098)--(11.136,3.098)--(11.136,3.098)--(11.136,3.098)--(11.136,3.098)%
    --(11.136,3.098)--(11.136,3.098)--(11.136,3.098)--(11.136,3.098)--(11.136,3.098)%
    --(11.136,3.098)--(11.136,3.098)--(11.136,3.098)--(11.136,3.098)--(11.136,3.098)%
    --(11.136,3.098)--(11.136,3.098)--(11.136,3.098)--(11.136,3.098)--(11.136,3.098)%
    --(11.136,3.098)--(11.136,3.098)--(11.136,3.098)--(11.136,3.098)--(11.136,3.098)%
    --(11.136,3.098)--(11.136,3.098)--(11.136,3.098)--(11.136,3.098)--(11.136,3.098)%
    --(11.136,3.098)--(11.136,3.098)--(11.136,3.098)--(11.136,3.098)--(11.136,3.098)%
    --(11.136,3.098)--(11.136,3.098)--(11.136,3.098)--(11.136,3.098)--(11.136,3.098)%
    --(11.136,3.098)--(11.136,3.098)--(11.136,3.098)--(11.136,3.098)--(11.136,3.098)%
    --(11.136,3.098)--(11.136,3.098)--(11.136,3.098)--(11.136,3.098)--(11.136,3.098)%
    --(11.136,3.098)--cycle;
\gpfill{color=gp lt color border,opacity=0.50} (11.947,4.155)--(11.947,4.155)--(11.947,4.155)--(11.947,4.155)%
    --(11.947,4.155)--(11.947,4.155)--(11.947,4.155)--(11.947,4.155)--(11.947,4.155)%
    --(11.947,4.155)--(11.947,4.155)--(11.947,4.155)--(11.947,4.155)--(11.947,4.155)%
    --(11.947,4.155)--(11.947,4.155)--(11.947,4.155)--(11.947,4.155)--(11.947,4.155)%
    --(11.947,4.155)--(11.947,4.155)--(11.947,4.155)--(11.947,4.155)--(11.947,4.155)%
    --(11.947,4.155)--(11.947,4.155)--(11.947,4.155)--(11.947,4.155)--(11.947,4.155)%
    --(11.947,4.155)--(11.947,4.155)--(11.947,4.155)--(11.947,4.155)--(11.947,4.155)%
    --(11.947,4.155)--(11.947,4.155)--(11.947,4.155)--(11.947,4.155)--(11.947,4.155)%
    --(11.947,4.155)--(11.947,4.155)--(11.947,4.155)--(11.947,4.155)--(11.947,4.155)%
    --(11.947,4.155)--(11.947,4.155)--(11.947,4.155)--(11.947,4.155)--(11.947,4.155)%
    --(11.947,4.155)--(11.947,4.155)--(11.947,4.155)--(11.947,4.155)--(11.947,4.155)%
    --(11.947,4.155)--(11.947,4.155)--(11.947,4.155)--(11.947,4.155)--(11.947,4.155)%
    --(11.947,4.155)--(11.947,4.155)--(11.947,4.155)--(11.947,4.155)--(11.947,4.155)%
    --(11.947,4.155)--(11.947,4.155)--(11.947,4.155)--(11.947,4.155)--(11.947,4.155)%
    --(11.947,4.155)--(11.947,4.155)--(11.947,4.155)--(11.947,4.155)--(11.947,4.155)%
    --(11.947,4.155)--(11.947,4.155)--(11.947,4.155)--(11.947,4.155)--(11.947,4.155)%
    --(11.947,4.155)--(11.947,4.155)--(11.947,4.155)--(11.947,4.155)--(11.947,4.155)%
    --(11.947,4.155)--(11.947,4.155)--(11.947,4.155)--(11.947,4.155)--(11.947,4.155)%
    --(11.947,4.155)--(11.947,4.155)--(11.947,4.155)--(11.947,4.155)--(11.947,4.155)%
    --(11.947,4.155)--(11.947,4.155)--(11.947,4.155)--(11.947,4.155)--(11.947,4.155)%
    --(11.947,4.155)--(11.947,4.155)--(11.947,4.155)--(11.947,4.155)--(11.947,4.155)%
    --(11.947,4.155)--(11.947,4.155)--(11.947,4.155)--(11.947,4.155)--(11.947,4.155)%
    --(11.947,4.155)--(11.947,4.155)--(11.947,4.155)--(11.947,4.155)--(11.947,4.155)%
    --(11.947,4.155)--(11.947,4.155)--(11.947,4.155)--(11.947,4.155)--(11.947,4.155)%
    --(11.947,4.155)--cycle;
\gpfill{color=gp lt color border,opacity=0.50} (8.622,4.155)--(8.621,4.164)--(8.620,4.174)--(8.619,4.184)%
    --(8.617,4.194)--(8.615,4.203)--(8.612,4.213)--(8.609,4.222)--(8.605,4.231)%
    --(8.601,4.240)--(8.596,4.249)--(8.591,4.257)--(8.585,4.266)--(8.579,4.273)%
    --(8.573,4.281)--(8.566,4.288)--(8.559,4.295)--(8.551,4.301)--(8.544,4.307)%
    --(8.535,4.313)--(8.527,4.318)--(8.518,4.323)--(8.509,4.327)--(8.500,4.331)%
    --(8.491,4.334)--(8.481,4.337)--(8.472,4.339)--(8.462,4.341)--(8.452,4.342)%
    --(8.442,4.343)--(8.433,4.344)--(8.423,4.343)--(8.413,4.342)--(8.403,4.341)%
    --(8.393,4.339)--(8.384,4.337)--(8.374,4.334)--(8.365,4.331)--(8.356,4.327)%
    --(8.347,4.323)--(8.338,4.318)--(8.330,4.313)--(8.321,4.307)--(8.314,4.301)%
    --(8.306,4.295)--(8.299,4.288)--(8.292,4.281)--(8.286,4.273)--(8.280,4.266)%
    --(8.274,4.257)--(8.269,4.249)--(8.264,4.240)--(8.260,4.231)--(8.256,4.222)%
    --(8.253,4.213)--(8.250,4.203)--(8.248,4.194)--(8.246,4.184)--(8.245,4.174)%
    --(8.244,4.164)--(8.244,4.155)--(8.244,4.145)--(8.245,4.135)--(8.246,4.125)%
    --(8.248,4.115)--(8.250,4.106)--(8.253,4.096)--(8.256,4.087)--(8.260,4.078)%
    --(8.264,4.069)--(8.269,4.060)--(8.274,4.052)--(8.280,4.043)--(8.286,4.036)%
    --(8.292,4.028)--(8.299,4.021)--(8.306,4.014)--(8.314,4.008)--(8.321,4.002)%
    --(8.330,3.996)--(8.338,3.991)--(8.347,3.986)--(8.356,3.982)--(8.365,3.978)%
    --(8.374,3.975)--(8.384,3.972)--(8.393,3.970)--(8.403,3.968)--(8.413,3.967)%
    --(8.423,3.966)--(8.432,3.966)--(8.442,3.966)--(8.452,3.967)--(8.462,3.968)%
    --(8.472,3.970)--(8.481,3.972)--(8.491,3.975)--(8.500,3.978)--(8.509,3.982)%
    --(8.518,3.986)--(8.527,3.991)--(8.535,3.996)--(8.544,4.002)--(8.551,4.008)%
    --(8.559,4.014)--(8.566,4.021)--(8.573,4.028)--(8.579,4.036)--(8.585,4.043)%
    --(8.591,4.052)--(8.596,4.060)--(8.601,4.069)--(8.605,4.078)--(8.609,4.087)%
    --(8.612,4.096)--(8.615,4.106)--(8.617,4.115)--(8.619,4.125)--(8.620,4.135)%
    --(8.621,4.145)--(8.622,4.154)--cycle;
\gpfill{color=gp lt color border,opacity=0.50} (3.866,7.324)--(3.865,7.325)--(3.865,7.326)--(3.865,7.328)%
    --(3.865,7.329)--(3.865,7.330)--(3.864,7.332)--(3.864,7.333)--(3.863,7.334)%
    --(3.863,7.336)--(3.862,7.337)--(3.861,7.338)--(3.860,7.339)--(3.859,7.340)%
    --(3.859,7.342)--(3.858,7.343)--(3.857,7.344)--(3.855,7.344)--(3.854,7.345)%
    --(3.853,7.346)--(3.852,7.347)--(3.851,7.348)--(3.849,7.348)--(3.848,7.349)%
    --(3.847,7.349)--(3.845,7.350)--(3.844,7.350)--(3.843,7.350)--(3.841,7.350)%
    --(3.840,7.350)--(3.839,7.351)--(3.837,7.350)--(3.836,7.350)--(3.834,7.350)%
    --(3.833,7.350)--(3.832,7.350)--(3.830,7.349)--(3.829,7.349)--(3.828,7.348)%
    --(3.826,7.348)--(3.825,7.347)--(3.824,7.346)--(3.823,7.345)--(3.822,7.344)%
    --(3.820,7.344)--(3.819,7.343)--(3.818,7.342)--(3.818,7.340)--(3.817,7.339)%
    --(3.816,7.338)--(3.815,7.337)--(3.814,7.336)--(3.814,7.334)--(3.813,7.333)%
    --(3.813,7.332)--(3.812,7.330)--(3.812,7.329)--(3.812,7.328)--(3.812,7.326)%
    --(3.812,7.325)--(3.812,7.324)--(3.812,7.322)--(3.812,7.321)--(3.812,7.319)%
    --(3.812,7.318)--(3.812,7.317)--(3.813,7.315)--(3.813,7.314)--(3.814,7.313)%
    --(3.814,7.311)--(3.815,7.310)--(3.816,7.309)--(3.817,7.308)--(3.818,7.307)%
    --(3.818,7.305)--(3.819,7.304)--(3.820,7.303)--(3.822,7.303)--(3.823,7.302)%
    --(3.824,7.301)--(3.825,7.300)--(3.826,7.299)--(3.828,7.299)--(3.829,7.298)%
    --(3.830,7.298)--(3.832,7.297)--(3.833,7.297)--(3.834,7.297)--(3.836,7.297)%
    --(3.837,7.297)--(3.838,7.297)--(3.840,7.297)--(3.841,7.297)--(3.843,7.297)%
    --(3.844,7.297)--(3.845,7.297)--(3.847,7.298)--(3.848,7.298)--(3.849,7.299)%
    --(3.851,7.299)--(3.852,7.300)--(3.853,7.301)--(3.854,7.302)--(3.855,7.303)%
    --(3.857,7.303)--(3.858,7.304)--(3.859,7.305)--(3.859,7.307)--(3.860,7.308)%
    --(3.861,7.309)--(3.862,7.310)--(3.863,7.311)--(3.863,7.313)--(3.864,7.314)%
    --(3.864,7.315)--(3.865,7.317)--(3.865,7.318)--(3.865,7.319)--(3.865,7.321)%
    --(3.865,7.322)--(3.866,7.323)--cycle;
\gpfill{color=gp lt color border,opacity=0.50} (2.487,7.324)--(2.487,7.324)--(2.487,7.324)--(2.487,7.324)%
    --(2.487,7.324)--(2.487,7.324)--(2.487,7.324)--(2.487,7.324)--(2.487,7.324)%
    --(2.487,7.324)--(2.487,7.324)--(2.487,7.324)--(2.487,7.324)--(2.487,7.324)%
    --(2.487,7.324)--(2.487,7.324)--(2.487,7.324)--(2.487,7.324)--(2.487,7.324)%
    --(2.487,7.324)--(2.487,7.324)--(2.487,7.324)--(2.487,7.324)--(2.487,7.324)%
    --(2.487,7.324)--(2.487,7.324)--(2.487,7.324)--(2.487,7.324)--(2.487,7.324)%
    --(2.487,7.324)--(2.487,7.324)--(2.487,7.324)--(2.487,7.324)--(2.487,7.324)%
    --(2.487,7.324)--(2.487,7.324)--(2.487,7.324)--(2.487,7.324)--(2.487,7.324)%
    --(2.487,7.324)--(2.487,7.324)--(2.487,7.324)--(2.487,7.324)--(2.487,7.324)%
    --(2.487,7.324)--(2.487,7.324)--(2.487,7.324)--(2.487,7.324)--(2.487,7.324)%
    --(2.487,7.324)--(2.487,7.324)--(2.487,7.324)--(2.487,7.324)--(2.487,7.324)%
    --(2.487,7.324)--(2.487,7.324)--(2.487,7.324)--(2.487,7.324)--(2.487,7.324)%
    --(2.487,7.324)--(2.487,7.324)--(2.487,7.324)--(2.487,7.324)--(2.487,7.324)%
    --(2.487,7.324)--(2.487,7.324)--(2.487,7.324)--(2.487,7.324)--(2.487,7.324)%
    --(2.487,7.324)--(2.487,7.324)--(2.487,7.324)--(2.487,7.324)--(2.487,7.324)%
    --(2.487,7.324)--(2.487,7.324)--(2.487,7.324)--(2.487,7.324)--(2.487,7.324)%
    --(2.487,7.324)--(2.487,7.324)--(2.487,7.324)--(2.487,7.324)--(2.487,7.324)%
    --(2.487,7.324)--(2.487,7.324)--(2.487,7.324)--(2.487,7.324)--(2.487,7.324)%
    --(2.487,7.324)--(2.487,7.324)--(2.487,7.324)--(2.487,7.324)--(2.487,7.324)%
    --(2.487,7.324)--(2.487,7.324)--(2.487,7.324)--(2.487,7.324)--(2.487,7.324)%
    --(2.487,7.324)--(2.487,7.324)--(2.487,7.324)--(2.487,7.324)--(2.487,7.324)%
    --(2.487,7.324)--(2.487,7.324)--(2.487,7.324)--(2.487,7.324)--(2.487,7.324)%
    --(2.487,7.324)--(2.487,7.324)--(2.487,7.324)--(2.487,7.324)--(2.487,7.324)%
    --(2.487,7.324)--(2.487,7.324)--(2.487,7.324)--(2.487,7.324)--(2.487,7.324)%
    --(2.487,7.324)--cycle;
\gpfill{color=gp lt color border,opacity=0.50} (9.460,4.155)--(9.459,4.166)--(9.458,4.177)--(9.457,4.188)%
    --(9.455,4.199)--(9.452,4.210)--(9.449,4.221)--(9.445,4.232)--(9.441,4.242)%
    --(9.436,4.253)--(9.431,4.262)--(9.425,4.272)--(9.418,4.281)--(9.411,4.290)%
    --(9.404,4.299)--(9.396,4.307)--(9.388,4.315)--(9.379,4.322)--(9.370,4.329)%
    --(9.361,4.336)--(9.352,4.342)--(9.342,4.347)--(9.331,4.352)--(9.321,4.356)%
    --(9.310,4.360)--(9.299,4.363)--(9.288,4.366)--(9.277,4.368)--(9.266,4.369)%
    --(9.255,4.370)--(9.244,4.371)--(9.232,4.370)--(9.221,4.369)--(9.210,4.368)%
    --(9.199,4.366)--(9.188,4.363)--(9.177,4.360)--(9.166,4.356)--(9.156,4.352)%
    --(9.145,4.347)--(9.136,4.342)--(9.126,4.336)--(9.117,4.329)--(9.108,4.322)%
    --(9.099,4.315)--(9.091,4.307)--(9.083,4.299)--(9.076,4.290)--(9.069,4.281)%
    --(9.062,4.272)--(9.056,4.262)--(9.051,4.253)--(9.046,4.242)--(9.042,4.232)%
    --(9.038,4.221)--(9.035,4.210)--(9.032,4.199)--(9.030,4.188)--(9.029,4.177)%
    --(9.028,4.166)--(9.028,4.155)--(9.028,4.143)--(9.029,4.132)--(9.030,4.121)%
    --(9.032,4.110)--(9.035,4.099)--(9.038,4.088)--(9.042,4.077)--(9.046,4.067)%
    --(9.051,4.056)--(9.056,4.046)--(9.062,4.037)--(9.069,4.028)--(9.076,4.019)%
    --(9.083,4.010)--(9.091,4.002)--(9.099,3.994)--(9.108,3.987)--(9.117,3.980)%
    --(9.126,3.973)--(9.135,3.967)--(9.145,3.962)--(9.156,3.957)--(9.166,3.953)%
    --(9.177,3.949)--(9.188,3.946)--(9.199,3.943)--(9.210,3.941)--(9.221,3.940)%
    --(9.232,3.939)--(9.243,3.939)--(9.255,3.939)--(9.266,3.940)--(9.277,3.941)%
    --(9.288,3.943)--(9.299,3.946)--(9.310,3.949)--(9.321,3.953)--(9.331,3.957)%
    --(9.342,3.962)--(9.352,3.967)--(9.361,3.973)--(9.370,3.980)--(9.379,3.987)%
    --(9.388,3.994)--(9.396,4.002)--(9.404,4.010)--(9.411,4.019)--(9.418,4.028)%
    --(9.425,4.037)--(9.431,4.046)--(9.436,4.056)--(9.441,4.067)--(9.445,4.077)%
    --(9.449,4.088)--(9.452,4.099)--(9.455,4.110)--(9.457,4.121)--(9.458,4.132)%
    --(9.459,4.143)--(9.460,4.154)--cycle;
\gpfill{color=gp lt color border,opacity=0.50} (6.029,4.155)--(6.028,4.170)--(6.027,4.186)--(6.025,4.201)%
    --(6.022,4.216)--(6.018,4.232)--(6.014,4.247)--(6.009,4.261)--(6.003,4.276)%
    --(5.996,4.290)--(5.989,4.303)--(5.980,4.317)--(5.972,4.330)--(5.962,4.342)%
    --(5.952,4.354)--(5.941,4.365)--(5.930,4.376)--(5.918,4.386)--(5.906,4.396)%
    --(5.893,4.404)--(5.880,4.413)--(5.866,4.420)--(5.852,4.427)--(5.837,4.433)%
    --(5.823,4.438)--(5.808,4.442)--(5.792,4.446)--(5.777,4.449)--(5.762,4.451)%
    --(5.746,4.452)--(5.731,4.453)--(5.715,4.452)--(5.699,4.451)--(5.684,4.449)%
    --(5.669,4.446)--(5.653,4.442)--(5.638,4.438)--(5.624,4.433)--(5.609,4.427)%
    --(5.595,4.420)--(5.582,4.413)--(5.568,4.404)--(5.555,4.396)--(5.543,4.386)%
    --(5.531,4.376)--(5.520,4.365)--(5.509,4.354)--(5.499,4.342)--(5.489,4.330)%
    --(5.481,4.317)--(5.472,4.303)--(5.465,4.290)--(5.458,4.276)--(5.452,4.261)%
    --(5.447,4.247)--(5.443,4.232)--(5.439,4.216)--(5.436,4.201)--(5.434,4.186)%
    --(5.433,4.170)--(5.433,4.155)--(5.433,4.139)--(5.434,4.123)--(5.436,4.108)%
    --(5.439,4.093)--(5.443,4.077)--(5.447,4.062)--(5.452,4.048)--(5.458,4.033)%
    --(5.465,4.019)--(5.472,4.005)--(5.481,3.992)--(5.489,3.979)--(5.499,3.967)%
    --(5.509,3.955)--(5.520,3.944)--(5.531,3.933)--(5.543,3.923)--(5.555,3.913)%
    --(5.568,3.905)--(5.581,3.896)--(5.595,3.889)--(5.609,3.882)--(5.624,3.876)%
    --(5.638,3.871)--(5.653,3.867)--(5.669,3.863)--(5.684,3.860)--(5.699,3.858)%
    --(5.715,3.857)--(5.730,3.857)--(5.746,3.857)--(5.762,3.858)--(5.777,3.860)%
    --(5.792,3.863)--(5.808,3.867)--(5.823,3.871)--(5.837,3.876)--(5.852,3.882)%
    --(5.866,3.889)--(5.880,3.896)--(5.893,3.905)--(5.906,3.913)--(5.918,3.923)%
    --(5.930,3.933)--(5.941,3.944)--(5.952,3.955)--(5.962,3.967)--(5.972,3.979)%
    --(5.980,3.992)--(5.989,4.005)--(5.996,4.019)--(6.003,4.033)--(6.009,4.048)%
    --(6.014,4.062)--(6.018,4.077)--(6.022,4.093)--(6.025,4.108)--(6.027,4.123)%
    --(6.028,4.139)--(6.029,4.154)--cycle;
\gpfill{color=gp lt color border,opacity=0.50} (8.487,3.098)--(8.486,3.100)--(8.486,3.103)--(8.486,3.106)%
    --(8.485,3.109)--(8.485,3.111)--(8.484,3.114)--(8.483,3.117)--(8.482,3.119)%
    --(8.481,3.122)--(8.479,3.124)--(8.478,3.127)--(8.476,3.129)--(8.474,3.131)%
    --(8.473,3.134)--(8.471,3.136)--(8.469,3.138)--(8.466,3.139)--(8.464,3.141)%
    --(8.462,3.143)--(8.460,3.144)--(8.457,3.146)--(8.454,3.147)--(8.452,3.148)%
    --(8.449,3.149)--(8.446,3.150)--(8.444,3.150)--(8.441,3.151)--(8.438,3.151)%
    --(8.435,3.151)--(8.433,3.152)--(8.430,3.151)--(8.427,3.151)--(8.424,3.151)%
    --(8.421,3.150)--(8.419,3.150)--(8.416,3.149)--(8.413,3.148)--(8.411,3.147)%
    --(8.408,3.146)--(8.406,3.144)--(8.403,3.143)--(8.401,3.141)--(8.399,3.139)%
    --(8.396,3.138)--(8.394,3.136)--(8.392,3.134)--(8.391,3.131)--(8.389,3.129)%
    --(8.387,3.127)--(8.386,3.124)--(8.384,3.122)--(8.383,3.119)--(8.382,3.117)%
    --(8.381,3.114)--(8.380,3.111)--(8.380,3.109)--(8.379,3.106)--(8.379,3.103)%
    --(8.379,3.100)--(8.379,3.098)--(8.379,3.095)--(8.379,3.092)--(8.379,3.089)%
    --(8.380,3.086)--(8.380,3.084)--(8.381,3.081)--(8.382,3.078)--(8.383,3.076)%
    --(8.384,3.073)--(8.386,3.070)--(8.387,3.068)--(8.389,3.066)--(8.391,3.064)%
    --(8.392,3.061)--(8.394,3.059)--(8.396,3.057)--(8.399,3.056)--(8.401,3.054)%
    --(8.403,3.052)--(8.405,3.051)--(8.408,3.049)--(8.411,3.048)--(8.413,3.047)%
    --(8.416,3.046)--(8.419,3.045)--(8.421,3.045)--(8.424,3.044)--(8.427,3.044)%
    --(8.430,3.044)--(8.432,3.044)--(8.435,3.044)--(8.438,3.044)--(8.441,3.044)%
    --(8.444,3.045)--(8.446,3.045)--(8.449,3.046)--(8.452,3.047)--(8.454,3.048)%
    --(8.457,3.049)--(8.460,3.051)--(8.462,3.052)--(8.464,3.054)--(8.466,3.056)%
    --(8.469,3.057)--(8.471,3.059)--(8.473,3.061)--(8.474,3.064)--(8.476,3.066)%
    --(8.478,3.068)--(8.479,3.070)--(8.481,3.073)--(8.482,3.076)--(8.483,3.078)%
    --(8.484,3.081)--(8.485,3.084)--(8.485,3.086)--(8.486,3.089)--(8.486,3.092)%
    --(8.486,3.095)--(8.487,3.097)--cycle;
\gpfill{color=gp lt color border,opacity=0.50} (11.136,6.268)--(11.136,6.268)--(11.136,6.268)--(11.136,6.268)%
    --(11.136,6.268)--(11.136,6.268)--(11.136,6.268)--(11.136,6.268)--(11.136,6.268)%
    --(11.136,6.268)--(11.136,6.268)--(11.136,6.268)--(11.136,6.268)--(11.136,6.268)%
    --(11.136,6.268)--(11.136,6.268)--(11.136,6.268)--(11.136,6.268)--(11.136,6.268)%
    --(11.136,6.268)--(11.136,6.268)--(11.136,6.268)--(11.136,6.268)--(11.136,6.268)%
    --(11.136,6.268)--(11.136,6.268)--(11.136,6.268)--(11.136,6.268)--(11.136,6.268)%
    --(11.136,6.268)--(11.136,6.268)--(11.136,6.268)--(11.136,6.268)--(11.136,6.268)%
    --(11.136,6.268)--(11.136,6.268)--(11.136,6.268)--(11.136,6.268)--(11.136,6.268)%
    --(11.136,6.268)--(11.136,6.268)--(11.136,6.268)--(11.136,6.268)--(11.136,6.268)%
    --(11.136,6.268)--(11.136,6.268)--(11.136,6.268)--(11.136,6.268)--(11.136,6.268)%
    --(11.136,6.268)--(11.136,6.268)--(11.136,6.268)--(11.136,6.268)--(11.136,6.268)%
    --(11.136,6.268)--(11.136,6.268)--(11.136,6.268)--(11.136,6.268)--(11.136,6.268)%
    --(11.136,6.268)--(11.136,6.268)--(11.136,6.268)--(11.136,6.268)--(11.136,6.268)%
    --(11.136,6.268)--(11.136,6.268)--(11.136,6.268)--(11.136,6.268)--(11.136,6.268)%
    --(11.136,6.268)--(11.136,6.268)--(11.136,6.268)--(11.136,6.268)--(11.136,6.268)%
    --(11.136,6.268)--(11.136,6.268)--(11.136,6.268)--(11.136,6.268)--(11.136,6.268)%
    --(11.136,6.268)--(11.136,6.268)--(11.136,6.268)--(11.136,6.268)--(11.136,6.268)%
    --(11.136,6.268)--(11.136,6.268)--(11.136,6.268)--(11.136,6.268)--(11.136,6.268)%
    --(11.136,6.268)--(11.136,6.268)--(11.136,6.268)--(11.136,6.268)--(11.136,6.268)%
    --(11.136,6.268)--(11.136,6.268)--(11.136,6.268)--(11.136,6.268)--(11.136,6.268)%
    --(11.136,6.268)--(11.136,6.268)--(11.136,6.268)--(11.136,6.268)--(11.136,6.268)%
    --(11.136,6.268)--(11.136,6.268)--(11.136,6.268)--(11.136,6.268)--(11.136,6.268)%
    --(11.136,6.268)--(11.136,6.268)--(11.136,6.268)--(11.136,6.268)--(11.136,6.268)%
    --(11.136,6.268)--(11.136,6.268)--(11.136,6.268)--(11.136,6.268)--(11.136,6.268)%
    --(11.136,6.268)--cycle;
\gpfill{color=gp lt color border,opacity=0.50} (9.785,6.268)--(9.785,6.268)--(9.785,6.268)--(9.785,6.268)%
    --(9.785,6.268)--(9.785,6.268)--(9.785,6.268)--(9.785,6.268)--(9.785,6.268)%
    --(9.785,6.268)--(9.785,6.268)--(9.785,6.268)--(9.785,6.268)--(9.785,6.268)%
    --(9.785,6.268)--(9.785,6.268)--(9.785,6.268)--(9.785,6.268)--(9.785,6.268)%
    --(9.785,6.268)--(9.785,6.268)--(9.785,6.268)--(9.785,6.268)--(9.785,6.268)%
    --(9.785,6.268)--(9.785,6.268)--(9.785,6.268)--(9.785,6.268)--(9.785,6.268)%
    --(9.785,6.268)--(9.785,6.268)--(9.785,6.268)--(9.785,6.268)--(9.785,6.268)%
    --(9.785,6.268)--(9.785,6.268)--(9.785,6.268)--(9.785,6.268)--(9.785,6.268)%
    --(9.785,6.268)--(9.785,6.268)--(9.785,6.268)--(9.785,6.268)--(9.785,6.268)%
    --(9.785,6.268)--(9.785,6.268)--(9.785,6.268)--(9.785,6.268)--(9.785,6.268)%
    --(9.785,6.268)--(9.785,6.268)--(9.785,6.268)--(9.785,6.268)--(9.785,6.268)%
    --(9.785,6.268)--(9.785,6.268)--(9.785,6.268)--(9.785,6.268)--(9.785,6.268)%
    --(9.785,6.268)--(9.785,6.268)--(9.785,6.268)--(9.785,6.268)--(9.785,6.268)%
    --(9.785,6.268)--(9.785,6.268)--(9.785,6.268)--(9.785,6.268)--(9.785,6.268)%
    --(9.785,6.268)--(9.785,6.268)--(9.785,6.268)--(9.785,6.268)--(9.785,6.268)%
    --(9.785,6.268)--(9.785,6.268)--(9.785,6.268)--(9.785,6.268)--(9.785,6.268)%
    --(9.785,6.268)--(9.785,6.268)--(9.785,6.268)--(9.785,6.268)--(9.785,6.268)%
    --(9.785,6.268)--(9.785,6.268)--(9.785,6.268)--(9.785,6.268)--(9.785,6.268)%
    --(9.785,6.268)--(9.785,6.268)--(9.785,6.268)--(9.785,6.268)--(9.785,6.268)%
    --(9.785,6.268)--(9.785,6.268)--(9.785,6.268)--(9.785,6.268)--(9.785,6.268)%
    --(9.785,6.268)--(9.785,6.268)--(9.785,6.268)--(9.785,6.268)--(9.785,6.268)%
    --(9.785,6.268)--(9.785,6.268)--(9.785,6.268)--(9.785,6.268)--(9.785,6.268)%
    --(9.785,6.268)--(9.785,6.268)--(9.785,6.268)--(9.785,6.268)--(9.785,6.268)%
    --(9.785,6.268)--(9.785,6.268)--(9.785,6.268)--(9.785,6.268)--(9.785,6.268)%
    --(9.785,6.268)--cycle;
\gpfill{color=gp lt color border,opacity=0.50} (6.811,4.155)--(6.810,4.169)--(6.809,4.183)--(6.807,4.197)%
    --(6.805,4.211)--(6.801,4.224)--(6.797,4.238)--(6.793,4.251)--(6.787,4.264)%
    --(6.781,4.277)--(6.774,4.289)--(6.767,4.302)--(6.759,4.313)--(6.750,4.324)%
    --(6.741,4.335)--(6.731,4.345)--(6.721,4.355)--(6.710,4.364)--(6.699,4.373)%
    --(6.688,4.381)--(6.676,4.388)--(6.663,4.395)--(6.650,4.401)--(6.637,4.407)%
    --(6.624,4.411)--(6.610,4.415)--(6.597,4.419)--(6.583,4.421)--(6.569,4.423)%
    --(6.555,4.424)--(6.541,4.425)--(6.526,4.424)--(6.512,4.423)--(6.498,4.421)%
    --(6.484,4.419)--(6.471,4.415)--(6.457,4.411)--(6.444,4.407)--(6.431,4.401)%
    --(6.418,4.395)--(6.406,4.388)--(6.393,4.381)--(6.382,4.373)--(6.371,4.364)%
    --(6.360,4.355)--(6.350,4.345)--(6.340,4.335)--(6.331,4.324)--(6.322,4.313)%
    --(6.314,4.302)--(6.307,4.289)--(6.300,4.277)--(6.294,4.264)--(6.288,4.251)%
    --(6.284,4.238)--(6.280,4.224)--(6.276,4.211)--(6.274,4.197)--(6.272,4.183)%
    --(6.271,4.169)--(6.271,4.155)--(6.271,4.140)--(6.272,4.126)--(6.274,4.112)%
    --(6.276,4.098)--(6.280,4.085)--(6.284,4.071)--(6.288,4.058)--(6.294,4.045)%
    --(6.300,4.032)--(6.307,4.019)--(6.314,4.007)--(6.322,3.996)--(6.331,3.985)%
    --(6.340,3.974)--(6.350,3.964)--(6.360,3.954)--(6.371,3.945)--(6.382,3.936)%
    --(6.393,3.928)--(6.405,3.921)--(6.418,3.914)--(6.431,3.908)--(6.444,3.902)%
    --(6.457,3.898)--(6.471,3.894)--(6.484,3.890)--(6.498,3.888)--(6.512,3.886)%
    --(6.526,3.885)--(6.540,3.885)--(6.555,3.885)--(6.569,3.886)--(6.583,3.888)%
    --(6.597,3.890)--(6.610,3.894)--(6.624,3.898)--(6.637,3.902)--(6.650,3.908)%
    --(6.663,3.914)--(6.676,3.921)--(6.688,3.928)--(6.699,3.936)--(6.710,3.945)%
    --(6.721,3.954)--(6.731,3.964)--(6.741,3.974)--(6.750,3.985)--(6.759,3.996)%
    --(6.767,4.007)--(6.774,4.019)--(6.781,4.032)--(6.787,4.045)--(6.793,4.058)%
    --(6.797,4.071)--(6.801,4.085)--(6.805,4.098)--(6.807,4.112)--(6.809,4.126)%
    --(6.810,4.140)--(6.811,4.154)--cycle;
\gpfill{color=gp lt color border,opacity=0.50} (3.379,4.155)--(3.378,4.173)--(3.377,4.191)--(3.374,4.209)%
    --(3.371,4.227)--(3.367,4.245)--(3.361,4.263)--(3.355,4.280)--(3.348,4.297)%
    --(3.340,4.314)--(3.331,4.330)--(3.322,4.346)--(3.311,4.361)--(3.300,4.375)%
    --(3.288,4.389)--(3.276,4.403)--(3.262,4.415)--(3.248,4.427)--(3.234,4.438)%
    --(3.219,4.449)--(3.203,4.458)--(3.187,4.467)--(3.170,4.475)--(3.153,4.482)%
    --(3.136,4.488)--(3.118,4.494)--(3.100,4.498)--(3.082,4.501)--(3.064,4.504)%
    --(3.046,4.505)--(3.028,4.506)--(3.009,4.505)--(2.991,4.504)--(2.973,4.501)%
    --(2.955,4.498)--(2.937,4.494)--(2.919,4.488)--(2.902,4.482)--(2.885,4.475)%
    --(2.868,4.467)--(2.852,4.458)--(2.836,4.449)--(2.821,4.438)--(2.807,4.427)%
    --(2.793,4.415)--(2.779,4.403)--(2.767,4.389)--(2.755,4.375)--(2.744,4.361)%
    --(2.733,4.346)--(2.724,4.330)--(2.715,4.314)--(2.707,4.297)--(2.700,4.280)%
    --(2.694,4.263)--(2.688,4.245)--(2.684,4.227)--(2.681,4.209)--(2.678,4.191)%
    --(2.677,4.173)--(2.677,4.155)--(2.677,4.136)--(2.678,4.118)--(2.681,4.100)%
    --(2.684,4.082)--(2.688,4.064)--(2.694,4.046)--(2.700,4.029)--(2.707,4.012)%
    --(2.715,3.995)--(2.724,3.979)--(2.733,3.963)--(2.744,3.948)--(2.755,3.934)%
    --(2.767,3.920)--(2.779,3.906)--(2.793,3.894)--(2.807,3.882)--(2.821,3.871)%
    --(2.836,3.860)--(2.852,3.851)--(2.868,3.842)--(2.885,3.834)--(2.902,3.827)%
    --(2.919,3.821)--(2.937,3.815)--(2.955,3.811)--(2.973,3.808)--(2.991,3.805)%
    --(3.009,3.804)--(3.027,3.804)--(3.046,3.804)--(3.064,3.805)--(3.082,3.808)%
    --(3.100,3.811)--(3.118,3.815)--(3.136,3.821)--(3.153,3.827)--(3.170,3.834)%
    --(3.187,3.842)--(3.203,3.851)--(3.219,3.860)--(3.234,3.871)--(3.248,3.882)%
    --(3.262,3.894)--(3.276,3.906)--(3.288,3.920)--(3.300,3.934)--(3.311,3.948)%
    --(3.322,3.963)--(3.331,3.979)--(3.340,3.995)--(3.348,4.012)--(3.355,4.029)%
    --(3.361,4.046)--(3.367,4.064)--(3.371,4.082)--(3.374,4.100)--(3.377,4.118)%
    --(3.378,4.136)--(3.379,4.154)--cycle;
\gpfill{color=gp lt color border,opacity=0.50} (5.839,3.098)--(5.838,3.103)--(5.838,3.109)--(5.837,3.114)%
    --(5.836,3.120)--(5.835,3.125)--(5.833,3.131)--(5.831,3.136)--(5.829,3.141)%
    --(5.827,3.147)--(5.824,3.151)--(5.821,3.156)--(5.818,3.161)--(5.814,3.165)%
    --(5.811,3.170)--(5.807,3.174)--(5.803,3.178)--(5.798,3.181)--(5.794,3.185)%
    --(5.789,3.188)--(5.785,3.191)--(5.780,3.194)--(5.774,3.196)--(5.769,3.198)%
    --(5.764,3.200)--(5.758,3.202)--(5.753,3.203)--(5.747,3.204)--(5.742,3.205)%
    --(5.736,3.205)--(5.731,3.206)--(5.725,3.205)--(5.719,3.205)--(5.714,3.204)%
    --(5.708,3.203)--(5.703,3.202)--(5.697,3.200)--(5.692,3.198)--(5.687,3.196)%
    --(5.681,3.194)--(5.677,3.191)--(5.672,3.188)--(5.667,3.185)--(5.663,3.181)%
    --(5.658,3.178)--(5.654,3.174)--(5.650,3.170)--(5.647,3.165)--(5.643,3.161)%
    --(5.640,3.156)--(5.637,3.151)--(5.634,3.147)--(5.632,3.141)--(5.630,3.136)%
    --(5.628,3.131)--(5.626,3.125)--(5.625,3.120)--(5.624,3.114)--(5.623,3.109)%
    --(5.623,3.103)--(5.623,3.098)--(5.623,3.092)--(5.623,3.086)--(5.624,3.081)%
    --(5.625,3.075)--(5.626,3.070)--(5.628,3.064)--(5.630,3.059)--(5.632,3.054)%
    --(5.634,3.048)--(5.637,3.043)--(5.640,3.039)--(5.643,3.034)--(5.647,3.030)%
    --(5.650,3.025)--(5.654,3.021)--(5.658,3.017)--(5.663,3.014)--(5.667,3.010)%
    --(5.672,3.007)--(5.676,3.004)--(5.681,3.001)--(5.687,2.999)--(5.692,2.997)%
    --(5.697,2.995)--(5.703,2.993)--(5.708,2.992)--(5.714,2.991)--(5.719,2.990)%
    --(5.725,2.990)--(5.730,2.990)--(5.736,2.990)--(5.742,2.990)--(5.747,2.991)%
    --(5.753,2.992)--(5.758,2.993)--(5.764,2.995)--(5.769,2.997)--(5.774,2.999)%
    --(5.780,3.001)--(5.785,3.004)--(5.789,3.007)--(5.794,3.010)--(5.798,3.014)%
    --(5.803,3.017)--(5.807,3.021)--(5.811,3.025)--(5.814,3.030)--(5.818,3.034)%
    --(5.821,3.039)--(5.824,3.043)--(5.827,3.048)--(5.829,3.054)--(5.831,3.059)%
    --(5.833,3.064)--(5.835,3.070)--(5.836,3.075)--(5.837,3.081)--(5.838,3.086)%
    --(5.838,3.092)--(5.839,3.097)--cycle;
\gpfill{color=gp lt color border,opacity=0.50} (8.514,6.268)--(8.513,6.272)--(8.513,6.276)--(8.513,6.280)%
    --(8.512,6.284)--(8.511,6.288)--(8.510,6.293)--(8.508,6.297)--(8.506,6.300)%
    --(8.505,6.304)--(8.503,6.308)--(8.500,6.312)--(8.498,6.315)--(8.495,6.318)%
    --(8.493,6.322)--(8.490,6.325)--(8.487,6.328)--(8.483,6.330)--(8.480,6.333)%
    --(8.477,6.335)--(8.473,6.338)--(8.469,6.340)--(8.465,6.341)--(8.462,6.343)%
    --(8.458,6.345)--(8.453,6.346)--(8.449,6.347)--(8.445,6.348)--(8.441,6.348)%
    --(8.437,6.348)--(8.433,6.349)--(8.428,6.348)--(8.424,6.348)--(8.420,6.348)%
    --(8.416,6.347)--(8.412,6.346)--(8.407,6.345)--(8.403,6.343)--(8.400,6.341)%
    --(8.396,6.340)--(8.392,6.338)--(8.388,6.335)--(8.385,6.333)--(8.382,6.330)%
    --(8.378,6.328)--(8.375,6.325)--(8.372,6.322)--(8.370,6.318)--(8.367,6.315)%
    --(8.365,6.312)--(8.362,6.308)--(8.360,6.304)--(8.359,6.300)--(8.357,6.297)%
    --(8.355,6.293)--(8.354,6.288)--(8.353,6.284)--(8.352,6.280)--(8.352,6.276)%
    --(8.352,6.272)--(8.352,6.268)--(8.352,6.263)--(8.352,6.259)--(8.352,6.255)%
    --(8.353,6.251)--(8.354,6.247)--(8.355,6.242)--(8.357,6.238)--(8.359,6.235)%
    --(8.360,6.231)--(8.362,6.227)--(8.365,6.223)--(8.367,6.220)--(8.370,6.217)%
    --(8.372,6.213)--(8.375,6.210)--(8.378,6.207)--(8.382,6.205)--(8.385,6.202)%
    --(8.388,6.200)--(8.392,6.197)--(8.396,6.195)--(8.400,6.194)--(8.403,6.192)%
    --(8.407,6.190)--(8.412,6.189)--(8.416,6.188)--(8.420,6.187)--(8.424,6.187)%
    --(8.428,6.187)--(8.432,6.187)--(8.437,6.187)--(8.441,6.187)--(8.445,6.187)%
    --(8.449,6.188)--(8.453,6.189)--(8.458,6.190)--(8.462,6.192)--(8.465,6.194)%
    --(8.469,6.195)--(8.473,6.197)--(8.477,6.200)--(8.480,6.202)--(8.483,6.205)%
    --(8.487,6.207)--(8.490,6.210)--(8.493,6.213)--(8.495,6.217)--(8.498,6.220)%
    --(8.500,6.223)--(8.503,6.227)--(8.505,6.231)--(8.506,6.235)--(8.508,6.238)%
    --(8.510,6.242)--(8.511,6.247)--(8.512,6.251)--(8.513,6.255)--(8.513,6.259)%
    --(8.513,6.263)--(8.514,6.267)--cycle;
\gpfill{color=gp lt color border,opacity=0.50} (11.190,5.211)--(11.189,5.213)--(11.189,5.216)--(11.189,5.219)%
    --(11.188,5.222)--(11.188,5.224)--(11.187,5.227)--(11.186,5.230)--(11.185,5.232)%
    --(11.184,5.235)--(11.182,5.237)--(11.181,5.240)--(11.179,5.242)--(11.177,5.244)%
    --(11.176,5.247)--(11.174,5.249)--(11.172,5.251)--(11.169,5.252)--(11.167,5.254)%
    --(11.165,5.256)--(11.163,5.257)--(11.160,5.259)--(11.157,5.260)--(11.155,5.261)%
    --(11.152,5.262)--(11.149,5.263)--(11.147,5.263)--(11.144,5.264)--(11.141,5.264)%
    --(11.138,5.264)--(11.136,5.265)--(11.133,5.264)--(11.130,5.264)--(11.127,5.264)%
    --(11.124,5.263)--(11.122,5.263)--(11.119,5.262)--(11.116,5.261)--(11.114,5.260)%
    --(11.111,5.259)--(11.109,5.257)--(11.106,5.256)--(11.104,5.254)--(11.102,5.252)%
    --(11.099,5.251)--(11.097,5.249)--(11.095,5.247)--(11.094,5.244)--(11.092,5.242)%
    --(11.090,5.240)--(11.089,5.237)--(11.087,5.235)--(11.086,5.232)--(11.085,5.230)%
    --(11.084,5.227)--(11.083,5.224)--(11.083,5.222)--(11.082,5.219)--(11.082,5.216)%
    --(11.082,5.213)--(11.082,5.211)--(11.082,5.208)--(11.082,5.205)--(11.082,5.202)%
    --(11.083,5.199)--(11.083,5.197)--(11.084,5.194)--(11.085,5.191)--(11.086,5.189)%
    --(11.087,5.186)--(11.089,5.183)--(11.090,5.181)--(11.092,5.179)--(11.094,5.177)%
    --(11.095,5.174)--(11.097,5.172)--(11.099,5.170)--(11.102,5.169)--(11.104,5.167)%
    --(11.106,5.165)--(11.108,5.164)--(11.111,5.162)--(11.114,5.161)--(11.116,5.160)%
    --(11.119,5.159)--(11.122,5.158)--(11.124,5.158)--(11.127,5.157)--(11.130,5.157)%
    --(11.133,5.157)--(11.135,5.157)--(11.138,5.157)--(11.141,5.157)--(11.144,5.157)%
    --(11.147,5.158)--(11.149,5.158)--(11.152,5.159)--(11.155,5.160)--(11.157,5.161)%
    --(11.160,5.162)--(11.163,5.164)--(11.165,5.165)--(11.167,5.167)--(11.169,5.169)%
    --(11.172,5.170)--(11.174,5.172)--(11.176,5.174)--(11.177,5.177)--(11.179,5.179)%
    --(11.181,5.181)--(11.182,5.183)--(11.184,5.186)--(11.185,5.189)--(11.186,5.191)%
    --(11.187,5.194)--(11.188,5.197)--(11.188,5.199)--(11.189,5.202)--(11.189,5.205)%
    --(11.189,5.208)--(11.190,5.210)--cycle;
\gpfill{color=gp lt color border,opacity=0.50} (7.217,6.268)--(7.216,6.275)--(7.216,6.282)--(7.215,6.289)%
    --(7.214,6.296)--(7.212,6.302)--(7.210,6.309)--(7.208,6.316)--(7.205,6.322)%
    --(7.202,6.329)--(7.198,6.335)--(7.195,6.341)--(7.191,6.347)--(7.186,6.352)%
    --(7.182,6.358)--(7.177,6.363)--(7.172,6.368)--(7.166,6.372)--(7.161,6.377)%
    --(7.155,6.381)--(7.149,6.384)--(7.143,6.388)--(7.136,6.391)--(7.130,6.394)%
    --(7.123,6.396)--(7.116,6.398)--(7.110,6.400)--(7.103,6.401)--(7.096,6.402)%
    --(7.089,6.402)--(7.082,6.403)--(7.074,6.402)--(7.067,6.402)--(7.060,6.401)%
    --(7.053,6.400)--(7.047,6.398)--(7.040,6.396)--(7.033,6.394)--(7.027,6.391)%
    --(7.020,6.388)--(7.014,6.384)--(7.008,6.381)--(7.002,6.377)--(6.997,6.372)%
    --(6.991,6.368)--(6.986,6.363)--(6.981,6.358)--(6.977,6.352)--(6.972,6.347)%
    --(6.968,6.341)--(6.965,6.335)--(6.961,6.329)--(6.958,6.322)--(6.955,6.316)%
    --(6.953,6.309)--(6.951,6.302)--(6.949,6.296)--(6.948,6.289)--(6.947,6.282)%
    --(6.947,6.275)--(6.947,6.268)--(6.947,6.260)--(6.947,6.253)--(6.948,6.246)%
    --(6.949,6.239)--(6.951,6.233)--(6.953,6.226)--(6.955,6.219)--(6.958,6.213)%
    --(6.961,6.206)--(6.965,6.200)--(6.968,6.194)--(6.972,6.188)--(6.977,6.183)%
    --(6.981,6.177)--(6.986,6.172)--(6.991,6.167)--(6.997,6.163)--(7.002,6.158)%
    --(7.008,6.154)--(7.014,6.151)--(7.020,6.147)--(7.027,6.144)--(7.033,6.141)%
    --(7.040,6.139)--(7.047,6.137)--(7.053,6.135)--(7.060,6.134)--(7.067,6.133)%
    --(7.074,6.133)--(7.081,6.133)--(7.089,6.133)--(7.096,6.133)--(7.103,6.134)%
    --(7.110,6.135)--(7.116,6.137)--(7.123,6.139)--(7.130,6.141)--(7.136,6.144)%
    --(7.143,6.147)--(7.149,6.151)--(7.155,6.154)--(7.161,6.158)--(7.166,6.163)%
    --(7.172,6.167)--(7.177,6.172)--(7.182,6.177)--(7.186,6.183)--(7.191,6.188)%
    --(7.195,6.194)--(7.198,6.200)--(7.202,6.206)--(7.205,6.213)--(7.208,6.219)%
    --(7.210,6.226)--(7.212,6.233)--(7.214,6.239)--(7.215,6.246)--(7.216,6.253)%
    --(7.216,6.260)--(7.217,6.267)--cycle;
\gpfill{color=gp lt color border,opacity=0.50} (9.866,5.211)--(9.865,5.215)--(9.865,5.219)--(9.865,5.223)%
    --(9.864,5.227)--(9.863,5.231)--(9.862,5.236)--(9.860,5.240)--(9.858,5.243)%
    --(9.857,5.247)--(9.855,5.251)--(9.852,5.255)--(9.850,5.258)--(9.847,5.261)%
    --(9.845,5.265)--(9.842,5.268)--(9.839,5.271)--(9.835,5.273)--(9.832,5.276)%
    --(9.829,5.278)--(9.825,5.281)--(9.821,5.283)--(9.817,5.284)--(9.814,5.286)%
    --(9.810,5.288)--(9.805,5.289)--(9.801,5.290)--(9.797,5.291)--(9.793,5.291)%
    --(9.789,5.291)--(9.785,5.292)--(9.780,5.291)--(9.776,5.291)--(9.772,5.291)%
    --(9.768,5.290)--(9.764,5.289)--(9.759,5.288)--(9.755,5.286)--(9.752,5.284)%
    --(9.748,5.283)--(9.744,5.281)--(9.740,5.278)--(9.737,5.276)--(9.734,5.273)%
    --(9.730,5.271)--(9.727,5.268)--(9.724,5.265)--(9.722,5.261)--(9.719,5.258)%
    --(9.717,5.255)--(9.714,5.251)--(9.712,5.247)--(9.711,5.243)--(9.709,5.240)%
    --(9.707,5.236)--(9.706,5.231)--(9.705,5.227)--(9.704,5.223)--(9.704,5.219)%
    --(9.704,5.215)--(9.704,5.211)--(9.704,5.206)--(9.704,5.202)--(9.704,5.198)%
    --(9.705,5.194)--(9.706,5.190)--(9.707,5.185)--(9.709,5.181)--(9.711,5.178)%
    --(9.712,5.174)--(9.714,5.170)--(9.717,5.166)--(9.719,5.163)--(9.722,5.160)%
    --(9.724,5.156)--(9.727,5.153)--(9.730,5.150)--(9.734,5.148)--(9.737,5.145)%
    --(9.740,5.143)--(9.744,5.140)--(9.748,5.138)--(9.752,5.137)--(9.755,5.135)%
    --(9.759,5.133)--(9.764,5.132)--(9.768,5.131)--(9.772,5.130)--(9.776,5.130)%
    --(9.780,5.130)--(9.784,5.130)--(9.789,5.130)--(9.793,5.130)--(9.797,5.130)%
    --(9.801,5.131)--(9.805,5.132)--(9.810,5.133)--(9.814,5.135)--(9.817,5.137)%
    --(9.821,5.138)--(9.825,5.140)--(9.829,5.143)--(9.832,5.145)--(9.835,5.148)%
    --(9.839,5.150)--(9.842,5.153)--(9.845,5.156)--(9.847,5.160)--(9.850,5.163)%
    --(9.852,5.166)--(9.855,5.170)--(9.857,5.174)--(9.858,5.178)--(9.860,5.181)%
    --(9.862,5.185)--(9.863,5.190)--(9.864,5.194)--(9.865,5.198)--(9.865,5.202)%
    --(9.865,5.206)--(9.866,5.210)--cycle;
\gpfill{color=gp lt color border,opacity=0.50} (3.325,3.098)--(3.324,3.113)--(3.323,3.129)--(3.321,3.144)%
    --(3.318,3.159)--(3.314,3.174)--(3.310,3.189)--(3.305,3.204)--(3.299,3.218)%
    --(3.292,3.232)--(3.285,3.246)--(3.277,3.259)--(3.268,3.272)--(3.258,3.284)%
    --(3.248,3.296)--(3.238,3.308)--(3.226,3.318)--(3.214,3.328)--(3.202,3.338)%
    --(3.189,3.347)--(3.176,3.355)--(3.162,3.362)--(3.148,3.369)--(3.134,3.375)%
    --(3.119,3.380)--(3.104,3.384)--(3.089,3.388)--(3.074,3.391)--(3.059,3.393)%
    --(3.043,3.394)--(3.028,3.395)--(3.012,3.394)--(2.996,3.393)--(2.981,3.391)%
    --(2.966,3.388)--(2.951,3.384)--(2.936,3.380)--(2.921,3.375)--(2.907,3.369)%
    --(2.893,3.362)--(2.879,3.355)--(2.866,3.347)--(2.853,3.338)--(2.841,3.328)%
    --(2.829,3.318)--(2.817,3.308)--(2.807,3.296)--(2.797,3.284)--(2.787,3.272)%
    --(2.778,3.259)--(2.770,3.246)--(2.763,3.232)--(2.756,3.218)--(2.750,3.204)%
    --(2.745,3.189)--(2.741,3.174)--(2.737,3.159)--(2.734,3.144)--(2.732,3.129)%
    --(2.731,3.113)--(2.731,3.098)--(2.731,3.082)--(2.732,3.066)--(2.734,3.051)%
    --(2.737,3.036)--(2.741,3.021)--(2.745,3.006)--(2.750,2.991)--(2.756,2.977)%
    --(2.763,2.963)--(2.770,2.949)--(2.778,2.936)--(2.787,2.923)--(2.797,2.911)%
    --(2.807,2.899)--(2.817,2.887)--(2.829,2.877)--(2.841,2.867)--(2.853,2.857)%
    --(2.866,2.848)--(2.879,2.840)--(2.893,2.833)--(2.907,2.826)--(2.921,2.820)%
    --(2.936,2.815)--(2.951,2.811)--(2.966,2.807)--(2.981,2.804)--(2.996,2.802)%
    --(3.012,2.801)--(3.027,2.801)--(3.043,2.801)--(3.059,2.802)--(3.074,2.804)%
    --(3.089,2.807)--(3.104,2.811)--(3.119,2.815)--(3.134,2.820)--(3.148,2.826)%
    --(3.162,2.833)--(3.176,2.840)--(3.189,2.848)--(3.202,2.857)--(3.214,2.867)%
    --(3.226,2.877)--(3.238,2.887)--(3.248,2.899)--(3.258,2.911)--(3.268,2.923)%
    --(3.277,2.936)--(3.285,2.949)--(3.292,2.963)--(3.299,2.977)--(3.305,2.991)%
    --(3.310,3.006)--(3.314,3.021)--(3.318,3.036)--(3.321,3.051)--(3.323,3.066)%
    --(3.324,3.082)--(3.325,3.097)--cycle;
\gpfill{color=gp lt color border,opacity=0.50} (5.921,6.268)--(5.920,6.277)--(5.919,6.287)--(5.918,6.297)%
    --(5.916,6.307)--(5.914,6.317)--(5.911,6.326)--(5.908,6.336)--(5.904,6.345)%
    --(5.900,6.354)--(5.895,6.362)--(5.890,6.371)--(5.884,6.379)--(5.878,6.387)%
    --(5.872,6.395)--(5.865,6.402)--(5.858,6.409)--(5.850,6.415)--(5.842,6.421)%
    --(5.834,6.427)--(5.826,6.432)--(5.817,6.437)--(5.808,6.441)--(5.799,6.445)%
    --(5.789,6.448)--(5.780,6.451)--(5.770,6.453)--(5.760,6.455)--(5.750,6.456)%
    --(5.740,6.457)--(5.731,6.458)--(5.721,6.457)--(5.711,6.456)--(5.701,6.455)%
    --(5.691,6.453)--(5.681,6.451)--(5.672,6.448)--(5.662,6.445)--(5.653,6.441)%
    --(5.644,6.437)--(5.636,6.432)--(5.627,6.427)--(5.619,6.421)--(5.611,6.415)%
    --(5.603,6.409)--(5.596,6.402)--(5.589,6.395)--(5.583,6.387)--(5.577,6.379)%
    --(5.571,6.371)--(5.566,6.362)--(5.561,6.354)--(5.557,6.345)--(5.553,6.336)%
    --(5.550,6.326)--(5.547,6.317)--(5.545,6.307)--(5.543,6.297)--(5.542,6.287)%
    --(5.541,6.277)--(5.541,6.268)--(5.541,6.258)--(5.542,6.248)--(5.543,6.238)%
    --(5.545,6.228)--(5.547,6.218)--(5.550,6.209)--(5.553,6.199)--(5.557,6.190)%
    --(5.561,6.181)--(5.566,6.172)--(5.571,6.164)--(5.577,6.156)--(5.583,6.148)%
    --(5.589,6.140)--(5.596,6.133)--(5.603,6.126)--(5.611,6.120)--(5.619,6.114)%
    --(5.627,6.108)--(5.635,6.103)--(5.644,6.098)--(5.653,6.094)--(5.662,6.090)%
    --(5.672,6.087)--(5.681,6.084)--(5.691,6.082)--(5.701,6.080)--(5.711,6.079)%
    --(5.721,6.078)--(5.730,6.078)--(5.740,6.078)--(5.750,6.079)--(5.760,6.080)%
    --(5.770,6.082)--(5.780,6.084)--(5.789,6.087)--(5.799,6.090)--(5.808,6.094)%
    --(5.817,6.098)--(5.826,6.103)--(5.834,6.108)--(5.842,6.114)--(5.850,6.120)%
    --(5.858,6.126)--(5.865,6.133)--(5.872,6.140)--(5.878,6.148)--(5.884,6.156)%
    --(5.890,6.164)--(5.895,6.172)--(5.900,6.181)--(5.904,6.190)--(5.908,6.199)%
    --(5.911,6.209)--(5.914,6.218)--(5.916,6.228)--(5.918,6.238)--(5.919,6.248)%
    --(5.920,6.258)--(5.921,6.267)--cycle;
\gpfill{color=gp lt color border,opacity=0.50} (8.541,5.211)--(8.540,5.216)--(8.540,5.222)--(8.539,5.227)%
    --(8.538,5.233)--(8.537,5.238)--(8.535,5.244)--(8.533,5.249)--(8.531,5.254)%
    --(8.529,5.260)--(8.526,5.264)--(8.523,5.269)--(8.520,5.274)--(8.516,5.278)%
    --(8.513,5.283)--(8.509,5.287)--(8.505,5.291)--(8.500,5.294)--(8.496,5.298)%
    --(8.491,5.301)--(8.487,5.304)--(8.482,5.307)--(8.476,5.309)--(8.471,5.311)%
    --(8.466,5.313)--(8.460,5.315)--(8.455,5.316)--(8.449,5.317)--(8.444,5.318)%
    --(8.438,5.318)--(8.433,5.319)--(8.427,5.318)--(8.421,5.318)--(8.416,5.317)%
    --(8.410,5.316)--(8.405,5.315)--(8.399,5.313)--(8.394,5.311)--(8.389,5.309)%
    --(8.383,5.307)--(8.379,5.304)--(8.374,5.301)--(8.369,5.298)--(8.365,5.294)%
    --(8.360,5.291)--(8.356,5.287)--(8.352,5.283)--(8.349,5.278)--(8.345,5.274)%
    --(8.342,5.269)--(8.339,5.264)--(8.336,5.260)--(8.334,5.254)--(8.332,5.249)%
    --(8.330,5.244)--(8.328,5.238)--(8.327,5.233)--(8.326,5.227)--(8.325,5.222)%
    --(8.325,5.216)--(8.325,5.211)--(8.325,5.205)--(8.325,5.199)--(8.326,5.194)%
    --(8.327,5.188)--(8.328,5.183)--(8.330,5.177)--(8.332,5.172)--(8.334,5.167)%
    --(8.336,5.161)--(8.339,5.156)--(8.342,5.152)--(8.345,5.147)--(8.349,5.143)%
    --(8.352,5.138)--(8.356,5.134)--(8.360,5.130)--(8.365,5.127)--(8.369,5.123)%
    --(8.374,5.120)--(8.378,5.117)--(8.383,5.114)--(8.389,5.112)--(8.394,5.110)%
    --(8.399,5.108)--(8.405,5.106)--(8.410,5.105)--(8.416,5.104)--(8.421,5.103)%
    --(8.427,5.103)--(8.432,5.103)--(8.438,5.103)--(8.444,5.103)--(8.449,5.104)%
    --(8.455,5.105)--(8.460,5.106)--(8.466,5.108)--(8.471,5.110)--(8.476,5.112)%
    --(8.482,5.114)--(8.487,5.117)--(8.491,5.120)--(8.496,5.123)--(8.500,5.127)%
    --(8.505,5.130)--(8.509,5.134)--(8.513,5.138)--(8.516,5.143)--(8.520,5.147)%
    --(8.523,5.152)--(8.526,5.156)--(8.529,5.161)--(8.531,5.167)--(8.533,5.172)%
    --(8.535,5.177)--(8.537,5.183)--(8.538,5.188)--(8.539,5.194)--(8.540,5.199)%
    --(8.540,5.205)--(8.541,5.210)--cycle;
\gpfill{color=gp lt color border,opacity=0.50} (4.164,4.155)--(4.163,4.172)--(4.162,4.188)--(4.159,4.205)%
    --(4.156,4.222)--(4.152,4.239)--(4.148,4.255)--(4.142,4.271)--(4.135,4.287)%
    --(4.128,4.302)--(4.120,4.317)--(4.111,4.332)--(4.101,4.346)--(4.091,4.359)%
    --(4.080,4.372)--(4.068,4.384)--(4.056,4.396)--(4.043,4.407)--(4.030,4.417)%
    --(4.016,4.427)--(4.001,4.436)--(3.986,4.444)--(3.971,4.451)--(3.955,4.458)%
    --(3.939,4.464)--(3.923,4.468)--(3.906,4.472)--(3.889,4.475)--(3.872,4.478)%
    --(3.856,4.479)--(3.839,4.480)--(3.821,4.479)--(3.805,4.478)--(3.788,4.475)%
    --(3.771,4.472)--(3.754,4.468)--(3.738,4.464)--(3.722,4.458)--(3.706,4.451)%
    --(3.691,4.444)--(3.676,4.436)--(3.661,4.427)--(3.647,4.417)--(3.634,4.407)%
    --(3.621,4.396)--(3.609,4.384)--(3.597,4.372)--(3.586,4.359)--(3.576,4.346)%
    --(3.566,4.332)--(3.557,4.317)--(3.549,4.302)--(3.542,4.287)--(3.535,4.271)%
    --(3.529,4.255)--(3.525,4.239)--(3.521,4.222)--(3.518,4.205)--(3.515,4.188)%
    --(3.514,4.172)--(3.514,4.155)--(3.514,4.137)--(3.515,4.121)--(3.518,4.104)%
    --(3.521,4.087)--(3.525,4.070)--(3.529,4.054)--(3.535,4.038)--(3.542,4.022)%
    --(3.549,4.007)--(3.557,3.992)--(3.566,3.977)--(3.576,3.963)--(3.586,3.950)%
    --(3.597,3.937)--(3.609,3.925)--(3.621,3.913)--(3.634,3.902)--(3.647,3.892)%
    --(3.661,3.882)--(3.676,3.873)--(3.691,3.865)--(3.706,3.858)--(3.722,3.851)%
    --(3.738,3.845)--(3.754,3.841)--(3.771,3.837)--(3.788,3.834)--(3.805,3.831)%
    --(3.821,3.830)--(3.838,3.830)--(3.856,3.830)--(3.872,3.831)--(3.889,3.834)%
    --(3.906,3.837)--(3.923,3.841)--(3.939,3.845)--(3.955,3.851)--(3.971,3.858)%
    --(3.986,3.865)--(4.001,3.873)--(4.016,3.882)--(4.030,3.892)--(4.043,3.902)%
    --(4.056,3.913)--(4.068,3.925)--(4.080,3.937)--(4.091,3.950)--(4.101,3.963)%
    --(4.111,3.977)--(4.120,3.992)--(4.128,4.007)--(4.135,4.022)--(4.142,4.038)%
    --(4.148,4.054)--(4.152,4.070)--(4.156,4.087)--(4.159,4.104)--(4.162,4.121)%
    --(4.163,4.137)--(4.164,4.154)--cycle;
\gpfill{color=gp lt color border,opacity=0.50} (6.622,3.098)--(6.621,3.102)--(6.621,3.106)--(6.621,3.110)%
    --(6.620,3.114)--(6.619,3.118)--(6.618,3.123)--(6.616,3.127)--(6.614,3.130)%
    --(6.613,3.134)--(6.611,3.138)--(6.608,3.142)--(6.606,3.145)--(6.603,3.148)%
    --(6.601,3.152)--(6.598,3.155)--(6.595,3.158)--(6.591,3.160)--(6.588,3.163)%
    --(6.585,3.165)--(6.581,3.168)--(6.577,3.170)--(6.573,3.171)--(6.570,3.173)%
    --(6.566,3.175)--(6.561,3.176)--(6.557,3.177)--(6.553,3.178)--(6.549,3.178)%
    --(6.545,3.178)--(6.541,3.179)--(6.536,3.178)--(6.532,3.178)--(6.528,3.178)%
    --(6.524,3.177)--(6.520,3.176)--(6.515,3.175)--(6.511,3.173)--(6.508,3.171)%
    --(6.504,3.170)--(6.500,3.168)--(6.496,3.165)--(6.493,3.163)--(6.490,3.160)%
    --(6.486,3.158)--(6.483,3.155)--(6.480,3.152)--(6.478,3.148)--(6.475,3.145)%
    --(6.473,3.142)--(6.470,3.138)--(6.468,3.134)--(6.467,3.130)--(6.465,3.127)%
    --(6.463,3.123)--(6.462,3.118)--(6.461,3.114)--(6.460,3.110)--(6.460,3.106)%
    --(6.460,3.102)--(6.460,3.098)--(6.460,3.093)--(6.460,3.089)--(6.460,3.085)%
    --(6.461,3.081)--(6.462,3.077)--(6.463,3.072)--(6.465,3.068)--(6.467,3.065)%
    --(6.468,3.061)--(6.470,3.057)--(6.473,3.053)--(6.475,3.050)--(6.478,3.047)%
    --(6.480,3.043)--(6.483,3.040)--(6.486,3.037)--(6.490,3.035)--(6.493,3.032)%
    --(6.496,3.030)--(6.500,3.027)--(6.504,3.025)--(6.508,3.024)--(6.511,3.022)%
    --(6.515,3.020)--(6.520,3.019)--(6.524,3.018)--(6.528,3.017)--(6.532,3.017)%
    --(6.536,3.017)--(6.540,3.017)--(6.545,3.017)--(6.549,3.017)--(6.553,3.017)%
    --(6.557,3.018)--(6.561,3.019)--(6.566,3.020)--(6.570,3.022)--(6.573,3.024)%
    --(6.577,3.025)--(6.581,3.027)--(6.585,3.030)--(6.588,3.032)--(6.591,3.035)%
    --(6.595,3.037)--(6.598,3.040)--(6.601,3.043)--(6.603,3.047)--(6.606,3.050)%
    --(6.608,3.053)--(6.611,3.057)--(6.613,3.061)--(6.614,3.065)--(6.616,3.068)%
    --(6.618,3.072)--(6.619,3.077)--(6.620,3.081)--(6.621,3.085)--(6.621,3.089)%
    --(6.621,3.093)--(6.622,3.097)--cycle;
\gpfill{color=gp lt color border,opacity=0.50} (7.298,5.211)--(7.297,5.222)--(7.296,5.233)--(7.295,5.244)%
    --(7.293,5.255)--(7.290,5.266)--(7.287,5.277)--(7.283,5.288)--(7.279,5.298)%
    --(7.274,5.309)--(7.269,5.318)--(7.263,5.328)--(7.256,5.337)--(7.249,5.346)%
    --(7.242,5.355)--(7.234,5.363)--(7.226,5.371)--(7.217,5.378)--(7.208,5.385)%
    --(7.199,5.392)--(7.190,5.398)--(7.180,5.403)--(7.169,5.408)--(7.159,5.412)%
    --(7.148,5.416)--(7.137,5.419)--(7.126,5.422)--(7.115,5.424)--(7.104,5.425)%
    --(7.093,5.426)--(7.082,5.427)--(7.070,5.426)--(7.059,5.425)--(7.048,5.424)%
    --(7.037,5.422)--(7.026,5.419)--(7.015,5.416)--(7.004,5.412)--(6.994,5.408)%
    --(6.983,5.403)--(6.974,5.398)--(6.964,5.392)--(6.955,5.385)--(6.946,5.378)%
    --(6.937,5.371)--(6.929,5.363)--(6.921,5.355)--(6.914,5.346)--(6.907,5.337)%
    --(6.900,5.328)--(6.894,5.318)--(6.889,5.309)--(6.884,5.298)--(6.880,5.288)%
    --(6.876,5.277)--(6.873,5.266)--(6.870,5.255)--(6.868,5.244)--(6.867,5.233)%
    --(6.866,5.222)--(6.866,5.211)--(6.866,5.199)--(6.867,5.188)--(6.868,5.177)%
    --(6.870,5.166)--(6.873,5.155)--(6.876,5.144)--(6.880,5.133)--(6.884,5.123)%
    --(6.889,5.112)--(6.894,5.102)--(6.900,5.093)--(6.907,5.084)--(6.914,5.075)%
    --(6.921,5.066)--(6.929,5.058)--(6.937,5.050)--(6.946,5.043)--(6.955,5.036)%
    --(6.964,5.029)--(6.973,5.023)--(6.983,5.018)--(6.994,5.013)--(7.004,5.009)%
    --(7.015,5.005)--(7.026,5.002)--(7.037,4.999)--(7.048,4.997)--(7.059,4.996)%
    --(7.070,4.995)--(7.081,4.995)--(7.093,4.995)--(7.104,4.996)--(7.115,4.997)%
    --(7.126,4.999)--(7.137,5.002)--(7.148,5.005)--(7.159,5.009)--(7.169,5.013)%
    --(7.180,5.018)--(7.190,5.023)--(7.199,5.029)--(7.208,5.036)--(7.217,5.043)%
    --(7.226,5.050)--(7.234,5.058)--(7.242,5.066)--(7.249,5.075)--(7.256,5.084)%
    --(7.263,5.093)--(7.269,5.102)--(7.274,5.112)--(7.279,5.123)--(7.283,5.133)%
    --(7.287,5.144)--(7.290,5.155)--(7.293,5.166)--(7.295,5.177)--(7.296,5.188)%
    --(7.297,5.199)--(7.298,5.210)--cycle;
\gpfill{color=gp lt color border,opacity=0.50} (4.568,6.268)--(4.567,6.277)--(4.566,6.287)--(4.565,6.297)%
    --(4.563,6.307)--(4.561,6.316)--(4.558,6.326)--(4.555,6.335)--(4.551,6.344)%
    --(4.547,6.353)--(4.542,6.362)--(4.537,6.370)--(4.531,6.379)--(4.525,6.386)%
    --(4.519,6.394)--(4.512,6.401)--(4.505,6.408)--(4.497,6.414)--(4.490,6.420)%
    --(4.481,6.426)--(4.473,6.431)--(4.464,6.436)--(4.455,6.440)--(4.446,6.444)%
    --(4.437,6.447)--(4.427,6.450)--(4.418,6.452)--(4.408,6.454)--(4.398,6.455)%
    --(4.388,6.456)--(4.379,6.457)--(4.369,6.456)--(4.359,6.455)--(4.349,6.454)%
    --(4.339,6.452)--(4.330,6.450)--(4.320,6.447)--(4.311,6.444)--(4.302,6.440)%
    --(4.293,6.436)--(4.284,6.431)--(4.276,6.426)--(4.267,6.420)--(4.260,6.414)%
    --(4.252,6.408)--(4.245,6.401)--(4.238,6.394)--(4.232,6.386)--(4.226,6.379)%
    --(4.220,6.370)--(4.215,6.362)--(4.210,6.353)--(4.206,6.344)--(4.202,6.335)%
    --(4.199,6.326)--(4.196,6.316)--(4.194,6.307)--(4.192,6.297)--(4.191,6.287)%
    --(4.190,6.277)--(4.190,6.268)--(4.190,6.258)--(4.191,6.248)--(4.192,6.238)%
    --(4.194,6.228)--(4.196,6.219)--(4.199,6.209)--(4.202,6.200)--(4.206,6.191)%
    --(4.210,6.182)--(4.215,6.173)--(4.220,6.165)--(4.226,6.156)--(4.232,6.149)%
    --(4.238,6.141)--(4.245,6.134)--(4.252,6.127)--(4.260,6.121)--(4.267,6.115)%
    --(4.276,6.109)--(4.284,6.104)--(4.293,6.099)--(4.302,6.095)--(4.311,6.091)%
    --(4.320,6.088)--(4.330,6.085)--(4.339,6.083)--(4.349,6.081)--(4.359,6.080)%
    --(4.369,6.079)--(4.378,6.079)--(4.388,6.079)--(4.398,6.080)--(4.408,6.081)%
    --(4.418,6.083)--(4.427,6.085)--(4.437,6.088)--(4.446,6.091)--(4.455,6.095)%
    --(4.464,6.099)--(4.473,6.104)--(4.481,6.109)--(4.490,6.115)--(4.497,6.121)%
    --(4.505,6.127)--(4.512,6.134)--(4.519,6.141)--(4.525,6.149)--(4.531,6.156)%
    --(4.537,6.165)--(4.542,6.173)--(4.547,6.182)--(4.551,6.191)--(4.555,6.200)%
    --(4.558,6.209)--(4.561,6.219)--(4.563,6.228)--(4.565,6.238)--(4.566,6.248)%
    --(4.567,6.258)--(4.568,6.267)--cycle;
\gpfill{color=gp lt color border,opacity=0.50} (3.217,6.268)--(3.216,6.277)--(3.215,6.287)--(3.214,6.297)%
    --(3.212,6.307)--(3.210,6.316)--(3.207,6.326)--(3.204,6.335)--(3.200,6.344)%
    --(3.196,6.353)--(3.191,6.362)--(3.186,6.370)--(3.180,6.379)--(3.174,6.386)%
    --(3.168,6.394)--(3.161,6.401)--(3.154,6.408)--(3.146,6.414)--(3.139,6.420)%
    --(3.130,6.426)--(3.122,6.431)--(3.113,6.436)--(3.104,6.440)--(3.095,6.444)%
    --(3.086,6.447)--(3.076,6.450)--(3.067,6.452)--(3.057,6.454)--(3.047,6.455)%
    --(3.037,6.456)--(3.028,6.457)--(3.018,6.456)--(3.008,6.455)--(2.998,6.454)%
    --(2.988,6.452)--(2.979,6.450)--(2.969,6.447)--(2.960,6.444)--(2.951,6.440)%
    --(2.942,6.436)--(2.933,6.431)--(2.925,6.426)--(2.916,6.420)--(2.909,6.414)%
    --(2.901,6.408)--(2.894,6.401)--(2.887,6.394)--(2.881,6.386)--(2.875,6.379)%
    --(2.869,6.370)--(2.864,6.362)--(2.859,6.353)--(2.855,6.344)--(2.851,6.335)%
    --(2.848,6.326)--(2.845,6.316)--(2.843,6.307)--(2.841,6.297)--(2.840,6.287)%
    --(2.839,6.277)--(2.839,6.268)--(2.839,6.258)--(2.840,6.248)--(2.841,6.238)%
    --(2.843,6.228)--(2.845,6.219)--(2.848,6.209)--(2.851,6.200)--(2.855,6.191)%
    --(2.859,6.182)--(2.864,6.173)--(2.869,6.165)--(2.875,6.156)--(2.881,6.149)%
    --(2.887,6.141)--(2.894,6.134)--(2.901,6.127)--(2.909,6.121)--(2.916,6.115)%
    --(2.925,6.109)--(2.933,6.104)--(2.942,6.099)--(2.951,6.095)--(2.960,6.091)%
    --(2.969,6.088)--(2.979,6.085)--(2.988,6.083)--(2.998,6.081)--(3.008,6.080)%
    --(3.018,6.079)--(3.027,6.079)--(3.037,6.079)--(3.047,6.080)--(3.057,6.081)%
    --(3.067,6.083)--(3.076,6.085)--(3.086,6.088)--(3.095,6.091)--(3.104,6.095)%
    --(3.113,6.099)--(3.122,6.104)--(3.130,6.109)--(3.139,6.115)--(3.146,6.121)%
    --(3.154,6.127)--(3.161,6.134)--(3.168,6.141)--(3.174,6.149)--(3.180,6.156)%
    --(3.186,6.165)--(3.191,6.173)--(3.196,6.182)--(3.200,6.191)--(3.204,6.200)%
    --(3.207,6.209)--(3.210,6.219)--(3.212,6.228)--(3.214,6.238)--(3.215,6.248)%
    --(3.216,6.258)--(3.217,6.267)--cycle;
\gpfill{color=gp lt color border,opacity=0.50} (6.002,5.211)--(6.001,5.225)--(6.000,5.239)--(5.998,5.253)%
    --(5.996,5.267)--(5.992,5.281)--(5.988,5.294)--(5.984,5.308)--(5.978,5.321)%
    --(5.972,5.334)--(5.965,5.346)--(5.958,5.358)--(5.950,5.370)--(5.941,5.381)%
    --(5.932,5.392)--(5.922,5.402)--(5.912,5.412)--(5.901,5.421)--(5.890,5.430)%
    --(5.878,5.438)--(5.866,5.445)--(5.854,5.452)--(5.841,5.458)--(5.828,5.464)%
    --(5.814,5.468)--(5.801,5.472)--(5.787,5.476)--(5.773,5.478)--(5.759,5.480)%
    --(5.745,5.481)--(5.731,5.482)--(5.716,5.481)--(5.702,5.480)--(5.688,5.478)%
    --(5.674,5.476)--(5.660,5.472)--(5.647,5.468)--(5.633,5.464)--(5.620,5.458)%
    --(5.607,5.452)--(5.595,5.445)--(5.583,5.438)--(5.571,5.430)--(5.560,5.421)%
    --(5.549,5.412)--(5.539,5.402)--(5.529,5.392)--(5.520,5.381)--(5.511,5.370)%
    --(5.503,5.358)--(5.496,5.346)--(5.489,5.334)--(5.483,5.321)--(5.477,5.308)%
    --(5.473,5.294)--(5.469,5.281)--(5.465,5.267)--(5.463,5.253)--(5.461,5.239)%
    --(5.460,5.225)--(5.460,5.211)--(5.460,5.196)--(5.461,5.182)--(5.463,5.168)%
    --(5.465,5.154)--(5.469,5.140)--(5.473,5.127)--(5.477,5.113)--(5.483,5.100)%
    --(5.489,5.087)--(5.496,5.075)--(5.503,5.063)--(5.511,5.051)--(5.520,5.040)%
    --(5.529,5.029)--(5.539,5.019)--(5.549,5.009)--(5.560,5.000)--(5.571,4.991)%
    --(5.583,4.983)--(5.595,4.976)--(5.607,4.969)--(5.620,4.963)--(5.633,4.957)%
    --(5.647,4.953)--(5.660,4.949)--(5.674,4.945)--(5.688,4.943)--(5.702,4.941)%
    --(5.716,4.940)--(5.730,4.940)--(5.745,4.940)--(5.759,4.941)--(5.773,4.943)%
    --(5.787,4.945)--(5.801,4.949)--(5.814,4.953)--(5.828,4.957)--(5.841,4.963)%
    --(5.854,4.969)--(5.866,4.976)--(5.878,4.983)--(5.890,4.991)--(5.901,5.000)%
    --(5.912,5.009)--(5.922,5.019)--(5.932,5.029)--(5.941,5.040)--(5.950,5.051)%
    --(5.958,5.063)--(5.965,5.075)--(5.972,5.087)--(5.978,5.100)--(5.984,5.113)%
    --(5.988,5.127)--(5.992,5.140)--(5.996,5.154)--(5.998,5.168)--(6.000,5.182)%
    --(6.001,5.196)--(6.002,5.210)--cycle;
\gpfill{color=gp lt color border,opacity=0.50} (1.460,4.155)--(1.459,4.171)--(1.458,4.188)--(1.456,4.205)%
    --(1.452,4.222)--(1.448,4.238)--(1.444,4.255)--(1.438,4.271)--(1.431,4.286)%
    --(1.424,4.302)--(1.416,4.316)--(1.407,4.331)--(1.398,4.345)--(1.387,4.358)%
    --(1.376,4.371)--(1.365,4.384)--(1.352,4.395)--(1.339,4.406)--(1.326,4.417)%
    --(1.312,4.426)--(1.298,4.435)--(1.283,4.443)--(1.267,4.450)--(1.252,4.457)%
    --(1.236,4.463)--(1.219,4.467)--(1.203,4.471)--(1.186,4.475)--(1.169,4.477)%
    --(1.152,4.478)--(1.136,4.479)--(1.119,4.478)--(1.102,4.477)--(1.085,4.475)%
    --(1.068,4.471)--(1.052,4.467)--(1.035,4.463)--(1.019,4.457)--(1.004,4.450)%
    --(0.988,4.443)--(0.974,4.435)--(0.959,4.426)--(0.945,4.417)--(0.932,4.406)%
    --(0.919,4.395)--(0.906,4.384)--(0.895,4.371)--(0.884,4.358)--(0.873,4.345)%
    --(0.864,4.331)--(0.855,4.316)--(0.847,4.302)--(0.840,4.286)--(0.833,4.271)%
    --(0.827,4.255)--(0.823,4.238)--(0.819,4.222)--(0.815,4.205)--(0.813,4.188)%
    --(0.812,4.171)--(0.812,4.155)--(0.812,4.138)--(0.813,4.121)--(0.815,4.104)%
    --(0.819,4.087)--(0.823,4.071)--(0.827,4.054)--(0.833,4.038)--(0.840,4.023)%
    --(0.847,4.007)--(0.855,3.992)--(0.864,3.978)--(0.873,3.964)--(0.884,3.951)%
    --(0.895,3.938)--(0.906,3.925)--(0.919,3.914)--(0.932,3.903)--(0.945,3.892)%
    --(0.959,3.883)--(0.973,3.874)--(0.988,3.866)--(1.004,3.859)--(1.019,3.852)%
    --(1.035,3.846)--(1.052,3.842)--(1.068,3.838)--(1.085,3.834)--(1.102,3.832)%
    --(1.119,3.831)--(1.135,3.831)--(1.152,3.831)--(1.169,3.832)--(1.186,3.834)%
    --(1.203,3.838)--(1.219,3.842)--(1.236,3.846)--(1.252,3.852)--(1.267,3.859)%
    --(1.283,3.866)--(1.298,3.874)--(1.312,3.883)--(1.326,3.892)--(1.339,3.903)%
    --(1.352,3.914)--(1.365,3.925)--(1.376,3.938)--(1.387,3.951)--(1.398,3.964)%
    --(1.407,3.978)--(1.416,3.992)--(1.424,4.007)--(1.431,4.023)--(1.438,4.038)%
    --(1.444,4.054)--(1.448,4.071)--(1.452,4.087)--(1.456,4.104)--(1.458,4.121)%
    --(1.459,4.138)--(1.460,4.154)--cycle;
\gpfill{color=gp lt color border,opacity=0.50} (4.082,3.098)--(4.081,3.110)--(4.080,3.123)--(4.079,3.136)%
    --(4.076,3.148)--(4.073,3.160)--(4.070,3.173)--(4.065,3.185)--(4.060,3.196)%
    --(4.055,3.208)--(4.049,3.219)--(4.042,3.230)--(4.035,3.240)--(4.027,3.250)%
    --(4.019,3.260)--(4.010,3.269)--(4.001,3.278)--(3.991,3.286)--(3.981,3.294)%
    --(3.971,3.301)--(3.960,3.308)--(3.949,3.314)--(3.937,3.319)--(3.926,3.324)%
    --(3.914,3.329)--(3.901,3.332)--(3.889,3.335)--(3.877,3.338)--(3.864,3.339)%
    --(3.851,3.340)--(3.839,3.341)--(3.826,3.340)--(3.813,3.339)--(3.800,3.338)%
    --(3.788,3.335)--(3.776,3.332)--(3.763,3.329)--(3.751,3.324)--(3.740,3.319)%
    --(3.728,3.314)--(3.717,3.308)--(3.706,3.301)--(3.696,3.294)--(3.686,3.286)%
    --(3.676,3.278)--(3.667,3.269)--(3.658,3.260)--(3.650,3.250)--(3.642,3.240)%
    --(3.635,3.230)--(3.628,3.219)--(3.622,3.208)--(3.617,3.196)--(3.612,3.185)%
    --(3.607,3.173)--(3.604,3.160)--(3.601,3.148)--(3.598,3.136)--(3.597,3.123)%
    --(3.596,3.110)--(3.596,3.098)--(3.596,3.085)--(3.597,3.072)--(3.598,3.059)%
    --(3.601,3.047)--(3.604,3.035)--(3.607,3.022)--(3.612,3.010)--(3.617,2.999)%
    --(3.622,2.987)--(3.628,2.976)--(3.635,2.965)--(3.642,2.955)--(3.650,2.945)%
    --(3.658,2.935)--(3.667,2.926)--(3.676,2.917)--(3.686,2.909)--(3.696,2.901)%
    --(3.706,2.894)--(3.717,2.887)--(3.728,2.881)--(3.740,2.876)--(3.751,2.871)%
    --(3.763,2.866)--(3.776,2.863)--(3.788,2.860)--(3.800,2.857)--(3.813,2.856)%
    --(3.826,2.855)--(3.838,2.855)--(3.851,2.855)--(3.864,2.856)--(3.877,2.857)%
    --(3.889,2.860)--(3.901,2.863)--(3.914,2.866)--(3.926,2.871)--(3.937,2.876)%
    --(3.949,2.881)--(3.960,2.887)--(3.971,2.894)--(3.981,2.901)--(3.991,2.909)%
    --(4.001,2.917)--(4.010,2.926)--(4.019,2.935)--(4.027,2.945)--(4.035,2.955)%
    --(4.042,2.965)--(4.049,2.976)--(4.055,2.987)--(4.060,2.999)--(4.065,3.010)%
    --(4.070,3.022)--(4.073,3.035)--(4.076,3.047)--(4.079,3.059)--(4.080,3.072)%
    --(4.081,3.085)--(4.082,3.097)--cycle;
\gpfill{color=gp lt color border,opacity=0.50} (1.759,6.268)--(1.758,6.272)--(1.758,6.276)--(1.757,6.280)%
    --(1.757,6.285)--(1.756,6.289)--(1.754,6.293)--(1.753,6.297)--(1.751,6.301)%
    --(1.750,6.305)--(1.748,6.308)--(1.745,6.312)--(1.743,6.316)--(1.740,6.319)%
    --(1.737,6.322)--(1.734,6.325)--(1.731,6.328)--(1.728,6.331)--(1.725,6.334)%
    --(1.721,6.336)--(1.718,6.339)--(1.714,6.341)--(1.710,6.342)--(1.706,6.344)%
    --(1.702,6.345)--(1.698,6.347)--(1.694,6.348)--(1.689,6.348)--(1.685,6.349)%
    --(1.681,6.349)--(1.677,6.350)--(1.672,6.349)--(1.668,6.349)--(1.664,6.348)%
    --(1.659,6.348)--(1.655,6.347)--(1.651,6.345)--(1.647,6.344)--(1.643,6.342)%
    --(1.639,6.341)--(1.636,6.339)--(1.632,6.336)--(1.628,6.334)--(1.625,6.331)%
    --(1.622,6.328)--(1.619,6.325)--(1.616,6.322)--(1.613,6.319)--(1.610,6.316)%
    --(1.608,6.312)--(1.605,6.308)--(1.603,6.305)--(1.602,6.301)--(1.600,6.297)%
    --(1.599,6.293)--(1.597,6.289)--(1.596,6.285)--(1.596,6.280)--(1.595,6.276)%
    --(1.595,6.272)--(1.595,6.268)--(1.595,6.263)--(1.595,6.259)--(1.596,6.255)%
    --(1.596,6.250)--(1.597,6.246)--(1.599,6.242)--(1.600,6.238)--(1.602,6.234)%
    --(1.603,6.230)--(1.605,6.226)--(1.608,6.223)--(1.610,6.219)--(1.613,6.216)%
    --(1.616,6.213)--(1.619,6.210)--(1.622,6.207)--(1.625,6.204)--(1.628,6.201)%
    --(1.632,6.199)--(1.635,6.196)--(1.639,6.194)--(1.643,6.193)--(1.647,6.191)%
    --(1.651,6.190)--(1.655,6.188)--(1.659,6.187)--(1.664,6.187)--(1.668,6.186)%
    --(1.672,6.186)--(1.676,6.186)--(1.681,6.186)--(1.685,6.186)--(1.689,6.187)%
    --(1.694,6.187)--(1.698,6.188)--(1.702,6.190)--(1.706,6.191)--(1.710,6.193)%
    --(1.714,6.194)--(1.718,6.196)--(1.721,6.199)--(1.725,6.201)--(1.728,6.204)%
    --(1.731,6.207)--(1.734,6.210)--(1.737,6.213)--(1.740,6.216)--(1.743,6.219)%
    --(1.745,6.223)--(1.748,6.226)--(1.750,6.230)--(1.751,6.234)--(1.753,6.238)%
    --(1.754,6.242)--(1.756,6.246)--(1.757,6.250)--(1.757,6.255)--(1.758,6.259)%
    --(1.758,6.263)--(1.759,6.267)--cycle;
\gpfill{color=gp lt color border,opacity=0.50} (4.703,5.211)--(4.702,5.227)--(4.701,5.244)--(4.699,5.261)%
    --(4.695,5.278)--(4.691,5.294)--(4.687,5.311)--(4.681,5.327)--(4.674,5.342)%
    --(4.667,5.358)--(4.659,5.372)--(4.650,5.387)--(4.641,5.401)--(4.630,5.414)%
    --(4.619,5.427)--(4.608,5.440)--(4.595,5.451)--(4.582,5.462)--(4.569,5.473)%
    --(4.555,5.482)--(4.541,5.491)--(4.526,5.499)--(4.510,5.506)--(4.495,5.513)%
    --(4.479,5.519)--(4.462,5.523)--(4.446,5.527)--(4.429,5.531)--(4.412,5.533)%
    --(4.395,5.534)--(4.379,5.535)--(4.362,5.534)--(4.345,5.533)--(4.328,5.531)%
    --(4.311,5.527)--(4.295,5.523)--(4.278,5.519)--(4.262,5.513)--(4.247,5.506)%
    --(4.231,5.499)--(4.217,5.491)--(4.202,5.482)--(4.188,5.473)--(4.175,5.462)%
    --(4.162,5.451)--(4.149,5.440)--(4.138,5.427)--(4.127,5.414)--(4.116,5.401)%
    --(4.107,5.387)--(4.098,5.372)--(4.090,5.358)--(4.083,5.342)--(4.076,5.327)%
    --(4.070,5.311)--(4.066,5.294)--(4.062,5.278)--(4.058,5.261)--(4.056,5.244)%
    --(4.055,5.227)--(4.055,5.211)--(4.055,5.194)--(4.056,5.177)--(4.058,5.160)%
    --(4.062,5.143)--(4.066,5.127)--(4.070,5.110)--(4.076,5.094)--(4.083,5.079)%
    --(4.090,5.063)--(4.098,5.048)--(4.107,5.034)--(4.116,5.020)--(4.127,5.007)%
    --(4.138,4.994)--(4.149,4.981)--(4.162,4.970)--(4.175,4.959)--(4.188,4.948)%
    --(4.202,4.939)--(4.216,4.930)--(4.231,4.922)--(4.247,4.915)--(4.262,4.908)%
    --(4.278,4.902)--(4.295,4.898)--(4.311,4.894)--(4.328,4.890)--(4.345,4.888)%
    --(4.362,4.887)--(4.378,4.887)--(4.395,4.887)--(4.412,4.888)--(4.429,4.890)%
    --(4.446,4.894)--(4.462,4.898)--(4.479,4.902)--(4.495,4.908)--(4.510,4.915)%
    --(4.526,4.922)--(4.541,4.930)--(4.555,4.939)--(4.569,4.948)--(4.582,4.959)%
    --(4.595,4.970)--(4.608,4.981)--(4.619,4.994)--(4.630,5.007)--(4.641,5.020)%
    --(4.650,5.034)--(4.659,5.048)--(4.667,5.063)--(4.674,5.079)--(4.681,5.094)%
    --(4.687,5.110)--(4.691,5.127)--(4.695,5.143)--(4.699,5.160)--(4.701,5.177)%
    --(4.702,5.194)--(4.703,5.210)--cycle;
\gpfill{color=gp lt color border,opacity=0.50} (3.298,5.211)--(3.297,5.225)--(3.296,5.239)--(3.294,5.253)%
    --(3.292,5.267)--(3.288,5.280)--(3.284,5.294)--(3.280,5.307)--(3.274,5.320)%
    --(3.268,5.333)--(3.261,5.345)--(3.254,5.358)--(3.246,5.369)--(3.237,5.380)%
    --(3.228,5.391)--(3.218,5.401)--(3.208,5.411)--(3.197,5.420)--(3.186,5.429)%
    --(3.175,5.437)--(3.163,5.444)--(3.150,5.451)--(3.137,5.457)--(3.124,5.463)%
    --(3.111,5.467)--(3.097,5.471)--(3.084,5.475)--(3.070,5.477)--(3.056,5.479)%
    --(3.042,5.480)--(3.028,5.481)--(3.013,5.480)--(2.999,5.479)--(2.985,5.477)%
    --(2.971,5.475)--(2.958,5.471)--(2.944,5.467)--(2.931,5.463)--(2.918,5.457)%
    --(2.905,5.451)--(2.893,5.444)--(2.880,5.437)--(2.869,5.429)--(2.858,5.420)%
    --(2.847,5.411)--(2.837,5.401)--(2.827,5.391)--(2.818,5.380)--(2.809,5.369)%
    --(2.801,5.358)--(2.794,5.345)--(2.787,5.333)--(2.781,5.320)--(2.775,5.307)%
    --(2.771,5.294)--(2.767,5.280)--(2.763,5.267)--(2.761,5.253)--(2.759,5.239)%
    --(2.758,5.225)--(2.758,5.211)--(2.758,5.196)--(2.759,5.182)--(2.761,5.168)%
    --(2.763,5.154)--(2.767,5.141)--(2.771,5.127)--(2.775,5.114)--(2.781,5.101)%
    --(2.787,5.088)--(2.794,5.075)--(2.801,5.063)--(2.809,5.052)--(2.818,5.041)%
    --(2.827,5.030)--(2.837,5.020)--(2.847,5.010)--(2.858,5.001)--(2.869,4.992)%
    --(2.880,4.984)--(2.892,4.977)--(2.905,4.970)--(2.918,4.964)--(2.931,4.958)%
    --(2.944,4.954)--(2.958,4.950)--(2.971,4.946)--(2.985,4.944)--(2.999,4.942)%
    --(3.013,4.941)--(3.027,4.941)--(3.042,4.941)--(3.056,4.942)--(3.070,4.944)%
    --(3.084,4.946)--(3.097,4.950)--(3.111,4.954)--(3.124,4.958)--(3.137,4.964)%
    --(3.150,4.970)--(3.163,4.977)--(3.175,4.984)--(3.186,4.992)--(3.197,5.001)%
    --(3.208,5.010)--(3.218,5.020)--(3.228,5.030)--(3.237,5.041)--(3.246,5.052)%
    --(3.254,5.063)--(3.261,5.075)--(3.268,5.088)--(3.274,5.101)--(3.280,5.114)%
    --(3.284,5.127)--(3.288,5.141)--(3.292,5.154)--(3.294,5.168)--(3.296,5.182)%
    --(3.297,5.196)--(3.298,5.210)--cycle;
\gpfill{color=gp lt color border,opacity=0.50} (1.568,3.098)--(1.567,3.120)--(1.565,3.143)--(1.562,3.165)%
    --(1.558,3.187)--(1.553,3.209)--(1.546,3.231)--(1.539,3.252)--(1.530,3.273)%
    --(1.520,3.294)--(1.510,3.313)--(1.498,3.333)--(1.485,3.351)--(1.471,3.369)%
    --(1.457,3.387)--(1.441,3.403)--(1.425,3.419)--(1.407,3.433)--(1.389,3.447)%
    --(1.371,3.460)--(1.352,3.472)--(1.332,3.482)--(1.311,3.492)--(1.290,3.501)%
    --(1.269,3.508)--(1.247,3.515)--(1.225,3.520)--(1.203,3.524)--(1.181,3.527)%
    --(1.158,3.529)--(1.136,3.530)--(1.113,3.529)--(1.090,3.527)--(1.068,3.524)%
    --(1.046,3.520)--(1.024,3.515)--(1.002,3.508)--(0.981,3.501)--(0.960,3.492)%
    --(0.939,3.482)--(0.920,3.472)--(0.900,3.460)--(0.882,3.447)--(0.864,3.433)%
    --(0.846,3.419)--(0.830,3.403)--(0.814,3.387)--(0.800,3.369)--(0.786,3.351)%
    --(0.773,3.333)--(0.761,3.313)--(0.751,3.294)--(0.741,3.273)--(0.732,3.252)%
    --(0.725,3.231)--(0.718,3.209)--(0.713,3.187)--(0.709,3.165)--(0.706,3.143)%
    --(0.704,3.120)--(0.704,3.098)--(0.704,3.075)--(0.706,3.052)--(0.709,3.030)%
    --(0.713,3.008)--(0.718,2.986)--(0.725,2.964)--(0.732,2.943)--(0.741,2.922)%
    --(0.751,2.901)--(0.761,2.881)--(0.773,2.862)--(0.786,2.844)--(0.800,2.826)%
    --(0.814,2.808)--(0.830,2.792)--(0.846,2.776)--(0.864,2.762)--(0.882,2.748)%
    --(0.900,2.735)--(0.919,2.723)--(0.939,2.713)--(0.960,2.703)--(0.981,2.694)%
    --(1.002,2.687)--(1.024,2.680)--(1.046,2.675)--(1.068,2.671)--(1.090,2.668)%
    --(1.113,2.666)--(1.135,2.666)--(1.158,2.666)--(1.181,2.668)--(1.203,2.671)%
    --(1.225,2.675)--(1.247,2.680)--(1.269,2.687)--(1.290,2.694)--(1.311,2.703)%
    --(1.332,2.713)--(1.352,2.723)--(1.371,2.735)--(1.389,2.748)--(1.407,2.762)%
    --(1.425,2.776)--(1.441,2.792)--(1.457,2.808)--(1.471,2.826)--(1.485,2.844)%
    --(1.498,2.862)--(1.510,2.881)--(1.520,2.901)--(1.530,2.922)--(1.539,2.943)%
    --(1.546,2.964)--(1.553,2.986)--(1.558,3.008)--(1.562,3.030)--(1.565,3.052)%
    --(1.567,3.075)--(1.568,3.097)--cycle;
\gpfill{color=gp lt color border,opacity=0.50} (8.433,7.324)--(8.433,7.324)--(8.433,7.324)--(8.433,7.324)%
    --(8.433,7.324)--(8.433,7.324)--(8.433,7.324)--(8.433,7.324)--(8.433,7.324)%
    --(8.433,7.324)--(8.433,7.324)--(8.433,7.324)--(8.433,7.324)--(8.433,7.324)%
    --(8.433,7.324)--(8.433,7.324)--(8.433,7.324)--(8.433,7.324)--(8.433,7.324)%
    --(8.433,7.324)--(8.433,7.324)--(8.433,7.324)--(8.433,7.324)--(8.433,7.324)%
    --(8.433,7.324)--(8.433,7.324)--(8.433,7.324)--(8.433,7.324)--(8.433,7.324)%
    --(8.433,7.324)--(8.433,7.324)--(8.433,7.324)--(8.433,7.324)--(8.433,7.324)%
    --(8.433,7.324)--(8.433,7.324)--(8.433,7.324)--(8.433,7.324)--(8.433,7.324)%
    --(8.433,7.324)--(8.433,7.324)--(8.433,7.324)--(8.433,7.324)--(8.433,7.324)%
    --(8.433,7.324)--(8.433,7.324)--(8.433,7.324)--(8.433,7.324)--(8.433,7.324)%
    --(8.433,7.324)--(8.433,7.324)--(8.433,7.324)--(8.433,7.324)--(8.433,7.324)%
    --(8.433,7.324)--(8.433,7.324)--(8.433,7.324)--(8.433,7.324)--(8.433,7.324)%
    --(8.433,7.324)--(8.433,7.324)--(8.433,7.324)--(8.433,7.324)--(8.433,7.324)%
    --(8.433,7.324)--(8.433,7.324)--(8.433,7.324)--(8.433,7.324)--(8.433,7.324)%
    --(8.433,7.324)--(8.433,7.324)--(8.433,7.324)--(8.433,7.324)--(8.433,7.324)%
    --(8.433,7.324)--(8.433,7.324)--(8.433,7.324)--(8.433,7.324)--(8.433,7.324)%
    --(8.433,7.324)--(8.433,7.324)--(8.433,7.324)--(8.433,7.324)--(8.433,7.324)%
    --(8.433,7.324)--(8.433,7.324)--(8.433,7.324)--(8.433,7.324)--(8.433,7.324)%
    --(8.433,7.324)--(8.433,7.324)--(8.433,7.324)--(8.433,7.324)--(8.433,7.324)%
    --(8.433,7.324)--(8.433,7.324)--(8.433,7.324)--(8.433,7.324)--(8.433,7.324)%
    --(8.433,7.324)--(8.433,7.324)--(8.433,7.324)--(8.433,7.324)--(8.433,7.324)%
    --(8.433,7.324)--(8.433,7.324)--(8.433,7.324)--(8.433,7.324)--(8.433,7.324)%
    --(8.433,7.324)--(8.433,7.324)--(8.433,7.324)--(8.433,7.324)--(8.433,7.324)%
    --(8.433,7.324)--(8.433,7.324)--(8.433,7.324)--(8.433,7.324)--(8.433,7.324)%
    --(8.433,7.324)--cycle;
\gpfill{color=gp lt color border,opacity=0.50} (7.109,7.324)--(7.108,7.325)--(7.108,7.326)--(7.108,7.328)%
    --(7.108,7.329)--(7.108,7.330)--(7.107,7.332)--(7.107,7.333)--(7.106,7.334)%
    --(7.106,7.336)--(7.105,7.337)--(7.104,7.338)--(7.103,7.339)--(7.102,7.340)%
    --(7.102,7.342)--(7.101,7.343)--(7.100,7.344)--(7.098,7.344)--(7.097,7.345)%
    --(7.096,7.346)--(7.095,7.347)--(7.094,7.348)--(7.092,7.348)--(7.091,7.349)%
    --(7.090,7.349)--(7.088,7.350)--(7.087,7.350)--(7.086,7.350)--(7.084,7.350)%
    --(7.083,7.350)--(7.082,7.351)--(7.080,7.350)--(7.079,7.350)--(7.077,7.350)%
    --(7.076,7.350)--(7.075,7.350)--(7.073,7.349)--(7.072,7.349)--(7.071,7.348)%
    --(7.069,7.348)--(7.068,7.347)--(7.067,7.346)--(7.066,7.345)--(7.065,7.344)%
    --(7.063,7.344)--(7.062,7.343)--(7.061,7.342)--(7.061,7.340)--(7.060,7.339)%
    --(7.059,7.338)--(7.058,7.337)--(7.057,7.336)--(7.057,7.334)--(7.056,7.333)%
    --(7.056,7.332)--(7.055,7.330)--(7.055,7.329)--(7.055,7.328)--(7.055,7.326)%
    --(7.055,7.325)--(7.055,7.324)--(7.055,7.322)--(7.055,7.321)--(7.055,7.319)%
    --(7.055,7.318)--(7.055,7.317)--(7.056,7.315)--(7.056,7.314)--(7.057,7.313)%
    --(7.057,7.311)--(7.058,7.310)--(7.059,7.309)--(7.060,7.308)--(7.061,7.307)%
    --(7.061,7.305)--(7.062,7.304)--(7.063,7.303)--(7.065,7.303)--(7.066,7.302)%
    --(7.067,7.301)--(7.068,7.300)--(7.069,7.299)--(7.071,7.299)--(7.072,7.298)%
    --(7.073,7.298)--(7.075,7.297)--(7.076,7.297)--(7.077,7.297)--(7.079,7.297)%
    --(7.080,7.297)--(7.081,7.297)--(7.083,7.297)--(7.084,7.297)--(7.086,7.297)%
    --(7.087,7.297)--(7.088,7.297)--(7.090,7.298)--(7.091,7.298)--(7.092,7.299)%
    --(7.094,7.299)--(7.095,7.300)--(7.096,7.301)--(7.097,7.302)--(7.098,7.303)%
    --(7.100,7.303)--(7.101,7.304)--(7.102,7.305)--(7.102,7.307)--(7.103,7.308)%
    --(7.104,7.309)--(7.105,7.310)--(7.106,7.311)--(7.106,7.313)--(7.107,7.314)%
    --(7.107,7.315)--(7.108,7.317)--(7.108,7.318)--(7.108,7.319)--(7.108,7.321)%
    --(7.108,7.322)--(7.109,7.323)--cycle;
\gpfill{color=gp lt color border,opacity=0.50} (1.975,5.211)--(1.974,5.226)--(1.973,5.242)--(1.971,5.257)%
    --(1.968,5.272)--(1.964,5.288)--(1.960,5.303)--(1.955,5.317)--(1.949,5.332)%
    --(1.942,5.346)--(1.935,5.359)--(1.926,5.373)--(1.918,5.386)--(1.908,5.398)%
    --(1.898,5.410)--(1.887,5.421)--(1.876,5.432)--(1.864,5.442)--(1.852,5.452)%
    --(1.839,5.460)--(1.826,5.469)--(1.812,5.476)--(1.798,5.483)--(1.783,5.489)%
    --(1.769,5.494)--(1.754,5.498)--(1.738,5.502)--(1.723,5.505)--(1.708,5.507)%
    --(1.692,5.508)--(1.677,5.509)--(1.661,5.508)--(1.645,5.507)--(1.630,5.505)%
    --(1.615,5.502)--(1.599,5.498)--(1.584,5.494)--(1.570,5.489)--(1.555,5.483)%
    --(1.541,5.476)--(1.528,5.469)--(1.514,5.460)--(1.501,5.452)--(1.489,5.442)%
    --(1.477,5.432)--(1.466,5.421)--(1.455,5.410)--(1.445,5.398)--(1.435,5.386)%
    --(1.427,5.373)--(1.418,5.359)--(1.411,5.346)--(1.404,5.332)--(1.398,5.317)%
    --(1.393,5.303)--(1.389,5.288)--(1.385,5.272)--(1.382,5.257)--(1.380,5.242)%
    --(1.379,5.226)--(1.379,5.211)--(1.379,5.195)--(1.380,5.179)--(1.382,5.164)%
    --(1.385,5.149)--(1.389,5.133)--(1.393,5.118)--(1.398,5.104)--(1.404,5.089)%
    --(1.411,5.075)--(1.418,5.061)--(1.427,5.048)--(1.435,5.035)--(1.445,5.023)%
    --(1.455,5.011)--(1.466,5.000)--(1.477,4.989)--(1.489,4.979)--(1.501,4.969)%
    --(1.514,4.961)--(1.527,4.952)--(1.541,4.945)--(1.555,4.938)--(1.570,4.932)%
    --(1.584,4.927)--(1.599,4.923)--(1.615,4.919)--(1.630,4.916)--(1.645,4.914)%
    --(1.661,4.913)--(1.676,4.913)--(1.692,4.913)--(1.708,4.914)--(1.723,4.916)%
    --(1.738,4.919)--(1.754,4.923)--(1.769,4.927)--(1.783,4.932)--(1.798,4.938)%
    --(1.812,4.945)--(1.826,4.952)--(1.839,4.961)--(1.852,4.969)--(1.864,4.979)%
    --(1.876,4.989)--(1.887,5.000)--(1.898,5.011)--(1.908,5.023)--(1.918,5.035)%
    --(1.926,5.048)--(1.935,5.061)--(1.942,5.075)--(1.949,5.089)--(1.955,5.104)%
    --(1.960,5.118)--(1.964,5.133)--(1.968,5.149)--(1.971,5.164)--(1.973,5.179)%
    --(1.974,5.195)--(1.975,5.210)--cycle;
\gpfill{color=gp lt color border,opacity=0.50} (4.379,8.381)--(4.379,8.381)--(4.379,8.381)--(4.379,8.381)%
    --(4.379,8.381)--(4.379,8.381)--(4.379,8.381)--(4.379,8.381)--(4.379,8.381)%
    --(4.379,8.381)--(4.379,8.381)--(4.379,8.381)--(4.379,8.381)--(4.379,8.381)%
    --(4.379,8.381)--(4.379,8.381)--(4.379,8.381)--(4.379,8.381)--(4.379,8.381)%
    --(4.379,8.381)--(4.379,8.381)--(4.379,8.381)--(4.379,8.381)--(4.379,8.381)%
    --(4.379,8.381)--(4.379,8.381)--(4.379,8.381)--(4.379,8.381)--(4.379,8.381)%
    --(4.379,8.381)--(4.379,8.381)--(4.379,8.381)--(4.379,8.381)--(4.379,8.381)%
    --(4.379,8.381)--(4.379,8.381)--(4.379,8.381)--(4.379,8.381)--(4.379,8.381)%
    --(4.379,8.381)--(4.379,8.381)--(4.379,8.381)--(4.379,8.381)--(4.379,8.381)%
    --(4.379,8.381)--(4.379,8.381)--(4.379,8.381)--(4.379,8.381)--(4.379,8.381)%
    --(4.379,8.381)--(4.379,8.381)--(4.379,8.381)--(4.379,8.381)--(4.379,8.381)%
    --(4.379,8.381)--(4.379,8.381)--(4.379,8.381)--(4.379,8.381)--(4.379,8.381)%
    --(4.379,8.381)--(4.379,8.381)--(4.379,8.381)--(4.379,8.381)--(4.379,8.381)%
    --(4.379,8.381)--(4.379,8.381)--(4.379,8.381)--(4.379,8.381)--(4.379,8.381)%
    --(4.379,8.381)--(4.379,8.381)--(4.379,8.381)--(4.379,8.381)--(4.379,8.381)%
    --(4.379,8.381)--(4.379,8.381)--(4.379,8.381)--(4.379,8.381)--(4.379,8.381)%
    --(4.379,8.381)--(4.379,8.381)--(4.379,8.381)--(4.379,8.381)--(4.379,8.381)%
    --(4.379,8.381)--(4.379,8.381)--(4.379,8.381)--(4.379,8.381)--(4.379,8.381)%
    --(4.379,8.381)--(4.379,8.381)--(4.379,8.381)--(4.379,8.381)--(4.379,8.381)%
    --(4.379,8.381)--(4.379,8.381)--(4.379,8.381)--(4.379,8.381)--(4.379,8.381)%
    --(4.379,8.381)--(4.379,8.381)--(4.379,8.381)--(4.379,8.381)--(4.379,8.381)%
    --(4.379,8.381)--(4.379,8.381)--(4.379,8.381)--(4.379,8.381)--(4.379,8.381)%
    --(4.379,8.381)--(4.379,8.381)--(4.379,8.381)--(4.379,8.381)--(4.379,8.381)%
    --(4.379,8.381)--(4.379,8.381)--(4.379,8.381)--(4.379,8.381)--(4.379,8.381)%
    --(4.379,8.381)--cycle;
\gpfill{color=gp lt color border,opacity=0.50} (11.758,4.155)--(11.757,4.159)--(11.757,4.163)--(11.757,4.167)%
    --(11.756,4.171)--(11.755,4.175)--(11.754,4.180)--(11.752,4.184)--(11.750,4.187)%
    --(11.749,4.191)--(11.747,4.195)--(11.744,4.199)--(11.742,4.202)--(11.739,4.205)%
    --(11.737,4.209)--(11.734,4.212)--(11.731,4.215)--(11.727,4.217)--(11.724,4.220)%
    --(11.721,4.222)--(11.717,4.225)--(11.713,4.227)--(11.709,4.228)--(11.706,4.230)%
    --(11.702,4.232)--(11.697,4.233)--(11.693,4.234)--(11.689,4.235)--(11.685,4.235)%
    --(11.681,4.235)--(11.677,4.236)--(11.672,4.235)--(11.668,4.235)--(11.664,4.235)%
    --(11.660,4.234)--(11.656,4.233)--(11.651,4.232)--(11.647,4.230)--(11.644,4.228)%
    --(11.640,4.227)--(11.636,4.225)--(11.632,4.222)--(11.629,4.220)--(11.626,4.217)%
    --(11.622,4.215)--(11.619,4.212)--(11.616,4.209)--(11.614,4.205)--(11.611,4.202)%
    --(11.609,4.199)--(11.606,4.195)--(11.604,4.191)--(11.603,4.187)--(11.601,4.184)%
    --(11.599,4.180)--(11.598,4.175)--(11.597,4.171)--(11.596,4.167)--(11.596,4.163)%
    --(11.596,4.159)--(11.596,4.155)--(11.596,4.150)--(11.596,4.146)--(11.596,4.142)%
    --(11.597,4.138)--(11.598,4.134)--(11.599,4.129)--(11.601,4.125)--(11.603,4.122)%
    --(11.604,4.118)--(11.606,4.114)--(11.609,4.110)--(11.611,4.107)--(11.614,4.104)%
    --(11.616,4.100)--(11.619,4.097)--(11.622,4.094)--(11.626,4.092)--(11.629,4.089)%
    --(11.632,4.087)--(11.636,4.084)--(11.640,4.082)--(11.644,4.081)--(11.647,4.079)%
    --(11.651,4.077)--(11.656,4.076)--(11.660,4.075)--(11.664,4.074)--(11.668,4.074)%
    --(11.672,4.074)--(11.676,4.074)--(11.681,4.074)--(11.685,4.074)--(11.689,4.074)%
    --(11.693,4.075)--(11.697,4.076)--(11.702,4.077)--(11.706,4.079)--(11.709,4.081)%
    --(11.713,4.082)--(11.717,4.084)--(11.721,4.087)--(11.724,4.089)--(11.727,4.092)%
    --(11.731,4.094)--(11.734,4.097)--(11.737,4.100)--(11.739,4.104)--(11.742,4.107)%
    --(11.744,4.110)--(11.747,4.114)--(11.749,4.118)--(11.750,4.122)--(11.752,4.125)%
    --(11.754,4.129)--(11.755,4.134)--(11.756,4.138)--(11.757,4.142)--(11.757,4.146)%
    --(11.757,4.150)--(11.758,4.154)--cycle;
\gpfill{color=gp lt color border,opacity=0.50} (3.028,8.381)--(3.028,8.381)--(3.028,8.381)--(3.028,8.381)%
    --(3.028,8.381)--(3.028,8.381)--(3.028,8.381)--(3.028,8.381)--(3.028,8.381)%
    --(3.028,8.381)--(3.028,8.381)--(3.028,8.381)--(3.028,8.381)--(3.028,8.381)%
    --(3.028,8.381)--(3.028,8.381)--(3.028,8.381)--(3.028,8.381)--(3.028,8.381)%
    --(3.028,8.381)--(3.028,8.381)--(3.028,8.381)--(3.028,8.381)--(3.028,8.381)%
    --(3.028,8.381)--(3.028,8.381)--(3.028,8.381)--(3.028,8.381)--(3.028,8.381)%
    --(3.028,8.381)--(3.028,8.381)--(3.028,8.381)--(3.028,8.381)--(3.028,8.381)%
    --(3.028,8.381)--(3.028,8.381)--(3.028,8.381)--(3.028,8.381)--(3.028,8.381)%
    --(3.028,8.381)--(3.028,8.381)--(3.028,8.381)--(3.028,8.381)--(3.028,8.381)%
    --(3.028,8.381)--(3.028,8.381)--(3.028,8.381)--(3.028,8.381)--(3.028,8.381)%
    --(3.028,8.381)--(3.028,8.381)--(3.028,8.381)--(3.028,8.381)--(3.028,8.381)%
    --(3.028,8.381)--(3.028,8.381)--(3.028,8.381)--(3.028,8.381)--(3.028,8.381)%
    --(3.028,8.381)--(3.028,8.381)--(3.028,8.381)--(3.028,8.381)--(3.028,8.381)%
    --(3.028,8.381)--(3.028,8.381)--(3.028,8.381)--(3.028,8.381)--(3.028,8.381)%
    --(3.028,8.381)--(3.028,8.381)--(3.028,8.381)--(3.028,8.381)--(3.028,8.381)%
    --(3.028,8.381)--(3.028,8.381)--(3.028,8.381)--(3.028,8.381)--(3.028,8.381)%
    --(3.028,8.381)--(3.028,8.381)--(3.028,8.381)--(3.028,8.381)--(3.028,8.381)%
    --(3.028,8.381)--(3.028,8.381)--(3.028,8.381)--(3.028,8.381)--(3.028,8.381)%
    --(3.028,8.381)--(3.028,8.381)--(3.028,8.381)--(3.028,8.381)--(3.028,8.381)%
    --(3.028,8.381)--(3.028,8.381)--(3.028,8.381)--(3.028,8.381)--(3.028,8.381)%
    --(3.028,8.381)--(3.028,8.381)--(3.028,8.381)--(3.028,8.381)--(3.028,8.381)%
    --(3.028,8.381)--(3.028,8.381)--(3.028,8.381)--(3.028,8.381)--(3.028,8.381)%
    --(3.028,8.381)--(3.028,8.381)--(3.028,8.381)--(3.028,8.381)--(3.028,8.381)%
    --(3.028,8.381)--(3.028,8.381)--(3.028,8.381)--(3.028,8.381)--(3.028,8.381)%
    --(3.028,8.381)--cycle;
\gpfill{color=gp lt color border,opacity=0.50} (5.785,7.324)--(5.784,7.326)--(5.784,7.329)--(5.784,7.332)%
    --(5.783,7.335)--(5.783,7.337)--(5.782,7.340)--(5.781,7.343)--(5.780,7.345)%
    --(5.779,7.348)--(5.777,7.350)--(5.776,7.353)--(5.774,7.355)--(5.772,7.357)%
    --(5.771,7.360)--(5.769,7.362)--(5.767,7.364)--(5.764,7.365)--(5.762,7.367)%
    --(5.760,7.369)--(5.758,7.370)--(5.755,7.372)--(5.752,7.373)--(5.750,7.374)%
    --(5.747,7.375)--(5.744,7.376)--(5.742,7.376)--(5.739,7.377)--(5.736,7.377)%
    --(5.733,7.377)--(5.731,7.378)--(5.728,7.377)--(5.725,7.377)--(5.722,7.377)%
    --(5.719,7.376)--(5.717,7.376)--(5.714,7.375)--(5.711,7.374)--(5.709,7.373)%
    --(5.706,7.372)--(5.704,7.370)--(5.701,7.369)--(5.699,7.367)--(5.697,7.365)%
    --(5.694,7.364)--(5.692,7.362)--(5.690,7.360)--(5.689,7.357)--(5.687,7.355)%
    --(5.685,7.353)--(5.684,7.350)--(5.682,7.348)--(5.681,7.345)--(5.680,7.343)%
    --(5.679,7.340)--(5.678,7.337)--(5.678,7.335)--(5.677,7.332)--(5.677,7.329)%
    --(5.677,7.326)--(5.677,7.324)--(5.677,7.321)--(5.677,7.318)--(5.677,7.315)%
    --(5.678,7.312)--(5.678,7.310)--(5.679,7.307)--(5.680,7.304)--(5.681,7.302)%
    --(5.682,7.299)--(5.684,7.296)--(5.685,7.294)--(5.687,7.292)--(5.689,7.290)%
    --(5.690,7.287)--(5.692,7.285)--(5.694,7.283)--(5.697,7.282)--(5.699,7.280)%
    --(5.701,7.278)--(5.703,7.277)--(5.706,7.275)--(5.709,7.274)--(5.711,7.273)%
    --(5.714,7.272)--(5.717,7.271)--(5.719,7.271)--(5.722,7.270)--(5.725,7.270)%
    --(5.728,7.270)--(5.730,7.270)--(5.733,7.270)--(5.736,7.270)--(5.739,7.270)%
    --(5.742,7.271)--(5.744,7.271)--(5.747,7.272)--(5.750,7.273)--(5.752,7.274)%
    --(5.755,7.275)--(5.758,7.277)--(5.760,7.278)--(5.762,7.280)--(5.764,7.282)%
    --(5.767,7.283)--(5.769,7.285)--(5.771,7.287)--(5.772,7.290)--(5.774,7.292)%
    --(5.776,7.294)--(5.777,7.296)--(5.779,7.299)--(5.780,7.302)--(5.781,7.304)%
    --(5.782,7.307)--(5.783,7.310)--(5.783,7.312)--(5.784,7.315)--(5.784,7.318)%
    --(5.784,7.321)--(5.785,7.323)--cycle;
\gpfill{color=gp lt color border,opacity=0.50} (9.190,4.155)--(9.189,4.166)--(9.188,4.177)--(9.187,4.188)%
    --(9.185,4.199)--(9.182,4.210)--(9.179,4.221)--(9.175,4.232)--(9.171,4.242)%
    --(9.166,4.253)--(9.161,4.262)--(9.155,4.272)--(9.148,4.281)--(9.141,4.290)%
    --(9.134,4.299)--(9.126,4.307)--(9.118,4.315)--(9.109,4.322)--(9.100,4.329)%
    --(9.091,4.336)--(9.082,4.342)--(9.072,4.347)--(9.061,4.352)--(9.051,4.356)%
    --(9.040,4.360)--(9.029,4.363)--(9.018,4.366)--(9.007,4.368)--(8.996,4.369)%
    --(8.985,4.370)--(8.974,4.371)--(8.962,4.370)--(8.951,4.369)--(8.940,4.368)%
    --(8.929,4.366)--(8.918,4.363)--(8.907,4.360)--(8.896,4.356)--(8.886,4.352)%
    --(8.875,4.347)--(8.866,4.342)--(8.856,4.336)--(8.847,4.329)--(8.838,4.322)%
    --(8.829,4.315)--(8.821,4.307)--(8.813,4.299)--(8.806,4.290)--(8.799,4.281)%
    --(8.792,4.272)--(8.786,4.262)--(8.781,4.253)--(8.776,4.242)--(8.772,4.232)%
    --(8.768,4.221)--(8.765,4.210)--(8.762,4.199)--(8.760,4.188)--(8.759,4.177)%
    --(8.758,4.166)--(8.758,4.155)--(8.758,4.143)--(8.759,4.132)--(8.760,4.121)%
    --(8.762,4.110)--(8.765,4.099)--(8.768,4.088)--(8.772,4.077)--(8.776,4.067)%
    --(8.781,4.056)--(8.786,4.046)--(8.792,4.037)--(8.799,4.028)--(8.806,4.019)%
    --(8.813,4.010)--(8.821,4.002)--(8.829,3.994)--(8.838,3.987)--(8.847,3.980)%
    --(8.856,3.973)--(8.865,3.967)--(8.875,3.962)--(8.886,3.957)--(8.896,3.953)%
    --(8.907,3.949)--(8.918,3.946)--(8.929,3.943)--(8.940,3.941)--(8.951,3.940)%
    --(8.962,3.939)--(8.973,3.939)--(8.985,3.939)--(8.996,3.940)--(9.007,3.941)%
    --(9.018,3.943)--(9.029,3.946)--(9.040,3.949)--(9.051,3.953)--(9.061,3.957)%
    --(9.072,3.962)--(9.082,3.967)--(9.091,3.973)--(9.100,3.980)--(9.109,3.987)%
    --(9.118,3.994)--(9.126,4.002)--(9.134,4.010)--(9.141,4.019)--(9.148,4.028)%
    --(9.155,4.037)--(9.161,4.046)--(9.166,4.056)--(9.171,4.067)--(9.175,4.077)%
    --(9.179,4.088)--(9.182,4.099)--(9.185,4.110)--(9.187,4.121)--(9.188,4.132)%
    --(9.189,4.143)--(9.190,4.154)--cycle;
\gpfill{color=gp lt color border,opacity=0.50} (9.893,4.155)--(9.892,4.160)--(9.892,4.166)--(9.891,4.171)%
    --(9.890,4.177)--(9.889,4.182)--(9.887,4.188)--(9.885,4.193)--(9.883,4.198)%
    --(9.881,4.204)--(9.878,4.208)--(9.875,4.213)--(9.872,4.218)--(9.868,4.222)%
    --(9.865,4.227)--(9.861,4.231)--(9.857,4.235)--(9.852,4.238)--(9.848,4.242)%
    --(9.843,4.245)--(9.839,4.248)--(9.834,4.251)--(9.828,4.253)--(9.823,4.255)%
    --(9.818,4.257)--(9.812,4.259)--(9.807,4.260)--(9.801,4.261)--(9.796,4.262)%
    --(9.790,4.262)--(9.785,4.263)--(9.779,4.262)--(9.773,4.262)--(9.768,4.261)%
    --(9.762,4.260)--(9.757,4.259)--(9.751,4.257)--(9.746,4.255)--(9.741,4.253)%
    --(9.735,4.251)--(9.731,4.248)--(9.726,4.245)--(9.721,4.242)--(9.717,4.238)%
    --(9.712,4.235)--(9.708,4.231)--(9.704,4.227)--(9.701,4.222)--(9.697,4.218)%
    --(9.694,4.213)--(9.691,4.208)--(9.688,4.204)--(9.686,4.198)--(9.684,4.193)%
    --(9.682,4.188)--(9.680,4.182)--(9.679,4.177)--(9.678,4.171)--(9.677,4.166)%
    --(9.677,4.160)--(9.677,4.155)--(9.677,4.149)--(9.677,4.143)--(9.678,4.138)%
    --(9.679,4.132)--(9.680,4.127)--(9.682,4.121)--(9.684,4.116)--(9.686,4.111)%
    --(9.688,4.105)--(9.691,4.100)--(9.694,4.096)--(9.697,4.091)--(9.701,4.087)%
    --(9.704,4.082)--(9.708,4.078)--(9.712,4.074)--(9.717,4.071)--(9.721,4.067)%
    --(9.726,4.064)--(9.730,4.061)--(9.735,4.058)--(9.741,4.056)--(9.746,4.054)%
    --(9.751,4.052)--(9.757,4.050)--(9.762,4.049)--(9.768,4.048)--(9.773,4.047)%
    --(9.779,4.047)--(9.784,4.047)--(9.790,4.047)--(9.796,4.047)--(9.801,4.048)%
    --(9.807,4.049)--(9.812,4.050)--(9.818,4.052)--(9.823,4.054)--(9.828,4.056)%
    --(9.834,4.058)--(9.839,4.061)--(9.843,4.064)--(9.848,4.067)--(9.852,4.071)%
    --(9.857,4.074)--(9.861,4.078)--(9.865,4.082)--(9.868,4.087)--(9.872,4.091)%
    --(9.875,4.096)--(9.878,4.100)--(9.881,4.105)--(9.883,4.111)--(9.885,4.116)%
    --(9.887,4.121)--(9.889,4.127)--(9.890,4.132)--(9.891,4.138)--(9.892,4.143)%
    --(9.892,4.149)--(9.893,4.154)--cycle;
\gpfill{color=gp lt color border,opacity=0.50} (6.541,4.155)--(6.540,4.169)--(6.539,4.183)--(6.537,4.197)%
    --(6.535,4.211)--(6.531,4.224)--(6.527,4.238)--(6.523,4.251)--(6.517,4.264)%
    --(6.511,4.277)--(6.504,4.289)--(6.497,4.302)--(6.489,4.313)--(6.480,4.324)%
    --(6.471,4.335)--(6.461,4.345)--(6.451,4.355)--(6.440,4.364)--(6.429,4.373)%
    --(6.418,4.381)--(6.406,4.388)--(6.393,4.395)--(6.380,4.401)--(6.367,4.407)%
    --(6.354,4.411)--(6.340,4.415)--(6.327,4.419)--(6.313,4.421)--(6.299,4.423)%
    --(6.285,4.424)--(6.271,4.425)--(6.256,4.424)--(6.242,4.423)--(6.228,4.421)%
    --(6.214,4.419)--(6.201,4.415)--(6.187,4.411)--(6.174,4.407)--(6.161,4.401)%
    --(6.148,4.395)--(6.136,4.388)--(6.123,4.381)--(6.112,4.373)--(6.101,4.364)%
    --(6.090,4.355)--(6.080,4.345)--(6.070,4.335)--(6.061,4.324)--(6.052,4.313)%
    --(6.044,4.302)--(6.037,4.289)--(6.030,4.277)--(6.024,4.264)--(6.018,4.251)%
    --(6.014,4.238)--(6.010,4.224)--(6.006,4.211)--(6.004,4.197)--(6.002,4.183)%
    --(6.001,4.169)--(6.001,4.155)--(6.001,4.140)--(6.002,4.126)--(6.004,4.112)%
    --(6.006,4.098)--(6.010,4.085)--(6.014,4.071)--(6.018,4.058)--(6.024,4.045)%
    --(6.030,4.032)--(6.037,4.019)--(6.044,4.007)--(6.052,3.996)--(6.061,3.985)%
    --(6.070,3.974)--(6.080,3.964)--(6.090,3.954)--(6.101,3.945)--(6.112,3.936)%
    --(6.123,3.928)--(6.135,3.921)--(6.148,3.914)--(6.161,3.908)--(6.174,3.902)%
    --(6.187,3.898)--(6.201,3.894)--(6.214,3.890)--(6.228,3.888)--(6.242,3.886)%
    --(6.256,3.885)--(6.270,3.885)--(6.285,3.885)--(6.299,3.886)--(6.313,3.888)%
    --(6.327,3.890)--(6.340,3.894)--(6.354,3.898)--(6.367,3.902)--(6.380,3.908)%
    --(6.393,3.914)--(6.406,3.921)--(6.418,3.928)--(6.429,3.936)--(6.440,3.945)%
    --(6.451,3.954)--(6.461,3.964)--(6.471,3.974)--(6.480,3.985)--(6.489,3.996)%
    --(6.497,4.007)--(6.504,4.019)--(6.511,4.032)--(6.517,4.045)--(6.523,4.058)%
    --(6.527,4.071)--(6.531,4.085)--(6.535,4.098)--(6.537,4.112)--(6.539,4.126)%
    --(6.540,4.140)--(6.541,4.154)--cycle;
\gpfill{color=gp lt color border,opacity=0.50} (1.677,7.324)--(1.677,7.324)--(1.677,7.324)--(1.677,7.324)%
    --(1.677,7.324)--(1.677,7.324)--(1.677,7.324)--(1.677,7.324)--(1.677,7.324)%
    --(1.677,7.324)--(1.677,7.324)--(1.677,7.324)--(1.677,7.324)--(1.677,7.324)%
    --(1.677,7.324)--(1.677,7.324)--(1.677,7.324)--(1.677,7.324)--(1.677,7.324)%
    --(1.677,7.324)--(1.677,7.324)--(1.677,7.324)--(1.677,7.324)--(1.677,7.324)%
    --(1.677,7.324)--(1.677,7.324)--(1.677,7.324)--(1.677,7.324)--(1.677,7.324)%
    --(1.677,7.324)--(1.677,7.324)--(1.677,7.324)--(1.677,7.324)--(1.677,7.324)%
    --(1.677,7.324)--(1.677,7.324)--(1.677,7.324)--(1.677,7.324)--(1.677,7.324)%
    --(1.677,7.324)--(1.677,7.324)--(1.677,7.324)--(1.677,7.324)--(1.677,7.324)%
    --(1.677,7.324)--(1.677,7.324)--(1.677,7.324)--(1.677,7.324)--(1.677,7.324)%
    --(1.677,7.324)--(1.677,7.324)--(1.677,7.324)--(1.677,7.324)--(1.677,7.324)%
    --(1.677,7.324)--(1.677,7.324)--(1.677,7.324)--(1.677,7.324)--(1.677,7.324)%
    --(1.677,7.324)--(1.677,7.324)--(1.677,7.324)--(1.677,7.324)--(1.677,7.324)%
    --(1.677,7.324)--(1.677,7.324)--(1.677,7.324)--(1.677,7.324)--(1.677,7.324)%
    --(1.677,7.324)--(1.677,7.324)--(1.677,7.324)--(1.677,7.324)--(1.677,7.324)%
    --(1.677,7.324)--(1.677,7.324)--(1.677,7.324)--(1.677,7.324)--(1.677,7.324)%
    --(1.677,7.324)--(1.677,7.324)--(1.677,7.324)--(1.677,7.324)--(1.677,7.324)%
    --(1.677,7.324)--(1.677,7.324)--(1.677,7.324)--(1.677,7.324)--(1.677,7.324)%
    --(1.677,7.324)--(1.677,7.324)--(1.677,7.324)--(1.677,7.324)--(1.677,7.324)%
    --(1.677,7.324)--(1.677,7.324)--(1.677,7.324)--(1.677,7.324)--(1.677,7.324)%
    --(1.677,7.324)--(1.677,7.324)--(1.677,7.324)--(1.677,7.324)--(1.677,7.324)%
    --(1.677,7.324)--(1.677,7.324)--(1.677,7.324)--(1.677,7.324)--(1.677,7.324)%
    --(1.677,7.324)--(1.677,7.324)--(1.677,7.324)--(1.677,7.324)--(1.677,7.324)%
    --(1.677,7.324)--(1.677,7.324)--(1.677,7.324)--(1.677,7.324)--(1.677,7.324)%
    --(1.677,7.324)--cycle;
\gpfill{color=gp lt color border,opacity=0.50} (9.001,3.098)--(9.000,3.099)--(9.000,3.100)--(9.000,3.102)%
    --(9.000,3.103)--(9.000,3.104)--(8.999,3.106)--(8.999,3.107)--(8.998,3.108)%
    --(8.998,3.110)--(8.997,3.111)--(8.996,3.112)--(8.995,3.113)--(8.994,3.114)%
    --(8.994,3.116)--(8.993,3.117)--(8.992,3.118)--(8.990,3.118)--(8.989,3.119)%
    --(8.988,3.120)--(8.987,3.121)--(8.986,3.122)--(8.984,3.122)--(8.983,3.123)%
    --(8.982,3.123)--(8.980,3.124)--(8.979,3.124)--(8.978,3.124)--(8.976,3.124)%
    --(8.975,3.124)--(8.974,3.125)--(8.972,3.124)--(8.971,3.124)--(8.969,3.124)%
    --(8.968,3.124)--(8.967,3.124)--(8.965,3.123)--(8.964,3.123)--(8.963,3.122)%
    --(8.961,3.122)--(8.960,3.121)--(8.959,3.120)--(8.958,3.119)--(8.957,3.118)%
    --(8.955,3.118)--(8.954,3.117)--(8.953,3.116)--(8.953,3.114)--(8.952,3.113)%
    --(8.951,3.112)--(8.950,3.111)--(8.949,3.110)--(8.949,3.108)--(8.948,3.107)%
    --(8.948,3.106)--(8.947,3.104)--(8.947,3.103)--(8.947,3.102)--(8.947,3.100)%
    --(8.947,3.099)--(8.947,3.098)--(8.947,3.096)--(8.947,3.095)--(8.947,3.093)%
    --(8.947,3.092)--(8.947,3.091)--(8.948,3.089)--(8.948,3.088)--(8.949,3.087)%
    --(8.949,3.085)--(8.950,3.084)--(8.951,3.083)--(8.952,3.082)--(8.953,3.081)%
    --(8.953,3.079)--(8.954,3.078)--(8.955,3.077)--(8.957,3.077)--(8.958,3.076)%
    --(8.959,3.075)--(8.960,3.074)--(8.961,3.073)--(8.963,3.073)--(8.964,3.072)%
    --(8.965,3.072)--(8.967,3.071)--(8.968,3.071)--(8.969,3.071)--(8.971,3.071)%
    --(8.972,3.071)--(8.973,3.071)--(8.975,3.071)--(8.976,3.071)--(8.978,3.071)%
    --(8.979,3.071)--(8.980,3.071)--(8.982,3.072)--(8.983,3.072)--(8.984,3.073)%
    --(8.986,3.073)--(8.987,3.074)--(8.988,3.075)--(8.989,3.076)--(8.990,3.077)%
    --(8.992,3.077)--(8.993,3.078)--(8.994,3.079)--(8.994,3.081)--(8.995,3.082)%
    --(8.996,3.083)--(8.997,3.084)--(8.998,3.085)--(8.998,3.087)--(8.999,3.088)%
    --(8.999,3.089)--(9.000,3.091)--(9.000,3.092)--(9.000,3.093)--(9.000,3.095)%
    --(9.000,3.096)--(9.001,3.097)--cycle;
\gpfill{color=gp lt color border,opacity=0.50} (11.677,6.268)--(11.677,6.268)--(11.677,6.268)--(11.677,6.268)%
    --(11.677,6.268)--(11.677,6.268)--(11.677,6.268)--(11.677,6.268)--(11.677,6.268)%
    --(11.677,6.268)--(11.677,6.268)--(11.677,6.268)--(11.677,6.268)--(11.677,6.268)%
    --(11.677,6.268)--(11.677,6.268)--(11.677,6.268)--(11.677,6.268)--(11.677,6.268)%
    --(11.677,6.268)--(11.677,6.268)--(11.677,6.268)--(11.677,6.268)--(11.677,6.268)%
    --(11.677,6.268)--(11.677,6.268)--(11.677,6.268)--(11.677,6.268)--(11.677,6.268)%
    --(11.677,6.268)--(11.677,6.268)--(11.677,6.268)--(11.677,6.268)--(11.677,6.268)%
    --(11.677,6.268)--(11.677,6.268)--(11.677,6.268)--(11.677,6.268)--(11.677,6.268)%
    --(11.677,6.268)--(11.677,6.268)--(11.677,6.268)--(11.677,6.268)--(11.677,6.268)%
    --(11.677,6.268)--(11.677,6.268)--(11.677,6.268)--(11.677,6.268)--(11.677,6.268)%
    --(11.677,6.268)--(11.677,6.268)--(11.677,6.268)--(11.677,6.268)--(11.677,6.268)%
    --(11.677,6.268)--(11.677,6.268)--(11.677,6.268)--(11.677,6.268)--(11.677,6.268)%
    --(11.677,6.268)--(11.677,6.268)--(11.677,6.268)--(11.677,6.268)--(11.677,6.268)%
    --(11.677,6.268)--(11.677,6.268)--(11.677,6.268)--(11.677,6.268)--(11.677,6.268)%
    --(11.677,6.268)--(11.677,6.268)--(11.677,6.268)--(11.677,6.268)--(11.677,6.268)%
    --(11.677,6.268)--(11.677,6.268)--(11.677,6.268)--(11.677,6.268)--(11.677,6.268)%
    --(11.677,6.268)--(11.677,6.268)--(11.677,6.268)--(11.677,6.268)--(11.677,6.268)%
    --(11.677,6.268)--(11.677,6.268)--(11.677,6.268)--(11.677,6.268)--(11.677,6.268)%
    --(11.677,6.268)--(11.677,6.268)--(11.677,6.268)--(11.677,6.268)--(11.677,6.268)%
    --(11.677,6.268)--(11.677,6.268)--(11.677,6.268)--(11.677,6.268)--(11.677,6.268)%
    --(11.677,6.268)--(11.677,6.268)--(11.677,6.268)--(11.677,6.268)--(11.677,6.268)%
    --(11.677,6.268)--(11.677,6.268)--(11.677,6.268)--(11.677,6.268)--(11.677,6.268)%
    --(11.677,6.268)--(11.677,6.268)--(11.677,6.268)--(11.677,6.268)--(11.677,6.268)%
    --(11.677,6.268)--(11.677,6.268)--(11.677,6.268)--(11.677,6.268)--(11.677,6.268)%
    --(11.677,6.268)--cycle;
\gpfill{color=gp lt color border,opacity=0.50} (3.919,4.155)--(3.918,4.173)--(3.917,4.191)--(3.914,4.209)%
    --(3.911,4.227)--(3.907,4.245)--(3.901,4.263)--(3.895,4.280)--(3.888,4.297)%
    --(3.880,4.314)--(3.871,4.330)--(3.862,4.346)--(3.851,4.361)--(3.840,4.375)%
    --(3.828,4.389)--(3.816,4.403)--(3.802,4.415)--(3.788,4.427)--(3.774,4.438)%
    --(3.759,4.449)--(3.743,4.458)--(3.727,4.467)--(3.710,4.475)--(3.693,4.482)%
    --(3.676,4.488)--(3.658,4.494)--(3.640,4.498)--(3.622,4.501)--(3.604,4.504)%
    --(3.586,4.505)--(3.568,4.506)--(3.549,4.505)--(3.531,4.504)--(3.513,4.501)%
    --(3.495,4.498)--(3.477,4.494)--(3.459,4.488)--(3.442,4.482)--(3.425,4.475)%
    --(3.408,4.467)--(3.392,4.458)--(3.376,4.449)--(3.361,4.438)--(3.347,4.427)%
    --(3.333,4.415)--(3.319,4.403)--(3.307,4.389)--(3.295,4.375)--(3.284,4.361)%
    --(3.273,4.346)--(3.264,4.330)--(3.255,4.314)--(3.247,4.297)--(3.240,4.280)%
    --(3.234,4.263)--(3.228,4.245)--(3.224,4.227)--(3.221,4.209)--(3.218,4.191)%
    --(3.217,4.173)--(3.217,4.155)--(3.217,4.136)--(3.218,4.118)--(3.221,4.100)%
    --(3.224,4.082)--(3.228,4.064)--(3.234,4.046)--(3.240,4.029)--(3.247,4.012)%
    --(3.255,3.995)--(3.264,3.979)--(3.273,3.963)--(3.284,3.948)--(3.295,3.934)%
    --(3.307,3.920)--(3.319,3.906)--(3.333,3.894)--(3.347,3.882)--(3.361,3.871)%
    --(3.376,3.860)--(3.392,3.851)--(3.408,3.842)--(3.425,3.834)--(3.442,3.827)%
    --(3.459,3.821)--(3.477,3.815)--(3.495,3.811)--(3.513,3.808)--(3.531,3.805)%
    --(3.549,3.804)--(3.567,3.804)--(3.586,3.804)--(3.604,3.805)--(3.622,3.808)%
    --(3.640,3.811)--(3.658,3.815)--(3.676,3.821)--(3.693,3.827)--(3.710,3.834)%
    --(3.727,3.842)--(3.743,3.851)--(3.759,3.860)--(3.774,3.871)--(3.788,3.882)%
    --(3.802,3.894)--(3.816,3.906)--(3.828,3.920)--(3.840,3.934)--(3.851,3.948)%
    --(3.862,3.963)--(3.871,3.979)--(3.880,3.995)--(3.888,4.012)--(3.895,4.029)%
    --(3.901,4.046)--(3.907,4.064)--(3.911,4.082)--(3.914,4.100)--(3.917,4.118)%
    --(3.918,4.136)--(3.919,4.154)--cycle;
\gpfill{color=gp lt color border,opacity=0.50} (6.406,3.098)--(6.405,3.105)--(6.405,3.112)--(6.404,3.119)%
    --(6.403,3.126)--(6.401,3.132)--(6.399,3.139)--(6.397,3.146)--(6.394,3.152)%
    --(6.391,3.159)--(6.387,3.165)--(6.384,3.171)--(6.380,3.177)--(6.375,3.182)%
    --(6.371,3.188)--(6.366,3.193)--(6.361,3.198)--(6.355,3.202)--(6.350,3.207)%
    --(6.344,3.211)--(6.338,3.214)--(6.332,3.218)--(6.325,3.221)--(6.319,3.224)%
    --(6.312,3.226)--(6.305,3.228)--(6.299,3.230)--(6.292,3.231)--(6.285,3.232)%
    --(6.278,3.232)--(6.271,3.233)--(6.263,3.232)--(6.256,3.232)--(6.249,3.231)%
    --(6.242,3.230)--(6.236,3.228)--(6.229,3.226)--(6.222,3.224)--(6.216,3.221)%
    --(6.209,3.218)--(6.203,3.214)--(6.197,3.211)--(6.191,3.207)--(6.186,3.202)%
    --(6.180,3.198)--(6.175,3.193)--(6.170,3.188)--(6.166,3.182)--(6.161,3.177)%
    --(6.157,3.171)--(6.154,3.165)--(6.150,3.159)--(6.147,3.152)--(6.144,3.146)%
    --(6.142,3.139)--(6.140,3.132)--(6.138,3.126)--(6.137,3.119)--(6.136,3.112)%
    --(6.136,3.105)--(6.136,3.098)--(6.136,3.090)--(6.136,3.083)--(6.137,3.076)%
    --(6.138,3.069)--(6.140,3.063)--(6.142,3.056)--(6.144,3.049)--(6.147,3.043)%
    --(6.150,3.036)--(6.154,3.030)--(6.157,3.024)--(6.161,3.018)--(6.166,3.013)%
    --(6.170,3.007)--(6.175,3.002)--(6.180,2.997)--(6.186,2.993)--(6.191,2.988)%
    --(6.197,2.984)--(6.203,2.981)--(6.209,2.977)--(6.216,2.974)--(6.222,2.971)%
    --(6.229,2.969)--(6.236,2.967)--(6.242,2.965)--(6.249,2.964)--(6.256,2.963)%
    --(6.263,2.963)--(6.270,2.963)--(6.278,2.963)--(6.285,2.963)--(6.292,2.964)%
    --(6.299,2.965)--(6.305,2.967)--(6.312,2.969)--(6.319,2.971)--(6.325,2.974)%
    --(6.332,2.977)--(6.338,2.981)--(6.344,2.984)--(6.350,2.988)--(6.355,2.993)%
    --(6.361,2.997)--(6.366,3.002)--(6.371,3.007)--(6.375,3.013)--(6.380,3.018)%
    --(6.384,3.024)--(6.387,3.030)--(6.391,3.036)--(6.394,3.043)--(6.397,3.049)%
    --(6.399,3.056)--(6.401,3.063)--(6.403,3.069)--(6.404,3.076)--(6.405,3.083)%
    --(6.405,3.090)--(6.406,3.097)--cycle;
\gpfill{color=gp lt color border,opacity=0.50} (9.082,6.268)--(9.081,6.273)--(9.081,6.279)--(9.080,6.284)%
    --(9.079,6.290)--(9.078,6.295)--(9.076,6.301)--(9.074,6.306)--(9.072,6.311)%
    --(9.070,6.317)--(9.067,6.321)--(9.064,6.326)--(9.061,6.331)--(9.057,6.335)%
    --(9.054,6.340)--(9.050,6.344)--(9.046,6.348)--(9.041,6.351)--(9.037,6.355)%
    --(9.032,6.358)--(9.028,6.361)--(9.023,6.364)--(9.017,6.366)--(9.012,6.368)%
    --(9.007,6.370)--(9.001,6.372)--(8.996,6.373)--(8.990,6.374)--(8.985,6.375)%
    --(8.979,6.375)--(8.974,6.376)--(8.968,6.375)--(8.962,6.375)--(8.957,6.374)%
    --(8.951,6.373)--(8.946,6.372)--(8.940,6.370)--(8.935,6.368)--(8.930,6.366)%
    --(8.924,6.364)--(8.920,6.361)--(8.915,6.358)--(8.910,6.355)--(8.906,6.351)%
    --(8.901,6.348)--(8.897,6.344)--(8.893,6.340)--(8.890,6.335)--(8.886,6.331)%
    --(8.883,6.326)--(8.880,6.321)--(8.877,6.317)--(8.875,6.311)--(8.873,6.306)%
    --(8.871,6.301)--(8.869,6.295)--(8.868,6.290)--(8.867,6.284)--(8.866,6.279)%
    --(8.866,6.273)--(8.866,6.268)--(8.866,6.262)--(8.866,6.256)--(8.867,6.251)%
    --(8.868,6.245)--(8.869,6.240)--(8.871,6.234)--(8.873,6.229)--(8.875,6.224)%
    --(8.877,6.218)--(8.880,6.213)--(8.883,6.209)--(8.886,6.204)--(8.890,6.200)%
    --(8.893,6.195)--(8.897,6.191)--(8.901,6.187)--(8.906,6.184)--(8.910,6.180)%
    --(8.915,6.177)--(8.919,6.174)--(8.924,6.171)--(8.930,6.169)--(8.935,6.167)%
    --(8.940,6.165)--(8.946,6.163)--(8.951,6.162)--(8.957,6.161)--(8.962,6.160)%
    --(8.968,6.160)--(8.973,6.160)--(8.979,6.160)--(8.985,6.160)--(8.990,6.161)%
    --(8.996,6.162)--(9.001,6.163)--(9.007,6.165)--(9.012,6.167)--(9.017,6.169)%
    --(9.023,6.171)--(9.028,6.174)--(9.032,6.177)--(9.037,6.180)--(9.041,6.184)%
    --(9.046,6.187)--(9.050,6.191)--(9.054,6.195)--(9.057,6.200)--(9.061,6.204)%
    --(9.064,6.209)--(9.067,6.213)--(9.070,6.218)--(9.072,6.224)--(9.074,6.229)%
    --(9.076,6.234)--(9.078,6.240)--(9.079,6.245)--(9.080,6.251)--(9.081,6.256)%
    --(9.081,6.262)--(9.082,6.267)--cycle;
\gpfill{color=gp lt color border,opacity=0.50} (11.704,5.211)--(11.703,5.212)--(11.703,5.213)--(11.703,5.215)%
    --(11.703,5.216)--(11.703,5.217)--(11.702,5.219)--(11.702,5.220)--(11.701,5.221)%
    --(11.701,5.223)--(11.700,5.224)--(11.699,5.225)--(11.698,5.226)--(11.697,5.227)%
    --(11.697,5.229)--(11.696,5.230)--(11.695,5.231)--(11.693,5.231)--(11.692,5.232)%
    --(11.691,5.233)--(11.690,5.234)--(11.689,5.235)--(11.687,5.235)--(11.686,5.236)%
    --(11.685,5.236)--(11.683,5.237)--(11.682,5.237)--(11.681,5.237)--(11.679,5.237)%
    --(11.678,5.237)--(11.677,5.238)--(11.675,5.237)--(11.674,5.237)--(11.672,5.237)%
    --(11.671,5.237)--(11.670,5.237)--(11.668,5.236)--(11.667,5.236)--(11.666,5.235)%
    --(11.664,5.235)--(11.663,5.234)--(11.662,5.233)--(11.661,5.232)--(11.660,5.231)%
    --(11.658,5.231)--(11.657,5.230)--(11.656,5.229)--(11.656,5.227)--(11.655,5.226)%
    --(11.654,5.225)--(11.653,5.224)--(11.652,5.223)--(11.652,5.221)--(11.651,5.220)%
    --(11.651,5.219)--(11.650,5.217)--(11.650,5.216)--(11.650,5.215)--(11.650,5.213)%
    --(11.650,5.212)--(11.650,5.211)--(11.650,5.209)--(11.650,5.208)--(11.650,5.206)%
    --(11.650,5.205)--(11.650,5.204)--(11.651,5.202)--(11.651,5.201)--(11.652,5.200)%
    --(11.652,5.198)--(11.653,5.197)--(11.654,5.196)--(11.655,5.195)--(11.656,5.194)%
    --(11.656,5.192)--(11.657,5.191)--(11.658,5.190)--(11.660,5.190)--(11.661,5.189)%
    --(11.662,5.188)--(11.663,5.187)--(11.664,5.186)--(11.666,5.186)--(11.667,5.185)%
    --(11.668,5.185)--(11.670,5.184)--(11.671,5.184)--(11.672,5.184)--(11.674,5.184)%
    --(11.675,5.184)--(11.676,5.184)--(11.678,5.184)--(11.679,5.184)--(11.681,5.184)%
    --(11.682,5.184)--(11.683,5.184)--(11.685,5.185)--(11.686,5.185)--(11.687,5.186)%
    --(11.689,5.186)--(11.690,5.187)--(11.691,5.188)--(11.692,5.189)--(11.693,5.190)%
    --(11.695,5.190)--(11.696,5.191)--(11.697,5.192)--(11.697,5.194)--(11.698,5.195)%
    --(11.699,5.196)--(11.700,5.197)--(11.701,5.198)--(11.701,5.200)--(11.702,5.201)%
    --(11.702,5.202)--(11.703,5.204)--(11.703,5.205)--(11.703,5.206)--(11.703,5.208)%
    --(11.703,5.209)--(11.704,5.210)--cycle;
\gpfill{color=gp lt color border,opacity=0.50} (7.352,4.155)--(7.351,4.169)--(7.350,4.183)--(7.348,4.197)%
    --(7.346,4.211)--(7.342,4.224)--(7.338,4.238)--(7.334,4.251)--(7.328,4.264)%
    --(7.322,4.277)--(7.315,4.289)--(7.308,4.302)--(7.300,4.313)--(7.291,4.324)%
    --(7.282,4.335)--(7.272,4.345)--(7.262,4.355)--(7.251,4.364)--(7.240,4.373)%
    --(7.229,4.381)--(7.217,4.388)--(7.204,4.395)--(7.191,4.401)--(7.178,4.407)%
    --(7.165,4.411)--(7.151,4.415)--(7.138,4.419)--(7.124,4.421)--(7.110,4.423)%
    --(7.096,4.424)--(7.082,4.425)--(7.067,4.424)--(7.053,4.423)--(7.039,4.421)%
    --(7.025,4.419)--(7.012,4.415)--(6.998,4.411)--(6.985,4.407)--(6.972,4.401)%
    --(6.959,4.395)--(6.947,4.388)--(6.934,4.381)--(6.923,4.373)--(6.912,4.364)%
    --(6.901,4.355)--(6.891,4.345)--(6.881,4.335)--(6.872,4.324)--(6.863,4.313)%
    --(6.855,4.302)--(6.848,4.289)--(6.841,4.277)--(6.835,4.264)--(6.829,4.251)%
    --(6.825,4.238)--(6.821,4.224)--(6.817,4.211)--(6.815,4.197)--(6.813,4.183)%
    --(6.812,4.169)--(6.812,4.155)--(6.812,4.140)--(6.813,4.126)--(6.815,4.112)%
    --(6.817,4.098)--(6.821,4.085)--(6.825,4.071)--(6.829,4.058)--(6.835,4.045)%
    --(6.841,4.032)--(6.848,4.019)--(6.855,4.007)--(6.863,3.996)--(6.872,3.985)%
    --(6.881,3.974)--(6.891,3.964)--(6.901,3.954)--(6.912,3.945)--(6.923,3.936)%
    --(6.934,3.928)--(6.946,3.921)--(6.959,3.914)--(6.972,3.908)--(6.985,3.902)%
    --(6.998,3.898)--(7.012,3.894)--(7.025,3.890)--(7.039,3.888)--(7.053,3.886)%
    --(7.067,3.885)--(7.081,3.885)--(7.096,3.885)--(7.110,3.886)--(7.124,3.888)%
    --(7.138,3.890)--(7.151,3.894)--(7.165,3.898)--(7.178,3.902)--(7.191,3.908)%
    --(7.204,3.914)--(7.217,3.921)--(7.229,3.928)--(7.240,3.936)--(7.251,3.945)%
    --(7.262,3.954)--(7.272,3.964)--(7.282,3.974)--(7.291,3.985)--(7.300,3.996)%
    --(7.308,4.007)--(7.315,4.019)--(7.322,4.032)--(7.328,4.045)--(7.334,4.058)%
    --(7.338,4.071)--(7.342,4.085)--(7.346,4.098)--(7.348,4.112)--(7.350,4.126)%
    --(7.351,4.140)--(7.352,4.154)--cycle;
\gpfill{color=gp lt color border,opacity=0.50} (9.812,3.098)--(9.811,3.099)--(9.811,3.100)--(9.811,3.102)%
    --(9.811,3.103)--(9.811,3.104)--(9.810,3.106)--(9.810,3.107)--(9.809,3.108)%
    --(9.809,3.110)--(9.808,3.111)--(9.807,3.112)--(9.806,3.113)--(9.805,3.114)%
    --(9.805,3.116)--(9.804,3.117)--(9.803,3.118)--(9.801,3.118)--(9.800,3.119)%
    --(9.799,3.120)--(9.798,3.121)--(9.797,3.122)--(9.795,3.122)--(9.794,3.123)%
    --(9.793,3.123)--(9.791,3.124)--(9.790,3.124)--(9.789,3.124)--(9.787,3.124)%
    --(9.786,3.124)--(9.785,3.125)--(9.783,3.124)--(9.782,3.124)--(9.780,3.124)%
    --(9.779,3.124)--(9.778,3.124)--(9.776,3.123)--(9.775,3.123)--(9.774,3.122)%
    --(9.772,3.122)--(9.771,3.121)--(9.770,3.120)--(9.769,3.119)--(9.768,3.118)%
    --(9.766,3.118)--(9.765,3.117)--(9.764,3.116)--(9.764,3.114)--(9.763,3.113)%
    --(9.762,3.112)--(9.761,3.111)--(9.760,3.110)--(9.760,3.108)--(9.759,3.107)%
    --(9.759,3.106)--(9.758,3.104)--(9.758,3.103)--(9.758,3.102)--(9.758,3.100)%
    --(9.758,3.099)--(9.758,3.098)--(9.758,3.096)--(9.758,3.095)--(9.758,3.093)%
    --(9.758,3.092)--(9.758,3.091)--(9.759,3.089)--(9.759,3.088)--(9.760,3.087)%
    --(9.760,3.085)--(9.761,3.084)--(9.762,3.083)--(9.763,3.082)--(9.764,3.081)%
    --(9.764,3.079)--(9.765,3.078)--(9.766,3.077)--(9.768,3.077)--(9.769,3.076)%
    --(9.770,3.075)--(9.771,3.074)--(9.772,3.073)--(9.774,3.073)--(9.775,3.072)%
    --(9.776,3.072)--(9.778,3.071)--(9.779,3.071)--(9.780,3.071)--(9.782,3.071)%
    --(9.783,3.071)--(9.784,3.071)--(9.786,3.071)--(9.787,3.071)--(9.789,3.071)%
    --(9.790,3.071)--(9.791,3.071)--(9.793,3.072)--(9.794,3.072)--(9.795,3.073)%
    --(9.797,3.073)--(9.798,3.074)--(9.799,3.075)--(9.800,3.076)--(9.801,3.077)%
    --(9.803,3.077)--(9.804,3.078)--(9.805,3.079)--(9.805,3.081)--(9.806,3.082)%
    --(9.807,3.083)--(9.808,3.084)--(9.809,3.085)--(9.809,3.087)--(9.810,3.088)%
    --(9.810,3.089)--(9.811,3.091)--(9.811,3.092)--(9.811,3.093)--(9.811,3.095)%
    --(9.811,3.096)--(9.812,3.097)--cycle;
\gpfill{color=gp lt color border,opacity=0.50} (7.704,6.268)--(7.703,6.272)--(7.703,6.276)--(7.703,6.280)%
    --(7.702,6.284)--(7.701,6.288)--(7.700,6.293)--(7.698,6.297)--(7.696,6.300)%
    --(7.695,6.304)--(7.693,6.308)--(7.690,6.312)--(7.688,6.315)--(7.685,6.318)%
    --(7.683,6.322)--(7.680,6.325)--(7.677,6.328)--(7.673,6.330)--(7.670,6.333)%
    --(7.667,6.335)--(7.663,6.338)--(7.659,6.340)--(7.655,6.341)--(7.652,6.343)%
    --(7.648,6.345)--(7.643,6.346)--(7.639,6.347)--(7.635,6.348)--(7.631,6.348)%
    --(7.627,6.348)--(7.623,6.349)--(7.618,6.348)--(7.614,6.348)--(7.610,6.348)%
    --(7.606,6.347)--(7.602,6.346)--(7.597,6.345)--(7.593,6.343)--(7.590,6.341)%
    --(7.586,6.340)--(7.582,6.338)--(7.578,6.335)--(7.575,6.333)--(7.572,6.330)%
    --(7.568,6.328)--(7.565,6.325)--(7.562,6.322)--(7.560,6.318)--(7.557,6.315)%
    --(7.555,6.312)--(7.552,6.308)--(7.550,6.304)--(7.549,6.300)--(7.547,6.297)%
    --(7.545,6.293)--(7.544,6.288)--(7.543,6.284)--(7.542,6.280)--(7.542,6.276)%
    --(7.542,6.272)--(7.542,6.268)--(7.542,6.263)--(7.542,6.259)--(7.542,6.255)%
    --(7.543,6.251)--(7.544,6.247)--(7.545,6.242)--(7.547,6.238)--(7.549,6.235)%
    --(7.550,6.231)--(7.552,6.227)--(7.555,6.223)--(7.557,6.220)--(7.560,6.217)%
    --(7.562,6.213)--(7.565,6.210)--(7.568,6.207)--(7.572,6.205)--(7.575,6.202)%
    --(7.578,6.200)--(7.582,6.197)--(7.586,6.195)--(7.590,6.194)--(7.593,6.192)%
    --(7.597,6.190)--(7.602,6.189)--(7.606,6.188)--(7.610,6.187)--(7.614,6.187)%
    --(7.618,6.187)--(7.622,6.187)--(7.627,6.187)--(7.631,6.187)--(7.635,6.187)%
    --(7.639,6.188)--(7.643,6.189)--(7.648,6.190)--(7.652,6.192)--(7.655,6.194)%
    --(7.659,6.195)--(7.663,6.197)--(7.667,6.200)--(7.670,6.202)--(7.673,6.205)%
    --(7.677,6.207)--(7.680,6.210)--(7.683,6.213)--(7.685,6.217)--(7.688,6.220)%
    --(7.690,6.223)--(7.693,6.227)--(7.695,6.231)--(7.696,6.235)--(7.698,6.238)%
    --(7.700,6.242)--(7.701,6.247)--(7.702,6.251)--(7.703,6.255)--(7.703,6.259)%
    --(7.703,6.263)--(7.704,6.267)--cycle;
\gpfill{color=gp lt color border,opacity=0.50} (10.325,5.211)--(10.325,5.211)--(10.325,5.211)--(10.325,5.211)%
    --(10.325,5.211)--(10.325,5.211)--(10.325,5.211)--(10.325,5.211)--(10.325,5.211)%
    --(10.325,5.211)--(10.325,5.211)--(10.325,5.211)--(10.325,5.211)--(10.325,5.211)%
    --(10.325,5.211)--(10.325,5.211)--(10.325,5.211)--(10.325,5.211)--(10.325,5.211)%
    --(10.325,5.211)--(10.325,5.211)--(10.325,5.211)--(10.325,5.211)--(10.325,5.211)%
    --(10.325,5.211)--(10.325,5.211)--(10.325,5.211)--(10.325,5.211)--(10.325,5.211)%
    --(10.325,5.211)--(10.325,5.211)--(10.325,5.211)--(10.325,5.211)--(10.325,5.211)%
    --(10.325,5.211)--(10.325,5.211)--(10.325,5.211)--(10.325,5.211)--(10.325,5.211)%
    --(10.325,5.211)--(10.325,5.211)--(10.325,5.211)--(10.325,5.211)--(10.325,5.211)%
    --(10.325,5.211)--(10.325,5.211)--(10.325,5.211)--(10.325,5.211)--(10.325,5.211)%
    --(10.325,5.211)--(10.325,5.211)--(10.325,5.211)--(10.325,5.211)--(10.325,5.211)%
    --(10.325,5.211)--(10.325,5.211)--(10.325,5.211)--(10.325,5.211)--(10.325,5.211)%
    --(10.325,5.211)--(10.325,5.211)--(10.325,5.211)--(10.325,5.211)--(10.325,5.211)%
    --(10.325,5.211)--(10.325,5.211)--(10.325,5.211)--(10.325,5.211)--(10.325,5.211)%
    --(10.325,5.211)--(10.325,5.211)--(10.325,5.211)--(10.325,5.211)--(10.325,5.211)%
    --(10.325,5.211)--(10.325,5.211)--(10.325,5.211)--(10.325,5.211)--(10.325,5.211)%
    --(10.325,5.211)--(10.325,5.211)--(10.325,5.211)--(10.325,5.211)--(10.325,5.211)%
    --(10.325,5.211)--(10.325,5.211)--(10.325,5.211)--(10.325,5.211)--(10.325,5.211)%
    --(10.325,5.211)--(10.325,5.211)--(10.325,5.211)--(10.325,5.211)--(10.325,5.211)%
    --(10.325,5.211)--(10.325,5.211)--(10.325,5.211)--(10.325,5.211)--(10.325,5.211)%
    --(10.325,5.211)--(10.325,5.211)--(10.325,5.211)--(10.325,5.211)--(10.325,5.211)%
    --(10.325,5.211)--(10.325,5.211)--(10.325,5.211)--(10.325,5.211)--(10.325,5.211)%
    --(10.325,5.211)--(10.325,5.211)--(10.325,5.211)--(10.325,5.211)--(10.325,5.211)%
    --(10.325,5.211)--(10.325,5.211)--(10.325,5.211)--(10.325,5.211)--(10.325,5.211)%
    --(10.325,5.211)--cycle;
\gpfill{color=gp lt color border,opacity=0.50} (3.784,3.098)--(3.783,3.109)--(3.782,3.120)--(3.781,3.131)%
    --(3.779,3.142)--(3.776,3.153)--(3.773,3.164)--(3.769,3.175)--(3.765,3.185)%
    --(3.760,3.196)--(3.755,3.205)--(3.749,3.215)--(3.742,3.224)--(3.735,3.233)%
    --(3.728,3.242)--(3.720,3.250)--(3.712,3.258)--(3.703,3.265)--(3.694,3.272)%
    --(3.685,3.279)--(3.676,3.285)--(3.666,3.290)--(3.655,3.295)--(3.645,3.299)%
    --(3.634,3.303)--(3.623,3.306)--(3.612,3.309)--(3.601,3.311)--(3.590,3.312)%
    --(3.579,3.313)--(3.568,3.314)--(3.556,3.313)--(3.545,3.312)--(3.534,3.311)%
    --(3.523,3.309)--(3.512,3.306)--(3.501,3.303)--(3.490,3.299)--(3.480,3.295)%
    --(3.469,3.290)--(3.460,3.285)--(3.450,3.279)--(3.441,3.272)--(3.432,3.265)%
    --(3.423,3.258)--(3.415,3.250)--(3.407,3.242)--(3.400,3.233)--(3.393,3.224)%
    --(3.386,3.215)--(3.380,3.205)--(3.375,3.196)--(3.370,3.185)--(3.366,3.175)%
    --(3.362,3.164)--(3.359,3.153)--(3.356,3.142)--(3.354,3.131)--(3.353,3.120)%
    --(3.352,3.109)--(3.352,3.098)--(3.352,3.086)--(3.353,3.075)--(3.354,3.064)%
    --(3.356,3.053)--(3.359,3.042)--(3.362,3.031)--(3.366,3.020)--(3.370,3.010)%
    --(3.375,2.999)--(3.380,2.989)--(3.386,2.980)--(3.393,2.971)--(3.400,2.962)%
    --(3.407,2.953)--(3.415,2.945)--(3.423,2.937)--(3.432,2.930)--(3.441,2.923)%
    --(3.450,2.916)--(3.459,2.910)--(3.469,2.905)--(3.480,2.900)--(3.490,2.896)%
    --(3.501,2.892)--(3.512,2.889)--(3.523,2.886)--(3.534,2.884)--(3.545,2.883)%
    --(3.556,2.882)--(3.567,2.882)--(3.579,2.882)--(3.590,2.883)--(3.601,2.884)%
    --(3.612,2.886)--(3.623,2.889)--(3.634,2.892)--(3.645,2.896)--(3.655,2.900)%
    --(3.666,2.905)--(3.676,2.910)--(3.685,2.916)--(3.694,2.923)--(3.703,2.930)%
    --(3.712,2.937)--(3.720,2.945)--(3.728,2.953)--(3.735,2.962)--(3.742,2.971)%
    --(3.749,2.980)--(3.755,2.989)--(3.760,2.999)--(3.765,3.010)--(3.769,3.020)%
    --(3.773,3.031)--(3.776,3.042)--(3.779,3.053)--(3.781,3.064)--(3.782,3.075)%
    --(3.783,3.086)--(3.784,3.097)--cycle;
\gpfill{color=gp lt color border,opacity=0.50} (6.460,6.268)--(6.459,6.277)--(6.458,6.287)--(6.457,6.297)%
    --(6.455,6.307)--(6.453,6.316)--(6.450,6.326)--(6.447,6.335)--(6.443,6.344)%
    --(6.439,6.353)--(6.434,6.362)--(6.429,6.370)--(6.423,6.379)--(6.417,6.386)%
    --(6.411,6.394)--(6.404,6.401)--(6.397,6.408)--(6.389,6.414)--(6.382,6.420)%
    --(6.373,6.426)--(6.365,6.431)--(6.356,6.436)--(6.347,6.440)--(6.338,6.444)%
    --(6.329,6.447)--(6.319,6.450)--(6.310,6.452)--(6.300,6.454)--(6.290,6.455)%
    --(6.280,6.456)--(6.271,6.457)--(6.261,6.456)--(6.251,6.455)--(6.241,6.454)%
    --(6.231,6.452)--(6.222,6.450)--(6.212,6.447)--(6.203,6.444)--(6.194,6.440)%
    --(6.185,6.436)--(6.176,6.431)--(6.168,6.426)--(6.159,6.420)--(6.152,6.414)%
    --(6.144,6.408)--(6.137,6.401)--(6.130,6.394)--(6.124,6.386)--(6.118,6.379)%
    --(6.112,6.370)--(6.107,6.362)--(6.102,6.353)--(6.098,6.344)--(6.094,6.335)%
    --(6.091,6.326)--(6.088,6.316)--(6.086,6.307)--(6.084,6.297)--(6.083,6.287)%
    --(6.082,6.277)--(6.082,6.268)--(6.082,6.258)--(6.083,6.248)--(6.084,6.238)%
    --(6.086,6.228)--(6.088,6.219)--(6.091,6.209)--(6.094,6.200)--(6.098,6.191)%
    --(6.102,6.182)--(6.107,6.173)--(6.112,6.165)--(6.118,6.156)--(6.124,6.149)%
    --(6.130,6.141)--(6.137,6.134)--(6.144,6.127)--(6.152,6.121)--(6.159,6.115)%
    --(6.168,6.109)--(6.176,6.104)--(6.185,6.099)--(6.194,6.095)--(6.203,6.091)%
    --(6.212,6.088)--(6.222,6.085)--(6.231,6.083)--(6.241,6.081)--(6.251,6.080)%
    --(6.261,6.079)--(6.270,6.079)--(6.280,6.079)--(6.290,6.080)--(6.300,6.081)%
    --(6.310,6.083)--(6.319,6.085)--(6.329,6.088)--(6.338,6.091)--(6.347,6.095)%
    --(6.356,6.099)--(6.365,6.104)--(6.373,6.109)--(6.382,6.115)--(6.389,6.121)%
    --(6.397,6.127)--(6.404,6.134)--(6.411,6.141)--(6.417,6.149)--(6.423,6.156)%
    --(6.429,6.165)--(6.434,6.173)--(6.439,6.182)--(6.443,6.191)--(6.447,6.200)%
    --(6.450,6.209)--(6.453,6.219)--(6.455,6.228)--(6.457,6.238)--(6.458,6.248)%
    --(6.459,6.258)--(6.460,6.267)--cycle;
\gpfill{color=gp lt color border,opacity=0.50} (9.136,5.211)--(9.135,5.219)--(9.135,5.227)--(9.134,5.236)%
    --(9.132,5.244)--(9.130,5.252)--(9.128,5.261)--(9.125,5.269)--(9.121,5.276)%
    --(9.118,5.284)--(9.114,5.291)--(9.109,5.299)--(9.105,5.306)--(9.099,5.312)%
    --(9.094,5.319)--(9.088,5.325)--(9.082,5.331)--(9.075,5.336)--(9.069,5.342)%
    --(9.062,5.346)--(9.055,5.351)--(9.047,5.355)--(9.039,5.358)--(9.032,5.362)%
    --(9.024,5.365)--(9.015,5.367)--(9.007,5.369)--(8.999,5.371)--(8.990,5.372)%
    --(8.982,5.372)--(8.974,5.373)--(8.965,5.372)--(8.957,5.372)--(8.948,5.371)%
    --(8.940,5.369)--(8.932,5.367)--(8.923,5.365)--(8.915,5.362)--(8.908,5.358)%
    --(8.900,5.355)--(8.893,5.351)--(8.885,5.346)--(8.878,5.342)--(8.872,5.336)%
    --(8.865,5.331)--(8.859,5.325)--(8.853,5.319)--(8.848,5.312)--(8.842,5.306)%
    --(8.838,5.299)--(8.833,5.291)--(8.829,5.284)--(8.826,5.276)--(8.822,5.269)%
    --(8.819,5.261)--(8.817,5.252)--(8.815,5.244)--(8.813,5.236)--(8.812,5.227)%
    --(8.812,5.219)--(8.812,5.211)--(8.812,5.202)--(8.812,5.194)--(8.813,5.185)%
    --(8.815,5.177)--(8.817,5.169)--(8.819,5.160)--(8.822,5.152)--(8.826,5.145)%
    --(8.829,5.137)--(8.833,5.129)--(8.838,5.122)--(8.842,5.115)--(8.848,5.109)%
    --(8.853,5.102)--(8.859,5.096)--(8.865,5.090)--(8.872,5.085)--(8.878,5.079)%
    --(8.885,5.075)--(8.892,5.070)--(8.900,5.066)--(8.908,5.063)--(8.915,5.059)%
    --(8.923,5.056)--(8.932,5.054)--(8.940,5.052)--(8.948,5.050)--(8.957,5.049)%
    --(8.965,5.049)--(8.973,5.049)--(8.982,5.049)--(8.990,5.049)--(8.999,5.050)%
    --(9.007,5.052)--(9.015,5.054)--(9.024,5.056)--(9.032,5.059)--(9.039,5.063)%
    --(9.047,5.066)--(9.055,5.070)--(9.062,5.075)--(9.069,5.079)--(9.075,5.085)%
    --(9.082,5.090)--(9.088,5.096)--(9.094,5.102)--(9.099,5.109)--(9.105,5.115)%
    --(9.109,5.122)--(9.114,5.129)--(9.118,5.137)--(9.121,5.145)--(9.125,5.152)%
    --(9.128,5.160)--(9.130,5.169)--(9.132,5.177)--(9.134,5.185)--(9.135,5.194)%
    --(9.135,5.202)--(9.136,5.210)--cycle;
\gpfill{color=gp lt color border,opacity=0.50} (4.703,4.155)--(4.702,4.171)--(4.701,4.188)--(4.699,4.205)%
    --(4.695,4.222)--(4.691,4.238)--(4.687,4.255)--(4.681,4.271)--(4.674,4.286)%
    --(4.667,4.302)--(4.659,4.316)--(4.650,4.331)--(4.641,4.345)--(4.630,4.358)%
    --(4.619,4.371)--(4.608,4.384)--(4.595,4.395)--(4.582,4.406)--(4.569,4.417)%
    --(4.555,4.426)--(4.541,4.435)--(4.526,4.443)--(4.510,4.450)--(4.495,4.457)%
    --(4.479,4.463)--(4.462,4.467)--(4.446,4.471)--(4.429,4.475)--(4.412,4.477)%
    --(4.395,4.478)--(4.379,4.479)--(4.362,4.478)--(4.345,4.477)--(4.328,4.475)%
    --(4.311,4.471)--(4.295,4.467)--(4.278,4.463)--(4.262,4.457)--(4.247,4.450)%
    --(4.231,4.443)--(4.217,4.435)--(4.202,4.426)--(4.188,4.417)--(4.175,4.406)%
    --(4.162,4.395)--(4.149,4.384)--(4.138,4.371)--(4.127,4.358)--(4.116,4.345)%
    --(4.107,4.331)--(4.098,4.316)--(4.090,4.302)--(4.083,4.286)--(4.076,4.271)%
    --(4.070,4.255)--(4.066,4.238)--(4.062,4.222)--(4.058,4.205)--(4.056,4.188)%
    --(4.055,4.171)--(4.055,4.155)--(4.055,4.138)--(4.056,4.121)--(4.058,4.104)%
    --(4.062,4.087)--(4.066,4.071)--(4.070,4.054)--(4.076,4.038)--(4.083,4.023)%
    --(4.090,4.007)--(4.098,3.992)--(4.107,3.978)--(4.116,3.964)--(4.127,3.951)%
    --(4.138,3.938)--(4.149,3.925)--(4.162,3.914)--(4.175,3.903)--(4.188,3.892)%
    --(4.202,3.883)--(4.216,3.874)--(4.231,3.866)--(4.247,3.859)--(4.262,3.852)%
    --(4.278,3.846)--(4.295,3.842)--(4.311,3.838)--(4.328,3.834)--(4.345,3.832)%
    --(4.362,3.831)--(4.378,3.831)--(4.395,3.831)--(4.412,3.832)--(4.429,3.834)%
    --(4.446,3.838)--(4.462,3.842)--(4.479,3.846)--(4.495,3.852)--(4.510,3.859)%
    --(4.526,3.866)--(4.541,3.874)--(4.555,3.883)--(4.569,3.892)--(4.582,3.903)%
    --(4.595,3.914)--(4.608,3.925)--(4.619,3.938)--(4.630,3.951)--(4.641,3.964)%
    --(4.650,3.978)--(4.659,3.992)--(4.667,4.007)--(4.674,4.023)--(4.681,4.038)%
    --(4.687,4.054)--(4.691,4.071)--(4.695,4.087)--(4.699,4.104)--(4.701,4.121)%
    --(4.702,4.138)--(4.703,4.154)--cycle;
\gpfill{color=gp lt color border,opacity=0.50} (7.163,3.098)--(7.162,3.102)--(7.162,3.106)--(7.162,3.110)%
    --(7.161,3.114)--(7.160,3.118)--(7.159,3.123)--(7.157,3.127)--(7.155,3.130)%
    --(7.154,3.134)--(7.152,3.138)--(7.149,3.142)--(7.147,3.145)--(7.144,3.148)%
    --(7.142,3.152)--(7.139,3.155)--(7.136,3.158)--(7.132,3.160)--(7.129,3.163)%
    --(7.126,3.165)--(7.122,3.168)--(7.118,3.170)--(7.114,3.171)--(7.111,3.173)%
    --(7.107,3.175)--(7.102,3.176)--(7.098,3.177)--(7.094,3.178)--(7.090,3.178)%
    --(7.086,3.178)--(7.082,3.179)--(7.077,3.178)--(7.073,3.178)--(7.069,3.178)%
    --(7.065,3.177)--(7.061,3.176)--(7.056,3.175)--(7.052,3.173)--(7.049,3.171)%
    --(7.045,3.170)--(7.041,3.168)--(7.037,3.165)--(7.034,3.163)--(7.031,3.160)%
    --(7.027,3.158)--(7.024,3.155)--(7.021,3.152)--(7.019,3.148)--(7.016,3.145)%
    --(7.014,3.142)--(7.011,3.138)--(7.009,3.134)--(7.008,3.130)--(7.006,3.127)%
    --(7.004,3.123)--(7.003,3.118)--(7.002,3.114)--(7.001,3.110)--(7.001,3.106)%
    --(7.001,3.102)--(7.001,3.098)--(7.001,3.093)--(7.001,3.089)--(7.001,3.085)%
    --(7.002,3.081)--(7.003,3.077)--(7.004,3.072)--(7.006,3.068)--(7.008,3.065)%
    --(7.009,3.061)--(7.011,3.057)--(7.014,3.053)--(7.016,3.050)--(7.019,3.047)%
    --(7.021,3.043)--(7.024,3.040)--(7.027,3.037)--(7.031,3.035)--(7.034,3.032)%
    --(7.037,3.030)--(7.041,3.027)--(7.045,3.025)--(7.049,3.024)--(7.052,3.022)%
    --(7.056,3.020)--(7.061,3.019)--(7.065,3.018)--(7.069,3.017)--(7.073,3.017)%
    --(7.077,3.017)--(7.081,3.017)--(7.086,3.017)--(7.090,3.017)--(7.094,3.017)%
    --(7.098,3.018)--(7.102,3.019)--(7.107,3.020)--(7.111,3.022)--(7.114,3.024)%
    --(7.118,3.025)--(7.122,3.027)--(7.126,3.030)--(7.129,3.032)--(7.132,3.035)%
    --(7.136,3.037)--(7.139,3.040)--(7.142,3.043)--(7.144,3.047)--(7.147,3.050)%
    --(7.149,3.053)--(7.152,3.057)--(7.154,3.061)--(7.155,3.065)--(7.157,3.068)%
    --(7.159,3.072)--(7.160,3.077)--(7.161,3.081)--(7.162,3.085)--(7.162,3.089)%
    --(7.162,3.093)--(7.163,3.097)--cycle;
\gpfill{color=gp lt color border,opacity=0.50} (5.082,6.268)--(5.081,6.276)--(5.081,6.284)--(5.080,6.293)%
    --(5.078,6.301)--(5.076,6.309)--(5.074,6.318)--(5.071,6.326)--(5.067,6.333)%
    --(5.064,6.341)--(5.060,6.348)--(5.055,6.356)--(5.051,6.363)--(5.045,6.369)%
    --(5.040,6.376)--(5.034,6.382)--(5.028,6.388)--(5.021,6.393)--(5.015,6.399)%
    --(5.008,6.403)--(5.001,6.408)--(4.993,6.412)--(4.985,6.415)--(4.978,6.419)%
    --(4.970,6.422)--(4.961,6.424)--(4.953,6.426)--(4.945,6.428)--(4.936,6.429)%
    --(4.928,6.429)--(4.920,6.430)--(4.911,6.429)--(4.903,6.429)--(4.894,6.428)%
    --(4.886,6.426)--(4.878,6.424)--(4.869,6.422)--(4.861,6.419)--(4.854,6.415)%
    --(4.846,6.412)--(4.839,6.408)--(4.831,6.403)--(4.824,6.399)--(4.818,6.393)%
    --(4.811,6.388)--(4.805,6.382)--(4.799,6.376)--(4.794,6.369)--(4.788,6.363)%
    --(4.784,6.356)--(4.779,6.348)--(4.775,6.341)--(4.772,6.333)--(4.768,6.326)%
    --(4.765,6.318)--(4.763,6.309)--(4.761,6.301)--(4.759,6.293)--(4.758,6.284)%
    --(4.758,6.276)--(4.758,6.268)--(4.758,6.259)--(4.758,6.251)--(4.759,6.242)%
    --(4.761,6.234)--(4.763,6.226)--(4.765,6.217)--(4.768,6.209)--(4.772,6.202)%
    --(4.775,6.194)--(4.779,6.186)--(4.784,6.179)--(4.788,6.172)--(4.794,6.166)%
    --(4.799,6.159)--(4.805,6.153)--(4.811,6.147)--(4.818,6.142)--(4.824,6.136)%
    --(4.831,6.132)--(4.838,6.127)--(4.846,6.123)--(4.854,6.120)--(4.861,6.116)%
    --(4.869,6.113)--(4.878,6.111)--(4.886,6.109)--(4.894,6.107)--(4.903,6.106)%
    --(4.911,6.106)--(4.919,6.106)--(4.928,6.106)--(4.936,6.106)--(4.945,6.107)%
    --(4.953,6.109)--(4.961,6.111)--(4.970,6.113)--(4.978,6.116)--(4.985,6.120)%
    --(4.993,6.123)--(5.001,6.127)--(5.008,6.132)--(5.015,6.136)--(5.021,6.142)%
    --(5.028,6.147)--(5.034,6.153)--(5.040,6.159)--(5.045,6.166)--(5.051,6.172)%
    --(5.055,6.179)--(5.060,6.186)--(5.064,6.194)--(5.067,6.202)--(5.071,6.209)%
    --(5.074,6.217)--(5.076,6.226)--(5.078,6.234)--(5.080,6.242)--(5.081,6.251)%
    --(5.081,6.259)--(5.082,6.267)--cycle;
\gpfill{color=gp lt color border,opacity=0.50} (7.813,5.211)--(7.812,5.220)--(7.811,5.230)--(7.810,5.240)%
    --(7.808,5.250)--(7.806,5.260)--(7.803,5.269)--(7.800,5.279)--(7.796,5.288)%
    --(7.792,5.297)--(7.787,5.305)--(7.782,5.314)--(7.776,5.322)--(7.770,5.330)%
    --(7.764,5.338)--(7.757,5.345)--(7.750,5.352)--(7.742,5.358)--(7.734,5.364)%
    --(7.726,5.370)--(7.718,5.375)--(7.709,5.380)--(7.700,5.384)--(7.691,5.388)%
    --(7.681,5.391)--(7.672,5.394)--(7.662,5.396)--(7.652,5.398)--(7.642,5.399)%
    --(7.632,5.400)--(7.623,5.401)--(7.613,5.400)--(7.603,5.399)--(7.593,5.398)%
    --(7.583,5.396)--(7.573,5.394)--(7.564,5.391)--(7.554,5.388)--(7.545,5.384)%
    --(7.536,5.380)--(7.528,5.375)--(7.519,5.370)--(7.511,5.364)--(7.503,5.358)%
    --(7.495,5.352)--(7.488,5.345)--(7.481,5.338)--(7.475,5.330)--(7.469,5.322)%
    --(7.463,5.314)--(7.458,5.305)--(7.453,5.297)--(7.449,5.288)--(7.445,5.279)%
    --(7.442,5.269)--(7.439,5.260)--(7.437,5.250)--(7.435,5.240)--(7.434,5.230)%
    --(7.433,5.220)--(7.433,5.211)--(7.433,5.201)--(7.434,5.191)--(7.435,5.181)%
    --(7.437,5.171)--(7.439,5.161)--(7.442,5.152)--(7.445,5.142)--(7.449,5.133)%
    --(7.453,5.124)--(7.458,5.115)--(7.463,5.107)--(7.469,5.099)--(7.475,5.091)%
    --(7.481,5.083)--(7.488,5.076)--(7.495,5.069)--(7.503,5.063)--(7.511,5.057)%
    --(7.519,5.051)--(7.527,5.046)--(7.536,5.041)--(7.545,5.037)--(7.554,5.033)%
    --(7.564,5.030)--(7.573,5.027)--(7.583,5.025)--(7.593,5.023)--(7.603,5.022)%
    --(7.613,5.021)--(7.622,5.021)--(7.632,5.021)--(7.642,5.022)--(7.652,5.023)%
    --(7.662,5.025)--(7.672,5.027)--(7.681,5.030)--(7.691,5.033)--(7.700,5.037)%
    --(7.709,5.041)--(7.718,5.046)--(7.726,5.051)--(7.734,5.057)--(7.742,5.063)%
    --(7.750,5.069)--(7.757,5.076)--(7.764,5.083)--(7.770,5.091)--(7.776,5.099)%
    --(7.782,5.107)--(7.787,5.115)--(7.792,5.124)--(7.796,5.133)--(7.800,5.142)%
    --(7.803,5.152)--(7.806,5.161)--(7.808,5.171)--(7.810,5.181)--(7.811,5.191)%
    --(7.812,5.201)--(7.813,5.210)--cycle;
\gpfill{color=gp lt color border,opacity=0.50} (3.757,6.268)--(3.756,6.277)--(3.755,6.287)--(3.754,6.297)%
    --(3.752,6.307)--(3.750,6.316)--(3.747,6.326)--(3.744,6.335)--(3.740,6.344)%
    --(3.736,6.353)--(3.731,6.362)--(3.726,6.370)--(3.720,6.379)--(3.714,6.386)%
    --(3.708,6.394)--(3.701,6.401)--(3.694,6.408)--(3.686,6.414)--(3.679,6.420)%
    --(3.670,6.426)--(3.662,6.431)--(3.653,6.436)--(3.644,6.440)--(3.635,6.444)%
    --(3.626,6.447)--(3.616,6.450)--(3.607,6.452)--(3.597,6.454)--(3.587,6.455)%
    --(3.577,6.456)--(3.568,6.457)--(3.558,6.456)--(3.548,6.455)--(3.538,6.454)%
    --(3.528,6.452)--(3.519,6.450)--(3.509,6.447)--(3.500,6.444)--(3.491,6.440)%
    --(3.482,6.436)--(3.473,6.431)--(3.465,6.426)--(3.456,6.420)--(3.449,6.414)%
    --(3.441,6.408)--(3.434,6.401)--(3.427,6.394)--(3.421,6.386)--(3.415,6.379)%
    --(3.409,6.370)--(3.404,6.362)--(3.399,6.353)--(3.395,6.344)--(3.391,6.335)%
    --(3.388,6.326)--(3.385,6.316)--(3.383,6.307)--(3.381,6.297)--(3.380,6.287)%
    --(3.379,6.277)--(3.379,6.268)--(3.379,6.258)--(3.380,6.248)--(3.381,6.238)%
    --(3.383,6.228)--(3.385,6.219)--(3.388,6.209)--(3.391,6.200)--(3.395,6.191)%
    --(3.399,6.182)--(3.404,6.173)--(3.409,6.165)--(3.415,6.156)--(3.421,6.149)%
    --(3.427,6.141)--(3.434,6.134)--(3.441,6.127)--(3.449,6.121)--(3.456,6.115)%
    --(3.465,6.109)--(3.473,6.104)--(3.482,6.099)--(3.491,6.095)--(3.500,6.091)%
    --(3.509,6.088)--(3.519,6.085)--(3.528,6.083)--(3.538,6.081)--(3.548,6.080)%
    --(3.558,6.079)--(3.567,6.079)--(3.577,6.079)--(3.587,6.080)--(3.597,6.081)%
    --(3.607,6.083)--(3.616,6.085)--(3.626,6.088)--(3.635,6.091)--(3.644,6.095)%
    --(3.653,6.099)--(3.662,6.104)--(3.670,6.109)--(3.679,6.115)--(3.686,6.121)%
    --(3.694,6.127)--(3.701,6.134)--(3.708,6.141)--(3.714,6.149)--(3.720,6.156)%
    --(3.726,6.165)--(3.731,6.173)--(3.736,6.182)--(3.740,6.191)--(3.744,6.200)%
    --(3.747,6.209)--(3.750,6.219)--(3.752,6.228)--(3.754,6.238)--(3.755,6.248)%
    --(3.756,6.258)--(3.757,6.267)--cycle;
\gpfill{color=gp lt color border,opacity=0.50} (6.514,5.211)--(6.513,5.223)--(6.512,5.236)--(6.511,5.249)%
    --(6.508,5.261)--(6.505,5.273)--(6.502,5.286)--(6.497,5.298)--(6.492,5.309)%
    --(6.487,5.321)--(6.481,5.332)--(6.474,5.343)--(6.467,5.353)--(6.459,5.363)%
    --(6.451,5.373)--(6.442,5.382)--(6.433,5.391)--(6.423,5.399)--(6.413,5.407)%
    --(6.403,5.414)--(6.392,5.421)--(6.381,5.427)--(6.369,5.432)--(6.358,5.437)%
    --(6.346,5.442)--(6.333,5.445)--(6.321,5.448)--(6.309,5.451)--(6.296,5.452)%
    --(6.283,5.453)--(6.271,5.454)--(6.258,5.453)--(6.245,5.452)--(6.232,5.451)%
    --(6.220,5.448)--(6.208,5.445)--(6.195,5.442)--(6.183,5.437)--(6.172,5.432)%
    --(6.160,5.427)--(6.149,5.421)--(6.138,5.414)--(6.128,5.407)--(6.118,5.399)%
    --(6.108,5.391)--(6.099,5.382)--(6.090,5.373)--(6.082,5.363)--(6.074,5.353)%
    --(6.067,5.343)--(6.060,5.332)--(6.054,5.321)--(6.049,5.309)--(6.044,5.298)%
    --(6.039,5.286)--(6.036,5.273)--(6.033,5.261)--(6.030,5.249)--(6.029,5.236)%
    --(6.028,5.223)--(6.028,5.211)--(6.028,5.198)--(6.029,5.185)--(6.030,5.172)%
    --(6.033,5.160)--(6.036,5.148)--(6.039,5.135)--(6.044,5.123)--(6.049,5.112)%
    --(6.054,5.100)--(6.060,5.089)--(6.067,5.078)--(6.074,5.068)--(6.082,5.058)%
    --(6.090,5.048)--(6.099,5.039)--(6.108,5.030)--(6.118,5.022)--(6.128,5.014)%
    --(6.138,5.007)--(6.149,5.000)--(6.160,4.994)--(6.172,4.989)--(6.183,4.984)%
    --(6.195,4.979)--(6.208,4.976)--(6.220,4.973)--(6.232,4.970)--(6.245,4.969)%
    --(6.258,4.968)--(6.270,4.968)--(6.283,4.968)--(6.296,4.969)--(6.309,4.970)%
    --(6.321,4.973)--(6.333,4.976)--(6.346,4.979)--(6.358,4.984)--(6.369,4.989)%
    --(6.381,4.994)--(6.392,5.000)--(6.403,5.007)--(6.413,5.014)--(6.423,5.022)%
    --(6.433,5.030)--(6.442,5.039)--(6.451,5.048)--(6.459,5.058)--(6.467,5.068)%
    --(6.474,5.078)--(6.481,5.089)--(6.487,5.100)--(6.492,5.112)--(6.497,5.123)%
    --(6.502,5.135)--(6.505,5.148)--(6.508,5.160)--(6.511,5.172)--(6.512,5.185)%
    --(6.513,5.198)--(6.514,5.210)--cycle;
\gpfill{color=gp lt color border,opacity=0.50} (2.137,4.155)--(2.136,4.179)--(2.134,4.203)--(2.131,4.226)%
    --(2.126,4.250)--(2.121,4.274)--(2.114,4.297)--(2.106,4.319)--(2.097,4.342)%
    --(2.086,4.363)--(2.075,4.384)--(2.062,4.405)--(2.049,4.425)--(2.034,4.444)%
    --(2.018,4.462)--(2.002,4.480)--(1.984,4.496)--(1.966,4.512)--(1.947,4.527)%
    --(1.927,4.540)--(1.907,4.553)--(1.885,4.564)--(1.864,4.575)--(1.841,4.584)%
    --(1.819,4.592)--(1.796,4.599)--(1.772,4.604)--(1.748,4.609)--(1.725,4.612)%
    --(1.701,4.614)--(1.677,4.615)--(1.652,4.614)--(1.628,4.612)--(1.605,4.609)%
    --(1.581,4.604)--(1.557,4.599)--(1.534,4.592)--(1.512,4.584)--(1.489,4.575)%
    --(1.468,4.564)--(1.447,4.553)--(1.426,4.540)--(1.406,4.527)--(1.387,4.512)%
    --(1.369,4.496)--(1.351,4.480)--(1.335,4.462)--(1.319,4.444)--(1.304,4.425)%
    --(1.291,4.405)--(1.278,4.384)--(1.267,4.363)--(1.256,4.342)--(1.247,4.319)%
    --(1.239,4.297)--(1.232,4.274)--(1.227,4.250)--(1.222,4.226)--(1.219,4.203)%
    --(1.217,4.179)--(1.217,4.155)--(1.217,4.130)--(1.219,4.106)--(1.222,4.083)%
    --(1.227,4.059)--(1.232,4.035)--(1.239,4.012)--(1.247,3.990)--(1.256,3.967)%
    --(1.267,3.946)--(1.278,3.924)--(1.291,3.904)--(1.304,3.884)--(1.319,3.865)%
    --(1.335,3.847)--(1.351,3.829)--(1.369,3.813)--(1.387,3.797)--(1.406,3.782)%
    --(1.426,3.769)--(1.446,3.756)--(1.468,3.745)--(1.489,3.734)--(1.512,3.725)%
    --(1.534,3.717)--(1.557,3.710)--(1.581,3.705)--(1.605,3.700)--(1.628,3.697)%
    --(1.652,3.695)--(1.676,3.695)--(1.701,3.695)--(1.725,3.697)--(1.748,3.700)%
    --(1.772,3.705)--(1.796,3.710)--(1.819,3.717)--(1.841,3.725)--(1.864,3.734)%
    --(1.885,3.745)--(1.907,3.756)--(1.927,3.769)--(1.947,3.782)--(1.966,3.797)%
    --(1.984,3.813)--(2.002,3.829)--(2.018,3.847)--(2.034,3.865)--(2.049,3.884)%
    --(2.062,3.904)--(2.075,3.924)--(2.086,3.946)--(2.097,3.967)--(2.106,3.990)%
    --(2.114,4.012)--(2.121,4.035)--(2.126,4.059)--(2.131,4.083)--(2.134,4.106)%
    --(2.136,4.130)--(2.137,4.154)--cycle;
\gpfill{color=gp lt color border,opacity=0.50} (4.568,3.098)--(4.567,3.107)--(4.566,3.117)--(4.565,3.127)%
    --(4.563,3.137)--(4.561,3.146)--(4.558,3.156)--(4.555,3.165)--(4.551,3.174)%
    --(4.547,3.183)--(4.542,3.192)--(4.537,3.200)--(4.531,3.209)--(4.525,3.216)%
    --(4.519,3.224)--(4.512,3.231)--(4.505,3.238)--(4.497,3.244)--(4.490,3.250)%
    --(4.481,3.256)--(4.473,3.261)--(4.464,3.266)--(4.455,3.270)--(4.446,3.274)%
    --(4.437,3.277)--(4.427,3.280)--(4.418,3.282)--(4.408,3.284)--(4.398,3.285)%
    --(4.388,3.286)--(4.379,3.287)--(4.369,3.286)--(4.359,3.285)--(4.349,3.284)%
    --(4.339,3.282)--(4.330,3.280)--(4.320,3.277)--(4.311,3.274)--(4.302,3.270)%
    --(4.293,3.266)--(4.284,3.261)--(4.276,3.256)--(4.267,3.250)--(4.260,3.244)%
    --(4.252,3.238)--(4.245,3.231)--(4.238,3.224)--(4.232,3.216)--(4.226,3.209)%
    --(4.220,3.200)--(4.215,3.192)--(4.210,3.183)--(4.206,3.174)--(4.202,3.165)%
    --(4.199,3.156)--(4.196,3.146)--(4.194,3.137)--(4.192,3.127)--(4.191,3.117)%
    --(4.190,3.107)--(4.190,3.098)--(4.190,3.088)--(4.191,3.078)--(4.192,3.068)%
    --(4.194,3.058)--(4.196,3.049)--(4.199,3.039)--(4.202,3.030)--(4.206,3.021)%
    --(4.210,3.012)--(4.215,3.003)--(4.220,2.995)--(4.226,2.986)--(4.232,2.979)%
    --(4.238,2.971)--(4.245,2.964)--(4.252,2.957)--(4.260,2.951)--(4.267,2.945)%
    --(4.276,2.939)--(4.284,2.934)--(4.293,2.929)--(4.302,2.925)--(4.311,2.921)%
    --(4.320,2.918)--(4.330,2.915)--(4.339,2.913)--(4.349,2.911)--(4.359,2.910)%
    --(4.369,2.909)--(4.378,2.909)--(4.388,2.909)--(4.398,2.910)--(4.408,2.911)%
    --(4.418,2.913)--(4.427,2.915)--(4.437,2.918)--(4.446,2.921)--(4.455,2.925)%
    --(4.464,2.929)--(4.473,2.934)--(4.481,2.939)--(4.490,2.945)--(4.497,2.951)%
    --(4.505,2.957)--(4.512,2.964)--(4.519,2.971)--(4.525,2.979)--(4.531,2.986)%
    --(4.537,2.995)--(4.542,3.003)--(4.547,3.012)--(4.551,3.021)--(4.555,3.030)%
    --(4.558,3.039)--(4.561,3.049)--(4.563,3.058)--(4.565,3.068)--(4.566,3.078)%
    --(4.567,3.088)--(4.568,3.097)--cycle;
\gpfill{color=gp lt color border,opacity=0.50} (2.379,6.268)--(2.378,6.276)--(2.378,6.284)--(2.377,6.293)%
    --(2.375,6.301)--(2.373,6.309)--(2.371,6.318)--(2.368,6.326)--(2.364,6.333)%
    --(2.361,6.341)--(2.357,6.348)--(2.352,6.356)--(2.348,6.363)--(2.342,6.369)%
    --(2.337,6.376)--(2.331,6.382)--(2.325,6.388)--(2.318,6.393)--(2.312,6.399)%
    --(2.305,6.403)--(2.298,6.408)--(2.290,6.412)--(2.282,6.415)--(2.275,6.419)%
    --(2.267,6.422)--(2.258,6.424)--(2.250,6.426)--(2.242,6.428)--(2.233,6.429)%
    --(2.225,6.429)--(2.217,6.430)--(2.208,6.429)--(2.200,6.429)--(2.191,6.428)%
    --(2.183,6.426)--(2.175,6.424)--(2.166,6.422)--(2.158,6.419)--(2.151,6.415)%
    --(2.143,6.412)--(2.136,6.408)--(2.128,6.403)--(2.121,6.399)--(2.115,6.393)%
    --(2.108,6.388)--(2.102,6.382)--(2.096,6.376)--(2.091,6.369)--(2.085,6.363)%
    --(2.081,6.356)--(2.076,6.348)--(2.072,6.341)--(2.069,6.333)--(2.065,6.326)%
    --(2.062,6.318)--(2.060,6.309)--(2.058,6.301)--(2.056,6.293)--(2.055,6.284)%
    --(2.055,6.276)--(2.055,6.268)--(2.055,6.259)--(2.055,6.251)--(2.056,6.242)%
    --(2.058,6.234)--(2.060,6.226)--(2.062,6.217)--(2.065,6.209)--(2.069,6.202)%
    --(2.072,6.194)--(2.076,6.186)--(2.081,6.179)--(2.085,6.172)--(2.091,6.166)%
    --(2.096,6.159)--(2.102,6.153)--(2.108,6.147)--(2.115,6.142)--(2.121,6.136)%
    --(2.128,6.132)--(2.135,6.127)--(2.143,6.123)--(2.151,6.120)--(2.158,6.116)%
    --(2.166,6.113)--(2.175,6.111)--(2.183,6.109)--(2.191,6.107)--(2.200,6.106)%
    --(2.208,6.106)--(2.216,6.106)--(2.225,6.106)--(2.233,6.106)--(2.242,6.107)%
    --(2.250,6.109)--(2.258,6.111)--(2.267,6.113)--(2.275,6.116)--(2.282,6.120)%
    --(2.290,6.123)--(2.298,6.127)--(2.305,6.132)--(2.312,6.136)--(2.318,6.142)%
    --(2.325,6.147)--(2.331,6.153)--(2.337,6.159)--(2.342,6.166)--(2.348,6.172)%
    --(2.352,6.179)--(2.357,6.186)--(2.361,6.194)--(2.364,6.202)--(2.368,6.209)%
    --(2.371,6.217)--(2.373,6.226)--(2.375,6.234)--(2.377,6.242)--(2.378,6.251)%
    --(2.378,6.259)--(2.379,6.267)--cycle;
\gpfill{color=gp lt color border,opacity=0.50} (5.190,5.211)--(5.189,5.225)--(5.188,5.239)--(5.186,5.253)%
    --(5.184,5.267)--(5.180,5.280)--(5.176,5.294)--(5.172,5.307)--(5.166,5.320)%
    --(5.160,5.333)--(5.153,5.345)--(5.146,5.358)--(5.138,5.369)--(5.129,5.380)%
    --(5.120,5.391)--(5.110,5.401)--(5.100,5.411)--(5.089,5.420)--(5.078,5.429)%
    --(5.067,5.437)--(5.055,5.444)--(5.042,5.451)--(5.029,5.457)--(5.016,5.463)%
    --(5.003,5.467)--(4.989,5.471)--(4.976,5.475)--(4.962,5.477)--(4.948,5.479)%
    --(4.934,5.480)--(4.920,5.481)--(4.905,5.480)--(4.891,5.479)--(4.877,5.477)%
    --(4.863,5.475)--(4.850,5.471)--(4.836,5.467)--(4.823,5.463)--(4.810,5.457)%
    --(4.797,5.451)--(4.785,5.444)--(4.772,5.437)--(4.761,5.429)--(4.750,5.420)%
    --(4.739,5.411)--(4.729,5.401)--(4.719,5.391)--(4.710,5.380)--(4.701,5.369)%
    --(4.693,5.358)--(4.686,5.345)--(4.679,5.333)--(4.673,5.320)--(4.667,5.307)%
    --(4.663,5.294)--(4.659,5.280)--(4.655,5.267)--(4.653,5.253)--(4.651,5.239)%
    --(4.650,5.225)--(4.650,5.211)--(4.650,5.196)--(4.651,5.182)--(4.653,5.168)%
    --(4.655,5.154)--(4.659,5.141)--(4.663,5.127)--(4.667,5.114)--(4.673,5.101)%
    --(4.679,5.088)--(4.686,5.075)--(4.693,5.063)--(4.701,5.052)--(4.710,5.041)%
    --(4.719,5.030)--(4.729,5.020)--(4.739,5.010)--(4.750,5.001)--(4.761,4.992)%
    --(4.772,4.984)--(4.784,4.977)--(4.797,4.970)--(4.810,4.964)--(4.823,4.958)%
    --(4.836,4.954)--(4.850,4.950)--(4.863,4.946)--(4.877,4.944)--(4.891,4.942)%
    --(4.905,4.941)--(4.919,4.941)--(4.934,4.941)--(4.948,4.942)--(4.962,4.944)%
    --(4.976,4.946)--(4.989,4.950)--(5.003,4.954)--(5.016,4.958)--(5.029,4.964)%
    --(5.042,4.970)--(5.055,4.977)--(5.067,4.984)--(5.078,4.992)--(5.089,5.001)%
    --(5.100,5.010)--(5.110,5.020)--(5.120,5.030)--(5.129,5.041)--(5.138,5.052)%
    --(5.146,5.063)--(5.153,5.075)--(5.160,5.088)--(5.166,5.101)--(5.172,5.114)%
    --(5.176,5.127)--(5.180,5.141)--(5.184,5.154)--(5.186,5.168)--(5.188,5.182)%
    --(5.189,5.196)--(5.190,5.210)--cycle;
\gpfill{color=gp lt color border,opacity=0.50} (3.865,5.211)--(3.864,5.226)--(3.863,5.242)--(3.861,5.257)%
    --(3.858,5.272)--(3.854,5.287)--(3.850,5.302)--(3.845,5.317)--(3.839,5.331)%
    --(3.832,5.345)--(3.825,5.359)--(3.817,5.372)--(3.808,5.385)--(3.798,5.397)%
    --(3.788,5.409)--(3.778,5.421)--(3.766,5.431)--(3.754,5.441)--(3.742,5.451)%
    --(3.729,5.460)--(3.716,5.468)--(3.702,5.475)--(3.688,5.482)--(3.674,5.488)%
    --(3.659,5.493)--(3.644,5.497)--(3.629,5.501)--(3.614,5.504)--(3.599,5.506)%
    --(3.583,5.507)--(3.568,5.508)--(3.552,5.507)--(3.536,5.506)--(3.521,5.504)%
    --(3.506,5.501)--(3.491,5.497)--(3.476,5.493)--(3.461,5.488)--(3.447,5.482)%
    --(3.433,5.475)--(3.419,5.468)--(3.406,5.460)--(3.393,5.451)--(3.381,5.441)%
    --(3.369,5.431)--(3.357,5.421)--(3.347,5.409)--(3.337,5.397)--(3.327,5.385)%
    --(3.318,5.372)--(3.310,5.359)--(3.303,5.345)--(3.296,5.331)--(3.290,5.317)%
    --(3.285,5.302)--(3.281,5.287)--(3.277,5.272)--(3.274,5.257)--(3.272,5.242)%
    --(3.271,5.226)--(3.271,5.211)--(3.271,5.195)--(3.272,5.179)--(3.274,5.164)%
    --(3.277,5.149)--(3.281,5.134)--(3.285,5.119)--(3.290,5.104)--(3.296,5.090)%
    --(3.303,5.076)--(3.310,5.062)--(3.318,5.049)--(3.327,5.036)--(3.337,5.024)%
    --(3.347,5.012)--(3.357,5.000)--(3.369,4.990)--(3.381,4.980)--(3.393,4.970)%
    --(3.406,4.961)--(3.419,4.953)--(3.433,4.946)--(3.447,4.939)--(3.461,4.933)%
    --(3.476,4.928)--(3.491,4.924)--(3.506,4.920)--(3.521,4.917)--(3.536,4.915)%
    --(3.552,4.914)--(3.567,4.914)--(3.583,4.914)--(3.599,4.915)--(3.614,4.917)%
    --(3.629,4.920)--(3.644,4.924)--(3.659,4.928)--(3.674,4.933)--(3.688,4.939)%
    --(3.702,4.946)--(3.716,4.953)--(3.729,4.961)--(3.742,4.970)--(3.754,4.980)%
    --(3.766,4.990)--(3.778,5.000)--(3.788,5.012)--(3.798,5.024)--(3.808,5.036)%
    --(3.817,5.049)--(3.825,5.062)--(3.832,5.076)--(3.839,5.090)--(3.845,5.104)%
    --(3.850,5.119)--(3.854,5.134)--(3.858,5.149)--(3.861,5.164)--(3.863,5.179)%
    --(3.864,5.195)--(3.865,5.210)--cycle;
\gpfill{color=gp lt color border,opacity=0.50} (6.271,8.381)--(6.271,8.381)--(6.271,8.381)--(6.271,8.381)%
    --(6.271,8.381)--(6.271,8.381)--(6.271,8.381)--(6.271,8.381)--(6.271,8.381)%
    --(6.271,8.381)--(6.271,8.381)--(6.271,8.381)--(6.271,8.381)--(6.271,8.381)%
    --(6.271,8.381)--(6.271,8.381)--(6.271,8.381)--(6.271,8.381)--(6.271,8.381)%
    --(6.271,8.381)--(6.271,8.381)--(6.271,8.381)--(6.271,8.381)--(6.271,8.381)%
    --(6.271,8.381)--(6.271,8.381)--(6.271,8.381)--(6.271,8.381)--(6.271,8.381)%
    --(6.271,8.381)--(6.271,8.381)--(6.271,8.381)--(6.271,8.381)--(6.271,8.381)%
    --(6.271,8.381)--(6.271,8.381)--(6.271,8.381)--(6.271,8.381)--(6.271,8.381)%
    --(6.271,8.381)--(6.271,8.381)--(6.271,8.381)--(6.271,8.381)--(6.271,8.381)%
    --(6.271,8.381)--(6.271,8.381)--(6.271,8.381)--(6.271,8.381)--(6.271,8.381)%
    --(6.271,8.381)--(6.271,8.381)--(6.271,8.381)--(6.271,8.381)--(6.271,8.381)%
    --(6.271,8.381)--(6.271,8.381)--(6.271,8.381)--(6.271,8.381)--(6.271,8.381)%
    --(6.271,8.381)--(6.271,8.381)--(6.271,8.381)--(6.271,8.381)--(6.271,8.381)%
    --(6.271,8.381)--(6.271,8.381)--(6.271,8.381)--(6.271,8.381)--(6.271,8.381)%
    --(6.271,8.381)--(6.271,8.381)--(6.271,8.381)--(6.271,8.381)--(6.271,8.381)%
    --(6.271,8.381)--(6.271,8.381)--(6.271,8.381)--(6.271,8.381)--(6.271,8.381)%
    --(6.271,8.381)--(6.271,8.381)--(6.271,8.381)--(6.271,8.381)--(6.271,8.381)%
    --(6.271,8.381)--(6.271,8.381)--(6.271,8.381)--(6.271,8.381)--(6.271,8.381)%
    --(6.271,8.381)--(6.271,8.381)--(6.271,8.381)--(6.271,8.381)--(6.271,8.381)%
    --(6.271,8.381)--(6.271,8.381)--(6.271,8.381)--(6.271,8.381)--(6.271,8.381)%
    --(6.271,8.381)--(6.271,8.381)--(6.271,8.381)--(6.271,8.381)--(6.271,8.381)%
    --(6.271,8.381)--(6.271,8.381)--(6.271,8.381)--(6.271,8.381)--(6.271,8.381)%
    --(6.271,8.381)--(6.271,8.381)--(6.271,8.381)--(6.271,8.381)--(6.271,8.381)%
    --(6.271,8.381)--(6.271,8.381)--(6.271,8.381)--(6.271,8.381)--(6.271,8.381)%
    --(6.271,8.381)--cycle;
\gpfill{color=gp lt color border,opacity=0.50} (2.137,3.098)--(2.136,3.122)--(2.134,3.146)--(2.131,3.169)%
    --(2.126,3.193)--(2.121,3.217)--(2.114,3.240)--(2.106,3.262)--(2.097,3.285)%
    --(2.086,3.306)--(2.075,3.327)--(2.062,3.348)--(2.049,3.368)--(2.034,3.387)%
    --(2.018,3.405)--(2.002,3.423)--(1.984,3.439)--(1.966,3.455)--(1.947,3.470)%
    --(1.927,3.483)--(1.907,3.496)--(1.885,3.507)--(1.864,3.518)--(1.841,3.527)%
    --(1.819,3.535)--(1.796,3.542)--(1.772,3.547)--(1.748,3.552)--(1.725,3.555)%
    --(1.701,3.557)--(1.677,3.558)--(1.652,3.557)--(1.628,3.555)--(1.605,3.552)%
    --(1.581,3.547)--(1.557,3.542)--(1.534,3.535)--(1.512,3.527)--(1.489,3.518)%
    --(1.468,3.507)--(1.447,3.496)--(1.426,3.483)--(1.406,3.470)--(1.387,3.455)%
    --(1.369,3.439)--(1.351,3.423)--(1.335,3.405)--(1.319,3.387)--(1.304,3.368)%
    --(1.291,3.348)--(1.278,3.327)--(1.267,3.306)--(1.256,3.285)--(1.247,3.262)%
    --(1.239,3.240)--(1.232,3.217)--(1.227,3.193)--(1.222,3.169)--(1.219,3.146)%
    --(1.217,3.122)--(1.217,3.098)--(1.217,3.073)--(1.219,3.049)--(1.222,3.026)%
    --(1.227,3.002)--(1.232,2.978)--(1.239,2.955)--(1.247,2.933)--(1.256,2.910)%
    --(1.267,2.889)--(1.278,2.867)--(1.291,2.847)--(1.304,2.827)--(1.319,2.808)%
    --(1.335,2.790)--(1.351,2.772)--(1.369,2.756)--(1.387,2.740)--(1.406,2.725)%
    --(1.426,2.712)--(1.446,2.699)--(1.468,2.688)--(1.489,2.677)--(1.512,2.668)%
    --(1.534,2.660)--(1.557,2.653)--(1.581,2.648)--(1.605,2.643)--(1.628,2.640)%
    --(1.652,2.638)--(1.676,2.638)--(1.701,2.638)--(1.725,2.640)--(1.748,2.643)%
    --(1.772,2.648)--(1.796,2.653)--(1.819,2.660)--(1.841,2.668)--(1.864,2.677)%
    --(1.885,2.688)--(1.907,2.699)--(1.927,2.712)--(1.947,2.725)--(1.966,2.740)%
    --(1.984,2.756)--(2.002,2.772)--(2.018,2.790)--(2.034,2.808)--(2.049,2.827)%
    --(2.062,2.847)--(2.075,2.867)--(2.086,2.889)--(2.097,2.910)--(2.106,2.933)%
    --(2.114,2.955)--(2.121,2.978)--(2.126,3.002)--(2.131,3.026)--(2.134,3.049)%
    --(2.136,3.073)--(2.137,3.097)--cycle;
\gpfill{color=gp lt color border,opacity=0.50} (2.541,5.211)--(2.540,5.227)--(2.539,5.244)--(2.537,5.261)%
    --(2.533,5.278)--(2.529,5.294)--(2.525,5.311)--(2.519,5.327)--(2.512,5.342)%
    --(2.505,5.358)--(2.497,5.372)--(2.488,5.387)--(2.479,5.401)--(2.468,5.414)%
    --(2.457,5.427)--(2.446,5.440)--(2.433,5.451)--(2.420,5.462)--(2.407,5.473)%
    --(2.393,5.482)--(2.379,5.491)--(2.364,5.499)--(2.348,5.506)--(2.333,5.513)%
    --(2.317,5.519)--(2.300,5.523)--(2.284,5.527)--(2.267,5.531)--(2.250,5.533)%
    --(2.233,5.534)--(2.217,5.535)--(2.200,5.534)--(2.183,5.533)--(2.166,5.531)%
    --(2.149,5.527)--(2.133,5.523)--(2.116,5.519)--(2.100,5.513)--(2.085,5.506)%
    --(2.069,5.499)--(2.055,5.491)--(2.040,5.482)--(2.026,5.473)--(2.013,5.462)%
    --(2.000,5.451)--(1.987,5.440)--(1.976,5.427)--(1.965,5.414)--(1.954,5.401)%
    --(1.945,5.387)--(1.936,5.372)--(1.928,5.358)--(1.921,5.342)--(1.914,5.327)%
    --(1.908,5.311)--(1.904,5.294)--(1.900,5.278)--(1.896,5.261)--(1.894,5.244)%
    --(1.893,5.227)--(1.893,5.211)--(1.893,5.194)--(1.894,5.177)--(1.896,5.160)%
    --(1.900,5.143)--(1.904,5.127)--(1.908,5.110)--(1.914,5.094)--(1.921,5.079)%
    --(1.928,5.063)--(1.936,5.048)--(1.945,5.034)--(1.954,5.020)--(1.965,5.007)%
    --(1.976,4.994)--(1.987,4.981)--(2.000,4.970)--(2.013,4.959)--(2.026,4.948)%
    --(2.040,4.939)--(2.054,4.930)--(2.069,4.922)--(2.085,4.915)--(2.100,4.908)%
    --(2.116,4.902)--(2.133,4.898)--(2.149,4.894)--(2.166,4.890)--(2.183,4.888)%
    --(2.200,4.887)--(2.216,4.887)--(2.233,4.887)--(2.250,4.888)--(2.267,4.890)%
    --(2.284,4.894)--(2.300,4.898)--(2.317,4.902)--(2.333,4.908)--(2.348,4.915)%
    --(2.364,4.922)--(2.379,4.930)--(2.393,4.939)--(2.407,4.948)--(2.420,4.959)%
    --(2.433,4.970)--(2.446,4.981)--(2.457,4.994)--(2.468,5.007)--(2.479,5.020)%
    --(2.488,5.034)--(2.497,5.048)--(2.505,5.063)--(2.512,5.079)--(2.519,5.094)%
    --(2.525,5.110)--(2.529,5.127)--(2.533,5.143)--(2.537,5.160)--(2.539,5.177)%
    --(2.540,5.194)--(2.541,5.210)--cycle;
\gpfill{color=gp lt color border,opacity=0.50} (4.920,8.381)--(4.920,8.381)--(4.920,8.381)--(4.920,8.381)%
    --(4.920,8.381)--(4.920,8.381)--(4.920,8.381)--(4.920,8.381)--(4.920,8.381)%
    --(4.920,8.381)--(4.920,8.381)--(4.920,8.381)--(4.920,8.381)--(4.920,8.381)%
    --(4.920,8.381)--(4.920,8.381)--(4.920,8.381)--(4.920,8.381)--(4.920,8.381)%
    --(4.920,8.381)--(4.920,8.381)--(4.920,8.381)--(4.920,8.381)--(4.920,8.381)%
    --(4.920,8.381)--(4.920,8.381)--(4.920,8.381)--(4.920,8.381)--(4.920,8.381)%
    --(4.920,8.381)--(4.920,8.381)--(4.920,8.381)--(4.920,8.381)--(4.920,8.381)%
    --(4.920,8.381)--(4.920,8.381)--(4.920,8.381)--(4.920,8.381)--(4.920,8.381)%
    --(4.920,8.381)--(4.920,8.381)--(4.920,8.381)--(4.920,8.381)--(4.920,8.381)%
    --(4.920,8.381)--(4.920,8.381)--(4.920,8.381)--(4.920,8.381)--(4.920,8.381)%
    --(4.920,8.381)--(4.920,8.381)--(4.920,8.381)--(4.920,8.381)--(4.920,8.381)%
    --(4.920,8.381)--(4.920,8.381)--(4.920,8.381)--(4.920,8.381)--(4.920,8.381)%
    --(4.920,8.381)--(4.920,8.381)--(4.920,8.381)--(4.920,8.381)--(4.920,8.381)%
    --(4.920,8.381)--(4.920,8.381)--(4.920,8.381)--(4.920,8.381)--(4.920,8.381)%
    --(4.920,8.381)--(4.920,8.381)--(4.920,8.381)--(4.920,8.381)--(4.920,8.381)%
    --(4.920,8.381)--(4.920,8.381)--(4.920,8.381)--(4.920,8.381)--(4.920,8.381)%
    --(4.920,8.381)--(4.920,8.381)--(4.920,8.381)--(4.920,8.381)--(4.920,8.381)%
    --(4.920,8.381)--(4.920,8.381)--(4.920,8.381)--(4.920,8.381)--(4.920,8.381)%
    --(4.920,8.381)--(4.920,8.381)--(4.920,8.381)--(4.920,8.381)--(4.920,8.381)%
    --(4.920,8.381)--(4.920,8.381)--(4.920,8.381)--(4.920,8.381)--(4.920,8.381)%
    --(4.920,8.381)--(4.920,8.381)--(4.920,8.381)--(4.920,8.381)--(4.920,8.381)%
    --(4.920,8.381)--(4.920,8.381)--(4.920,8.381)--(4.920,8.381)--(4.920,8.381)%
    --(4.920,8.381)--(4.920,8.381)--(4.920,8.381)--(4.920,8.381)--(4.920,8.381)%
    --(4.920,8.381)--(4.920,8.381)--(4.920,8.381)--(4.920,8.381)--(4.920,8.381)%
    --(4.920,8.381)--cycle;
\gpfill{color=gp lt color border,opacity=0.50} (7.623,7.324)--(7.623,7.324)--(7.623,7.324)--(7.623,7.324)%
    --(7.623,7.324)--(7.623,7.324)--(7.623,7.324)--(7.623,7.324)--(7.623,7.324)%
    --(7.623,7.324)--(7.623,7.324)--(7.623,7.324)--(7.623,7.324)--(7.623,7.324)%
    --(7.623,7.324)--(7.623,7.324)--(7.623,7.324)--(7.623,7.324)--(7.623,7.324)%
    --(7.623,7.324)--(7.623,7.324)--(7.623,7.324)--(7.623,7.324)--(7.623,7.324)%
    --(7.623,7.324)--(7.623,7.324)--(7.623,7.324)--(7.623,7.324)--(7.623,7.324)%
    --(7.623,7.324)--(7.623,7.324)--(7.623,7.324)--(7.623,7.324)--(7.623,7.324)%
    --(7.623,7.324)--(7.623,7.324)--(7.623,7.324)--(7.623,7.324)--(7.623,7.324)%
    --(7.623,7.324)--(7.623,7.324)--(7.623,7.324)--(7.623,7.324)--(7.623,7.324)%
    --(7.623,7.324)--(7.623,7.324)--(7.623,7.324)--(7.623,7.324)--(7.623,7.324)%
    --(7.623,7.324)--(7.623,7.324)--(7.623,7.324)--(7.623,7.324)--(7.623,7.324)%
    --(7.623,7.324)--(7.623,7.324)--(7.623,7.324)--(7.623,7.324)--(7.623,7.324)%
    --(7.623,7.324)--(7.623,7.324)--(7.623,7.324)--(7.623,7.324)--(7.623,7.324)%
    --(7.623,7.324)--(7.623,7.324)--(7.623,7.324)--(7.623,7.324)--(7.623,7.324)%
    --(7.623,7.324)--(7.623,7.324)--(7.623,7.324)--(7.623,7.324)--(7.623,7.324)%
    --(7.623,7.324)--(7.623,7.324)--(7.623,7.324)--(7.623,7.324)--(7.623,7.324)%
    --(7.623,7.324)--(7.623,7.324)--(7.623,7.324)--(7.623,7.324)--(7.623,7.324)%
    --(7.623,7.324)--(7.623,7.324)--(7.623,7.324)--(7.623,7.324)--(7.623,7.324)%
    --(7.623,7.324)--(7.623,7.324)--(7.623,7.324)--(7.623,7.324)--(7.623,7.324)%
    --(7.623,7.324)--(7.623,7.324)--(7.623,7.324)--(7.623,7.324)--(7.623,7.324)%
    --(7.623,7.324)--(7.623,7.324)--(7.623,7.324)--(7.623,7.324)--(7.623,7.324)%
    --(7.623,7.324)--(7.623,7.324)--(7.623,7.324)--(7.623,7.324)--(7.623,7.324)%
    --(7.623,7.324)--(7.623,7.324)--(7.623,7.324)--(7.623,7.324)--(7.623,7.324)%
    --(7.623,7.324)--(7.623,7.324)--(7.623,7.324)--(7.623,7.324)--(7.623,7.324)%
    --(7.623,7.324)--cycle;
\gpfill{color=gp lt color border,opacity=0.50} (9.678,4.155)--(9.677,4.163)--(9.677,4.172)--(9.675,4.180)%
    --(9.674,4.188)--(9.672,4.197)--(9.670,4.205)--(9.667,4.213)--(9.663,4.221)%
    --(9.660,4.229)--(9.656,4.236)--(9.651,4.243)--(9.646,4.250)--(9.641,4.257)%
    --(9.636,4.264)--(9.630,4.270)--(9.624,4.276)--(9.617,4.281)--(9.610,4.286)%
    --(9.603,4.291)--(9.596,4.296)--(9.589,4.300)--(9.581,4.303)--(9.573,4.307)%
    --(9.565,4.310)--(9.557,4.312)--(9.548,4.314)--(9.540,4.315)--(9.532,4.317)%
    --(9.523,4.317)--(9.515,4.318)--(9.506,4.317)--(9.497,4.317)--(9.489,4.315)%
    --(9.481,4.314)--(9.472,4.312)--(9.464,4.310)--(9.456,4.307)--(9.448,4.303)%
    --(9.440,4.300)--(9.433,4.296)--(9.426,4.291)--(9.419,4.286)--(9.412,4.281)%
    --(9.405,4.276)--(9.399,4.270)--(9.393,4.264)--(9.388,4.257)--(9.383,4.250)%
    --(9.378,4.243)--(9.373,4.236)--(9.369,4.229)--(9.366,4.221)--(9.362,4.213)%
    --(9.359,4.205)--(9.357,4.197)--(9.355,4.188)--(9.354,4.180)--(9.352,4.172)%
    --(9.352,4.163)--(9.352,4.155)--(9.352,4.146)--(9.352,4.137)--(9.354,4.129)%
    --(9.355,4.121)--(9.357,4.112)--(9.359,4.104)--(9.362,4.096)--(9.366,4.088)%
    --(9.369,4.080)--(9.373,4.073)--(9.378,4.066)--(9.383,4.059)--(9.388,4.052)%
    --(9.393,4.045)--(9.399,4.039)--(9.405,4.033)--(9.412,4.028)--(9.419,4.023)%
    --(9.426,4.018)--(9.433,4.013)--(9.440,4.009)--(9.448,4.006)--(9.456,4.002)%
    --(9.464,3.999)--(9.472,3.997)--(9.481,3.995)--(9.489,3.994)--(9.497,3.992)%
    --(9.506,3.992)--(9.514,3.992)--(9.523,3.992)--(9.532,3.992)--(9.540,3.994)%
    --(9.548,3.995)--(9.557,3.997)--(9.565,3.999)--(9.573,4.002)--(9.581,4.006)%
    --(9.589,4.009)--(9.596,4.013)--(9.603,4.018)--(9.610,4.023)--(9.617,4.028)%
    --(9.624,4.033)--(9.630,4.039)--(9.636,4.045)--(9.641,4.052)--(9.646,4.059)%
    --(9.651,4.066)--(9.656,4.073)--(9.660,4.080)--(9.663,4.088)--(9.667,4.096)%
    --(9.670,4.104)--(9.672,4.112)--(9.674,4.121)--(9.675,4.129)--(9.677,4.137)%
    --(9.677,4.146)--(9.678,4.154)--cycle;
\gpfill{color=gp lt color border,opacity=0.50} (4.920,7.324)--(4.920,7.324)--(4.920,7.324)--(4.920,7.324)%
    --(4.920,7.324)--(4.920,7.324)--(4.920,7.324)--(4.920,7.324)--(4.920,7.324)%
    --(4.920,7.324)--(4.920,7.324)--(4.920,7.324)--(4.920,7.324)--(4.920,7.324)%
    --(4.920,7.324)--(4.920,7.324)--(4.920,7.324)--(4.920,7.324)--(4.920,7.324)%
    --(4.920,7.324)--(4.920,7.324)--(4.920,7.324)--(4.920,7.324)--(4.920,7.324)%
    --(4.920,7.324)--(4.920,7.324)--(4.920,7.324)--(4.920,7.324)--(4.920,7.324)%
    --(4.920,7.324)--(4.920,7.324)--(4.920,7.324)--(4.920,7.324)--(4.920,7.324)%
    --(4.920,7.324)--(4.920,7.324)--(4.920,7.324)--(4.920,7.324)--(4.920,7.324)%
    --(4.920,7.324)--(4.920,7.324)--(4.920,7.324)--(4.920,7.324)--(4.920,7.324)%
    --(4.920,7.324)--(4.920,7.324)--(4.920,7.324)--(4.920,7.324)--(4.920,7.324)%
    --(4.920,7.324)--(4.920,7.324)--(4.920,7.324)--(4.920,7.324)--(4.920,7.324)%
    --(4.920,7.324)--(4.920,7.324)--(4.920,7.324)--(4.920,7.324)--(4.920,7.324)%
    --(4.920,7.324)--(4.920,7.324)--(4.920,7.324)--(4.920,7.324)--(4.920,7.324)%
    --(4.920,7.324)--(4.920,7.324)--(4.920,7.324)--(4.920,7.324)--(4.920,7.324)%
    --(4.920,7.324)--(4.920,7.324)--(4.920,7.324)--(4.920,7.324)--(4.920,7.324)%
    --(4.920,7.324)--(4.920,7.324)--(4.920,7.324)--(4.920,7.324)--(4.920,7.324)%
    --(4.920,7.324)--(4.920,7.324)--(4.920,7.324)--(4.920,7.324)--(4.920,7.324)%
    --(4.920,7.324)--(4.920,7.324)--(4.920,7.324)--(4.920,7.324)--(4.920,7.324)%
    --(4.920,7.324)--(4.920,7.324)--(4.920,7.324)--(4.920,7.324)--(4.920,7.324)%
    --(4.920,7.324)--(4.920,7.324)--(4.920,7.324)--(4.920,7.324)--(4.920,7.324)%
    --(4.920,7.324)--(4.920,7.324)--(4.920,7.324)--(4.920,7.324)--(4.920,7.324)%
    --(4.920,7.324)--(4.920,7.324)--(4.920,7.324)--(4.920,7.324)--(4.920,7.324)%
    --(4.920,7.324)--(4.920,7.324)--(4.920,7.324)--(4.920,7.324)--(4.920,7.324)%
    --(4.920,7.324)--(4.920,7.324)--(4.920,7.324)--(4.920,7.324)--(4.920,7.324)%
    --(4.920,7.324)--cycle;
\gpfill{color=gp lt color border,opacity=0.50} (3.568,7.324)--(3.568,7.324)--(3.568,7.324)--(3.568,7.324)%
    --(3.568,7.324)--(3.568,7.324)--(3.568,7.324)--(3.568,7.324)--(3.568,7.324)%
    --(3.568,7.324)--(3.568,7.324)--(3.568,7.324)--(3.568,7.324)--(3.568,7.324)%
    --(3.568,7.324)--(3.568,7.324)--(3.568,7.324)--(3.568,7.324)--(3.568,7.324)%
    --(3.568,7.324)--(3.568,7.324)--(3.568,7.324)--(3.568,7.324)--(3.568,7.324)%
    --(3.568,7.324)--(3.568,7.324)--(3.568,7.324)--(3.568,7.324)--(3.568,7.324)%
    --(3.568,7.324)--(3.568,7.324)--(3.568,7.324)--(3.568,7.324)--(3.568,7.324)%
    --(3.568,7.324)--(3.568,7.324)--(3.568,7.324)--(3.568,7.324)--(3.568,7.324)%
    --(3.568,7.324)--(3.568,7.324)--(3.568,7.324)--(3.568,7.324)--(3.568,7.324)%
    --(3.568,7.324)--(3.568,7.324)--(3.568,7.324)--(3.568,7.324)--(3.568,7.324)%
    --(3.568,7.324)--(3.568,7.324)--(3.568,7.324)--(3.568,7.324)--(3.568,7.324)%
    --(3.568,7.324)--(3.568,7.324)--(3.568,7.324)--(3.568,7.324)--(3.568,7.324)%
    --(3.568,7.324)--(3.568,7.324)--(3.568,7.324)--(3.568,7.324)--(3.568,7.324)%
    --(3.568,7.324)--(3.568,7.324)--(3.568,7.324)--(3.568,7.324)--(3.568,7.324)%
    --(3.568,7.324)--(3.568,7.324)--(3.568,7.324)--(3.568,7.324)--(3.568,7.324)%
    --(3.568,7.324)--(3.568,7.324)--(3.568,7.324)--(3.568,7.324)--(3.568,7.324)%
    --(3.568,7.324)--(3.568,7.324)--(3.568,7.324)--(3.568,7.324)--(3.568,7.324)%
    --(3.568,7.324)--(3.568,7.324)--(3.568,7.324)--(3.568,7.324)--(3.568,7.324)%
    --(3.568,7.324)--(3.568,7.324)--(3.568,7.324)--(3.568,7.324)--(3.568,7.324)%
    --(3.568,7.324)--(3.568,7.324)--(3.568,7.324)--(3.568,7.324)--(3.568,7.324)%
    --(3.568,7.324)--(3.568,7.324)--(3.568,7.324)--(3.568,7.324)--(3.568,7.324)%
    --(3.568,7.324)--(3.568,7.324)--(3.568,7.324)--(3.568,7.324)--(3.568,7.324)%
    --(3.568,7.324)--(3.568,7.324)--(3.568,7.324)--(3.568,7.324)--(3.568,7.324)%
    --(3.568,7.324)--(3.568,7.324)--(3.568,7.324)--(3.568,7.324)--(3.568,7.324)%
    --(3.568,7.324)--cycle;
\gpfill{color=gp lt color border,opacity=0.50} (7.083,4.155)--(7.082,4.169)--(7.081,4.183)--(7.079,4.197)%
    --(7.077,4.211)--(7.073,4.225)--(7.069,4.238)--(7.065,4.252)--(7.059,4.265)%
    --(7.053,4.278)--(7.046,4.290)--(7.039,4.302)--(7.031,4.314)--(7.022,4.325)%
    --(7.013,4.336)--(7.003,4.346)--(6.993,4.356)--(6.982,4.365)--(6.971,4.374)%
    --(6.959,4.382)--(6.947,4.389)--(6.935,4.396)--(6.922,4.402)--(6.909,4.408)%
    --(6.895,4.412)--(6.882,4.416)--(6.868,4.420)--(6.854,4.422)--(6.840,4.424)%
    --(6.826,4.425)--(6.812,4.426)--(6.797,4.425)--(6.783,4.424)--(6.769,4.422)%
    --(6.755,4.420)--(6.741,4.416)--(6.728,4.412)--(6.714,4.408)--(6.701,4.402)%
    --(6.688,4.396)--(6.676,4.389)--(6.664,4.382)--(6.652,4.374)--(6.641,4.365)%
    --(6.630,4.356)--(6.620,4.346)--(6.610,4.336)--(6.601,4.325)--(6.592,4.314)%
    --(6.584,4.302)--(6.577,4.290)--(6.570,4.278)--(6.564,4.265)--(6.558,4.252)%
    --(6.554,4.238)--(6.550,4.225)--(6.546,4.211)--(6.544,4.197)--(6.542,4.183)%
    --(6.541,4.169)--(6.541,4.155)--(6.541,4.140)--(6.542,4.126)--(6.544,4.112)%
    --(6.546,4.098)--(6.550,4.084)--(6.554,4.071)--(6.558,4.057)--(6.564,4.044)%
    --(6.570,4.031)--(6.577,4.019)--(6.584,4.007)--(6.592,3.995)--(6.601,3.984)%
    --(6.610,3.973)--(6.620,3.963)--(6.630,3.953)--(6.641,3.944)--(6.652,3.935)%
    --(6.664,3.927)--(6.676,3.920)--(6.688,3.913)--(6.701,3.907)--(6.714,3.901)%
    --(6.728,3.897)--(6.741,3.893)--(6.755,3.889)--(6.769,3.887)--(6.783,3.885)%
    --(6.797,3.884)--(6.811,3.884)--(6.826,3.884)--(6.840,3.885)--(6.854,3.887)%
    --(6.868,3.889)--(6.882,3.893)--(6.895,3.897)--(6.909,3.901)--(6.922,3.907)%
    --(6.935,3.913)--(6.947,3.920)--(6.959,3.927)--(6.971,3.935)--(6.982,3.944)%
    --(6.993,3.953)--(7.003,3.963)--(7.013,3.973)--(7.022,3.984)--(7.031,3.995)%
    --(7.039,4.007)--(7.046,4.019)--(7.053,4.031)--(7.059,4.044)--(7.065,4.057)%
    --(7.069,4.071)--(7.073,4.084)--(7.077,4.098)--(7.079,4.112)--(7.081,4.126)%
    --(7.082,4.140)--(7.083,4.154)--cycle;
\gpfill{color=gp lt color border,opacity=0.50} (9.515,3.098)--(9.515,3.098)--(9.515,3.098)--(9.515,3.098)%
    --(9.515,3.098)--(9.515,3.098)--(9.515,3.098)--(9.515,3.098)--(9.515,3.098)%
    --(9.515,3.098)--(9.515,3.098)--(9.515,3.098)--(9.515,3.098)--(9.515,3.098)%
    --(9.515,3.098)--(9.515,3.098)--(9.515,3.098)--(9.515,3.098)--(9.515,3.098)%
    --(9.515,3.098)--(9.515,3.098)--(9.515,3.098)--(9.515,3.098)--(9.515,3.098)%
    --(9.515,3.098)--(9.515,3.098)--(9.515,3.098)--(9.515,3.098)--(9.515,3.098)%
    --(9.515,3.098)--(9.515,3.098)--(9.515,3.098)--(9.515,3.098)--(9.515,3.098)%
    --(9.515,3.098)--(9.515,3.098)--(9.515,3.098)--(9.515,3.098)--(9.515,3.098)%
    --(9.515,3.098)--(9.515,3.098)--(9.515,3.098)--(9.515,3.098)--(9.515,3.098)%
    --(9.515,3.098)--(9.515,3.098)--(9.515,3.098)--(9.515,3.098)--(9.515,3.098)%
    --(9.515,3.098)--(9.515,3.098)--(9.515,3.098)--(9.515,3.098)--(9.515,3.098)%
    --(9.515,3.098)--(9.515,3.098)--(9.515,3.098)--(9.515,3.098)--(9.515,3.098)%
    --(9.515,3.098)--(9.515,3.098)--(9.515,3.098)--(9.515,3.098)--(9.515,3.098)%
    --(9.515,3.098)--(9.515,3.098)--(9.515,3.098)--(9.515,3.098)--(9.515,3.098)%
    --(9.515,3.098)--(9.515,3.098)--(9.515,3.098)--(9.515,3.098)--(9.515,3.098)%
    --(9.515,3.098)--(9.515,3.098)--(9.515,3.098)--(9.515,3.098)--(9.515,3.098)%
    --(9.515,3.098)--(9.515,3.098)--(9.515,3.098)--(9.515,3.098)--(9.515,3.098)%
    --(9.515,3.098)--(9.515,3.098)--(9.515,3.098)--(9.515,3.098)--(9.515,3.098)%
    --(9.515,3.098)--(9.515,3.098)--(9.515,3.098)--(9.515,3.098)--(9.515,3.098)%
    --(9.515,3.098)--(9.515,3.098)--(9.515,3.098)--(9.515,3.098)--(9.515,3.098)%
    --(9.515,3.098)--(9.515,3.098)--(9.515,3.098)--(9.515,3.098)--(9.515,3.098)%
    --(9.515,3.098)--(9.515,3.098)--(9.515,3.098)--(9.515,3.098)--(9.515,3.098)%
    --(9.515,3.098)--(9.515,3.098)--(9.515,3.098)--(9.515,3.098)--(9.515,3.098)%
    --(9.515,3.098)--(9.515,3.098)--(9.515,3.098)--(9.515,3.098)--(9.515,3.098)%
    --(9.515,3.098)--cycle;
\gpfill{color=gp lt color border,opacity=0.50} (10.406,4.155)--(10.405,4.159)--(10.405,4.163)--(10.405,4.167)%
    --(10.404,4.171)--(10.403,4.175)--(10.402,4.180)--(10.400,4.184)--(10.398,4.187)%
    --(10.397,4.191)--(10.395,4.195)--(10.392,4.199)--(10.390,4.202)--(10.387,4.205)%
    --(10.385,4.209)--(10.382,4.212)--(10.379,4.215)--(10.375,4.217)--(10.372,4.220)%
    --(10.369,4.222)--(10.365,4.225)--(10.361,4.227)--(10.357,4.228)--(10.354,4.230)%
    --(10.350,4.232)--(10.345,4.233)--(10.341,4.234)--(10.337,4.235)--(10.333,4.235)%
    --(10.329,4.235)--(10.325,4.236)--(10.320,4.235)--(10.316,4.235)--(10.312,4.235)%
    --(10.308,4.234)--(10.304,4.233)--(10.299,4.232)--(10.295,4.230)--(10.292,4.228)%
    --(10.288,4.227)--(10.284,4.225)--(10.280,4.222)--(10.277,4.220)--(10.274,4.217)%
    --(10.270,4.215)--(10.267,4.212)--(10.264,4.209)--(10.262,4.205)--(10.259,4.202)%
    --(10.257,4.199)--(10.254,4.195)--(10.252,4.191)--(10.251,4.187)--(10.249,4.184)%
    --(10.247,4.180)--(10.246,4.175)--(10.245,4.171)--(10.244,4.167)--(10.244,4.163)%
    --(10.244,4.159)--(10.244,4.155)--(10.244,4.150)--(10.244,4.146)--(10.244,4.142)%
    --(10.245,4.138)--(10.246,4.134)--(10.247,4.129)--(10.249,4.125)--(10.251,4.122)%
    --(10.252,4.118)--(10.254,4.114)--(10.257,4.110)--(10.259,4.107)--(10.262,4.104)%
    --(10.264,4.100)--(10.267,4.097)--(10.270,4.094)--(10.274,4.092)--(10.277,4.089)%
    --(10.280,4.087)--(10.284,4.084)--(10.288,4.082)--(10.292,4.081)--(10.295,4.079)%
    --(10.299,4.077)--(10.304,4.076)--(10.308,4.075)--(10.312,4.074)--(10.316,4.074)%
    --(10.320,4.074)--(10.324,4.074)--(10.329,4.074)--(10.333,4.074)--(10.337,4.074)%
    --(10.341,4.075)--(10.345,4.076)--(10.350,4.077)--(10.354,4.079)--(10.357,4.081)%
    --(10.361,4.082)--(10.365,4.084)--(10.369,4.087)--(10.372,4.089)--(10.375,4.092)%
    --(10.379,4.094)--(10.382,4.097)--(10.385,4.100)--(10.387,4.104)--(10.390,4.107)%
    --(10.392,4.110)--(10.395,4.114)--(10.397,4.118)--(10.398,4.122)--(10.400,4.125)%
    --(10.402,4.129)--(10.403,4.134)--(10.404,4.138)--(10.405,4.142)--(10.405,4.146)%
    --(10.405,4.150)--(10.406,4.154)--cycle;
\gpfill{color=gp lt color border,opacity=0.50} (4.433,4.155)--(4.432,4.171)--(4.431,4.188)--(4.429,4.205)%
    --(4.425,4.222)--(4.421,4.238)--(4.417,4.255)--(4.411,4.271)--(4.404,4.286)%
    --(4.397,4.302)--(4.389,4.316)--(4.380,4.331)--(4.371,4.345)--(4.360,4.358)%
    --(4.349,4.371)--(4.338,4.384)--(4.325,4.395)--(4.312,4.406)--(4.299,4.417)%
    --(4.285,4.426)--(4.271,4.435)--(4.256,4.443)--(4.240,4.450)--(4.225,4.457)%
    --(4.209,4.463)--(4.192,4.467)--(4.176,4.471)--(4.159,4.475)--(4.142,4.477)%
    --(4.125,4.478)--(4.109,4.479)--(4.092,4.478)--(4.075,4.477)--(4.058,4.475)%
    --(4.041,4.471)--(4.025,4.467)--(4.008,4.463)--(3.992,4.457)--(3.977,4.450)%
    --(3.961,4.443)--(3.947,4.435)--(3.932,4.426)--(3.918,4.417)--(3.905,4.406)%
    --(3.892,4.395)--(3.879,4.384)--(3.868,4.371)--(3.857,4.358)--(3.846,4.345)%
    --(3.837,4.331)--(3.828,4.316)--(3.820,4.302)--(3.813,4.286)--(3.806,4.271)%
    --(3.800,4.255)--(3.796,4.238)--(3.792,4.222)--(3.788,4.205)--(3.786,4.188)%
    --(3.785,4.171)--(3.785,4.155)--(3.785,4.138)--(3.786,4.121)--(3.788,4.104)%
    --(3.792,4.087)--(3.796,4.071)--(3.800,4.054)--(3.806,4.038)--(3.813,4.023)%
    --(3.820,4.007)--(3.828,3.992)--(3.837,3.978)--(3.846,3.964)--(3.857,3.951)%
    --(3.868,3.938)--(3.879,3.925)--(3.892,3.914)--(3.905,3.903)--(3.918,3.892)%
    --(3.932,3.883)--(3.946,3.874)--(3.961,3.866)--(3.977,3.859)--(3.992,3.852)%
    --(4.008,3.846)--(4.025,3.842)--(4.041,3.838)--(4.058,3.834)--(4.075,3.832)%
    --(4.092,3.831)--(4.108,3.831)--(4.125,3.831)--(4.142,3.832)--(4.159,3.834)%
    --(4.176,3.838)--(4.192,3.842)--(4.209,3.846)--(4.225,3.852)--(4.240,3.859)%
    --(4.256,3.866)--(4.271,3.874)--(4.285,3.883)--(4.299,3.892)--(4.312,3.903)%
    --(4.325,3.914)--(4.338,3.925)--(4.349,3.938)--(4.360,3.951)--(4.371,3.964)%
    --(4.380,3.978)--(4.389,3.992)--(4.397,4.007)--(4.404,4.023)--(4.411,4.038)%
    --(4.417,4.054)--(4.421,4.071)--(4.425,4.087)--(4.429,4.104)--(4.431,4.121)%
    --(4.432,4.138)--(4.433,4.154)--cycle;
\gpfill{color=gp lt color border,opacity=0.50} (6.839,3.098)--(6.838,3.099)--(6.838,3.100)--(6.838,3.102)%
    --(6.838,3.103)--(6.838,3.104)--(6.837,3.106)--(6.837,3.107)--(6.836,3.108)%
    --(6.836,3.110)--(6.835,3.111)--(6.834,3.112)--(6.833,3.113)--(6.832,3.114)%
    --(6.832,3.116)--(6.831,3.117)--(6.830,3.118)--(6.828,3.118)--(6.827,3.119)%
    --(6.826,3.120)--(6.825,3.121)--(6.824,3.122)--(6.822,3.122)--(6.821,3.123)%
    --(6.820,3.123)--(6.818,3.124)--(6.817,3.124)--(6.816,3.124)--(6.814,3.124)%
    --(6.813,3.124)--(6.812,3.125)--(6.810,3.124)--(6.809,3.124)--(6.807,3.124)%
    --(6.806,3.124)--(6.805,3.124)--(6.803,3.123)--(6.802,3.123)--(6.801,3.122)%
    --(6.799,3.122)--(6.798,3.121)--(6.797,3.120)--(6.796,3.119)--(6.795,3.118)%
    --(6.793,3.118)--(6.792,3.117)--(6.791,3.116)--(6.791,3.114)--(6.790,3.113)%
    --(6.789,3.112)--(6.788,3.111)--(6.787,3.110)--(6.787,3.108)--(6.786,3.107)%
    --(6.786,3.106)--(6.785,3.104)--(6.785,3.103)--(6.785,3.102)--(6.785,3.100)%
    --(6.785,3.099)--(6.785,3.098)--(6.785,3.096)--(6.785,3.095)--(6.785,3.093)%
    --(6.785,3.092)--(6.785,3.091)--(6.786,3.089)--(6.786,3.088)--(6.787,3.087)%
    --(6.787,3.085)--(6.788,3.084)--(6.789,3.083)--(6.790,3.082)--(6.791,3.081)%
    --(6.791,3.079)--(6.792,3.078)--(6.793,3.077)--(6.795,3.077)--(6.796,3.076)%
    --(6.797,3.075)--(6.798,3.074)--(6.799,3.073)--(6.801,3.073)--(6.802,3.072)%
    --(6.803,3.072)--(6.805,3.071)--(6.806,3.071)--(6.807,3.071)--(6.809,3.071)%
    --(6.810,3.071)--(6.811,3.071)--(6.813,3.071)--(6.814,3.071)--(6.816,3.071)%
    --(6.817,3.071)--(6.818,3.071)--(6.820,3.072)--(6.821,3.072)--(6.822,3.073)%
    --(6.824,3.073)--(6.825,3.074)--(6.826,3.075)--(6.827,3.076)--(6.828,3.077)%
    --(6.830,3.077)--(6.831,3.078)--(6.832,3.079)--(6.832,3.081)--(6.833,3.082)%
    --(6.834,3.083)--(6.835,3.084)--(6.836,3.085)--(6.836,3.087)--(6.837,3.088)%
    --(6.837,3.089)--(6.838,3.091)--(6.838,3.092)--(6.838,3.093)--(6.838,3.095)%
    --(6.838,3.096)--(6.839,3.097)--cycle;
\gpfill{color=gp lt color border,opacity=0.50} (9.543,6.268)--(9.542,6.269)--(9.542,6.270)--(9.542,6.272)%
    --(9.542,6.273)--(9.542,6.275)--(9.541,6.276)--(9.541,6.278)--(9.540,6.279)%
    --(9.539,6.280)--(9.539,6.281)--(9.538,6.283)--(9.537,6.284)--(9.536,6.285)%
    --(9.535,6.286)--(9.534,6.287)--(9.533,6.288)--(9.532,6.289)--(9.531,6.290)%
    --(9.530,6.291)--(9.529,6.292)--(9.527,6.292)--(9.526,6.293)--(9.525,6.294)%
    --(9.523,6.294)--(9.522,6.295)--(9.520,6.295)--(9.519,6.295)--(9.517,6.295)%
    --(9.516,6.295)--(9.515,6.296)--(9.513,6.295)--(9.512,6.295)--(9.510,6.295)%
    --(9.509,6.295)--(9.507,6.295)--(9.506,6.294)--(9.504,6.294)--(9.503,6.293)%
    --(9.502,6.292)--(9.501,6.292)--(9.499,6.291)--(9.498,6.290)--(9.497,6.289)%
    --(9.496,6.288)--(9.495,6.287)--(9.494,6.286)--(9.493,6.285)--(9.492,6.284)%
    --(9.491,6.283)--(9.490,6.281)--(9.490,6.280)--(9.489,6.279)--(9.488,6.278)%
    --(9.488,6.276)--(9.487,6.275)--(9.487,6.273)--(9.487,6.272)--(9.487,6.270)%
    --(9.487,6.269)--(9.487,6.268)--(9.487,6.266)--(9.487,6.265)--(9.487,6.263)%
    --(9.487,6.262)--(9.487,6.260)--(9.488,6.259)--(9.488,6.257)--(9.489,6.256)%
    --(9.490,6.255)--(9.490,6.253)--(9.491,6.252)--(9.492,6.251)--(9.493,6.250)%
    --(9.494,6.249)--(9.495,6.248)--(9.496,6.247)--(9.497,6.246)--(9.498,6.245)%
    --(9.499,6.244)--(9.500,6.243)--(9.502,6.243)--(9.503,6.242)--(9.504,6.241)%
    --(9.506,6.241)--(9.507,6.240)--(9.509,6.240)--(9.510,6.240)--(9.512,6.240)%
    --(9.513,6.240)--(9.514,6.240)--(9.516,6.240)--(9.517,6.240)--(9.519,6.240)%
    --(9.520,6.240)--(9.522,6.240)--(9.523,6.241)--(9.525,6.241)--(9.526,6.242)%
    --(9.527,6.243)--(9.529,6.243)--(9.530,6.244)--(9.531,6.245)--(9.532,6.246)%
    --(9.533,6.247)--(9.534,6.248)--(9.535,6.249)--(9.536,6.250)--(9.537,6.251)%
    --(9.538,6.252)--(9.539,6.253)--(9.539,6.255)--(9.540,6.256)--(9.541,6.257)%
    --(9.541,6.259)--(9.542,6.260)--(9.542,6.262)--(9.542,6.263)--(9.542,6.265)%
    --(9.542,6.266)--(9.543,6.267)--cycle;
\gpfill{color=gp lt color border,opacity=0.50} (7.813,4.155)--(7.812,4.164)--(7.811,4.174)--(7.810,4.184)%
    --(7.808,4.194)--(7.806,4.204)--(7.803,4.213)--(7.800,4.223)--(7.796,4.232)%
    --(7.792,4.241)--(7.787,4.249)--(7.782,4.258)--(7.776,4.266)--(7.770,4.274)%
    --(7.764,4.282)--(7.757,4.289)--(7.750,4.296)--(7.742,4.302)--(7.734,4.308)%
    --(7.726,4.314)--(7.718,4.319)--(7.709,4.324)--(7.700,4.328)--(7.691,4.332)%
    --(7.681,4.335)--(7.672,4.338)--(7.662,4.340)--(7.652,4.342)--(7.642,4.343)%
    --(7.632,4.344)--(7.623,4.345)--(7.613,4.344)--(7.603,4.343)--(7.593,4.342)%
    --(7.583,4.340)--(7.573,4.338)--(7.564,4.335)--(7.554,4.332)--(7.545,4.328)%
    --(7.536,4.324)--(7.528,4.319)--(7.519,4.314)--(7.511,4.308)--(7.503,4.302)%
    --(7.495,4.296)--(7.488,4.289)--(7.481,4.282)--(7.475,4.274)--(7.469,4.266)%
    --(7.463,4.258)--(7.458,4.249)--(7.453,4.241)--(7.449,4.232)--(7.445,4.223)%
    --(7.442,4.213)--(7.439,4.204)--(7.437,4.194)--(7.435,4.184)--(7.434,4.174)%
    --(7.433,4.164)--(7.433,4.155)--(7.433,4.145)--(7.434,4.135)--(7.435,4.125)%
    --(7.437,4.115)--(7.439,4.105)--(7.442,4.096)--(7.445,4.086)--(7.449,4.077)%
    --(7.453,4.068)--(7.458,4.059)--(7.463,4.051)--(7.469,4.043)--(7.475,4.035)%
    --(7.481,4.027)--(7.488,4.020)--(7.495,4.013)--(7.503,4.007)--(7.511,4.001)%
    --(7.519,3.995)--(7.527,3.990)--(7.536,3.985)--(7.545,3.981)--(7.554,3.977)%
    --(7.564,3.974)--(7.573,3.971)--(7.583,3.969)--(7.593,3.967)--(7.603,3.966)%
    --(7.613,3.965)--(7.622,3.965)--(7.632,3.965)--(7.642,3.966)--(7.652,3.967)%
    --(7.662,3.969)--(7.672,3.971)--(7.681,3.974)--(7.691,3.977)--(7.700,3.981)%
    --(7.709,3.985)--(7.718,3.990)--(7.726,3.995)--(7.734,4.001)--(7.742,4.007)%
    --(7.750,4.013)--(7.757,4.020)--(7.764,4.027)--(7.770,4.035)--(7.776,4.043)%
    --(7.782,4.051)--(7.787,4.059)--(7.792,4.068)--(7.796,4.077)--(7.800,4.086)%
    --(7.803,4.096)--(7.806,4.105)--(7.808,4.115)--(7.810,4.125)--(7.811,4.135)%
    --(7.812,4.145)--(7.813,4.154)--cycle;
\gpfill{color=gp lt color border,opacity=0.50} (10.325,3.098)--(10.325,3.098)--(10.325,3.098)--(10.325,3.098)%
    --(10.325,3.098)--(10.325,3.098)--(10.325,3.098)--(10.325,3.098)--(10.325,3.098)%
    --(10.325,3.098)--(10.325,3.098)--(10.325,3.098)--(10.325,3.098)--(10.325,3.098)%
    --(10.325,3.098)--(10.325,3.098)--(10.325,3.098)--(10.325,3.098)--(10.325,3.098)%
    --(10.325,3.098)--(10.325,3.098)--(10.325,3.098)--(10.325,3.098)--(10.325,3.098)%
    --(10.325,3.098)--(10.325,3.098)--(10.325,3.098)--(10.325,3.098)--(10.325,3.098)%
    --(10.325,3.098)--(10.325,3.098)--(10.325,3.098)--(10.325,3.098)--(10.325,3.098)%
    --(10.325,3.098)--(10.325,3.098)--(10.325,3.098)--(10.325,3.098)--(10.325,3.098)%
    --(10.325,3.098)--(10.325,3.098)--(10.325,3.098)--(10.325,3.098)--(10.325,3.098)%
    --(10.325,3.098)--(10.325,3.098)--(10.325,3.098)--(10.325,3.098)--(10.325,3.098)%
    --(10.325,3.098)--(10.325,3.098)--(10.325,3.098)--(10.325,3.098)--(10.325,3.098)%
    --(10.325,3.098)--(10.325,3.098)--(10.325,3.098)--(10.325,3.098)--(10.325,3.098)%
    --(10.325,3.098)--(10.325,3.098)--(10.325,3.098)--(10.325,3.098)--(10.325,3.098)%
    --(10.325,3.098)--(10.325,3.098)--(10.325,3.098)--(10.325,3.098)--(10.325,3.098)%
    --(10.325,3.098)--(10.325,3.098)--(10.325,3.098)--(10.325,3.098)--(10.325,3.098)%
    --(10.325,3.098)--(10.325,3.098)--(10.325,3.098)--(10.325,3.098)--(10.325,3.098)%
    --(10.325,3.098)--(10.325,3.098)--(10.325,3.098)--(10.325,3.098)--(10.325,3.098)%
    --(10.325,3.098)--(10.325,3.098)--(10.325,3.098)--(10.325,3.098)--(10.325,3.098)%
    --(10.325,3.098)--(10.325,3.098)--(10.325,3.098)--(10.325,3.098)--(10.325,3.098)%
    --(10.325,3.098)--(10.325,3.098)--(10.325,3.098)--(10.325,3.098)--(10.325,3.098)%
    --(10.325,3.098)--(10.325,3.098)--(10.325,3.098)--(10.325,3.098)--(10.325,3.098)%
    --(10.325,3.098)--(10.325,3.098)--(10.325,3.098)--(10.325,3.098)--(10.325,3.098)%
    --(10.325,3.098)--(10.325,3.098)--(10.325,3.098)--(10.325,3.098)--(10.325,3.098)%
    --(10.325,3.098)--(10.325,3.098)--(10.325,3.098)--(10.325,3.098)--(10.325,3.098)%
    --(10.325,3.098)--cycle;
\gpfill{color=gp lt color border,opacity=0.50} (8.217,6.268)--(8.216,6.270)--(8.216,6.273)--(8.216,6.276)%
    --(8.215,6.279)--(8.215,6.281)--(8.214,6.284)--(8.213,6.287)--(8.212,6.289)%
    --(8.211,6.292)--(8.209,6.294)--(8.208,6.297)--(8.206,6.299)--(8.204,6.301)%
    --(8.203,6.304)--(8.201,6.306)--(8.199,6.308)--(8.196,6.309)--(8.194,6.311)%
    --(8.192,6.313)--(8.190,6.314)--(8.187,6.316)--(8.184,6.317)--(8.182,6.318)%
    --(8.179,6.319)--(8.176,6.320)--(8.174,6.320)--(8.171,6.321)--(8.168,6.321)%
    --(8.165,6.321)--(8.163,6.322)--(8.160,6.321)--(8.157,6.321)--(8.154,6.321)%
    --(8.151,6.320)--(8.149,6.320)--(8.146,6.319)--(8.143,6.318)--(8.141,6.317)%
    --(8.138,6.316)--(8.136,6.314)--(8.133,6.313)--(8.131,6.311)--(8.129,6.309)%
    --(8.126,6.308)--(8.124,6.306)--(8.122,6.304)--(8.121,6.301)--(8.119,6.299)%
    --(8.117,6.297)--(8.116,6.294)--(8.114,6.292)--(8.113,6.289)--(8.112,6.287)%
    --(8.111,6.284)--(8.110,6.281)--(8.110,6.279)--(8.109,6.276)--(8.109,6.273)%
    --(8.109,6.270)--(8.109,6.268)--(8.109,6.265)--(8.109,6.262)--(8.109,6.259)%
    --(8.110,6.256)--(8.110,6.254)--(8.111,6.251)--(8.112,6.248)--(8.113,6.246)%
    --(8.114,6.243)--(8.116,6.240)--(8.117,6.238)--(8.119,6.236)--(8.121,6.234)%
    --(8.122,6.231)--(8.124,6.229)--(8.126,6.227)--(8.129,6.226)--(8.131,6.224)%
    --(8.133,6.222)--(8.135,6.221)--(8.138,6.219)--(8.141,6.218)--(8.143,6.217)%
    --(8.146,6.216)--(8.149,6.215)--(8.151,6.215)--(8.154,6.214)--(8.157,6.214)%
    --(8.160,6.214)--(8.162,6.214)--(8.165,6.214)--(8.168,6.214)--(8.171,6.214)%
    --(8.174,6.215)--(8.176,6.215)--(8.179,6.216)--(8.182,6.217)--(8.184,6.218)%
    --(8.187,6.219)--(8.190,6.221)--(8.192,6.222)--(8.194,6.224)--(8.196,6.226)%
    --(8.199,6.227)--(8.201,6.229)--(8.203,6.231)--(8.204,6.234)--(8.206,6.236)%
    --(8.208,6.238)--(8.209,6.240)--(8.211,6.243)--(8.212,6.246)--(8.213,6.248)%
    --(8.214,6.251)--(8.215,6.254)--(8.215,6.256)--(8.216,6.259)--(8.216,6.262)%
    --(8.216,6.265)--(8.217,6.267)--cycle;
\gpfill{color=gp lt color border,opacity=0.50} (10.920,5.211)--(10.919,5.213)--(10.919,5.216)--(10.919,5.219)%
    --(10.918,5.222)--(10.918,5.224)--(10.917,5.227)--(10.916,5.230)--(10.915,5.232)%
    --(10.914,5.235)--(10.912,5.237)--(10.911,5.240)--(10.909,5.242)--(10.907,5.244)%
    --(10.906,5.247)--(10.904,5.249)--(10.902,5.251)--(10.899,5.252)--(10.897,5.254)%
    --(10.895,5.256)--(10.893,5.257)--(10.890,5.259)--(10.887,5.260)--(10.885,5.261)%
    --(10.882,5.262)--(10.879,5.263)--(10.877,5.263)--(10.874,5.264)--(10.871,5.264)%
    --(10.868,5.264)--(10.866,5.265)--(10.863,5.264)--(10.860,5.264)--(10.857,5.264)%
    --(10.854,5.263)--(10.852,5.263)--(10.849,5.262)--(10.846,5.261)--(10.844,5.260)%
    --(10.841,5.259)--(10.839,5.257)--(10.836,5.256)--(10.834,5.254)--(10.832,5.252)%
    --(10.829,5.251)--(10.827,5.249)--(10.825,5.247)--(10.824,5.244)--(10.822,5.242)%
    --(10.820,5.240)--(10.819,5.237)--(10.817,5.235)--(10.816,5.232)--(10.815,5.230)%
    --(10.814,5.227)--(10.813,5.224)--(10.813,5.222)--(10.812,5.219)--(10.812,5.216)%
    --(10.812,5.213)--(10.812,5.211)--(10.812,5.208)--(10.812,5.205)--(10.812,5.202)%
    --(10.813,5.199)--(10.813,5.197)--(10.814,5.194)--(10.815,5.191)--(10.816,5.189)%
    --(10.817,5.186)--(10.819,5.183)--(10.820,5.181)--(10.822,5.179)--(10.824,5.177)%
    --(10.825,5.174)--(10.827,5.172)--(10.829,5.170)--(10.832,5.169)--(10.834,5.167)%
    --(10.836,5.165)--(10.838,5.164)--(10.841,5.162)--(10.844,5.161)--(10.846,5.160)%
    --(10.849,5.159)--(10.852,5.158)--(10.854,5.158)--(10.857,5.157)--(10.860,5.157)%
    --(10.863,5.157)--(10.865,5.157)--(10.868,5.157)--(10.871,5.157)--(10.874,5.157)%
    --(10.877,5.158)--(10.879,5.158)--(10.882,5.159)--(10.885,5.160)--(10.887,5.161)%
    --(10.890,5.162)--(10.893,5.164)--(10.895,5.165)--(10.897,5.167)--(10.899,5.169)%
    --(10.902,5.170)--(10.904,5.172)--(10.906,5.174)--(10.907,5.177)--(10.909,5.179)%
    --(10.911,5.181)--(10.912,5.183)--(10.914,5.186)--(10.915,5.189)--(10.916,5.191)%
    --(10.917,5.194)--(10.918,5.197)--(10.918,5.199)--(10.919,5.202)--(10.919,5.205)%
    --(10.919,5.208)--(10.920,5.210)--cycle;
\gpfill{color=gp lt color border,opacity=0.50} (4.298,3.098)--(4.297,3.107)--(4.296,3.117)--(4.295,3.127)%
    --(4.293,3.137)--(4.291,3.146)--(4.288,3.156)--(4.285,3.165)--(4.281,3.174)%
    --(4.277,3.183)--(4.272,3.192)--(4.267,3.200)--(4.261,3.209)--(4.255,3.216)%
    --(4.249,3.224)--(4.242,3.231)--(4.235,3.238)--(4.227,3.244)--(4.220,3.250)%
    --(4.211,3.256)--(4.203,3.261)--(4.194,3.266)--(4.185,3.270)--(4.176,3.274)%
    --(4.167,3.277)--(4.157,3.280)--(4.148,3.282)--(4.138,3.284)--(4.128,3.285)%
    --(4.118,3.286)--(4.109,3.287)--(4.099,3.286)--(4.089,3.285)--(4.079,3.284)%
    --(4.069,3.282)--(4.060,3.280)--(4.050,3.277)--(4.041,3.274)--(4.032,3.270)%
    --(4.023,3.266)--(4.014,3.261)--(4.006,3.256)--(3.997,3.250)--(3.990,3.244)%
    --(3.982,3.238)--(3.975,3.231)--(3.968,3.224)--(3.962,3.216)--(3.956,3.209)%
    --(3.950,3.200)--(3.945,3.192)--(3.940,3.183)--(3.936,3.174)--(3.932,3.165)%
    --(3.929,3.156)--(3.926,3.146)--(3.924,3.137)--(3.922,3.127)--(3.921,3.117)%
    --(3.920,3.107)--(3.920,3.098)--(3.920,3.088)--(3.921,3.078)--(3.922,3.068)%
    --(3.924,3.058)--(3.926,3.049)--(3.929,3.039)--(3.932,3.030)--(3.936,3.021)%
    --(3.940,3.012)--(3.945,3.003)--(3.950,2.995)--(3.956,2.986)--(3.962,2.979)%
    --(3.968,2.971)--(3.975,2.964)--(3.982,2.957)--(3.990,2.951)--(3.997,2.945)%
    --(4.006,2.939)--(4.014,2.934)--(4.023,2.929)--(4.032,2.925)--(4.041,2.921)%
    --(4.050,2.918)--(4.060,2.915)--(4.069,2.913)--(4.079,2.911)--(4.089,2.910)%
    --(4.099,2.909)--(4.108,2.909)--(4.118,2.909)--(4.128,2.910)--(4.138,2.911)%
    --(4.148,2.913)--(4.157,2.915)--(4.167,2.918)--(4.176,2.921)--(4.185,2.925)%
    --(4.194,2.929)--(4.203,2.934)--(4.211,2.939)--(4.220,2.945)--(4.227,2.951)%
    --(4.235,2.957)--(4.242,2.964)--(4.249,2.971)--(4.255,2.979)--(4.261,2.986)%
    --(4.267,2.995)--(4.272,3.003)--(4.277,3.012)--(4.281,3.021)--(4.285,3.030)%
    --(4.288,3.039)--(4.291,3.049)--(4.293,3.058)--(4.295,3.068)--(4.296,3.078)%
    --(4.297,3.088)--(4.298,3.097)--cycle;
\gpfill{color=gp lt color border,opacity=0.50} (6.947,6.268)--(6.946,6.275)--(6.946,6.282)--(6.945,6.289)%
    --(6.944,6.296)--(6.942,6.302)--(6.940,6.309)--(6.938,6.316)--(6.935,6.322)%
    --(6.932,6.329)--(6.928,6.335)--(6.925,6.341)--(6.921,6.347)--(6.916,6.352)%
    --(6.912,6.358)--(6.907,6.363)--(6.902,6.368)--(6.896,6.372)--(6.891,6.377)%
    --(6.885,6.381)--(6.879,6.384)--(6.873,6.388)--(6.866,6.391)--(6.860,6.394)%
    --(6.853,6.396)--(6.846,6.398)--(6.840,6.400)--(6.833,6.401)--(6.826,6.402)%
    --(6.819,6.402)--(6.812,6.403)--(6.804,6.402)--(6.797,6.402)--(6.790,6.401)%
    --(6.783,6.400)--(6.777,6.398)--(6.770,6.396)--(6.763,6.394)--(6.757,6.391)%
    --(6.750,6.388)--(6.744,6.384)--(6.738,6.381)--(6.732,6.377)--(6.727,6.372)%
    --(6.721,6.368)--(6.716,6.363)--(6.711,6.358)--(6.707,6.352)--(6.702,6.347)%
    --(6.698,6.341)--(6.695,6.335)--(6.691,6.329)--(6.688,6.322)--(6.685,6.316)%
    --(6.683,6.309)--(6.681,6.302)--(6.679,6.296)--(6.678,6.289)--(6.677,6.282)%
    --(6.677,6.275)--(6.677,6.268)--(6.677,6.260)--(6.677,6.253)--(6.678,6.246)%
    --(6.679,6.239)--(6.681,6.233)--(6.683,6.226)--(6.685,6.219)--(6.688,6.213)%
    --(6.691,6.206)--(6.695,6.200)--(6.698,6.194)--(6.702,6.188)--(6.707,6.183)%
    --(6.711,6.177)--(6.716,6.172)--(6.721,6.167)--(6.727,6.163)--(6.732,6.158)%
    --(6.738,6.154)--(6.744,6.151)--(6.750,6.147)--(6.757,6.144)--(6.763,6.141)%
    --(6.770,6.139)--(6.777,6.137)--(6.783,6.135)--(6.790,6.134)--(6.797,6.133)%
    --(6.804,6.133)--(6.811,6.133)--(6.819,6.133)--(6.826,6.133)--(6.833,6.134)%
    --(6.840,6.135)--(6.846,6.137)--(6.853,6.139)--(6.860,6.141)--(6.866,6.144)%
    --(6.873,6.147)--(6.879,6.151)--(6.885,6.154)--(6.891,6.158)--(6.896,6.163)%
    --(6.902,6.167)--(6.907,6.172)--(6.912,6.177)--(6.916,6.183)--(6.921,6.188)%
    --(6.925,6.194)--(6.928,6.200)--(6.932,6.206)--(6.935,6.213)--(6.938,6.219)%
    --(6.940,6.226)--(6.942,6.233)--(6.944,6.239)--(6.945,6.246)--(6.946,6.253)%
    --(6.946,6.260)--(6.947,6.267)--cycle;
\gpfill{color=gp lt color border,opacity=0.50} (9.651,5.211)--(9.650,5.218)--(9.650,5.225)--(9.649,5.232)%
    --(9.648,5.239)--(9.646,5.246)--(9.644,5.253)--(9.641,5.259)--(9.639,5.266)%
    --(9.636,5.272)--(9.632,5.278)--(9.629,5.285)--(9.625,5.290)--(9.620,5.296)%
    --(9.616,5.302)--(9.611,5.307)--(9.606,5.312)--(9.600,5.316)--(9.594,5.321)%
    --(9.589,5.325)--(9.583,5.328)--(9.576,5.332)--(9.570,5.335)--(9.563,5.337)%
    --(9.557,5.340)--(9.550,5.342)--(9.543,5.344)--(9.536,5.345)--(9.529,5.346)%
    --(9.522,5.346)--(9.515,5.347)--(9.507,5.346)--(9.500,5.346)--(9.493,5.345)%
    --(9.486,5.344)--(9.479,5.342)--(9.472,5.340)--(9.466,5.337)--(9.459,5.335)%
    --(9.453,5.332)--(9.447,5.328)--(9.440,5.325)--(9.435,5.321)--(9.429,5.316)%
    --(9.423,5.312)--(9.418,5.307)--(9.413,5.302)--(9.409,5.296)--(9.404,5.290)%
    --(9.400,5.285)--(9.397,5.278)--(9.393,5.272)--(9.390,5.266)--(9.388,5.259)%
    --(9.385,5.253)--(9.383,5.246)--(9.381,5.239)--(9.380,5.232)--(9.379,5.225)%
    --(9.379,5.218)--(9.379,5.211)--(9.379,5.203)--(9.379,5.196)--(9.380,5.189)%
    --(9.381,5.182)--(9.383,5.175)--(9.385,5.168)--(9.388,5.162)--(9.390,5.155)%
    --(9.393,5.149)--(9.397,5.142)--(9.400,5.136)--(9.404,5.131)--(9.409,5.125)%
    --(9.413,5.119)--(9.418,5.114)--(9.423,5.109)--(9.429,5.105)--(9.435,5.100)%
    --(9.440,5.096)--(9.446,5.093)--(9.453,5.089)--(9.459,5.086)--(9.466,5.084)%
    --(9.472,5.081)--(9.479,5.079)--(9.486,5.077)--(9.493,5.076)--(9.500,5.075)%
    --(9.507,5.075)--(9.514,5.075)--(9.522,5.075)--(9.529,5.075)--(9.536,5.076)%
    --(9.543,5.077)--(9.550,5.079)--(9.557,5.081)--(9.563,5.084)--(9.570,5.086)%
    --(9.576,5.089)--(9.583,5.093)--(9.589,5.096)--(9.594,5.100)--(9.600,5.105)%
    --(9.606,5.109)--(9.611,5.114)--(9.616,5.119)--(9.620,5.125)--(9.625,5.131)%
    --(9.629,5.136)--(9.632,5.142)--(9.636,5.149)--(9.639,5.155)--(9.641,5.162)%
    --(9.644,5.168)--(9.646,5.175)--(9.648,5.182)--(9.649,5.189)--(9.650,5.196)%
    --(9.650,5.203)--(9.651,5.210)--cycle;
\gpfill{color=gp lt color border,opacity=0.50} (5.217,4.155)--(5.216,4.170)--(5.215,4.186)--(5.213,4.201)%
    --(5.210,4.216)--(5.206,4.231)--(5.202,4.246)--(5.197,4.261)--(5.191,4.275)%
    --(5.184,4.289)--(5.177,4.303)--(5.169,4.316)--(5.160,4.329)--(5.150,4.341)%
    --(5.140,4.353)--(5.130,4.365)--(5.118,4.375)--(5.106,4.385)--(5.094,4.395)%
    --(5.081,4.404)--(5.068,4.412)--(5.054,4.419)--(5.040,4.426)--(5.026,4.432)%
    --(5.011,4.437)--(4.996,4.441)--(4.981,4.445)--(4.966,4.448)--(4.951,4.450)%
    --(4.935,4.451)--(4.920,4.452)--(4.904,4.451)--(4.888,4.450)--(4.873,4.448)%
    --(4.858,4.445)--(4.843,4.441)--(4.828,4.437)--(4.813,4.432)--(4.799,4.426)%
    --(4.785,4.419)--(4.771,4.412)--(4.758,4.404)--(4.745,4.395)--(4.733,4.385)%
    --(4.721,4.375)--(4.709,4.365)--(4.699,4.353)--(4.689,4.341)--(4.679,4.329)%
    --(4.670,4.316)--(4.662,4.303)--(4.655,4.289)--(4.648,4.275)--(4.642,4.261)%
    --(4.637,4.246)--(4.633,4.231)--(4.629,4.216)--(4.626,4.201)--(4.624,4.186)%
    --(4.623,4.170)--(4.623,4.155)--(4.623,4.139)--(4.624,4.123)--(4.626,4.108)%
    --(4.629,4.093)--(4.633,4.078)--(4.637,4.063)--(4.642,4.048)--(4.648,4.034)%
    --(4.655,4.020)--(4.662,4.006)--(4.670,3.993)--(4.679,3.980)--(4.689,3.968)%
    --(4.699,3.956)--(4.709,3.944)--(4.721,3.934)--(4.733,3.924)--(4.745,3.914)%
    --(4.758,3.905)--(4.771,3.897)--(4.785,3.890)--(4.799,3.883)--(4.813,3.877)%
    --(4.828,3.872)--(4.843,3.868)--(4.858,3.864)--(4.873,3.861)--(4.888,3.859)%
    --(4.904,3.858)--(4.919,3.858)--(4.935,3.858)--(4.951,3.859)--(4.966,3.861)%
    --(4.981,3.864)--(4.996,3.868)--(5.011,3.872)--(5.026,3.877)--(5.040,3.883)%
    --(5.054,3.890)--(5.068,3.897)--(5.081,3.905)--(5.094,3.914)--(5.106,3.924)%
    --(5.118,3.934)--(5.130,3.944)--(5.140,3.956)--(5.150,3.968)--(5.160,3.980)%
    --(5.169,3.993)--(5.177,4.006)--(5.184,4.020)--(5.191,4.034)--(5.197,4.048)%
    --(5.202,4.063)--(5.206,4.078)--(5.210,4.093)--(5.213,4.108)--(5.215,4.123)%
    --(5.216,4.139)--(5.217,4.154)--cycle;
\gpfill{color=gp lt color border,opacity=0.50} (1.865,4.155)--(1.864,4.179)--(1.862,4.202)--(1.859,4.226)%
    --(1.854,4.250)--(1.849,4.273)--(1.842,4.296)--(1.834,4.319)--(1.825,4.341)%
    --(1.814,4.363)--(1.803,4.384)--(1.790,4.404)--(1.777,4.424)--(1.762,4.443)%
    --(1.747,4.462)--(1.730,4.479)--(1.713,4.496)--(1.694,4.511)--(1.675,4.526)%
    --(1.655,4.539)--(1.635,4.552)--(1.614,4.563)--(1.592,4.574)--(1.570,4.583)%
    --(1.547,4.591)--(1.524,4.598)--(1.501,4.603)--(1.477,4.608)--(1.453,4.611)%
    --(1.430,4.613)--(1.406,4.614)--(1.381,4.613)--(1.358,4.611)--(1.334,4.608)%
    --(1.310,4.603)--(1.287,4.598)--(1.264,4.591)--(1.241,4.583)--(1.219,4.574)%
    --(1.197,4.563)--(1.176,4.552)--(1.156,4.539)--(1.136,4.526)--(1.117,4.511)%
    --(1.098,4.496)--(1.081,4.479)--(1.064,4.462)--(1.049,4.443)--(1.034,4.424)%
    --(1.021,4.404)--(1.008,4.384)--(0.997,4.363)--(0.986,4.341)--(0.977,4.319)%
    --(0.969,4.296)--(0.962,4.273)--(0.957,4.250)--(0.952,4.226)--(0.949,4.202)%
    --(0.947,4.179)--(0.947,4.155)--(0.947,4.130)--(0.949,4.107)--(0.952,4.083)%
    --(0.957,4.059)--(0.962,4.036)--(0.969,4.013)--(0.977,3.990)--(0.986,3.968)%
    --(0.997,3.946)--(1.008,3.925)--(1.021,3.905)--(1.034,3.885)--(1.049,3.866)%
    --(1.064,3.847)--(1.081,3.830)--(1.098,3.813)--(1.117,3.798)--(1.136,3.783)%
    --(1.156,3.770)--(1.176,3.757)--(1.197,3.746)--(1.219,3.735)--(1.241,3.726)%
    --(1.264,3.718)--(1.287,3.711)--(1.310,3.706)--(1.334,3.701)--(1.358,3.698)%
    --(1.381,3.696)--(1.405,3.696)--(1.430,3.696)--(1.453,3.698)--(1.477,3.701)%
    --(1.501,3.706)--(1.524,3.711)--(1.547,3.718)--(1.570,3.726)--(1.592,3.735)%
    --(1.614,3.746)--(1.635,3.757)--(1.655,3.770)--(1.675,3.783)--(1.694,3.798)%
    --(1.713,3.813)--(1.730,3.830)--(1.747,3.847)--(1.762,3.866)--(1.777,3.885)%
    --(1.790,3.905)--(1.803,3.925)--(1.814,3.946)--(1.825,3.968)--(1.834,3.990)%
    --(1.842,4.013)--(1.849,4.036)--(1.854,4.059)--(1.859,4.083)--(1.862,4.107)%
    --(1.864,4.130)--(1.865,4.154)--cycle;
\gpfill{color=gp lt color border,opacity=0.50} (5.649,6.268)--(5.648,6.277)--(5.647,6.287)--(5.646,6.297)%
    --(5.644,6.307)--(5.642,6.316)--(5.639,6.326)--(5.636,6.335)--(5.632,6.344)%
    --(5.628,6.353)--(5.623,6.362)--(5.618,6.370)--(5.612,6.379)--(5.606,6.386)%
    --(5.600,6.394)--(5.593,6.401)--(5.586,6.408)--(5.578,6.414)--(5.571,6.420)%
    --(5.562,6.426)--(5.554,6.431)--(5.545,6.436)--(5.536,6.440)--(5.527,6.444)%
    --(5.518,6.447)--(5.508,6.450)--(5.499,6.452)--(5.489,6.454)--(5.479,6.455)%
    --(5.469,6.456)--(5.460,6.457)--(5.450,6.456)--(5.440,6.455)--(5.430,6.454)%
    --(5.420,6.452)--(5.411,6.450)--(5.401,6.447)--(5.392,6.444)--(5.383,6.440)%
    --(5.374,6.436)--(5.365,6.431)--(5.357,6.426)--(5.348,6.420)--(5.341,6.414)%
    --(5.333,6.408)--(5.326,6.401)--(5.319,6.394)--(5.313,6.386)--(5.307,6.379)%
    --(5.301,6.370)--(5.296,6.362)--(5.291,6.353)--(5.287,6.344)--(5.283,6.335)%
    --(5.280,6.326)--(5.277,6.316)--(5.275,6.307)--(5.273,6.297)--(5.272,6.287)%
    --(5.271,6.277)--(5.271,6.268)--(5.271,6.258)--(5.272,6.248)--(5.273,6.238)%
    --(5.275,6.228)--(5.277,6.219)--(5.280,6.209)--(5.283,6.200)--(5.287,6.191)%
    --(5.291,6.182)--(5.296,6.173)--(5.301,6.165)--(5.307,6.156)--(5.313,6.149)%
    --(5.319,6.141)--(5.326,6.134)--(5.333,6.127)--(5.341,6.121)--(5.348,6.115)%
    --(5.357,6.109)--(5.365,6.104)--(5.374,6.099)--(5.383,6.095)--(5.392,6.091)%
    --(5.401,6.088)--(5.411,6.085)--(5.420,6.083)--(5.430,6.081)--(5.440,6.080)%
    --(5.450,6.079)--(5.459,6.079)--(5.469,6.079)--(5.479,6.080)--(5.489,6.081)%
    --(5.499,6.083)--(5.508,6.085)--(5.518,6.088)--(5.527,6.091)--(5.536,6.095)%
    --(5.545,6.099)--(5.554,6.104)--(5.562,6.109)--(5.571,6.115)--(5.578,6.121)%
    --(5.586,6.127)--(5.593,6.134)--(5.600,6.141)--(5.606,6.149)--(5.612,6.156)%
    --(5.618,6.165)--(5.623,6.173)--(5.628,6.182)--(5.632,6.191)--(5.636,6.200)%
    --(5.639,6.209)--(5.642,6.219)--(5.644,6.228)--(5.646,6.238)--(5.647,6.248)%
    --(5.648,6.258)--(5.649,6.267)--cycle;
\gpfill{color=gp lt color border,opacity=0.50} (8.379,5.211)--(8.378,5.222)--(8.377,5.233)--(8.376,5.244)%
    --(8.374,5.255)--(8.371,5.266)--(8.368,5.277)--(8.364,5.288)--(8.360,5.298)%
    --(8.355,5.309)--(8.350,5.318)--(8.344,5.328)--(8.337,5.337)--(8.330,5.346)%
    --(8.323,5.355)--(8.315,5.363)--(8.307,5.371)--(8.298,5.378)--(8.289,5.385)%
    --(8.280,5.392)--(8.271,5.398)--(8.261,5.403)--(8.250,5.408)--(8.240,5.412)%
    --(8.229,5.416)--(8.218,5.419)--(8.207,5.422)--(8.196,5.424)--(8.185,5.425)%
    --(8.174,5.426)--(8.163,5.427)--(8.151,5.426)--(8.140,5.425)--(8.129,5.424)%
    --(8.118,5.422)--(8.107,5.419)--(8.096,5.416)--(8.085,5.412)--(8.075,5.408)%
    --(8.064,5.403)--(8.055,5.398)--(8.045,5.392)--(8.036,5.385)--(8.027,5.378)%
    --(8.018,5.371)--(8.010,5.363)--(8.002,5.355)--(7.995,5.346)--(7.988,5.337)%
    --(7.981,5.328)--(7.975,5.318)--(7.970,5.309)--(7.965,5.298)--(7.961,5.288)%
    --(7.957,5.277)--(7.954,5.266)--(7.951,5.255)--(7.949,5.244)--(7.948,5.233)%
    --(7.947,5.222)--(7.947,5.211)--(7.947,5.199)--(7.948,5.188)--(7.949,5.177)%
    --(7.951,5.166)--(7.954,5.155)--(7.957,5.144)--(7.961,5.133)--(7.965,5.123)%
    --(7.970,5.112)--(7.975,5.102)--(7.981,5.093)--(7.988,5.084)--(7.995,5.075)%
    --(8.002,5.066)--(8.010,5.058)--(8.018,5.050)--(8.027,5.043)--(8.036,5.036)%
    --(8.045,5.029)--(8.054,5.023)--(8.064,5.018)--(8.075,5.013)--(8.085,5.009)%
    --(8.096,5.005)--(8.107,5.002)--(8.118,4.999)--(8.129,4.997)--(8.140,4.996)%
    --(8.151,4.995)--(8.162,4.995)--(8.174,4.995)--(8.185,4.996)--(8.196,4.997)%
    --(8.207,4.999)--(8.218,5.002)--(8.229,5.005)--(8.240,5.009)--(8.250,5.013)%
    --(8.261,5.018)--(8.271,5.023)--(8.280,5.029)--(8.289,5.036)--(8.298,5.043)%
    --(8.307,5.050)--(8.315,5.058)--(8.323,5.066)--(8.330,5.075)--(8.337,5.084)%
    --(8.344,5.093)--(8.350,5.102)--(8.355,5.112)--(8.360,5.123)--(8.364,5.133)%
    --(8.368,5.144)--(8.371,5.155)--(8.374,5.166)--(8.376,5.177)--(8.377,5.188)%
    --(8.378,5.199)--(8.379,5.210)--cycle;
\gpfill{color=gp lt color border,opacity=0.50} (4.271,6.268)--(4.270,6.276)--(4.270,6.284)--(4.269,6.293)%
    --(4.267,6.301)--(4.265,6.309)--(4.263,6.318)--(4.260,6.326)--(4.256,6.333)%
    --(4.253,6.341)--(4.249,6.348)--(4.244,6.356)--(4.240,6.363)--(4.234,6.369)%
    --(4.229,6.376)--(4.223,6.382)--(4.217,6.388)--(4.210,6.393)--(4.204,6.399)%
    --(4.197,6.403)--(4.190,6.408)--(4.182,6.412)--(4.174,6.415)--(4.167,6.419)%
    --(4.159,6.422)--(4.150,6.424)--(4.142,6.426)--(4.134,6.428)--(4.125,6.429)%
    --(4.117,6.429)--(4.109,6.430)--(4.100,6.429)--(4.092,6.429)--(4.083,6.428)%
    --(4.075,6.426)--(4.067,6.424)--(4.058,6.422)--(4.050,6.419)--(4.043,6.415)%
    --(4.035,6.412)--(4.028,6.408)--(4.020,6.403)--(4.013,6.399)--(4.007,6.393)%
    --(4.000,6.388)--(3.994,6.382)--(3.988,6.376)--(3.983,6.369)--(3.977,6.363)%
    --(3.973,6.356)--(3.968,6.348)--(3.964,6.341)--(3.961,6.333)--(3.957,6.326)%
    --(3.954,6.318)--(3.952,6.309)--(3.950,6.301)--(3.948,6.293)--(3.947,6.284)%
    --(3.947,6.276)--(3.947,6.268)--(3.947,6.259)--(3.947,6.251)--(3.948,6.242)%
    --(3.950,6.234)--(3.952,6.226)--(3.954,6.217)--(3.957,6.209)--(3.961,6.202)%
    --(3.964,6.194)--(3.968,6.186)--(3.973,6.179)--(3.977,6.172)--(3.983,6.166)%
    --(3.988,6.159)--(3.994,6.153)--(4.000,6.147)--(4.007,6.142)--(4.013,6.136)%
    --(4.020,6.132)--(4.027,6.127)--(4.035,6.123)--(4.043,6.120)--(4.050,6.116)%
    --(4.058,6.113)--(4.067,6.111)--(4.075,6.109)--(4.083,6.107)--(4.092,6.106)%
    --(4.100,6.106)--(4.108,6.106)--(4.117,6.106)--(4.125,6.106)--(4.134,6.107)%
    --(4.142,6.109)--(4.150,6.111)--(4.159,6.113)--(4.167,6.116)--(4.174,6.120)%
    --(4.182,6.123)--(4.190,6.127)--(4.197,6.132)--(4.204,6.136)--(4.210,6.142)%
    --(4.217,6.147)--(4.223,6.153)--(4.229,6.159)--(4.234,6.166)--(4.240,6.172)%
    --(4.244,6.179)--(4.249,6.186)--(4.253,6.194)--(4.256,6.202)--(4.260,6.209)%
    --(4.263,6.217)--(4.265,6.226)--(4.267,6.234)--(4.269,6.242)--(4.270,6.251)%
    --(4.270,6.259)--(4.271,6.267)--cycle;
\gpfill{color=gp lt color border,opacity=0.50} (7.055,5.211)--(7.054,5.223)--(7.053,5.236)--(7.052,5.249)%
    --(7.049,5.261)--(7.046,5.273)--(7.043,5.286)--(7.038,5.298)--(7.033,5.309)%
    --(7.028,5.321)--(7.022,5.332)--(7.015,5.343)--(7.008,5.353)--(7.000,5.363)%
    --(6.992,5.373)--(6.983,5.382)--(6.974,5.391)--(6.964,5.399)--(6.954,5.407)%
    --(6.944,5.414)--(6.933,5.421)--(6.922,5.427)--(6.910,5.432)--(6.899,5.437)%
    --(6.887,5.442)--(6.874,5.445)--(6.862,5.448)--(6.850,5.451)--(6.837,5.452)%
    --(6.824,5.453)--(6.812,5.454)--(6.799,5.453)--(6.786,5.452)--(6.773,5.451)%
    --(6.761,5.448)--(6.749,5.445)--(6.736,5.442)--(6.724,5.437)--(6.713,5.432)%
    --(6.701,5.427)--(6.690,5.421)--(6.679,5.414)--(6.669,5.407)--(6.659,5.399)%
    --(6.649,5.391)--(6.640,5.382)--(6.631,5.373)--(6.623,5.363)--(6.615,5.353)%
    --(6.608,5.343)--(6.601,5.332)--(6.595,5.321)--(6.590,5.309)--(6.585,5.298)%
    --(6.580,5.286)--(6.577,5.273)--(6.574,5.261)--(6.571,5.249)--(6.570,5.236)%
    --(6.569,5.223)--(6.569,5.211)--(6.569,5.198)--(6.570,5.185)--(6.571,5.172)%
    --(6.574,5.160)--(6.577,5.148)--(6.580,5.135)--(6.585,5.123)--(6.590,5.112)%
    --(6.595,5.100)--(6.601,5.089)--(6.608,5.078)--(6.615,5.068)--(6.623,5.058)%
    --(6.631,5.048)--(6.640,5.039)--(6.649,5.030)--(6.659,5.022)--(6.669,5.014)%
    --(6.679,5.007)--(6.690,5.000)--(6.701,4.994)--(6.713,4.989)--(6.724,4.984)%
    --(6.736,4.979)--(6.749,4.976)--(6.761,4.973)--(6.773,4.970)--(6.786,4.969)%
    --(6.799,4.968)--(6.811,4.968)--(6.824,4.968)--(6.837,4.969)--(6.850,4.970)%
    --(6.862,4.973)--(6.874,4.976)--(6.887,4.979)--(6.899,4.984)--(6.910,4.989)%
    --(6.922,4.994)--(6.933,5.000)--(6.944,5.007)--(6.954,5.014)--(6.964,5.022)%
    --(6.974,5.030)--(6.983,5.039)--(6.992,5.048)--(7.000,5.058)--(7.008,5.068)%
    --(7.015,5.078)--(7.022,5.089)--(7.028,5.100)--(7.033,5.112)--(7.038,5.123)%
    --(7.043,5.135)--(7.046,5.148)--(7.049,5.160)--(7.052,5.172)--(7.053,5.185)%
    --(7.054,5.198)--(7.055,5.210)--cycle;
\gpfill{color=gp lt color border,opacity=0.50} (2.622,4.155)--(2.621,4.176)--(2.619,4.197)--(2.617,4.218)%
    --(2.613,4.239)--(2.608,4.259)--(2.602,4.280)--(2.595,4.300)--(2.586,4.319)%
    --(2.577,4.338)--(2.567,4.357)--(2.556,4.375)--(2.544,4.393)--(2.531,4.409)%
    --(2.517,4.425)--(2.503,4.441)--(2.487,4.455)--(2.471,4.469)--(2.455,4.482)%
    --(2.437,4.494)--(2.419,4.505)--(2.400,4.515)--(2.381,4.524)--(2.362,4.533)%
    --(2.342,4.540)--(2.321,4.546)--(2.301,4.551)--(2.280,4.555)--(2.259,4.557)%
    --(2.238,4.559)--(2.217,4.560)--(2.195,4.559)--(2.174,4.557)--(2.153,4.555)%
    --(2.132,4.551)--(2.112,4.546)--(2.091,4.540)--(2.071,4.533)--(2.052,4.524)%
    --(2.033,4.515)--(2.014,4.505)--(1.996,4.494)--(1.978,4.482)--(1.962,4.469)%
    --(1.946,4.455)--(1.930,4.441)--(1.916,4.425)--(1.902,4.409)--(1.889,4.393)%
    --(1.877,4.375)--(1.866,4.357)--(1.856,4.338)--(1.847,4.319)--(1.838,4.300)%
    --(1.831,4.280)--(1.825,4.259)--(1.820,4.239)--(1.816,4.218)--(1.814,4.197)%
    --(1.812,4.176)--(1.812,4.155)--(1.812,4.133)--(1.814,4.112)--(1.816,4.091)%
    --(1.820,4.070)--(1.825,4.050)--(1.831,4.029)--(1.838,4.009)--(1.847,3.990)%
    --(1.856,3.971)--(1.866,3.952)--(1.877,3.934)--(1.889,3.916)--(1.902,3.900)%
    --(1.916,3.884)--(1.930,3.868)--(1.946,3.854)--(1.962,3.840)--(1.978,3.827)%
    --(1.996,3.815)--(2.014,3.804)--(2.033,3.794)--(2.052,3.785)--(2.071,3.776)%
    --(2.091,3.769)--(2.112,3.763)--(2.132,3.758)--(2.153,3.754)--(2.174,3.752)%
    --(2.195,3.750)--(2.216,3.750)--(2.238,3.750)--(2.259,3.752)--(2.280,3.754)%
    --(2.301,3.758)--(2.321,3.763)--(2.342,3.769)--(2.362,3.776)--(2.381,3.785)%
    --(2.400,3.794)--(2.419,3.804)--(2.437,3.815)--(2.455,3.827)--(2.471,3.840)%
    --(2.487,3.854)--(2.503,3.868)--(2.517,3.884)--(2.531,3.900)--(2.544,3.916)%
    --(2.556,3.934)--(2.567,3.952)--(2.577,3.971)--(2.586,3.990)--(2.595,4.009)%
    --(2.602,4.029)--(2.608,4.050)--(2.613,4.070)--(2.617,4.091)--(2.619,4.112)%
    --(2.621,4.133)--(2.622,4.154)--cycle;
\gpfill{color=gp lt color border,opacity=0.50} (5.055,3.098)--(5.054,3.105)--(5.054,3.112)--(5.053,3.119)%
    --(5.052,3.126)--(5.050,3.132)--(5.048,3.139)--(5.046,3.146)--(5.043,3.152)%
    --(5.040,3.159)--(5.036,3.165)--(5.033,3.171)--(5.029,3.177)--(5.024,3.182)%
    --(5.020,3.188)--(5.015,3.193)--(5.010,3.198)--(5.004,3.202)--(4.999,3.207)%
    --(4.993,3.211)--(4.987,3.214)--(4.981,3.218)--(4.974,3.221)--(4.968,3.224)%
    --(4.961,3.226)--(4.954,3.228)--(4.948,3.230)--(4.941,3.231)--(4.934,3.232)%
    --(4.927,3.232)--(4.920,3.233)--(4.912,3.232)--(4.905,3.232)--(4.898,3.231)%
    --(4.891,3.230)--(4.885,3.228)--(4.878,3.226)--(4.871,3.224)--(4.865,3.221)%
    --(4.858,3.218)--(4.852,3.214)--(4.846,3.211)--(4.840,3.207)--(4.835,3.202)%
    --(4.829,3.198)--(4.824,3.193)--(4.819,3.188)--(4.815,3.182)--(4.810,3.177)%
    --(4.806,3.171)--(4.803,3.165)--(4.799,3.159)--(4.796,3.152)--(4.793,3.146)%
    --(4.791,3.139)--(4.789,3.132)--(4.787,3.126)--(4.786,3.119)--(4.785,3.112)%
    --(4.785,3.105)--(4.785,3.098)--(4.785,3.090)--(4.785,3.083)--(4.786,3.076)%
    --(4.787,3.069)--(4.789,3.063)--(4.791,3.056)--(4.793,3.049)--(4.796,3.043)%
    --(4.799,3.036)--(4.803,3.030)--(4.806,3.024)--(4.810,3.018)--(4.815,3.013)%
    --(4.819,3.007)--(4.824,3.002)--(4.829,2.997)--(4.835,2.993)--(4.840,2.988)%
    --(4.846,2.984)--(4.852,2.981)--(4.858,2.977)--(4.865,2.974)--(4.871,2.971)%
    --(4.878,2.969)--(4.885,2.967)--(4.891,2.965)--(4.898,2.964)--(4.905,2.963)%
    --(4.912,2.963)--(4.919,2.963)--(4.927,2.963)--(4.934,2.963)--(4.941,2.964)%
    --(4.948,2.965)--(4.954,2.967)--(4.961,2.969)--(4.968,2.971)--(4.974,2.974)%
    --(4.981,2.977)--(4.987,2.981)--(4.993,2.984)--(4.999,2.988)--(5.004,2.993)%
    --(5.010,2.997)--(5.015,3.002)--(5.020,3.007)--(5.024,3.013)--(5.029,3.018)%
    --(5.033,3.024)--(5.036,3.030)--(5.040,3.036)--(5.043,3.043)--(5.046,3.049)%
    --(5.048,3.056)--(5.050,3.063)--(5.052,3.069)--(5.053,3.076)--(5.054,3.083)%
    --(5.054,3.090)--(5.055,3.097)--cycle;
\gpfill{color=gp lt color border,opacity=0.50} (1.946,3.098)--(1.945,3.126)--(1.943,3.154)--(1.939,3.182)%
    --(1.934,3.210)--(1.927,3.237)--(1.919,3.264)--(1.910,3.291)--(1.899,3.317)%
    --(1.887,3.343)--(1.873,3.367)--(1.858,3.392)--(1.842,3.415)--(1.825,3.437)%
    --(1.807,3.459)--(1.787,3.479)--(1.767,3.499)--(1.745,3.517)--(1.723,3.534)%
    --(1.700,3.550)--(1.676,3.565)--(1.651,3.579)--(1.625,3.591)--(1.599,3.602)%
    --(1.572,3.611)--(1.545,3.619)--(1.518,3.626)--(1.490,3.631)--(1.462,3.635)%
    --(1.434,3.637)--(1.406,3.638)--(1.377,3.637)--(1.349,3.635)--(1.321,3.631)%
    --(1.293,3.626)--(1.266,3.619)--(1.239,3.611)--(1.212,3.602)--(1.186,3.591)%
    --(1.160,3.579)--(1.136,3.565)--(1.111,3.550)--(1.088,3.534)--(1.066,3.517)%
    --(1.044,3.499)--(1.024,3.479)--(1.004,3.459)--(0.986,3.437)--(0.969,3.415)%
    --(0.953,3.392)--(0.938,3.367)--(0.924,3.343)--(0.912,3.317)--(0.901,3.291)%
    --(0.892,3.264)--(0.884,3.237)--(0.877,3.210)--(0.872,3.182)--(0.868,3.154)%
    --(0.866,3.126)--(0.866,3.098)--(0.866,3.069)--(0.868,3.041)--(0.872,3.013)%
    --(0.877,2.985)--(0.884,2.958)--(0.892,2.931)--(0.901,2.904)--(0.912,2.878)%
    --(0.924,2.852)--(0.938,2.827)--(0.953,2.803)--(0.969,2.780)--(0.986,2.758)%
    --(1.004,2.736)--(1.024,2.716)--(1.044,2.696)--(1.066,2.678)--(1.088,2.661)%
    --(1.111,2.645)--(1.135,2.630)--(1.160,2.616)--(1.186,2.604)--(1.212,2.593)%
    --(1.239,2.584)--(1.266,2.576)--(1.293,2.569)--(1.321,2.564)--(1.349,2.560)%
    --(1.377,2.558)--(1.405,2.558)--(1.434,2.558)--(1.462,2.560)--(1.490,2.564)%
    --(1.518,2.569)--(1.545,2.576)--(1.572,2.584)--(1.599,2.593)--(1.625,2.604)%
    --(1.651,2.616)--(1.676,2.630)--(1.700,2.645)--(1.723,2.661)--(1.745,2.678)%
    --(1.767,2.696)--(1.787,2.716)--(1.807,2.736)--(1.825,2.758)--(1.842,2.780)%
    --(1.858,2.803)--(1.873,2.827)--(1.887,2.852)--(1.899,2.878)--(1.910,2.904)%
    --(1.919,2.931)--(1.927,2.958)--(1.934,2.985)--(1.939,3.013)--(1.943,3.041)%
    --(1.945,3.069)--(1.946,3.097)--cycle;
\gpfill{color=gp lt color border,opacity=0.50} (2.921,6.268)--(2.920,6.276)--(2.920,6.285)--(2.918,6.293)%
    --(2.917,6.301)--(2.915,6.310)--(2.913,6.318)--(2.910,6.326)--(2.906,6.334)%
    --(2.903,6.342)--(2.899,6.349)--(2.894,6.356)--(2.889,6.363)--(2.884,6.370)%
    --(2.879,6.377)--(2.873,6.383)--(2.867,6.389)--(2.860,6.394)--(2.853,6.399)%
    --(2.846,6.404)--(2.839,6.409)--(2.832,6.413)--(2.824,6.416)--(2.816,6.420)%
    --(2.808,6.423)--(2.800,6.425)--(2.791,6.427)--(2.783,6.428)--(2.775,6.430)%
    --(2.766,6.430)--(2.758,6.431)--(2.749,6.430)--(2.740,6.430)--(2.732,6.428)%
    --(2.724,6.427)--(2.715,6.425)--(2.707,6.423)--(2.699,6.420)--(2.691,6.416)%
    --(2.683,6.413)--(2.676,6.409)--(2.669,6.404)--(2.662,6.399)--(2.655,6.394)%
    --(2.648,6.389)--(2.642,6.383)--(2.636,6.377)--(2.631,6.370)--(2.626,6.363)%
    --(2.621,6.356)--(2.616,6.349)--(2.612,6.342)--(2.609,6.334)--(2.605,6.326)%
    --(2.602,6.318)--(2.600,6.310)--(2.598,6.301)--(2.597,6.293)--(2.595,6.285)%
    --(2.595,6.276)--(2.595,6.268)--(2.595,6.259)--(2.595,6.250)--(2.597,6.242)%
    --(2.598,6.234)--(2.600,6.225)--(2.602,6.217)--(2.605,6.209)--(2.609,6.201)%
    --(2.612,6.193)--(2.616,6.186)--(2.621,6.179)--(2.626,6.172)--(2.631,6.165)%
    --(2.636,6.158)--(2.642,6.152)--(2.648,6.146)--(2.655,6.141)--(2.662,6.136)%
    --(2.669,6.131)--(2.676,6.126)--(2.683,6.122)--(2.691,6.119)--(2.699,6.115)%
    --(2.707,6.112)--(2.715,6.110)--(2.724,6.108)--(2.732,6.107)--(2.740,6.105)%
    --(2.749,6.105)--(2.757,6.105)--(2.766,6.105)--(2.775,6.105)--(2.783,6.107)%
    --(2.791,6.108)--(2.800,6.110)--(2.808,6.112)--(2.816,6.115)--(2.824,6.119)%
    --(2.832,6.122)--(2.839,6.126)--(2.846,6.131)--(2.853,6.136)--(2.860,6.141)%
    --(2.867,6.146)--(2.873,6.152)--(2.879,6.158)--(2.884,6.165)--(2.889,6.172)%
    --(2.894,6.179)--(2.899,6.186)--(2.903,6.193)--(2.906,6.201)--(2.910,6.209)%
    --(2.913,6.217)--(2.915,6.225)--(2.917,6.234)--(2.918,6.242)--(2.920,6.250)%
    --(2.920,6.259)--(2.921,6.267)--cycle;
\gpfill{color=gp lt color border,opacity=0.50} (5.730,5.211)--(5.729,5.225)--(5.728,5.239)--(5.726,5.253)%
    --(5.724,5.267)--(5.720,5.280)--(5.716,5.294)--(5.712,5.307)--(5.706,5.320)%
    --(5.700,5.333)--(5.693,5.345)--(5.686,5.358)--(5.678,5.369)--(5.669,5.380)%
    --(5.660,5.391)--(5.650,5.401)--(5.640,5.411)--(5.629,5.420)--(5.618,5.429)%
    --(5.607,5.437)--(5.595,5.444)--(5.582,5.451)--(5.569,5.457)--(5.556,5.463)%
    --(5.543,5.467)--(5.529,5.471)--(5.516,5.475)--(5.502,5.477)--(5.488,5.479)%
    --(5.474,5.480)--(5.460,5.481)--(5.445,5.480)--(5.431,5.479)--(5.417,5.477)%
    --(5.403,5.475)--(5.390,5.471)--(5.376,5.467)--(5.363,5.463)--(5.350,5.457)%
    --(5.337,5.451)--(5.325,5.444)--(5.312,5.437)--(5.301,5.429)--(5.290,5.420)%
    --(5.279,5.411)--(5.269,5.401)--(5.259,5.391)--(5.250,5.380)--(5.241,5.369)%
    --(5.233,5.358)--(5.226,5.345)--(5.219,5.333)--(5.213,5.320)--(5.207,5.307)%
    --(5.203,5.294)--(5.199,5.280)--(5.195,5.267)--(5.193,5.253)--(5.191,5.239)%
    --(5.190,5.225)--(5.190,5.211)--(5.190,5.196)--(5.191,5.182)--(5.193,5.168)%
    --(5.195,5.154)--(5.199,5.141)--(5.203,5.127)--(5.207,5.114)--(5.213,5.101)%
    --(5.219,5.088)--(5.226,5.075)--(5.233,5.063)--(5.241,5.052)--(5.250,5.041)%
    --(5.259,5.030)--(5.269,5.020)--(5.279,5.010)--(5.290,5.001)--(5.301,4.992)%
    --(5.312,4.984)--(5.324,4.977)--(5.337,4.970)--(5.350,4.964)--(5.363,4.958)%
    --(5.376,4.954)--(5.390,4.950)--(5.403,4.946)--(5.417,4.944)--(5.431,4.942)%
    --(5.445,4.941)--(5.459,4.941)--(5.474,4.941)--(5.488,4.942)--(5.502,4.944)%
    --(5.516,4.946)--(5.529,4.950)--(5.543,4.954)--(5.556,4.958)--(5.569,4.964)%
    --(5.582,4.970)--(5.595,4.977)--(5.607,4.984)--(5.618,4.992)--(5.629,5.001)%
    --(5.640,5.010)--(5.650,5.020)--(5.660,5.030)--(5.669,5.041)--(5.678,5.052)%
    --(5.686,5.063)--(5.693,5.075)--(5.700,5.088)--(5.706,5.101)--(5.712,5.114)%
    --(5.716,5.127)--(5.720,5.141)--(5.724,5.154)--(5.726,5.168)--(5.728,5.182)%
    --(5.729,5.196)--(5.730,5.210)--cycle;
\gpfill{color=gp lt color border,opacity=0.50} (6.812,8.381)--(6.812,8.381)--(6.812,8.381)--(6.812,8.381)%
    --(6.812,8.381)--(6.812,8.381)--(6.812,8.381)--(6.812,8.381)--(6.812,8.381)%
    --(6.812,8.381)--(6.812,8.381)--(6.812,8.381)--(6.812,8.381)--(6.812,8.381)%
    --(6.812,8.381)--(6.812,8.381)--(6.812,8.381)--(6.812,8.381)--(6.812,8.381)%
    --(6.812,8.381)--(6.812,8.381)--(6.812,8.381)--(6.812,8.381)--(6.812,8.381)%
    --(6.812,8.381)--(6.812,8.381)--(6.812,8.381)--(6.812,8.381)--(6.812,8.381)%
    --(6.812,8.381)--(6.812,8.381)--(6.812,8.381)--(6.812,8.381)--(6.812,8.381)%
    --(6.812,8.381)--(6.812,8.381)--(6.812,8.381)--(6.812,8.381)--(6.812,8.381)%
    --(6.812,8.381)--(6.812,8.381)--(6.812,8.381)--(6.812,8.381)--(6.812,8.381)%
    --(6.812,8.381)--(6.812,8.381)--(6.812,8.381)--(6.812,8.381)--(6.812,8.381)%
    --(6.812,8.381)--(6.812,8.381)--(6.812,8.381)--(6.812,8.381)--(6.812,8.381)%
    --(6.812,8.381)--(6.812,8.381)--(6.812,8.381)--(6.812,8.381)--(6.812,8.381)%
    --(6.812,8.381)--(6.812,8.381)--(6.812,8.381)--(6.812,8.381)--(6.812,8.381)%
    --(6.812,8.381)--(6.812,8.381)--(6.812,8.381)--(6.812,8.381)--(6.812,8.381)%
    --(6.812,8.381)--(6.812,8.381)--(6.812,8.381)--(6.812,8.381)--(6.812,8.381)%
    --(6.812,8.381)--(6.812,8.381)--(6.812,8.381)--(6.812,8.381)--(6.812,8.381)%
    --(6.812,8.381)--(6.812,8.381)--(6.812,8.381)--(6.812,8.381)--(6.812,8.381)%
    --(6.812,8.381)--(6.812,8.381)--(6.812,8.381)--(6.812,8.381)--(6.812,8.381)%
    --(6.812,8.381)--(6.812,8.381)--(6.812,8.381)--(6.812,8.381)--(6.812,8.381)%
    --(6.812,8.381)--(6.812,8.381)--(6.812,8.381)--(6.812,8.381)--(6.812,8.381)%
    --(6.812,8.381)--(6.812,8.381)--(6.812,8.381)--(6.812,8.381)--(6.812,8.381)%
    --(6.812,8.381)--(6.812,8.381)--(6.812,8.381)--(6.812,8.381)--(6.812,8.381)%
    --(6.812,8.381)--(6.812,8.381)--(6.812,8.381)--(6.812,8.381)--(6.812,8.381)%
    --(6.812,8.381)--(6.812,8.381)--(6.812,8.381)--(6.812,8.381)--(6.812,8.381)%
    --(6.812,8.381)--cycle;
\gpfill{color=gp lt color border,opacity=0.50} (4.379,5.211)--(4.378,5.225)--(4.377,5.239)--(4.375,5.253)%
    --(4.373,5.267)--(4.369,5.280)--(4.365,5.294)--(4.361,5.307)--(4.355,5.320)%
    --(4.349,5.333)--(4.342,5.345)--(4.335,5.358)--(4.327,5.369)--(4.318,5.380)%
    --(4.309,5.391)--(4.299,5.401)--(4.289,5.411)--(4.278,5.420)--(4.267,5.429)%
    --(4.256,5.437)--(4.244,5.444)--(4.231,5.451)--(4.218,5.457)--(4.205,5.463)%
    --(4.192,5.467)--(4.178,5.471)--(4.165,5.475)--(4.151,5.477)--(4.137,5.479)%
    --(4.123,5.480)--(4.109,5.481)--(4.094,5.480)--(4.080,5.479)--(4.066,5.477)%
    --(4.052,5.475)--(4.039,5.471)--(4.025,5.467)--(4.012,5.463)--(3.999,5.457)%
    --(3.986,5.451)--(3.974,5.444)--(3.961,5.437)--(3.950,5.429)--(3.939,5.420)%
    --(3.928,5.411)--(3.918,5.401)--(3.908,5.391)--(3.899,5.380)--(3.890,5.369)%
    --(3.882,5.358)--(3.875,5.345)--(3.868,5.333)--(3.862,5.320)--(3.856,5.307)%
    --(3.852,5.294)--(3.848,5.280)--(3.844,5.267)--(3.842,5.253)--(3.840,5.239)%
    --(3.839,5.225)--(3.839,5.211)--(3.839,5.196)--(3.840,5.182)--(3.842,5.168)%
    --(3.844,5.154)--(3.848,5.141)--(3.852,5.127)--(3.856,5.114)--(3.862,5.101)%
    --(3.868,5.088)--(3.875,5.075)--(3.882,5.063)--(3.890,5.052)--(3.899,5.041)%
    --(3.908,5.030)--(3.918,5.020)--(3.928,5.010)--(3.939,5.001)--(3.950,4.992)%
    --(3.961,4.984)--(3.973,4.977)--(3.986,4.970)--(3.999,4.964)--(4.012,4.958)%
    --(4.025,4.954)--(4.039,4.950)--(4.052,4.946)--(4.066,4.944)--(4.080,4.942)%
    --(4.094,4.941)--(4.108,4.941)--(4.123,4.941)--(4.137,4.942)--(4.151,4.944)%
    --(4.165,4.946)--(4.178,4.950)--(4.192,4.954)--(4.205,4.958)--(4.218,4.964)%
    --(4.231,4.970)--(4.244,4.977)--(4.256,4.984)--(4.267,4.992)--(4.278,5.001)%
    --(4.289,5.010)--(4.299,5.020)--(4.309,5.030)--(4.318,5.041)--(4.327,5.052)%
    --(4.335,5.063)--(4.342,5.075)--(4.349,5.088)--(4.355,5.101)--(4.361,5.114)%
    --(4.365,5.127)--(4.369,5.141)--(4.373,5.154)--(4.375,5.168)--(4.377,5.182)%
    --(4.378,5.196)--(4.379,5.210)--cycle;
\gpfill{color=gp lt color border,opacity=0.50} (2.595,3.098)--(2.594,3.117)--(2.592,3.137)--(2.590,3.157)%
    --(2.586,3.176)--(2.582,3.195)--(2.576,3.214)--(2.569,3.233)--(2.562,3.251)%
    --(2.553,3.269)--(2.544,3.286)--(2.534,3.303)--(2.522,3.320)--(2.510,3.335)%
    --(2.497,3.350)--(2.484,3.365)--(2.469,3.378)--(2.454,3.391)--(2.439,3.403)%
    --(2.422,3.415)--(2.406,3.425)--(2.388,3.434)--(2.370,3.443)--(2.352,3.450)%
    --(2.333,3.457)--(2.314,3.463)--(2.295,3.467)--(2.276,3.471)--(2.256,3.473)%
    --(2.236,3.475)--(2.217,3.476)--(2.197,3.475)--(2.177,3.473)--(2.157,3.471)%
    --(2.138,3.467)--(2.119,3.463)--(2.100,3.457)--(2.081,3.450)--(2.063,3.443)%
    --(2.045,3.434)--(2.028,3.425)--(2.011,3.415)--(1.994,3.403)--(1.979,3.391)%
    --(1.964,3.378)--(1.949,3.365)--(1.936,3.350)--(1.923,3.335)--(1.911,3.320)%
    --(1.899,3.303)--(1.889,3.286)--(1.880,3.269)--(1.871,3.251)--(1.864,3.233)%
    --(1.857,3.214)--(1.851,3.195)--(1.847,3.176)--(1.843,3.157)--(1.841,3.137)%
    --(1.839,3.117)--(1.839,3.098)--(1.839,3.078)--(1.841,3.058)--(1.843,3.038)%
    --(1.847,3.019)--(1.851,3.000)--(1.857,2.981)--(1.864,2.962)--(1.871,2.944)%
    --(1.880,2.926)--(1.889,2.908)--(1.899,2.892)--(1.911,2.875)--(1.923,2.860)%
    --(1.936,2.845)--(1.949,2.830)--(1.964,2.817)--(1.979,2.804)--(1.994,2.792)%
    --(2.011,2.780)--(2.027,2.770)--(2.045,2.761)--(2.063,2.752)--(2.081,2.745)%
    --(2.100,2.738)--(2.119,2.732)--(2.138,2.728)--(2.157,2.724)--(2.177,2.722)%
    --(2.197,2.720)--(2.216,2.720)--(2.236,2.720)--(2.256,2.722)--(2.276,2.724)%
    --(2.295,2.728)--(2.314,2.732)--(2.333,2.738)--(2.352,2.745)--(2.370,2.752)%
    --(2.388,2.761)--(2.406,2.770)--(2.422,2.780)--(2.439,2.792)--(2.454,2.804)%
    --(2.469,2.817)--(2.484,2.830)--(2.497,2.845)--(2.510,2.860)--(2.522,2.875)%
    --(2.534,2.892)--(2.544,2.908)--(2.553,2.926)--(2.562,2.944)--(2.569,2.962)%
    --(2.576,2.981)--(2.582,3.000)--(2.586,3.019)--(2.590,3.038)--(2.592,3.058)%
    --(2.594,3.078)--(2.595,3.097)--cycle;
\gpfill{color=gp lt color border,opacity=0.50} (9.515,7.324)--(9.515,7.324)--(9.515,7.324)--(9.515,7.324)%
    --(9.515,7.324)--(9.515,7.324)--(9.515,7.324)--(9.515,7.324)--(9.515,7.324)%
    --(9.515,7.324)--(9.515,7.324)--(9.515,7.324)--(9.515,7.324)--(9.515,7.324)%
    --(9.515,7.324)--(9.515,7.324)--(9.515,7.324)--(9.515,7.324)--(9.515,7.324)%
    --(9.515,7.324)--(9.515,7.324)--(9.515,7.324)--(9.515,7.324)--(9.515,7.324)%
    --(9.515,7.324)--(9.515,7.324)--(9.515,7.324)--(9.515,7.324)--(9.515,7.324)%
    --(9.515,7.324)--(9.515,7.324)--(9.515,7.324)--(9.515,7.324)--(9.515,7.324)%
    --(9.515,7.324)--(9.515,7.324)--(9.515,7.324)--(9.515,7.324)--(9.515,7.324)%
    --(9.515,7.324)--(9.515,7.324)--(9.515,7.324)--(9.515,7.324)--(9.515,7.324)%
    --(9.515,7.324)--(9.515,7.324)--(9.515,7.324)--(9.515,7.324)--(9.515,7.324)%
    --(9.515,7.324)--(9.515,7.324)--(9.515,7.324)--(9.515,7.324)--(9.515,7.324)%
    --(9.515,7.324)--(9.515,7.324)--(9.515,7.324)--(9.515,7.324)--(9.515,7.324)%
    --(9.515,7.324)--(9.515,7.324)--(9.515,7.324)--(9.515,7.324)--(9.515,7.324)%
    --(9.515,7.324)--(9.515,7.324)--(9.515,7.324)--(9.515,7.324)--(9.515,7.324)%
    --(9.515,7.324)--(9.515,7.324)--(9.515,7.324)--(9.515,7.324)--(9.515,7.324)%
    --(9.515,7.324)--(9.515,7.324)--(9.515,7.324)--(9.515,7.324)--(9.515,7.324)%
    --(9.515,7.324)--(9.515,7.324)--(9.515,7.324)--(9.515,7.324)--(9.515,7.324)%
    --(9.515,7.324)--(9.515,7.324)--(9.515,7.324)--(9.515,7.324)--(9.515,7.324)%
    --(9.515,7.324)--(9.515,7.324)--(9.515,7.324)--(9.515,7.324)--(9.515,7.324)%
    --(9.515,7.324)--(9.515,7.324)--(9.515,7.324)--(9.515,7.324)--(9.515,7.324)%
    --(9.515,7.324)--(9.515,7.324)--(9.515,7.324)--(9.515,7.324)--(9.515,7.324)%
    --(9.515,7.324)--(9.515,7.324)--(9.515,7.324)--(9.515,7.324)--(9.515,7.324)%
    --(9.515,7.324)--(9.515,7.324)--(9.515,7.324)--(9.515,7.324)--(9.515,7.324)%
    --(9.515,7.324)--(9.515,7.324)--(9.515,7.324)--(9.515,7.324)--(9.515,7.324)%
    --(9.515,7.324)--cycle;
\gpfill{color=gp lt color border,opacity=0.50} (1.487,6.268)--(1.486,6.272)--(1.486,6.276)--(1.486,6.280)%
    --(1.485,6.284)--(1.484,6.288)--(1.483,6.293)--(1.481,6.297)--(1.479,6.300)%
    --(1.478,6.304)--(1.476,6.308)--(1.473,6.312)--(1.471,6.315)--(1.468,6.318)%
    --(1.466,6.322)--(1.463,6.325)--(1.460,6.328)--(1.456,6.330)--(1.453,6.333)%
    --(1.450,6.335)--(1.446,6.338)--(1.442,6.340)--(1.438,6.341)--(1.435,6.343)%
    --(1.431,6.345)--(1.426,6.346)--(1.422,6.347)--(1.418,6.348)--(1.414,6.348)%
    --(1.410,6.348)--(1.406,6.349)--(1.401,6.348)--(1.397,6.348)--(1.393,6.348)%
    --(1.389,6.347)--(1.385,6.346)--(1.380,6.345)--(1.376,6.343)--(1.373,6.341)%
    --(1.369,6.340)--(1.365,6.338)--(1.361,6.335)--(1.358,6.333)--(1.355,6.330)%
    --(1.351,6.328)--(1.348,6.325)--(1.345,6.322)--(1.343,6.318)--(1.340,6.315)%
    --(1.338,6.312)--(1.335,6.308)--(1.333,6.304)--(1.332,6.300)--(1.330,6.297)%
    --(1.328,6.293)--(1.327,6.288)--(1.326,6.284)--(1.325,6.280)--(1.325,6.276)%
    --(1.325,6.272)--(1.325,6.268)--(1.325,6.263)--(1.325,6.259)--(1.325,6.255)%
    --(1.326,6.251)--(1.327,6.247)--(1.328,6.242)--(1.330,6.238)--(1.332,6.235)%
    --(1.333,6.231)--(1.335,6.227)--(1.338,6.223)--(1.340,6.220)--(1.343,6.217)%
    --(1.345,6.213)--(1.348,6.210)--(1.351,6.207)--(1.355,6.205)--(1.358,6.202)%
    --(1.361,6.200)--(1.365,6.197)--(1.369,6.195)--(1.373,6.194)--(1.376,6.192)%
    --(1.380,6.190)--(1.385,6.189)--(1.389,6.188)--(1.393,6.187)--(1.397,6.187)%
    --(1.401,6.187)--(1.405,6.187)--(1.410,6.187)--(1.414,6.187)--(1.418,6.187)%
    --(1.422,6.188)--(1.426,6.189)--(1.431,6.190)--(1.435,6.192)--(1.438,6.194)%
    --(1.442,6.195)--(1.446,6.197)--(1.450,6.200)--(1.453,6.202)--(1.456,6.205)%
    --(1.460,6.207)--(1.463,6.210)--(1.466,6.213)--(1.468,6.217)--(1.471,6.220)%
    --(1.473,6.223)--(1.476,6.227)--(1.478,6.231)--(1.479,6.235)--(1.481,6.238)%
    --(1.483,6.242)--(1.484,6.247)--(1.485,6.251)--(1.486,6.255)--(1.486,6.259)%
    --(1.486,6.263)--(1.487,6.267)--cycle;
\gpfill{color=gp lt color border,opacity=0.50} (3.056,5.211)--(3.055,5.226)--(3.054,5.242)--(3.052,5.257)%
    --(3.049,5.272)--(3.045,5.288)--(3.041,5.303)--(3.036,5.317)--(3.030,5.332)%
    --(3.023,5.346)--(3.016,5.359)--(3.007,5.373)--(2.999,5.386)--(2.989,5.398)%
    --(2.979,5.410)--(2.968,5.421)--(2.957,5.432)--(2.945,5.442)--(2.933,5.452)%
    --(2.920,5.460)--(2.907,5.469)--(2.893,5.476)--(2.879,5.483)--(2.864,5.489)%
    --(2.850,5.494)--(2.835,5.498)--(2.819,5.502)--(2.804,5.505)--(2.789,5.507)%
    --(2.773,5.508)--(2.758,5.509)--(2.742,5.508)--(2.726,5.507)--(2.711,5.505)%
    --(2.696,5.502)--(2.680,5.498)--(2.665,5.494)--(2.651,5.489)--(2.636,5.483)%
    --(2.622,5.476)--(2.609,5.469)--(2.595,5.460)--(2.582,5.452)--(2.570,5.442)%
    --(2.558,5.432)--(2.547,5.421)--(2.536,5.410)--(2.526,5.398)--(2.516,5.386)%
    --(2.508,5.373)--(2.499,5.359)--(2.492,5.346)--(2.485,5.332)--(2.479,5.317)%
    --(2.474,5.303)--(2.470,5.288)--(2.466,5.272)--(2.463,5.257)--(2.461,5.242)%
    --(2.460,5.226)--(2.460,5.211)--(2.460,5.195)--(2.461,5.179)--(2.463,5.164)%
    --(2.466,5.149)--(2.470,5.133)--(2.474,5.118)--(2.479,5.104)--(2.485,5.089)%
    --(2.492,5.075)--(2.499,5.061)--(2.508,5.048)--(2.516,5.035)--(2.526,5.023)%
    --(2.536,5.011)--(2.547,5.000)--(2.558,4.989)--(2.570,4.979)--(2.582,4.969)%
    --(2.595,4.961)--(2.608,4.952)--(2.622,4.945)--(2.636,4.938)--(2.651,4.932)%
    --(2.665,4.927)--(2.680,4.923)--(2.696,4.919)--(2.711,4.916)--(2.726,4.914)%
    --(2.742,4.913)--(2.757,4.913)--(2.773,4.913)--(2.789,4.914)--(2.804,4.916)%
    --(2.819,4.919)--(2.835,4.923)--(2.850,4.927)--(2.864,4.932)--(2.879,4.938)%
    --(2.893,4.945)--(2.907,4.952)--(2.920,4.961)--(2.933,4.969)--(2.945,4.979)%
    --(2.957,4.989)--(2.968,5.000)--(2.979,5.011)--(2.989,5.023)--(2.999,5.035)%
    --(3.007,5.048)--(3.016,5.061)--(3.023,5.075)--(3.030,5.089)--(3.036,5.104)%
    --(3.041,5.118)--(3.045,5.133)--(3.049,5.149)--(3.052,5.164)--(3.054,5.179)%
    --(3.055,5.195)--(3.056,5.210)--cycle;
\gpfill{color=gp lt color border,opacity=0.50} (6.839,7.324)--(6.838,7.325)--(6.838,7.326)--(6.838,7.328)%
    --(6.838,7.329)--(6.838,7.330)--(6.837,7.332)--(6.837,7.333)--(6.836,7.334)%
    --(6.836,7.336)--(6.835,7.337)--(6.834,7.338)--(6.833,7.339)--(6.832,7.340)%
    --(6.832,7.342)--(6.831,7.343)--(6.830,7.344)--(6.828,7.344)--(6.827,7.345)%
    --(6.826,7.346)--(6.825,7.347)--(6.824,7.348)--(6.822,7.348)--(6.821,7.349)%
    --(6.820,7.349)--(6.818,7.350)--(6.817,7.350)--(6.816,7.350)--(6.814,7.350)%
    --(6.813,7.350)--(6.812,7.351)--(6.810,7.350)--(6.809,7.350)--(6.807,7.350)%
    --(6.806,7.350)--(6.805,7.350)--(6.803,7.349)--(6.802,7.349)--(6.801,7.348)%
    --(6.799,7.348)--(6.798,7.347)--(6.797,7.346)--(6.796,7.345)--(6.795,7.344)%
    --(6.793,7.344)--(6.792,7.343)--(6.791,7.342)--(6.791,7.340)--(6.790,7.339)%
    --(6.789,7.338)--(6.788,7.337)--(6.787,7.336)--(6.787,7.334)--(6.786,7.333)%
    --(6.786,7.332)--(6.785,7.330)--(6.785,7.329)--(6.785,7.328)--(6.785,7.326)%
    --(6.785,7.325)--(6.785,7.324)--(6.785,7.322)--(6.785,7.321)--(6.785,7.319)%
    --(6.785,7.318)--(6.785,7.317)--(6.786,7.315)--(6.786,7.314)--(6.787,7.313)%
    --(6.787,7.311)--(6.788,7.310)--(6.789,7.309)--(6.790,7.308)--(6.791,7.307)%
    --(6.791,7.305)--(6.792,7.304)--(6.793,7.303)--(6.795,7.303)--(6.796,7.302)%
    --(6.797,7.301)--(6.798,7.300)--(6.799,7.299)--(6.801,7.299)--(6.802,7.298)%
    --(6.803,7.298)--(6.805,7.297)--(6.806,7.297)--(6.807,7.297)--(6.809,7.297)%
    --(6.810,7.297)--(6.811,7.297)--(6.813,7.297)--(6.814,7.297)--(6.816,7.297)%
    --(6.817,7.297)--(6.818,7.297)--(6.820,7.298)--(6.821,7.298)--(6.822,7.299)%
    --(6.824,7.299)--(6.825,7.300)--(6.826,7.301)--(6.827,7.302)--(6.828,7.303)%
    --(6.830,7.303)--(6.831,7.304)--(6.832,7.305)--(6.832,7.307)--(6.833,7.308)%
    --(6.834,7.309)--(6.835,7.310)--(6.836,7.311)--(6.836,7.313)--(6.837,7.314)%
    --(6.837,7.315)--(6.838,7.317)--(6.838,7.318)--(6.838,7.319)--(6.838,7.321)%
    --(6.838,7.322)--(6.839,7.323)--cycle;
\gpfill{color=gp lt color border,opacity=0.50} (1.649,5.211)--(1.648,5.223)--(1.647,5.236)--(1.646,5.249)%
    --(1.643,5.261)--(1.640,5.273)--(1.637,5.286)--(1.632,5.298)--(1.627,5.309)%
    --(1.622,5.321)--(1.616,5.332)--(1.609,5.343)--(1.602,5.353)--(1.594,5.363)%
    --(1.586,5.373)--(1.577,5.382)--(1.568,5.391)--(1.558,5.399)--(1.548,5.407)%
    --(1.538,5.414)--(1.527,5.421)--(1.516,5.427)--(1.504,5.432)--(1.493,5.437)%
    --(1.481,5.442)--(1.468,5.445)--(1.456,5.448)--(1.444,5.451)--(1.431,5.452)%
    --(1.418,5.453)--(1.406,5.454)--(1.393,5.453)--(1.380,5.452)--(1.367,5.451)%
    --(1.355,5.448)--(1.343,5.445)--(1.330,5.442)--(1.318,5.437)--(1.307,5.432)%
    --(1.295,5.427)--(1.284,5.421)--(1.273,5.414)--(1.263,5.407)--(1.253,5.399)%
    --(1.243,5.391)--(1.234,5.382)--(1.225,5.373)--(1.217,5.363)--(1.209,5.353)%
    --(1.202,5.343)--(1.195,5.332)--(1.189,5.321)--(1.184,5.309)--(1.179,5.298)%
    --(1.174,5.286)--(1.171,5.273)--(1.168,5.261)--(1.165,5.249)--(1.164,5.236)%
    --(1.163,5.223)--(1.163,5.211)--(1.163,5.198)--(1.164,5.185)--(1.165,5.172)%
    --(1.168,5.160)--(1.171,5.148)--(1.174,5.135)--(1.179,5.123)--(1.184,5.112)%
    --(1.189,5.100)--(1.195,5.089)--(1.202,5.078)--(1.209,5.068)--(1.217,5.058)%
    --(1.225,5.048)--(1.234,5.039)--(1.243,5.030)--(1.253,5.022)--(1.263,5.014)%
    --(1.273,5.007)--(1.284,5.000)--(1.295,4.994)--(1.307,4.989)--(1.318,4.984)%
    --(1.330,4.979)--(1.343,4.976)--(1.355,4.973)--(1.367,4.970)--(1.380,4.969)%
    --(1.393,4.968)--(1.405,4.968)--(1.418,4.968)--(1.431,4.969)--(1.444,4.970)%
    --(1.456,4.973)--(1.468,4.976)--(1.481,4.979)--(1.493,4.984)--(1.504,4.989)%
    --(1.516,4.994)--(1.527,5.000)--(1.538,5.007)--(1.548,5.014)--(1.558,5.022)%
    --(1.568,5.030)--(1.577,5.039)--(1.586,5.048)--(1.594,5.058)--(1.602,5.068)%
    --(1.609,5.078)--(1.616,5.089)--(1.622,5.100)--(1.627,5.112)--(1.632,5.123)%
    --(1.637,5.135)--(1.640,5.148)--(1.643,5.160)--(1.646,5.172)--(1.647,5.185)%
    --(1.648,5.198)--(1.649,5.210)--cycle;
\gpfill{color=gp lt color border,opacity=0.50} (10.163,4.155)--(10.162,4.160)--(10.162,4.166)--(10.161,4.171)%
    --(10.160,4.177)--(10.159,4.182)--(10.157,4.188)--(10.155,4.193)--(10.153,4.198)%
    --(10.151,4.204)--(10.148,4.208)--(10.145,4.213)--(10.142,4.218)--(10.138,4.222)%
    --(10.135,4.227)--(10.131,4.231)--(10.127,4.235)--(10.122,4.238)--(10.118,4.242)%
    --(10.113,4.245)--(10.109,4.248)--(10.104,4.251)--(10.098,4.253)--(10.093,4.255)%
    --(10.088,4.257)--(10.082,4.259)--(10.077,4.260)--(10.071,4.261)--(10.066,4.262)%
    --(10.060,4.262)--(10.055,4.263)--(10.049,4.262)--(10.043,4.262)--(10.038,4.261)%
    --(10.032,4.260)--(10.027,4.259)--(10.021,4.257)--(10.016,4.255)--(10.011,4.253)%
    --(10.005,4.251)--(10.001,4.248)--(9.996,4.245)--(9.991,4.242)--(9.987,4.238)%
    --(9.982,4.235)--(9.978,4.231)--(9.974,4.227)--(9.971,4.222)--(9.967,4.218)%
    --(9.964,4.213)--(9.961,4.208)--(9.958,4.204)--(9.956,4.198)--(9.954,4.193)%
    --(9.952,4.188)--(9.950,4.182)--(9.949,4.177)--(9.948,4.171)--(9.947,4.166)%
    --(9.947,4.160)--(9.947,4.155)--(9.947,4.149)--(9.947,4.143)--(9.948,4.138)%
    --(9.949,4.132)--(9.950,4.127)--(9.952,4.121)--(9.954,4.116)--(9.956,4.111)%
    --(9.958,4.105)--(9.961,4.100)--(9.964,4.096)--(9.967,4.091)--(9.971,4.087)%
    --(9.974,4.082)--(9.978,4.078)--(9.982,4.074)--(9.987,4.071)--(9.991,4.067)%
    --(9.996,4.064)--(10.000,4.061)--(10.005,4.058)--(10.011,4.056)--(10.016,4.054)%
    --(10.021,4.052)--(10.027,4.050)--(10.032,4.049)--(10.038,4.048)--(10.043,4.047)%
    --(10.049,4.047)--(10.054,4.047)--(10.060,4.047)--(10.066,4.047)--(10.071,4.048)%
    --(10.077,4.049)--(10.082,4.050)--(10.088,4.052)--(10.093,4.054)--(10.098,4.056)%
    --(10.104,4.058)--(10.109,4.061)--(10.113,4.064)--(10.118,4.067)--(10.122,4.071)%
    --(10.127,4.074)--(10.131,4.078)--(10.135,4.082)--(10.138,4.087)--(10.142,4.091)%
    --(10.145,4.096)--(10.148,4.100)--(10.151,4.105)--(10.153,4.111)--(10.155,4.116)%
    --(10.157,4.121)--(10.159,4.127)--(10.160,4.132)--(10.161,4.138)--(10.162,4.143)%
    --(10.162,4.149)--(10.163,4.154)--cycle;
\gpfill{color=gp lt color border,opacity=0.50} (5.514,7.324)--(5.513,7.326)--(5.513,7.329)--(5.513,7.332)%
    --(5.512,7.335)--(5.512,7.337)--(5.511,7.340)--(5.510,7.343)--(5.509,7.345)%
    --(5.508,7.348)--(5.506,7.350)--(5.505,7.353)--(5.503,7.355)--(5.501,7.357)%
    --(5.500,7.360)--(5.498,7.362)--(5.496,7.364)--(5.493,7.365)--(5.491,7.367)%
    --(5.489,7.369)--(5.487,7.370)--(5.484,7.372)--(5.481,7.373)--(5.479,7.374)%
    --(5.476,7.375)--(5.473,7.376)--(5.471,7.376)--(5.468,7.377)--(5.465,7.377)%
    --(5.462,7.377)--(5.460,7.378)--(5.457,7.377)--(5.454,7.377)--(5.451,7.377)%
    --(5.448,7.376)--(5.446,7.376)--(5.443,7.375)--(5.440,7.374)--(5.438,7.373)%
    --(5.435,7.372)--(5.433,7.370)--(5.430,7.369)--(5.428,7.367)--(5.426,7.365)%
    --(5.423,7.364)--(5.421,7.362)--(5.419,7.360)--(5.418,7.357)--(5.416,7.355)%
    --(5.414,7.353)--(5.413,7.350)--(5.411,7.348)--(5.410,7.345)--(5.409,7.343)%
    --(5.408,7.340)--(5.407,7.337)--(5.407,7.335)--(5.406,7.332)--(5.406,7.329)%
    --(5.406,7.326)--(5.406,7.324)--(5.406,7.321)--(5.406,7.318)--(5.406,7.315)%
    --(5.407,7.312)--(5.407,7.310)--(5.408,7.307)--(5.409,7.304)--(5.410,7.302)%
    --(5.411,7.299)--(5.413,7.296)--(5.414,7.294)--(5.416,7.292)--(5.418,7.290)%
    --(5.419,7.287)--(5.421,7.285)--(5.423,7.283)--(5.426,7.282)--(5.428,7.280)%
    --(5.430,7.278)--(5.432,7.277)--(5.435,7.275)--(5.438,7.274)--(5.440,7.273)%
    --(5.443,7.272)--(5.446,7.271)--(5.448,7.271)--(5.451,7.270)--(5.454,7.270)%
    --(5.457,7.270)--(5.459,7.270)--(5.462,7.270)--(5.465,7.270)--(5.468,7.270)%
    --(5.471,7.271)--(5.473,7.271)--(5.476,7.272)--(5.479,7.273)--(5.481,7.274)%
    --(5.484,7.275)--(5.487,7.277)--(5.489,7.278)--(5.491,7.280)--(5.493,7.282)%
    --(5.496,7.283)--(5.498,7.285)--(5.500,7.287)--(5.501,7.290)--(5.503,7.292)%
    --(5.505,7.294)--(5.506,7.296)--(5.508,7.299)--(5.509,7.302)--(5.510,7.304)%
    --(5.511,7.307)--(5.512,7.310)--(5.512,7.312)--(5.513,7.315)--(5.513,7.318)%
    --(5.513,7.321)--(5.514,7.323)--cycle;
\gpfill{color=gp lt color border,opacity=0.50} (4.163,7.324)--(4.162,7.326)--(4.162,7.329)--(4.162,7.332)%
    --(4.161,7.335)--(4.161,7.337)--(4.160,7.340)--(4.159,7.343)--(4.158,7.345)%
    --(4.157,7.348)--(4.155,7.350)--(4.154,7.353)--(4.152,7.355)--(4.150,7.357)%
    --(4.149,7.360)--(4.147,7.362)--(4.145,7.364)--(4.142,7.365)--(4.140,7.367)%
    --(4.138,7.369)--(4.136,7.370)--(4.133,7.372)--(4.130,7.373)--(4.128,7.374)%
    --(4.125,7.375)--(4.122,7.376)--(4.120,7.376)--(4.117,7.377)--(4.114,7.377)%
    --(4.111,7.377)--(4.109,7.378)--(4.106,7.377)--(4.103,7.377)--(4.100,7.377)%
    --(4.097,7.376)--(4.095,7.376)--(4.092,7.375)--(4.089,7.374)--(4.087,7.373)%
    --(4.084,7.372)--(4.082,7.370)--(4.079,7.369)--(4.077,7.367)--(4.075,7.365)%
    --(4.072,7.364)--(4.070,7.362)--(4.068,7.360)--(4.067,7.357)--(4.065,7.355)%
    --(4.063,7.353)--(4.062,7.350)--(4.060,7.348)--(4.059,7.345)--(4.058,7.343)%
    --(4.057,7.340)--(4.056,7.337)--(4.056,7.335)--(4.055,7.332)--(4.055,7.329)%
    --(4.055,7.326)--(4.055,7.324)--(4.055,7.321)--(4.055,7.318)--(4.055,7.315)%
    --(4.056,7.312)--(4.056,7.310)--(4.057,7.307)--(4.058,7.304)--(4.059,7.302)%
    --(4.060,7.299)--(4.062,7.296)--(4.063,7.294)--(4.065,7.292)--(4.067,7.290)%
    --(4.068,7.287)--(4.070,7.285)--(4.072,7.283)--(4.075,7.282)--(4.077,7.280)%
    --(4.079,7.278)--(4.081,7.277)--(4.084,7.275)--(4.087,7.274)--(4.089,7.273)%
    --(4.092,7.272)--(4.095,7.271)--(4.097,7.271)--(4.100,7.270)--(4.103,7.270)%
    --(4.106,7.270)--(4.108,7.270)--(4.111,7.270)--(4.114,7.270)--(4.117,7.270)%
    --(4.120,7.271)--(4.122,7.271)--(4.125,7.272)--(4.128,7.273)--(4.130,7.274)%
    --(4.133,7.275)--(4.136,7.277)--(4.138,7.278)--(4.140,7.280)--(4.142,7.282)%
    --(4.145,7.283)--(4.147,7.285)--(4.149,7.287)--(4.150,7.290)--(4.152,7.292)%
    --(4.154,7.294)--(4.155,7.296)--(4.157,7.299)--(4.158,7.302)--(4.159,7.304)%
    --(4.160,7.307)--(4.161,7.310)--(4.161,7.312)--(4.162,7.315)--(4.162,7.318)%
    --(4.162,7.321)--(4.163,7.323)--cycle;
\gpfill{color=gp lt color border,opacity=0.50} (7.595,4.155)--(7.594,4.167)--(7.593,4.180)--(7.592,4.193)%
    --(7.589,4.205)--(7.586,4.217)--(7.583,4.230)--(7.578,4.242)--(7.573,4.253)%
    --(7.568,4.265)--(7.562,4.276)--(7.555,4.287)--(7.548,4.297)--(7.540,4.307)%
    --(7.532,4.317)--(7.523,4.326)--(7.514,4.335)--(7.504,4.343)--(7.494,4.351)%
    --(7.484,4.358)--(7.473,4.365)--(7.462,4.371)--(7.450,4.376)--(7.439,4.381)%
    --(7.427,4.386)--(7.414,4.389)--(7.402,4.392)--(7.390,4.395)--(7.377,4.396)%
    --(7.364,4.397)--(7.352,4.398)--(7.339,4.397)--(7.326,4.396)--(7.313,4.395)%
    --(7.301,4.392)--(7.289,4.389)--(7.276,4.386)--(7.264,4.381)--(7.253,4.376)%
    --(7.241,4.371)--(7.230,4.365)--(7.219,4.358)--(7.209,4.351)--(7.199,4.343)%
    --(7.189,4.335)--(7.180,4.326)--(7.171,4.317)--(7.163,4.307)--(7.155,4.297)%
    --(7.148,4.287)--(7.141,4.276)--(7.135,4.265)--(7.130,4.253)--(7.125,4.242)%
    --(7.120,4.230)--(7.117,4.217)--(7.114,4.205)--(7.111,4.193)--(7.110,4.180)%
    --(7.109,4.167)--(7.109,4.155)--(7.109,4.142)--(7.110,4.129)--(7.111,4.116)%
    --(7.114,4.104)--(7.117,4.092)--(7.120,4.079)--(7.125,4.067)--(7.130,4.056)%
    --(7.135,4.044)--(7.141,4.033)--(7.148,4.022)--(7.155,4.012)--(7.163,4.002)%
    --(7.171,3.992)--(7.180,3.983)--(7.189,3.974)--(7.199,3.966)--(7.209,3.958)%
    --(7.219,3.951)--(7.230,3.944)--(7.241,3.938)--(7.253,3.933)--(7.264,3.928)%
    --(7.276,3.923)--(7.289,3.920)--(7.301,3.917)--(7.313,3.914)--(7.326,3.913)%
    --(7.339,3.912)--(7.351,3.912)--(7.364,3.912)--(7.377,3.913)--(7.390,3.914)%
    --(7.402,3.917)--(7.414,3.920)--(7.427,3.923)--(7.439,3.928)--(7.450,3.933)%
    --(7.462,3.938)--(7.473,3.944)--(7.484,3.951)--(7.494,3.958)--(7.504,3.966)%
    --(7.514,3.974)--(7.523,3.983)--(7.532,3.992)--(7.540,4.002)--(7.548,4.012)%
    --(7.555,4.022)--(7.562,4.033)--(7.568,4.044)--(7.573,4.056)--(7.578,4.067)%
    --(7.583,4.079)--(7.586,4.092)--(7.589,4.104)--(7.592,4.116)--(7.593,4.129)%
    --(7.594,4.142)--(7.595,4.154)--cycle;
\gpfill{color=gp lt color border,opacity=0.50} (10.947,4.155)--(10.946,4.159)--(10.946,4.163)--(10.946,4.167)%
    --(10.945,4.171)--(10.944,4.175)--(10.943,4.180)--(10.941,4.184)--(10.939,4.187)%
    --(10.938,4.191)--(10.936,4.195)--(10.933,4.199)--(10.931,4.202)--(10.928,4.205)%
    --(10.926,4.209)--(10.923,4.212)--(10.920,4.215)--(10.916,4.217)--(10.913,4.220)%
    --(10.910,4.222)--(10.906,4.225)--(10.902,4.227)--(10.898,4.228)--(10.895,4.230)%
    --(10.891,4.232)--(10.886,4.233)--(10.882,4.234)--(10.878,4.235)--(10.874,4.235)%
    --(10.870,4.235)--(10.866,4.236)--(10.861,4.235)--(10.857,4.235)--(10.853,4.235)%
    --(10.849,4.234)--(10.845,4.233)--(10.840,4.232)--(10.836,4.230)--(10.833,4.228)%
    --(10.829,4.227)--(10.825,4.225)--(10.821,4.222)--(10.818,4.220)--(10.815,4.217)%
    --(10.811,4.215)--(10.808,4.212)--(10.805,4.209)--(10.803,4.205)--(10.800,4.202)%
    --(10.798,4.199)--(10.795,4.195)--(10.793,4.191)--(10.792,4.187)--(10.790,4.184)%
    --(10.788,4.180)--(10.787,4.175)--(10.786,4.171)--(10.785,4.167)--(10.785,4.163)%
    --(10.785,4.159)--(10.785,4.155)--(10.785,4.150)--(10.785,4.146)--(10.785,4.142)%
    --(10.786,4.138)--(10.787,4.134)--(10.788,4.129)--(10.790,4.125)--(10.792,4.122)%
    --(10.793,4.118)--(10.795,4.114)--(10.798,4.110)--(10.800,4.107)--(10.803,4.104)%
    --(10.805,4.100)--(10.808,4.097)--(10.811,4.094)--(10.815,4.092)--(10.818,4.089)%
    --(10.821,4.087)--(10.825,4.084)--(10.829,4.082)--(10.833,4.081)--(10.836,4.079)%
    --(10.840,4.077)--(10.845,4.076)--(10.849,4.075)--(10.853,4.074)--(10.857,4.074)%
    --(10.861,4.074)--(10.865,4.074)--(10.870,4.074)--(10.874,4.074)--(10.878,4.074)%
    --(10.882,4.075)--(10.886,4.076)--(10.891,4.077)--(10.895,4.079)--(10.898,4.081)%
    --(10.902,4.082)--(10.906,4.084)--(10.910,4.087)--(10.913,4.089)--(10.916,4.092)%
    --(10.920,4.094)--(10.923,4.097)--(10.926,4.100)--(10.928,4.104)--(10.931,4.107)%
    --(10.933,4.110)--(10.936,4.114)--(10.938,4.118)--(10.939,4.122)--(10.941,4.125)%
    --(10.943,4.129)--(10.944,4.134)--(10.945,4.138)--(10.946,4.142)--(10.946,4.146)%
    --(10.946,4.150)--(10.947,4.154)--cycle;
\gpfill{color=gp lt color border,opacity=0.50} (4.948,4.155)--(4.947,4.170)--(4.946,4.186)--(4.944,4.201)%
    --(4.941,4.216)--(4.937,4.232)--(4.933,4.247)--(4.928,4.261)--(4.922,4.276)%
    --(4.915,4.290)--(4.908,4.303)--(4.899,4.317)--(4.891,4.330)--(4.881,4.342)%
    --(4.871,4.354)--(4.860,4.365)--(4.849,4.376)--(4.837,4.386)--(4.825,4.396)%
    --(4.812,4.404)--(4.799,4.413)--(4.785,4.420)--(4.771,4.427)--(4.756,4.433)%
    --(4.742,4.438)--(4.727,4.442)--(4.711,4.446)--(4.696,4.449)--(4.681,4.451)%
    --(4.665,4.452)--(4.650,4.453)--(4.634,4.452)--(4.618,4.451)--(4.603,4.449)%
    --(4.588,4.446)--(4.572,4.442)--(4.557,4.438)--(4.543,4.433)--(4.528,4.427)%
    --(4.514,4.420)--(4.501,4.413)--(4.487,4.404)--(4.474,4.396)--(4.462,4.386)%
    --(4.450,4.376)--(4.439,4.365)--(4.428,4.354)--(4.418,4.342)--(4.408,4.330)%
    --(4.400,4.317)--(4.391,4.303)--(4.384,4.290)--(4.377,4.276)--(4.371,4.261)%
    --(4.366,4.247)--(4.362,4.232)--(4.358,4.216)--(4.355,4.201)--(4.353,4.186)%
    --(4.352,4.170)--(4.352,4.155)--(4.352,4.139)--(4.353,4.123)--(4.355,4.108)%
    --(4.358,4.093)--(4.362,4.077)--(4.366,4.062)--(4.371,4.048)--(4.377,4.033)%
    --(4.384,4.019)--(4.391,4.005)--(4.400,3.992)--(4.408,3.979)--(4.418,3.967)%
    --(4.428,3.955)--(4.439,3.944)--(4.450,3.933)--(4.462,3.923)--(4.474,3.913)%
    --(4.487,3.905)--(4.500,3.896)--(4.514,3.889)--(4.528,3.882)--(4.543,3.876)%
    --(4.557,3.871)--(4.572,3.867)--(4.588,3.863)--(4.603,3.860)--(4.618,3.858)%
    --(4.634,3.857)--(4.649,3.857)--(4.665,3.857)--(4.681,3.858)--(4.696,3.860)%
    --(4.711,3.863)--(4.727,3.867)--(4.742,3.871)--(4.756,3.876)--(4.771,3.882)%
    --(4.785,3.889)--(4.799,3.896)--(4.812,3.905)--(4.825,3.913)--(4.837,3.923)%
    --(4.849,3.933)--(4.860,3.944)--(4.871,3.955)--(4.881,3.967)--(4.891,3.979)%
    --(4.899,3.992)--(4.908,4.005)--(4.915,4.019)--(4.922,4.033)--(4.928,4.048)%
    --(4.933,4.062)--(4.937,4.077)--(4.941,4.093)--(4.944,4.108)--(4.946,4.123)%
    --(4.947,4.139)--(4.948,4.154)--cycle;
\gpfill{color=gp lt color border,opacity=0.50} (7.379,3.098)--(7.378,3.099)--(7.378,3.100)--(7.378,3.102)%
    --(7.378,3.103)--(7.378,3.104)--(7.377,3.106)--(7.377,3.107)--(7.376,3.108)%
    --(7.376,3.110)--(7.375,3.111)--(7.374,3.112)--(7.373,3.113)--(7.372,3.114)%
    --(7.372,3.116)--(7.371,3.117)--(7.370,3.118)--(7.368,3.118)--(7.367,3.119)%
    --(7.366,3.120)--(7.365,3.121)--(7.364,3.122)--(7.362,3.122)--(7.361,3.123)%
    --(7.360,3.123)--(7.358,3.124)--(7.357,3.124)--(7.356,3.124)--(7.354,3.124)%
    --(7.353,3.124)--(7.352,3.125)--(7.350,3.124)--(7.349,3.124)--(7.347,3.124)%
    --(7.346,3.124)--(7.345,3.124)--(7.343,3.123)--(7.342,3.123)--(7.341,3.122)%
    --(7.339,3.122)--(7.338,3.121)--(7.337,3.120)--(7.336,3.119)--(7.335,3.118)%
    --(7.333,3.118)--(7.332,3.117)--(7.331,3.116)--(7.331,3.114)--(7.330,3.113)%
    --(7.329,3.112)--(7.328,3.111)--(7.327,3.110)--(7.327,3.108)--(7.326,3.107)%
    --(7.326,3.106)--(7.325,3.104)--(7.325,3.103)--(7.325,3.102)--(7.325,3.100)%
    --(7.325,3.099)--(7.325,3.098)--(7.325,3.096)--(7.325,3.095)--(7.325,3.093)%
    --(7.325,3.092)--(7.325,3.091)--(7.326,3.089)--(7.326,3.088)--(7.327,3.087)%
    --(7.327,3.085)--(7.328,3.084)--(7.329,3.083)--(7.330,3.082)--(7.331,3.081)%
    --(7.331,3.079)--(7.332,3.078)--(7.333,3.077)--(7.335,3.077)--(7.336,3.076)%
    --(7.337,3.075)--(7.338,3.074)--(7.339,3.073)--(7.341,3.073)--(7.342,3.072)%
    --(7.343,3.072)--(7.345,3.071)--(7.346,3.071)--(7.347,3.071)--(7.349,3.071)%
    --(7.350,3.071)--(7.351,3.071)--(7.353,3.071)--(7.354,3.071)--(7.356,3.071)%
    --(7.357,3.071)--(7.358,3.071)--(7.360,3.072)--(7.361,3.072)--(7.362,3.073)%
    --(7.364,3.073)--(7.365,3.074)--(7.366,3.075)--(7.367,3.076)--(7.368,3.077)%
    --(7.370,3.077)--(7.371,3.078)--(7.372,3.079)--(7.372,3.081)--(7.373,3.082)%
    --(7.374,3.083)--(7.375,3.084)--(7.376,3.085)--(7.376,3.087)--(7.377,3.088)%
    --(7.377,3.089)--(7.378,3.091)--(7.378,3.092)--(7.378,3.093)--(7.378,3.095)%
    --(7.378,3.096)--(7.379,3.097)--cycle;
\gpfill{color=gp lt color border,opacity=0.50} (10.109,6.268)--(10.108,6.270)--(10.108,6.273)--(10.108,6.276)%
    --(10.107,6.279)--(10.107,6.281)--(10.106,6.284)--(10.105,6.287)--(10.104,6.289)%
    --(10.103,6.292)--(10.101,6.294)--(10.100,6.297)--(10.098,6.299)--(10.096,6.301)%
    --(10.095,6.304)--(10.093,6.306)--(10.091,6.308)--(10.088,6.309)--(10.086,6.311)%
    --(10.084,6.313)--(10.082,6.314)--(10.079,6.316)--(10.076,6.317)--(10.074,6.318)%
    --(10.071,6.319)--(10.068,6.320)--(10.066,6.320)--(10.063,6.321)--(10.060,6.321)%
    --(10.057,6.321)--(10.055,6.322)--(10.052,6.321)--(10.049,6.321)--(10.046,6.321)%
    --(10.043,6.320)--(10.041,6.320)--(10.038,6.319)--(10.035,6.318)--(10.033,6.317)%
    --(10.030,6.316)--(10.028,6.314)--(10.025,6.313)--(10.023,6.311)--(10.021,6.309)%
    --(10.018,6.308)--(10.016,6.306)--(10.014,6.304)--(10.013,6.301)--(10.011,6.299)%
    --(10.009,6.297)--(10.008,6.294)--(10.006,6.292)--(10.005,6.289)--(10.004,6.287)%
    --(10.003,6.284)--(10.002,6.281)--(10.002,6.279)--(10.001,6.276)--(10.001,6.273)%
    --(10.001,6.270)--(10.001,6.268)--(10.001,6.265)--(10.001,6.262)--(10.001,6.259)%
    --(10.002,6.256)--(10.002,6.254)--(10.003,6.251)--(10.004,6.248)--(10.005,6.246)%
    --(10.006,6.243)--(10.008,6.240)--(10.009,6.238)--(10.011,6.236)--(10.013,6.234)%
    --(10.014,6.231)--(10.016,6.229)--(10.018,6.227)--(10.021,6.226)--(10.023,6.224)%
    --(10.025,6.222)--(10.027,6.221)--(10.030,6.219)--(10.033,6.218)--(10.035,6.217)%
    --(10.038,6.216)--(10.041,6.215)--(10.043,6.215)--(10.046,6.214)--(10.049,6.214)%
    --(10.052,6.214)--(10.054,6.214)--(10.057,6.214)--(10.060,6.214)--(10.063,6.214)%
    --(10.066,6.215)--(10.068,6.215)--(10.071,6.216)--(10.074,6.217)--(10.076,6.218)%
    --(10.079,6.219)--(10.082,6.221)--(10.084,6.222)--(10.086,6.224)--(10.088,6.226)%
    --(10.091,6.227)--(10.093,6.229)--(10.095,6.231)--(10.096,6.234)--(10.098,6.236)%
    --(10.100,6.238)--(10.101,6.240)--(10.103,6.243)--(10.104,6.246)--(10.105,6.248)%
    --(10.106,6.251)--(10.107,6.254)--(10.107,6.256)--(10.108,6.259)--(10.108,6.262)%
    --(10.108,6.265)--(10.109,6.267)--cycle;
\gpfill{color=gp lt color border,opacity=0.50} (8.379,4.155)--(8.378,4.166)--(8.377,4.177)--(8.376,4.188)%
    --(8.374,4.199)--(8.371,4.210)--(8.368,4.221)--(8.364,4.232)--(8.360,4.242)%
    --(8.355,4.253)--(8.350,4.262)--(8.344,4.272)--(8.337,4.281)--(8.330,4.290)%
    --(8.323,4.299)--(8.315,4.307)--(8.307,4.315)--(8.298,4.322)--(8.289,4.329)%
    --(8.280,4.336)--(8.271,4.342)--(8.261,4.347)--(8.250,4.352)--(8.240,4.356)%
    --(8.229,4.360)--(8.218,4.363)--(8.207,4.366)--(8.196,4.368)--(8.185,4.369)%
    --(8.174,4.370)--(8.163,4.371)--(8.151,4.370)--(8.140,4.369)--(8.129,4.368)%
    --(8.118,4.366)--(8.107,4.363)--(8.096,4.360)--(8.085,4.356)--(8.075,4.352)%
    --(8.064,4.347)--(8.055,4.342)--(8.045,4.336)--(8.036,4.329)--(8.027,4.322)%
    --(8.018,4.315)--(8.010,4.307)--(8.002,4.299)--(7.995,4.290)--(7.988,4.281)%
    --(7.981,4.272)--(7.975,4.262)--(7.970,4.253)--(7.965,4.242)--(7.961,4.232)%
    --(7.957,4.221)--(7.954,4.210)--(7.951,4.199)--(7.949,4.188)--(7.948,4.177)%
    --(7.947,4.166)--(7.947,4.155)--(7.947,4.143)--(7.948,4.132)--(7.949,4.121)%
    --(7.951,4.110)--(7.954,4.099)--(7.957,4.088)--(7.961,4.077)--(7.965,4.067)%
    --(7.970,4.056)--(7.975,4.046)--(7.981,4.037)--(7.988,4.028)--(7.995,4.019)%
    --(8.002,4.010)--(8.010,4.002)--(8.018,3.994)--(8.027,3.987)--(8.036,3.980)%
    --(8.045,3.973)--(8.054,3.967)--(8.064,3.962)--(8.075,3.957)--(8.085,3.953)%
    --(8.096,3.949)--(8.107,3.946)--(8.118,3.943)--(8.129,3.941)--(8.140,3.940)%
    --(8.151,3.939)--(8.162,3.939)--(8.174,3.939)--(8.185,3.940)--(8.196,3.941)%
    --(8.207,3.943)--(8.218,3.946)--(8.229,3.949)--(8.240,3.953)--(8.250,3.957)%
    --(8.261,3.962)--(8.271,3.967)--(8.280,3.973)--(8.289,3.980)--(8.298,3.987)%
    --(8.307,3.994)--(8.315,4.002)--(8.323,4.010)--(8.330,4.019)--(8.337,4.028)%
    --(8.344,4.037)--(8.350,4.046)--(8.355,4.056)--(8.360,4.067)--(8.364,4.077)%
    --(8.368,4.088)--(8.371,4.099)--(8.374,4.110)--(8.376,4.121)--(8.377,4.132)%
    --(8.378,4.143)--(8.379,4.154)--cycle;
\gpfill{color=gp lt color border,opacity=0.50} (8.785,6.268)--(8.784,6.272)--(8.784,6.276)--(8.784,6.280)%
    --(8.783,6.284)--(8.782,6.288)--(8.781,6.293)--(8.779,6.297)--(8.777,6.300)%
    --(8.776,6.304)--(8.774,6.308)--(8.771,6.312)--(8.769,6.315)--(8.766,6.318)%
    --(8.764,6.322)--(8.761,6.325)--(8.758,6.328)--(8.754,6.330)--(8.751,6.333)%
    --(8.748,6.335)--(8.744,6.338)--(8.740,6.340)--(8.736,6.341)--(8.733,6.343)%
    --(8.729,6.345)--(8.724,6.346)--(8.720,6.347)--(8.716,6.348)--(8.712,6.348)%
    --(8.708,6.348)--(8.704,6.349)--(8.699,6.348)--(8.695,6.348)--(8.691,6.348)%
    --(8.687,6.347)--(8.683,6.346)--(8.678,6.345)--(8.674,6.343)--(8.671,6.341)%
    --(8.667,6.340)--(8.663,6.338)--(8.659,6.335)--(8.656,6.333)--(8.653,6.330)%
    --(8.649,6.328)--(8.646,6.325)--(8.643,6.322)--(8.641,6.318)--(8.638,6.315)%
    --(8.636,6.312)--(8.633,6.308)--(8.631,6.304)--(8.630,6.300)--(8.628,6.297)%
    --(8.626,6.293)--(8.625,6.288)--(8.624,6.284)--(8.623,6.280)--(8.623,6.276)%
    --(8.623,6.272)--(8.623,6.268)--(8.623,6.263)--(8.623,6.259)--(8.623,6.255)%
    --(8.624,6.251)--(8.625,6.247)--(8.626,6.242)--(8.628,6.238)--(8.630,6.235)%
    --(8.631,6.231)--(8.633,6.227)--(8.636,6.223)--(8.638,6.220)--(8.641,6.217)%
    --(8.643,6.213)--(8.646,6.210)--(8.649,6.207)--(8.653,6.205)--(8.656,6.202)%
    --(8.659,6.200)--(8.663,6.197)--(8.667,6.195)--(8.671,6.194)--(8.674,6.192)%
    --(8.678,6.190)--(8.683,6.189)--(8.687,6.188)--(8.691,6.187)--(8.695,6.187)%
    --(8.699,6.187)--(8.703,6.187)--(8.708,6.187)--(8.712,6.187)--(8.716,6.187)%
    --(8.720,6.188)--(8.724,6.189)--(8.729,6.190)--(8.733,6.192)--(8.736,6.194)%
    --(8.740,6.195)--(8.744,6.197)--(8.748,6.200)--(8.751,6.202)--(8.754,6.205)%
    --(8.758,6.207)--(8.761,6.210)--(8.764,6.213)--(8.766,6.217)--(8.769,6.220)%
    --(8.771,6.223)--(8.774,6.227)--(8.776,6.231)--(8.777,6.235)--(8.779,6.238)%
    --(8.781,6.242)--(8.782,6.247)--(8.783,6.251)--(8.784,6.255)--(8.784,6.259)%
    --(8.784,6.263)--(8.785,6.267)--cycle;
\gpfill{color=gp lt color border,opacity=0.50} (7.487,6.268)--(7.486,6.275)--(7.486,6.282)--(7.485,6.289)%
    --(7.484,6.296)--(7.482,6.302)--(7.480,6.309)--(7.478,6.316)--(7.475,6.322)%
    --(7.472,6.329)--(7.468,6.335)--(7.465,6.341)--(7.461,6.347)--(7.456,6.352)%
    --(7.452,6.358)--(7.447,6.363)--(7.442,6.368)--(7.436,6.372)--(7.431,6.377)%
    --(7.425,6.381)--(7.419,6.384)--(7.413,6.388)--(7.406,6.391)--(7.400,6.394)%
    --(7.393,6.396)--(7.386,6.398)--(7.380,6.400)--(7.373,6.401)--(7.366,6.402)%
    --(7.359,6.402)--(7.352,6.403)--(7.344,6.402)--(7.337,6.402)--(7.330,6.401)%
    --(7.323,6.400)--(7.317,6.398)--(7.310,6.396)--(7.303,6.394)--(7.297,6.391)%
    --(7.290,6.388)--(7.284,6.384)--(7.278,6.381)--(7.272,6.377)--(7.267,6.372)%
    --(7.261,6.368)--(7.256,6.363)--(7.251,6.358)--(7.247,6.352)--(7.242,6.347)%
    --(7.238,6.341)--(7.235,6.335)--(7.231,6.329)--(7.228,6.322)--(7.225,6.316)%
    --(7.223,6.309)--(7.221,6.302)--(7.219,6.296)--(7.218,6.289)--(7.217,6.282)%
    --(7.217,6.275)--(7.217,6.268)--(7.217,6.260)--(7.217,6.253)--(7.218,6.246)%
    --(7.219,6.239)--(7.221,6.233)--(7.223,6.226)--(7.225,6.219)--(7.228,6.213)%
    --(7.231,6.206)--(7.235,6.200)--(7.238,6.194)--(7.242,6.188)--(7.247,6.183)%
    --(7.251,6.177)--(7.256,6.172)--(7.261,6.167)--(7.267,6.163)--(7.272,6.158)%
    --(7.278,6.154)--(7.284,6.151)--(7.290,6.147)--(7.297,6.144)--(7.303,6.141)%
    --(7.310,6.139)--(7.317,6.137)--(7.323,6.135)--(7.330,6.134)--(7.337,6.133)%
    --(7.344,6.133)--(7.351,6.133)--(7.359,6.133)--(7.366,6.133)--(7.373,6.134)%
    --(7.380,6.135)--(7.386,6.137)--(7.393,6.139)--(7.400,6.141)--(7.406,6.144)%
    --(7.413,6.147)--(7.419,6.151)--(7.425,6.154)--(7.431,6.158)--(7.436,6.163)%
    --(7.442,6.167)--(7.447,6.172)--(7.452,6.177)--(7.456,6.183)--(7.461,6.188)%
    --(7.465,6.194)--(7.468,6.200)--(7.472,6.206)--(7.475,6.213)--(7.478,6.219)%
    --(7.480,6.226)--(7.482,6.233)--(7.484,6.239)--(7.485,6.246)--(7.486,6.253)%
    --(7.486,6.260)--(7.487,6.267)--cycle;
\gpfill{color=gp lt color border,opacity=0.50} (10.163,5.211)--(10.162,5.216)--(10.162,5.222)--(10.161,5.227)%
    --(10.160,5.233)--(10.159,5.238)--(10.157,5.244)--(10.155,5.249)--(10.153,5.254)%
    --(10.151,5.260)--(10.148,5.264)--(10.145,5.269)--(10.142,5.274)--(10.138,5.278)%
    --(10.135,5.283)--(10.131,5.287)--(10.127,5.291)--(10.122,5.294)--(10.118,5.298)%
    --(10.113,5.301)--(10.109,5.304)--(10.104,5.307)--(10.098,5.309)--(10.093,5.311)%
    --(10.088,5.313)--(10.082,5.315)--(10.077,5.316)--(10.071,5.317)--(10.066,5.318)%
    --(10.060,5.318)--(10.055,5.319)--(10.049,5.318)--(10.043,5.318)--(10.038,5.317)%
    --(10.032,5.316)--(10.027,5.315)--(10.021,5.313)--(10.016,5.311)--(10.011,5.309)%
    --(10.005,5.307)--(10.001,5.304)--(9.996,5.301)--(9.991,5.298)--(9.987,5.294)%
    --(9.982,5.291)--(9.978,5.287)--(9.974,5.283)--(9.971,5.278)--(9.967,5.274)%
    --(9.964,5.269)--(9.961,5.264)--(9.958,5.260)--(9.956,5.254)--(9.954,5.249)%
    --(9.952,5.244)--(9.950,5.238)--(9.949,5.233)--(9.948,5.227)--(9.947,5.222)%
    --(9.947,5.216)--(9.947,5.211)--(9.947,5.205)--(9.947,5.199)--(9.948,5.194)%
    --(9.949,5.188)--(9.950,5.183)--(9.952,5.177)--(9.954,5.172)--(9.956,5.167)%
    --(9.958,5.161)--(9.961,5.156)--(9.964,5.152)--(9.967,5.147)--(9.971,5.143)%
    --(9.974,5.138)--(9.978,5.134)--(9.982,5.130)--(9.987,5.127)--(9.991,5.123)%
    --(9.996,5.120)--(10.000,5.117)--(10.005,5.114)--(10.011,5.112)--(10.016,5.110)%
    --(10.021,5.108)--(10.027,5.106)--(10.032,5.105)--(10.038,5.104)--(10.043,5.103)%
    --(10.049,5.103)--(10.054,5.103)--(10.060,5.103)--(10.066,5.103)--(10.071,5.104)%
    --(10.077,5.105)--(10.082,5.106)--(10.088,5.108)--(10.093,5.110)--(10.098,5.112)%
    --(10.104,5.114)--(10.109,5.117)--(10.113,5.120)--(10.118,5.123)--(10.122,5.127)%
    --(10.127,5.130)--(10.131,5.134)--(10.135,5.138)--(10.138,5.143)--(10.142,5.147)%
    --(10.145,5.152)--(10.148,5.156)--(10.151,5.161)--(10.153,5.167)--(10.155,5.172)%
    --(10.157,5.177)--(10.159,5.183)--(10.160,5.188)--(10.161,5.194)--(10.162,5.199)%
    --(10.162,5.205)--(10.163,5.210)--cycle;
\gpfill{color=gp lt color border,opacity=0.50} (5.757,4.155)--(5.756,4.170)--(5.755,4.186)--(5.753,4.201)%
    --(5.750,4.216)--(5.746,4.231)--(5.742,4.246)--(5.737,4.261)--(5.731,4.275)%
    --(5.724,4.289)--(5.717,4.303)--(5.709,4.316)--(5.700,4.329)--(5.690,4.341)%
    --(5.680,4.353)--(5.670,4.365)--(5.658,4.375)--(5.646,4.385)--(5.634,4.395)%
    --(5.621,4.404)--(5.608,4.412)--(5.594,4.419)--(5.580,4.426)--(5.566,4.432)%
    --(5.551,4.437)--(5.536,4.441)--(5.521,4.445)--(5.506,4.448)--(5.491,4.450)%
    --(5.475,4.451)--(5.460,4.452)--(5.444,4.451)--(5.428,4.450)--(5.413,4.448)%
    --(5.398,4.445)--(5.383,4.441)--(5.368,4.437)--(5.353,4.432)--(5.339,4.426)%
    --(5.325,4.419)--(5.311,4.412)--(5.298,4.404)--(5.285,4.395)--(5.273,4.385)%
    --(5.261,4.375)--(5.249,4.365)--(5.239,4.353)--(5.229,4.341)--(5.219,4.329)%
    --(5.210,4.316)--(5.202,4.303)--(5.195,4.289)--(5.188,4.275)--(5.182,4.261)%
    --(5.177,4.246)--(5.173,4.231)--(5.169,4.216)--(5.166,4.201)--(5.164,4.186)%
    --(5.163,4.170)--(5.163,4.155)--(5.163,4.139)--(5.164,4.123)--(5.166,4.108)%
    --(5.169,4.093)--(5.173,4.078)--(5.177,4.063)--(5.182,4.048)--(5.188,4.034)%
    --(5.195,4.020)--(5.202,4.006)--(5.210,3.993)--(5.219,3.980)--(5.229,3.968)%
    --(5.239,3.956)--(5.249,3.944)--(5.261,3.934)--(5.273,3.924)--(5.285,3.914)%
    --(5.298,3.905)--(5.311,3.897)--(5.325,3.890)--(5.339,3.883)--(5.353,3.877)%
    --(5.368,3.872)--(5.383,3.868)--(5.398,3.864)--(5.413,3.861)--(5.428,3.859)%
    --(5.444,3.858)--(5.459,3.858)--(5.475,3.858)--(5.491,3.859)--(5.506,3.861)%
    --(5.521,3.864)--(5.536,3.868)--(5.551,3.872)--(5.566,3.877)--(5.580,3.883)%
    --(5.594,3.890)--(5.608,3.897)--(5.621,3.905)--(5.634,3.914)--(5.646,3.924)%
    --(5.658,3.934)--(5.670,3.944)--(5.680,3.956)--(5.690,3.968)--(5.700,3.980)%
    --(5.709,3.993)--(5.717,4.006)--(5.724,4.020)--(5.731,4.034)--(5.737,4.048)%
    --(5.742,4.063)--(5.746,4.078)--(5.750,4.093)--(5.753,4.108)--(5.755,4.123)%
    --(5.756,4.139)--(5.757,4.154)--cycle;
\gpfill{color=gp lt color border,opacity=0.50} (8.163,3.098)--(8.163,3.098)--(8.163,3.098)--(8.163,3.098)%
    --(8.163,3.098)--(8.163,3.098)--(8.163,3.098)--(8.163,3.098)--(8.163,3.098)%
    --(8.163,3.098)--(8.163,3.098)--(8.163,3.098)--(8.163,3.098)--(8.163,3.098)%
    --(8.163,3.098)--(8.163,3.098)--(8.163,3.098)--(8.163,3.098)--(8.163,3.098)%
    --(8.163,3.098)--(8.163,3.098)--(8.163,3.098)--(8.163,3.098)--(8.163,3.098)%
    --(8.163,3.098)--(8.163,3.098)--(8.163,3.098)--(8.163,3.098)--(8.163,3.098)%
    --(8.163,3.098)--(8.163,3.098)--(8.163,3.098)--(8.163,3.098)--(8.163,3.098)%
    --(8.163,3.098)--(8.163,3.098)--(8.163,3.098)--(8.163,3.098)--(8.163,3.098)%
    --(8.163,3.098)--(8.163,3.098)--(8.163,3.098)--(8.163,3.098)--(8.163,3.098)%
    --(8.163,3.098)--(8.163,3.098)--(8.163,3.098)--(8.163,3.098)--(8.163,3.098)%
    --(8.163,3.098)--(8.163,3.098)--(8.163,3.098)--(8.163,3.098)--(8.163,3.098)%
    --(8.163,3.098)--(8.163,3.098)--(8.163,3.098)--(8.163,3.098)--(8.163,3.098)%
    --(8.163,3.098)--(8.163,3.098)--(8.163,3.098)--(8.163,3.098)--(8.163,3.098)%
    --(8.163,3.098)--(8.163,3.098)--(8.163,3.098)--(8.163,3.098)--(8.163,3.098)%
    --(8.163,3.098)--(8.163,3.098)--(8.163,3.098)--(8.163,3.098)--(8.163,3.098)%
    --(8.163,3.098)--(8.163,3.098)--(8.163,3.098)--(8.163,3.098)--(8.163,3.098)%
    --(8.163,3.098)--(8.163,3.098)--(8.163,3.098)--(8.163,3.098)--(8.163,3.098)%
    --(8.163,3.098)--(8.163,3.098)--(8.163,3.098)--(8.163,3.098)--(8.163,3.098)%
    --(8.163,3.098)--(8.163,3.098)--(8.163,3.098)--(8.163,3.098)--(8.163,3.098)%
    --(8.163,3.098)--(8.163,3.098)--(8.163,3.098)--(8.163,3.098)--(8.163,3.098)%
    --(8.163,3.098)--(8.163,3.098)--(8.163,3.098)--(8.163,3.098)--(8.163,3.098)%
    --(8.163,3.098)--(8.163,3.098)--(8.163,3.098)--(8.163,3.098)--(8.163,3.098)%
    --(8.163,3.098)--(8.163,3.098)--(8.163,3.098)--(8.163,3.098)--(8.163,3.098)%
    --(8.163,3.098)--(8.163,3.098)--(8.163,3.098)--(8.163,3.098)--(8.163,3.098)%
    --(8.163,3.098)--cycle;
\gpfill{color=gp lt color border,opacity=0.50} (2.380,4.155)--(2.379,4.177)--(2.377,4.200)--(2.374,4.222)%
    --(2.370,4.245)--(2.365,4.267)--(2.358,4.288)--(2.351,4.310)--(2.342,4.331)%
    --(2.332,4.351)--(2.321,4.371)--(2.310,4.390)--(2.297,4.409)--(2.283,4.427)%
    --(2.268,4.444)--(2.253,4.461)--(2.236,4.476)--(2.219,4.491)--(2.201,4.505)%
    --(2.182,4.518)--(2.163,4.529)--(2.143,4.540)--(2.123,4.550)--(2.102,4.559)%
    --(2.080,4.566)--(2.059,4.573)--(2.037,4.578)--(2.014,4.582)--(1.992,4.585)%
    --(1.969,4.587)--(1.947,4.588)--(1.924,4.587)--(1.901,4.585)--(1.879,4.582)%
    --(1.856,4.578)--(1.834,4.573)--(1.813,4.566)--(1.791,4.559)--(1.770,4.550)%
    --(1.750,4.540)--(1.730,4.529)--(1.711,4.518)--(1.692,4.505)--(1.674,4.491)%
    --(1.657,4.476)--(1.640,4.461)--(1.625,4.444)--(1.610,4.427)--(1.596,4.409)%
    --(1.583,4.390)--(1.572,4.371)--(1.561,4.351)--(1.551,4.331)--(1.542,4.310)%
    --(1.535,4.288)--(1.528,4.267)--(1.523,4.245)--(1.519,4.222)--(1.516,4.200)%
    --(1.514,4.177)--(1.514,4.155)--(1.514,4.132)--(1.516,4.109)--(1.519,4.087)%
    --(1.523,4.064)--(1.528,4.042)--(1.535,4.021)--(1.542,3.999)--(1.551,3.978)%
    --(1.561,3.958)--(1.572,3.938)--(1.583,3.919)--(1.596,3.900)--(1.610,3.882)%
    --(1.625,3.865)--(1.640,3.848)--(1.657,3.833)--(1.674,3.818)--(1.692,3.804)%
    --(1.711,3.791)--(1.730,3.780)--(1.750,3.769)--(1.770,3.759)--(1.791,3.750)%
    --(1.813,3.743)--(1.834,3.736)--(1.856,3.731)--(1.879,3.727)--(1.901,3.724)%
    --(1.924,3.722)--(1.946,3.722)--(1.969,3.722)--(1.992,3.724)--(2.014,3.727)%
    --(2.037,3.731)--(2.059,3.736)--(2.080,3.743)--(2.102,3.750)--(2.123,3.759)%
    --(2.143,3.769)--(2.163,3.780)--(2.182,3.791)--(2.201,3.804)--(2.219,3.818)%
    --(2.236,3.833)--(2.253,3.848)--(2.268,3.865)--(2.283,3.882)--(2.297,3.900)%
    --(2.310,3.919)--(2.321,3.938)--(2.332,3.958)--(2.342,3.978)--(2.351,3.999)%
    --(2.358,4.021)--(2.365,4.042)--(2.370,4.064)--(2.374,4.087)--(2.377,4.109)%
    --(2.379,4.132)--(2.380,4.154)--cycle;
\gpfill{color=gp lt color border,opacity=0.50} (4.813,3.098)--(4.812,3.106)--(4.812,3.115)--(4.810,3.123)%
    --(4.809,3.131)--(4.807,3.140)--(4.805,3.148)--(4.802,3.156)--(4.798,3.164)%
    --(4.795,3.172)--(4.791,3.179)--(4.786,3.186)--(4.781,3.193)--(4.776,3.200)%
    --(4.771,3.207)--(4.765,3.213)--(4.759,3.219)--(4.752,3.224)--(4.745,3.229)%
    --(4.738,3.234)--(4.731,3.239)--(4.724,3.243)--(4.716,3.246)--(4.708,3.250)%
    --(4.700,3.253)--(4.692,3.255)--(4.683,3.257)--(4.675,3.258)--(4.667,3.260)%
    --(4.658,3.260)--(4.650,3.261)--(4.641,3.260)--(4.632,3.260)--(4.624,3.258)%
    --(4.616,3.257)--(4.607,3.255)--(4.599,3.253)--(4.591,3.250)--(4.583,3.246)%
    --(4.575,3.243)--(4.568,3.239)--(4.561,3.234)--(4.554,3.229)--(4.547,3.224)%
    --(4.540,3.219)--(4.534,3.213)--(4.528,3.207)--(4.523,3.200)--(4.518,3.193)%
    --(4.513,3.186)--(4.508,3.179)--(4.504,3.172)--(4.501,3.164)--(4.497,3.156)%
    --(4.494,3.148)--(4.492,3.140)--(4.490,3.131)--(4.489,3.123)--(4.487,3.115)%
    --(4.487,3.106)--(4.487,3.098)--(4.487,3.089)--(4.487,3.080)--(4.489,3.072)%
    --(4.490,3.064)--(4.492,3.055)--(4.494,3.047)--(4.497,3.039)--(4.501,3.031)%
    --(4.504,3.023)--(4.508,3.016)--(4.513,3.009)--(4.518,3.002)--(4.523,2.995)%
    --(4.528,2.988)--(4.534,2.982)--(4.540,2.976)--(4.547,2.971)--(4.554,2.966)%
    --(4.561,2.961)--(4.568,2.956)--(4.575,2.952)--(4.583,2.949)--(4.591,2.945)%
    --(4.599,2.942)--(4.607,2.940)--(4.616,2.938)--(4.624,2.937)--(4.632,2.935)%
    --(4.641,2.935)--(4.649,2.935)--(4.658,2.935)--(4.667,2.935)--(4.675,2.937)%
    --(4.683,2.938)--(4.692,2.940)--(4.700,2.942)--(4.708,2.945)--(4.716,2.949)%
    --(4.724,2.952)--(4.731,2.956)--(4.738,2.961)--(4.745,2.966)--(4.752,2.971)%
    --(4.759,2.976)--(4.765,2.982)--(4.771,2.988)--(4.776,2.995)--(4.781,3.002)%
    --(4.786,3.009)--(4.791,3.016)--(4.795,3.023)--(4.798,3.031)--(4.802,3.039)%
    --(4.805,3.047)--(4.807,3.055)--(4.809,3.064)--(4.810,3.072)--(4.812,3.080)%
    --(4.812,3.089)--(4.813,3.097)--cycle;
\gpfill{color=gp lt color border,opacity=0.50} (6.163,6.268)--(6.162,6.276)--(6.162,6.284)--(6.161,6.293)%
    --(6.159,6.301)--(6.157,6.309)--(6.155,6.318)--(6.152,6.326)--(6.148,6.333)%
    --(6.145,6.341)--(6.141,6.348)--(6.136,6.356)--(6.132,6.363)--(6.126,6.369)%
    --(6.121,6.376)--(6.115,6.382)--(6.109,6.388)--(6.102,6.393)--(6.096,6.399)%
    --(6.089,6.403)--(6.082,6.408)--(6.074,6.412)--(6.066,6.415)--(6.059,6.419)%
    --(6.051,6.422)--(6.042,6.424)--(6.034,6.426)--(6.026,6.428)--(6.017,6.429)%
    --(6.009,6.429)--(6.001,6.430)--(5.992,6.429)--(5.984,6.429)--(5.975,6.428)%
    --(5.967,6.426)--(5.959,6.424)--(5.950,6.422)--(5.942,6.419)--(5.935,6.415)%
    --(5.927,6.412)--(5.920,6.408)--(5.912,6.403)--(5.905,6.399)--(5.899,6.393)%
    --(5.892,6.388)--(5.886,6.382)--(5.880,6.376)--(5.875,6.369)--(5.869,6.363)%
    --(5.865,6.356)--(5.860,6.348)--(5.856,6.341)--(5.853,6.333)--(5.849,6.326)%
    --(5.846,6.318)--(5.844,6.309)--(5.842,6.301)--(5.840,6.293)--(5.839,6.284)%
    --(5.839,6.276)--(5.839,6.268)--(5.839,6.259)--(5.839,6.251)--(5.840,6.242)%
    --(5.842,6.234)--(5.844,6.226)--(5.846,6.217)--(5.849,6.209)--(5.853,6.202)%
    --(5.856,6.194)--(5.860,6.186)--(5.865,6.179)--(5.869,6.172)--(5.875,6.166)%
    --(5.880,6.159)--(5.886,6.153)--(5.892,6.147)--(5.899,6.142)--(5.905,6.136)%
    --(5.912,6.132)--(5.919,6.127)--(5.927,6.123)--(5.935,6.120)--(5.942,6.116)%
    --(5.950,6.113)--(5.959,6.111)--(5.967,6.109)--(5.975,6.107)--(5.984,6.106)%
    --(5.992,6.106)--(6.000,6.106)--(6.009,6.106)--(6.017,6.106)--(6.026,6.107)%
    --(6.034,6.109)--(6.042,6.111)--(6.051,6.113)--(6.059,6.116)--(6.066,6.120)%
    --(6.074,6.123)--(6.082,6.127)--(6.089,6.132)--(6.096,6.136)--(6.102,6.142)%
    --(6.109,6.147)--(6.115,6.153)--(6.121,6.159)--(6.126,6.166)--(6.132,6.172)%
    --(6.136,6.179)--(6.141,6.186)--(6.145,6.194)--(6.148,6.202)--(6.152,6.209)%
    --(6.155,6.217)--(6.157,6.226)--(6.159,6.234)--(6.161,6.242)--(6.162,6.251)%
    --(6.162,6.259)--(6.163,6.267)--cycle;
\gpfill{color=gp lt color border,opacity=0.50} (8.866,5.211)--(8.865,5.219)--(8.865,5.227)--(8.864,5.236)%
    --(8.862,5.244)--(8.860,5.252)--(8.858,5.261)--(8.855,5.269)--(8.851,5.276)%
    --(8.848,5.284)--(8.844,5.291)--(8.839,5.299)--(8.835,5.306)--(8.829,5.312)%
    --(8.824,5.319)--(8.818,5.325)--(8.812,5.331)--(8.805,5.336)--(8.799,5.342)%
    --(8.792,5.346)--(8.785,5.351)--(8.777,5.355)--(8.769,5.358)--(8.762,5.362)%
    --(8.754,5.365)--(8.745,5.367)--(8.737,5.369)--(8.729,5.371)--(8.720,5.372)%
    --(8.712,5.372)--(8.704,5.373)--(8.695,5.372)--(8.687,5.372)--(8.678,5.371)%
    --(8.670,5.369)--(8.662,5.367)--(8.653,5.365)--(8.645,5.362)--(8.638,5.358)%
    --(8.630,5.355)--(8.623,5.351)--(8.615,5.346)--(8.608,5.342)--(8.602,5.336)%
    --(8.595,5.331)--(8.589,5.325)--(8.583,5.319)--(8.578,5.312)--(8.572,5.306)%
    --(8.568,5.299)--(8.563,5.291)--(8.559,5.284)--(8.556,5.276)--(8.552,5.269)%
    --(8.549,5.261)--(8.547,5.252)--(8.545,5.244)--(8.543,5.236)--(8.542,5.227)%
    --(8.542,5.219)--(8.542,5.211)--(8.542,5.202)--(8.542,5.194)--(8.543,5.185)%
    --(8.545,5.177)--(8.547,5.169)--(8.549,5.160)--(8.552,5.152)--(8.556,5.145)%
    --(8.559,5.137)--(8.563,5.129)--(8.568,5.122)--(8.572,5.115)--(8.578,5.109)%
    --(8.583,5.102)--(8.589,5.096)--(8.595,5.090)--(8.602,5.085)--(8.608,5.079)%
    --(8.615,5.075)--(8.622,5.070)--(8.630,5.066)--(8.638,5.063)--(8.645,5.059)%
    --(8.653,5.056)--(8.662,5.054)--(8.670,5.052)--(8.678,5.050)--(8.687,5.049)%
    --(8.695,5.049)--(8.703,5.049)--(8.712,5.049)--(8.720,5.049)--(8.729,5.050)%
    --(8.737,5.052)--(8.745,5.054)--(8.754,5.056)--(8.762,5.059)--(8.769,5.063)%
    --(8.777,5.066)--(8.785,5.070)--(8.792,5.075)--(8.799,5.079)--(8.805,5.085)%
    --(8.812,5.090)--(8.818,5.096)--(8.824,5.102)--(8.829,5.109)--(8.835,5.115)%
    --(8.839,5.122)--(8.844,5.129)--(8.848,5.137)--(8.851,5.145)--(8.855,5.152)%
    --(8.858,5.160)--(8.860,5.169)--(8.862,5.177)--(8.864,5.185)--(8.865,5.194)%
    --(8.865,5.202)--(8.866,5.210)--cycle;
\gpfill{color=gp lt color border,opacity=0.50} (7.568,5.211)--(7.567,5.222)--(7.566,5.233)--(7.565,5.244)%
    --(7.563,5.255)--(7.560,5.266)--(7.557,5.277)--(7.553,5.288)--(7.549,5.298)%
    --(7.544,5.309)--(7.539,5.318)--(7.533,5.328)--(7.526,5.337)--(7.519,5.346)%
    --(7.512,5.355)--(7.504,5.363)--(7.496,5.371)--(7.487,5.378)--(7.478,5.385)%
    --(7.469,5.392)--(7.460,5.398)--(7.450,5.403)--(7.439,5.408)--(7.429,5.412)%
    --(7.418,5.416)--(7.407,5.419)--(7.396,5.422)--(7.385,5.424)--(7.374,5.425)%
    --(7.363,5.426)--(7.352,5.427)--(7.340,5.426)--(7.329,5.425)--(7.318,5.424)%
    --(7.307,5.422)--(7.296,5.419)--(7.285,5.416)--(7.274,5.412)--(7.264,5.408)%
    --(7.253,5.403)--(7.244,5.398)--(7.234,5.392)--(7.225,5.385)--(7.216,5.378)%
    --(7.207,5.371)--(7.199,5.363)--(7.191,5.355)--(7.184,5.346)--(7.177,5.337)%
    --(7.170,5.328)--(7.164,5.318)--(7.159,5.309)--(7.154,5.298)--(7.150,5.288)%
    --(7.146,5.277)--(7.143,5.266)--(7.140,5.255)--(7.138,5.244)--(7.137,5.233)%
    --(7.136,5.222)--(7.136,5.211)--(7.136,5.199)--(7.137,5.188)--(7.138,5.177)%
    --(7.140,5.166)--(7.143,5.155)--(7.146,5.144)--(7.150,5.133)--(7.154,5.123)%
    --(7.159,5.112)--(7.164,5.102)--(7.170,5.093)--(7.177,5.084)--(7.184,5.075)%
    --(7.191,5.066)--(7.199,5.058)--(7.207,5.050)--(7.216,5.043)--(7.225,5.036)%
    --(7.234,5.029)--(7.243,5.023)--(7.253,5.018)--(7.264,5.013)--(7.274,5.009)%
    --(7.285,5.005)--(7.296,5.002)--(7.307,4.999)--(7.318,4.997)--(7.329,4.996)%
    --(7.340,4.995)--(7.351,4.995)--(7.363,4.995)--(7.374,4.996)--(7.385,4.997)%
    --(7.396,4.999)--(7.407,5.002)--(7.418,5.005)--(7.429,5.009)--(7.439,5.013)%
    --(7.450,5.018)--(7.460,5.023)--(7.469,5.029)--(7.478,5.036)--(7.487,5.043)%
    --(7.496,5.050)--(7.504,5.058)--(7.512,5.066)--(7.519,5.075)--(7.526,5.084)%
    --(7.533,5.093)--(7.539,5.102)--(7.544,5.112)--(7.549,5.123)--(7.553,5.133)%
    --(7.557,5.144)--(7.560,5.155)--(7.563,5.166)--(7.565,5.177)--(7.566,5.188)%
    --(7.567,5.199)--(7.568,5.210)--cycle;
\gpfill{color=gp lt color border,opacity=0.50} (3.137,4.155)--(3.136,4.174)--(3.134,4.194)--(3.132,4.214)%
    --(3.128,4.233)--(3.124,4.253)--(3.118,4.272)--(3.111,4.290)--(3.104,4.309)%
    --(3.095,4.327)--(3.086,4.344)--(3.075,4.361)--(3.064,4.377)--(3.052,4.393)%
    --(3.039,4.408)--(3.025,4.422)--(3.011,4.436)--(2.996,4.449)--(2.980,4.461)%
    --(2.964,4.472)--(2.947,4.483)--(2.930,4.492)--(2.912,4.501)--(2.893,4.508)%
    --(2.875,4.515)--(2.856,4.521)--(2.836,4.525)--(2.817,4.529)--(2.797,4.531)%
    --(2.777,4.533)--(2.758,4.534)--(2.738,4.533)--(2.718,4.531)--(2.698,4.529)%
    --(2.679,4.525)--(2.659,4.521)--(2.640,4.515)--(2.622,4.508)--(2.603,4.501)%
    --(2.585,4.492)--(2.568,4.483)--(2.551,4.472)--(2.535,4.461)--(2.519,4.449)%
    --(2.504,4.436)--(2.490,4.422)--(2.476,4.408)--(2.463,4.393)--(2.451,4.377)%
    --(2.440,4.361)--(2.429,4.344)--(2.420,4.327)--(2.411,4.309)--(2.404,4.290)%
    --(2.397,4.272)--(2.391,4.253)--(2.387,4.233)--(2.383,4.214)--(2.381,4.194)%
    --(2.379,4.174)--(2.379,4.155)--(2.379,4.135)--(2.381,4.115)--(2.383,4.095)%
    --(2.387,4.076)--(2.391,4.056)--(2.397,4.037)--(2.404,4.019)--(2.411,4.000)%
    --(2.420,3.982)--(2.429,3.965)--(2.440,3.948)--(2.451,3.932)--(2.463,3.916)%
    --(2.476,3.901)--(2.490,3.887)--(2.504,3.873)--(2.519,3.860)--(2.535,3.848)%
    --(2.551,3.837)--(2.568,3.826)--(2.585,3.817)--(2.603,3.808)--(2.622,3.801)%
    --(2.640,3.794)--(2.659,3.788)--(2.679,3.784)--(2.698,3.780)--(2.718,3.778)%
    --(2.738,3.776)--(2.757,3.776)--(2.777,3.776)--(2.797,3.778)--(2.817,3.780)%
    --(2.836,3.784)--(2.856,3.788)--(2.875,3.794)--(2.893,3.801)--(2.912,3.808)%
    --(2.930,3.817)--(2.947,3.826)--(2.964,3.837)--(2.980,3.848)--(2.996,3.860)%
    --(3.011,3.873)--(3.025,3.887)--(3.039,3.901)--(3.052,3.916)--(3.064,3.932)%
    --(3.075,3.948)--(3.086,3.965)--(3.095,3.982)--(3.104,4.000)--(3.111,4.019)%
    --(3.118,4.037)--(3.124,4.056)--(3.128,4.076)--(3.132,4.095)--(3.134,4.115)%
    --(3.136,4.135)--(3.137,4.154)--cycle;
\gpfill{color=gp lt color border,opacity=0.50} (5.595,3.098)--(5.594,3.105)--(5.594,3.112)--(5.593,3.119)%
    --(5.592,3.126)--(5.590,3.132)--(5.588,3.139)--(5.586,3.146)--(5.583,3.152)%
    --(5.580,3.159)--(5.576,3.165)--(5.573,3.171)--(5.569,3.177)--(5.564,3.182)%
    --(5.560,3.188)--(5.555,3.193)--(5.550,3.198)--(5.544,3.202)--(5.539,3.207)%
    --(5.533,3.211)--(5.527,3.214)--(5.521,3.218)--(5.514,3.221)--(5.508,3.224)%
    --(5.501,3.226)--(5.494,3.228)--(5.488,3.230)--(5.481,3.231)--(5.474,3.232)%
    --(5.467,3.232)--(5.460,3.233)--(5.452,3.232)--(5.445,3.232)--(5.438,3.231)%
    --(5.431,3.230)--(5.425,3.228)--(5.418,3.226)--(5.411,3.224)--(5.405,3.221)%
    --(5.398,3.218)--(5.392,3.214)--(5.386,3.211)--(5.380,3.207)--(5.375,3.202)%
    --(5.369,3.198)--(5.364,3.193)--(5.359,3.188)--(5.355,3.182)--(5.350,3.177)%
    --(5.346,3.171)--(5.343,3.165)--(5.339,3.159)--(5.336,3.152)--(5.333,3.146)%
    --(5.331,3.139)--(5.329,3.132)--(5.327,3.126)--(5.326,3.119)--(5.325,3.112)%
    --(5.325,3.105)--(5.325,3.098)--(5.325,3.090)--(5.325,3.083)--(5.326,3.076)%
    --(5.327,3.069)--(5.329,3.063)--(5.331,3.056)--(5.333,3.049)--(5.336,3.043)%
    --(5.339,3.036)--(5.343,3.030)--(5.346,3.024)--(5.350,3.018)--(5.355,3.013)%
    --(5.359,3.007)--(5.364,3.002)--(5.369,2.997)--(5.375,2.993)--(5.380,2.988)%
    --(5.386,2.984)--(5.392,2.981)--(5.398,2.977)--(5.405,2.974)--(5.411,2.971)%
    --(5.418,2.969)--(5.425,2.967)--(5.431,2.965)--(5.438,2.964)--(5.445,2.963)%
    --(5.452,2.963)--(5.459,2.963)--(5.467,2.963)--(5.474,2.963)--(5.481,2.964)%
    --(5.488,2.965)--(5.494,2.967)--(5.501,2.969)--(5.508,2.971)--(5.514,2.974)%
    --(5.521,2.977)--(5.527,2.981)--(5.533,2.984)--(5.539,2.988)--(5.544,2.993)%
    --(5.550,2.997)--(5.555,3.002)--(5.560,3.007)--(5.564,3.013)--(5.569,3.018)%
    --(5.573,3.024)--(5.576,3.030)--(5.580,3.036)--(5.583,3.043)--(5.586,3.049)%
    --(5.588,3.056)--(5.590,3.063)--(5.592,3.069)--(5.593,3.076)--(5.594,3.083)%
    --(5.594,3.090)--(5.595,3.097)--cycle;
\gpfill{color=gp lt color border,opacity=0.50} (2.353,3.098)--(2.352,3.119)--(2.350,3.140)--(2.348,3.161)%
    --(2.344,3.182)--(2.339,3.203)--(2.333,3.223)--(2.326,3.243)--(2.317,3.263)%
    --(2.308,3.282)--(2.298,3.300)--(2.287,3.319)--(2.275,3.336)--(2.262,3.353)%
    --(2.248,3.369)--(2.234,3.385)--(2.218,3.399)--(2.202,3.413)--(2.185,3.426)%
    --(2.168,3.438)--(2.150,3.449)--(2.131,3.459)--(2.112,3.468)--(2.092,3.477)%
    --(2.072,3.484)--(2.052,3.490)--(2.031,3.495)--(2.010,3.499)--(1.989,3.501)%
    --(1.968,3.503)--(1.947,3.504)--(1.925,3.503)--(1.904,3.501)--(1.883,3.499)%
    --(1.862,3.495)--(1.841,3.490)--(1.821,3.484)--(1.801,3.477)--(1.781,3.468)%
    --(1.762,3.459)--(1.744,3.449)--(1.725,3.438)--(1.708,3.426)--(1.691,3.413)%
    --(1.675,3.399)--(1.659,3.385)--(1.645,3.369)--(1.631,3.353)--(1.618,3.336)%
    --(1.606,3.319)--(1.595,3.300)--(1.585,3.282)--(1.576,3.263)--(1.567,3.243)%
    --(1.560,3.223)--(1.554,3.203)--(1.549,3.182)--(1.545,3.161)--(1.543,3.140)%
    --(1.541,3.119)--(1.541,3.098)--(1.541,3.076)--(1.543,3.055)--(1.545,3.034)%
    --(1.549,3.013)--(1.554,2.992)--(1.560,2.972)--(1.567,2.952)--(1.576,2.932)%
    --(1.585,2.913)--(1.595,2.894)--(1.606,2.876)--(1.618,2.859)--(1.631,2.842)%
    --(1.645,2.826)--(1.659,2.810)--(1.675,2.796)--(1.691,2.782)--(1.708,2.769)%
    --(1.725,2.757)--(1.743,2.746)--(1.762,2.736)--(1.781,2.727)--(1.801,2.718)%
    --(1.821,2.711)--(1.841,2.705)--(1.862,2.700)--(1.883,2.696)--(1.904,2.694)%
    --(1.925,2.692)--(1.946,2.692)--(1.968,2.692)--(1.989,2.694)--(2.010,2.696)%
    --(2.031,2.700)--(2.052,2.705)--(2.072,2.711)--(2.092,2.718)--(2.112,2.727)%
    --(2.131,2.736)--(2.150,2.746)--(2.168,2.757)--(2.185,2.769)--(2.202,2.782)%
    --(2.218,2.796)--(2.234,2.810)--(2.248,2.826)--(2.262,2.842)--(2.275,2.859)%
    --(2.287,2.876)--(2.298,2.894)--(2.308,2.913)--(2.317,2.932)--(2.326,2.952)%
    --(2.333,2.972)--(2.339,2.992)--(2.344,3.013)--(2.348,3.034)--(2.350,3.055)%
    --(2.352,3.076)--(2.353,3.097)--cycle;
\gpfill{color=gp lt color border,opacity=0.50} (4.840,6.268)--(4.839,6.277)--(4.838,6.287)--(4.837,6.297)%
    --(4.835,6.307)--(4.833,6.317)--(4.830,6.326)--(4.827,6.336)--(4.823,6.345)%
    --(4.819,6.354)--(4.814,6.362)--(4.809,6.371)--(4.803,6.379)--(4.797,6.387)%
    --(4.791,6.395)--(4.784,6.402)--(4.777,6.409)--(4.769,6.415)--(4.761,6.421)%
    --(4.753,6.427)--(4.745,6.432)--(4.736,6.437)--(4.727,6.441)--(4.718,6.445)%
    --(4.708,6.448)--(4.699,6.451)--(4.689,6.453)--(4.679,6.455)--(4.669,6.456)%
    --(4.659,6.457)--(4.650,6.458)--(4.640,6.457)--(4.630,6.456)--(4.620,6.455)%
    --(4.610,6.453)--(4.600,6.451)--(4.591,6.448)--(4.581,6.445)--(4.572,6.441)%
    --(4.563,6.437)--(4.555,6.432)--(4.546,6.427)--(4.538,6.421)--(4.530,6.415)%
    --(4.522,6.409)--(4.515,6.402)--(4.508,6.395)--(4.502,6.387)--(4.496,6.379)%
    --(4.490,6.371)--(4.485,6.362)--(4.480,6.354)--(4.476,6.345)--(4.472,6.336)%
    --(4.469,6.326)--(4.466,6.317)--(4.464,6.307)--(4.462,6.297)--(4.461,6.287)%
    --(4.460,6.277)--(4.460,6.268)--(4.460,6.258)--(4.461,6.248)--(4.462,6.238)%
    --(4.464,6.228)--(4.466,6.218)--(4.469,6.209)--(4.472,6.199)--(4.476,6.190)%
    --(4.480,6.181)--(4.485,6.172)--(4.490,6.164)--(4.496,6.156)--(4.502,6.148)%
    --(4.508,6.140)--(4.515,6.133)--(4.522,6.126)--(4.530,6.120)--(4.538,6.114)%
    --(4.546,6.108)--(4.554,6.103)--(4.563,6.098)--(4.572,6.094)--(4.581,6.090)%
    --(4.591,6.087)--(4.600,6.084)--(4.610,6.082)--(4.620,6.080)--(4.630,6.079)%
    --(4.640,6.078)--(4.649,6.078)--(4.659,6.078)--(4.669,6.079)--(4.679,6.080)%
    --(4.689,6.082)--(4.699,6.084)--(4.708,6.087)--(4.718,6.090)--(4.727,6.094)%
    --(4.736,6.098)--(4.745,6.103)--(4.753,6.108)--(4.761,6.114)--(4.769,6.120)%
    --(4.777,6.126)--(4.784,6.133)--(4.791,6.140)--(4.797,6.148)--(4.803,6.156)%
    --(4.809,6.164)--(4.814,6.172)--(4.819,6.181)--(4.823,6.190)--(4.827,6.199)%
    --(4.830,6.209)--(4.833,6.218)--(4.835,6.228)--(4.837,6.238)--(4.838,6.248)%
    --(4.839,6.258)--(4.840,6.267)--cycle;
\gpfill{color=gp lt color border,opacity=0.50} (3.433,6.268)--(3.432,6.275)--(3.432,6.282)--(3.431,6.289)%
    --(3.430,6.296)--(3.428,6.302)--(3.426,6.309)--(3.424,6.316)--(3.421,6.322)%
    --(3.418,6.329)--(3.414,6.335)--(3.411,6.341)--(3.407,6.347)--(3.402,6.352)%
    --(3.398,6.358)--(3.393,6.363)--(3.388,6.368)--(3.382,6.372)--(3.377,6.377)%
    --(3.371,6.381)--(3.365,6.384)--(3.359,6.388)--(3.352,6.391)--(3.346,6.394)%
    --(3.339,6.396)--(3.332,6.398)--(3.326,6.400)--(3.319,6.401)--(3.312,6.402)%
    --(3.305,6.402)--(3.298,6.403)--(3.290,6.402)--(3.283,6.402)--(3.276,6.401)%
    --(3.269,6.400)--(3.263,6.398)--(3.256,6.396)--(3.249,6.394)--(3.243,6.391)%
    --(3.236,6.388)--(3.230,6.384)--(3.224,6.381)--(3.218,6.377)--(3.213,6.372)%
    --(3.207,6.368)--(3.202,6.363)--(3.197,6.358)--(3.193,6.352)--(3.188,6.347)%
    --(3.184,6.341)--(3.181,6.335)--(3.177,6.329)--(3.174,6.322)--(3.171,6.316)%
    --(3.169,6.309)--(3.167,6.302)--(3.165,6.296)--(3.164,6.289)--(3.163,6.282)%
    --(3.163,6.275)--(3.163,6.268)--(3.163,6.260)--(3.163,6.253)--(3.164,6.246)%
    --(3.165,6.239)--(3.167,6.233)--(3.169,6.226)--(3.171,6.219)--(3.174,6.213)%
    --(3.177,6.206)--(3.181,6.200)--(3.184,6.194)--(3.188,6.188)--(3.193,6.183)%
    --(3.197,6.177)--(3.202,6.172)--(3.207,6.167)--(3.213,6.163)--(3.218,6.158)%
    --(3.224,6.154)--(3.230,6.151)--(3.236,6.147)--(3.243,6.144)--(3.249,6.141)%
    --(3.256,6.139)--(3.263,6.137)--(3.269,6.135)--(3.276,6.134)--(3.283,6.133)%
    --(3.290,6.133)--(3.297,6.133)--(3.305,6.133)--(3.312,6.133)--(3.319,6.134)%
    --(3.326,6.135)--(3.332,6.137)--(3.339,6.139)--(3.346,6.141)--(3.352,6.144)%
    --(3.359,6.147)--(3.365,6.151)--(3.371,6.154)--(3.377,6.158)--(3.382,6.163)%
    --(3.388,6.167)--(3.393,6.172)--(3.398,6.177)--(3.402,6.183)--(3.407,6.188)%
    --(3.411,6.194)--(3.414,6.200)--(3.418,6.206)--(3.421,6.213)--(3.424,6.219)%
    --(3.426,6.226)--(3.428,6.233)--(3.430,6.239)--(3.431,6.246)--(3.432,6.253)%
    --(3.432,6.260)--(3.433,6.267)--cycle;
\gpfill{color=gp lt color border,opacity=0.50} (6.244,5.211)--(6.243,5.223)--(6.242,5.236)--(6.241,5.249)%
    --(6.238,5.261)--(6.235,5.273)--(6.232,5.286)--(6.227,5.298)--(6.222,5.309)%
    --(6.217,5.321)--(6.211,5.332)--(6.204,5.343)--(6.197,5.353)--(6.189,5.363)%
    --(6.181,5.373)--(6.172,5.382)--(6.163,5.391)--(6.153,5.399)--(6.143,5.407)%
    --(6.133,5.414)--(6.122,5.421)--(6.111,5.427)--(6.099,5.432)--(6.088,5.437)%
    --(6.076,5.442)--(6.063,5.445)--(6.051,5.448)--(6.039,5.451)--(6.026,5.452)%
    --(6.013,5.453)--(6.001,5.454)--(5.988,5.453)--(5.975,5.452)--(5.962,5.451)%
    --(5.950,5.448)--(5.938,5.445)--(5.925,5.442)--(5.913,5.437)--(5.902,5.432)%
    --(5.890,5.427)--(5.879,5.421)--(5.868,5.414)--(5.858,5.407)--(5.848,5.399)%
    --(5.838,5.391)--(5.829,5.382)--(5.820,5.373)--(5.812,5.363)--(5.804,5.353)%
    --(5.797,5.343)--(5.790,5.332)--(5.784,5.321)--(5.779,5.309)--(5.774,5.298)%
    --(5.769,5.286)--(5.766,5.273)--(5.763,5.261)--(5.760,5.249)--(5.759,5.236)%
    --(5.758,5.223)--(5.758,5.211)--(5.758,5.198)--(5.759,5.185)--(5.760,5.172)%
    --(5.763,5.160)--(5.766,5.148)--(5.769,5.135)--(5.774,5.123)--(5.779,5.112)%
    --(5.784,5.100)--(5.790,5.089)--(5.797,5.078)--(5.804,5.068)--(5.812,5.058)%
    --(5.820,5.048)--(5.829,5.039)--(5.838,5.030)--(5.848,5.022)--(5.858,5.014)%
    --(5.868,5.007)--(5.879,5.000)--(5.890,4.994)--(5.902,4.989)--(5.913,4.984)%
    --(5.925,4.979)--(5.938,4.976)--(5.950,4.973)--(5.962,4.970)--(5.975,4.969)%
    --(5.988,4.968)--(6.000,4.968)--(6.013,4.968)--(6.026,4.969)--(6.039,4.970)%
    --(6.051,4.973)--(6.063,4.976)--(6.076,4.979)--(6.088,4.984)--(6.099,4.989)%
    --(6.111,4.994)--(6.122,5.000)--(6.133,5.007)--(6.143,5.014)--(6.153,5.022)%
    --(6.163,5.030)--(6.172,5.039)--(6.181,5.048)--(6.189,5.058)--(6.197,5.068)%
    --(6.204,5.078)--(6.211,5.089)--(6.217,5.100)--(6.222,5.112)--(6.227,5.123)%
    --(6.232,5.135)--(6.235,5.148)--(6.238,5.160)--(6.241,5.172)--(6.242,5.185)%
    --(6.243,5.198)--(6.244,5.210)--cycle;
\gpfill{color=gp lt color border,opacity=0.50} (3.056,3.098)--(3.055,3.113)--(3.054,3.129)--(3.052,3.144)%
    --(3.049,3.159)--(3.045,3.175)--(3.041,3.190)--(3.036,3.204)--(3.030,3.219)%
    --(3.023,3.233)--(3.016,3.246)--(3.007,3.260)--(2.999,3.273)--(2.989,3.285)%
    --(2.979,3.297)--(2.968,3.308)--(2.957,3.319)--(2.945,3.329)--(2.933,3.339)%
    --(2.920,3.347)--(2.907,3.356)--(2.893,3.363)--(2.879,3.370)--(2.864,3.376)%
    --(2.850,3.381)--(2.835,3.385)--(2.819,3.389)--(2.804,3.392)--(2.789,3.394)%
    --(2.773,3.395)--(2.758,3.396)--(2.742,3.395)--(2.726,3.394)--(2.711,3.392)%
    --(2.696,3.389)--(2.680,3.385)--(2.665,3.381)--(2.651,3.376)--(2.636,3.370)%
    --(2.622,3.363)--(2.609,3.356)--(2.595,3.347)--(2.582,3.339)--(2.570,3.329)%
    --(2.558,3.319)--(2.547,3.308)--(2.536,3.297)--(2.526,3.285)--(2.516,3.273)%
    --(2.508,3.260)--(2.499,3.246)--(2.492,3.233)--(2.485,3.219)--(2.479,3.204)%
    --(2.474,3.190)--(2.470,3.175)--(2.466,3.159)--(2.463,3.144)--(2.461,3.129)%
    --(2.460,3.113)--(2.460,3.098)--(2.460,3.082)--(2.461,3.066)--(2.463,3.051)%
    --(2.466,3.036)--(2.470,3.020)--(2.474,3.005)--(2.479,2.991)--(2.485,2.976)%
    --(2.492,2.962)--(2.499,2.948)--(2.508,2.935)--(2.516,2.922)--(2.526,2.910)%
    --(2.536,2.898)--(2.547,2.887)--(2.558,2.876)--(2.570,2.866)--(2.582,2.856)%
    --(2.595,2.848)--(2.608,2.839)--(2.622,2.832)--(2.636,2.825)--(2.651,2.819)%
    --(2.665,2.814)--(2.680,2.810)--(2.696,2.806)--(2.711,2.803)--(2.726,2.801)%
    --(2.742,2.800)--(2.757,2.800)--(2.773,2.800)--(2.789,2.801)--(2.804,2.803)%
    --(2.819,2.806)--(2.835,2.810)--(2.850,2.814)--(2.864,2.819)--(2.879,2.825)%
    --(2.893,2.832)--(2.907,2.839)--(2.920,2.848)--(2.933,2.856)--(2.945,2.866)%
    --(2.957,2.876)--(2.968,2.887)--(2.979,2.898)--(2.989,2.910)--(2.999,2.922)%
    --(3.007,2.935)--(3.016,2.948)--(3.023,2.962)--(3.030,2.976)--(3.036,2.991)%
    --(3.041,3.005)--(3.045,3.020)--(3.049,3.036)--(3.052,3.051)--(3.054,3.066)%
    --(3.055,3.082)--(3.056,3.097)--cycle;
\gpfill{color=gp lt color border,opacity=0.50} (2.136,6.268)--(2.135,6.277)--(2.134,6.287)--(2.133,6.297)%
    --(2.131,6.307)--(2.129,6.316)--(2.126,6.326)--(2.123,6.335)--(2.119,6.344)%
    --(2.115,6.353)--(2.110,6.362)--(2.105,6.370)--(2.099,6.379)--(2.093,6.386)%
    --(2.087,6.394)--(2.080,6.401)--(2.073,6.408)--(2.065,6.414)--(2.058,6.420)%
    --(2.049,6.426)--(2.041,6.431)--(2.032,6.436)--(2.023,6.440)--(2.014,6.444)%
    --(2.005,6.447)--(1.995,6.450)--(1.986,6.452)--(1.976,6.454)--(1.966,6.455)%
    --(1.956,6.456)--(1.947,6.457)--(1.937,6.456)--(1.927,6.455)--(1.917,6.454)%
    --(1.907,6.452)--(1.898,6.450)--(1.888,6.447)--(1.879,6.444)--(1.870,6.440)%
    --(1.861,6.436)--(1.852,6.431)--(1.844,6.426)--(1.835,6.420)--(1.828,6.414)%
    --(1.820,6.408)--(1.813,6.401)--(1.806,6.394)--(1.800,6.386)--(1.794,6.379)%
    --(1.788,6.370)--(1.783,6.362)--(1.778,6.353)--(1.774,6.344)--(1.770,6.335)%
    --(1.767,6.326)--(1.764,6.316)--(1.762,6.307)--(1.760,6.297)--(1.759,6.287)%
    --(1.758,6.277)--(1.758,6.268)--(1.758,6.258)--(1.759,6.248)--(1.760,6.238)%
    --(1.762,6.228)--(1.764,6.219)--(1.767,6.209)--(1.770,6.200)--(1.774,6.191)%
    --(1.778,6.182)--(1.783,6.173)--(1.788,6.165)--(1.794,6.156)--(1.800,6.149)%
    --(1.806,6.141)--(1.813,6.134)--(1.820,6.127)--(1.828,6.121)--(1.835,6.115)%
    --(1.844,6.109)--(1.852,6.104)--(1.861,6.099)--(1.870,6.095)--(1.879,6.091)%
    --(1.888,6.088)--(1.898,6.085)--(1.907,6.083)--(1.917,6.081)--(1.927,6.080)%
    --(1.937,6.079)--(1.946,6.079)--(1.956,6.079)--(1.966,6.080)--(1.976,6.081)%
    --(1.986,6.083)--(1.995,6.085)--(2.005,6.088)--(2.014,6.091)--(2.023,6.095)%
    --(2.032,6.099)--(2.041,6.104)--(2.049,6.109)--(2.058,6.115)--(2.065,6.121)%
    --(2.073,6.127)--(2.080,6.134)--(2.087,6.141)--(2.093,6.149)--(2.099,6.156)%
    --(2.105,6.165)--(2.110,6.173)--(2.115,6.182)--(2.119,6.191)--(2.123,6.200)%
    --(2.126,6.209)--(2.129,6.219)--(2.131,6.228)--(2.133,6.238)--(2.134,6.248)%
    --(2.135,6.258)--(2.136,6.267)--cycle;
\gpfill{color=gp lt color border,opacity=0.50} (4.948,5.211)--(4.947,5.226)--(4.946,5.242)--(4.944,5.257)%
    --(4.941,5.272)--(4.937,5.288)--(4.933,5.303)--(4.928,5.317)--(4.922,5.332)%
    --(4.915,5.346)--(4.908,5.359)--(4.899,5.373)--(4.891,5.386)--(4.881,5.398)%
    --(4.871,5.410)--(4.860,5.421)--(4.849,5.432)--(4.837,5.442)--(4.825,5.452)%
    --(4.812,5.460)--(4.799,5.469)--(4.785,5.476)--(4.771,5.483)--(4.756,5.489)%
    --(4.742,5.494)--(4.727,5.498)--(4.711,5.502)--(4.696,5.505)--(4.681,5.507)%
    --(4.665,5.508)--(4.650,5.509)--(4.634,5.508)--(4.618,5.507)--(4.603,5.505)%
    --(4.588,5.502)--(4.572,5.498)--(4.557,5.494)--(4.543,5.489)--(4.528,5.483)%
    --(4.514,5.476)--(4.501,5.469)--(4.487,5.460)--(4.474,5.452)--(4.462,5.442)%
    --(4.450,5.432)--(4.439,5.421)--(4.428,5.410)--(4.418,5.398)--(4.408,5.386)%
    --(4.400,5.373)--(4.391,5.359)--(4.384,5.346)--(4.377,5.332)--(4.371,5.317)%
    --(4.366,5.303)--(4.362,5.288)--(4.358,5.272)--(4.355,5.257)--(4.353,5.242)%
    --(4.352,5.226)--(4.352,5.211)--(4.352,5.195)--(4.353,5.179)--(4.355,5.164)%
    --(4.358,5.149)--(4.362,5.133)--(4.366,5.118)--(4.371,5.104)--(4.377,5.089)%
    --(4.384,5.075)--(4.391,5.061)--(4.400,5.048)--(4.408,5.035)--(4.418,5.023)%
    --(4.428,5.011)--(4.439,5.000)--(4.450,4.989)--(4.462,4.979)--(4.474,4.969)%
    --(4.487,4.961)--(4.500,4.952)--(4.514,4.945)--(4.528,4.938)--(4.543,4.932)%
    --(4.557,4.927)--(4.572,4.923)--(4.588,4.919)--(4.603,4.916)--(4.618,4.914)%
    --(4.634,4.913)--(4.649,4.913)--(4.665,4.913)--(4.681,4.914)--(4.696,4.916)%
    --(4.711,4.919)--(4.727,4.923)--(4.742,4.927)--(4.756,4.932)--(4.771,4.938)%
    --(4.785,4.945)--(4.799,4.952)--(4.812,4.961)--(4.825,4.969)--(4.837,4.979)%
    --(4.849,4.989)--(4.860,5.000)--(4.871,5.011)--(4.881,5.023)--(4.891,5.035)%
    --(4.899,5.048)--(4.908,5.061)--(4.915,5.075)--(4.922,5.089)--(4.928,5.104)%
    --(4.933,5.118)--(4.937,5.133)--(4.941,5.149)--(4.944,5.164)--(4.946,5.179)%
    --(4.947,5.195)--(4.948,5.210)--cycle;
\gpfill{color=gp lt color border,opacity=0.50} (3.595,5.211)--(3.594,5.226)--(3.593,5.242)--(3.591,5.257)%
    --(3.588,5.272)--(3.584,5.287)--(3.580,5.302)--(3.575,5.317)--(3.569,5.331)%
    --(3.562,5.345)--(3.555,5.359)--(3.547,5.372)--(3.538,5.385)--(3.528,5.397)%
    --(3.518,5.409)--(3.508,5.421)--(3.496,5.431)--(3.484,5.441)--(3.472,5.451)%
    --(3.459,5.460)--(3.446,5.468)--(3.432,5.475)--(3.418,5.482)--(3.404,5.488)%
    --(3.389,5.493)--(3.374,5.497)--(3.359,5.501)--(3.344,5.504)--(3.329,5.506)%
    --(3.313,5.507)--(3.298,5.508)--(3.282,5.507)--(3.266,5.506)--(3.251,5.504)%
    --(3.236,5.501)--(3.221,5.497)--(3.206,5.493)--(3.191,5.488)--(3.177,5.482)%
    --(3.163,5.475)--(3.149,5.468)--(3.136,5.460)--(3.123,5.451)--(3.111,5.441)%
    --(3.099,5.431)--(3.087,5.421)--(3.077,5.409)--(3.067,5.397)--(3.057,5.385)%
    --(3.048,5.372)--(3.040,5.359)--(3.033,5.345)--(3.026,5.331)--(3.020,5.317)%
    --(3.015,5.302)--(3.011,5.287)--(3.007,5.272)--(3.004,5.257)--(3.002,5.242)%
    --(3.001,5.226)--(3.001,5.211)--(3.001,5.195)--(3.002,5.179)--(3.004,5.164)%
    --(3.007,5.149)--(3.011,5.134)--(3.015,5.119)--(3.020,5.104)--(3.026,5.090)%
    --(3.033,5.076)--(3.040,5.062)--(3.048,5.049)--(3.057,5.036)--(3.067,5.024)%
    --(3.077,5.012)--(3.087,5.000)--(3.099,4.990)--(3.111,4.980)--(3.123,4.970)%
    --(3.136,4.961)--(3.149,4.953)--(3.163,4.946)--(3.177,4.939)--(3.191,4.933)%
    --(3.206,4.928)--(3.221,4.924)--(3.236,4.920)--(3.251,4.917)--(3.266,4.915)%
    --(3.282,4.914)--(3.297,4.914)--(3.313,4.914)--(3.329,4.915)--(3.344,4.917)%
    --(3.359,4.920)--(3.374,4.924)--(3.389,4.928)--(3.404,4.933)--(3.418,4.939)%
    --(3.432,4.946)--(3.446,4.953)--(3.459,4.961)--(3.472,4.970)--(3.484,4.980)%
    --(3.496,4.990)--(3.508,5.000)--(3.518,5.012)--(3.528,5.024)--(3.538,5.036)%
    --(3.547,5.049)--(3.555,5.062)--(3.562,5.076)--(3.569,5.090)--(3.575,5.104)%
    --(3.580,5.119)--(3.584,5.134)--(3.588,5.149)--(3.591,5.164)--(3.593,5.179)%
    --(3.594,5.195)--(3.595,5.210)--cycle;
\gpfill{color=gp lt color border,opacity=0.50} (8.704,7.324)--(8.704,7.324)--(8.704,7.324)--(8.704,7.324)%
    --(8.704,7.324)--(8.704,7.324)--(8.704,7.324)--(8.704,7.324)--(8.704,7.324)%
    --(8.704,7.324)--(8.704,7.324)--(8.704,7.324)--(8.704,7.324)--(8.704,7.324)%
    --(8.704,7.324)--(8.704,7.324)--(8.704,7.324)--(8.704,7.324)--(8.704,7.324)%
    --(8.704,7.324)--(8.704,7.324)--(8.704,7.324)--(8.704,7.324)--(8.704,7.324)%
    --(8.704,7.324)--(8.704,7.324)--(8.704,7.324)--(8.704,7.324)--(8.704,7.324)%
    --(8.704,7.324)--(8.704,7.324)--(8.704,7.324)--(8.704,7.324)--(8.704,7.324)%
    --(8.704,7.324)--(8.704,7.324)--(8.704,7.324)--(8.704,7.324)--(8.704,7.324)%
    --(8.704,7.324)--(8.704,7.324)--(8.704,7.324)--(8.704,7.324)--(8.704,7.324)%
    --(8.704,7.324)--(8.704,7.324)--(8.704,7.324)--(8.704,7.324)--(8.704,7.324)%
    --(8.704,7.324)--(8.704,7.324)--(8.704,7.324)--(8.704,7.324)--(8.704,7.324)%
    --(8.704,7.324)--(8.704,7.324)--(8.704,7.324)--(8.704,7.324)--(8.704,7.324)%
    --(8.704,7.324)--(8.704,7.324)--(8.704,7.324)--(8.704,7.324)--(8.704,7.324)%
    --(8.704,7.324)--(8.704,7.324)--(8.704,7.324)--(8.704,7.324)--(8.704,7.324)%
    --(8.704,7.324)--(8.704,7.324)--(8.704,7.324)--(8.704,7.324)--(8.704,7.324)%
    --(8.704,7.324)--(8.704,7.324)--(8.704,7.324)--(8.704,7.324)--(8.704,7.324)%
    --(8.704,7.324)--(8.704,7.324)--(8.704,7.324)--(8.704,7.324)--(8.704,7.324)%
    --(8.704,7.324)--(8.704,7.324)--(8.704,7.324)--(8.704,7.324)--(8.704,7.324)%
    --(8.704,7.324)--(8.704,7.324)--(8.704,7.324)--(8.704,7.324)--(8.704,7.324)%
    --(8.704,7.324)--(8.704,7.324)--(8.704,7.324)--(8.704,7.324)--(8.704,7.324)%
    --(8.704,7.324)--(8.704,7.324)--(8.704,7.324)--(8.704,7.324)--(8.704,7.324)%
    --(8.704,7.324)--(8.704,7.324)--(8.704,7.324)--(8.704,7.324)--(8.704,7.324)%
    --(8.704,7.324)--(8.704,7.324)--(8.704,7.324)--(8.704,7.324)--(8.704,7.324)%
    --(8.704,7.324)--(8.704,7.324)--(8.704,7.324)--(8.704,7.324)--(8.704,7.324)%
    --(8.704,7.324)--cycle;
\gpfill{color=gp lt color border,opacity=0.50} (2.244,5.211)--(2.243,5.226)--(2.242,5.242)--(2.240,5.257)%
    --(2.237,5.272)--(2.233,5.287)--(2.229,5.302)--(2.224,5.317)--(2.218,5.331)%
    --(2.211,5.345)--(2.204,5.359)--(2.196,5.372)--(2.187,5.385)--(2.177,5.397)%
    --(2.167,5.409)--(2.157,5.421)--(2.145,5.431)--(2.133,5.441)--(2.121,5.451)%
    --(2.108,5.460)--(2.095,5.468)--(2.081,5.475)--(2.067,5.482)--(2.053,5.488)%
    --(2.038,5.493)--(2.023,5.497)--(2.008,5.501)--(1.993,5.504)--(1.978,5.506)%
    --(1.962,5.507)--(1.947,5.508)--(1.931,5.507)--(1.915,5.506)--(1.900,5.504)%
    --(1.885,5.501)--(1.870,5.497)--(1.855,5.493)--(1.840,5.488)--(1.826,5.482)%
    --(1.812,5.475)--(1.798,5.468)--(1.785,5.460)--(1.772,5.451)--(1.760,5.441)%
    --(1.748,5.431)--(1.736,5.421)--(1.726,5.409)--(1.716,5.397)--(1.706,5.385)%
    --(1.697,5.372)--(1.689,5.359)--(1.682,5.345)--(1.675,5.331)--(1.669,5.317)%
    --(1.664,5.302)--(1.660,5.287)--(1.656,5.272)--(1.653,5.257)--(1.651,5.242)%
    --(1.650,5.226)--(1.650,5.211)--(1.650,5.195)--(1.651,5.179)--(1.653,5.164)%
    --(1.656,5.149)--(1.660,5.134)--(1.664,5.119)--(1.669,5.104)--(1.675,5.090)%
    --(1.682,5.076)--(1.689,5.062)--(1.697,5.049)--(1.706,5.036)--(1.716,5.024)%
    --(1.726,5.012)--(1.736,5.000)--(1.748,4.990)--(1.760,4.980)--(1.772,4.970)%
    --(1.785,4.961)--(1.798,4.953)--(1.812,4.946)--(1.826,4.939)--(1.840,4.933)%
    --(1.855,4.928)--(1.870,4.924)--(1.885,4.920)--(1.900,4.917)--(1.915,4.915)%
    --(1.931,4.914)--(1.946,4.914)--(1.962,4.914)--(1.978,4.915)--(1.993,4.917)%
    --(2.008,4.920)--(2.023,4.924)--(2.038,4.928)--(2.053,4.933)--(2.067,4.939)%
    --(2.081,4.946)--(2.095,4.953)--(2.108,4.961)--(2.121,4.970)--(2.133,4.980)%
    --(2.145,4.990)--(2.157,5.000)--(2.167,5.012)--(2.177,5.024)--(2.187,5.036)%
    --(2.196,5.049)--(2.204,5.062)--(2.211,5.076)--(2.218,5.090)--(2.224,5.104)%
    --(2.229,5.119)--(2.233,5.134)--(2.237,5.149)--(2.240,5.164)--(2.242,5.179)%
    --(2.243,5.195)--(2.244,5.210)--cycle;
\gpfill{color=gp lt color border,opacity=0.50} (10.650,4.155)--(10.649,4.157)--(10.649,4.160)--(10.649,4.163)%
    --(10.648,4.166)--(10.648,4.168)--(10.647,4.171)--(10.646,4.174)--(10.645,4.176)%
    --(10.644,4.179)--(10.642,4.181)--(10.641,4.184)--(10.639,4.186)--(10.637,4.188)%
    --(10.636,4.191)--(10.634,4.193)--(10.632,4.195)--(10.629,4.196)--(10.627,4.198)%
    --(10.625,4.200)--(10.623,4.201)--(10.620,4.203)--(10.617,4.204)--(10.615,4.205)%
    --(10.612,4.206)--(10.609,4.207)--(10.607,4.207)--(10.604,4.208)--(10.601,4.208)%
    --(10.598,4.208)--(10.596,4.209)--(10.593,4.208)--(10.590,4.208)--(10.587,4.208)%
    --(10.584,4.207)--(10.582,4.207)--(10.579,4.206)--(10.576,4.205)--(10.574,4.204)%
    --(10.571,4.203)--(10.569,4.201)--(10.566,4.200)--(10.564,4.198)--(10.562,4.196)%
    --(10.559,4.195)--(10.557,4.193)--(10.555,4.191)--(10.554,4.188)--(10.552,4.186)%
    --(10.550,4.184)--(10.549,4.181)--(10.547,4.179)--(10.546,4.176)--(10.545,4.174)%
    --(10.544,4.171)--(10.543,4.168)--(10.543,4.166)--(10.542,4.163)--(10.542,4.160)%
    --(10.542,4.157)--(10.542,4.155)--(10.542,4.152)--(10.542,4.149)--(10.542,4.146)%
    --(10.543,4.143)--(10.543,4.141)--(10.544,4.138)--(10.545,4.135)--(10.546,4.133)%
    --(10.547,4.130)--(10.549,4.127)--(10.550,4.125)--(10.552,4.123)--(10.554,4.121)%
    --(10.555,4.118)--(10.557,4.116)--(10.559,4.114)--(10.562,4.113)--(10.564,4.111)%
    --(10.566,4.109)--(10.568,4.108)--(10.571,4.106)--(10.574,4.105)--(10.576,4.104)%
    --(10.579,4.103)--(10.582,4.102)--(10.584,4.102)--(10.587,4.101)--(10.590,4.101)%
    --(10.593,4.101)--(10.595,4.101)--(10.598,4.101)--(10.601,4.101)--(10.604,4.101)%
    --(10.607,4.102)--(10.609,4.102)--(10.612,4.103)--(10.615,4.104)--(10.617,4.105)%
    --(10.620,4.106)--(10.623,4.108)--(10.625,4.109)--(10.627,4.111)--(10.629,4.113)%
    --(10.632,4.114)--(10.634,4.116)--(10.636,4.118)--(10.637,4.121)--(10.639,4.123)%
    --(10.641,4.125)--(10.642,4.127)--(10.644,4.130)--(10.645,4.133)--(10.646,4.135)%
    --(10.647,4.138)--(10.648,4.141)--(10.648,4.143)--(10.649,4.146)--(10.649,4.149)%
    --(10.649,4.152)--(10.650,4.154)--cycle;
\gpfill{color=gp lt color border,opacity=0.50} (6.028,7.324)--(6.027,7.325)--(6.027,7.326)--(6.027,7.328)%
    --(6.027,7.329)--(6.027,7.330)--(6.026,7.332)--(6.026,7.333)--(6.025,7.334)%
    --(6.025,7.336)--(6.024,7.337)--(6.023,7.338)--(6.022,7.339)--(6.021,7.340)%
    --(6.021,7.342)--(6.020,7.343)--(6.019,7.344)--(6.017,7.344)--(6.016,7.345)%
    --(6.015,7.346)--(6.014,7.347)--(6.013,7.348)--(6.011,7.348)--(6.010,7.349)%
    --(6.009,7.349)--(6.007,7.350)--(6.006,7.350)--(6.005,7.350)--(6.003,7.350)%
    --(6.002,7.350)--(6.001,7.351)--(5.999,7.350)--(5.998,7.350)--(5.996,7.350)%
    --(5.995,7.350)--(5.994,7.350)--(5.992,7.349)--(5.991,7.349)--(5.990,7.348)%
    --(5.988,7.348)--(5.987,7.347)--(5.986,7.346)--(5.985,7.345)--(5.984,7.344)%
    --(5.982,7.344)--(5.981,7.343)--(5.980,7.342)--(5.980,7.340)--(5.979,7.339)%
    --(5.978,7.338)--(5.977,7.337)--(5.976,7.336)--(5.976,7.334)--(5.975,7.333)%
    --(5.975,7.332)--(5.974,7.330)--(5.974,7.329)--(5.974,7.328)--(5.974,7.326)%
    --(5.974,7.325)--(5.974,7.324)--(5.974,7.322)--(5.974,7.321)--(5.974,7.319)%
    --(5.974,7.318)--(5.974,7.317)--(5.975,7.315)--(5.975,7.314)--(5.976,7.313)%
    --(5.976,7.311)--(5.977,7.310)--(5.978,7.309)--(5.979,7.308)--(5.980,7.307)%
    --(5.980,7.305)--(5.981,7.304)--(5.982,7.303)--(5.984,7.303)--(5.985,7.302)%
    --(5.986,7.301)--(5.987,7.300)--(5.988,7.299)--(5.990,7.299)--(5.991,7.298)%
    --(5.992,7.298)--(5.994,7.297)--(5.995,7.297)--(5.996,7.297)--(5.998,7.297)%
    --(5.999,7.297)--(6.000,7.297)--(6.002,7.297)--(6.003,7.297)--(6.005,7.297)%
    --(6.006,7.297)--(6.007,7.297)--(6.009,7.298)--(6.010,7.298)--(6.011,7.299)%
    --(6.013,7.299)--(6.014,7.300)--(6.015,7.301)--(6.016,7.302)--(6.017,7.303)%
    --(6.019,7.303)--(6.020,7.304)--(6.021,7.305)--(6.021,7.307)--(6.022,7.308)%
    --(6.023,7.309)--(6.024,7.310)--(6.025,7.311)--(6.025,7.313)--(6.026,7.314)%
    --(6.026,7.315)--(6.027,7.317)--(6.027,7.318)--(6.027,7.319)--(6.027,7.321)%
    --(6.027,7.322)--(6.028,7.323)--cycle;
\gpfill{color=gp lt color border,opacity=0.50} (4.732,7.324)--(4.731,7.328)--(4.731,7.332)--(4.730,7.336)%
    --(4.730,7.341)--(4.729,7.345)--(4.727,7.349)--(4.726,7.353)--(4.724,7.357)%
    --(4.723,7.361)--(4.721,7.364)--(4.718,7.368)--(4.716,7.372)--(4.713,7.375)%
    --(4.710,7.378)--(4.707,7.381)--(4.704,7.384)--(4.701,7.387)--(4.698,7.390)%
    --(4.694,7.392)--(4.691,7.395)--(4.687,7.397)--(4.683,7.398)--(4.679,7.400)%
    --(4.675,7.401)--(4.671,7.403)--(4.667,7.404)--(4.662,7.404)--(4.658,7.405)%
    --(4.654,7.405)--(4.650,7.406)--(4.645,7.405)--(4.641,7.405)--(4.637,7.404)%
    --(4.632,7.404)--(4.628,7.403)--(4.624,7.401)--(4.620,7.400)--(4.616,7.398)%
    --(4.612,7.397)--(4.609,7.395)--(4.605,7.392)--(4.601,7.390)--(4.598,7.387)%
    --(4.595,7.384)--(4.592,7.381)--(4.589,7.378)--(4.586,7.375)--(4.583,7.372)%
    --(4.581,7.368)--(4.578,7.364)--(4.576,7.361)--(4.575,7.357)--(4.573,7.353)%
    --(4.572,7.349)--(4.570,7.345)--(4.569,7.341)--(4.569,7.336)--(4.568,7.332)%
    --(4.568,7.328)--(4.568,7.324)--(4.568,7.319)--(4.568,7.315)--(4.569,7.311)%
    --(4.569,7.306)--(4.570,7.302)--(4.572,7.298)--(4.573,7.294)--(4.575,7.290)%
    --(4.576,7.286)--(4.578,7.282)--(4.581,7.279)--(4.583,7.275)--(4.586,7.272)%
    --(4.589,7.269)--(4.592,7.266)--(4.595,7.263)--(4.598,7.260)--(4.601,7.257)%
    --(4.605,7.255)--(4.608,7.252)--(4.612,7.250)--(4.616,7.249)--(4.620,7.247)%
    --(4.624,7.246)--(4.628,7.244)--(4.632,7.243)--(4.637,7.243)--(4.641,7.242)%
    --(4.645,7.242)--(4.649,7.242)--(4.654,7.242)--(4.658,7.242)--(4.662,7.243)%
    --(4.667,7.243)--(4.671,7.244)--(4.675,7.246)--(4.679,7.247)--(4.683,7.249)%
    --(4.687,7.250)--(4.691,7.252)--(4.694,7.255)--(4.698,7.257)--(4.701,7.260)%
    --(4.704,7.263)--(4.707,7.266)--(4.710,7.269)--(4.713,7.272)--(4.716,7.275)%
    --(4.718,7.279)--(4.721,7.282)--(4.723,7.286)--(4.724,7.290)--(4.726,7.294)%
    --(4.727,7.298)--(4.729,7.302)--(4.730,7.306)--(4.730,7.311)--(4.731,7.315)%
    --(4.731,7.319)--(4.732,7.323)--cycle;
\gpfill{color=gp lt color border,opacity=0.50} (8.082,4.155)--(8.081,4.164)--(8.080,4.174)--(8.079,4.184)%
    --(8.077,4.194)--(8.075,4.203)--(8.072,4.213)--(8.069,4.222)--(8.065,4.231)%
    --(8.061,4.240)--(8.056,4.249)--(8.051,4.257)--(8.045,4.266)--(8.039,4.273)%
    --(8.033,4.281)--(8.026,4.288)--(8.019,4.295)--(8.011,4.301)--(8.004,4.307)%
    --(7.995,4.313)--(7.987,4.318)--(7.978,4.323)--(7.969,4.327)--(7.960,4.331)%
    --(7.951,4.334)--(7.941,4.337)--(7.932,4.339)--(7.922,4.341)--(7.912,4.342)%
    --(7.902,4.343)--(7.893,4.344)--(7.883,4.343)--(7.873,4.342)--(7.863,4.341)%
    --(7.853,4.339)--(7.844,4.337)--(7.834,4.334)--(7.825,4.331)--(7.816,4.327)%
    --(7.807,4.323)--(7.798,4.318)--(7.790,4.313)--(7.781,4.307)--(7.774,4.301)%
    --(7.766,4.295)--(7.759,4.288)--(7.752,4.281)--(7.746,4.273)--(7.740,4.266)%
    --(7.734,4.257)--(7.729,4.249)--(7.724,4.240)--(7.720,4.231)--(7.716,4.222)%
    --(7.713,4.213)--(7.710,4.203)--(7.708,4.194)--(7.706,4.184)--(7.705,4.174)%
    --(7.704,4.164)--(7.704,4.155)--(7.704,4.145)--(7.705,4.135)--(7.706,4.125)%
    --(7.708,4.115)--(7.710,4.106)--(7.713,4.096)--(7.716,4.087)--(7.720,4.078)%
    --(7.724,4.069)--(7.729,4.060)--(7.734,4.052)--(7.740,4.043)--(7.746,4.036)%
    --(7.752,4.028)--(7.759,4.021)--(7.766,4.014)--(7.774,4.008)--(7.781,4.002)%
    --(7.790,3.996)--(7.798,3.991)--(7.807,3.986)--(7.816,3.982)--(7.825,3.978)%
    --(7.834,3.975)--(7.844,3.972)--(7.853,3.970)--(7.863,3.968)--(7.873,3.967)%
    --(7.883,3.966)--(7.892,3.966)--(7.902,3.966)--(7.912,3.967)--(7.922,3.968)%
    --(7.932,3.970)--(7.941,3.972)--(7.951,3.975)--(7.960,3.978)--(7.969,3.982)%
    --(7.978,3.986)--(7.987,3.991)--(7.995,3.996)--(8.004,4.002)--(8.011,4.008)%
    --(8.019,4.014)--(8.026,4.021)--(8.033,4.028)--(8.039,4.036)--(8.045,4.043)%
    --(8.051,4.052)--(8.056,4.060)--(8.061,4.069)--(8.065,4.078)--(8.069,4.087)%
    --(8.072,4.096)--(8.075,4.106)--(8.077,4.115)--(8.079,4.125)--(8.080,4.135)%
    --(8.081,4.145)--(8.082,4.154)--cycle;
\gpfill{color=gp lt color border,opacity=0.50} (3.379,7.324)--(3.378,7.328)--(3.378,7.332)--(3.378,7.336)%
    --(3.377,7.340)--(3.376,7.344)--(3.375,7.349)--(3.373,7.353)--(3.371,7.356)%
    --(3.370,7.360)--(3.368,7.364)--(3.365,7.368)--(3.363,7.371)--(3.360,7.374)%
    --(3.358,7.378)--(3.355,7.381)--(3.352,7.384)--(3.348,7.386)--(3.345,7.389)%
    --(3.342,7.391)--(3.338,7.394)--(3.334,7.396)--(3.330,7.397)--(3.327,7.399)%
    --(3.323,7.401)--(3.318,7.402)--(3.314,7.403)--(3.310,7.404)--(3.306,7.404)%
    --(3.302,7.404)--(3.298,7.405)--(3.293,7.404)--(3.289,7.404)--(3.285,7.404)%
    --(3.281,7.403)--(3.277,7.402)--(3.272,7.401)--(3.268,7.399)--(3.265,7.397)%
    --(3.261,7.396)--(3.257,7.394)--(3.253,7.391)--(3.250,7.389)--(3.247,7.386)%
    --(3.243,7.384)--(3.240,7.381)--(3.237,7.378)--(3.235,7.374)--(3.232,7.371)%
    --(3.230,7.368)--(3.227,7.364)--(3.225,7.360)--(3.224,7.356)--(3.222,7.353)%
    --(3.220,7.349)--(3.219,7.344)--(3.218,7.340)--(3.217,7.336)--(3.217,7.332)%
    --(3.217,7.328)--(3.217,7.324)--(3.217,7.319)--(3.217,7.315)--(3.217,7.311)%
    --(3.218,7.307)--(3.219,7.303)--(3.220,7.298)--(3.222,7.294)--(3.224,7.291)%
    --(3.225,7.287)--(3.227,7.283)--(3.230,7.279)--(3.232,7.276)--(3.235,7.273)%
    --(3.237,7.269)--(3.240,7.266)--(3.243,7.263)--(3.247,7.261)--(3.250,7.258)%
    --(3.253,7.256)--(3.257,7.253)--(3.261,7.251)--(3.265,7.250)--(3.268,7.248)%
    --(3.272,7.246)--(3.277,7.245)--(3.281,7.244)--(3.285,7.243)--(3.289,7.243)%
    --(3.293,7.243)--(3.297,7.243)--(3.302,7.243)--(3.306,7.243)--(3.310,7.243)%
    --(3.314,7.244)--(3.318,7.245)--(3.323,7.246)--(3.327,7.248)--(3.330,7.250)%
    --(3.334,7.251)--(3.338,7.253)--(3.342,7.256)--(3.345,7.258)--(3.348,7.261)%
    --(3.352,7.263)--(3.355,7.266)--(3.358,7.269)--(3.360,7.273)--(3.363,7.276)%
    --(3.365,7.279)--(3.368,7.283)--(3.370,7.287)--(3.371,7.291)--(3.373,7.294)%
    --(3.375,7.298)--(3.376,7.303)--(3.377,7.307)--(3.378,7.311)--(3.378,7.315)%
    --(3.378,7.319)--(3.379,7.323)--cycle;
\gpfill{color=gp lt color border,opacity=0.50} (10.596,3.098)--(10.596,3.098)--(10.596,3.098)--(10.596,3.098)%
    --(10.596,3.098)--(10.596,3.098)--(10.596,3.098)--(10.596,3.098)--(10.596,3.098)%
    --(10.596,3.098)--(10.596,3.098)--(10.596,3.098)--(10.596,3.098)--(10.596,3.098)%
    --(10.596,3.098)--(10.596,3.098)--(10.596,3.098)--(10.596,3.098)--(10.596,3.098)%
    --(10.596,3.098)--(10.596,3.098)--(10.596,3.098)--(10.596,3.098)--(10.596,3.098)%
    --(10.596,3.098)--(10.596,3.098)--(10.596,3.098)--(10.596,3.098)--(10.596,3.098)%
    --(10.596,3.098)--(10.596,3.098)--(10.596,3.098)--(10.596,3.098)--(10.596,3.098)%
    --(10.596,3.098)--(10.596,3.098)--(10.596,3.098)--(10.596,3.098)--(10.596,3.098)%
    --(10.596,3.098)--(10.596,3.098)--(10.596,3.098)--(10.596,3.098)--(10.596,3.098)%
    --(10.596,3.098)--(10.596,3.098)--(10.596,3.098)--(10.596,3.098)--(10.596,3.098)%
    --(10.596,3.098)--(10.596,3.098)--(10.596,3.098)--(10.596,3.098)--(10.596,3.098)%
    --(10.596,3.098)--(10.596,3.098)--(10.596,3.098)--(10.596,3.098)--(10.596,3.098)%
    --(10.596,3.098)--(10.596,3.098)--(10.596,3.098)--(10.596,3.098)--(10.596,3.098)%
    --(10.596,3.098)--(10.596,3.098)--(10.596,3.098)--(10.596,3.098)--(10.596,3.098)%
    --(10.596,3.098)--(10.596,3.098)--(10.596,3.098)--(10.596,3.098)--(10.596,3.098)%
    --(10.596,3.098)--(10.596,3.098)--(10.596,3.098)--(10.596,3.098)--(10.596,3.098)%
    --(10.596,3.098)--(10.596,3.098)--(10.596,3.098)--(10.596,3.098)--(10.596,3.098)%
    --(10.596,3.098)--(10.596,3.098)--(10.596,3.098)--(10.596,3.098)--(10.596,3.098)%
    --(10.596,3.098)--(10.596,3.098)--(10.596,3.098)--(10.596,3.098)--(10.596,3.098)%
    --(10.596,3.098)--(10.596,3.098)--(10.596,3.098)--(10.596,3.098)--(10.596,3.098)%
    --(10.596,3.098)--(10.596,3.098)--(10.596,3.098)--(10.596,3.098)--(10.596,3.098)%
    --(10.596,3.098)--(10.596,3.098)--(10.596,3.098)--(10.596,3.098)--(10.596,3.098)%
    --(10.596,3.098)--(10.596,3.098)--(10.596,3.098)--(10.596,3.098)--(10.596,3.098)%
    --(10.596,3.098)--(10.596,3.098)--(10.596,3.098)--(10.596,3.098)--(10.596,3.098)%
    --(10.596,3.098)--cycle;
\gpfill{color=gp lt color border,opacity=0.50} (11.433,4.155)--(11.432,4.156)--(11.432,4.157)--(11.432,4.159)%
    --(11.432,4.160)--(11.432,4.161)--(11.431,4.163)--(11.431,4.164)--(11.430,4.165)%
    --(11.430,4.167)--(11.429,4.168)--(11.428,4.169)--(11.427,4.170)--(11.426,4.171)%
    --(11.426,4.173)--(11.425,4.174)--(11.424,4.175)--(11.422,4.175)--(11.421,4.176)%
    --(11.420,4.177)--(11.419,4.178)--(11.418,4.179)--(11.416,4.179)--(11.415,4.180)%
    --(11.414,4.180)--(11.412,4.181)--(11.411,4.181)--(11.410,4.181)--(11.408,4.181)%
    --(11.407,4.181)--(11.406,4.182)--(11.404,4.181)--(11.403,4.181)--(11.401,4.181)%
    --(11.400,4.181)--(11.399,4.181)--(11.397,4.180)--(11.396,4.180)--(11.395,4.179)%
    --(11.393,4.179)--(11.392,4.178)--(11.391,4.177)--(11.390,4.176)--(11.389,4.175)%
    --(11.387,4.175)--(11.386,4.174)--(11.385,4.173)--(11.385,4.171)--(11.384,4.170)%
    --(11.383,4.169)--(11.382,4.168)--(11.381,4.167)--(11.381,4.165)--(11.380,4.164)%
    --(11.380,4.163)--(11.379,4.161)--(11.379,4.160)--(11.379,4.159)--(11.379,4.157)%
    --(11.379,4.156)--(11.379,4.155)--(11.379,4.153)--(11.379,4.152)--(11.379,4.150)%
    --(11.379,4.149)--(11.379,4.148)--(11.380,4.146)--(11.380,4.145)--(11.381,4.144)%
    --(11.381,4.142)--(11.382,4.141)--(11.383,4.140)--(11.384,4.139)--(11.385,4.138)%
    --(11.385,4.136)--(11.386,4.135)--(11.387,4.134)--(11.389,4.134)--(11.390,4.133)%
    --(11.391,4.132)--(11.392,4.131)--(11.393,4.130)--(11.395,4.130)--(11.396,4.129)%
    --(11.397,4.129)--(11.399,4.128)--(11.400,4.128)--(11.401,4.128)--(11.403,4.128)%
    --(11.404,4.128)--(11.405,4.128)--(11.407,4.128)--(11.408,4.128)--(11.410,4.128)%
    --(11.411,4.128)--(11.412,4.128)--(11.414,4.129)--(11.415,4.129)--(11.416,4.130)%
    --(11.418,4.130)--(11.419,4.131)--(11.420,4.132)--(11.421,4.133)--(11.422,4.134)%
    --(11.424,4.134)--(11.425,4.135)--(11.426,4.136)--(11.426,4.138)--(11.427,4.139)%
    --(11.428,4.140)--(11.429,4.141)--(11.430,4.142)--(11.430,4.144)--(11.431,4.145)%
    --(11.431,4.146)--(11.432,4.148)--(11.432,4.149)--(11.432,4.150)--(11.432,4.152)%
    --(11.432,4.153)--(11.433,4.154)--cycle;
\gpfill{color=gp lt color border,opacity=0.50} (11.974,6.268)--(11.973,6.269)--(11.973,6.270)--(11.973,6.272)%
    --(11.973,6.273)--(11.973,6.274)--(11.972,6.276)--(11.972,6.277)--(11.971,6.278)%
    --(11.971,6.280)--(11.970,6.281)--(11.969,6.282)--(11.968,6.283)--(11.967,6.284)%
    --(11.967,6.286)--(11.966,6.287)--(11.965,6.288)--(11.963,6.288)--(11.962,6.289)%
    --(11.961,6.290)--(11.960,6.291)--(11.959,6.292)--(11.957,6.292)--(11.956,6.293)%
    --(11.955,6.293)--(11.953,6.294)--(11.952,6.294)--(11.951,6.294)--(11.949,6.294)%
    --(11.948,6.294)--(11.947,6.295)--(11.945,6.294)--(11.944,6.294)--(11.942,6.294)%
    --(11.941,6.294)--(11.940,6.294)--(11.938,6.293)--(11.937,6.293)--(11.936,6.292)%
    --(11.934,6.292)--(11.933,6.291)--(11.932,6.290)--(11.931,6.289)--(11.930,6.288)%
    --(11.928,6.288)--(11.927,6.287)--(11.926,6.286)--(11.926,6.284)--(11.925,6.283)%
    --(11.924,6.282)--(11.923,6.281)--(11.922,6.280)--(11.922,6.278)--(11.921,6.277)%
    --(11.921,6.276)--(11.920,6.274)--(11.920,6.273)--(11.920,6.272)--(11.920,6.270)%
    --(11.920,6.269)--(11.920,6.268)--(11.920,6.266)--(11.920,6.265)--(11.920,6.263)%
    --(11.920,6.262)--(11.920,6.261)--(11.921,6.259)--(11.921,6.258)--(11.922,6.257)%
    --(11.922,6.255)--(11.923,6.254)--(11.924,6.253)--(11.925,6.252)--(11.926,6.251)%
    --(11.926,6.249)--(11.927,6.248)--(11.928,6.247)--(11.930,6.247)--(11.931,6.246)%
    --(11.932,6.245)--(11.933,6.244)--(11.934,6.243)--(11.936,6.243)--(11.937,6.242)%
    --(11.938,6.242)--(11.940,6.241)--(11.941,6.241)--(11.942,6.241)--(11.944,6.241)%
    --(11.945,6.241)--(11.946,6.241)--(11.948,6.241)--(11.949,6.241)--(11.951,6.241)%
    --(11.952,6.241)--(11.953,6.241)--(11.955,6.242)--(11.956,6.242)--(11.957,6.243)%
    --(11.959,6.243)--(11.960,6.244)--(11.961,6.245)--(11.962,6.246)--(11.963,6.247)%
    --(11.965,6.247)--(11.966,6.248)--(11.967,6.249)--(11.967,6.251)--(11.968,6.252)%
    --(11.969,6.253)--(11.970,6.254)--(11.971,6.255)--(11.971,6.257)--(11.972,6.258)%
    --(11.972,6.259)--(11.973,6.261)--(11.973,6.262)--(11.973,6.263)--(11.973,6.265)%
    --(11.973,6.266)--(11.974,6.267)--cycle;
\gpfill{color=gp lt color border,opacity=0.50} (10.623,6.268)--(10.622,6.269)--(10.622,6.270)--(10.622,6.272)%
    --(10.622,6.273)--(10.622,6.274)--(10.621,6.276)--(10.621,6.277)--(10.620,6.278)%
    --(10.620,6.280)--(10.619,6.281)--(10.618,6.282)--(10.617,6.283)--(10.616,6.284)%
    --(10.616,6.286)--(10.615,6.287)--(10.614,6.288)--(10.612,6.288)--(10.611,6.289)%
    --(10.610,6.290)--(10.609,6.291)--(10.608,6.292)--(10.606,6.292)--(10.605,6.293)%
    --(10.604,6.293)--(10.602,6.294)--(10.601,6.294)--(10.600,6.294)--(10.598,6.294)%
    --(10.597,6.294)--(10.596,6.295)--(10.594,6.294)--(10.593,6.294)--(10.591,6.294)%
    --(10.590,6.294)--(10.589,6.294)--(10.587,6.293)--(10.586,6.293)--(10.585,6.292)%
    --(10.583,6.292)--(10.582,6.291)--(10.581,6.290)--(10.580,6.289)--(10.579,6.288)%
    --(10.577,6.288)--(10.576,6.287)--(10.575,6.286)--(10.575,6.284)--(10.574,6.283)%
    --(10.573,6.282)--(10.572,6.281)--(10.571,6.280)--(10.571,6.278)--(10.570,6.277)%
    --(10.570,6.276)--(10.569,6.274)--(10.569,6.273)--(10.569,6.272)--(10.569,6.270)%
    --(10.569,6.269)--(10.569,6.268)--(10.569,6.266)--(10.569,6.265)--(10.569,6.263)%
    --(10.569,6.262)--(10.569,6.261)--(10.570,6.259)--(10.570,6.258)--(10.571,6.257)%
    --(10.571,6.255)--(10.572,6.254)--(10.573,6.253)--(10.574,6.252)--(10.575,6.251)%
    --(10.575,6.249)--(10.576,6.248)--(10.577,6.247)--(10.579,6.247)--(10.580,6.246)%
    --(10.581,6.245)--(10.582,6.244)--(10.583,6.243)--(10.585,6.243)--(10.586,6.242)%
    --(10.587,6.242)--(10.589,6.241)--(10.590,6.241)--(10.591,6.241)--(10.593,6.241)%
    --(10.594,6.241)--(10.595,6.241)--(10.597,6.241)--(10.598,6.241)--(10.600,6.241)%
    --(10.601,6.241)--(10.602,6.241)--(10.604,6.242)--(10.605,6.242)--(10.606,6.243)%
    --(10.608,6.243)--(10.609,6.244)--(10.610,6.245)--(10.611,6.246)--(10.612,6.247)%
    --(10.614,6.247)--(10.615,6.248)--(10.616,6.249)--(10.616,6.251)--(10.617,6.252)%
    --(10.618,6.253)--(10.619,6.254)--(10.620,6.255)--(10.620,6.257)--(10.621,6.258)%
    --(10.621,6.259)--(10.622,6.261)--(10.622,6.262)--(10.622,6.263)--(10.622,6.265)%
    --(10.622,6.266)--(10.623,6.267)--cycle;
\gpfill{color=gp lt color border,opacity=0.50} (8.921,4.155)--(8.920,4.166)--(8.919,4.177)--(8.918,4.188)%
    --(8.916,4.200)--(8.913,4.211)--(8.910,4.222)--(8.906,4.232)--(8.902,4.243)%
    --(8.897,4.253)--(8.891,4.263)--(8.885,4.273)--(8.879,4.282)--(8.872,4.291)%
    --(8.865,4.300)--(8.857,4.308)--(8.849,4.316)--(8.840,4.323)--(8.831,4.330)%
    --(8.822,4.336)--(8.812,4.342)--(8.802,4.348)--(8.792,4.353)--(8.781,4.357)%
    --(8.771,4.361)--(8.760,4.364)--(8.749,4.367)--(8.737,4.369)--(8.726,4.370)%
    --(8.715,4.371)--(8.704,4.372)--(8.692,4.371)--(8.681,4.370)--(8.670,4.369)%
    --(8.658,4.367)--(8.647,4.364)--(8.636,4.361)--(8.626,4.357)--(8.615,4.353)%
    --(8.605,4.348)--(8.595,4.342)--(8.585,4.336)--(8.576,4.330)--(8.567,4.323)%
    --(8.558,4.316)--(8.550,4.308)--(8.542,4.300)--(8.535,4.291)--(8.528,4.282)%
    --(8.522,4.273)--(8.516,4.263)--(8.510,4.253)--(8.505,4.243)--(8.501,4.232)%
    --(8.497,4.222)--(8.494,4.211)--(8.491,4.200)--(8.489,4.188)--(8.488,4.177)%
    --(8.487,4.166)--(8.487,4.155)--(8.487,4.143)--(8.488,4.132)--(8.489,4.121)%
    --(8.491,4.109)--(8.494,4.098)--(8.497,4.087)--(8.501,4.077)--(8.505,4.066)%
    --(8.510,4.056)--(8.516,4.046)--(8.522,4.036)--(8.528,4.027)--(8.535,4.018)%
    --(8.542,4.009)--(8.550,4.001)--(8.558,3.993)--(8.567,3.986)--(8.576,3.979)%
    --(8.585,3.973)--(8.595,3.967)--(8.605,3.961)--(8.615,3.956)--(8.626,3.952)%
    --(8.636,3.948)--(8.647,3.945)--(8.658,3.942)--(8.670,3.940)--(8.681,3.939)%
    --(8.692,3.938)--(8.703,3.938)--(8.715,3.938)--(8.726,3.939)--(8.737,3.940)%
    --(8.749,3.942)--(8.760,3.945)--(8.771,3.948)--(8.781,3.952)--(8.792,3.956)%
    --(8.802,3.961)--(8.812,3.967)--(8.822,3.973)--(8.831,3.979)--(8.840,3.986)%
    --(8.849,3.993)--(8.857,4.001)--(8.865,4.009)--(8.872,4.018)--(8.879,4.027)%
    --(8.885,4.036)--(8.891,4.046)--(8.897,4.056)--(8.902,4.066)--(8.906,4.077)%
    --(8.910,4.087)--(8.913,4.098)--(8.916,4.109)--(8.918,4.121)--(8.919,4.132)%
    --(8.920,4.143)--(8.921,4.154)--cycle;
\gpfill{color=gp lt color border,opacity=0.50} (11.433,3.098)--(11.432,3.099)--(11.432,3.100)--(11.432,3.102)%
    --(11.432,3.103)--(11.432,3.104)--(11.431,3.106)--(11.431,3.107)--(11.430,3.108)%
    --(11.430,3.110)--(11.429,3.111)--(11.428,3.112)--(11.427,3.113)--(11.426,3.114)%
    --(11.426,3.116)--(11.425,3.117)--(11.424,3.118)--(11.422,3.118)--(11.421,3.119)%
    --(11.420,3.120)--(11.419,3.121)--(11.418,3.122)--(11.416,3.122)--(11.415,3.123)%
    --(11.414,3.123)--(11.412,3.124)--(11.411,3.124)--(11.410,3.124)--(11.408,3.124)%
    --(11.407,3.124)--(11.406,3.125)--(11.404,3.124)--(11.403,3.124)--(11.401,3.124)%
    --(11.400,3.124)--(11.399,3.124)--(11.397,3.123)--(11.396,3.123)--(11.395,3.122)%
    --(11.393,3.122)--(11.392,3.121)--(11.391,3.120)--(11.390,3.119)--(11.389,3.118)%
    --(11.387,3.118)--(11.386,3.117)--(11.385,3.116)--(11.385,3.114)--(11.384,3.113)%
    --(11.383,3.112)--(11.382,3.111)--(11.381,3.110)--(11.381,3.108)--(11.380,3.107)%
    --(11.380,3.106)--(11.379,3.104)--(11.379,3.103)--(11.379,3.102)--(11.379,3.100)%
    --(11.379,3.099)--(11.379,3.098)--(11.379,3.096)--(11.379,3.095)--(11.379,3.093)%
    --(11.379,3.092)--(11.379,3.091)--(11.380,3.089)--(11.380,3.088)--(11.381,3.087)%
    --(11.381,3.085)--(11.382,3.084)--(11.383,3.083)--(11.384,3.082)--(11.385,3.081)%
    --(11.385,3.079)--(11.386,3.078)--(11.387,3.077)--(11.389,3.077)--(11.390,3.076)%
    --(11.391,3.075)--(11.392,3.074)--(11.393,3.073)--(11.395,3.073)--(11.396,3.072)%
    --(11.397,3.072)--(11.399,3.071)--(11.400,3.071)--(11.401,3.071)--(11.403,3.071)%
    --(11.404,3.071)--(11.405,3.071)--(11.407,3.071)--(11.408,3.071)--(11.410,3.071)%
    --(11.411,3.071)--(11.412,3.071)--(11.414,3.072)--(11.415,3.072)--(11.416,3.073)%
    --(11.418,3.073)--(11.419,3.074)--(11.420,3.075)--(11.421,3.076)--(11.422,3.077)%
    --(11.424,3.077)--(11.425,3.078)--(11.426,3.079)--(11.426,3.081)--(11.427,3.082)%
    --(11.428,3.083)--(11.429,3.084)--(11.430,3.085)--(11.430,3.087)--(11.431,3.088)%
    --(11.431,3.089)--(11.432,3.091)--(11.432,3.092)--(11.432,3.093)--(11.432,3.095)%
    --(11.432,3.096)--(11.433,3.097)--cycle;
\gpfill{color=gp lt color border,opacity=0.50} (5.487,4.155)--(5.486,4.170)--(5.485,4.186)--(5.483,4.201)%
    --(5.480,4.216)--(5.476,4.231)--(5.472,4.246)--(5.467,4.261)--(5.461,4.275)%
    --(5.454,4.289)--(5.447,4.303)--(5.439,4.316)--(5.430,4.329)--(5.420,4.341)%
    --(5.410,4.353)--(5.400,4.365)--(5.388,4.375)--(5.376,4.385)--(5.364,4.395)%
    --(5.351,4.404)--(5.338,4.412)--(5.324,4.419)--(5.310,4.426)--(5.296,4.432)%
    --(5.281,4.437)--(5.266,4.441)--(5.251,4.445)--(5.236,4.448)--(5.221,4.450)%
    --(5.205,4.451)--(5.190,4.452)--(5.174,4.451)--(5.158,4.450)--(5.143,4.448)%
    --(5.128,4.445)--(5.113,4.441)--(5.098,4.437)--(5.083,4.432)--(5.069,4.426)%
    --(5.055,4.419)--(5.041,4.412)--(5.028,4.404)--(5.015,4.395)--(5.003,4.385)%
    --(4.991,4.375)--(4.979,4.365)--(4.969,4.353)--(4.959,4.341)--(4.949,4.329)%
    --(4.940,4.316)--(4.932,4.303)--(4.925,4.289)--(4.918,4.275)--(4.912,4.261)%
    --(4.907,4.246)--(4.903,4.231)--(4.899,4.216)--(4.896,4.201)--(4.894,4.186)%
    --(4.893,4.170)--(4.893,4.155)--(4.893,4.139)--(4.894,4.123)--(4.896,4.108)%
    --(4.899,4.093)--(4.903,4.078)--(4.907,4.063)--(4.912,4.048)--(4.918,4.034)%
    --(4.925,4.020)--(4.932,4.006)--(4.940,3.993)--(4.949,3.980)--(4.959,3.968)%
    --(4.969,3.956)--(4.979,3.944)--(4.991,3.934)--(5.003,3.924)--(5.015,3.914)%
    --(5.028,3.905)--(5.041,3.897)--(5.055,3.890)--(5.069,3.883)--(5.083,3.877)%
    --(5.098,3.872)--(5.113,3.868)--(5.128,3.864)--(5.143,3.861)--(5.158,3.859)%
    --(5.174,3.858)--(5.189,3.858)--(5.205,3.858)--(5.221,3.859)--(5.236,3.861)%
    --(5.251,3.864)--(5.266,3.868)--(5.281,3.872)--(5.296,3.877)--(5.310,3.883)%
    --(5.324,3.890)--(5.338,3.897)--(5.351,3.905)--(5.364,3.914)--(5.376,3.924)%
    --(5.388,3.934)--(5.400,3.944)--(5.410,3.956)--(5.420,3.968)--(5.430,3.980)%
    --(5.439,3.993)--(5.447,4.006)--(5.454,4.020)--(5.461,4.034)--(5.467,4.048)%
    --(5.472,4.063)--(5.476,4.078)--(5.480,4.093)--(5.483,4.108)--(5.485,4.123)%
    --(5.486,4.139)--(5.487,4.154)--cycle;
\gpfill{color=gp lt color border,opacity=0.50} (7.974,3.098)--(7.973,3.102)--(7.973,3.106)--(7.973,3.110)%
    --(7.972,3.114)--(7.971,3.118)--(7.970,3.123)--(7.968,3.127)--(7.966,3.130)%
    --(7.965,3.134)--(7.963,3.138)--(7.960,3.142)--(7.958,3.145)--(7.955,3.148)%
    --(7.953,3.152)--(7.950,3.155)--(7.947,3.158)--(7.943,3.160)--(7.940,3.163)%
    --(7.937,3.165)--(7.933,3.168)--(7.929,3.170)--(7.925,3.171)--(7.922,3.173)%
    --(7.918,3.175)--(7.913,3.176)--(7.909,3.177)--(7.905,3.178)--(7.901,3.178)%
    --(7.897,3.178)--(7.893,3.179)--(7.888,3.178)--(7.884,3.178)--(7.880,3.178)%
    --(7.876,3.177)--(7.872,3.176)--(7.867,3.175)--(7.863,3.173)--(7.860,3.171)%
    --(7.856,3.170)--(7.852,3.168)--(7.848,3.165)--(7.845,3.163)--(7.842,3.160)%
    --(7.838,3.158)--(7.835,3.155)--(7.832,3.152)--(7.830,3.148)--(7.827,3.145)%
    --(7.825,3.142)--(7.822,3.138)--(7.820,3.134)--(7.819,3.130)--(7.817,3.127)%
    --(7.815,3.123)--(7.814,3.118)--(7.813,3.114)--(7.812,3.110)--(7.812,3.106)%
    --(7.812,3.102)--(7.812,3.098)--(7.812,3.093)--(7.812,3.089)--(7.812,3.085)%
    --(7.813,3.081)--(7.814,3.077)--(7.815,3.072)--(7.817,3.068)--(7.819,3.065)%
    --(7.820,3.061)--(7.822,3.057)--(7.825,3.053)--(7.827,3.050)--(7.830,3.047)%
    --(7.832,3.043)--(7.835,3.040)--(7.838,3.037)--(7.842,3.035)--(7.845,3.032)%
    --(7.848,3.030)--(7.852,3.027)--(7.856,3.025)--(7.860,3.024)--(7.863,3.022)%
    --(7.867,3.020)--(7.872,3.019)--(7.876,3.018)--(7.880,3.017)--(7.884,3.017)%
    --(7.888,3.017)--(7.892,3.017)--(7.897,3.017)--(7.901,3.017)--(7.905,3.017)%
    --(7.909,3.018)--(7.913,3.019)--(7.918,3.020)--(7.922,3.022)--(7.925,3.024)%
    --(7.929,3.025)--(7.933,3.027)--(7.937,3.030)--(7.940,3.032)--(7.943,3.035)%
    --(7.947,3.037)--(7.950,3.040)--(7.953,3.043)--(7.955,3.047)--(7.958,3.050)%
    --(7.960,3.053)--(7.963,3.057)--(7.965,3.061)--(7.966,3.065)--(7.968,3.068)%
    --(7.970,3.072)--(7.971,3.077)--(7.972,3.081)--(7.973,3.085)--(7.973,3.089)%
    --(7.973,3.093)--(7.974,3.097)--cycle;
\gpfill{color=gp lt color border,opacity=0.50} (9.271,6.268)--(9.270,6.269)--(9.270,6.270)--(9.270,6.272)%
    --(9.270,6.273)--(9.270,6.274)--(9.269,6.276)--(9.269,6.277)--(9.268,6.278)%
    --(9.268,6.280)--(9.267,6.281)--(9.266,6.282)--(9.265,6.283)--(9.264,6.284)%
    --(9.264,6.286)--(9.263,6.287)--(9.262,6.288)--(9.260,6.288)--(9.259,6.289)%
    --(9.258,6.290)--(9.257,6.291)--(9.256,6.292)--(9.254,6.292)--(9.253,6.293)%
    --(9.252,6.293)--(9.250,6.294)--(9.249,6.294)--(9.248,6.294)--(9.246,6.294)%
    --(9.245,6.294)--(9.244,6.295)--(9.242,6.294)--(9.241,6.294)--(9.239,6.294)%
    --(9.238,6.294)--(9.237,6.294)--(9.235,6.293)--(9.234,6.293)--(9.233,6.292)%
    --(9.231,6.292)--(9.230,6.291)--(9.229,6.290)--(9.228,6.289)--(9.227,6.288)%
    --(9.225,6.288)--(9.224,6.287)--(9.223,6.286)--(9.223,6.284)--(9.222,6.283)%
    --(9.221,6.282)--(9.220,6.281)--(9.219,6.280)--(9.219,6.278)--(9.218,6.277)%
    --(9.218,6.276)--(9.217,6.274)--(9.217,6.273)--(9.217,6.272)--(9.217,6.270)%
    --(9.217,6.269)--(9.217,6.268)--(9.217,6.266)--(9.217,6.265)--(9.217,6.263)%
    --(9.217,6.262)--(9.217,6.261)--(9.218,6.259)--(9.218,6.258)--(9.219,6.257)%
    --(9.219,6.255)--(9.220,6.254)--(9.221,6.253)--(9.222,6.252)--(9.223,6.251)%
    --(9.223,6.249)--(9.224,6.248)--(9.225,6.247)--(9.227,6.247)--(9.228,6.246)%
    --(9.229,6.245)--(9.230,6.244)--(9.231,6.243)--(9.233,6.243)--(9.234,6.242)%
    --(9.235,6.242)--(9.237,6.241)--(9.238,6.241)--(9.239,6.241)--(9.241,6.241)%
    --(9.242,6.241)--(9.243,6.241)--(9.245,6.241)--(9.246,6.241)--(9.248,6.241)%
    --(9.249,6.241)--(9.250,6.241)--(9.252,6.242)--(9.253,6.242)--(9.254,6.243)%
    --(9.256,6.243)--(9.257,6.244)--(9.258,6.245)--(9.259,6.246)--(9.260,6.247)%
    --(9.262,6.247)--(9.263,6.248)--(9.264,6.249)--(9.264,6.251)--(9.265,6.252)%
    --(9.266,6.253)--(9.267,6.254)--(9.268,6.255)--(9.268,6.257)--(9.269,6.258)%
    --(9.269,6.259)--(9.270,6.261)--(9.270,6.262)--(9.270,6.263)--(9.270,6.265)%
    --(9.270,6.266)--(9.271,6.267)--cycle;
\gpfill{color=gp lt color border,opacity=0.50} (10.704,5.211)--(10.703,5.216)--(10.703,5.222)--(10.702,5.227)%
    --(10.701,5.233)--(10.700,5.238)--(10.698,5.244)--(10.696,5.249)--(10.694,5.254)%
    --(10.692,5.260)--(10.689,5.264)--(10.686,5.269)--(10.683,5.274)--(10.679,5.278)%
    --(10.676,5.283)--(10.672,5.287)--(10.668,5.291)--(10.663,5.294)--(10.659,5.298)%
    --(10.654,5.301)--(10.650,5.304)--(10.645,5.307)--(10.639,5.309)--(10.634,5.311)%
    --(10.629,5.313)--(10.623,5.315)--(10.618,5.316)--(10.612,5.317)--(10.607,5.318)%
    --(10.601,5.318)--(10.596,5.319)--(10.590,5.318)--(10.584,5.318)--(10.579,5.317)%
    --(10.573,5.316)--(10.568,5.315)--(10.562,5.313)--(10.557,5.311)--(10.552,5.309)%
    --(10.546,5.307)--(10.542,5.304)--(10.537,5.301)--(10.532,5.298)--(10.528,5.294)%
    --(10.523,5.291)--(10.519,5.287)--(10.515,5.283)--(10.512,5.278)--(10.508,5.274)%
    --(10.505,5.269)--(10.502,5.264)--(10.499,5.260)--(10.497,5.254)--(10.495,5.249)%
    --(10.493,5.244)--(10.491,5.238)--(10.490,5.233)--(10.489,5.227)--(10.488,5.222)%
    --(10.488,5.216)--(10.488,5.211)--(10.488,5.205)--(10.488,5.199)--(10.489,5.194)%
    --(10.490,5.188)--(10.491,5.183)--(10.493,5.177)--(10.495,5.172)--(10.497,5.167)%
    --(10.499,5.161)--(10.502,5.156)--(10.505,5.152)--(10.508,5.147)--(10.512,5.143)%
    --(10.515,5.138)--(10.519,5.134)--(10.523,5.130)--(10.528,5.127)--(10.532,5.123)%
    --(10.537,5.120)--(10.541,5.117)--(10.546,5.114)--(10.552,5.112)--(10.557,5.110)%
    --(10.562,5.108)--(10.568,5.106)--(10.573,5.105)--(10.579,5.104)--(10.584,5.103)%
    --(10.590,5.103)--(10.595,5.103)--(10.601,5.103)--(10.607,5.103)--(10.612,5.104)%
    --(10.618,5.105)--(10.623,5.106)--(10.629,5.108)--(10.634,5.110)--(10.639,5.112)%
    --(10.645,5.114)--(10.650,5.117)--(10.654,5.120)--(10.659,5.123)--(10.663,5.127)%
    --(10.668,5.130)--(10.672,5.134)--(10.676,5.138)--(10.679,5.143)--(10.683,5.147)%
    --(10.686,5.152)--(10.689,5.156)--(10.692,5.161)--(10.694,5.167)--(10.696,5.172)%
    --(10.698,5.177)--(10.700,5.183)--(10.701,5.188)--(10.702,5.194)--(10.703,5.199)%
    --(10.703,5.205)--(10.704,5.210)--cycle;
\gpfill{color=gp lt color border,opacity=0.50} (6.298,4.155)--(6.297,4.170)--(6.296,4.186)--(6.294,4.201)%
    --(6.291,4.216)--(6.287,4.231)--(6.283,4.246)--(6.278,4.261)--(6.272,4.275)%
    --(6.265,4.289)--(6.258,4.303)--(6.250,4.316)--(6.241,4.329)--(6.231,4.341)%
    --(6.221,4.353)--(6.211,4.365)--(6.199,4.375)--(6.187,4.385)--(6.175,4.395)%
    --(6.162,4.404)--(6.149,4.412)--(6.135,4.419)--(6.121,4.426)--(6.107,4.432)%
    --(6.092,4.437)--(6.077,4.441)--(6.062,4.445)--(6.047,4.448)--(6.032,4.450)%
    --(6.016,4.451)--(6.001,4.452)--(5.985,4.451)--(5.969,4.450)--(5.954,4.448)%
    --(5.939,4.445)--(5.924,4.441)--(5.909,4.437)--(5.894,4.432)--(5.880,4.426)%
    --(5.866,4.419)--(5.852,4.412)--(5.839,4.404)--(5.826,4.395)--(5.814,4.385)%
    --(5.802,4.375)--(5.790,4.365)--(5.780,4.353)--(5.770,4.341)--(5.760,4.329)%
    --(5.751,4.316)--(5.743,4.303)--(5.736,4.289)--(5.729,4.275)--(5.723,4.261)%
    --(5.718,4.246)--(5.714,4.231)--(5.710,4.216)--(5.707,4.201)--(5.705,4.186)%
    --(5.704,4.170)--(5.704,4.155)--(5.704,4.139)--(5.705,4.123)--(5.707,4.108)%
    --(5.710,4.093)--(5.714,4.078)--(5.718,4.063)--(5.723,4.048)--(5.729,4.034)%
    --(5.736,4.020)--(5.743,4.006)--(5.751,3.993)--(5.760,3.980)--(5.770,3.968)%
    --(5.780,3.956)--(5.790,3.944)--(5.802,3.934)--(5.814,3.924)--(5.826,3.914)%
    --(5.839,3.905)--(5.852,3.897)--(5.866,3.890)--(5.880,3.883)--(5.894,3.877)%
    --(5.909,3.872)--(5.924,3.868)--(5.939,3.864)--(5.954,3.861)--(5.969,3.859)%
    --(5.985,3.858)--(6.000,3.858)--(6.016,3.858)--(6.032,3.859)--(6.047,3.861)%
    --(6.062,3.864)--(6.077,3.868)--(6.092,3.872)--(6.107,3.877)--(6.121,3.883)%
    --(6.135,3.890)--(6.149,3.897)--(6.162,3.905)--(6.175,3.914)--(6.187,3.924)%
    --(6.199,3.934)--(6.211,3.944)--(6.221,3.956)--(6.231,3.968)--(6.241,3.980)%
    --(6.250,3.993)--(6.258,4.006)--(6.265,4.020)--(6.272,4.034)--(6.278,4.048)%
    --(6.283,4.063)--(6.287,4.078)--(6.291,4.093)--(6.294,4.108)--(6.296,4.123)%
    --(6.297,4.139)--(6.298,4.154)--cycle;
\gpfill{color=gp lt color border,opacity=0.50} (8.731,3.098)--(8.730,3.099)--(8.730,3.100)--(8.730,3.102)%
    --(8.730,3.103)--(8.730,3.104)--(8.729,3.106)--(8.729,3.107)--(8.728,3.108)%
    --(8.728,3.110)--(8.727,3.111)--(8.726,3.112)--(8.725,3.113)--(8.724,3.114)%
    --(8.724,3.116)--(8.723,3.117)--(8.722,3.118)--(8.720,3.118)--(8.719,3.119)%
    --(8.718,3.120)--(8.717,3.121)--(8.716,3.122)--(8.714,3.122)--(8.713,3.123)%
    --(8.712,3.123)--(8.710,3.124)--(8.709,3.124)--(8.708,3.124)--(8.706,3.124)%
    --(8.705,3.124)--(8.704,3.125)--(8.702,3.124)--(8.701,3.124)--(8.699,3.124)%
    --(8.698,3.124)--(8.697,3.124)--(8.695,3.123)--(8.694,3.123)--(8.693,3.122)%
    --(8.691,3.122)--(8.690,3.121)--(8.689,3.120)--(8.688,3.119)--(8.687,3.118)%
    --(8.685,3.118)--(8.684,3.117)--(8.683,3.116)--(8.683,3.114)--(8.682,3.113)%
    --(8.681,3.112)--(8.680,3.111)--(8.679,3.110)--(8.679,3.108)--(8.678,3.107)%
    --(8.678,3.106)--(8.677,3.104)--(8.677,3.103)--(8.677,3.102)--(8.677,3.100)%
    --(8.677,3.099)--(8.677,3.098)--(8.677,3.096)--(8.677,3.095)--(8.677,3.093)%
    --(8.677,3.092)--(8.677,3.091)--(8.678,3.089)--(8.678,3.088)--(8.679,3.087)%
    --(8.679,3.085)--(8.680,3.084)--(8.681,3.083)--(8.682,3.082)--(8.683,3.081)%
    --(8.683,3.079)--(8.684,3.078)--(8.685,3.077)--(8.687,3.077)--(8.688,3.076)%
    --(8.689,3.075)--(8.690,3.074)--(8.691,3.073)--(8.693,3.073)--(8.694,3.072)%
    --(8.695,3.072)--(8.697,3.071)--(8.698,3.071)--(8.699,3.071)--(8.701,3.071)%
    --(8.702,3.071)--(8.703,3.071)--(8.705,3.071)--(8.706,3.071)--(8.708,3.071)%
    --(8.709,3.071)--(8.710,3.071)--(8.712,3.072)--(8.713,3.072)--(8.714,3.073)%
    --(8.716,3.073)--(8.717,3.074)--(8.718,3.075)--(8.719,3.076)--(8.720,3.077)%
    --(8.722,3.077)--(8.723,3.078)--(8.724,3.079)--(8.724,3.081)--(8.725,3.082)%
    --(8.726,3.083)--(8.727,3.084)--(8.728,3.085)--(8.728,3.087)--(8.729,3.088)%
    --(8.729,3.089)--(8.730,3.091)--(8.730,3.092)--(8.730,3.093)--(8.730,3.095)%
    --(8.730,3.096)--(8.731,3.097)--cycle;
\gpfill{color=gp lt color border,opacity=0.50} (2.892,4.155)--(2.891,4.176)--(2.889,4.197)--(2.887,4.218)%
    --(2.883,4.239)--(2.878,4.259)--(2.872,4.280)--(2.865,4.300)--(2.856,4.319)%
    --(2.847,4.338)--(2.837,4.357)--(2.826,4.375)--(2.814,4.393)--(2.801,4.409)%
    --(2.787,4.425)--(2.773,4.441)--(2.757,4.455)--(2.741,4.469)--(2.725,4.482)%
    --(2.707,4.494)--(2.689,4.505)--(2.670,4.515)--(2.651,4.524)--(2.632,4.533)%
    --(2.612,4.540)--(2.591,4.546)--(2.571,4.551)--(2.550,4.555)--(2.529,4.557)%
    --(2.508,4.559)--(2.487,4.560)--(2.465,4.559)--(2.444,4.557)--(2.423,4.555)%
    --(2.402,4.551)--(2.382,4.546)--(2.361,4.540)--(2.341,4.533)--(2.322,4.524)%
    --(2.303,4.515)--(2.284,4.505)--(2.266,4.494)--(2.248,4.482)--(2.232,4.469)%
    --(2.216,4.455)--(2.200,4.441)--(2.186,4.425)--(2.172,4.409)--(2.159,4.393)%
    --(2.147,4.375)--(2.136,4.357)--(2.126,4.338)--(2.117,4.319)--(2.108,4.300)%
    --(2.101,4.280)--(2.095,4.259)--(2.090,4.239)--(2.086,4.218)--(2.084,4.197)%
    --(2.082,4.176)--(2.082,4.155)--(2.082,4.133)--(2.084,4.112)--(2.086,4.091)%
    --(2.090,4.070)--(2.095,4.050)--(2.101,4.029)--(2.108,4.009)--(2.117,3.990)%
    --(2.126,3.971)--(2.136,3.952)--(2.147,3.934)--(2.159,3.916)--(2.172,3.900)%
    --(2.186,3.884)--(2.200,3.868)--(2.216,3.854)--(2.232,3.840)--(2.248,3.827)%
    --(2.266,3.815)--(2.284,3.804)--(2.303,3.794)--(2.322,3.785)--(2.341,3.776)%
    --(2.361,3.769)--(2.382,3.763)--(2.402,3.758)--(2.423,3.754)--(2.444,3.752)%
    --(2.465,3.750)--(2.486,3.750)--(2.508,3.750)--(2.529,3.752)--(2.550,3.754)%
    --(2.571,3.758)--(2.591,3.763)--(2.612,3.769)--(2.632,3.776)--(2.651,3.785)%
    --(2.670,3.794)--(2.689,3.804)--(2.707,3.815)--(2.725,3.827)--(2.741,3.840)%
    --(2.757,3.854)--(2.773,3.868)--(2.787,3.884)--(2.801,3.900)--(2.814,3.916)%
    --(2.826,3.934)--(2.837,3.952)--(2.847,3.971)--(2.856,3.990)--(2.865,4.009)%
    --(2.872,4.029)--(2.878,4.050)--(2.883,4.070)--(2.887,4.091)--(2.889,4.112)%
    --(2.891,4.133)--(2.892,4.154)--cycle;
\gpfill{color=gp lt color border,opacity=0.50} (5.325,3.098)--(5.324,3.105)--(5.324,3.112)--(5.323,3.119)%
    --(5.322,3.126)--(5.320,3.132)--(5.318,3.139)--(5.316,3.146)--(5.313,3.152)%
    --(5.310,3.159)--(5.306,3.165)--(5.303,3.171)--(5.299,3.177)--(5.294,3.182)%
    --(5.290,3.188)--(5.285,3.193)--(5.280,3.198)--(5.274,3.202)--(5.269,3.207)%
    --(5.263,3.211)--(5.257,3.214)--(5.251,3.218)--(5.244,3.221)--(5.238,3.224)%
    --(5.231,3.226)--(5.224,3.228)--(5.218,3.230)--(5.211,3.231)--(5.204,3.232)%
    --(5.197,3.232)--(5.190,3.233)--(5.182,3.232)--(5.175,3.232)--(5.168,3.231)%
    --(5.161,3.230)--(5.155,3.228)--(5.148,3.226)--(5.141,3.224)--(5.135,3.221)%
    --(5.128,3.218)--(5.122,3.214)--(5.116,3.211)--(5.110,3.207)--(5.105,3.202)%
    --(5.099,3.198)--(5.094,3.193)--(5.089,3.188)--(5.085,3.182)--(5.080,3.177)%
    --(5.076,3.171)--(5.073,3.165)--(5.069,3.159)--(5.066,3.152)--(5.063,3.146)%
    --(5.061,3.139)--(5.059,3.132)--(5.057,3.126)--(5.056,3.119)--(5.055,3.112)%
    --(5.055,3.105)--(5.055,3.098)--(5.055,3.090)--(5.055,3.083)--(5.056,3.076)%
    --(5.057,3.069)--(5.059,3.063)--(5.061,3.056)--(5.063,3.049)--(5.066,3.043)%
    --(5.069,3.036)--(5.073,3.030)--(5.076,3.024)--(5.080,3.018)--(5.085,3.013)%
    --(5.089,3.007)--(5.094,3.002)--(5.099,2.997)--(5.105,2.993)--(5.110,2.988)%
    --(5.116,2.984)--(5.122,2.981)--(5.128,2.977)--(5.135,2.974)--(5.141,2.971)%
    --(5.148,2.969)--(5.155,2.967)--(5.161,2.965)--(5.168,2.964)--(5.175,2.963)%
    --(5.182,2.963)--(5.189,2.963)--(5.197,2.963)--(5.204,2.963)--(5.211,2.964)%
    --(5.218,2.965)--(5.224,2.967)--(5.231,2.969)--(5.238,2.971)--(5.244,2.974)%
    --(5.251,2.977)--(5.257,2.981)--(5.263,2.984)--(5.269,2.988)--(5.274,2.993)%
    --(5.280,2.997)--(5.285,3.002)--(5.290,3.007)--(5.294,3.013)--(5.299,3.018)%
    --(5.303,3.024)--(5.306,3.030)--(5.310,3.036)--(5.313,3.043)--(5.316,3.049)%
    --(5.318,3.056)--(5.320,3.063)--(5.322,3.069)--(5.323,3.076)--(5.324,3.083)%
    --(5.324,3.090)--(5.325,3.097)--cycle;
\gpfill{color=gp lt color border,opacity=0.50} (8.028,6.268)--(8.027,6.275)--(8.027,6.282)--(8.026,6.289)%
    --(8.025,6.296)--(8.023,6.302)--(8.021,6.309)--(8.019,6.316)--(8.016,6.322)%
    --(8.013,6.329)--(8.009,6.335)--(8.006,6.341)--(8.002,6.347)--(7.997,6.352)%
    --(7.993,6.358)--(7.988,6.363)--(7.983,6.368)--(7.977,6.372)--(7.972,6.377)%
    --(7.966,6.381)--(7.960,6.384)--(7.954,6.388)--(7.947,6.391)--(7.941,6.394)%
    --(7.934,6.396)--(7.927,6.398)--(7.921,6.400)--(7.914,6.401)--(7.907,6.402)%
    --(7.900,6.402)--(7.893,6.403)--(7.885,6.402)--(7.878,6.402)--(7.871,6.401)%
    --(7.864,6.400)--(7.858,6.398)--(7.851,6.396)--(7.844,6.394)--(7.838,6.391)%
    --(7.831,6.388)--(7.825,6.384)--(7.819,6.381)--(7.813,6.377)--(7.808,6.372)%
    --(7.802,6.368)--(7.797,6.363)--(7.792,6.358)--(7.788,6.352)--(7.783,6.347)%
    --(7.779,6.341)--(7.776,6.335)--(7.772,6.329)--(7.769,6.322)--(7.766,6.316)%
    --(7.764,6.309)--(7.762,6.302)--(7.760,6.296)--(7.759,6.289)--(7.758,6.282)%
    --(7.758,6.275)--(7.758,6.268)--(7.758,6.260)--(7.758,6.253)--(7.759,6.246)%
    --(7.760,6.239)--(7.762,6.233)--(7.764,6.226)--(7.766,6.219)--(7.769,6.213)%
    --(7.772,6.206)--(7.776,6.200)--(7.779,6.194)--(7.783,6.188)--(7.788,6.183)%
    --(7.792,6.177)--(7.797,6.172)--(7.802,6.167)--(7.808,6.163)--(7.813,6.158)%
    --(7.819,6.154)--(7.825,6.151)--(7.831,6.147)--(7.838,6.144)--(7.844,6.141)%
    --(7.851,6.139)--(7.858,6.137)--(7.864,6.135)--(7.871,6.134)--(7.878,6.133)%
    --(7.885,6.133)--(7.892,6.133)--(7.900,6.133)--(7.907,6.133)--(7.914,6.134)%
    --(7.921,6.135)--(7.927,6.137)--(7.934,6.139)--(7.941,6.141)--(7.947,6.144)%
    --(7.954,6.147)--(7.960,6.151)--(7.966,6.154)--(7.972,6.158)--(7.977,6.163)%
    --(7.983,6.167)--(7.988,6.172)--(7.993,6.177)--(7.997,6.183)--(8.002,6.188)%
    --(8.006,6.194)--(8.009,6.200)--(8.013,6.206)--(8.016,6.213)--(8.019,6.219)%
    --(8.021,6.226)--(8.023,6.233)--(8.025,6.239)--(8.026,6.246)--(8.027,6.253)%
    --(8.027,6.260)--(8.028,6.267)--cycle;
\gpfill{color=gp lt color border,opacity=0.50} (6.703,6.268)--(6.702,6.276)--(6.702,6.284)--(6.701,6.293)%
    --(6.699,6.301)--(6.697,6.309)--(6.695,6.318)--(6.692,6.326)--(6.688,6.333)%
    --(6.685,6.341)--(6.681,6.348)--(6.676,6.356)--(6.672,6.363)--(6.666,6.369)%
    --(6.661,6.376)--(6.655,6.382)--(6.649,6.388)--(6.642,6.393)--(6.636,6.399)%
    --(6.629,6.403)--(6.622,6.408)--(6.614,6.412)--(6.606,6.415)--(6.599,6.419)%
    --(6.591,6.422)--(6.582,6.424)--(6.574,6.426)--(6.566,6.428)--(6.557,6.429)%
    --(6.549,6.429)--(6.541,6.430)--(6.532,6.429)--(6.524,6.429)--(6.515,6.428)%
    --(6.507,6.426)--(6.499,6.424)--(6.490,6.422)--(6.482,6.419)--(6.475,6.415)%
    --(6.467,6.412)--(6.460,6.408)--(6.452,6.403)--(6.445,6.399)--(6.439,6.393)%
    --(6.432,6.388)--(6.426,6.382)--(6.420,6.376)--(6.415,6.369)--(6.409,6.363)%
    --(6.405,6.356)--(6.400,6.348)--(6.396,6.341)--(6.393,6.333)--(6.389,6.326)%
    --(6.386,6.318)--(6.384,6.309)--(6.382,6.301)--(6.380,6.293)--(6.379,6.284)%
    --(6.379,6.276)--(6.379,6.268)--(6.379,6.259)--(6.379,6.251)--(6.380,6.242)%
    --(6.382,6.234)--(6.384,6.226)--(6.386,6.217)--(6.389,6.209)--(6.393,6.202)%
    --(6.396,6.194)--(6.400,6.186)--(6.405,6.179)--(6.409,6.172)--(6.415,6.166)%
    --(6.420,6.159)--(6.426,6.153)--(6.432,6.147)--(6.439,6.142)--(6.445,6.136)%
    --(6.452,6.132)--(6.459,6.127)--(6.467,6.123)--(6.475,6.120)--(6.482,6.116)%
    --(6.490,6.113)--(6.499,6.111)--(6.507,6.109)--(6.515,6.107)--(6.524,6.106)%
    --(6.532,6.106)--(6.540,6.106)--(6.549,6.106)--(6.557,6.106)--(6.566,6.107)%
    --(6.574,6.109)--(6.582,6.111)--(6.591,6.113)--(6.599,6.116)--(6.606,6.120)%
    --(6.614,6.123)--(6.622,6.127)--(6.629,6.132)--(6.636,6.136)--(6.642,6.142)%
    --(6.649,6.147)--(6.655,6.153)--(6.661,6.159)--(6.666,6.166)--(6.672,6.172)%
    --(6.676,6.179)--(6.681,6.186)--(6.685,6.194)--(6.688,6.202)--(6.692,6.209)%
    --(6.695,6.217)--(6.697,6.226)--(6.699,6.234)--(6.701,6.242)--(6.702,6.251)%
    --(6.702,6.259)--(6.703,6.267)--cycle;
\gpfill{color=gp lt color border,opacity=0.50} (9.406,5.211)--(9.405,5.219)--(9.405,5.227)--(9.404,5.236)%
    --(9.402,5.244)--(9.400,5.252)--(9.398,5.261)--(9.395,5.269)--(9.391,5.276)%
    --(9.388,5.284)--(9.384,5.291)--(9.379,5.299)--(9.375,5.306)--(9.369,5.312)%
    --(9.364,5.319)--(9.358,5.325)--(9.352,5.331)--(9.345,5.336)--(9.339,5.342)%
    --(9.332,5.346)--(9.325,5.351)--(9.317,5.355)--(9.309,5.358)--(9.302,5.362)%
    --(9.294,5.365)--(9.285,5.367)--(9.277,5.369)--(9.269,5.371)--(9.260,5.372)%
    --(9.252,5.372)--(9.244,5.373)--(9.235,5.372)--(9.227,5.372)--(9.218,5.371)%
    --(9.210,5.369)--(9.202,5.367)--(9.193,5.365)--(9.185,5.362)--(9.178,5.358)%
    --(9.170,5.355)--(9.163,5.351)--(9.155,5.346)--(9.148,5.342)--(9.142,5.336)%
    --(9.135,5.331)--(9.129,5.325)--(9.123,5.319)--(9.118,5.312)--(9.112,5.306)%
    --(9.108,5.299)--(9.103,5.291)--(9.099,5.284)--(9.096,5.276)--(9.092,5.269)%
    --(9.089,5.261)--(9.087,5.252)--(9.085,5.244)--(9.083,5.236)--(9.082,5.227)%
    --(9.082,5.219)--(9.082,5.211)--(9.082,5.202)--(9.082,5.194)--(9.083,5.185)%
    --(9.085,5.177)--(9.087,5.169)--(9.089,5.160)--(9.092,5.152)--(9.096,5.145)%
    --(9.099,5.137)--(9.103,5.129)--(9.108,5.122)--(9.112,5.115)--(9.118,5.109)%
    --(9.123,5.102)--(9.129,5.096)--(9.135,5.090)--(9.142,5.085)--(9.148,5.079)%
    --(9.155,5.075)--(9.162,5.070)--(9.170,5.066)--(9.178,5.063)--(9.185,5.059)%
    --(9.193,5.056)--(9.202,5.054)--(9.210,5.052)--(9.218,5.050)--(9.227,5.049)%
    --(9.235,5.049)--(9.243,5.049)--(9.252,5.049)--(9.260,5.049)--(9.269,5.050)%
    --(9.277,5.052)--(9.285,5.054)--(9.294,5.056)--(9.302,5.059)--(9.309,5.063)%
    --(9.317,5.066)--(9.325,5.070)--(9.332,5.075)--(9.339,5.079)--(9.345,5.085)%
    --(9.352,5.090)--(9.358,5.096)--(9.364,5.102)--(9.369,5.109)--(9.375,5.115)%
    --(9.379,5.122)--(9.384,5.129)--(9.388,5.137)--(9.391,5.145)--(9.395,5.152)%
    --(9.398,5.160)--(9.400,5.169)--(9.402,5.177)--(9.404,5.185)--(9.405,5.194)%
    --(9.405,5.202)--(9.406,5.210)--cycle;
\gpfill{color=gp lt color border,opacity=0.50} (6.163,3.098)--(6.162,3.106)--(6.162,3.114)--(6.161,3.123)%
    --(6.159,3.131)--(6.157,3.139)--(6.155,3.148)--(6.152,3.156)--(6.148,3.163)%
    --(6.145,3.171)--(6.141,3.178)--(6.136,3.186)--(6.132,3.193)--(6.126,3.199)%
    --(6.121,3.206)--(6.115,3.212)--(6.109,3.218)--(6.102,3.223)--(6.096,3.229)%
    --(6.089,3.233)--(6.082,3.238)--(6.074,3.242)--(6.066,3.245)--(6.059,3.249)%
    --(6.051,3.252)--(6.042,3.254)--(6.034,3.256)--(6.026,3.258)--(6.017,3.259)%
    --(6.009,3.259)--(6.001,3.260)--(5.992,3.259)--(5.984,3.259)--(5.975,3.258)%
    --(5.967,3.256)--(5.959,3.254)--(5.950,3.252)--(5.942,3.249)--(5.935,3.245)%
    --(5.927,3.242)--(5.920,3.238)--(5.912,3.233)--(5.905,3.229)--(5.899,3.223)%
    --(5.892,3.218)--(5.886,3.212)--(5.880,3.206)--(5.875,3.199)--(5.869,3.193)%
    --(5.865,3.186)--(5.860,3.178)--(5.856,3.171)--(5.853,3.163)--(5.849,3.156)%
    --(5.846,3.148)--(5.844,3.139)--(5.842,3.131)--(5.840,3.123)--(5.839,3.114)%
    --(5.839,3.106)--(5.839,3.098)--(5.839,3.089)--(5.839,3.081)--(5.840,3.072)%
    --(5.842,3.064)--(5.844,3.056)--(5.846,3.047)--(5.849,3.039)--(5.853,3.032)%
    --(5.856,3.024)--(5.860,3.016)--(5.865,3.009)--(5.869,3.002)--(5.875,2.996)%
    --(5.880,2.989)--(5.886,2.983)--(5.892,2.977)--(5.899,2.972)--(5.905,2.966)%
    --(5.912,2.962)--(5.919,2.957)--(5.927,2.953)--(5.935,2.950)--(5.942,2.946)%
    --(5.950,2.943)--(5.959,2.941)--(5.967,2.939)--(5.975,2.937)--(5.984,2.936)%
    --(5.992,2.936)--(6.000,2.936)--(6.009,2.936)--(6.017,2.936)--(6.026,2.937)%
    --(6.034,2.939)--(6.042,2.941)--(6.051,2.943)--(6.059,2.946)--(6.066,2.950)%
    --(6.074,2.953)--(6.082,2.957)--(6.089,2.962)--(6.096,2.966)--(6.102,2.972)%
    --(6.109,2.977)--(6.115,2.983)--(6.121,2.989)--(6.126,2.996)--(6.132,3.002)%
    --(6.136,3.009)--(6.141,3.016)--(6.145,3.024)--(6.148,3.032)--(6.152,3.039)%
    --(6.155,3.047)--(6.157,3.056)--(6.159,3.064)--(6.161,3.072)--(6.162,3.081)%
    --(6.162,3.089)--(6.163,3.097)--cycle;
\gpfill{color=gp lt color border,opacity=0.50} (2.811,3.098)--(2.810,3.114)--(2.809,3.131)--(2.807,3.148)%
    --(2.803,3.165)--(2.799,3.181)--(2.795,3.198)--(2.789,3.214)--(2.782,3.229)%
    --(2.775,3.245)--(2.767,3.259)--(2.758,3.274)--(2.749,3.288)--(2.738,3.301)%
    --(2.727,3.314)--(2.716,3.327)--(2.703,3.338)--(2.690,3.349)--(2.677,3.360)%
    --(2.663,3.369)--(2.649,3.378)--(2.634,3.386)--(2.618,3.393)--(2.603,3.400)%
    --(2.587,3.406)--(2.570,3.410)--(2.554,3.414)--(2.537,3.418)--(2.520,3.420)%
    --(2.503,3.421)--(2.487,3.422)--(2.470,3.421)--(2.453,3.420)--(2.436,3.418)%
    --(2.419,3.414)--(2.403,3.410)--(2.386,3.406)--(2.370,3.400)--(2.355,3.393)%
    --(2.339,3.386)--(2.325,3.378)--(2.310,3.369)--(2.296,3.360)--(2.283,3.349)%
    --(2.270,3.338)--(2.257,3.327)--(2.246,3.314)--(2.235,3.301)--(2.224,3.288)%
    --(2.215,3.274)--(2.206,3.259)--(2.198,3.245)--(2.191,3.229)--(2.184,3.214)%
    --(2.178,3.198)--(2.174,3.181)--(2.170,3.165)--(2.166,3.148)--(2.164,3.131)%
    --(2.163,3.114)--(2.163,3.098)--(2.163,3.081)--(2.164,3.064)--(2.166,3.047)%
    --(2.170,3.030)--(2.174,3.014)--(2.178,2.997)--(2.184,2.981)--(2.191,2.966)%
    --(2.198,2.950)--(2.206,2.935)--(2.215,2.921)--(2.224,2.907)--(2.235,2.894)%
    --(2.246,2.881)--(2.257,2.868)--(2.270,2.857)--(2.283,2.846)--(2.296,2.835)%
    --(2.310,2.826)--(2.324,2.817)--(2.339,2.809)--(2.355,2.802)--(2.370,2.795)%
    --(2.386,2.789)--(2.403,2.785)--(2.419,2.781)--(2.436,2.777)--(2.453,2.775)%
    --(2.470,2.774)--(2.486,2.774)--(2.503,2.774)--(2.520,2.775)--(2.537,2.777)%
    --(2.554,2.781)--(2.570,2.785)--(2.587,2.789)--(2.603,2.795)--(2.618,2.802)%
    --(2.634,2.809)--(2.649,2.817)--(2.663,2.826)--(2.677,2.835)--(2.690,2.846)%
    --(2.703,2.857)--(2.716,2.868)--(2.727,2.881)--(2.738,2.894)--(2.749,2.907)%
    --(2.758,2.921)--(2.767,2.935)--(2.775,2.950)--(2.782,2.966)--(2.789,2.981)%
    --(2.795,2.997)--(2.799,3.014)--(2.803,3.030)--(2.807,3.047)--(2.809,3.064)%
    --(2.810,3.081)--(2.811,3.097)--cycle;
\gpfill{color=gp lt color border,opacity=0.50} (5.325,6.268)--(5.324,6.275)--(5.324,6.282)--(5.323,6.289)%
    --(5.322,6.296)--(5.320,6.302)--(5.318,6.309)--(5.316,6.316)--(5.313,6.322)%
    --(5.310,6.329)--(5.306,6.335)--(5.303,6.341)--(5.299,6.347)--(5.294,6.352)%
    --(5.290,6.358)--(5.285,6.363)--(5.280,6.368)--(5.274,6.372)--(5.269,6.377)%
    --(5.263,6.381)--(5.257,6.384)--(5.251,6.388)--(5.244,6.391)--(5.238,6.394)%
    --(5.231,6.396)--(5.224,6.398)--(5.218,6.400)--(5.211,6.401)--(5.204,6.402)%
    --(5.197,6.402)--(5.190,6.403)--(5.182,6.402)--(5.175,6.402)--(5.168,6.401)%
    --(5.161,6.400)--(5.155,6.398)--(5.148,6.396)--(5.141,6.394)--(5.135,6.391)%
    --(5.128,6.388)--(5.122,6.384)--(5.116,6.381)--(5.110,6.377)--(5.105,6.372)%
    --(5.099,6.368)--(5.094,6.363)--(5.089,6.358)--(5.085,6.352)--(5.080,6.347)%
    --(5.076,6.341)--(5.073,6.335)--(5.069,6.329)--(5.066,6.322)--(5.063,6.316)%
    --(5.061,6.309)--(5.059,6.302)--(5.057,6.296)--(5.056,6.289)--(5.055,6.282)%
    --(5.055,6.275)--(5.055,6.268)--(5.055,6.260)--(5.055,6.253)--(5.056,6.246)%
    --(5.057,6.239)--(5.059,6.233)--(5.061,6.226)--(5.063,6.219)--(5.066,6.213)%
    --(5.069,6.206)--(5.073,6.200)--(5.076,6.194)--(5.080,6.188)--(5.085,6.183)%
    --(5.089,6.177)--(5.094,6.172)--(5.099,6.167)--(5.105,6.163)--(5.110,6.158)%
    --(5.116,6.154)--(5.122,6.151)--(5.128,6.147)--(5.135,6.144)--(5.141,6.141)%
    --(5.148,6.139)--(5.155,6.137)--(5.161,6.135)--(5.168,6.134)--(5.175,6.133)%
    --(5.182,6.133)--(5.189,6.133)--(5.197,6.133)--(5.204,6.133)--(5.211,6.134)%
    --(5.218,6.135)--(5.224,6.137)--(5.231,6.139)--(5.238,6.141)--(5.244,6.144)%
    --(5.251,6.147)--(5.257,6.151)--(5.263,6.154)--(5.269,6.158)--(5.274,6.163)%
    --(5.280,6.167)--(5.285,6.172)--(5.290,6.177)--(5.294,6.183)--(5.299,6.188)%
    --(5.303,6.194)--(5.306,6.200)--(5.310,6.206)--(5.313,6.213)--(5.316,6.219)%
    --(5.318,6.226)--(5.320,6.233)--(5.322,6.239)--(5.323,6.246)--(5.324,6.253)%
    --(5.324,6.260)--(5.325,6.267)--cycle;
\gpfill{color=gp lt color border,opacity=0.50} (8.055,5.211)--(8.054,5.219)--(8.054,5.227)--(8.053,5.236)%
    --(8.051,5.244)--(8.049,5.252)--(8.047,5.261)--(8.044,5.269)--(8.040,5.276)%
    --(8.037,5.284)--(8.033,5.291)--(8.028,5.299)--(8.024,5.306)--(8.018,5.312)%
    --(8.013,5.319)--(8.007,5.325)--(8.001,5.331)--(7.994,5.336)--(7.988,5.342)%
    --(7.981,5.346)--(7.974,5.351)--(7.966,5.355)--(7.958,5.358)--(7.951,5.362)%
    --(7.943,5.365)--(7.934,5.367)--(7.926,5.369)--(7.918,5.371)--(7.909,5.372)%
    --(7.901,5.372)--(7.893,5.373)--(7.884,5.372)--(7.876,5.372)--(7.867,5.371)%
    --(7.859,5.369)--(7.851,5.367)--(7.842,5.365)--(7.834,5.362)--(7.827,5.358)%
    --(7.819,5.355)--(7.812,5.351)--(7.804,5.346)--(7.797,5.342)--(7.791,5.336)%
    --(7.784,5.331)--(7.778,5.325)--(7.772,5.319)--(7.767,5.312)--(7.761,5.306)%
    --(7.757,5.299)--(7.752,5.291)--(7.748,5.284)--(7.745,5.276)--(7.741,5.269)%
    --(7.738,5.261)--(7.736,5.252)--(7.734,5.244)--(7.732,5.236)--(7.731,5.227)%
    --(7.731,5.219)--(7.731,5.211)--(7.731,5.202)--(7.731,5.194)--(7.732,5.185)%
    --(7.734,5.177)--(7.736,5.169)--(7.738,5.160)--(7.741,5.152)--(7.745,5.145)%
    --(7.748,5.137)--(7.752,5.129)--(7.757,5.122)--(7.761,5.115)--(7.767,5.109)%
    --(7.772,5.102)--(7.778,5.096)--(7.784,5.090)--(7.791,5.085)--(7.797,5.079)%
    --(7.804,5.075)--(7.811,5.070)--(7.819,5.066)--(7.827,5.063)--(7.834,5.059)%
    --(7.842,5.056)--(7.851,5.054)--(7.859,5.052)--(7.867,5.050)--(7.876,5.049)%
    --(7.884,5.049)--(7.892,5.049)--(7.901,5.049)--(7.909,5.049)--(7.918,5.050)%
    --(7.926,5.052)--(7.934,5.054)--(7.943,5.056)--(7.951,5.059)--(7.958,5.063)%
    --(7.966,5.066)--(7.974,5.070)--(7.981,5.075)--(7.988,5.079)--(7.994,5.085)%
    --(8.001,5.090)--(8.007,5.096)--(8.013,5.102)--(8.018,5.109)--(8.024,5.115)%
    --(8.028,5.122)--(8.033,5.129)--(8.037,5.137)--(8.040,5.145)--(8.044,5.152)%
    --(8.047,5.160)--(8.049,5.169)--(8.051,5.177)--(8.053,5.185)--(8.054,5.194)%
    --(8.054,5.202)--(8.055,5.210)--cycle;
\gpfill{color=gp lt color border,opacity=0.50} (3.649,4.155)--(3.648,4.173)--(3.647,4.191)--(3.644,4.209)%
    --(3.641,4.227)--(3.637,4.245)--(3.631,4.263)--(3.625,4.280)--(3.618,4.297)%
    --(3.610,4.314)--(3.601,4.330)--(3.592,4.346)--(3.581,4.361)--(3.570,4.375)%
    --(3.558,4.389)--(3.546,4.403)--(3.532,4.415)--(3.518,4.427)--(3.504,4.438)%
    --(3.489,4.449)--(3.473,4.458)--(3.457,4.467)--(3.440,4.475)--(3.423,4.482)%
    --(3.406,4.488)--(3.388,4.494)--(3.370,4.498)--(3.352,4.501)--(3.334,4.504)%
    --(3.316,4.505)--(3.298,4.506)--(3.279,4.505)--(3.261,4.504)--(3.243,4.501)%
    --(3.225,4.498)--(3.207,4.494)--(3.189,4.488)--(3.172,4.482)--(3.155,4.475)%
    --(3.138,4.467)--(3.122,4.458)--(3.106,4.449)--(3.091,4.438)--(3.077,4.427)%
    --(3.063,4.415)--(3.049,4.403)--(3.037,4.389)--(3.025,4.375)--(3.014,4.361)%
    --(3.003,4.346)--(2.994,4.330)--(2.985,4.314)--(2.977,4.297)--(2.970,4.280)%
    --(2.964,4.263)--(2.958,4.245)--(2.954,4.227)--(2.951,4.209)--(2.948,4.191)%
    --(2.947,4.173)--(2.947,4.155)--(2.947,4.136)--(2.948,4.118)--(2.951,4.100)%
    --(2.954,4.082)--(2.958,4.064)--(2.964,4.046)--(2.970,4.029)--(2.977,4.012)%
    --(2.985,3.995)--(2.994,3.979)--(3.003,3.963)--(3.014,3.948)--(3.025,3.934)%
    --(3.037,3.920)--(3.049,3.906)--(3.063,3.894)--(3.077,3.882)--(3.091,3.871)%
    --(3.106,3.860)--(3.122,3.851)--(3.138,3.842)--(3.155,3.834)--(3.172,3.827)%
    --(3.189,3.821)--(3.207,3.815)--(3.225,3.811)--(3.243,3.808)--(3.261,3.805)%
    --(3.279,3.804)--(3.297,3.804)--(3.316,3.804)--(3.334,3.805)--(3.352,3.808)%
    --(3.370,3.811)--(3.388,3.815)--(3.406,3.821)--(3.423,3.827)--(3.440,3.834)%
    --(3.457,3.842)--(3.473,3.851)--(3.489,3.860)--(3.504,3.871)--(3.518,3.882)%
    --(3.532,3.894)--(3.546,3.906)--(3.558,3.920)--(3.570,3.934)--(3.581,3.948)%
    --(3.592,3.963)--(3.601,3.979)--(3.610,3.995)--(3.618,4.012)--(3.625,4.029)%
    --(3.631,4.046)--(3.637,4.064)--(3.641,4.082)--(3.644,4.100)--(3.647,4.118)%
    --(3.648,4.136)--(3.649,4.154)--cycle;
\gpfill{color=gp lt color border,opacity=0.50} (4.028,6.268)--(4.027,6.277)--(4.026,6.287)--(4.025,6.297)%
    --(4.023,6.307)--(4.021,6.316)--(4.018,6.326)--(4.015,6.335)--(4.011,6.344)%
    --(4.007,6.353)--(4.002,6.362)--(3.997,6.370)--(3.991,6.379)--(3.985,6.386)%
    --(3.979,6.394)--(3.972,6.401)--(3.965,6.408)--(3.957,6.414)--(3.950,6.420)%
    --(3.941,6.426)--(3.933,6.431)--(3.924,6.436)--(3.915,6.440)--(3.906,6.444)%
    --(3.897,6.447)--(3.887,6.450)--(3.878,6.452)--(3.868,6.454)--(3.858,6.455)%
    --(3.848,6.456)--(3.839,6.457)--(3.829,6.456)--(3.819,6.455)--(3.809,6.454)%
    --(3.799,6.452)--(3.790,6.450)--(3.780,6.447)--(3.771,6.444)--(3.762,6.440)%
    --(3.753,6.436)--(3.744,6.431)--(3.736,6.426)--(3.727,6.420)--(3.720,6.414)%
    --(3.712,6.408)--(3.705,6.401)--(3.698,6.394)--(3.692,6.386)--(3.686,6.379)%
    --(3.680,6.370)--(3.675,6.362)--(3.670,6.353)--(3.666,6.344)--(3.662,6.335)%
    --(3.659,6.326)--(3.656,6.316)--(3.654,6.307)--(3.652,6.297)--(3.651,6.287)%
    --(3.650,6.277)--(3.650,6.268)--(3.650,6.258)--(3.651,6.248)--(3.652,6.238)%
    --(3.654,6.228)--(3.656,6.219)--(3.659,6.209)--(3.662,6.200)--(3.666,6.191)%
    --(3.670,6.182)--(3.675,6.173)--(3.680,6.165)--(3.686,6.156)--(3.692,6.149)%
    --(3.698,6.141)--(3.705,6.134)--(3.712,6.127)--(3.720,6.121)--(3.727,6.115)%
    --(3.736,6.109)--(3.744,6.104)--(3.753,6.099)--(3.762,6.095)--(3.771,6.091)%
    --(3.780,6.088)--(3.790,6.085)--(3.799,6.083)--(3.809,6.081)--(3.819,6.080)%
    --(3.829,6.079)--(3.838,6.079)--(3.848,6.079)--(3.858,6.080)--(3.868,6.081)%
    --(3.878,6.083)--(3.887,6.085)--(3.897,6.088)--(3.906,6.091)--(3.915,6.095)%
    --(3.924,6.099)--(3.933,6.104)--(3.941,6.109)--(3.950,6.115)--(3.957,6.121)%
    --(3.965,6.127)--(3.972,6.134)--(3.979,6.141)--(3.985,6.149)--(3.991,6.156)%
    --(3.997,6.165)--(4.002,6.173)--(4.007,6.182)--(4.011,6.191)--(4.015,6.200)%
    --(4.018,6.209)--(4.021,6.219)--(4.023,6.228)--(4.025,6.238)--(4.026,6.248)%
    --(4.027,6.258)--(4.028,6.267)--cycle;
\gpfill{color=gp lt color border,opacity=0.50} (6.784,5.211)--(6.783,5.223)--(6.782,5.236)--(6.781,5.249)%
    --(6.778,5.261)--(6.775,5.273)--(6.772,5.286)--(6.767,5.298)--(6.762,5.309)%
    --(6.757,5.321)--(6.751,5.332)--(6.744,5.343)--(6.737,5.353)--(6.729,5.363)%
    --(6.721,5.373)--(6.712,5.382)--(6.703,5.391)--(6.693,5.399)--(6.683,5.407)%
    --(6.673,5.414)--(6.662,5.421)--(6.651,5.427)--(6.639,5.432)--(6.628,5.437)%
    --(6.616,5.442)--(6.603,5.445)--(6.591,5.448)--(6.579,5.451)--(6.566,5.452)%
    --(6.553,5.453)--(6.541,5.454)--(6.528,5.453)--(6.515,5.452)--(6.502,5.451)%
    --(6.490,5.448)--(6.478,5.445)--(6.465,5.442)--(6.453,5.437)--(6.442,5.432)%
    --(6.430,5.427)--(6.419,5.421)--(6.408,5.414)--(6.398,5.407)--(6.388,5.399)%
    --(6.378,5.391)--(6.369,5.382)--(6.360,5.373)--(6.352,5.363)--(6.344,5.353)%
    --(6.337,5.343)--(6.330,5.332)--(6.324,5.321)--(6.319,5.309)--(6.314,5.298)%
    --(6.309,5.286)--(6.306,5.273)--(6.303,5.261)--(6.300,5.249)--(6.299,5.236)%
    --(6.298,5.223)--(6.298,5.211)--(6.298,5.198)--(6.299,5.185)--(6.300,5.172)%
    --(6.303,5.160)--(6.306,5.148)--(6.309,5.135)--(6.314,5.123)--(6.319,5.112)%
    --(6.324,5.100)--(6.330,5.089)--(6.337,5.078)--(6.344,5.068)--(6.352,5.058)%
    --(6.360,5.048)--(6.369,5.039)--(6.378,5.030)--(6.388,5.022)--(6.398,5.014)%
    --(6.408,5.007)--(6.419,5.000)--(6.430,4.994)--(6.442,4.989)--(6.453,4.984)%
    --(6.465,4.979)--(6.478,4.976)--(6.490,4.973)--(6.502,4.970)--(6.515,4.969)%
    --(6.528,4.968)--(6.540,4.968)--(6.553,4.968)--(6.566,4.969)--(6.579,4.970)%
    --(6.591,4.973)--(6.603,4.976)--(6.616,4.979)--(6.628,4.984)--(6.639,4.989)%
    --(6.651,4.994)--(6.662,5.000)--(6.673,5.007)--(6.683,5.014)--(6.693,5.022)%
    --(6.703,5.030)--(6.712,5.039)--(6.721,5.048)--(6.729,5.058)--(6.737,5.068)%
    --(6.744,5.078)--(6.751,5.089)--(6.757,5.100)--(6.762,5.112)--(6.767,5.123)%
    --(6.772,5.135)--(6.775,5.148)--(6.778,5.160)--(6.781,5.172)--(6.782,5.185)%
    --(6.783,5.198)--(6.784,5.210)--cycle;
\draw[gp path] (1.136,8.381)--(1.136,0.985)--(11.947,0.985)--(11.947,8.381)--cycle;
%% coordinates of the plot area
\gpdefrectangularnode{gp plot 1}{\pgfpoint{1.136cm}{0.985cm}}{\pgfpoint{11.947cm}{8.381cm}}
\end{tikzpicture}
%% gnuplot variables

      \end{myplot}

      \begin{myplot}%
        {Зависимость числа операций объединения множеств от размера метода}%
        {plot:merge_ops}
        \begin{tikzpicture}[gnuplot]
%% generated with GNUPLOT 4.5p0 (Lua 5.1; terminal rev. 99, script rev. 98)
%% 27.05.2011 12:41:08
\path (0.000,0.000) rectangle (12.500,8.750);
\gpcolor{color=gp lt color border}
\gpsetlinetype{gp lt border}
\gpsetlinewidth{1.00}
\draw[gp path] (1.320,0.985)--(1.500,0.985);
\draw[gp path] (11.947,0.985)--(11.767,0.985);
\node[gp node right] at (1.136,0.985) {$10^{0}$};
\draw[gp path] (1.320,1.555)--(1.410,1.555);
\draw[gp path] (11.947,1.555)--(11.857,1.555);
\draw[gp path] (1.320,1.889)--(1.410,1.889);
\draw[gp path] (11.947,1.889)--(11.857,1.889);
\draw[gp path] (1.320,2.126)--(1.410,2.126);
\draw[gp path] (11.947,2.126)--(11.857,2.126);
\draw[gp path] (1.320,2.309)--(1.410,2.309);
\draw[gp path] (11.947,2.309)--(11.857,2.309);
\draw[gp path] (1.320,2.460)--(1.410,2.460);
\draw[gp path] (11.947,2.460)--(11.857,2.460);
\draw[gp path] (1.320,2.586)--(1.410,2.586);
\draw[gp path] (11.947,2.586)--(11.857,2.586);
\draw[gp path] (1.320,2.696)--(1.410,2.696);
\draw[gp path] (11.947,2.696)--(11.857,2.696);
\draw[gp path] (1.320,2.793)--(1.410,2.793);
\draw[gp path] (11.947,2.793)--(11.857,2.793);
\draw[gp path] (1.320,2.880)--(1.500,2.880);
\draw[gp path] (11.947,2.880)--(11.767,2.880);
\node[gp node right] at (1.136,2.880) {$10^{1}$};
\draw[gp path] (1.320,3.450)--(1.410,3.450);
\draw[gp path] (11.947,3.450)--(11.857,3.450);
\draw[gp path] (1.320,3.784)--(1.410,3.784);
\draw[gp path] (11.947,3.784)--(11.857,3.784);
\draw[gp path] (1.320,4.021)--(1.410,4.021);
\draw[gp path] (11.947,4.021)--(11.857,4.021);
\draw[gp path] (1.320,4.204)--(1.410,4.204);
\draw[gp path] (11.947,4.204)--(11.857,4.204);
\draw[gp path] (1.320,4.354)--(1.410,4.354);
\draw[gp path] (11.947,4.354)--(11.857,4.354);
\draw[gp path] (1.320,4.481)--(1.410,4.481);
\draw[gp path] (11.947,4.481)--(11.857,4.481);
\draw[gp path] (1.320,4.591)--(1.410,4.591);
\draw[gp path] (11.947,4.591)--(11.857,4.591);
\draw[gp path] (1.320,4.688)--(1.410,4.688);
\draw[gp path] (11.947,4.688)--(11.857,4.688);
\draw[gp path] (1.320,4.775)--(1.500,4.775);
\draw[gp path] (11.947,4.775)--(11.767,4.775);
\node[gp node right] at (1.136,4.775) {$10^{2}$};
\draw[gp path] (1.320,5.345)--(1.410,5.345);
\draw[gp path] (11.947,5.345)--(11.857,5.345);
\draw[gp path] (1.320,5.679)--(1.410,5.679);
\draw[gp path] (11.947,5.679)--(11.857,5.679);
\draw[gp path] (1.320,5.916)--(1.410,5.916);
\draw[gp path] (11.947,5.916)--(11.857,5.916);
\draw[gp path] (1.320,6.099)--(1.410,6.099);
\draw[gp path] (11.947,6.099)--(11.857,6.099);
\draw[gp path] (1.320,6.249)--(1.410,6.249);
\draw[gp path] (11.947,6.249)--(11.857,6.249);
\draw[gp path] (1.320,6.376)--(1.410,6.376);
\draw[gp path] (11.947,6.376)--(11.857,6.376);
\draw[gp path] (1.320,6.486)--(1.410,6.486);
\draw[gp path] (11.947,6.486)--(11.857,6.486);
\draw[gp path] (1.320,6.583)--(1.410,6.583);
\draw[gp path] (11.947,6.583)--(11.857,6.583);
\draw[gp path] (1.320,6.670)--(1.500,6.670);
\draw[gp path] (11.947,6.670)--(11.767,6.670);
\node[gp node right] at (1.136,6.670) {$10^{3}$};
\draw[gp path] (1.320,7.240)--(1.410,7.240);
\draw[gp path] (11.947,7.240)--(11.857,7.240);
\draw[gp path] (1.320,7.574)--(1.410,7.574);
\draw[gp path] (11.947,7.574)--(11.857,7.574);
\draw[gp path] (1.320,7.811)--(1.410,7.811);
\draw[gp path] (11.947,7.811)--(11.857,7.811);
\draw[gp path] (1.320,7.994)--(1.410,7.994);
\draw[gp path] (11.947,7.994)--(11.857,7.994);
\draw[gp path] (1.320,8.144)--(1.410,8.144);
\draw[gp path] (11.947,8.144)--(11.857,8.144);
\draw[gp path] (1.320,8.271)--(1.410,8.271);
\draw[gp path] (11.947,8.271)--(11.857,8.271);
\draw[gp path] (1.320,8.381)--(1.410,8.381);
\draw[gp path] (11.947,8.381)--(11.857,8.381);
\draw[gp path] (1.320,0.985)--(1.320,1.165);
\draw[gp path] (1.320,8.381)--(1.320,8.201);
\node[gp node center] at (1.320,0.677) {$10^{0}$};
\draw[gp path] (2.140,0.985)--(2.140,1.075);
\draw[gp path] (2.140,8.381)--(2.140,8.291);
\draw[gp path] (2.619,0.985)--(2.619,1.075);
\draw[gp path] (2.619,8.381)--(2.619,8.291);
\draw[gp path] (2.959,0.985)--(2.959,1.075);
\draw[gp path] (2.959,8.381)--(2.959,8.291);
\draw[gp path] (3.223,0.985)--(3.223,1.075);
\draw[gp path] (3.223,8.381)--(3.223,8.291);
\draw[gp path] (3.439,0.985)--(3.439,1.075);
\draw[gp path] (3.439,8.381)--(3.439,8.291);
\draw[gp path] (3.621,0.985)--(3.621,1.075);
\draw[gp path] (3.621,8.381)--(3.621,8.291);
\draw[gp path] (3.779,0.985)--(3.779,1.075);
\draw[gp path] (3.779,8.381)--(3.779,8.291);
\draw[gp path] (3.918,0.985)--(3.918,1.075);
\draw[gp path] (3.918,8.381)--(3.918,8.291);
\draw[gp path] (4.043,0.985)--(4.043,1.165);
\draw[gp path] (4.043,8.381)--(4.043,8.201);
\node[gp node center] at (4.043,0.677) {$10^{1}$};
\draw[gp path] (4.862,0.985)--(4.862,1.075);
\draw[gp path] (4.862,8.381)--(4.862,8.291);
\draw[gp path] (5.342,0.985)--(5.342,1.075);
\draw[gp path] (5.342,8.381)--(5.342,8.291);
\draw[gp path] (5.682,0.985)--(5.682,1.075);
\draw[gp path] (5.682,8.381)--(5.682,8.291);
\draw[gp path] (5.946,0.985)--(5.946,1.075);
\draw[gp path] (5.946,8.381)--(5.946,8.291);
\draw[gp path] (6.161,0.985)--(6.161,1.075);
\draw[gp path] (6.161,8.381)--(6.161,8.291);
\draw[gp path] (6.344,0.985)--(6.344,1.075);
\draw[gp path] (6.344,8.381)--(6.344,8.291);
\draw[gp path] (6.502,0.985)--(6.502,1.075);
\draw[gp path] (6.502,8.381)--(6.502,8.291);
\draw[gp path] (6.641,0.985)--(6.641,1.075);
\draw[gp path] (6.641,8.381)--(6.641,8.291);
\draw[gp path] (6.765,0.985)--(6.765,1.165);
\draw[gp path] (6.765,8.381)--(6.765,8.201);
\node[gp node center] at (6.765,0.677) {$10^{2}$};
\draw[gp path] (7.585,0.985)--(7.585,1.075);
\draw[gp path] (7.585,8.381)--(7.585,8.291);
\draw[gp path] (8.064,0.985)--(8.064,1.075);
\draw[gp path] (8.064,8.381)--(8.064,8.291);
\draw[gp path] (8.405,0.985)--(8.405,1.075);
\draw[gp path] (8.405,8.381)--(8.405,8.291);
\draw[gp path] (8.669,0.985)--(8.669,1.075);
\draw[gp path] (8.669,8.381)--(8.669,8.291);
\draw[gp path] (8.884,0.985)--(8.884,1.075);
\draw[gp path] (8.884,8.381)--(8.884,8.291);
\draw[gp path] (9.066,0.985)--(9.066,1.075);
\draw[gp path] (9.066,8.381)--(9.066,8.291);
\draw[gp path] (9.224,0.985)--(9.224,1.075);
\draw[gp path] (9.224,8.381)--(9.224,8.291);
\draw[gp path] (9.364,0.985)--(9.364,1.075);
\draw[gp path] (9.364,8.381)--(9.364,8.291);
\draw[gp path] (9.488,0.985)--(9.488,1.165);
\draw[gp path] (9.488,8.381)--(9.488,8.201);
\node[gp node center] at (9.488,0.677) {$10^{3}$};
\draw[gp path] (10.308,0.985)--(10.308,1.075);
\draw[gp path] (10.308,8.381)--(10.308,8.291);
\draw[gp path] (10.787,0.985)--(10.787,1.075);
\draw[gp path] (10.787,8.381)--(10.787,8.291);
\draw[gp path] (11.127,0.985)--(11.127,1.075);
\draw[gp path] (11.127,8.381)--(11.127,8.291);
\draw[gp path] (11.391,0.985)--(11.391,1.075);
\draw[gp path] (11.391,8.381)--(11.391,8.291);
\draw[gp path] (11.607,0.985)--(11.607,1.075);
\draw[gp path] (11.607,8.381)--(11.607,8.291);
\draw[gp path] (11.789,0.985)--(11.789,1.075);
\draw[gp path] (11.789,8.381)--(11.789,8.291);
\draw[gp path] (11.947,0.985)--(11.947,1.075);
\draw[gp path] (11.947,8.381)--(11.947,8.291);
\draw[gp path] (1.320,8.381)--(1.320,0.985)--(11.947,0.985)--(11.947,8.381)--cycle;
\node[gp node center,rotate=-270] at (0.246,4.683) {Количество операций};
\node[gp node center] at (6.633,0.215) {Количество присваиваний};
\gpsetlinewidth{2.00}
\gpsetpointsize{4.00}
\gppoint{gp mark 0}{(3.223,2.586)}
\gppoint{gp mark 0}{(3.223,2.126)}
\gppoint{gp mark 0}{(3.223,2.126)}
\gppoint{gp mark 0}{(3.223,2.586)}
\gppoint{gp mark 0}{(3.223,2.460)}
\gppoint{gp mark 0}{(3.223,2.126)}
\gppoint{gp mark 0}{(3.223,2.586)}
\gppoint{gp mark 0}{(3.223,2.126)}
\gppoint{gp mark 0}{(3.223,2.586)}
\gppoint{gp mark 0}{(3.223,2.460)}
\gppoint{gp mark 0}{(3.223,2.460)}
\gppoint{gp mark 0}{(3.223,2.586)}
\gppoint{gp mark 0}{(3.223,2.586)}
\gppoint{gp mark 0}{(3.223,2.460)}
\gppoint{gp mark 0}{(3.223,2.586)}
\gppoint{gp mark 0}{(3.223,2.586)}
\gppoint{gp mark 0}{(3.223,2.586)}
\gppoint{gp mark 0}{(3.223,2.586)}
\gppoint{gp mark 0}{(3.223,2.126)}
\gppoint{gp mark 0}{(3.223,2.586)}
\gppoint{gp mark 0}{(3.223,2.126)}
\gppoint{gp mark 0}{(3.223,2.586)}
\gppoint{gp mark 0}{(3.223,2.126)}
\gppoint{gp mark 0}{(3.223,2.126)}
\gppoint{gp mark 0}{(3.223,2.586)}
\gppoint{gp mark 0}{(3.223,2.126)}
\gppoint{gp mark 0}{(3.223,2.126)}
\gppoint{gp mark 0}{(3.439,2.793)}
\gppoint{gp mark 0}{(3.439,2.793)}
\gppoint{gp mark 0}{(3.439,2.696)}
\gppoint{gp mark 0}{(3.439,2.880)}
\gppoint{gp mark 0}{(3.439,2.793)}
\gppoint{gp mark 0}{(3.439,2.793)}
\gppoint{gp mark 0}{(3.439,2.126)}
\gppoint{gp mark 0}{(3.439,2.460)}
\gppoint{gp mark 0}{(3.439,2.460)}
\gppoint{gp mark 0}{(3.439,2.793)}
\gppoint{gp mark 0}{(3.439,2.309)}
\gppoint{gp mark 0}{(3.439,2.793)}
\gppoint{gp mark 0}{(3.439,2.460)}
\gppoint{gp mark 0}{(3.439,2.793)}
\gppoint{gp mark 0}{(3.439,2.460)}
\gppoint{gp mark 0}{(3.439,2.126)}
\gppoint{gp mark 0}{(3.439,2.696)}
\gppoint{gp mark 0}{(3.439,2.309)}
\gppoint{gp mark 0}{(3.439,2.793)}
\gppoint{gp mark 0}{(3.439,2.696)}
\gppoint{gp mark 0}{(3.439,2.460)}
\gppoint{gp mark 0}{(3.439,2.460)}
\gppoint{gp mark 0}{(3.439,2.460)}
\gppoint{gp mark 0}{(3.439,2.793)}
\gppoint{gp mark 0}{(3.439,2.460)}
\gppoint{gp mark 0}{(3.439,2.460)}
\gppoint{gp mark 0}{(3.439,2.460)}
\gppoint{gp mark 0}{(3.439,2.460)}
\gppoint{gp mark 0}{(3.439,2.126)}
\gppoint{gp mark 0}{(3.621,2.586)}
\gppoint{gp mark 0}{(3.621,2.309)}
\gppoint{gp mark 0}{(3.621,2.880)}
\gppoint{gp mark 0}{(3.621,2.696)}
\gppoint{gp mark 0}{(3.621,2.696)}
\gppoint{gp mark 0}{(3.621,2.309)}
\gppoint{gp mark 0}{(3.621,2.880)}
\gppoint{gp mark 0}{(3.621,2.309)}
\gppoint{gp mark 0}{(3.621,2.880)}
\gppoint{gp mark 0}{(3.621,2.793)}
\gppoint{gp mark 0}{(3.621,2.460)}
\gppoint{gp mark 0}{(3.621,2.309)}
\gppoint{gp mark 0}{(3.621,2.309)}
\gppoint{gp mark 0}{(3.621,2.696)}
\gppoint{gp mark 0}{(3.621,2.696)}
\gppoint{gp mark 0}{(3.621,2.309)}
\gppoint{gp mark 0}{(3.621,2.696)}
\gppoint{gp mark 0}{(3.621,2.586)}
\gppoint{gp mark 0}{(3.621,2.309)}
\gppoint{gp mark 0}{(3.621,2.880)}
\gppoint{gp mark 0}{(3.621,2.696)}
\gppoint{gp mark 0}{(3.621,2.586)}
\gppoint{gp mark 0}{(3.621,2.696)}
\gppoint{gp mark 0}{(3.621,2.696)}
\gppoint{gp mark 0}{(3.621,2.696)}
\gppoint{gp mark 0}{(3.621,2.309)}
\gppoint{gp mark 0}{(3.621,2.309)}
\gppoint{gp mark 0}{(3.621,3.157)}
\gppoint{gp mark 0}{(3.621,2.309)}
\gppoint{gp mark 0}{(3.621,2.586)}
\gppoint{gp mark 0}{(3.621,2.309)}
\gppoint{gp mark 0}{(3.621,2.958)}
\gppoint{gp mark 0}{(3.621,2.309)}
\gppoint{gp mark 0}{(3.621,2.309)}
\gppoint{gp mark 0}{(3.621,2.696)}
\gppoint{gp mark 0}{(3.621,3.030)}
\gppoint{gp mark 0}{(3.621,2.309)}
\gppoint{gp mark 0}{(3.621,2.309)}
\gppoint{gp mark 0}{(3.621,2.586)}
\gppoint{gp mark 0}{(3.621,2.309)}
\gppoint{gp mark 0}{(3.621,2.793)}
\gppoint{gp mark 0}{(3.621,3.157)}
\gppoint{gp mark 0}{(3.621,3.030)}
\gppoint{gp mark 0}{(3.621,2.309)}
\gppoint{gp mark 0}{(3.621,3.030)}
\gppoint{gp mark 0}{(3.621,3.096)}
\gppoint{gp mark 0}{(3.621,2.309)}
\gppoint{gp mark 0}{(3.621,2.309)}
\gppoint{gp mark 0}{(3.621,2.958)}
\gppoint{gp mark 0}{(3.621,2.880)}
\gppoint{gp mark 0}{(3.621,3.157)}
\gppoint{gp mark 0}{(3.621,3.096)}
\gppoint{gp mark 0}{(3.621,2.309)}
\gppoint{gp mark 0}{(3.621,2.309)}
\gppoint{gp mark 0}{(3.621,2.309)}
\gppoint{gp mark 0}{(3.621,2.880)}
\gppoint{gp mark 0}{(3.621,2.309)}
\gppoint{gp mark 0}{(3.621,3.030)}
\gppoint{gp mark 0}{(3.621,3.030)}
\gppoint{gp mark 0}{(3.621,2.309)}
\gppoint{gp mark 0}{(3.621,3.214)}
\gppoint{gp mark 0}{(3.621,3.030)}
\gppoint{gp mark 0}{(3.621,2.309)}
\gppoint{gp mark 0}{(3.621,2.309)}
\gppoint{gp mark 0}{(3.621,2.309)}
\gppoint{gp mark 0}{(3.621,2.696)}
\gppoint{gp mark 0}{(3.621,2.126)}
\gppoint{gp mark 0}{(3.621,2.696)}
\gppoint{gp mark 0}{(3.621,3.157)}
\gppoint{gp mark 0}{(3.621,2.309)}
\gppoint{gp mark 0}{(3.621,2.309)}
\gppoint{gp mark 0}{(3.621,2.696)}
\gppoint{gp mark 0}{(3.621,2.696)}
\gppoint{gp mark 0}{(3.621,2.309)}
\gppoint{gp mark 0}{(3.621,2.696)}
\gppoint{gp mark 0}{(3.621,2.309)}
\gppoint{gp mark 0}{(3.621,2.696)}
\gppoint{gp mark 0}{(3.621,2.309)}
\gppoint{gp mark 0}{(3.621,3.030)}
\gppoint{gp mark 0}{(3.621,2.309)}
\gppoint{gp mark 0}{(3.621,2.696)}
\gppoint{gp mark 0}{(3.621,2.309)}
\gppoint{gp mark 0}{(3.621,2.696)}
\gppoint{gp mark 0}{(3.621,2.696)}
\gppoint{gp mark 0}{(3.621,2.793)}
\gppoint{gp mark 0}{(3.621,3.157)}
\gppoint{gp mark 0}{(3.621,2.309)}
\gppoint{gp mark 0}{(3.621,2.958)}
\gppoint{gp mark 0}{(3.621,2.696)}
\gppoint{gp mark 0}{(3.621,2.309)}
\gppoint{gp mark 0}{(3.621,2.309)}
\gppoint{gp mark 0}{(3.621,3.096)}
\gppoint{gp mark 0}{(3.621,2.793)}
\gppoint{gp mark 0}{(3.621,2.793)}
\gppoint{gp mark 0}{(3.621,2.586)}
\gppoint{gp mark 0}{(3.621,2.793)}
\gppoint{gp mark 0}{(3.621,3.364)}
\gppoint{gp mark 0}{(3.621,3.157)}
\gppoint{gp mark 0}{(3.621,2.696)}
\gppoint{gp mark 0}{(3.621,3.157)}
\gppoint{gp mark 0}{(3.621,3.214)}
\gppoint{gp mark 0}{(3.621,2.793)}
\gppoint{gp mark 0}{(3.621,2.696)}
\gppoint{gp mark 0}{(3.621,2.309)}
\gppoint{gp mark 0}{(3.621,2.793)}
\gppoint{gp mark 0}{(3.621,3.096)}
\gppoint{gp mark 0}{(3.621,2.309)}
\gppoint{gp mark 0}{(3.621,3.096)}
\gppoint{gp mark 0}{(3.621,3.214)}
\gppoint{gp mark 0}{(3.621,2.309)}
\gppoint{gp mark 0}{(3.621,2.696)}
\gppoint{gp mark 0}{(3.621,2.793)}
\gppoint{gp mark 0}{(3.621,2.309)}
\gppoint{gp mark 0}{(3.621,2.958)}
\gppoint{gp mark 0}{(3.621,2.309)}
\gppoint{gp mark 0}{(3.621,2.460)}
\gppoint{gp mark 0}{(3.621,2.793)}
\gppoint{gp mark 0}{(3.621,2.880)}
\gppoint{gp mark 0}{(3.621,3.030)}
\gppoint{gp mark 0}{(3.621,2.793)}
\gppoint{gp mark 0}{(3.621,2.793)}
\gppoint{gp mark 0}{(3.621,2.309)}
\gppoint{gp mark 0}{(3.621,2.309)}
\gppoint{gp mark 0}{(3.621,2.696)}
\gppoint{gp mark 0}{(3.621,2.793)}
\gppoint{gp mark 0}{(3.621,2.793)}
\gppoint{gp mark 0}{(3.621,2.309)}
\gppoint{gp mark 0}{(3.621,2.793)}
\gppoint{gp mark 0}{(3.621,2.793)}
\gppoint{gp mark 0}{(3.621,2.793)}
\gppoint{gp mark 0}{(3.621,2.793)}
\gppoint{gp mark 0}{(3.621,2.958)}
\gppoint{gp mark 0}{(3.621,2.880)}
\gppoint{gp mark 0}{(3.621,2.309)}
\gppoint{gp mark 0}{(3.621,3.267)}
\gppoint{gp mark 0}{(3.621,2.793)}
\gppoint{gp mark 0}{(3.621,2.460)}
\gppoint{gp mark 0}{(3.621,2.696)}
\gppoint{gp mark 0}{(3.621,2.309)}
\gppoint{gp mark 0}{(3.621,2.309)}
\gppoint{gp mark 0}{(3.621,2.696)}
\gppoint{gp mark 0}{(3.621,2.696)}
\gppoint{gp mark 0}{(3.621,3.267)}
\gppoint{gp mark 0}{(3.621,3.267)}
\gppoint{gp mark 0}{(3.621,2.793)}
\gppoint{gp mark 0}{(3.621,2.880)}
\gppoint{gp mark 0}{(3.621,2.793)}
\gppoint{gp mark 0}{(3.621,2.880)}
\gppoint{gp mark 0}{(3.621,2.586)}
\gppoint{gp mark 0}{(3.621,2.696)}
\gppoint{gp mark 0}{(3.621,2.793)}
\gppoint{gp mark 0}{(3.621,2.793)}
\gppoint{gp mark 0}{(3.621,2.793)}
\gppoint{gp mark 0}{(3.621,2.793)}
\gppoint{gp mark 0}{(3.621,2.793)}
\gppoint{gp mark 0}{(3.621,2.793)}
\gppoint{gp mark 0}{(3.621,2.793)}
\gppoint{gp mark 0}{(3.621,3.157)}
\gppoint{gp mark 0}{(3.621,2.793)}
\gppoint{gp mark 0}{(3.621,2.793)}
\gppoint{gp mark 0}{(3.621,2.793)}
\gppoint{gp mark 0}{(3.621,2.793)}
\gppoint{gp mark 0}{(3.621,2.793)}
\gppoint{gp mark 0}{(3.621,2.793)}
\gppoint{gp mark 0}{(3.621,2.793)}
\gppoint{gp mark 0}{(3.621,2.793)}
\gppoint{gp mark 0}{(3.621,2.793)}
\gppoint{gp mark 0}{(3.621,2.793)}
\gppoint{gp mark 0}{(3.621,2.793)}
\gppoint{gp mark 0}{(3.621,2.793)}
\gppoint{gp mark 0}{(3.621,2.793)}
\gppoint{gp mark 0}{(3.621,2.793)}
\gppoint{gp mark 0}{(3.621,2.793)}
\gppoint{gp mark 0}{(3.621,2.793)}
\gppoint{gp mark 0}{(3.621,2.793)}
\gppoint{gp mark 0}{(3.621,2.793)}
\gppoint{gp mark 0}{(3.621,2.793)}
\gppoint{gp mark 0}{(3.621,2.793)}
\gppoint{gp mark 0}{(3.621,2.793)}
\gppoint{gp mark 0}{(3.621,3.267)}
\gppoint{gp mark 0}{(3.621,2.793)}
\gppoint{gp mark 0}{(3.621,2.793)}
\gppoint{gp mark 0}{(3.621,2.793)}
\gppoint{gp mark 0}{(3.621,2.309)}
\gppoint{gp mark 0}{(3.621,2.793)}
\gppoint{gp mark 0}{(3.621,2.793)}
\gppoint{gp mark 0}{(3.621,2.696)}
\gppoint{gp mark 0}{(3.621,2.696)}
\gppoint{gp mark 0}{(3.621,2.793)}
\gppoint{gp mark 0}{(3.621,2.793)}
\gppoint{gp mark 0}{(3.621,2.793)}
\gppoint{gp mark 0}{(3.621,2.793)}
\gppoint{gp mark 0}{(3.621,2.793)}
\gppoint{gp mark 0}{(3.621,2.793)}
\gppoint{gp mark 0}{(3.621,2.793)}
\gppoint{gp mark 0}{(3.621,2.793)}
\gppoint{gp mark 0}{(3.621,2.793)}
\gppoint{gp mark 0}{(3.621,2.793)}
\gppoint{gp mark 0}{(3.621,2.793)}
\gppoint{gp mark 0}{(3.621,2.460)}
\gppoint{gp mark 0}{(3.621,2.696)}
\gppoint{gp mark 0}{(3.621,2.793)}
\gppoint{gp mark 0}{(3.621,2.309)}
\gppoint{gp mark 0}{(3.621,2.793)}
\gppoint{gp mark 0}{(3.621,2.793)}
\gppoint{gp mark 0}{(3.621,2.793)}
\gppoint{gp mark 0}{(3.621,2.460)}
\gppoint{gp mark 0}{(3.621,3.157)}
\gppoint{gp mark 0}{(3.621,2.793)}
\gppoint{gp mark 0}{(3.621,2.793)}
\gppoint{gp mark 0}{(3.621,2.793)}
\gppoint{gp mark 0}{(3.621,2.696)}
\gppoint{gp mark 0}{(3.621,2.793)}
\gppoint{gp mark 0}{(3.621,2.309)}
\gppoint{gp mark 0}{(3.621,2.793)}
\gppoint{gp mark 0}{(3.621,2.793)}
\gppoint{gp mark 0}{(3.621,2.793)}
\gppoint{gp mark 0}{(3.621,2.793)}
\gppoint{gp mark 0}{(3.621,2.793)}
\gppoint{gp mark 0}{(3.621,2.793)}
\gppoint{gp mark 0}{(3.621,2.309)}
\gppoint{gp mark 0}{(3.621,2.309)}
\gppoint{gp mark 0}{(3.621,2.793)}
\gppoint{gp mark 0}{(3.621,3.157)}
\gppoint{gp mark 0}{(3.621,2.793)}
\gppoint{gp mark 0}{(3.621,2.586)}
\gppoint{gp mark 0}{(3.621,2.309)}
\gppoint{gp mark 0}{(3.621,2.793)}
\gppoint{gp mark 0}{(3.621,2.793)}
\gppoint{gp mark 0}{(3.621,2.880)}
\gppoint{gp mark 0}{(3.621,2.793)}
\gppoint{gp mark 0}{(3.621,2.793)}
\gppoint{gp mark 0}{(3.621,2.793)}
\gppoint{gp mark 0}{(3.621,2.793)}
\gppoint{gp mark 0}{(3.621,2.793)}
\gppoint{gp mark 0}{(3.621,2.793)}
\gppoint{gp mark 0}{(3.621,2.309)}
\gppoint{gp mark 0}{(3.621,2.793)}
\gppoint{gp mark 0}{(3.621,2.793)}
\gppoint{gp mark 0}{(3.621,2.309)}
\gppoint{gp mark 0}{(3.621,2.793)}
\gppoint{gp mark 0}{(3.621,2.793)}
\gppoint{gp mark 0}{(3.621,2.793)}
\gppoint{gp mark 0}{(3.621,2.793)}
\gppoint{gp mark 0}{(3.621,2.696)}
\gppoint{gp mark 0}{(3.621,2.793)}
\gppoint{gp mark 0}{(3.621,2.793)}
\gppoint{gp mark 0}{(3.621,2.793)}
\gppoint{gp mark 0}{(3.621,2.793)}
\gppoint{gp mark 0}{(3.621,2.793)}
\gppoint{gp mark 0}{(3.621,2.696)}
\gppoint{gp mark 0}{(3.621,2.793)}
\gppoint{gp mark 0}{(3.621,2.793)}
\gppoint{gp mark 0}{(3.621,2.793)}
\gppoint{gp mark 0}{(3.621,2.793)}
\gppoint{gp mark 0}{(3.621,2.586)}
\gppoint{gp mark 0}{(3.621,2.793)}
\gppoint{gp mark 0}{(3.621,2.793)}
\gppoint{gp mark 0}{(3.621,2.793)}
\gppoint{gp mark 0}{(3.621,2.793)}
\gppoint{gp mark 0}{(3.621,2.793)}
\gppoint{gp mark 0}{(3.621,2.793)}
\gppoint{gp mark 0}{(3.621,2.793)}
\gppoint{gp mark 0}{(3.621,2.793)}
\gppoint{gp mark 0}{(3.621,2.793)}
\gppoint{gp mark 0}{(3.621,2.793)}
\gppoint{gp mark 0}{(3.621,2.793)}
\gppoint{gp mark 0}{(3.621,2.793)}
\gppoint{gp mark 0}{(3.621,2.793)}
\gppoint{gp mark 0}{(3.621,2.309)}
\gppoint{gp mark 0}{(3.779,2.696)}
\gppoint{gp mark 0}{(3.779,3.317)}
\gppoint{gp mark 0}{(3.779,3.096)}
\gppoint{gp mark 0}{(3.779,2.586)}
\gppoint{gp mark 0}{(3.779,3.096)}
\gppoint{gp mark 0}{(3.779,3.157)}
\gppoint{gp mark 0}{(3.779,3.267)}
\gppoint{gp mark 0}{(3.779,2.696)}
\gppoint{gp mark 0}{(3.779,3.096)}
\gppoint{gp mark 0}{(3.779,2.696)}
\gppoint{gp mark 0}{(3.779,2.696)}
\gppoint{gp mark 0}{(3.779,2.793)}
\gppoint{gp mark 0}{(3.779,2.586)}
\gppoint{gp mark 0}{(3.779,3.030)}
\gppoint{gp mark 0}{(3.779,2.696)}
\gppoint{gp mark 0}{(3.779,2.958)}
\gppoint{gp mark 0}{(3.779,3.408)}
\gppoint{gp mark 0}{(3.779,2.696)}
\gppoint{gp mark 0}{(3.779,2.793)}
\gppoint{gp mark 0}{(3.779,2.793)}
\gppoint{gp mark 0}{(3.779,2.793)}
\gppoint{gp mark 0}{(3.779,2.696)}
\gppoint{gp mark 0}{(3.779,2.958)}
\gppoint{gp mark 0}{(3.779,2.460)}
\gppoint{gp mark 0}{(3.779,3.267)}
\gppoint{gp mark 0}{(3.779,3.030)}
\gppoint{gp mark 0}{(3.779,3.030)}
\gppoint{gp mark 0}{(3.779,2.880)}
\gppoint{gp mark 0}{(3.779,3.214)}
\gppoint{gp mark 0}{(3.779,3.096)}
\gppoint{gp mark 0}{(3.779,3.267)}
\gppoint{gp mark 0}{(3.779,2.880)}
\gppoint{gp mark 0}{(3.779,3.096)}
\gppoint{gp mark 0}{(3.779,3.157)}
\gppoint{gp mark 0}{(3.779,2.696)}
\gppoint{gp mark 0}{(3.779,3.030)}
\gppoint{gp mark 0}{(3.779,3.214)}
\gppoint{gp mark 0}{(3.779,3.317)}
\gppoint{gp mark 0}{(3.779,3.030)}
\gppoint{gp mark 0}{(3.779,2.880)}
\gppoint{gp mark 0}{(3.779,2.958)}
\gppoint{gp mark 0}{(3.779,3.267)}
\gppoint{gp mark 0}{(3.779,2.460)}
\gppoint{gp mark 0}{(3.779,2.696)}
\gppoint{gp mark 0}{(3.779,2.880)}
\gppoint{gp mark 0}{(3.779,3.030)}
\gppoint{gp mark 0}{(3.779,2.958)}
\gppoint{gp mark 0}{(3.779,2.586)}
\gppoint{gp mark 0}{(3.779,2.880)}
\gppoint{gp mark 0}{(3.779,2.696)}
\gppoint{gp mark 0}{(3.779,3.267)}
\gppoint{gp mark 0}{(3.779,3.096)}
\gppoint{gp mark 0}{(3.779,3.317)}
\gppoint{gp mark 0}{(3.779,3.157)}
\gppoint{gp mark 0}{(3.779,3.096)}
\gppoint{gp mark 0}{(3.779,2.880)}
\gppoint{gp mark 0}{(3.779,2.696)}
\gppoint{gp mark 0}{(3.779,2.696)}
\gppoint{gp mark 0}{(3.779,2.880)}
\gppoint{gp mark 0}{(3.779,2.696)}
\gppoint{gp mark 0}{(3.779,2.586)}
\gppoint{gp mark 0}{(3.779,3.096)}
\gppoint{gp mark 0}{(3.779,3.096)}
\gppoint{gp mark 0}{(3.779,3.096)}
\gppoint{gp mark 0}{(3.779,2.793)}
\gppoint{gp mark 0}{(3.779,3.096)}
\gppoint{gp mark 0}{(3.779,3.096)}
\gppoint{gp mark 0}{(3.779,2.696)}
\gppoint{gp mark 0}{(3.779,2.586)}
\gppoint{gp mark 0}{(3.779,3.267)}
\gppoint{gp mark 0}{(3.779,3.214)}
\gppoint{gp mark 0}{(3.779,2.586)}
\gppoint{gp mark 0}{(3.779,3.096)}
\gppoint{gp mark 0}{(3.779,2.309)}
\gppoint{gp mark 0}{(3.779,2.880)}
\gppoint{gp mark 0}{(3.779,2.880)}
\gppoint{gp mark 0}{(3.779,2.460)}
\gppoint{gp mark 0}{(3.779,3.450)}
\gppoint{gp mark 0}{(3.779,3.157)}
\gppoint{gp mark 0}{(3.779,2.460)}
\gppoint{gp mark 0}{(3.779,2.460)}
\gppoint{gp mark 0}{(3.779,3.030)}
\gppoint{gp mark 0}{(3.779,3.096)}
\gppoint{gp mark 0}{(3.779,2.793)}
\gppoint{gp mark 0}{(3.779,2.793)}
\gppoint{gp mark 0}{(3.779,2.793)}
\gppoint{gp mark 0}{(3.779,2.586)}
\gppoint{gp mark 0}{(3.779,2.880)}
\gppoint{gp mark 0}{(3.779,2.696)}
\gppoint{gp mark 0}{(3.779,3.096)}
\gppoint{gp mark 0}{(3.779,3.030)}
\gppoint{gp mark 0}{(3.779,3.096)}
\gppoint{gp mark 0}{(3.779,3.096)}
\gppoint{gp mark 0}{(3.779,3.030)}
\gppoint{gp mark 0}{(3.779,2.586)}
\gppoint{gp mark 0}{(3.779,3.030)}
\gppoint{gp mark 0}{(3.779,3.096)}
\gppoint{gp mark 0}{(3.779,2.696)}
\gppoint{gp mark 0}{(3.779,2.586)}
\gppoint{gp mark 0}{(3.779,2.309)}
\gppoint{gp mark 0}{(3.779,2.958)}
\gppoint{gp mark 0}{(3.779,3.317)}
\gppoint{gp mark 0}{(3.779,3.214)}
\gppoint{gp mark 0}{(3.779,2.880)}
\gppoint{gp mark 0}{(3.779,2.586)}
\gppoint{gp mark 0}{(3.779,3.030)}
\gppoint{gp mark 0}{(3.779,3.267)}
\gppoint{gp mark 0}{(3.779,2.586)}
\gppoint{gp mark 0}{(3.779,2.880)}
\gppoint{gp mark 0}{(3.779,2.793)}
\gppoint{gp mark 0}{(3.779,2.880)}
\gppoint{gp mark 0}{(3.779,2.958)}
\gppoint{gp mark 0}{(3.779,3.267)}
\gppoint{gp mark 0}{(3.779,2.880)}
\gppoint{gp mark 0}{(3.779,3.157)}
\gppoint{gp mark 0}{(3.779,2.696)}
\gppoint{gp mark 0}{(3.779,2.880)}
\gppoint{gp mark 0}{(3.779,3.096)}
\gppoint{gp mark 0}{(3.779,3.214)}
\gppoint{gp mark 0}{(3.779,2.586)}
\gppoint{gp mark 0}{(3.779,2.586)}
\gppoint{gp mark 0}{(3.779,3.096)}
\gppoint{gp mark 0}{(3.779,2.696)}
\gppoint{gp mark 0}{(3.779,3.157)}
\gppoint{gp mark 0}{(3.779,2.696)}
\gppoint{gp mark 0}{(3.779,3.030)}
\gppoint{gp mark 0}{(3.779,2.793)}
\gppoint{gp mark 0}{(3.779,2.880)}
\gppoint{gp mark 0}{(3.779,2.793)}
\gppoint{gp mark 0}{(3.779,2.793)}
\gppoint{gp mark 0}{(3.779,2.958)}
\gppoint{gp mark 0}{(3.779,2.793)}
\gppoint{gp mark 0}{(3.779,2.586)}
\gppoint{gp mark 0}{(3.779,3.157)}
\gppoint{gp mark 0}{(3.779,2.793)}
\gppoint{gp mark 0}{(3.779,2.793)}
\gppoint{gp mark 0}{(3.779,2.460)}
\gppoint{gp mark 0}{(3.779,2.696)}
\gppoint{gp mark 0}{(3.779,2.696)}
\gppoint{gp mark 0}{(3.779,2.958)}
\gppoint{gp mark 0}{(3.779,2.586)}
\gppoint{gp mark 0}{(3.779,2.586)}
\gppoint{gp mark 0}{(3.779,3.157)}
\gppoint{gp mark 0}{(3.779,3.096)}
\gppoint{gp mark 0}{(3.779,2.880)}
\gppoint{gp mark 0}{(3.779,3.267)}
\gppoint{gp mark 0}{(3.779,2.793)}
\gppoint{gp mark 0}{(3.779,2.696)}
\gppoint{gp mark 0}{(3.779,2.880)}
\gppoint{gp mark 0}{(3.779,2.460)}
\gppoint{gp mark 0}{(3.779,2.793)}
\gppoint{gp mark 0}{(3.779,2.696)}
\gppoint{gp mark 0}{(3.779,2.696)}
\gppoint{gp mark 0}{(3.779,3.157)}
\gppoint{gp mark 0}{(3.779,2.696)}
\gppoint{gp mark 0}{(3.779,2.586)}
\gppoint{gp mark 0}{(3.779,3.157)}
\gppoint{gp mark 0}{(3.779,3.157)}
\gppoint{gp mark 0}{(3.779,2.696)}
\gppoint{gp mark 0}{(3.779,2.696)}
\gppoint{gp mark 0}{(3.779,2.586)}
\gppoint{gp mark 0}{(3.779,3.030)}
\gppoint{gp mark 0}{(3.779,2.793)}
\gppoint{gp mark 0}{(3.779,2.696)}
\gppoint{gp mark 0}{(3.779,2.696)}
\gppoint{gp mark 0}{(3.779,2.880)}
\gppoint{gp mark 0}{(3.779,3.096)}
\gppoint{gp mark 0}{(3.779,3.096)}
\gppoint{gp mark 0}{(3.779,2.586)}
\gppoint{gp mark 0}{(3.779,3.030)}
\gppoint{gp mark 0}{(3.779,2.696)}
\gppoint{gp mark 0}{(3.779,2.958)}
\gppoint{gp mark 0}{(3.779,2.696)}
\gppoint{gp mark 0}{(3.779,2.880)}
\gppoint{gp mark 0}{(3.779,2.586)}
\gppoint{gp mark 0}{(3.779,3.096)}
\gppoint{gp mark 0}{(3.779,3.096)}
\gppoint{gp mark 0}{(3.779,3.096)}
\gppoint{gp mark 0}{(3.779,3.096)}
\gppoint{gp mark 0}{(3.779,2.793)}
\gppoint{gp mark 0}{(3.779,3.030)}
\gppoint{gp mark 0}{(3.779,2.460)}
\gppoint{gp mark 0}{(3.779,2.793)}
\gppoint{gp mark 0}{(3.779,2.586)}
\gppoint{gp mark 0}{(3.779,2.880)}
\gppoint{gp mark 0}{(3.779,2.460)}
\gppoint{gp mark 0}{(3.779,3.096)}
\gppoint{gp mark 0}{(3.779,2.696)}
\gppoint{gp mark 0}{(3.779,2.880)}
\gppoint{gp mark 0}{(3.779,3.157)}
\gppoint{gp mark 0}{(3.779,3.096)}
\gppoint{gp mark 0}{(3.779,2.880)}
\gppoint{gp mark 0}{(3.779,2.793)}
\gppoint{gp mark 0}{(3.779,3.096)}
\gppoint{gp mark 0}{(3.779,3.267)}
\gppoint{gp mark 0}{(3.779,3.030)}
\gppoint{gp mark 0}{(3.779,2.958)}
\gppoint{gp mark 0}{(3.779,3.096)}
\gppoint{gp mark 0}{(3.918,2.586)}
\gppoint{gp mark 0}{(3.918,2.880)}
\gppoint{gp mark 0}{(3.918,2.958)}
\gppoint{gp mark 0}{(3.918,2.880)}
\gppoint{gp mark 0}{(3.918,2.793)}
\gppoint{gp mark 0}{(3.918,3.096)}
\gppoint{gp mark 0}{(3.918,2.793)}
\gppoint{gp mark 0}{(3.918,3.267)}
\gppoint{gp mark 0}{(3.918,2.460)}
\gppoint{gp mark 0}{(3.918,3.157)}
\gppoint{gp mark 0}{(3.918,3.030)}
\gppoint{gp mark 0}{(3.918,3.030)}
\gppoint{gp mark 0}{(3.918,2.586)}
\gppoint{gp mark 0}{(3.918,2.460)}
\gppoint{gp mark 0}{(3.918,3.214)}
\gppoint{gp mark 0}{(3.918,3.030)}
\gppoint{gp mark 0}{(3.918,2.696)}
\gppoint{gp mark 0}{(3.918,3.030)}
\gppoint{gp mark 0}{(3.918,2.958)}
\gppoint{gp mark 0}{(3.918,3.214)}
\gppoint{gp mark 0}{(3.918,3.317)}
\gppoint{gp mark 0}{(3.918,2.460)}
\gppoint{gp mark 0}{(3.918,2.696)}
\gppoint{gp mark 0}{(3.918,3.317)}
\gppoint{gp mark 0}{(3.918,3.030)}
\gppoint{gp mark 0}{(3.918,3.030)}
\gppoint{gp mark 0}{(3.918,2.958)}
\gppoint{gp mark 0}{(3.918,3.096)}
\gppoint{gp mark 0}{(3.918,3.267)}
\gppoint{gp mark 0}{(3.918,2.696)}
\gppoint{gp mark 0}{(3.918,2.958)}
\gppoint{gp mark 0}{(3.918,2.586)}
\gppoint{gp mark 0}{(3.918,2.793)}
\gppoint{gp mark 0}{(3.918,3.030)}
\gppoint{gp mark 0}{(3.918,3.030)}
\gppoint{gp mark 0}{(3.918,2.958)}
\gppoint{gp mark 0}{(3.918,3.030)}
\gppoint{gp mark 0}{(3.918,2.460)}
\gppoint{gp mark 0}{(3.918,2.958)}
\gppoint{gp mark 0}{(3.918,3.030)}
\gppoint{gp mark 0}{(3.918,3.364)}
\gppoint{gp mark 0}{(3.918,3.096)}
\gppoint{gp mark 0}{(3.918,3.157)}
\gppoint{gp mark 0}{(3.918,2.460)}
\gppoint{gp mark 0}{(3.918,2.460)}
\gppoint{gp mark 0}{(3.918,3.364)}
\gppoint{gp mark 0}{(3.918,3.157)}
\gppoint{gp mark 0}{(3.918,3.096)}
\gppoint{gp mark 0}{(3.918,3.157)}
\gppoint{gp mark 0}{(3.918,2.880)}
\gppoint{gp mark 0}{(3.918,3.096)}
\gppoint{gp mark 0}{(3.918,2.880)}
\gppoint{gp mark 0}{(3.918,2.880)}
\gppoint{gp mark 0}{(3.918,3.157)}
\gppoint{gp mark 0}{(3.918,3.030)}
\gppoint{gp mark 0}{(3.918,3.030)}
\gppoint{gp mark 0}{(3.918,3.030)}
\gppoint{gp mark 0}{(3.918,2.958)}
\gppoint{gp mark 0}{(3.918,3.030)}
\gppoint{gp mark 0}{(3.918,3.096)}
\gppoint{gp mark 0}{(3.918,2.880)}
\gppoint{gp mark 0}{(3.918,3.157)}
\gppoint{gp mark 0}{(3.918,3.096)}
\gppoint{gp mark 0}{(3.918,2.958)}
\gppoint{gp mark 0}{(3.918,3.030)}
\gppoint{gp mark 0}{(3.918,2.880)}
\gppoint{gp mark 0}{(3.918,2.696)}
\gppoint{gp mark 0}{(3.918,3.096)}
\gppoint{gp mark 0}{(3.918,3.030)}
\gppoint{gp mark 0}{(3.918,2.460)}
\gppoint{gp mark 0}{(3.918,2.460)}
\gppoint{gp mark 0}{(3.918,3.364)}
\gppoint{gp mark 0}{(3.918,2.696)}
\gppoint{gp mark 0}{(3.918,2.958)}
\gppoint{gp mark 0}{(3.918,2.958)}
\gppoint{gp mark 0}{(3.918,2.696)}
\gppoint{gp mark 0}{(3.918,3.267)}
\gppoint{gp mark 0}{(3.918,2.696)}
\gppoint{gp mark 0}{(3.918,2.880)}
\gppoint{gp mark 0}{(3.918,2.696)}
\gppoint{gp mark 0}{(3.918,3.096)}
\gppoint{gp mark 0}{(3.918,3.214)}
\gppoint{gp mark 0}{(3.918,3.030)}
\gppoint{gp mark 0}{(3.918,2.880)}
\gppoint{gp mark 0}{(3.918,3.096)}
\gppoint{gp mark 0}{(3.918,3.364)}
\gppoint{gp mark 0}{(3.918,2.793)}
\gppoint{gp mark 0}{(3.918,3.157)}
\gppoint{gp mark 0}{(3.918,2.958)}
\gppoint{gp mark 0}{(3.918,3.096)}
\gppoint{gp mark 0}{(3.918,2.958)}
\gppoint{gp mark 0}{(3.918,2.958)}
\gppoint{gp mark 0}{(3.918,3.030)}
\gppoint{gp mark 0}{(3.918,2.696)}
\gppoint{gp mark 0}{(3.918,2.958)}
\gppoint{gp mark 0}{(3.918,2.696)}
\gppoint{gp mark 0}{(3.918,2.958)}
\gppoint{gp mark 0}{(3.918,3.030)}
\gppoint{gp mark 0}{(3.918,2.696)}
\gppoint{gp mark 0}{(3.918,3.157)}
\gppoint{gp mark 0}{(3.918,3.030)}
\gppoint{gp mark 0}{(3.918,3.030)}
\gppoint{gp mark 0}{(3.918,3.364)}
\gppoint{gp mark 0}{(3.918,3.030)}
\gppoint{gp mark 0}{(3.918,3.157)}
\gppoint{gp mark 0}{(3.918,3.030)}
\gppoint{gp mark 0}{(3.918,3.364)}
\gppoint{gp mark 0}{(3.918,3.634)}
\gppoint{gp mark 0}{(3.918,2.880)}
\gppoint{gp mark 0}{(3.918,3.030)}
\gppoint{gp mark 0}{(3.918,2.696)}
\gppoint{gp mark 0}{(3.918,2.696)}
\gppoint{gp mark 0}{(3.918,2.696)}
\gppoint{gp mark 0}{(3.918,2.696)}
\gppoint{gp mark 0}{(3.918,3.030)}
\gppoint{gp mark 0}{(3.918,3.157)}
\gppoint{gp mark 0}{(3.918,2.309)}
\gppoint{gp mark 0}{(3.918,2.309)}
\gppoint{gp mark 0}{(3.918,3.157)}
\gppoint{gp mark 0}{(3.918,3.157)}
\gppoint{gp mark 0}{(3.918,2.696)}
\gppoint{gp mark 0}{(3.918,2.793)}
\gppoint{gp mark 0}{(3.918,2.880)}
\gppoint{gp mark 0}{(3.918,2.586)}
\gppoint{gp mark 0}{(3.918,2.586)}
\gppoint{gp mark 0}{(3.918,2.586)}
\gppoint{gp mark 0}{(3.918,2.696)}
\gppoint{gp mark 0}{(3.918,2.958)}
\gppoint{gp mark 0}{(3.918,3.030)}
\gppoint{gp mark 0}{(3.918,2.586)}
\gppoint{gp mark 0}{(3.918,2.880)}
\gppoint{gp mark 0}{(3.918,3.214)}
\gppoint{gp mark 0}{(3.918,2.696)}
\gppoint{gp mark 0}{(3.918,2.696)}
\gppoint{gp mark 0}{(3.918,3.157)}
\gppoint{gp mark 0}{(3.918,2.696)}
\gppoint{gp mark 0}{(3.918,2.696)}
\gppoint{gp mark 0}{(3.918,2.696)}
\gppoint{gp mark 0}{(3.918,3.490)}
\gppoint{gp mark 0}{(3.918,2.586)}
\gppoint{gp mark 0}{(3.918,3.030)}
\gppoint{gp mark 0}{(3.918,2.696)}
\gppoint{gp mark 0}{(3.918,2.696)}
\gppoint{gp mark 0}{(3.918,3.030)}
\gppoint{gp mark 0}{(3.918,2.586)}
\gppoint{gp mark 0}{(3.918,3.157)}
\gppoint{gp mark 0}{(3.918,2.793)}
\gppoint{gp mark 0}{(3.918,2.586)}
\gppoint{gp mark 0}{(3.918,2.586)}
\gppoint{gp mark 0}{(3.918,2.586)}
\gppoint{gp mark 0}{(3.918,2.696)}
\gppoint{gp mark 0}{(3.918,2.586)}
\gppoint{gp mark 0}{(3.918,2.460)}
\gppoint{gp mark 0}{(3.918,2.586)}
\gppoint{gp mark 0}{(3.918,2.793)}
\gppoint{gp mark 0}{(3.918,2.586)}
\gppoint{gp mark 0}{(3.918,2.586)}
\gppoint{gp mark 0}{(3.918,2.586)}
\gppoint{gp mark 0}{(3.918,2.586)}
\gppoint{gp mark 0}{(3.918,3.096)}
\gppoint{gp mark 0}{(3.918,3.096)}
\gppoint{gp mark 0}{(3.918,2.586)}
\gppoint{gp mark 0}{(3.918,2.793)}
\gppoint{gp mark 0}{(3.918,3.096)}
\gppoint{gp mark 0}{(3.918,2.586)}
\gppoint{gp mark 0}{(3.918,2.958)}
\gppoint{gp mark 0}{(3.918,2.793)}
\gppoint{gp mark 0}{(3.918,2.586)}
\gppoint{gp mark 0}{(3.918,3.030)}
\gppoint{gp mark 0}{(3.918,3.030)}
\gppoint{gp mark 0}{(3.918,3.096)}
\gppoint{gp mark 0}{(3.918,2.793)}
\gppoint{gp mark 0}{(3.918,3.096)}
\gppoint{gp mark 0}{(3.918,2.880)}
\gppoint{gp mark 0}{(3.918,2.793)}
\gppoint{gp mark 0}{(3.918,3.096)}
\gppoint{gp mark 0}{(3.918,3.267)}
\gppoint{gp mark 0}{(3.918,3.214)}
\gppoint{gp mark 0}{(3.918,2.793)}
\gppoint{gp mark 0}{(3.918,3.030)}
\gppoint{gp mark 0}{(3.918,2.958)}
\gppoint{gp mark 0}{(3.918,3.030)}
\gppoint{gp mark 0}{(3.918,2.958)}
\gppoint{gp mark 0}{(3.918,3.096)}
\gppoint{gp mark 0}{(3.918,3.030)}
\gppoint{gp mark 0}{(3.918,3.030)}
\gppoint{gp mark 0}{(3.918,3.030)}
\gppoint{gp mark 0}{(3.918,2.793)}
\gppoint{gp mark 0}{(3.918,3.030)}
\gppoint{gp mark 0}{(3.918,2.793)}
\gppoint{gp mark 0}{(3.918,2.793)}
\gppoint{gp mark 0}{(3.918,3.030)}
\gppoint{gp mark 0}{(3.918,2.793)}
\gppoint{gp mark 0}{(3.918,3.529)}
\gppoint{gp mark 0}{(3.918,3.030)}
\gppoint{gp mark 0}{(3.918,2.958)}
\gppoint{gp mark 0}{(3.918,2.958)}
\gppoint{gp mark 0}{(3.918,3.096)}
\gppoint{gp mark 0}{(3.918,2.880)}
\gppoint{gp mark 0}{(3.918,3.030)}
\gppoint{gp mark 0}{(3.918,3.157)}
\gppoint{gp mark 0}{(3.918,2.793)}
\gppoint{gp mark 0}{(3.918,3.157)}
\gppoint{gp mark 0}{(3.918,3.030)}
\gppoint{gp mark 0}{(3.918,3.157)}
\gppoint{gp mark 0}{(3.918,3.157)}
\gppoint{gp mark 0}{(3.918,2.793)}
\gppoint{gp mark 0}{(3.918,3.030)}
\gppoint{gp mark 0}{(3.918,3.030)}
\gppoint{gp mark 0}{(3.918,2.793)}
\gppoint{gp mark 0}{(3.918,3.030)}
\gppoint{gp mark 0}{(3.918,3.030)}
\gppoint{gp mark 0}{(3.918,2.958)}
\gppoint{gp mark 0}{(3.918,2.958)}
\gppoint{gp mark 0}{(3.918,2.696)}
\gppoint{gp mark 0}{(3.918,2.793)}
\gppoint{gp mark 0}{(3.918,3.096)}
\gppoint{gp mark 0}{(3.918,3.096)}
\gppoint{gp mark 0}{(3.918,2.793)}
\gppoint{gp mark 0}{(3.918,3.096)}
\gppoint{gp mark 0}{(3.918,2.880)}
\gppoint{gp mark 0}{(3.918,3.030)}
\gppoint{gp mark 0}{(3.918,3.030)}
\gppoint{gp mark 0}{(3.918,3.030)}
\gppoint{gp mark 0}{(3.918,3.096)}
\gppoint{gp mark 0}{(3.918,2.793)}
\gppoint{gp mark 0}{(3.918,3.214)}
\gppoint{gp mark 0}{(3.918,3.096)}
\gppoint{gp mark 0}{(3.918,2.793)}
\gppoint{gp mark 0}{(3.918,3.030)}
\gppoint{gp mark 0}{(3.918,3.030)}
\gppoint{gp mark 0}{(3.918,2.793)}
\gppoint{gp mark 0}{(3.918,2.880)}
\gppoint{gp mark 0}{(3.918,3.030)}
\gppoint{gp mark 0}{(3.918,2.460)}
\gppoint{gp mark 0}{(3.918,3.030)}
\gppoint{gp mark 0}{(3.918,3.214)}
\gppoint{gp mark 0}{(3.918,3.030)}
\gppoint{gp mark 0}{(3.918,3.030)}
\gppoint{gp mark 0}{(3.918,3.157)}
\gppoint{gp mark 0}{(3.918,3.030)}
\gppoint{gp mark 0}{(3.918,3.030)}
\gppoint{gp mark 0}{(3.918,3.030)}
\gppoint{gp mark 0}{(3.918,2.880)}
\gppoint{gp mark 0}{(3.918,3.030)}
\gppoint{gp mark 0}{(3.918,3.030)}
\gppoint{gp mark 0}{(3.918,3.317)}
\gppoint{gp mark 0}{(3.918,2.880)}
\gppoint{gp mark 0}{(3.918,3.408)}
\gppoint{gp mark 0}{(3.918,2.880)}
\gppoint{gp mark 0}{(3.918,2.880)}
\gppoint{gp mark 0}{(3.918,3.364)}
\gppoint{gp mark 0}{(3.918,2.793)}
\gppoint{gp mark 0}{(3.918,3.364)}
\gppoint{gp mark 0}{(3.918,3.030)}
\gppoint{gp mark 0}{(3.918,3.030)}
\gppoint{gp mark 0}{(3.918,2.696)}
\gppoint{gp mark 0}{(3.918,2.880)}
\gppoint{gp mark 0}{(3.918,2.880)}
\gppoint{gp mark 0}{(3.918,2.880)}
\gppoint{gp mark 0}{(3.918,2.958)}
\gppoint{gp mark 0}{(3.918,3.096)}
\gppoint{gp mark 0}{(3.918,3.030)}
\gppoint{gp mark 0}{(3.918,3.157)}
\gppoint{gp mark 0}{(3.918,2.696)}
\gppoint{gp mark 0}{(3.918,2.696)}
\gppoint{gp mark 0}{(3.918,2.958)}
\gppoint{gp mark 0}{(3.918,2.696)}
\gppoint{gp mark 0}{(3.918,2.958)}
\gppoint{gp mark 0}{(3.918,2.958)}
\gppoint{gp mark 0}{(3.918,2.958)}
\gppoint{gp mark 0}{(3.918,3.030)}
\gppoint{gp mark 0}{(3.918,2.958)}
\gppoint{gp mark 0}{(3.918,3.030)}
\gppoint{gp mark 0}{(3.918,2.696)}
\gppoint{gp mark 0}{(3.918,3.030)}
\gppoint{gp mark 0}{(3.918,2.958)}
\gppoint{gp mark 0}{(3.918,3.030)}
\gppoint{gp mark 0}{(3.918,3.030)}
\gppoint{gp mark 0}{(3.918,3.157)}
\gppoint{gp mark 0}{(3.918,2.696)}
\gppoint{gp mark 0}{(3.918,3.157)}
\gppoint{gp mark 0}{(3.918,2.696)}
\gppoint{gp mark 0}{(3.918,3.157)}
\gppoint{gp mark 0}{(3.918,3.096)}
\gppoint{gp mark 0}{(3.918,3.157)}
\gppoint{gp mark 0}{(3.918,3.096)}
\gppoint{gp mark 0}{(3.918,2.696)}
\gppoint{gp mark 0}{(3.918,3.030)}
\gppoint{gp mark 0}{(3.918,3.096)}
\gppoint{gp mark 0}{(3.918,3.030)}
\gppoint{gp mark 0}{(3.918,3.030)}
\gppoint{gp mark 0}{(3.918,3.030)}
\gppoint{gp mark 0}{(3.918,3.096)}
\gppoint{gp mark 0}{(3.918,2.793)}
\gppoint{gp mark 0}{(3.918,3.030)}
\gppoint{gp mark 0}{(3.918,3.030)}
\gppoint{gp mark 0}{(3.918,2.793)}
\gppoint{gp mark 0}{(3.918,3.030)}
\gppoint{gp mark 0}{(3.918,2.793)}
\gppoint{gp mark 0}{(3.918,2.793)}
\gppoint{gp mark 0}{(3.918,3.364)}
\gppoint{gp mark 0}{(3.918,3.030)}
\gppoint{gp mark 0}{(3.918,3.030)}
\gppoint{gp mark 0}{(3.918,2.696)}
\gppoint{gp mark 0}{(3.918,2.696)}
\gppoint{gp mark 0}{(3.918,3.030)}
\gppoint{gp mark 0}{(3.918,2.958)}
\gppoint{gp mark 0}{(3.918,2.880)}
\gppoint{gp mark 0}{(3.918,2.696)}
\gppoint{gp mark 0}{(3.918,2.696)}
\gppoint{gp mark 0}{(3.918,2.696)}
\gppoint{gp mark 0}{(3.918,2.586)}
\gppoint{gp mark 0}{(3.918,2.696)}
\gppoint{gp mark 0}{(3.918,3.030)}
\gppoint{gp mark 0}{(3.918,3.030)}
\gppoint{gp mark 0}{(3.918,3.030)}
\gppoint{gp mark 0}{(3.918,2.460)}
\gppoint{gp mark 0}{(3.918,2.696)}
\gppoint{gp mark 0}{(3.918,2.793)}
\gppoint{gp mark 0}{(3.918,2.696)}
\gppoint{gp mark 0}{(3.918,2.696)}
\gppoint{gp mark 0}{(3.918,3.030)}
\gppoint{gp mark 0}{(3.918,3.030)}
\gppoint{gp mark 0}{(3.918,2.696)}
\gppoint{gp mark 0}{(3.918,3.030)}
\gppoint{gp mark 0}{(3.918,3.030)}
\gppoint{gp mark 0}{(3.918,2.696)}
\gppoint{gp mark 0}{(3.918,2.880)}
\gppoint{gp mark 0}{(3.918,2.880)}
\gppoint{gp mark 0}{(3.918,2.958)}
\gppoint{gp mark 0}{(3.918,2.696)}
\gppoint{gp mark 0}{(3.918,2.880)}
\gppoint{gp mark 0}{(3.918,3.096)}
\gppoint{gp mark 0}{(3.918,2.880)}
\gppoint{gp mark 0}{(3.918,3.030)}
\gppoint{gp mark 0}{(3.918,2.880)}
\gppoint{gp mark 0}{(3.918,2.880)}
\gppoint{gp mark 0}{(3.918,2.793)}
\gppoint{gp mark 0}{(3.918,3.096)}
\gppoint{gp mark 0}{(3.918,2.880)}
\gppoint{gp mark 0}{(3.918,2.880)}
\gppoint{gp mark 0}{(3.918,3.096)}
\gppoint{gp mark 0}{(3.918,2.586)}
\gppoint{gp mark 0}{(3.918,2.958)}
\gppoint{gp mark 0}{(3.918,2.880)}
\gppoint{gp mark 0}{(3.918,3.214)}
\gppoint{gp mark 0}{(3.918,2.880)}
\gppoint{gp mark 0}{(3.918,2.696)}
\gppoint{gp mark 0}{(3.918,3.450)}
\gppoint{gp mark 0}{(3.918,2.880)}
\gppoint{gp mark 0}{(3.918,3.317)}
\gppoint{gp mark 0}{(3.918,3.096)}
\gppoint{gp mark 0}{(3.918,3.096)}
\gppoint{gp mark 0}{(3.918,2.586)}
\gppoint{gp mark 0}{(3.918,3.030)}
\gppoint{gp mark 0}{(3.918,3.030)}
\gppoint{gp mark 0}{(3.918,2.880)}
\gppoint{gp mark 0}{(3.918,3.030)}
\gppoint{gp mark 0}{(3.918,3.030)}
\gppoint{gp mark 0}{(3.918,2.696)}
\gppoint{gp mark 0}{(3.918,2.793)}
\gppoint{gp mark 0}{(3.918,2.958)}
\gppoint{gp mark 0}{(3.918,3.030)}
\gppoint{gp mark 0}{(3.918,3.030)}
\gppoint{gp mark 0}{(3.918,2.586)}
\gppoint{gp mark 0}{(3.918,2.696)}
\gppoint{gp mark 0}{(3.918,2.793)}
\gppoint{gp mark 0}{(3.918,3.096)}
\gppoint{gp mark 0}{(3.918,2.958)}
\gppoint{gp mark 0}{(3.918,2.460)}
\gppoint{gp mark 0}{(3.918,3.096)}
\gppoint{gp mark 0}{(3.918,3.030)}
\gppoint{gp mark 0}{(3.918,2.958)}
\gppoint{gp mark 0}{(3.918,3.030)}
\gppoint{gp mark 0}{(3.918,2.880)}
\gppoint{gp mark 0}{(3.918,2.958)}
\gppoint{gp mark 0}{(3.918,2.586)}
\gppoint{gp mark 0}{(3.918,2.696)}
\gppoint{gp mark 0}{(3.918,2.958)}
\gppoint{gp mark 0}{(3.918,3.030)}
\gppoint{gp mark 0}{(3.918,3.096)}
\gppoint{gp mark 0}{(3.918,2.696)}
\gppoint{gp mark 0}{(3.918,2.880)}
\gppoint{gp mark 0}{(3.918,3.030)}
\gppoint{gp mark 0}{(3.918,3.030)}
\gppoint{gp mark 0}{(3.918,3.030)}
\gppoint{gp mark 0}{(3.918,2.958)}
\gppoint{gp mark 0}{(3.918,3.030)}
\gppoint{gp mark 0}{(3.918,2.696)}
\gppoint{gp mark 0}{(3.918,3.030)}
\gppoint{gp mark 0}{(3.918,3.364)}
\gppoint{gp mark 0}{(3.918,2.958)}
\gppoint{gp mark 0}{(3.918,2.958)}
\gppoint{gp mark 0}{(3.918,3.030)}
\gppoint{gp mark 0}{(3.918,2.793)}
\gppoint{gp mark 0}{(3.918,3.096)}
\gppoint{gp mark 0}{(3.918,3.030)}
\gppoint{gp mark 0}{(3.918,2.880)}
\gppoint{gp mark 0}{(3.918,3.157)}
\gppoint{gp mark 0}{(3.918,3.030)}
\gppoint{gp mark 0}{(3.918,2.586)}
\gppoint{gp mark 0}{(3.918,2.696)}
\gppoint{gp mark 0}{(3.918,3.030)}
\gppoint{gp mark 0}{(3.918,2.460)}
\gppoint{gp mark 0}{(3.918,2.958)}
\gppoint{gp mark 0}{(3.918,3.096)}
\gppoint{gp mark 0}{(3.918,3.214)}
\gppoint{gp mark 0}{(3.918,3.096)}
\gppoint{gp mark 0}{(3.918,2.696)}
\gppoint{gp mark 0}{(3.918,2.880)}
\gppoint{gp mark 0}{(3.918,3.030)}
\gppoint{gp mark 0}{(3.918,2.696)}
\gppoint{gp mark 0}{(3.918,2.958)}
\gppoint{gp mark 0}{(3.918,3.030)}
\gppoint{gp mark 0}{(3.918,2.880)}
\gppoint{gp mark 0}{(3.918,2.958)}
\gppoint{gp mark 0}{(4.043,3.030)}
\gppoint{gp mark 0}{(4.043,2.958)}
\gppoint{gp mark 0}{(4.043,2.880)}
\gppoint{gp mark 0}{(4.043,3.030)}
\gppoint{gp mark 0}{(4.043,2.880)}
\gppoint{gp mark 0}{(4.043,2.880)}
\gppoint{gp mark 0}{(4.043,2.958)}
\gppoint{gp mark 0}{(4.043,2.880)}
\gppoint{gp mark 0}{(4.043,3.450)}
\gppoint{gp mark 0}{(4.043,3.030)}
\gppoint{gp mark 0}{(4.043,2.958)}
\gppoint{gp mark 0}{(4.043,2.958)}
\gppoint{gp mark 0}{(4.043,3.157)}
\gppoint{gp mark 0}{(4.043,2.958)}
\gppoint{gp mark 0}{(4.043,2.880)}
\gppoint{gp mark 0}{(4.043,3.450)}
\gppoint{gp mark 0}{(4.043,2.880)}
\gppoint{gp mark 0}{(4.043,2.880)}
\gppoint{gp mark 0}{(4.043,3.214)}
\gppoint{gp mark 0}{(4.043,3.214)}
\gppoint{gp mark 0}{(4.043,3.408)}
\gppoint{gp mark 0}{(4.043,3.157)}
\gppoint{gp mark 0}{(4.043,3.214)}
\gppoint{gp mark 0}{(4.043,2.958)}
\gppoint{gp mark 0}{(4.043,2.880)}
\gppoint{gp mark 0}{(4.043,3.450)}
\gppoint{gp mark 0}{(4.043,2.696)}
\gppoint{gp mark 0}{(4.043,3.214)}
\gppoint{gp mark 0}{(4.043,2.793)}
\gppoint{gp mark 0}{(4.043,2.958)}
\gppoint{gp mark 0}{(4.043,2.793)}
\gppoint{gp mark 0}{(4.043,3.317)}
\gppoint{gp mark 0}{(4.043,3.450)}
\gppoint{gp mark 0}{(4.043,3.030)}
\gppoint{gp mark 0}{(4.043,2.793)}
\gppoint{gp mark 0}{(4.043,3.450)}
\gppoint{gp mark 0}{(4.043,2.958)}
\gppoint{gp mark 0}{(4.043,2.793)}
\gppoint{gp mark 0}{(4.043,3.214)}
\gppoint{gp mark 0}{(4.043,3.450)}
\gppoint{gp mark 0}{(4.043,3.157)}
\gppoint{gp mark 0}{(4.043,2.793)}
\gppoint{gp mark 0}{(4.043,2.880)}
\gppoint{gp mark 0}{(4.043,3.096)}
\gppoint{gp mark 0}{(4.043,3.157)}
\gppoint{gp mark 0}{(4.043,3.214)}
\gppoint{gp mark 0}{(4.043,3.214)}
\gppoint{gp mark 0}{(4.043,2.880)}
\gppoint{gp mark 0}{(4.043,2.793)}
\gppoint{gp mark 0}{(4.043,2.958)}
\gppoint{gp mark 0}{(4.043,2.958)}
\gppoint{gp mark 0}{(4.043,3.214)}
\gppoint{gp mark 0}{(4.043,3.030)}
\gppoint{gp mark 0}{(4.043,2.793)}
\gppoint{gp mark 0}{(4.043,3.157)}
\gppoint{gp mark 0}{(4.043,3.157)}
\gppoint{gp mark 0}{(4.043,2.958)}
\gppoint{gp mark 0}{(4.043,3.096)}
\gppoint{gp mark 0}{(4.043,3.214)}
\gppoint{gp mark 0}{(4.043,3.030)}
\gppoint{gp mark 0}{(4.043,2.793)}
\gppoint{gp mark 0}{(4.043,3.214)}
\gppoint{gp mark 0}{(4.043,3.214)}
\gppoint{gp mark 0}{(4.043,2.958)}
\gppoint{gp mark 0}{(4.043,2.793)}
\gppoint{gp mark 0}{(4.043,2.958)}
\gppoint{gp mark 0}{(4.043,2.958)}
\gppoint{gp mark 0}{(4.043,3.157)}
\gppoint{gp mark 0}{(4.043,2.958)}
\gppoint{gp mark 0}{(4.043,3.157)}
\gppoint{gp mark 0}{(4.043,3.214)}
\gppoint{gp mark 0}{(4.043,3.030)}
\gppoint{gp mark 0}{(4.043,3.157)}
\gppoint{gp mark 0}{(4.043,3.364)}
\gppoint{gp mark 0}{(4.043,3.214)}
\gppoint{gp mark 0}{(4.043,2.696)}
\gppoint{gp mark 0}{(4.043,2.880)}
\gppoint{gp mark 0}{(4.043,3.364)}
\gppoint{gp mark 0}{(4.043,3.450)}
\gppoint{gp mark 0}{(4.043,2.958)}
\gppoint{gp mark 0}{(4.043,3.157)}
\gppoint{gp mark 0}{(4.043,3.157)}
\gppoint{gp mark 0}{(4.043,3.157)}
\gppoint{gp mark 0}{(4.043,3.450)}
\gppoint{gp mark 0}{(4.043,3.157)}
\gppoint{gp mark 0}{(4.043,2.793)}
\gppoint{gp mark 0}{(4.043,3.450)}
\gppoint{gp mark 0}{(4.043,3.317)}
\gppoint{gp mark 0}{(4.043,3.157)}
\gppoint{gp mark 0}{(4.043,3.096)}
\gppoint{gp mark 0}{(4.043,3.267)}
\gppoint{gp mark 0}{(4.043,3.157)}
\gppoint{gp mark 0}{(4.043,3.450)}
\gppoint{gp mark 0}{(4.043,3.157)}
\gppoint{gp mark 0}{(4.043,3.157)}
\gppoint{gp mark 0}{(4.043,2.586)}
\gppoint{gp mark 0}{(4.043,3.317)}
\gppoint{gp mark 0}{(4.043,3.450)}
\gppoint{gp mark 0}{(4.043,3.267)}
\gppoint{gp mark 0}{(4.043,3.450)}
\gppoint{gp mark 0}{(4.043,2.958)}
\gppoint{gp mark 0}{(4.043,3.214)}
\gppoint{gp mark 0}{(4.043,3.450)}
\gppoint{gp mark 0}{(4.043,3.030)}
\gppoint{gp mark 0}{(4.043,3.096)}
\gppoint{gp mark 0}{(4.043,3.096)}
\gppoint{gp mark 0}{(4.043,2.958)}
\gppoint{gp mark 0}{(4.043,3.096)}
\gppoint{gp mark 0}{(4.043,2.793)}
\gppoint{gp mark 0}{(4.043,3.096)}
\gppoint{gp mark 0}{(4.043,2.793)}
\gppoint{gp mark 0}{(4.043,3.317)}
\gppoint{gp mark 0}{(4.043,3.096)}
\gppoint{gp mark 0}{(4.043,2.958)}
\gppoint{gp mark 0}{(4.043,3.214)}
\gppoint{gp mark 0}{(4.043,3.317)}
\gppoint{gp mark 0}{(4.043,3.450)}
\gppoint{gp mark 0}{(4.043,2.880)}
\gppoint{gp mark 0}{(4.043,2.880)}
\gppoint{gp mark 0}{(4.043,2.793)}
\gppoint{gp mark 0}{(4.043,2.880)}
\gppoint{gp mark 0}{(4.043,2.793)}
\gppoint{gp mark 0}{(4.043,2.880)}
\gppoint{gp mark 0}{(4.043,3.096)}
\gppoint{gp mark 0}{(4.043,2.958)}
\gppoint{gp mark 0}{(4.043,3.157)}
\gppoint{gp mark 0}{(4.043,3.450)}
\gppoint{gp mark 0}{(4.043,2.958)}
\gppoint{gp mark 0}{(4.043,3.408)}
\gppoint{gp mark 0}{(4.043,2.460)}
\gppoint{gp mark 0}{(4.043,2.958)}
\gppoint{gp mark 0}{(4.043,2.880)}
\gppoint{gp mark 0}{(4.043,2.880)}
\gppoint{gp mark 0}{(4.043,2.958)}
\gppoint{gp mark 0}{(4.043,3.096)}
\gppoint{gp mark 0}{(4.043,2.958)}
\gppoint{gp mark 0}{(4.043,3.157)}
\gppoint{gp mark 0}{(4.043,3.157)}
\gppoint{gp mark 0}{(4.043,3.030)}
\gppoint{gp mark 0}{(4.043,3.364)}
\gppoint{gp mark 0}{(4.043,3.450)}
\gppoint{gp mark 0}{(4.043,2.958)}
\gppoint{gp mark 0}{(4.043,2.880)}
\gppoint{gp mark 0}{(4.043,2.958)}
\gppoint{gp mark 0}{(4.043,3.214)}
\gppoint{gp mark 0}{(4.043,3.030)}
\gppoint{gp mark 0}{(4.043,2.793)}
\gppoint{gp mark 0}{(4.043,3.030)}
\gppoint{gp mark 0}{(4.043,3.408)}
\gppoint{gp mark 0}{(4.043,3.450)}
\gppoint{gp mark 0}{(4.043,2.958)}
\gppoint{gp mark 0}{(4.043,3.030)}
\gppoint{gp mark 0}{(4.043,3.030)}
\gppoint{gp mark 0}{(4.043,2.696)}
\gppoint{gp mark 0}{(4.043,3.030)}
\gppoint{gp mark 0}{(4.043,3.157)}
\gppoint{gp mark 0}{(4.043,3.030)}
\gppoint{gp mark 0}{(4.043,2.696)}
\gppoint{gp mark 0}{(4.043,2.958)}
\gppoint{gp mark 0}{(4.043,3.030)}
\gppoint{gp mark 0}{(4.043,2.958)}
\gppoint{gp mark 0}{(4.043,2.793)}
\gppoint{gp mark 0}{(4.043,3.214)}
\gppoint{gp mark 0}{(4.043,3.214)}
\gppoint{gp mark 0}{(4.043,3.096)}
\gppoint{gp mark 0}{(4.043,2.696)}
\gppoint{gp mark 0}{(4.043,3.364)}
\gppoint{gp mark 0}{(4.043,3.157)}
\gppoint{gp mark 0}{(4.043,3.364)}
\gppoint{gp mark 0}{(4.043,3.030)}
\gppoint{gp mark 0}{(4.043,3.214)}
\gppoint{gp mark 0}{(4.043,3.214)}
\gppoint{gp mark 0}{(4.043,2.958)}
\gppoint{gp mark 0}{(4.043,3.030)}
\gppoint{gp mark 0}{(4.043,3.030)}
\gppoint{gp mark 0}{(4.043,3.317)}
\gppoint{gp mark 0}{(4.043,2.696)}
\gppoint{gp mark 0}{(4.043,2.586)}
\gppoint{gp mark 0}{(4.043,2.958)}
\gppoint{gp mark 0}{(4.043,3.214)}
\gppoint{gp mark 0}{(4.043,2.793)}
\gppoint{gp mark 0}{(4.043,2.793)}
\gppoint{gp mark 0}{(4.043,2.958)}
\gppoint{gp mark 0}{(4.043,3.096)}
\gppoint{gp mark 0}{(4.043,3.214)}
\gppoint{gp mark 0}{(4.043,2.880)}
\gppoint{gp mark 0}{(4.043,3.157)}
\gppoint{gp mark 0}{(4.043,3.267)}
\gppoint{gp mark 0}{(4.043,2.696)}
\gppoint{gp mark 0}{(4.043,2.696)}
\gppoint{gp mark 0}{(4.043,3.030)}
\gppoint{gp mark 0}{(4.043,2.958)}
\gppoint{gp mark 0}{(4.043,2.793)}
\gppoint{gp mark 0}{(4.043,3.030)}
\gppoint{gp mark 0}{(4.043,2.696)}
\gppoint{gp mark 0}{(4.043,2.793)}
\gppoint{gp mark 0}{(4.043,2.793)}
\gppoint{gp mark 0}{(4.043,2.880)}
\gppoint{gp mark 0}{(4.043,2.880)}
\gppoint{gp mark 0}{(4.043,3.214)}
\gppoint{gp mark 0}{(4.043,2.958)}
\gppoint{gp mark 0}{(4.043,3.157)}
\gppoint{gp mark 0}{(4.043,3.157)}
\gppoint{gp mark 0}{(4.043,3.317)}
\gppoint{gp mark 0}{(4.043,3.267)}
\gppoint{gp mark 0}{(4.043,2.793)}
\gppoint{gp mark 0}{(4.043,2.586)}
\gppoint{gp mark 0}{(4.043,3.096)}
\gppoint{gp mark 0}{(4.043,2.880)}
\gppoint{gp mark 0}{(4.043,2.880)}
\gppoint{gp mark 0}{(4.043,3.030)}
\gppoint{gp mark 0}{(4.043,2.793)}
\gppoint{gp mark 0}{(4.043,3.364)}
\gppoint{gp mark 0}{(4.043,3.030)}
\gppoint{gp mark 0}{(4.043,3.267)}
\gppoint{gp mark 0}{(4.043,2.880)}
\gppoint{gp mark 0}{(4.043,2.958)}
\gppoint{gp mark 0}{(4.043,2.880)}
\gppoint{gp mark 0}{(4.043,2.793)}
\gppoint{gp mark 0}{(4.043,3.030)}
\gppoint{gp mark 0}{(4.043,3.317)}
\gppoint{gp mark 0}{(4.043,3.157)}
\gppoint{gp mark 0}{(4.043,3.157)}
\gppoint{gp mark 0}{(4.043,3.157)}
\gppoint{gp mark 0}{(4.043,2.586)}
\gppoint{gp mark 0}{(4.043,3.214)}
\gppoint{gp mark 0}{(4.043,3.364)}
\gppoint{gp mark 0}{(4.043,3.030)}
\gppoint{gp mark 0}{(4.043,3.364)}
\gppoint{gp mark 0}{(4.043,3.317)}
\gppoint{gp mark 0}{(4.043,2.880)}
\gppoint{gp mark 0}{(4.043,2.958)}
\gppoint{gp mark 0}{(4.043,3.157)}
\gppoint{gp mark 0}{(4.043,2.880)}
\gppoint{gp mark 0}{(4.043,2.958)}
\gppoint{gp mark 0}{(4.043,2.880)}
\gppoint{gp mark 0}{(4.043,2.586)}
\gppoint{gp mark 0}{(4.043,3.267)}
\gppoint{gp mark 0}{(4.043,3.214)}
\gppoint{gp mark 0}{(4.043,3.096)}
\gppoint{gp mark 0}{(4.043,2.880)}
\gppoint{gp mark 0}{(4.043,3.317)}
\gppoint{gp mark 0}{(4.043,2.880)}
\gppoint{gp mark 0}{(4.043,2.696)}
\gppoint{gp mark 0}{(4.043,3.317)}
\gppoint{gp mark 0}{(4.043,3.030)}
\gppoint{gp mark 0}{(4.043,2.793)}
\gppoint{gp mark 0}{(4.043,2.958)}
\gppoint{gp mark 0}{(4.043,2.793)}
\gppoint{gp mark 0}{(4.043,2.958)}
\gppoint{gp mark 0}{(4.043,2.793)}
\gppoint{gp mark 0}{(4.043,3.157)}
\gppoint{gp mark 0}{(4.043,3.214)}
\gppoint{gp mark 0}{(4.043,2.880)}
\gppoint{gp mark 0}{(4.043,3.317)}
\gppoint{gp mark 0}{(4.043,2.793)}
\gppoint{gp mark 0}{(4.043,2.958)}
\gppoint{gp mark 0}{(4.043,3.157)}
\gppoint{gp mark 0}{(4.043,3.450)}
\gppoint{gp mark 0}{(4.043,2.793)}
\gppoint{gp mark 0}{(4.043,3.267)}
\gppoint{gp mark 0}{(4.043,2.793)}
\gppoint{gp mark 0}{(4.043,2.958)}
\gppoint{gp mark 0}{(4.043,2.793)}
\gppoint{gp mark 0}{(4.043,2.793)}
\gppoint{gp mark 0}{(4.043,2.793)}
\gppoint{gp mark 0}{(4.043,2.793)}
\gppoint{gp mark 0}{(4.043,2.880)}
\gppoint{gp mark 0}{(4.043,3.096)}
\gppoint{gp mark 0}{(4.043,3.157)}
\gppoint{gp mark 0}{(4.043,2.793)}
\gppoint{gp mark 0}{(4.043,2.793)}
\gppoint{gp mark 0}{(4.043,2.696)}
\gppoint{gp mark 0}{(4.043,2.793)}
\gppoint{gp mark 0}{(4.043,2.793)}
\gppoint{gp mark 0}{(4.043,3.030)}
\gppoint{gp mark 0}{(4.043,3.214)}
\gppoint{gp mark 0}{(4.043,3.030)}
\gppoint{gp mark 0}{(4.043,3.030)}
\gppoint{gp mark 0}{(4.043,2.793)}
\gppoint{gp mark 0}{(4.043,3.267)}
\gppoint{gp mark 0}{(4.043,3.364)}
\gppoint{gp mark 0}{(4.043,3.157)}
\gppoint{gp mark 0}{(4.043,3.096)}
\gppoint{gp mark 0}{(4.043,3.030)}
\gppoint{gp mark 0}{(4.043,3.030)}
\gppoint{gp mark 0}{(4.043,3.157)}
\gppoint{gp mark 0}{(4.043,2.793)}
\gppoint{gp mark 0}{(4.043,3.317)}
\gppoint{gp mark 0}{(4.043,2.460)}
\gppoint{gp mark 0}{(4.043,3.157)}
\gppoint{gp mark 0}{(4.043,2.958)}
\gppoint{gp mark 0}{(4.043,2.880)}
\gppoint{gp mark 0}{(4.043,3.030)}
\gppoint{gp mark 0}{(4.043,3.267)}
\gppoint{gp mark 0}{(4.043,3.030)}
\gppoint{gp mark 0}{(4.043,2.958)}
\gppoint{gp mark 0}{(4.043,3.030)}
\gppoint{gp mark 0}{(4.043,3.096)}
\gppoint{gp mark 0}{(4.043,3.157)}
\gppoint{gp mark 0}{(4.043,3.030)}
\gppoint{gp mark 0}{(4.043,3.030)}
\gppoint{gp mark 0}{(4.043,2.880)}
\gppoint{gp mark 0}{(4.043,3.317)}
\gppoint{gp mark 0}{(4.043,3.317)}
\gppoint{gp mark 0}{(4.043,2.958)}
\gppoint{gp mark 0}{(4.043,3.317)}
\gppoint{gp mark 0}{(4.043,3.157)}
\gppoint{gp mark 0}{(4.043,2.880)}
\gppoint{gp mark 0}{(4.043,3.096)}
\gppoint{gp mark 0}{(4.043,2.958)}
\gppoint{gp mark 0}{(4.043,2.958)}
\gppoint{gp mark 0}{(4.043,3.096)}
\gppoint{gp mark 0}{(4.043,3.565)}
\gppoint{gp mark 0}{(4.043,2.958)}
\gppoint{gp mark 0}{(4.043,3.565)}
\gppoint{gp mark 0}{(4.043,2.880)}
\gppoint{gp mark 0}{(4.043,3.565)}
\gppoint{gp mark 0}{(4.043,2.696)}
\gppoint{gp mark 0}{(4.043,3.096)}
\gppoint{gp mark 0}{(4.043,2.880)}
\gppoint{gp mark 0}{(4.043,3.157)}
\gppoint{gp mark 0}{(4.043,3.096)}
\gppoint{gp mark 0}{(4.043,3.267)}
\gppoint{gp mark 0}{(4.043,3.450)}
\gppoint{gp mark 0}{(4.043,3.157)}
\gppoint{gp mark 0}{(4.043,3.408)}
\gppoint{gp mark 0}{(4.043,3.214)}
\gppoint{gp mark 0}{(4.043,3.267)}
\gppoint{gp mark 0}{(4.043,3.267)}
\gppoint{gp mark 0}{(4.043,3.267)}
\gppoint{gp mark 0}{(4.043,3.267)}
\gppoint{gp mark 0}{(4.043,2.958)}
\gppoint{gp mark 0}{(4.043,3.157)}
\gppoint{gp mark 0}{(4.043,3.157)}
\gppoint{gp mark 0}{(4.043,3.157)}
\gppoint{gp mark 0}{(4.043,3.157)}
\gppoint{gp mark 0}{(4.043,3.030)}
\gppoint{gp mark 0}{(4.043,3.157)}
\gppoint{gp mark 0}{(4.043,3.030)}
\gppoint{gp mark 0}{(4.043,2.793)}
\gppoint{gp mark 0}{(4.043,2.958)}
\gppoint{gp mark 0}{(4.043,3.727)}
\gppoint{gp mark 0}{(4.043,2.880)}
\gppoint{gp mark 0}{(4.043,3.157)}
\gppoint{gp mark 0}{(4.043,2.958)}
\gppoint{gp mark 0}{(4.043,3.214)}
\gppoint{gp mark 0}{(4.043,2.958)}
\gppoint{gp mark 0}{(4.043,2.958)}
\gppoint{gp mark 0}{(4.043,2.793)}
\gppoint{gp mark 0}{(4.043,2.880)}
\gppoint{gp mark 0}{(4.043,2.958)}
\gppoint{gp mark 0}{(4.043,2.880)}
\gppoint{gp mark 0}{(4.043,2.793)}
\gppoint{gp mark 0}{(4.043,3.030)}
\gppoint{gp mark 0}{(4.043,2.793)}
\gppoint{gp mark 0}{(4.043,3.096)}
\gppoint{gp mark 0}{(4.043,2.958)}
\gppoint{gp mark 0}{(4.043,3.157)}
\gppoint{gp mark 0}{(4.043,3.030)}
\gppoint{gp mark 0}{(4.043,3.408)}
\gppoint{gp mark 0}{(4.043,3.030)}
\gppoint{gp mark 0}{(4.043,3.157)}
\gppoint{gp mark 0}{(4.043,2.958)}
\gppoint{gp mark 0}{(4.043,2.696)}
\gppoint{gp mark 0}{(4.043,3.214)}
\gppoint{gp mark 0}{(4.043,3.030)}
\gppoint{gp mark 0}{(4.043,3.157)}
\gppoint{gp mark 0}{(4.043,3.267)}
\gppoint{gp mark 0}{(4.043,3.096)}
\gppoint{gp mark 0}{(4.043,3.408)}
\gppoint{gp mark 0}{(4.043,3.030)}
\gppoint{gp mark 0}{(4.043,3.096)}
\gppoint{gp mark 0}{(4.043,3.157)}
\gppoint{gp mark 0}{(4.043,3.096)}
\gppoint{gp mark 0}{(4.043,3.267)}
\gppoint{gp mark 0}{(4.043,3.267)}
\gppoint{gp mark 0}{(4.043,3.096)}
\gppoint{gp mark 0}{(4.043,3.157)}
\gppoint{gp mark 0}{(4.043,3.267)}
\gppoint{gp mark 0}{(4.043,3.157)}
\gppoint{gp mark 0}{(4.043,3.450)}
\gppoint{gp mark 0}{(4.043,3.214)}
\gppoint{gp mark 0}{(4.043,2.793)}
\gppoint{gp mark 0}{(4.043,3.030)}
\gppoint{gp mark 0}{(4.043,3.267)}
\gppoint{gp mark 0}{(4.043,2.696)}
\gppoint{gp mark 0}{(4.043,3.267)}
\gppoint{gp mark 0}{(4.043,2.696)}
\gppoint{gp mark 0}{(4.043,2.880)}
\gppoint{gp mark 0}{(4.043,3.408)}
\gppoint{gp mark 0}{(4.043,3.157)}
\gppoint{gp mark 0}{(4.043,2.958)}
\gppoint{gp mark 0}{(4.043,3.030)}
\gppoint{gp mark 0}{(4.043,2.793)}
\gppoint{gp mark 0}{(4.043,3.096)}
\gppoint{gp mark 0}{(4.043,3.157)}
\gppoint{gp mark 0}{(4.043,2.696)}
\gppoint{gp mark 0}{(4.043,3.096)}
\gppoint{gp mark 0}{(4.043,2.880)}
\gppoint{gp mark 0}{(4.043,2.880)}
\gppoint{gp mark 0}{(4.043,3.096)}
\gppoint{gp mark 0}{(4.043,3.096)}
\gppoint{gp mark 0}{(4.043,3.096)}
\gppoint{gp mark 0}{(4.043,3.364)}
\gppoint{gp mark 0}{(4.043,2.880)}
\gppoint{gp mark 0}{(4.043,3.214)}
\gppoint{gp mark 0}{(4.043,2.793)}
\gppoint{gp mark 0}{(4.043,2.696)}
\gppoint{gp mark 0}{(4.043,3.267)}
\gppoint{gp mark 0}{(4.043,3.030)}
\gppoint{gp mark 0}{(4.043,3.030)}
\gppoint{gp mark 0}{(4.043,3.096)}
\gppoint{gp mark 0}{(4.043,3.214)}
\gppoint{gp mark 0}{(4.043,3.214)}
\gppoint{gp mark 0}{(4.043,3.157)}
\gppoint{gp mark 0}{(4.043,2.696)}
\gppoint{gp mark 0}{(4.043,2.880)}
\gppoint{gp mark 0}{(4.043,2.793)}
\gppoint{gp mark 0}{(4.043,2.793)}
\gppoint{gp mark 0}{(4.043,2.793)}
\gppoint{gp mark 0}{(4.043,3.096)}
\gppoint{gp mark 0}{(4.043,3.030)}
\gppoint{gp mark 0}{(4.043,2.958)}
\gppoint{gp mark 0}{(4.043,3.317)}
\gppoint{gp mark 0}{(4.043,2.586)}
\gppoint{gp mark 0}{(4.043,3.096)}
\gppoint{gp mark 0}{(4.043,2.696)}
\gppoint{gp mark 0}{(4.043,2.696)}
\gppoint{gp mark 0}{(4.043,3.214)}
\gppoint{gp mark 0}{(4.155,3.096)}
\gppoint{gp mark 0}{(4.155,3.157)}
\gppoint{gp mark 0}{(4.155,3.096)}
\gppoint{gp mark 0}{(4.155,2.696)}
\gppoint{gp mark 0}{(4.155,3.214)}
\gppoint{gp mark 0}{(4.155,3.157)}
\gppoint{gp mark 0}{(4.155,3.317)}
\gppoint{gp mark 0}{(4.155,3.157)}
\gppoint{gp mark 0}{(4.155,3.030)}
\gppoint{gp mark 0}{(4.155,3.157)}
\gppoint{gp mark 0}{(4.155,3.096)}
\gppoint{gp mark 0}{(4.155,2.958)}
\gppoint{gp mark 0}{(4.155,2.880)}
\gppoint{gp mark 0}{(4.155,3.157)}
\gppoint{gp mark 0}{(4.155,2.793)}
\gppoint{gp mark 0}{(4.155,3.030)}
\gppoint{gp mark 0}{(4.155,3.317)}
\gppoint{gp mark 0}{(4.155,2.958)}
\gppoint{gp mark 0}{(4.155,3.157)}
\gppoint{gp mark 0}{(4.155,2.696)}
\gppoint{gp mark 0}{(4.155,3.096)}
\gppoint{gp mark 0}{(4.155,2.793)}
\gppoint{gp mark 0}{(4.155,3.214)}
\gppoint{gp mark 0}{(4.155,3.157)}
\gppoint{gp mark 0}{(4.155,3.214)}
\gppoint{gp mark 0}{(4.155,3.157)}
\gppoint{gp mark 0}{(4.155,2.958)}
\gppoint{gp mark 0}{(4.155,3.030)}
\gppoint{gp mark 0}{(4.155,3.096)}
\gppoint{gp mark 0}{(4.155,3.214)}
\gppoint{gp mark 0}{(4.155,2.793)}
\gppoint{gp mark 0}{(4.155,3.214)}
\gppoint{gp mark 0}{(4.155,3.096)}
\gppoint{gp mark 0}{(4.155,3.030)}
\gppoint{gp mark 0}{(4.155,3.096)}
\gppoint{gp mark 0}{(4.155,3.267)}
\gppoint{gp mark 0}{(4.155,3.096)}
\gppoint{gp mark 0}{(4.155,3.157)}
\gppoint{gp mark 0}{(4.155,3.490)}
\gppoint{gp mark 0}{(4.155,2.880)}
\gppoint{gp mark 0}{(4.155,2.586)}
\gppoint{gp mark 0}{(4.155,3.157)}
\gppoint{gp mark 0}{(4.155,3.096)}
\gppoint{gp mark 0}{(4.155,3.157)}
\gppoint{gp mark 0}{(4.155,3.364)}
\gppoint{gp mark 0}{(4.155,2.880)}
\gppoint{gp mark 0}{(4.155,3.096)}
\gppoint{gp mark 0}{(4.155,3.096)}
\gppoint{gp mark 0}{(4.155,2.696)}
\gppoint{gp mark 0}{(4.155,3.096)}
\gppoint{gp mark 0}{(4.155,3.317)}
\gppoint{gp mark 0}{(4.155,2.880)}
\gppoint{gp mark 0}{(4.155,3.317)}
\gppoint{gp mark 0}{(4.155,2.880)}
\gppoint{gp mark 0}{(4.155,3.408)}
\gppoint{gp mark 0}{(4.155,3.030)}
\gppoint{gp mark 0}{(4.155,3.157)}
\gppoint{gp mark 0}{(4.155,2.586)}
\gppoint{gp mark 0}{(4.155,3.214)}
\gppoint{gp mark 0}{(4.155,2.696)}
\gppoint{gp mark 0}{(4.155,2.880)}
\gppoint{gp mark 0}{(4.155,3.214)}
\gppoint{gp mark 0}{(4.155,3.317)}
\gppoint{gp mark 0}{(4.155,3.096)}
\gppoint{gp mark 0}{(4.155,2.880)}
\gppoint{gp mark 0}{(4.155,2.460)}
\gppoint{gp mark 0}{(4.155,3.157)}
\gppoint{gp mark 0}{(4.155,3.096)}
\gppoint{gp mark 0}{(4.155,3.214)}
\gppoint{gp mark 0}{(4.155,3.267)}
\gppoint{gp mark 0}{(4.155,2.880)}
\gppoint{gp mark 0}{(4.155,2.958)}
\gppoint{gp mark 0}{(4.155,3.157)}
\gppoint{gp mark 0}{(4.155,3.267)}
\gppoint{gp mark 0}{(4.155,3.214)}
\gppoint{gp mark 0}{(4.155,3.157)}
\gppoint{gp mark 0}{(4.155,2.880)}
\gppoint{gp mark 0}{(4.155,2.586)}
\gppoint{gp mark 0}{(4.155,3.096)}
\gppoint{gp mark 0}{(4.155,2.958)}
\gppoint{gp mark 0}{(4.155,2.880)}
\gppoint{gp mark 0}{(4.155,3.096)}
\gppoint{gp mark 0}{(4.155,3.096)}
\gppoint{gp mark 0}{(4.155,3.267)}
\gppoint{gp mark 0}{(4.155,3.157)}
\gppoint{gp mark 0}{(4.155,2.793)}
\gppoint{gp mark 0}{(4.155,3.157)}
\gppoint{gp mark 0}{(4.155,3.317)}
\gppoint{gp mark 0}{(4.155,3.030)}
\gppoint{gp mark 0}{(4.155,3.030)}
\gppoint{gp mark 0}{(4.155,3.030)}
\gppoint{gp mark 0}{(4.155,3.157)}
\gppoint{gp mark 0}{(4.155,3.364)}
\gppoint{gp mark 0}{(4.155,3.157)}
\gppoint{gp mark 0}{(4.155,3.267)}
\gppoint{gp mark 0}{(4.155,3.157)}
\gppoint{gp mark 0}{(4.155,3.030)}
\gppoint{gp mark 0}{(4.155,3.529)}
\gppoint{gp mark 0}{(4.155,3.157)}
\gppoint{gp mark 0}{(4.155,3.096)}
\gppoint{gp mark 0}{(4.155,2.958)}
\gppoint{gp mark 0}{(4.155,3.317)}
\gppoint{gp mark 0}{(4.155,3.096)}
\gppoint{gp mark 0}{(4.155,3.157)}
\gppoint{gp mark 0}{(4.155,3.157)}
\gppoint{gp mark 0}{(4.155,3.157)}
\gppoint{gp mark 0}{(4.155,2.586)}
\gppoint{gp mark 0}{(4.155,3.214)}
\gppoint{gp mark 0}{(4.155,2.793)}
\gppoint{gp mark 0}{(4.155,2.793)}
\gppoint{gp mark 0}{(4.155,3.157)}
\gppoint{gp mark 0}{(4.155,3.030)}
\gppoint{gp mark 0}{(4.155,3.317)}
\gppoint{gp mark 0}{(4.155,2.880)}
\gppoint{gp mark 0}{(4.155,2.880)}
\gppoint{gp mark 0}{(4.155,3.157)}
\gppoint{gp mark 0}{(4.155,3.157)}
\gppoint{gp mark 0}{(4.155,3.214)}
\gppoint{gp mark 0}{(4.155,2.880)}
\gppoint{gp mark 0}{(4.155,2.880)}
\gppoint{gp mark 0}{(4.155,2.880)}
\gppoint{gp mark 0}{(4.155,3.214)}
\gppoint{gp mark 0}{(4.155,3.157)}
\gppoint{gp mark 0}{(4.155,3.096)}
\gppoint{gp mark 0}{(4.155,3.096)}
\gppoint{gp mark 0}{(4.155,3.030)}
\gppoint{gp mark 0}{(4.155,3.450)}
\gppoint{gp mark 0}{(4.155,2.958)}
\gppoint{gp mark 0}{(4.155,3.450)}
\gppoint{gp mark 0}{(4.155,3.408)}
\gppoint{gp mark 0}{(4.155,3.157)}
\gppoint{gp mark 0}{(4.155,2.696)}
\gppoint{gp mark 0}{(4.155,3.157)}
\gppoint{gp mark 0}{(4.155,3.030)}
\gppoint{gp mark 0}{(4.155,3.096)}
\gppoint{gp mark 0}{(4.155,3.157)}
\gppoint{gp mark 0}{(4.155,3.364)}
\gppoint{gp mark 0}{(4.155,3.030)}
\gppoint{gp mark 0}{(4.155,3.030)}
\gppoint{gp mark 0}{(4.155,3.214)}
\gppoint{gp mark 0}{(4.155,3.096)}
\gppoint{gp mark 0}{(4.155,3.157)}
\gppoint{gp mark 0}{(4.155,3.157)}
\gppoint{gp mark 0}{(4.155,3.267)}
\gppoint{gp mark 0}{(4.155,2.793)}
\gppoint{gp mark 0}{(4.155,3.317)}
\gppoint{gp mark 0}{(4.155,3.267)}
\gppoint{gp mark 0}{(4.155,2.880)}
\gppoint{gp mark 0}{(4.155,3.096)}
\gppoint{gp mark 0}{(4.155,3.157)}
\gppoint{gp mark 0}{(4.155,3.030)}
\gppoint{gp mark 0}{(4.155,3.214)}
\gppoint{gp mark 0}{(4.155,3.214)}
\gppoint{gp mark 0}{(4.155,3.157)}
\gppoint{gp mark 0}{(4.155,3.214)}
\gppoint{gp mark 0}{(4.155,3.096)}
\gppoint{gp mark 0}{(4.155,2.880)}
\gppoint{gp mark 0}{(4.155,3.317)}
\gppoint{gp mark 0}{(4.155,3.490)}
\gppoint{gp mark 0}{(4.155,2.793)}
\gppoint{gp mark 0}{(4.155,3.214)}
\gppoint{gp mark 0}{(4.155,3.529)}
\gppoint{gp mark 0}{(4.155,2.958)}
\gppoint{gp mark 0}{(4.155,3.450)}
\gppoint{gp mark 0}{(4.155,3.096)}
\gppoint{gp mark 0}{(4.155,3.450)}
\gppoint{gp mark 0}{(4.155,3.157)}
\gppoint{gp mark 0}{(4.155,2.793)}
\gppoint{gp mark 0}{(4.155,3.450)}
\gppoint{gp mark 0}{(4.155,3.030)}
\gppoint{gp mark 0}{(4.155,2.880)}
\gppoint{gp mark 0}{(4.155,2.958)}
\gppoint{gp mark 0}{(4.155,3.096)}
\gppoint{gp mark 0}{(4.155,2.586)}
\gppoint{gp mark 0}{(4.155,2.586)}
\gppoint{gp mark 0}{(4.155,2.958)}
\gppoint{gp mark 0}{(4.155,3.727)}
\gppoint{gp mark 0}{(4.155,2.696)}
\gppoint{gp mark 0}{(4.155,2.958)}
\gppoint{gp mark 0}{(4.155,3.096)}
\gppoint{gp mark 0}{(4.155,2.958)}
\gppoint{gp mark 0}{(4.155,3.214)}
\gppoint{gp mark 0}{(4.155,3.317)}
\gppoint{gp mark 0}{(4.155,3.030)}
\gppoint{gp mark 0}{(4.155,3.157)}
\gppoint{gp mark 0}{(4.155,3.214)}
\gppoint{gp mark 0}{(4.155,2.958)}
\gppoint{gp mark 0}{(4.155,3.267)}
\gppoint{gp mark 0}{(4.155,3.096)}
\gppoint{gp mark 0}{(4.155,3.267)}
\gppoint{gp mark 0}{(4.155,2.586)}
\gppoint{gp mark 0}{(4.155,3.157)}
\gppoint{gp mark 0}{(4.155,3.096)}
\gppoint{gp mark 0}{(4.155,3.030)}
\gppoint{gp mark 0}{(4.155,3.030)}
\gppoint{gp mark 0}{(4.155,3.157)}
\gppoint{gp mark 0}{(4.155,3.317)}
\gppoint{gp mark 0}{(4.155,3.157)}
\gppoint{gp mark 0}{(4.155,3.565)}
\gppoint{gp mark 0}{(4.155,3.214)}
\gppoint{gp mark 0}{(4.155,3.214)}
\gppoint{gp mark 0}{(4.155,3.317)}
\gppoint{gp mark 0}{(4.155,3.529)}
\gppoint{gp mark 0}{(4.155,3.214)}
\gppoint{gp mark 0}{(4.155,3.157)}
\gppoint{gp mark 0}{(4.155,2.793)}
\gppoint{gp mark 0}{(4.155,3.030)}
\gppoint{gp mark 0}{(4.155,2.880)}
\gppoint{gp mark 0}{(4.155,3.267)}
\gppoint{gp mark 0}{(4.155,3.214)}
\gppoint{gp mark 0}{(4.155,3.529)}
\gppoint{gp mark 0}{(4.155,3.157)}
\gppoint{gp mark 0}{(4.155,2.958)}
\gppoint{gp mark 0}{(4.155,2.880)}
\gppoint{gp mark 0}{(4.155,2.958)}
\gppoint{gp mark 0}{(4.155,3.157)}
\gppoint{gp mark 0}{(4.155,3.267)}
\gppoint{gp mark 0}{(4.155,3.096)}
\gppoint{gp mark 0}{(4.155,2.880)}
\gppoint{gp mark 0}{(4.155,3.214)}
\gppoint{gp mark 0}{(4.155,3.157)}
\gppoint{gp mark 0}{(4.155,3.697)}
\gppoint{gp mark 0}{(4.155,2.958)}
\gppoint{gp mark 0}{(4.155,3.267)}
\gppoint{gp mark 0}{(4.155,2.880)}
\gppoint{gp mark 0}{(4.155,2.958)}
\gppoint{gp mark 0}{(4.155,2.958)}
\gppoint{gp mark 0}{(4.155,2.793)}
\gppoint{gp mark 0}{(4.155,2.793)}
\gppoint{gp mark 0}{(4.155,3.450)}
\gppoint{gp mark 0}{(4.155,2.793)}
\gppoint{gp mark 0}{(4.155,3.030)}
\gppoint{gp mark 0}{(4.155,3.030)}
\gppoint{gp mark 0}{(4.155,2.958)}
\gppoint{gp mark 0}{(4.155,3.030)}
\gppoint{gp mark 0}{(4.155,2.958)}
\gppoint{gp mark 0}{(4.155,3.096)}
\gppoint{gp mark 0}{(4.155,3.096)}
\gppoint{gp mark 0}{(4.155,3.450)}
\gppoint{gp mark 0}{(4.155,3.214)}
\gppoint{gp mark 0}{(4.155,2.793)}
\gppoint{gp mark 0}{(4.155,3.096)}
\gppoint{gp mark 0}{(4.155,3.096)}
\gppoint{gp mark 0}{(4.155,2.958)}
\gppoint{gp mark 0}{(4.155,3.096)}
\gppoint{gp mark 0}{(4.155,2.958)}
\gppoint{gp mark 0}{(4.155,3.364)}
\gppoint{gp mark 0}{(4.155,2.958)}
\gppoint{gp mark 0}{(4.155,3.317)}
\gppoint{gp mark 0}{(4.155,3.096)}
\gppoint{gp mark 0}{(4.155,3.600)}
\gppoint{gp mark 0}{(4.155,3.096)}
\gppoint{gp mark 0}{(4.155,2.586)}
\gppoint{gp mark 0}{(4.155,2.793)}
\gppoint{gp mark 0}{(4.155,2.880)}
\gppoint{gp mark 0}{(4.155,3.030)}
\gppoint{gp mark 0}{(4.155,2.958)}
\gppoint{gp mark 0}{(4.155,2.958)}
\gppoint{gp mark 0}{(4.155,3.157)}
\gppoint{gp mark 0}{(4.155,2.880)}
\gppoint{gp mark 0}{(4.155,2.793)}
\gppoint{gp mark 0}{(4.155,3.214)}
\gppoint{gp mark 0}{(4.155,2.880)}
\gppoint{gp mark 0}{(4.155,3.317)}
\gppoint{gp mark 0}{(4.155,3.364)}
\gppoint{gp mark 0}{(4.155,3.157)}
\gppoint{gp mark 0}{(4.155,3.317)}
\gppoint{gp mark 0}{(4.155,3.450)}
\gppoint{gp mark 0}{(4.155,3.267)}
\gppoint{gp mark 0}{(4.155,3.096)}
\gppoint{gp mark 0}{(4.155,3.214)}
\gppoint{gp mark 0}{(4.155,3.096)}
\gppoint{gp mark 0}{(4.155,2.880)}
\gppoint{gp mark 0}{(4.155,3.157)}
\gppoint{gp mark 0}{(4.155,3.096)}
\gppoint{gp mark 0}{(4.155,3.096)}
\gppoint{gp mark 0}{(4.155,3.096)}
\gppoint{gp mark 0}{(4.155,3.096)}
\gppoint{gp mark 0}{(4.155,3.214)}
\gppoint{gp mark 0}{(4.155,2.880)}
\gppoint{gp mark 0}{(4.155,3.408)}
\gppoint{gp mark 0}{(4.155,3.096)}
\gppoint{gp mark 0}{(4.155,3.157)}
\gppoint{gp mark 0}{(4.155,3.096)}
\gppoint{gp mark 0}{(4.155,3.096)}
\gppoint{gp mark 0}{(4.155,2.880)}
\gppoint{gp mark 0}{(4.155,2.880)}
\gppoint{gp mark 0}{(4.155,3.096)}
\gppoint{gp mark 0}{(4.155,2.696)}
\gppoint{gp mark 0}{(4.155,3.267)}
\gppoint{gp mark 0}{(4.155,2.880)}
\gppoint{gp mark 0}{(4.155,2.958)}
\gppoint{gp mark 0}{(4.155,3.756)}
\gppoint{gp mark 0}{(4.155,3.096)}
\gppoint{gp mark 0}{(4.155,2.696)}
\gppoint{gp mark 0}{(4.155,3.096)}
\gppoint{gp mark 0}{(4.155,2.793)}
\gppoint{gp mark 0}{(4.155,3.096)}
\gppoint{gp mark 0}{(4.155,3.096)}
\gppoint{gp mark 0}{(4.155,2.696)}
\gppoint{gp mark 0}{(4.155,3.267)}
\gppoint{gp mark 0}{(4.155,3.157)}
\gppoint{gp mark 0}{(4.155,3.096)}
\gppoint{gp mark 0}{(4.155,3.096)}
\gppoint{gp mark 0}{(4.155,3.784)}
\gppoint{gp mark 0}{(4.155,3.096)}
\gppoint{gp mark 0}{(4.155,3.214)}
\gppoint{gp mark 0}{(4.155,3.096)}
\gppoint{gp mark 0}{(4.155,3.450)}
\gppoint{gp mark 0}{(4.155,2.880)}
\gppoint{gp mark 0}{(4.155,3.214)}
\gppoint{gp mark 0}{(4.155,3.364)}
\gppoint{gp mark 0}{(4.155,3.364)}
\gppoint{gp mark 0}{(4.155,3.096)}
\gppoint{gp mark 0}{(4.155,2.586)}
\gppoint{gp mark 0}{(4.155,2.958)}
\gppoint{gp mark 0}{(4.155,3.364)}
\gppoint{gp mark 0}{(4.155,2.958)}
\gppoint{gp mark 0}{(4.155,2.793)}
\gppoint{gp mark 0}{(4.155,3.030)}
\gppoint{gp mark 0}{(4.155,3.096)}
\gppoint{gp mark 0}{(4.155,3.157)}
\gppoint{gp mark 0}{(4.155,3.030)}
\gppoint{gp mark 0}{(4.155,3.157)}
\gppoint{gp mark 0}{(4.155,3.157)}
\gppoint{gp mark 0}{(4.155,3.408)}
\gppoint{gp mark 0}{(4.155,3.157)}
\gppoint{gp mark 0}{(4.155,2.880)}
\gppoint{gp mark 0}{(4.155,3.450)}
\gppoint{gp mark 0}{(4.155,3.030)}
\gppoint{gp mark 0}{(4.155,3.096)}
\gppoint{gp mark 0}{(4.155,2.793)}
\gppoint{gp mark 0}{(4.155,3.096)}
\gppoint{gp mark 0}{(4.155,2.793)}
\gppoint{gp mark 0}{(4.155,3.030)}
\gppoint{gp mark 0}{(4.155,2.586)}
\gppoint{gp mark 0}{(4.155,3.214)}
\gppoint{gp mark 0}{(4.155,3.214)}
\gppoint{gp mark 0}{(4.155,3.030)}
\gppoint{gp mark 0}{(4.155,3.157)}
\gppoint{gp mark 0}{(4.155,3.157)}
\gppoint{gp mark 0}{(4.155,3.096)}
\gppoint{gp mark 0}{(4.155,3.030)}
\gppoint{gp mark 0}{(4.155,3.157)}
\gppoint{gp mark 0}{(4.155,3.317)}
\gppoint{gp mark 0}{(4.155,3.157)}
\gppoint{gp mark 0}{(4.155,3.096)}
\gppoint{gp mark 0}{(4.155,2.880)}
\gppoint{gp mark 0}{(4.155,3.096)}
\gppoint{gp mark 0}{(4.155,3.157)}
\gppoint{gp mark 0}{(4.155,3.157)}
\gppoint{gp mark 0}{(4.155,2.696)}
\gppoint{gp mark 0}{(4.155,3.096)}
\gppoint{gp mark 0}{(4.155,3.096)}
\gppoint{gp mark 0}{(4.155,2.696)}
\gppoint{gp mark 0}{(4.155,2.793)}
\gppoint{gp mark 0}{(4.155,3.096)}
\gppoint{gp mark 0}{(4.155,3.096)}
\gppoint{gp mark 0}{(4.155,2.958)}
\gppoint{gp mark 0}{(4.155,3.450)}
\gppoint{gp mark 0}{(4.155,2.958)}
\gppoint{gp mark 0}{(4.155,2.793)}
\gppoint{gp mark 0}{(4.155,2.958)}
\gppoint{gp mark 0}{(4.155,3.096)}
\gppoint{gp mark 0}{(4.155,3.096)}
\gppoint{gp mark 0}{(4.155,2.958)}
\gppoint{gp mark 0}{(4.155,2.793)}
\gppoint{gp mark 0}{(4.155,3.317)}
\gppoint{gp mark 0}{(4.155,3.096)}
\gppoint{gp mark 0}{(4.155,2.793)}
\gppoint{gp mark 0}{(4.155,2.958)}
\gppoint{gp mark 0}{(4.155,2.793)}
\gppoint{gp mark 0}{(4.155,2.880)}
\gppoint{gp mark 0}{(4.155,3.096)}
\gppoint{gp mark 0}{(4.155,2.696)}
\gppoint{gp mark 0}{(4.155,3.214)}
\gppoint{gp mark 0}{(4.155,3.157)}
\gppoint{gp mark 0}{(4.155,2.958)}
\gppoint{gp mark 0}{(4.155,3.030)}
\gppoint{gp mark 0}{(4.155,3.096)}
\gppoint{gp mark 0}{(4.155,2.880)}
\gppoint{gp mark 0}{(4.155,3.096)}
\gppoint{gp mark 0}{(4.155,3.157)}
\gppoint{gp mark 0}{(4.155,3.267)}
\gppoint{gp mark 0}{(4.155,3.214)}
\gppoint{gp mark 0}{(4.155,3.267)}
\gppoint{gp mark 0}{(4.155,3.157)}
\gppoint{gp mark 0}{(4.155,3.096)}
\gppoint{gp mark 0}{(4.155,2.958)}
\gppoint{gp mark 0}{(4.155,3.364)}
\gppoint{gp mark 0}{(4.155,3.364)}
\gppoint{gp mark 0}{(4.155,3.096)}
\gppoint{gp mark 0}{(4.155,2.880)}
\gppoint{gp mark 0}{(4.155,3.096)}
\gppoint{gp mark 0}{(4.155,2.880)}
\gppoint{gp mark 0}{(4.155,3.267)}
\gppoint{gp mark 0}{(4.155,3.267)}
\gppoint{gp mark 0}{(4.155,2.793)}
\gppoint{gp mark 0}{(4.155,3.157)}
\gppoint{gp mark 0}{(4.155,3.157)}
\gppoint{gp mark 0}{(4.155,3.096)}
\gppoint{gp mark 0}{(4.155,3.157)}
\gppoint{gp mark 0}{(4.155,2.793)}
\gppoint{gp mark 0}{(4.155,2.880)}
\gppoint{gp mark 0}{(4.155,3.157)}
\gppoint{gp mark 0}{(4.155,2.586)}
\gppoint{gp mark 0}{(4.155,2.586)}
\gppoint{gp mark 0}{(4.155,3.157)}
\gppoint{gp mark 0}{(4.155,3.096)}
\gppoint{gp mark 0}{(4.155,3.157)}
\gppoint{gp mark 0}{(4.155,2.880)}
\gppoint{gp mark 0}{(4.155,3.096)}
\gppoint{gp mark 0}{(4.155,3.267)}
\gppoint{gp mark 0}{(4.155,3.096)}
\gppoint{gp mark 0}{(4.155,2.793)}
\gppoint{gp mark 0}{(4.155,2.696)}
\gppoint{gp mark 0}{(4.155,3.267)}
\gppoint{gp mark 0}{(4.155,3.096)}
\gppoint{gp mark 0}{(4.155,3.267)}
\gppoint{gp mark 0}{(4.155,3.096)}
\gppoint{gp mark 0}{(4.155,3.157)}
\gppoint{gp mark 0}{(4.155,3.030)}
\gppoint{gp mark 0}{(4.155,3.267)}
\gppoint{gp mark 0}{(4.155,3.096)}
\gppoint{gp mark 0}{(4.155,3.267)}
\gppoint{gp mark 0}{(4.155,3.267)}
\gppoint{gp mark 0}{(4.155,3.267)}
\gppoint{gp mark 0}{(4.155,3.096)}
\gppoint{gp mark 0}{(4.155,3.030)}
\gppoint{gp mark 0}{(4.155,2.958)}
\gppoint{gp mark 0}{(4.155,2.793)}
\gppoint{gp mark 0}{(4.155,2.958)}
\gppoint{gp mark 0}{(4.155,3.157)}
\gppoint{gp mark 0}{(4.155,2.880)}
\gppoint{gp mark 0}{(4.155,3.267)}
\gppoint{gp mark 0}{(4.155,2.958)}
\gppoint{gp mark 0}{(4.155,3.267)}
\gppoint{gp mark 0}{(4.155,3.096)}
\gppoint{gp mark 0}{(4.155,3.267)}
\gppoint{gp mark 0}{(4.155,2.880)}
\gppoint{gp mark 0}{(4.155,3.030)}
\gppoint{gp mark 0}{(4.155,3.267)}
\gppoint{gp mark 0}{(4.155,3.267)}
\gppoint{gp mark 0}{(4.155,3.267)}
\gppoint{gp mark 0}{(4.155,3.157)}
\gppoint{gp mark 0}{(4.155,3.267)}
\gppoint{gp mark 0}{(4.155,3.096)}
\gppoint{gp mark 0}{(4.155,3.267)}
\gppoint{gp mark 0}{(4.155,3.096)}
\gppoint{gp mark 0}{(4.155,3.267)}
\gppoint{gp mark 0}{(4.155,3.600)}
\gppoint{gp mark 0}{(4.155,3.157)}
\gppoint{gp mark 0}{(4.155,3.030)}
\gppoint{gp mark 0}{(4.155,3.267)}
\gppoint{gp mark 0}{(4.155,3.267)}
\gppoint{gp mark 0}{(4.155,3.030)}
\gppoint{gp mark 0}{(4.155,3.317)}
\gppoint{gp mark 0}{(4.155,2.880)}
\gppoint{gp mark 0}{(4.155,3.450)}
\gppoint{gp mark 0}{(4.155,3.030)}
\gppoint{gp mark 0}{(4.155,3.096)}
\gppoint{gp mark 0}{(4.155,3.096)}
\gppoint{gp mark 0}{(4.155,3.030)}
\gppoint{gp mark 0}{(4.155,2.958)}
\gppoint{gp mark 0}{(4.155,2.793)}
\gppoint{gp mark 0}{(4.155,2.793)}
\gppoint{gp mark 0}{(4.155,3.267)}
\gppoint{gp mark 0}{(4.155,3.267)}
\gppoint{gp mark 0}{(4.155,3.157)}
\gppoint{gp mark 0}{(4.155,3.157)}
\gppoint{gp mark 0}{(4.155,3.490)}
\gppoint{gp mark 0}{(4.155,3.157)}
\gppoint{gp mark 0}{(4.155,3.030)}
\gppoint{gp mark 0}{(4.155,2.880)}
\gppoint{gp mark 0}{(4.155,3.096)}
\gppoint{gp mark 0}{(4.155,3.096)}
\gppoint{gp mark 0}{(4.155,2.958)}
\gppoint{gp mark 0}{(4.155,3.096)}
\gppoint{gp mark 0}{(4.155,3.030)}
\gppoint{gp mark 0}{(4.155,2.880)}
\gppoint{gp mark 0}{(4.155,3.157)}
\gppoint{gp mark 0}{(4.155,2.696)}
\gppoint{gp mark 0}{(4.155,3.096)}
\gppoint{gp mark 0}{(4.155,3.096)}
\gppoint{gp mark 0}{(4.155,3.096)}
\gppoint{gp mark 0}{(4.155,3.157)}
\gppoint{gp mark 0}{(4.155,3.096)}
\gppoint{gp mark 0}{(4.155,3.030)}
\gppoint{gp mark 0}{(4.155,3.096)}
\gppoint{gp mark 0}{(4.155,2.460)}
\gppoint{gp mark 0}{(4.155,3.096)}
\gppoint{gp mark 0}{(4.155,3.096)}
\gppoint{gp mark 0}{(4.155,3.096)}
\gppoint{gp mark 0}{(4.155,3.157)}
\gppoint{gp mark 0}{(4.155,2.793)}
\gppoint{gp mark 0}{(4.155,3.096)}
\gppoint{gp mark 0}{(4.155,3.030)}
\gppoint{gp mark 0}{(4.155,3.030)}
\gppoint{gp mark 0}{(4.155,2.586)}
\gppoint{gp mark 0}{(4.155,3.030)}
\gppoint{gp mark 0}{(4.155,2.586)}
\gppoint{gp mark 0}{(4.155,3.096)}
\gppoint{gp mark 0}{(4.155,2.793)}
\gppoint{gp mark 0}{(4.155,3.096)}
\gppoint{gp mark 0}{(4.155,3.096)}
\gppoint{gp mark 0}{(4.155,3.096)}
\gppoint{gp mark 0}{(4.155,3.157)}
\gppoint{gp mark 0}{(4.155,3.364)}
\gppoint{gp mark 0}{(4.155,3.214)}
\gppoint{gp mark 0}{(4.155,3.157)}
\gppoint{gp mark 0}{(4.155,3.157)}
\gppoint{gp mark 0}{(4.155,3.096)}
\gppoint{gp mark 0}{(4.155,3.317)}
\gppoint{gp mark 0}{(4.155,2.793)}
\gppoint{gp mark 0}{(4.155,3.267)}
\gppoint{gp mark 0}{(4.155,3.450)}
\gppoint{gp mark 0}{(4.155,3.267)}
\gppoint{gp mark 0}{(4.155,3.096)}
\gppoint{gp mark 0}{(4.155,2.958)}
\gppoint{gp mark 0}{(4.155,2.586)}
\gppoint{gp mark 0}{(4.155,3.267)}
\gppoint{gp mark 0}{(4.155,2.880)}
\gppoint{gp mark 0}{(4.155,2.958)}
\gppoint{gp mark 0}{(4.155,2.880)}
\gppoint{gp mark 0}{(4.258,3.096)}
\gppoint{gp mark 0}{(4.258,3.157)}
\gppoint{gp mark 0}{(4.258,3.030)}
\gppoint{gp mark 0}{(4.258,3.408)}
\gppoint{gp mark 0}{(4.258,3.697)}
\gppoint{gp mark 0}{(4.258,3.030)}
\gppoint{gp mark 0}{(4.258,3.096)}
\gppoint{gp mark 0}{(4.258,2.696)}
\gppoint{gp mark 0}{(4.258,2.880)}
\gppoint{gp mark 0}{(4.258,3.030)}
\gppoint{gp mark 0}{(4.258,3.030)}
\gppoint{gp mark 0}{(4.258,3.030)}
\gppoint{gp mark 0}{(4.258,3.267)}
\gppoint{gp mark 0}{(4.258,3.157)}
\gppoint{gp mark 0}{(4.258,3.214)}
\gppoint{gp mark 0}{(4.258,3.600)}
\gppoint{gp mark 0}{(4.258,2.880)}
\gppoint{gp mark 0}{(4.258,3.096)}
\gppoint{gp mark 0}{(4.258,3.030)}
\gppoint{gp mark 0}{(4.258,3.364)}
\gppoint{gp mark 0}{(4.258,3.364)}
\gppoint{gp mark 0}{(4.258,3.157)}
\gppoint{gp mark 0}{(4.258,3.157)}
\gppoint{gp mark 0}{(4.258,3.157)}
\gppoint{gp mark 0}{(4.258,2.880)}
\gppoint{gp mark 0}{(4.258,3.267)}
\gppoint{gp mark 0}{(4.258,3.267)}
\gppoint{gp mark 0}{(4.258,3.600)}
\gppoint{gp mark 0}{(4.258,3.096)}
\gppoint{gp mark 0}{(4.258,3.157)}
\gppoint{gp mark 0}{(4.258,3.267)}
\gppoint{gp mark 0}{(4.258,2.793)}
\gppoint{gp mark 0}{(4.258,2.958)}
\gppoint{gp mark 0}{(4.258,2.793)}
\gppoint{gp mark 0}{(4.258,2.696)}
\gppoint{gp mark 0}{(4.258,2.696)}
\gppoint{gp mark 0}{(4.258,3.364)}
\gppoint{gp mark 0}{(4.258,3.096)}
\gppoint{gp mark 0}{(4.258,3.697)}
\gppoint{gp mark 0}{(4.258,3.364)}
\gppoint{gp mark 0}{(4.258,3.096)}
\gppoint{gp mark 0}{(4.258,3.214)}
\gppoint{gp mark 0}{(4.258,3.214)}
\gppoint{gp mark 0}{(4.258,3.317)}
\gppoint{gp mark 0}{(4.258,3.317)}
\gppoint{gp mark 0}{(4.258,3.214)}
\gppoint{gp mark 0}{(4.258,3.214)}
\gppoint{gp mark 0}{(4.258,3.157)}
\gppoint{gp mark 0}{(4.258,2.880)}
\gppoint{gp mark 0}{(4.258,2.460)}
\gppoint{gp mark 0}{(4.258,3.450)}
\gppoint{gp mark 0}{(4.258,3.600)}
\gppoint{gp mark 0}{(4.258,3.030)}
\gppoint{gp mark 0}{(4.258,3.317)}
\gppoint{gp mark 0}{(4.258,3.030)}
\gppoint{gp mark 0}{(4.258,2.793)}
\gppoint{gp mark 0}{(4.258,3.408)}
\gppoint{gp mark 0}{(4.258,2.793)}
\gppoint{gp mark 0}{(4.258,3.157)}
\gppoint{gp mark 0}{(4.258,2.793)}
\gppoint{gp mark 0}{(4.258,3.214)}
\gppoint{gp mark 0}{(4.258,2.880)}
\gppoint{gp mark 0}{(4.258,2.958)}
\gppoint{gp mark 0}{(4.258,3.267)}
\gppoint{gp mark 0}{(4.258,2.958)}
\gppoint{gp mark 0}{(4.258,3.267)}
\gppoint{gp mark 0}{(4.258,3.096)}
\gppoint{gp mark 0}{(4.258,3.214)}
\gppoint{gp mark 0}{(4.258,3.408)}
\gppoint{gp mark 0}{(4.258,3.096)}
\gppoint{gp mark 0}{(4.258,3.267)}
\gppoint{gp mark 0}{(4.258,3.267)}
\gppoint{gp mark 0}{(4.258,3.096)}
\gppoint{gp mark 0}{(4.258,3.096)}
\gppoint{gp mark 0}{(4.258,3.157)}
\gppoint{gp mark 0}{(4.258,3.408)}
\gppoint{gp mark 0}{(4.258,3.096)}
\gppoint{gp mark 0}{(4.258,3.096)}
\gppoint{gp mark 0}{(4.258,3.408)}
\gppoint{gp mark 0}{(4.258,3.214)}
\gppoint{gp mark 0}{(4.258,3.317)}
\gppoint{gp mark 0}{(4.258,2.793)}
\gppoint{gp mark 0}{(4.258,2.958)}
\gppoint{gp mark 0}{(4.258,3.267)}
\gppoint{gp mark 0}{(4.258,2.880)}
\gppoint{gp mark 0}{(4.258,2.696)}
\gppoint{gp mark 0}{(4.258,3.408)}
\gppoint{gp mark 0}{(4.258,3.157)}
\gppoint{gp mark 0}{(4.258,3.157)}
\gppoint{gp mark 0}{(4.258,3.214)}
\gppoint{gp mark 0}{(4.258,2.696)}
\gppoint{gp mark 0}{(4.258,3.529)}
\gppoint{gp mark 0}{(4.258,3.317)}
\gppoint{gp mark 0}{(4.258,2.696)}
\gppoint{gp mark 0}{(4.258,3.214)}
\gppoint{gp mark 0}{(4.258,3.096)}
\gppoint{gp mark 0}{(4.258,3.157)}
\gppoint{gp mark 0}{(4.258,3.408)}
\gppoint{gp mark 0}{(4.258,3.096)}
\gppoint{gp mark 0}{(4.258,3.408)}
\gppoint{gp mark 0}{(4.258,3.267)}
\gppoint{gp mark 0}{(4.258,3.317)}
\gppoint{gp mark 0}{(4.258,2.696)}
\gppoint{gp mark 0}{(4.258,3.490)}
\gppoint{gp mark 0}{(4.258,2.958)}
\gppoint{gp mark 0}{(4.258,3.157)}
\gppoint{gp mark 0}{(4.258,3.565)}
\gppoint{gp mark 0}{(4.258,3.157)}
\gppoint{gp mark 0}{(4.258,3.214)}
\gppoint{gp mark 0}{(4.258,2.958)}
\gppoint{gp mark 0}{(4.258,3.157)}
\gppoint{gp mark 0}{(4.258,3.267)}
\gppoint{gp mark 0}{(4.258,2.586)}
\gppoint{gp mark 0}{(4.258,3.157)}
\gppoint{gp mark 0}{(4.258,3.408)}
\gppoint{gp mark 0}{(4.258,3.096)}
\gppoint{gp mark 0}{(4.258,3.267)}
\gppoint{gp mark 0}{(4.258,3.096)}
\gppoint{gp mark 0}{(4.258,2.696)}
\gppoint{gp mark 0}{(4.258,3.408)}
\gppoint{gp mark 0}{(4.258,3.529)}
\gppoint{gp mark 0}{(4.258,3.408)}
\gppoint{gp mark 0}{(4.258,3.157)}
\gppoint{gp mark 0}{(4.258,3.096)}
\gppoint{gp mark 0}{(4.258,3.267)}
\gppoint{gp mark 0}{(4.258,2.958)}
\gppoint{gp mark 0}{(4.258,3.490)}
\gppoint{gp mark 0}{(4.258,2.880)}
\gppoint{gp mark 0}{(4.258,2.880)}
\gppoint{gp mark 0}{(4.258,2.958)}
\gppoint{gp mark 0}{(4.258,3.267)}
\gppoint{gp mark 0}{(4.258,3.267)}
\gppoint{gp mark 0}{(4.258,2.958)}
\gppoint{gp mark 0}{(4.258,3.267)}
\gppoint{gp mark 0}{(4.258,2.586)}
\gppoint{gp mark 0}{(4.258,2.958)}
\gppoint{gp mark 0}{(4.258,3.157)}
\gppoint{gp mark 0}{(4.258,3.157)}
\gppoint{gp mark 0}{(4.258,2.793)}
\gppoint{gp mark 0}{(4.258,3.408)}
\gppoint{gp mark 0}{(4.258,2.793)}
\gppoint{gp mark 0}{(4.258,3.096)}
\gppoint{gp mark 0}{(4.258,3.565)}
\gppoint{gp mark 0}{(4.258,3.096)}
\gppoint{gp mark 0}{(4.258,3.634)}
\gppoint{gp mark 0}{(4.258,2.880)}
\gppoint{gp mark 0}{(4.258,3.214)}
\gppoint{gp mark 0}{(4.258,3.408)}
\gppoint{gp mark 0}{(4.258,3.267)}
\gppoint{gp mark 0}{(4.258,3.096)}
\gppoint{gp mark 0}{(4.258,3.697)}
\gppoint{gp mark 0}{(4.258,3.157)}
\gppoint{gp mark 0}{(4.258,3.214)}
\gppoint{gp mark 0}{(4.258,3.157)}
\gppoint{gp mark 0}{(4.258,3.214)}
\gppoint{gp mark 0}{(4.258,3.317)}
\gppoint{gp mark 0}{(4.258,3.267)}
\gppoint{gp mark 0}{(4.258,3.267)}
\gppoint{gp mark 0}{(4.258,3.214)}
\gppoint{gp mark 0}{(4.258,3.267)}
\gppoint{gp mark 0}{(4.258,3.157)}
\gppoint{gp mark 0}{(4.258,3.157)}
\gppoint{gp mark 0}{(4.258,3.214)}
\gppoint{gp mark 0}{(4.258,2.880)}
\gppoint{gp mark 0}{(4.258,3.157)}
\gppoint{gp mark 0}{(4.258,3.157)}
\gppoint{gp mark 0}{(4.258,3.408)}
\gppoint{gp mark 0}{(4.258,3.214)}
\gppoint{gp mark 0}{(4.258,3.634)}
\gppoint{gp mark 0}{(4.258,3.317)}
\gppoint{gp mark 0}{(4.258,3.317)}
\gppoint{gp mark 0}{(4.258,3.267)}
\gppoint{gp mark 0}{(4.258,3.157)}
\gppoint{gp mark 0}{(4.258,3.157)}
\gppoint{gp mark 0}{(4.258,3.157)}
\gppoint{gp mark 0}{(4.258,3.157)}
\gppoint{gp mark 0}{(4.258,3.364)}
\gppoint{gp mark 0}{(4.258,3.364)}
\gppoint{gp mark 0}{(4.258,3.214)}
\gppoint{gp mark 0}{(4.258,3.408)}
\gppoint{gp mark 0}{(4.258,3.214)}
\gppoint{gp mark 0}{(4.258,3.096)}
\gppoint{gp mark 0}{(4.258,3.157)}
\gppoint{gp mark 0}{(4.258,2.793)}
\gppoint{gp mark 0}{(4.258,3.157)}
\gppoint{gp mark 0}{(4.258,3.214)}
\gppoint{gp mark 0}{(4.258,3.157)}
\gppoint{gp mark 0}{(4.258,3.666)}
\gppoint{gp mark 0}{(4.258,3.096)}
\gppoint{gp mark 0}{(4.258,3.364)}
\gppoint{gp mark 0}{(4.258,3.096)}
\gppoint{gp mark 0}{(4.258,2.958)}
\gppoint{gp mark 0}{(4.258,3.364)}
\gppoint{gp mark 0}{(4.258,3.408)}
\gppoint{gp mark 0}{(4.258,3.408)}
\gppoint{gp mark 0}{(4.258,3.450)}
\gppoint{gp mark 0}{(4.258,3.214)}
\gppoint{gp mark 0}{(4.258,2.793)}
\gppoint{gp mark 0}{(4.258,3.450)}
\gppoint{gp mark 0}{(4.258,3.030)}
\gppoint{gp mark 0}{(4.258,3.030)}
\gppoint{gp mark 0}{(4.258,3.364)}
\gppoint{gp mark 0}{(4.258,3.030)}
\gppoint{gp mark 0}{(4.258,3.030)}
\gppoint{gp mark 0}{(4.258,3.030)}
\gppoint{gp mark 0}{(4.258,3.600)}
\gppoint{gp mark 0}{(4.258,3.408)}
\gppoint{gp mark 0}{(4.258,3.408)}
\gppoint{gp mark 0}{(4.258,3.030)}
\gppoint{gp mark 0}{(4.258,2.793)}
\gppoint{gp mark 0}{(4.258,3.096)}
\gppoint{gp mark 0}{(4.258,3.267)}
\gppoint{gp mark 0}{(4.258,3.317)}
\gppoint{gp mark 0}{(4.258,2.793)}
\gppoint{gp mark 0}{(4.258,3.096)}
\gppoint{gp mark 0}{(4.258,3.214)}
\gppoint{gp mark 0}{(4.258,2.793)}
\gppoint{gp mark 0}{(4.258,2.958)}
\gppoint{gp mark 0}{(4.258,2.880)}
\gppoint{gp mark 0}{(4.258,3.157)}
\gppoint{gp mark 0}{(4.258,2.958)}
\gppoint{gp mark 0}{(4.258,2.958)}
\gppoint{gp mark 0}{(4.258,3.408)}
\gppoint{gp mark 0}{(4.258,3.408)}
\gppoint{gp mark 0}{(4.258,2.586)}
\gppoint{gp mark 0}{(4.258,2.793)}
\gppoint{gp mark 0}{(4.258,2.880)}
\gppoint{gp mark 0}{(4.258,3.317)}
\gppoint{gp mark 0}{(4.258,3.317)}
\gppoint{gp mark 0}{(4.258,3.317)}
\gppoint{gp mark 0}{(4.258,3.408)}
\gppoint{gp mark 0}{(4.258,2.958)}
\gppoint{gp mark 0}{(4.258,3.317)}
\gppoint{gp mark 0}{(4.258,3.317)}
\gppoint{gp mark 0}{(4.258,3.317)}
\gppoint{gp mark 0}{(4.258,3.030)}
\gppoint{gp mark 0}{(4.258,3.408)}
\gppoint{gp mark 0}{(4.258,3.317)}
\gppoint{gp mark 0}{(4.258,3.214)}
\gppoint{gp mark 0}{(4.258,2.793)}
\gppoint{gp mark 0}{(4.258,3.317)}
\gppoint{gp mark 0}{(4.258,3.267)}
\gppoint{gp mark 0}{(4.258,3.214)}
\gppoint{gp mark 0}{(4.258,3.317)}
\gppoint{gp mark 0}{(4.258,3.317)}
\gppoint{gp mark 0}{(4.258,3.030)}
\gppoint{gp mark 0}{(4.258,3.267)}
\gppoint{gp mark 0}{(4.258,3.408)}
\gppoint{gp mark 0}{(4.258,3.317)}
\gppoint{gp mark 0}{(4.258,2.958)}
\gppoint{gp mark 0}{(4.258,2.793)}
\gppoint{gp mark 0}{(4.258,3.317)}
\gppoint{gp mark 0}{(4.258,3.490)}
\gppoint{gp mark 0}{(4.258,3.408)}
\gppoint{gp mark 0}{(4.258,3.096)}
\gppoint{gp mark 0}{(4.258,3.267)}
\gppoint{gp mark 0}{(4.258,2.793)}
\gppoint{gp mark 0}{(4.258,2.793)}
\gppoint{gp mark 0}{(4.258,3.408)}
\gppoint{gp mark 0}{(4.258,3.157)}
\gppoint{gp mark 0}{(4.258,3.267)}
\gppoint{gp mark 0}{(4.258,3.030)}
\gppoint{gp mark 0}{(4.258,2.958)}
\gppoint{gp mark 0}{(4.258,3.267)}
\gppoint{gp mark 0}{(4.258,3.030)}
\gppoint{gp mark 0}{(4.258,3.267)}
\gppoint{gp mark 0}{(4.258,3.408)}
\gppoint{gp mark 0}{(4.258,3.030)}
\gppoint{gp mark 0}{(4.258,3.030)}
\gppoint{gp mark 0}{(4.258,3.450)}
\gppoint{gp mark 0}{(4.258,3.267)}
\gppoint{gp mark 0}{(4.258,3.450)}
\gppoint{gp mark 0}{(4.258,3.030)}
\gppoint{gp mark 0}{(4.258,3.450)}
\gppoint{gp mark 0}{(4.258,3.450)}
\gppoint{gp mark 0}{(4.258,3.317)}
\gppoint{gp mark 0}{(4.258,3.157)}
\gppoint{gp mark 0}{(4.258,2.696)}
\gppoint{gp mark 0}{(4.258,3.450)}
\gppoint{gp mark 0}{(4.258,3.450)}
\gppoint{gp mark 0}{(4.258,3.450)}
\gppoint{gp mark 0}{(4.258,3.450)}
\gppoint{gp mark 0}{(4.258,3.408)}
\gppoint{gp mark 0}{(4.258,3.214)}
\gppoint{gp mark 0}{(4.258,3.030)}
\gppoint{gp mark 0}{(4.258,3.030)}
\gppoint{gp mark 0}{(4.258,3.214)}
\gppoint{gp mark 0}{(4.258,3.529)}
\gppoint{gp mark 0}{(4.258,3.408)}
\gppoint{gp mark 0}{(4.258,3.157)}
\gppoint{gp mark 0}{(4.258,2.958)}
\gppoint{gp mark 0}{(4.258,3.157)}
\gppoint{gp mark 0}{(4.258,3.157)}
\gppoint{gp mark 0}{(4.258,3.096)}
\gppoint{gp mark 0}{(4.258,3.364)}
\gppoint{gp mark 0}{(4.258,3.214)}
\gppoint{gp mark 0}{(4.258,3.214)}
\gppoint{gp mark 0}{(4.258,3.157)}
\gppoint{gp mark 0}{(4.258,3.030)}
\gppoint{gp mark 0}{(4.258,3.214)}
\gppoint{gp mark 0}{(4.258,3.030)}
\gppoint{gp mark 0}{(4.258,3.214)}
\gppoint{gp mark 0}{(4.258,3.030)}
\gppoint{gp mark 0}{(4.258,3.157)}
\gppoint{gp mark 0}{(4.258,3.408)}
\gppoint{gp mark 0}{(4.258,3.214)}
\gppoint{gp mark 0}{(4.258,3.408)}
\gppoint{gp mark 0}{(4.258,3.267)}
\gppoint{gp mark 0}{(4.258,2.958)}
\gppoint{gp mark 0}{(4.258,3.096)}
\gppoint{gp mark 0}{(4.258,3.408)}
\gppoint{gp mark 0}{(4.258,3.408)}
\gppoint{gp mark 0}{(4.258,2.793)}
\gppoint{gp mark 0}{(4.258,3.267)}
\gppoint{gp mark 0}{(4.258,3.267)}
\gppoint{gp mark 0}{(4.258,3.408)}
\gppoint{gp mark 0}{(4.258,3.317)}
\gppoint{gp mark 0}{(4.258,3.030)}
\gppoint{gp mark 0}{(4.258,3.267)}
\gppoint{gp mark 0}{(4.258,3.267)}
\gppoint{gp mark 0}{(4.258,3.096)}
\gppoint{gp mark 0}{(4.258,3.408)}
\gppoint{gp mark 0}{(4.258,3.490)}
\gppoint{gp mark 0}{(4.258,2.958)}
\gppoint{gp mark 0}{(4.258,3.096)}
\gppoint{gp mark 0}{(4.258,3.096)}
\gppoint{gp mark 0}{(4.258,3.096)}
\gppoint{gp mark 0}{(4.258,2.958)}
\gppoint{gp mark 0}{(4.258,3.096)}
\gppoint{gp mark 0}{(4.258,3.030)}
\gppoint{gp mark 0}{(4.258,3.317)}
\gppoint{gp mark 0}{(4.258,3.157)}
\gppoint{gp mark 0}{(4.258,3.317)}
\gppoint{gp mark 0}{(4.258,3.096)}
\gppoint{gp mark 0}{(4.258,2.793)}
\gppoint{gp mark 0}{(4.258,2.958)}
\gppoint{gp mark 0}{(4.258,2.958)}
\gppoint{gp mark 0}{(4.258,3.490)}
\gppoint{gp mark 0}{(4.258,3.267)}
\gppoint{gp mark 0}{(4.258,3.267)}
\gppoint{gp mark 0}{(4.258,3.408)}
\gppoint{gp mark 0}{(4.258,3.096)}
\gppoint{gp mark 0}{(4.258,3.408)}
\gppoint{gp mark 0}{(4.258,2.586)}
\gppoint{gp mark 0}{(4.258,2.880)}
\gppoint{gp mark 0}{(4.258,3.490)}
\gppoint{gp mark 0}{(4.258,2.880)}
\gppoint{gp mark 0}{(4.258,3.317)}
\gppoint{gp mark 0}{(4.258,3.214)}
\gppoint{gp mark 0}{(4.258,3.096)}
\gppoint{gp mark 0}{(4.258,3.096)}
\gppoint{gp mark 0}{(4.258,3.450)}
\gppoint{gp mark 0}{(4.258,3.096)}
\gppoint{gp mark 0}{(4.258,3.214)}
\gppoint{gp mark 0}{(4.258,3.364)}
\gppoint{gp mark 0}{(4.258,3.096)}
\gppoint{gp mark 0}{(4.258,2.958)}
\gppoint{gp mark 0}{(4.258,3.214)}
\gppoint{gp mark 0}{(4.258,2.793)}
\gppoint{gp mark 0}{(4.258,3.267)}
\gppoint{gp mark 0}{(4.258,3.214)}
\gppoint{gp mark 0}{(4.258,3.214)}
\gppoint{gp mark 0}{(4.258,3.364)}
\gppoint{gp mark 0}{(4.258,3.096)}
\gppoint{gp mark 0}{(4.258,3.267)}
\gppoint{gp mark 0}{(4.258,3.267)}
\gppoint{gp mark 0}{(4.258,3.408)}
\gppoint{gp mark 0}{(4.258,3.096)}
\gppoint{gp mark 0}{(4.258,3.408)}
\gppoint{gp mark 0}{(4.258,3.096)}
\gppoint{gp mark 0}{(4.258,3.096)}
\gppoint{gp mark 0}{(4.258,3.096)}
\gppoint{gp mark 0}{(4.258,3.096)}
\gppoint{gp mark 0}{(4.258,3.364)}
\gppoint{gp mark 0}{(4.258,3.364)}
\gppoint{gp mark 0}{(4.258,3.490)}
\gppoint{gp mark 0}{(4.258,3.364)}
\gppoint{gp mark 0}{(4.258,3.490)}
\gppoint{gp mark 0}{(4.258,3.490)}
\gppoint{gp mark 0}{(4.258,3.030)}
\gppoint{gp mark 0}{(4.258,3.214)}
\gppoint{gp mark 0}{(4.258,3.157)}
\gppoint{gp mark 0}{(4.258,3.157)}
\gppoint{gp mark 0}{(4.258,3.490)}
\gppoint{gp mark 0}{(4.258,3.408)}
\gppoint{gp mark 0}{(4.258,3.490)}
\gppoint{gp mark 0}{(4.258,3.490)}
\gppoint{gp mark 0}{(4.258,3.214)}
\gppoint{gp mark 0}{(4.258,3.408)}
\gppoint{gp mark 0}{(4.258,3.490)}
\gppoint{gp mark 0}{(4.258,3.030)}
\gppoint{gp mark 0}{(4.258,3.030)}
\gppoint{gp mark 0}{(4.258,3.450)}
\gppoint{gp mark 0}{(4.258,3.030)}
\gppoint{gp mark 0}{(4.258,3.030)}
\gppoint{gp mark 0}{(4.258,2.586)}
\gppoint{gp mark 0}{(4.258,3.214)}
\gppoint{gp mark 0}{(4.258,3.267)}
\gppoint{gp mark 0}{(4.258,3.096)}
\gppoint{gp mark 0}{(4.258,3.096)}
\gppoint{gp mark 0}{(4.258,3.030)}
\gppoint{gp mark 0}{(4.258,3.096)}
\gppoint{gp mark 0}{(4.258,3.030)}
\gppoint{gp mark 0}{(4.258,3.408)}
\gppoint{gp mark 0}{(4.258,3.030)}
\gppoint{gp mark 0}{(4.258,3.317)}
\gppoint{gp mark 0}{(4.258,3.030)}
\gppoint{gp mark 0}{(4.258,3.408)}
\gppoint{gp mark 0}{(4.258,3.096)}
\gppoint{gp mark 0}{(4.258,3.030)}
\gppoint{gp mark 0}{(4.258,2.880)}
\gppoint{gp mark 0}{(4.258,3.267)}
\gppoint{gp mark 0}{(4.258,3.030)}
\gppoint{gp mark 0}{(4.258,2.586)}
\gppoint{gp mark 0}{(4.258,3.157)}
\gppoint{gp mark 0}{(4.258,3.490)}
\gppoint{gp mark 0}{(4.258,2.958)}
\gppoint{gp mark 0}{(4.258,2.958)}
\gppoint{gp mark 0}{(4.258,3.408)}
\gppoint{gp mark 0}{(4.258,3.408)}
\gppoint{gp mark 0}{(4.258,3.157)}
\gppoint{gp mark 0}{(4.258,3.157)}
\gppoint{gp mark 0}{(4.258,3.565)}
\gppoint{gp mark 0}{(4.258,2.958)}
\gppoint{gp mark 0}{(4.258,3.096)}
\gppoint{gp mark 0}{(4.258,2.958)}
\gppoint{gp mark 0}{(4.258,3.096)}
\gppoint{gp mark 0}{(4.258,3.267)}
\gppoint{gp mark 0}{(4.258,3.317)}
\gppoint{gp mark 0}{(4.258,2.586)}
\gppoint{gp mark 0}{(4.258,3.096)}
\gppoint{gp mark 0}{(4.258,3.267)}
\gppoint{gp mark 0}{(4.258,3.408)}
\gppoint{gp mark 0}{(4.258,2.958)}
\gppoint{gp mark 0}{(4.258,3.096)}
\gppoint{gp mark 0}{(4.258,3.030)}
\gppoint{gp mark 0}{(4.258,3.030)}
\gppoint{gp mark 0}{(4.258,3.096)}
\gppoint{gp mark 0}{(4.258,2.586)}
\gppoint{gp mark 0}{(4.258,3.490)}
\gppoint{gp mark 0}{(4.258,3.529)}
\gppoint{gp mark 0}{(4.258,3.408)}
\gppoint{gp mark 0}{(4.258,3.096)}
\gppoint{gp mark 0}{(4.258,3.490)}
\gppoint{gp mark 0}{(4.258,3.490)}
\gppoint{gp mark 0}{(4.258,3.490)}
\gppoint{gp mark 0}{(4.258,3.030)}
\gppoint{gp mark 0}{(4.258,3.490)}
\gppoint{gp mark 0}{(4.258,3.030)}
\gppoint{gp mark 0}{(4.258,2.696)}
\gppoint{gp mark 0}{(4.258,3.408)}
\gppoint{gp mark 0}{(4.258,3.450)}
\gppoint{gp mark 0}{(4.258,3.600)}
\gppoint{gp mark 0}{(4.258,3.408)}
\gppoint{gp mark 0}{(4.258,3.267)}
\gppoint{gp mark 0}{(4.258,3.030)}
\gppoint{gp mark 0}{(4.258,3.490)}
\gppoint{gp mark 0}{(4.258,3.030)}
\gppoint{gp mark 0}{(4.258,3.096)}
\gppoint{gp mark 0}{(4.258,3.030)}
\gppoint{gp mark 0}{(4.258,3.529)}
\gppoint{gp mark 0}{(4.258,3.408)}
\gppoint{gp mark 0}{(4.258,3.157)}
\gppoint{gp mark 0}{(4.258,3.267)}
\gppoint{gp mark 0}{(4.258,2.958)}
\gppoint{gp mark 0}{(4.258,3.096)}
\gppoint{gp mark 0}{(4.258,3.096)}
\gppoint{gp mark 0}{(4.258,3.157)}
\gppoint{gp mark 0}{(4.258,3.267)}
\gppoint{gp mark 0}{(4.258,3.364)}
\gppoint{gp mark 0}{(4.258,2.586)}
\gppoint{gp mark 0}{(4.258,3.364)}
\gppoint{gp mark 0}{(4.258,3.157)}
\gppoint{gp mark 0}{(4.258,3.030)}
\gppoint{gp mark 0}{(4.258,3.408)}
\gppoint{gp mark 0}{(4.258,3.450)}
\gppoint{gp mark 0}{(4.258,3.214)}
\gppoint{gp mark 0}{(4.258,3.600)}
\gppoint{gp mark 0}{(4.258,3.529)}
\gppoint{gp mark 0}{(4.258,3.267)}
\gppoint{gp mark 0}{(4.258,3.157)}
\gppoint{gp mark 0}{(4.258,3.214)}
\gppoint{gp mark 0}{(4.258,3.408)}
\gppoint{gp mark 0}{(4.258,3.214)}
\gppoint{gp mark 0}{(4.258,3.666)}
\gppoint{gp mark 0}{(4.258,3.666)}
\gppoint{gp mark 0}{(4.258,3.214)}
\gppoint{gp mark 0}{(4.258,3.267)}
\gppoint{gp mark 0}{(4.258,3.096)}
\gppoint{gp mark 0}{(4.258,3.096)}
\gppoint{gp mark 0}{(4.258,3.096)}
\gppoint{gp mark 0}{(4.258,2.793)}
\gppoint{gp mark 0}{(4.258,3.157)}
\gppoint{gp mark 0}{(4.258,2.793)}
\gppoint{gp mark 0}{(4.258,3.157)}
\gppoint{gp mark 0}{(4.258,3.096)}
\gppoint{gp mark 0}{(4.258,3.096)}
\gppoint{gp mark 0}{(4.258,3.267)}
\gppoint{gp mark 0}{(4.258,3.267)}
\gppoint{gp mark 0}{(4.258,3.267)}
\gppoint{gp mark 0}{(4.258,3.267)}
\gppoint{gp mark 0}{(4.258,3.267)}
\gppoint{gp mark 0}{(4.258,3.267)}
\gppoint{gp mark 0}{(4.258,3.267)}
\gppoint{gp mark 0}{(4.258,2.586)}
\gppoint{gp mark 0}{(4.258,3.157)}
\gppoint{gp mark 0}{(4.258,3.030)}
\gppoint{gp mark 0}{(4.258,3.408)}
\gppoint{gp mark 0}{(4.258,3.408)}
\gppoint{gp mark 0}{(4.258,3.364)}
\gppoint{gp mark 0}{(4.258,3.408)}
\gppoint{gp mark 0}{(4.258,3.096)}
\gppoint{gp mark 0}{(4.258,3.096)}
\gppoint{gp mark 0}{(4.258,3.317)}
\gppoint{gp mark 0}{(4.258,3.030)}
\gppoint{gp mark 0}{(4.258,3.408)}
\gppoint{gp mark 0}{(4.258,3.096)}
\gppoint{gp mark 0}{(4.258,3.408)}
\gppoint{gp mark 0}{(4.258,3.214)}
\gppoint{gp mark 0}{(4.258,3.096)}
\gppoint{gp mark 0}{(4.258,3.096)}
\gppoint{gp mark 0}{(4.258,3.157)}
\gppoint{gp mark 0}{(4.258,3.364)}
\gppoint{gp mark 0}{(4.353,3.157)}
\gppoint{gp mark 0}{(4.353,3.214)}
\gppoint{gp mark 0}{(4.353,2.696)}
\gppoint{gp mark 0}{(4.353,3.317)}
\gppoint{gp mark 0}{(4.353,3.030)}
\gppoint{gp mark 0}{(4.353,3.756)}
\gppoint{gp mark 0}{(4.353,3.634)}
\gppoint{gp mark 0}{(4.353,3.030)}
\gppoint{gp mark 0}{(4.353,3.317)}
\gppoint{gp mark 0}{(4.353,3.030)}
\gppoint{gp mark 0}{(4.353,3.096)}
\gppoint{gp mark 0}{(4.353,3.030)}
\gppoint{gp mark 0}{(4.353,3.030)}
\gppoint{gp mark 0}{(4.353,3.214)}
\gppoint{gp mark 0}{(4.353,3.214)}
\gppoint{gp mark 0}{(4.353,3.096)}
\gppoint{gp mark 0}{(4.353,3.096)}
\gppoint{gp mark 0}{(4.353,3.096)}
\gppoint{gp mark 0}{(4.353,3.096)}
\gppoint{gp mark 0}{(4.353,3.096)}
\gppoint{gp mark 0}{(4.353,3.096)}
\gppoint{gp mark 0}{(4.353,3.317)}
\gppoint{gp mark 0}{(4.353,3.157)}
\gppoint{gp mark 0}{(4.353,3.408)}
\gppoint{gp mark 0}{(4.353,3.317)}
\gppoint{gp mark 0}{(4.353,3.214)}
\gppoint{gp mark 0}{(4.353,3.408)}
\gppoint{gp mark 0}{(4.353,3.408)}
\gppoint{gp mark 0}{(4.353,3.096)}
\gppoint{gp mark 0}{(4.353,3.727)}
\gppoint{gp mark 0}{(4.353,3.157)}
\gppoint{gp mark 0}{(4.353,3.450)}
\gppoint{gp mark 0}{(4.353,3.214)}
\gppoint{gp mark 0}{(4.353,2.958)}
\gppoint{gp mark 0}{(4.353,3.030)}
\gppoint{gp mark 0}{(4.353,2.958)}
\gppoint{gp mark 0}{(4.353,3.030)}
\gppoint{gp mark 0}{(4.353,3.267)}
\gppoint{gp mark 0}{(4.353,2.880)}
\gppoint{gp mark 0}{(4.353,2.958)}
\gppoint{gp mark 0}{(4.353,2.880)}
\gppoint{gp mark 0}{(4.353,3.408)}
\gppoint{gp mark 0}{(4.353,3.408)}
\gppoint{gp mark 0}{(4.353,3.157)}
\gppoint{gp mark 0}{(4.353,3.317)}
\gppoint{gp mark 0}{(4.353,3.317)}
\gppoint{gp mark 0}{(4.353,3.600)}
\gppoint{gp mark 0}{(4.353,3.157)}
\gppoint{gp mark 0}{(4.353,3.157)}
\gppoint{gp mark 0}{(4.353,3.666)}
\gppoint{gp mark 0}{(4.353,2.958)}
\gppoint{gp mark 0}{(4.353,3.267)}
\gppoint{gp mark 0}{(4.353,3.450)}
\gppoint{gp mark 0}{(4.353,3.565)}
\gppoint{gp mark 0}{(4.353,3.214)}
\gppoint{gp mark 0}{(4.353,2.793)}
\gppoint{gp mark 0}{(4.353,3.214)}
\gppoint{gp mark 0}{(4.353,3.214)}
\gppoint{gp mark 0}{(4.353,3.450)}
\gppoint{gp mark 0}{(4.353,2.958)}
\gppoint{gp mark 0}{(4.353,2.958)}
\gppoint{gp mark 0}{(4.353,3.096)}
\gppoint{gp mark 0}{(4.353,3.529)}
\gppoint{gp mark 0}{(4.353,3.214)}
\gppoint{gp mark 0}{(4.353,2.958)}
\gppoint{gp mark 0}{(4.353,3.157)}
\gppoint{gp mark 0}{(4.353,3.450)}
\gppoint{gp mark 0}{(4.353,3.030)}
\gppoint{gp mark 0}{(4.353,3.030)}
\gppoint{gp mark 0}{(4.353,3.364)}
\gppoint{gp mark 0}{(4.353,3.030)}
\gppoint{gp mark 0}{(4.353,3.214)}
\gppoint{gp mark 0}{(4.353,3.214)}
\gppoint{gp mark 0}{(4.353,3.317)}
\gppoint{gp mark 0}{(4.353,2.958)}
\gppoint{gp mark 0}{(4.353,2.793)}
\gppoint{gp mark 0}{(4.353,3.267)}
\gppoint{gp mark 0}{(4.353,2.793)}
\gppoint{gp mark 0}{(4.353,3.364)}
\gppoint{gp mark 0}{(4.353,3.214)}
\gppoint{gp mark 0}{(4.353,3.157)}
\gppoint{gp mark 0}{(4.353,3.666)}
\gppoint{gp mark 0}{(4.353,3.096)}
\gppoint{gp mark 0}{(4.353,3.364)}
\gppoint{gp mark 0}{(4.353,2.880)}
\gppoint{gp mark 0}{(4.353,3.267)}
\gppoint{gp mark 0}{(4.353,3.214)}
\gppoint{gp mark 0}{(4.353,3.408)}
\gppoint{gp mark 0}{(4.353,3.214)}
\gppoint{gp mark 0}{(4.353,3.214)}
\gppoint{gp mark 0}{(4.353,2.958)}
\gppoint{gp mark 0}{(4.353,3.317)}
\gppoint{gp mark 0}{(4.353,3.529)}
\gppoint{gp mark 0}{(4.353,3.157)}
\gppoint{gp mark 0}{(4.353,3.096)}
\gppoint{gp mark 0}{(4.353,3.214)}
\gppoint{gp mark 0}{(4.353,2.958)}
\gppoint{gp mark 0}{(4.353,3.030)}
\gppoint{gp mark 0}{(4.353,3.529)}
\gppoint{gp mark 0}{(4.353,2.696)}
\gppoint{gp mark 0}{(4.353,3.096)}
\gppoint{gp mark 0}{(4.353,3.157)}
\gppoint{gp mark 0}{(4.353,3.267)}
\gppoint{gp mark 0}{(4.353,3.030)}
\gppoint{gp mark 0}{(4.353,3.157)}
\gppoint{gp mark 0}{(4.353,3.096)}
\gppoint{gp mark 0}{(4.353,3.364)}
\gppoint{gp mark 0}{(4.353,3.267)}
\gppoint{gp mark 0}{(4.353,3.214)}
\gppoint{gp mark 0}{(4.353,3.030)}
\gppoint{gp mark 0}{(4.353,3.364)}
\gppoint{gp mark 0}{(4.353,3.267)}
\gppoint{gp mark 0}{(4.353,3.096)}
\gppoint{gp mark 0}{(4.353,3.096)}
\gppoint{gp mark 0}{(4.353,3.214)}
\gppoint{gp mark 0}{(4.353,3.267)}
\gppoint{gp mark 0}{(4.353,3.030)}
\gppoint{gp mark 0}{(4.353,2.958)}
\gppoint{gp mark 0}{(4.353,2.958)}
\gppoint{gp mark 0}{(4.353,3.214)}
\gppoint{gp mark 0}{(4.353,3.030)}
\gppoint{gp mark 0}{(4.353,3.267)}
\gppoint{gp mark 0}{(4.353,3.157)}
\gppoint{gp mark 0}{(4.353,3.096)}
\gppoint{gp mark 0}{(4.353,2.958)}
\gppoint{gp mark 0}{(4.353,2.696)}
\gppoint{gp mark 0}{(4.353,3.267)}
\gppoint{gp mark 0}{(4.353,3.096)}
\gppoint{gp mark 0}{(4.353,3.634)}
\gppoint{gp mark 0}{(4.353,2.958)}
\gppoint{gp mark 0}{(4.353,3.030)}
\gppoint{gp mark 0}{(4.353,3.267)}
\gppoint{gp mark 0}{(4.353,3.450)}
\gppoint{gp mark 0}{(4.353,3.450)}
\gppoint{gp mark 0}{(4.353,3.911)}
\gppoint{gp mark 0}{(4.353,3.214)}
\gppoint{gp mark 0}{(4.353,3.450)}
\gppoint{gp mark 0}{(4.353,3.727)}
\gppoint{gp mark 0}{(4.353,2.958)}
\gppoint{gp mark 0}{(4.353,2.958)}
\gppoint{gp mark 0}{(4.353,3.600)}
\gppoint{gp mark 0}{(4.353,3.267)}
\gppoint{gp mark 0}{(4.353,3.727)}
\gppoint{gp mark 0}{(4.353,3.756)}
\gppoint{gp mark 0}{(4.353,2.880)}
\gppoint{gp mark 0}{(4.353,3.096)}
\gppoint{gp mark 0}{(4.353,2.880)}
\gppoint{gp mark 0}{(4.353,2.696)}
\gppoint{gp mark 0}{(4.353,3.214)}
\gppoint{gp mark 0}{(4.353,3.214)}
\gppoint{gp mark 0}{(4.353,2.793)}
\gppoint{gp mark 0}{(4.353,3.666)}
\gppoint{gp mark 0}{(4.353,2.793)}
\gppoint{gp mark 0}{(4.353,3.214)}
\gppoint{gp mark 0}{(4.353,3.267)}
\gppoint{gp mark 0}{(4.353,3.634)}
\gppoint{gp mark 0}{(4.353,3.214)}
\gppoint{gp mark 0}{(4.353,2.880)}
\gppoint{gp mark 0}{(4.353,3.450)}
\gppoint{gp mark 0}{(4.353,3.096)}
\gppoint{gp mark 0}{(4.353,3.096)}
\gppoint{gp mark 0}{(4.353,2.958)}
\gppoint{gp mark 0}{(4.353,3.157)}
\gppoint{gp mark 0}{(4.353,3.364)}
\gppoint{gp mark 0}{(4.353,3.096)}
\gppoint{gp mark 0}{(4.353,2.958)}
\gppoint{gp mark 0}{(4.353,3.267)}
\gppoint{gp mark 0}{(4.353,3.214)}
\gppoint{gp mark 0}{(4.353,3.666)}
\gppoint{gp mark 0}{(4.353,2.880)}
\gppoint{gp mark 0}{(4.353,3.157)}
\gppoint{gp mark 0}{(4.353,3.887)}
\gppoint{gp mark 0}{(4.353,3.490)}
\gppoint{gp mark 0}{(4.353,3.096)}
\gppoint{gp mark 0}{(4.353,3.600)}
\gppoint{gp mark 0}{(4.353,3.096)}
\gppoint{gp mark 0}{(4.353,3.157)}
\gppoint{gp mark 0}{(4.353,2.958)}
\gppoint{gp mark 0}{(4.353,2.958)}
\gppoint{gp mark 0}{(4.353,2.958)}
\gppoint{gp mark 0}{(4.353,3.030)}
\gppoint{gp mark 0}{(4.353,3.030)}
\gppoint{gp mark 0}{(4.353,3.364)}
\gppoint{gp mark 0}{(4.353,3.267)}
\gppoint{gp mark 0}{(4.353,2.586)}
\gppoint{gp mark 0}{(4.353,3.030)}
\gppoint{gp mark 0}{(4.353,3.727)}
\gppoint{gp mark 0}{(4.353,3.600)}
\gppoint{gp mark 0}{(4.353,3.267)}
\gppoint{gp mark 0}{(4.353,3.157)}
\gppoint{gp mark 0}{(4.353,3.157)}
\gppoint{gp mark 0}{(4.353,2.958)}
\gppoint{gp mark 0}{(4.353,3.267)}
\gppoint{gp mark 0}{(4.353,3.267)}
\gppoint{gp mark 0}{(4.353,2.880)}
\gppoint{gp mark 0}{(4.353,3.666)}
\gppoint{gp mark 0}{(4.353,3.214)}
\gppoint{gp mark 0}{(4.353,3.317)}
\gppoint{gp mark 0}{(4.353,3.214)}
\gppoint{gp mark 0}{(4.353,2.958)}
\gppoint{gp mark 0}{(4.353,3.364)}
\gppoint{gp mark 0}{(4.353,3.317)}
\gppoint{gp mark 0}{(4.353,3.267)}
\gppoint{gp mark 0}{(4.353,3.030)}
\gppoint{gp mark 0}{(4.353,3.490)}
\gppoint{gp mark 0}{(4.353,3.030)}
\gppoint{gp mark 0}{(4.353,2.958)}
\gppoint{gp mark 0}{(4.353,3.317)}
\gppoint{gp mark 0}{(4.353,3.408)}
\gppoint{gp mark 0}{(4.353,2.958)}
\gppoint{gp mark 0}{(4.353,3.214)}
\gppoint{gp mark 0}{(4.353,3.214)}
\gppoint{gp mark 0}{(4.353,3.364)}
\gppoint{gp mark 0}{(4.353,2.880)}
\gppoint{gp mark 0}{(4.353,2.958)}
\gppoint{gp mark 0}{(4.353,3.157)}
\gppoint{gp mark 0}{(4.353,3.096)}
\gppoint{gp mark 0}{(4.353,2.958)}
\gppoint{gp mark 0}{(4.353,2.958)}
\gppoint{gp mark 0}{(4.353,3.490)}
\gppoint{gp mark 0}{(4.353,3.214)}
\gppoint{gp mark 0}{(4.353,3.214)}
\gppoint{gp mark 0}{(4.353,3.214)}
\gppoint{gp mark 0}{(4.353,3.096)}
\gppoint{gp mark 0}{(4.353,3.529)}
\gppoint{gp mark 0}{(4.353,3.490)}
\gppoint{gp mark 0}{(4.353,3.267)}
\gppoint{gp mark 0}{(4.353,3.317)}
\gppoint{gp mark 0}{(4.353,3.364)}
\gppoint{gp mark 0}{(4.353,3.267)}
\gppoint{gp mark 0}{(4.353,3.600)}
\gppoint{gp mark 0}{(4.353,2.958)}
\gppoint{gp mark 0}{(4.353,2.958)}
\gppoint{gp mark 0}{(4.353,3.096)}
\gppoint{gp mark 0}{(4.353,3.408)}
\gppoint{gp mark 0}{(4.353,3.450)}
\gppoint{gp mark 0}{(4.353,3.267)}
\gppoint{gp mark 0}{(4.353,3.267)}
\gppoint{gp mark 0}{(4.353,3.157)}
\gppoint{gp mark 0}{(4.353,3.157)}
\gppoint{gp mark 0}{(4.353,3.157)}
\gppoint{gp mark 0}{(4.353,3.030)}
\gppoint{gp mark 0}{(4.353,2.958)}
\gppoint{gp mark 0}{(4.353,3.157)}
\gppoint{gp mark 0}{(4.353,2.958)}
\gppoint{gp mark 0}{(4.353,2.958)}
\gppoint{gp mark 0}{(4.353,3.096)}
\gppoint{gp mark 0}{(4.353,3.214)}
\gppoint{gp mark 0}{(4.353,3.490)}
\gppoint{gp mark 0}{(4.353,3.096)}
\gppoint{gp mark 0}{(4.353,3.214)}
\gppoint{gp mark 0}{(4.353,3.157)}
\gppoint{gp mark 0}{(4.353,3.030)}
\gppoint{gp mark 0}{(4.353,3.450)}
\gppoint{gp mark 0}{(4.353,3.214)}
\gppoint{gp mark 0}{(4.353,3.565)}
\gppoint{gp mark 0}{(4.353,2.793)}
\gppoint{gp mark 0}{(4.353,3.030)}
\gppoint{gp mark 0}{(4.353,3.157)}
\gppoint{gp mark 0}{(4.353,3.214)}
\gppoint{gp mark 0}{(4.353,3.529)}
\gppoint{gp mark 0}{(4.353,3.450)}
\gppoint{gp mark 0}{(4.353,3.030)}
\gppoint{gp mark 0}{(4.353,3.267)}
\gppoint{gp mark 0}{(4.353,3.214)}
\gppoint{gp mark 0}{(4.353,3.529)}
\gppoint{gp mark 0}{(4.353,3.214)}
\gppoint{gp mark 0}{(4.353,3.214)}
\gppoint{gp mark 0}{(4.353,3.214)}
\gppoint{gp mark 0}{(4.353,3.267)}
\gppoint{gp mark 0}{(4.353,3.267)}
\gppoint{gp mark 0}{(4.353,3.364)}
\gppoint{gp mark 0}{(4.353,3.267)}
\gppoint{gp mark 0}{(4.353,3.214)}
\gppoint{gp mark 0}{(4.353,3.364)}
\gppoint{gp mark 0}{(4.353,3.784)}
\gppoint{gp mark 0}{(4.353,3.214)}
\gppoint{gp mark 0}{(4.353,3.030)}
\gppoint{gp mark 0}{(4.353,3.267)}
\gppoint{gp mark 0}{(4.353,2.880)}
\gppoint{gp mark 0}{(4.353,3.450)}
\gppoint{gp mark 0}{(4.353,3.408)}
\gppoint{gp mark 0}{(4.353,3.490)}
\gppoint{gp mark 0}{(4.353,3.267)}
\gppoint{gp mark 0}{(4.353,3.157)}
\gppoint{gp mark 0}{(4.353,3.267)}
\gppoint{gp mark 0}{(4.353,3.030)}
\gppoint{gp mark 0}{(4.353,3.267)}
\gppoint{gp mark 0}{(4.353,3.157)}
\gppoint{gp mark 0}{(4.353,2.958)}
\gppoint{gp mark 0}{(4.353,3.408)}
\gppoint{gp mark 0}{(4.353,3.214)}
\gppoint{gp mark 0}{(4.353,3.214)}
\gppoint{gp mark 0}{(4.353,3.214)}
\gppoint{gp mark 0}{(4.353,2.958)}
\gppoint{gp mark 0}{(4.353,3.666)}
\gppoint{gp mark 0}{(4.353,3.408)}
\gppoint{gp mark 0}{(4.353,3.267)}
\gppoint{gp mark 0}{(4.353,3.408)}
\gppoint{gp mark 0}{(4.353,3.214)}
\gppoint{gp mark 0}{(4.353,3.214)}
\gppoint{gp mark 0}{(4.353,3.214)}
\gppoint{gp mark 0}{(4.353,3.157)}
\gppoint{gp mark 0}{(4.353,3.529)}
\gppoint{gp mark 0}{(4.353,3.214)}
\gppoint{gp mark 0}{(4.353,3.157)}
\gppoint{gp mark 0}{(4.353,3.030)}
\gppoint{gp mark 0}{(4.353,3.529)}
\gppoint{gp mark 0}{(4.353,3.408)}
\gppoint{gp mark 0}{(4.353,3.666)}
\gppoint{gp mark 0}{(4.353,3.267)}
\gppoint{gp mark 0}{(4.353,3.157)}
\gppoint{gp mark 0}{(4.353,3.214)}
\gppoint{gp mark 0}{(4.353,3.317)}
\gppoint{gp mark 0}{(4.353,3.157)}
\gppoint{gp mark 0}{(4.353,3.214)}
\gppoint{gp mark 0}{(4.353,3.600)}
\gppoint{gp mark 0}{(4.353,3.214)}
\gppoint{gp mark 0}{(4.353,3.214)}
\gppoint{gp mark 0}{(4.353,3.157)}
\gppoint{gp mark 0}{(4.353,2.880)}
\gppoint{gp mark 0}{(4.353,3.157)}
\gppoint{gp mark 0}{(4.353,3.157)}
\gppoint{gp mark 0}{(4.353,3.157)}
\gppoint{gp mark 0}{(4.353,3.030)}
\gppoint{gp mark 0}{(4.353,3.450)}
\gppoint{gp mark 0}{(4.353,3.214)}
\gppoint{gp mark 0}{(4.353,3.267)}
\gppoint{gp mark 0}{(4.353,3.490)}
\gppoint{gp mark 0}{(4.353,3.214)}
\gppoint{gp mark 0}{(4.353,2.793)}
\gppoint{gp mark 0}{(4.353,3.030)}
\gppoint{gp mark 0}{(4.353,3.214)}
\gppoint{gp mark 0}{(4.353,3.214)}
\gppoint{gp mark 0}{(4.353,3.450)}
\gppoint{gp mark 0}{(4.353,3.030)}
\gppoint{gp mark 0}{(4.353,3.317)}
\gppoint{gp mark 0}{(4.353,3.030)}
\gppoint{gp mark 0}{(4.353,3.934)}
\gppoint{gp mark 0}{(4.353,3.317)}
\gppoint{gp mark 0}{(4.353,3.030)}
\gppoint{gp mark 0}{(4.353,3.157)}
\gppoint{gp mark 0}{(4.353,3.096)}
\gppoint{gp mark 0}{(4.353,2.880)}
\gppoint{gp mark 0}{(4.353,3.214)}
\gppoint{gp mark 0}{(4.353,3.214)}
\gppoint{gp mark 0}{(4.353,2.793)}
\gppoint{gp mark 0}{(4.353,3.267)}
\gppoint{gp mark 0}{(4.353,3.267)}
\gppoint{gp mark 0}{(4.353,3.267)}
\gppoint{gp mark 0}{(4.353,3.096)}
\gppoint{gp mark 0}{(4.353,3.214)}
\gppoint{gp mark 0}{(4.353,3.030)}
\gppoint{gp mark 0}{(4.353,2.958)}
\gppoint{gp mark 0}{(4.353,3.529)}
\gppoint{gp mark 0}{(4.353,3.157)}
\gppoint{gp mark 0}{(4.353,3.450)}
\gppoint{gp mark 0}{(4.353,3.157)}
\gppoint{gp mark 0}{(4.353,3.157)}
\gppoint{gp mark 0}{(4.353,2.696)}
\gppoint{gp mark 0}{(4.353,3.450)}
\gppoint{gp mark 0}{(4.353,3.317)}
\gppoint{gp mark 0}{(4.353,3.157)}
\gppoint{gp mark 0}{(4.353,3.214)}
\gppoint{gp mark 0}{(4.353,3.214)}
\gppoint{gp mark 0}{(4.353,3.157)}
\gppoint{gp mark 0}{(4.353,3.529)}
\gppoint{gp mark 0}{(4.353,3.911)}
\gppoint{gp mark 0}{(4.353,3.157)}
\gppoint{gp mark 0}{(4.353,3.450)}
\gppoint{gp mark 0}{(4.353,3.157)}
\gppoint{gp mark 0}{(4.353,3.214)}
\gppoint{gp mark 0}{(4.353,3.096)}
\gppoint{gp mark 0}{(4.353,3.030)}
\gppoint{gp mark 0}{(4.353,2.958)}
\gppoint{gp mark 0}{(4.353,3.096)}
\gppoint{gp mark 0}{(4.353,3.157)}
\gppoint{gp mark 0}{(4.353,3.364)}
\gppoint{gp mark 0}{(4.353,3.317)}
\gppoint{gp mark 0}{(4.353,3.030)}
\gppoint{gp mark 0}{(4.353,3.214)}
\gppoint{gp mark 0}{(4.353,3.096)}
\gppoint{gp mark 0}{(4.353,3.157)}
\gppoint{gp mark 0}{(4.353,3.267)}
\gppoint{gp mark 0}{(4.353,3.408)}
\gppoint{gp mark 0}{(4.353,3.030)}
\gppoint{gp mark 0}{(4.353,3.364)}
\gppoint{gp mark 0}{(4.353,3.096)}
\gppoint{gp mark 0}{(4.353,3.600)}
\gppoint{gp mark 0}{(4.353,3.529)}
\gppoint{gp mark 0}{(4.353,2.793)}
\gppoint{gp mark 0}{(4.353,3.214)}
\gppoint{gp mark 0}{(4.353,3.096)}
\gppoint{gp mark 0}{(4.353,3.529)}
\gppoint{gp mark 0}{(4.353,3.529)}
\gppoint{gp mark 0}{(4.353,3.529)}
\gppoint{gp mark 0}{(4.353,3.096)}
\gppoint{gp mark 0}{(4.353,3.157)}
\gppoint{gp mark 0}{(4.353,3.157)}
\gppoint{gp mark 0}{(4.353,3.364)}
\gppoint{gp mark 0}{(4.353,3.529)}
\gppoint{gp mark 0}{(4.353,3.529)}
\gppoint{gp mark 0}{(4.353,3.490)}
\gppoint{gp mark 0}{(4.353,3.267)}
\gppoint{gp mark 0}{(4.353,3.529)}
\gppoint{gp mark 0}{(4.353,3.529)}
\gppoint{gp mark 0}{(4.353,3.157)}
\gppoint{gp mark 0}{(4.353,3.096)}
\gppoint{gp mark 0}{(4.353,3.529)}
\gppoint{gp mark 0}{(4.353,3.529)}
\gppoint{gp mark 0}{(4.353,3.529)}
\gppoint{gp mark 0}{(4.353,3.364)}
\gppoint{gp mark 0}{(4.353,3.529)}
\gppoint{gp mark 0}{(4.353,2.958)}
\gppoint{gp mark 0}{(4.353,3.529)}
\gppoint{gp mark 0}{(4.353,3.490)}
\gppoint{gp mark 0}{(4.353,3.529)}
\gppoint{gp mark 0}{(4.353,3.490)}
\gppoint{gp mark 0}{(4.353,3.529)}
\gppoint{gp mark 0}{(4.353,3.364)}
\gppoint{gp mark 0}{(4.353,3.364)}
\gppoint{gp mark 0}{(4.353,3.267)}
\gppoint{gp mark 0}{(4.353,3.364)}
\gppoint{gp mark 0}{(4.353,2.958)}
\gppoint{gp mark 0}{(4.353,3.030)}
\gppoint{gp mark 0}{(4.353,3.666)}
\gppoint{gp mark 0}{(4.353,3.666)}
\gppoint{gp mark 0}{(4.353,3.030)}
\gppoint{gp mark 0}{(4.353,3.267)}
\gppoint{gp mark 0}{(4.353,3.317)}
\gppoint{gp mark 0}{(4.353,3.096)}
\gppoint{gp mark 0}{(4.353,3.096)}
\gppoint{gp mark 0}{(4.353,3.157)}
\gppoint{gp mark 0}{(4.353,3.157)}
\gppoint{gp mark 0}{(4.353,3.096)}
\gppoint{gp mark 0}{(4.353,3.030)}
\gppoint{gp mark 0}{(4.353,3.490)}
\gppoint{gp mark 0}{(4.353,3.157)}
\gppoint{gp mark 0}{(4.353,2.958)}
\gppoint{gp mark 0}{(4.353,3.490)}
\gppoint{gp mark 0}{(4.353,3.450)}
\gppoint{gp mark 0}{(4.353,2.696)}
\gppoint{gp mark 0}{(4.353,2.696)}
\gppoint{gp mark 0}{(4.353,3.317)}
\gppoint{gp mark 0}{(4.353,2.958)}
\gppoint{gp mark 0}{(4.353,3.214)}
\gppoint{gp mark 0}{(4.353,3.267)}
\gppoint{gp mark 0}{(4.353,3.157)}
\gppoint{gp mark 0}{(4.353,3.030)}
\gppoint{gp mark 0}{(4.353,3.214)}
\gppoint{gp mark 0}{(4.353,3.214)}
\gppoint{gp mark 0}{(4.353,3.030)}
\gppoint{gp mark 0}{(4.353,3.490)}
\gppoint{gp mark 0}{(4.353,2.793)}
\gppoint{gp mark 0}{(4.353,3.490)}
\gppoint{gp mark 0}{(4.353,2.958)}
\gppoint{gp mark 0}{(4.353,3.600)}
\gppoint{gp mark 0}{(4.353,3.157)}
\gppoint{gp mark 0}{(4.353,3.096)}
\gppoint{gp mark 0}{(4.353,3.157)}
\gppoint{gp mark 0}{(4.353,3.317)}
\gppoint{gp mark 0}{(4.353,3.030)}
\gppoint{gp mark 0}{(4.353,3.157)}
\gppoint{gp mark 0}{(4.353,3.529)}
\gppoint{gp mark 0}{(4.353,3.529)}
\gppoint{gp mark 0}{(4.353,3.214)}
\gppoint{gp mark 0}{(4.353,3.267)}
\gppoint{gp mark 0}{(4.353,3.214)}
\gppoint{gp mark 0}{(4.353,3.364)}
\gppoint{gp mark 0}{(4.353,3.364)}
\gppoint{gp mark 0}{(4.353,3.364)}
\gppoint{gp mark 0}{(4.353,3.529)}
\gppoint{gp mark 0}{(4.353,3.096)}
\gppoint{gp mark 0}{(4.353,3.214)}
\gppoint{gp mark 0}{(4.353,3.317)}
\gppoint{gp mark 0}{(4.353,2.793)}
\gppoint{gp mark 0}{(4.353,3.096)}
\gppoint{gp mark 0}{(4.353,3.450)}
\gppoint{gp mark 0}{(4.353,3.600)}
\gppoint{gp mark 0}{(4.353,3.096)}
\gppoint{gp mark 0}{(4.353,3.317)}
\gppoint{gp mark 0}{(4.353,3.096)}
\gppoint{gp mark 0}{(4.353,3.096)}
\gppoint{gp mark 0}{(4.353,3.096)}
\gppoint{gp mark 0}{(4.353,4.061)}
\gppoint{gp mark 0}{(4.353,3.267)}
\gppoint{gp mark 0}{(4.353,3.096)}
\gppoint{gp mark 0}{(4.353,3.096)}
\gppoint{gp mark 0}{(4.353,3.096)}
\gppoint{gp mark 0}{(4.353,3.317)}
\gppoint{gp mark 0}{(4.353,3.364)}
\gppoint{gp mark 0}{(4.353,2.880)}
\gppoint{gp mark 0}{(4.353,2.958)}
\gppoint{gp mark 0}{(4.353,3.364)}
\gppoint{gp mark 0}{(4.353,2.880)}
\gppoint{gp mark 0}{(4.353,3.364)}
\gppoint{gp mark 0}{(4.353,3.214)}
\gppoint{gp mark 0}{(4.353,3.364)}
\gppoint{gp mark 0}{(4.353,3.529)}
\gppoint{gp mark 0}{(4.353,3.157)}
\gppoint{gp mark 0}{(4.353,3.157)}
\gppoint{gp mark 0}{(4.353,3.317)}
\gppoint{gp mark 0}{(4.353,3.317)}
\gppoint{gp mark 0}{(4.353,3.096)}
\gppoint{gp mark 0}{(4.353,3.214)}
\gppoint{gp mark 0}{(4.353,3.267)}
\gppoint{gp mark 0}{(4.353,3.317)}
\gppoint{gp mark 0}{(4.353,3.030)}
\gppoint{gp mark 0}{(4.353,3.030)}
\gppoint{gp mark 0}{(4.353,3.030)}
\gppoint{gp mark 0}{(4.353,3.267)}
\gppoint{gp mark 0}{(4.353,3.214)}
\gppoint{gp mark 0}{(4.353,3.267)}
\gppoint{gp mark 0}{(4.353,3.214)}
\gppoint{gp mark 0}{(4.353,3.317)}
\gppoint{gp mark 0}{(4.353,3.096)}
\gppoint{gp mark 0}{(4.353,3.214)}
\gppoint{gp mark 0}{(4.353,3.267)}
\gppoint{gp mark 0}{(4.353,3.267)}
\gppoint{gp mark 0}{(4.353,3.157)}
\gppoint{gp mark 0}{(4.353,3.157)}
\gppoint{gp mark 0}{(4.353,3.157)}
\gppoint{gp mark 0}{(4.353,3.157)}
\gppoint{gp mark 0}{(4.353,3.157)}
\gppoint{gp mark 0}{(4.353,3.157)}
\gppoint{gp mark 0}{(4.353,3.096)}
\gppoint{gp mark 0}{(4.353,3.157)}
\gppoint{gp mark 0}{(4.353,3.096)}
\gppoint{gp mark 0}{(4.353,3.529)}
\gppoint{gp mark 0}{(4.353,3.157)}
\gppoint{gp mark 0}{(4.353,3.157)}
\gppoint{gp mark 0}{(4.353,3.267)}
\gppoint{gp mark 0}{(4.353,3.214)}
\gppoint{gp mark 0}{(4.353,3.727)}
\gppoint{gp mark 0}{(4.353,3.157)}
\gppoint{gp mark 0}{(4.441,3.565)}
\gppoint{gp mark 0}{(4.441,3.214)}
\gppoint{gp mark 0}{(4.441,2.958)}
\gppoint{gp mark 0}{(4.441,3.529)}
\gppoint{gp mark 0}{(4.441,3.450)}
\gppoint{gp mark 0}{(4.441,3.490)}
\gppoint{gp mark 0}{(4.441,3.267)}
\gppoint{gp mark 0}{(4.441,3.364)}
\gppoint{gp mark 0}{(4.441,3.267)}
\gppoint{gp mark 0}{(4.441,3.408)}
\gppoint{gp mark 0}{(4.441,3.408)}
\gppoint{gp mark 0}{(4.441,3.408)}
\gppoint{gp mark 0}{(4.441,3.317)}
\gppoint{gp mark 0}{(4.441,3.634)}
\gppoint{gp mark 0}{(4.441,3.157)}
\gppoint{gp mark 0}{(4.441,3.490)}
\gppoint{gp mark 0}{(4.441,3.364)}
\gppoint{gp mark 0}{(4.441,2.793)}
\gppoint{gp mark 0}{(4.441,3.214)}
\gppoint{gp mark 0}{(4.441,3.030)}
\gppoint{gp mark 0}{(4.441,3.934)}
\gppoint{gp mark 0}{(4.441,3.317)}
\gppoint{gp mark 0}{(4.441,2.958)}
\gppoint{gp mark 0}{(4.441,3.267)}
\gppoint{gp mark 0}{(4.441,3.490)}
\gppoint{gp mark 0}{(4.441,3.490)}
\gppoint{gp mark 0}{(4.441,3.364)}
\gppoint{gp mark 0}{(4.441,3.364)}
\gppoint{gp mark 0}{(4.441,3.030)}
\gppoint{gp mark 0}{(4.441,3.096)}
\gppoint{gp mark 0}{(4.441,2.880)}
\gppoint{gp mark 0}{(4.441,3.490)}
\gppoint{gp mark 0}{(4.441,3.267)}
\gppoint{gp mark 0}{(4.441,3.214)}
\gppoint{gp mark 0}{(4.441,3.157)}
\gppoint{gp mark 0}{(4.441,3.214)}
\gppoint{gp mark 0}{(4.441,3.450)}
\gppoint{gp mark 0}{(4.441,3.030)}
\gppoint{gp mark 0}{(4.441,3.214)}
\gppoint{gp mark 0}{(4.441,3.267)}
\gppoint{gp mark 0}{(4.441,3.490)}
\gppoint{gp mark 0}{(4.441,3.157)}
\gppoint{gp mark 0}{(4.441,3.030)}
\gppoint{gp mark 0}{(4.441,3.157)}
\gppoint{gp mark 0}{(4.441,3.157)}
\gppoint{gp mark 0}{(4.441,2.958)}
\gppoint{gp mark 0}{(4.441,3.157)}
\gppoint{gp mark 0}{(4.441,3.490)}
\gppoint{gp mark 0}{(4.441,3.030)}
\gppoint{gp mark 0}{(4.441,3.408)}
\gppoint{gp mark 0}{(4.441,3.934)}
\gppoint{gp mark 0}{(4.441,3.157)}
\gppoint{gp mark 0}{(4.441,3.727)}
\gppoint{gp mark 0}{(4.441,3.030)}
\gppoint{gp mark 0}{(4.441,3.727)}
\gppoint{gp mark 0}{(4.441,3.030)}
\gppoint{gp mark 0}{(4.441,3.214)}
\gppoint{gp mark 0}{(4.441,3.214)}
\gppoint{gp mark 0}{(4.441,3.096)}
\gppoint{gp mark 0}{(4.441,3.030)}
\gppoint{gp mark 0}{(4.441,2.880)}
\gppoint{gp mark 0}{(4.441,3.364)}
\gppoint{gp mark 0}{(4.441,3.490)}
\gppoint{gp mark 0}{(4.441,3.490)}
\gppoint{gp mark 0}{(4.441,3.490)}
\gppoint{gp mark 0}{(4.441,3.030)}
\gppoint{gp mark 0}{(4.441,3.490)}
\gppoint{gp mark 0}{(4.441,3.490)}
\gppoint{gp mark 0}{(4.441,2.880)}
\gppoint{gp mark 0}{(4.441,3.490)}
\gppoint{gp mark 0}{(4.441,3.490)}
\gppoint{gp mark 0}{(4.441,3.490)}
\gppoint{gp mark 0}{(4.441,3.214)}
\gppoint{gp mark 0}{(4.441,3.157)}
\gppoint{gp mark 0}{(4.441,3.267)}
\gppoint{gp mark 0}{(4.441,3.490)}
\gppoint{gp mark 0}{(4.441,2.958)}
\gppoint{gp mark 0}{(4.441,3.697)}
\gppoint{gp mark 0}{(4.441,3.666)}
\gppoint{gp mark 0}{(4.441,3.666)}
\gppoint{gp mark 0}{(4.441,3.157)}
\gppoint{gp mark 0}{(4.441,3.267)}
\gppoint{gp mark 0}{(4.441,3.490)}
\gppoint{gp mark 0}{(4.441,3.096)}
\gppoint{gp mark 0}{(4.441,3.364)}
\gppoint{gp mark 0}{(4.441,3.490)}
\gppoint{gp mark 0}{(4.441,3.490)}
\gppoint{gp mark 0}{(4.441,3.600)}
\gppoint{gp mark 0}{(4.441,3.600)}
\gppoint{gp mark 0}{(4.441,3.317)}
\gppoint{gp mark 0}{(4.441,3.157)}
\gppoint{gp mark 0}{(4.441,3.490)}
\gppoint{gp mark 0}{(4.441,2.958)}
\gppoint{gp mark 0}{(4.441,3.450)}
\gppoint{gp mark 0}{(4.441,3.214)}
\gppoint{gp mark 0}{(4.441,3.214)}
\gppoint{gp mark 0}{(4.441,3.267)}
\gppoint{gp mark 0}{(4.441,3.157)}
\gppoint{gp mark 0}{(4.441,3.157)}
\gppoint{gp mark 0}{(4.441,3.565)}
\gppoint{gp mark 0}{(4.441,3.096)}
\gppoint{gp mark 0}{(4.441,3.600)}
\gppoint{gp mark 0}{(4.441,3.600)}
\gppoint{gp mark 0}{(4.441,3.529)}
\gppoint{gp mark 0}{(4.441,3.529)}
\gppoint{gp mark 0}{(4.441,3.600)}
\gppoint{gp mark 0}{(4.441,3.214)}
\gppoint{gp mark 0}{(4.441,3.030)}
\gppoint{gp mark 0}{(4.441,3.030)}
\gppoint{gp mark 0}{(4.441,3.529)}
\gppoint{gp mark 0}{(4.441,3.030)}
\gppoint{gp mark 0}{(4.441,3.364)}
\gppoint{gp mark 0}{(4.441,3.030)}
\gppoint{gp mark 0}{(4.441,3.837)}
\gppoint{gp mark 0}{(4.441,3.157)}
\gppoint{gp mark 0}{(4.441,3.030)}
\gppoint{gp mark 0}{(4.441,3.096)}
\gppoint{gp mark 0}{(4.441,3.364)}
\gppoint{gp mark 0}{(4.441,3.157)}
\gppoint{gp mark 0}{(4.441,3.030)}
\gppoint{gp mark 0}{(4.441,3.634)}
\gppoint{gp mark 0}{(4.441,3.450)}
\gppoint{gp mark 0}{(4.441,3.157)}
\gppoint{gp mark 0}{(4.441,3.267)}
\gppoint{gp mark 0}{(4.441,3.214)}
\gppoint{gp mark 0}{(4.441,3.096)}
\gppoint{gp mark 0}{(4.441,3.911)}
\gppoint{gp mark 0}{(4.441,3.214)}
\gppoint{gp mark 0}{(4.441,3.214)}
\gppoint{gp mark 0}{(4.441,3.157)}
\gppoint{gp mark 0}{(4.441,3.450)}
\gppoint{gp mark 0}{(4.441,3.666)}
\gppoint{gp mark 0}{(4.441,3.317)}
\gppoint{gp mark 0}{(4.441,3.030)}
\gppoint{gp mark 0}{(4.441,3.634)}
\gppoint{gp mark 0}{(4.441,3.214)}
\gppoint{gp mark 0}{(4.441,3.214)}
\gppoint{gp mark 0}{(4.441,3.030)}
\gppoint{gp mark 0}{(4.441,2.880)}
\gppoint{gp mark 0}{(4.441,3.529)}
\gppoint{gp mark 0}{(4.441,3.157)}
\gppoint{gp mark 0}{(4.441,3.364)}
\gppoint{gp mark 0}{(4.441,3.490)}
\gppoint{gp mark 0}{(4.441,3.030)}
\gppoint{gp mark 0}{(4.441,3.214)}
\gppoint{gp mark 0}{(4.441,3.317)}
\gppoint{gp mark 0}{(4.441,3.096)}
\gppoint{gp mark 0}{(4.441,2.696)}
\gppoint{gp mark 0}{(4.441,2.880)}
\gppoint{gp mark 0}{(4.441,3.096)}
\gppoint{gp mark 0}{(4.441,3.600)}
\gppoint{gp mark 0}{(4.441,3.408)}
\gppoint{gp mark 0}{(4.441,3.490)}
\gppoint{gp mark 0}{(4.441,3.267)}
\gppoint{gp mark 0}{(4.441,3.030)}
\gppoint{gp mark 0}{(4.441,3.490)}
\gppoint{gp mark 0}{(4.441,3.490)}
\gppoint{gp mark 0}{(4.441,3.490)}
\gppoint{gp mark 0}{(4.441,3.364)}
\gppoint{gp mark 0}{(4.441,3.450)}
\gppoint{gp mark 0}{(4.441,3.408)}
\gppoint{gp mark 0}{(4.441,3.267)}
\gppoint{gp mark 0}{(4.441,3.096)}
\gppoint{gp mark 0}{(4.441,3.408)}
\gppoint{gp mark 0}{(4.441,3.214)}
\gppoint{gp mark 0}{(4.441,3.811)}
\gppoint{gp mark 0}{(4.441,3.727)}
\gppoint{gp mark 0}{(4.441,3.317)}
\gppoint{gp mark 0}{(4.441,3.030)}
\gppoint{gp mark 0}{(4.441,3.666)}
\gppoint{gp mark 0}{(4.441,3.030)}
\gppoint{gp mark 0}{(4.441,3.157)}
\gppoint{gp mark 0}{(4.441,3.364)}
\gppoint{gp mark 0}{(4.441,3.096)}
\gppoint{gp mark 0}{(4.441,3.634)}
\gppoint{gp mark 0}{(4.441,3.096)}
\gppoint{gp mark 0}{(4.441,3.096)}
\gppoint{gp mark 0}{(4.441,3.317)}
\gppoint{gp mark 0}{(4.441,3.529)}
\gppoint{gp mark 0}{(4.441,3.490)}
\gppoint{gp mark 0}{(4.441,3.096)}
\gppoint{gp mark 0}{(4.441,3.096)}
\gppoint{gp mark 0}{(4.441,3.096)}
\gppoint{gp mark 0}{(4.441,3.490)}
\gppoint{gp mark 0}{(4.441,3.096)}
\gppoint{gp mark 0}{(4.441,3.490)}
\gppoint{gp mark 0}{(4.441,3.157)}
\gppoint{gp mark 0}{(4.441,3.267)}
\gppoint{gp mark 0}{(4.441,3.408)}
\gppoint{gp mark 0}{(4.441,3.157)}
\gppoint{gp mark 0}{(4.441,3.490)}
\gppoint{gp mark 0}{(4.441,3.364)}
\gppoint{gp mark 0}{(4.441,3.490)}
\gppoint{gp mark 0}{(4.441,3.364)}
\gppoint{gp mark 0}{(4.441,3.157)}
\gppoint{gp mark 0}{(4.441,3.811)}
\gppoint{gp mark 0}{(4.441,3.490)}
\gppoint{gp mark 0}{(4.441,3.157)}
\gppoint{gp mark 0}{(4.441,3.634)}
\gppoint{gp mark 0}{(4.441,3.214)}
\gppoint{gp mark 0}{(4.441,3.096)}
\gppoint{gp mark 0}{(4.441,3.030)}
\gppoint{gp mark 0}{(4.441,3.887)}
\gppoint{gp mark 0}{(4.441,3.157)}
\gppoint{gp mark 0}{(4.441,3.364)}
\gppoint{gp mark 0}{(4.441,3.096)}
\gppoint{gp mark 0}{(4.441,3.565)}
\gppoint{gp mark 0}{(4.441,2.958)}
\gppoint{gp mark 0}{(4.441,3.364)}
\gppoint{gp mark 0}{(4.441,3.697)}
\gppoint{gp mark 0}{(4.441,3.157)}
\gppoint{gp mark 0}{(4.441,3.408)}
\gppoint{gp mark 0}{(4.441,3.811)}
\gppoint{gp mark 0}{(4.441,3.364)}
\gppoint{gp mark 0}{(4.441,2.958)}
\gppoint{gp mark 0}{(4.441,2.880)}
\gppoint{gp mark 0}{(4.441,3.450)}
\gppoint{gp mark 0}{(4.441,3.364)}
\gppoint{gp mark 0}{(4.441,2.958)}
\gppoint{gp mark 0}{(4.441,3.634)}
\gppoint{gp mark 0}{(4.441,3.096)}
\gppoint{gp mark 0}{(4.441,3.267)}
\gppoint{gp mark 0}{(4.441,3.267)}
\gppoint{gp mark 0}{(4.441,3.529)}
\gppoint{gp mark 0}{(4.441,3.490)}
\gppoint{gp mark 0}{(4.441,2.958)}
\gppoint{gp mark 0}{(4.441,3.600)}
\gppoint{gp mark 0}{(4.441,3.364)}
\gppoint{gp mark 0}{(4.441,3.096)}
\gppoint{gp mark 0}{(4.441,3.364)}
\gppoint{gp mark 0}{(4.441,3.267)}
\gppoint{gp mark 0}{(4.441,3.408)}
\gppoint{gp mark 0}{(4.441,3.214)}
\gppoint{gp mark 0}{(4.441,2.586)}
\gppoint{gp mark 0}{(4.441,3.666)}
\gppoint{gp mark 0}{(4.441,2.958)}
\gppoint{gp mark 0}{(4.441,3.214)}
\gppoint{gp mark 0}{(4.441,2.586)}
\gppoint{gp mark 0}{(4.441,3.267)}
\gppoint{gp mark 0}{(4.441,3.030)}
\gppoint{gp mark 0}{(4.441,2.880)}
\gppoint{gp mark 0}{(4.441,3.030)}
\gppoint{gp mark 0}{(4.441,3.364)}
\gppoint{gp mark 0}{(4.441,3.267)}
\gppoint{gp mark 0}{(4.441,3.030)}
\gppoint{gp mark 0}{(4.441,3.565)}
\gppoint{gp mark 0}{(4.441,3.364)}
\gppoint{gp mark 0}{(4.441,3.267)}
\gppoint{gp mark 0}{(4.441,3.565)}
\gppoint{gp mark 0}{(4.441,3.450)}
\gppoint{gp mark 0}{(4.441,3.267)}
\gppoint{gp mark 0}{(4.441,3.267)}
\gppoint{gp mark 0}{(4.441,3.408)}
\gppoint{gp mark 0}{(4.441,3.364)}
\gppoint{gp mark 0}{(4.441,3.267)}
\gppoint{gp mark 0}{(4.441,3.096)}
\gppoint{gp mark 0}{(4.441,2.958)}
\gppoint{gp mark 0}{(4.441,3.267)}
\gppoint{gp mark 0}{(4.441,3.727)}
\gppoint{gp mark 0}{(4.441,3.529)}
\gppoint{gp mark 0}{(4.441,3.267)}
\gppoint{gp mark 0}{(4.441,3.364)}
\gppoint{gp mark 0}{(4.441,2.958)}
\gppoint{gp mark 0}{(4.441,3.267)}
\gppoint{gp mark 0}{(4.441,3.364)}
\gppoint{gp mark 0}{(4.441,3.267)}
\gppoint{gp mark 0}{(4.441,3.030)}
\gppoint{gp mark 0}{(4.441,2.696)}
\gppoint{gp mark 0}{(4.441,3.317)}
\gppoint{gp mark 0}{(4.441,3.096)}
\gppoint{gp mark 0}{(4.441,3.364)}
\gppoint{gp mark 0}{(4.441,3.157)}
\gppoint{gp mark 0}{(4.441,3.157)}
\gppoint{gp mark 0}{(4.441,3.267)}
\gppoint{gp mark 0}{(4.441,3.364)}
\gppoint{gp mark 0}{(4.441,3.490)}
\gppoint{gp mark 0}{(4.441,3.317)}
\gppoint{gp mark 0}{(4.441,3.450)}
\gppoint{gp mark 0}{(4.441,3.565)}
\gppoint{gp mark 0}{(4.441,3.214)}
\gppoint{gp mark 0}{(4.441,3.490)}
\gppoint{gp mark 0}{(4.441,3.364)}
\gppoint{gp mark 0}{(4.441,3.267)}
\gppoint{gp mark 0}{(4.441,3.267)}
\gppoint{gp mark 0}{(4.441,3.408)}
\gppoint{gp mark 0}{(4.441,3.450)}
\gppoint{gp mark 0}{(4.441,3.490)}
\gppoint{gp mark 0}{(4.441,3.364)}
\gppoint{gp mark 0}{(4.441,3.408)}
\gppoint{gp mark 0}{(4.441,2.696)}
\gppoint{gp mark 0}{(4.441,3.450)}
\gppoint{gp mark 0}{(4.441,3.267)}
\gppoint{gp mark 0}{(4.441,3.317)}
\gppoint{gp mark 0}{(4.441,3.157)}
\gppoint{gp mark 0}{(4.441,3.364)}
\gppoint{gp mark 0}{(4.441,3.364)}
\gppoint{gp mark 0}{(4.441,3.408)}
\gppoint{gp mark 0}{(4.441,3.030)}
\gppoint{gp mark 0}{(4.441,2.793)}
\gppoint{gp mark 0}{(4.441,3.490)}
\gppoint{gp mark 0}{(4.441,3.267)}
\gppoint{gp mark 0}{(4.441,2.880)}
\gppoint{gp mark 0}{(4.441,3.364)}
\gppoint{gp mark 0}{(4.441,3.214)}
\gppoint{gp mark 0}{(4.441,3.267)}
\gppoint{gp mark 0}{(4.441,3.784)}
\gppoint{gp mark 0}{(4.441,3.364)}
\gppoint{gp mark 0}{(4.441,3.214)}
\gppoint{gp mark 0}{(4.441,3.408)}
\gppoint{gp mark 0}{(4.441,3.157)}
\gppoint{gp mark 0}{(4.441,3.030)}
\gppoint{gp mark 0}{(4.441,3.364)}
\gppoint{gp mark 0}{(4.441,3.030)}
\gppoint{gp mark 0}{(4.441,3.267)}
\gppoint{gp mark 0}{(4.441,3.364)}
\gppoint{gp mark 0}{(4.441,3.634)}
\gppoint{gp mark 0}{(4.441,3.490)}
\gppoint{gp mark 0}{(4.441,3.096)}
\gppoint{gp mark 0}{(4.441,3.450)}
\gppoint{gp mark 0}{(4.441,3.214)}
\gppoint{gp mark 0}{(4.441,3.096)}
\gppoint{gp mark 0}{(4.441,3.408)}
\gppoint{gp mark 0}{(4.441,3.267)}
\gppoint{gp mark 0}{(4.441,2.696)}
\gppoint{gp mark 0}{(4.441,3.267)}
\gppoint{gp mark 0}{(4.441,3.214)}
\gppoint{gp mark 0}{(4.441,2.958)}
\gppoint{gp mark 0}{(4.441,3.267)}
\gppoint{gp mark 0}{(4.441,2.958)}
\gppoint{gp mark 0}{(4.441,3.634)}
\gppoint{gp mark 0}{(4.441,3.634)}
\gppoint{gp mark 0}{(4.441,2.958)}
\gppoint{gp mark 0}{(4.441,3.317)}
\gppoint{gp mark 0}{(4.441,3.267)}
\gppoint{gp mark 0}{(4.441,3.267)}
\gppoint{gp mark 0}{(4.441,3.565)}
\gppoint{gp mark 0}{(4.441,3.267)}
\gppoint{gp mark 0}{(4.441,2.958)}
\gppoint{gp mark 0}{(4.441,3.317)}
\gppoint{gp mark 0}{(4.441,3.408)}
\gppoint{gp mark 0}{(4.441,3.756)}
\gppoint{gp mark 0}{(4.441,2.880)}
\gppoint{gp mark 0}{(4.441,3.030)}
\gppoint{gp mark 0}{(4.441,3.317)}
\gppoint{gp mark 0}{(4.441,3.214)}
\gppoint{gp mark 0}{(4.441,3.697)}
\gppoint{gp mark 0}{(4.441,2.958)}
\gppoint{gp mark 0}{(4.441,3.490)}
\gppoint{gp mark 0}{(4.441,3.450)}
\gppoint{gp mark 0}{(4.441,3.600)}
\gppoint{gp mark 0}{(4.441,3.267)}
\gppoint{gp mark 0}{(4.441,2.958)}
\gppoint{gp mark 0}{(4.441,2.958)}
\gppoint{gp mark 0}{(4.441,3.862)}
\gppoint{gp mark 0}{(4.441,3.697)}
\gppoint{gp mark 0}{(4.441,3.666)}
\gppoint{gp mark 0}{(4.441,3.030)}
\gppoint{gp mark 0}{(4.441,3.666)}
\gppoint{gp mark 0}{(4.441,3.030)}
\gppoint{gp mark 0}{(4.441,3.214)}
\gppoint{gp mark 0}{(4.441,3.490)}
\gppoint{gp mark 0}{(4.441,3.267)}
\gppoint{gp mark 0}{(4.441,3.267)}
\gppoint{gp mark 0}{(4.441,3.030)}
\gppoint{gp mark 0}{(4.441,3.030)}
\gppoint{gp mark 0}{(4.441,3.157)}
\gppoint{gp mark 0}{(4.441,3.408)}
\gppoint{gp mark 0}{(4.441,3.267)}
\gppoint{gp mark 0}{(4.441,3.490)}
\gppoint{gp mark 0}{(4.441,3.529)}
\gppoint{gp mark 0}{(4.441,3.214)}
\gppoint{gp mark 0}{(4.441,2.880)}
\gppoint{gp mark 0}{(4.441,3.267)}
\gppoint{gp mark 0}{(4.441,3.911)}
\gppoint{gp mark 0}{(4.441,3.157)}
\gppoint{gp mark 0}{(4.441,3.214)}
\gppoint{gp mark 0}{(4.441,3.096)}
\gppoint{gp mark 0}{(4.441,4.118)}
\gppoint{gp mark 0}{(4.441,3.030)}
\gppoint{gp mark 0}{(4.441,3.214)}
\gppoint{gp mark 0}{(4.441,3.214)}
\gppoint{gp mark 0}{(4.441,3.784)}
\gppoint{gp mark 0}{(4.441,3.317)}
\gppoint{gp mark 0}{(4.441,3.756)}
\gppoint{gp mark 0}{(4.441,3.214)}
\gppoint{gp mark 0}{(4.441,3.030)}
\gppoint{gp mark 0}{(4.441,2.958)}
\gppoint{gp mark 0}{(4.441,3.157)}
\gppoint{gp mark 0}{(4.441,3.030)}
\gppoint{gp mark 0}{(4.441,3.756)}
\gppoint{gp mark 0}{(4.441,3.364)}
\gppoint{gp mark 0}{(4.441,3.565)}
\gppoint{gp mark 0}{(4.441,3.096)}
\gppoint{gp mark 0}{(4.441,3.096)}
\gppoint{gp mark 0}{(4.441,3.450)}
\gppoint{gp mark 0}{(4.441,3.214)}
\gppoint{gp mark 0}{(4.441,3.157)}
\gppoint{gp mark 0}{(4.441,3.267)}
\gppoint{gp mark 0}{(4.441,3.529)}
\gppoint{gp mark 0}{(4.441,2.880)}
\gppoint{gp mark 0}{(4.441,3.600)}
\gppoint{gp mark 0}{(4.441,3.267)}
\gppoint{gp mark 0}{(4.441,3.267)}
\gppoint{gp mark 0}{(4.441,3.364)}
\gppoint{gp mark 0}{(4.441,3.364)}
\gppoint{gp mark 0}{(4.441,3.267)}
\gppoint{gp mark 0}{(4.441,3.267)}
\gppoint{gp mark 0}{(4.441,3.600)}
\gppoint{gp mark 0}{(4.441,3.157)}
\gppoint{gp mark 0}{(4.441,3.096)}
\gppoint{gp mark 0}{(4.441,3.267)}
\gppoint{gp mark 0}{(4.441,3.600)}
\gppoint{gp mark 0}{(4.441,2.696)}
\gppoint{gp mark 0}{(4.441,3.408)}
\gppoint{gp mark 0}{(4.441,3.214)}
\gppoint{gp mark 0}{(4.441,3.408)}
\gppoint{gp mark 0}{(4.441,3.364)}
\gppoint{gp mark 0}{(4.441,3.157)}
\gppoint{gp mark 0}{(4.441,3.096)}
\gppoint{gp mark 0}{(4.441,3.096)}
\gppoint{gp mark 0}{(4.441,2.696)}
\gppoint{gp mark 0}{(4.441,3.096)}
\gppoint{gp mark 0}{(4.441,3.267)}
\gppoint{gp mark 0}{(4.441,3.096)}
\gppoint{gp mark 0}{(4.441,3.214)}
\gppoint{gp mark 0}{(4.441,3.096)}
\gppoint{gp mark 0}{(4.441,3.364)}
\gppoint{gp mark 0}{(4.441,2.958)}
\gppoint{gp mark 0}{(4.441,3.364)}
\gppoint{gp mark 0}{(4.441,2.880)}
\gppoint{gp mark 0}{(4.441,3.450)}
\gppoint{gp mark 0}{(4.441,3.214)}
\gppoint{gp mark 0}{(4.441,2.696)}
\gppoint{gp mark 0}{(4.441,3.666)}
\gppoint{gp mark 0}{(4.441,3.450)}
\gppoint{gp mark 0}{(4.441,3.030)}
\gppoint{gp mark 0}{(4.441,2.958)}
\gppoint{gp mark 0}{(4.441,3.096)}
\gppoint{gp mark 0}{(4.441,3.096)}
\gppoint{gp mark 0}{(4.441,3.364)}
\gppoint{gp mark 0}{(4.441,3.450)}
\gppoint{gp mark 0}{(4.441,3.408)}
\gppoint{gp mark 0}{(4.441,3.450)}
\gppoint{gp mark 0}{(4.441,3.364)}
\gppoint{gp mark 0}{(4.441,3.157)}
\gppoint{gp mark 0}{(4.441,3.157)}
\gppoint{gp mark 0}{(4.441,3.096)}
\gppoint{gp mark 0}{(4.441,3.727)}
\gppoint{gp mark 0}{(4.441,3.096)}
\gppoint{gp mark 0}{(4.441,2.793)}
\gppoint{gp mark 0}{(4.441,3.096)}
\gppoint{gp mark 0}{(4.441,3.408)}
\gppoint{gp mark 0}{(4.441,3.364)}
\gppoint{gp mark 0}{(4.441,3.096)}
\gppoint{gp mark 0}{(4.441,2.958)}
\gppoint{gp mark 0}{(4.441,3.364)}
\gppoint{gp mark 0}{(4.441,2.958)}
\gppoint{gp mark 0}{(4.441,2.958)}
\gppoint{gp mark 0}{(4.441,3.096)}
\gppoint{gp mark 0}{(4.441,3.666)}
\gppoint{gp mark 0}{(4.441,3.096)}
\gppoint{gp mark 0}{(4.441,3.408)}
\gppoint{gp mark 0}{(4.441,2.880)}
\gppoint{gp mark 0}{(4.441,3.727)}
\gppoint{gp mark 0}{(4.441,3.214)}
\gppoint{gp mark 0}{(4.441,2.958)}
\gppoint{gp mark 0}{(4.441,3.267)}
\gppoint{gp mark 0}{(4.441,3.408)}
\gppoint{gp mark 0}{(4.441,3.697)}
\gppoint{gp mark 0}{(4.441,3.267)}
\gppoint{gp mark 0}{(4.441,3.727)}
\gppoint{gp mark 0}{(4.441,3.214)}
\gppoint{gp mark 0}{(4.441,3.157)}
\gppoint{gp mark 0}{(4.441,3.408)}
\gppoint{gp mark 0}{(4.441,3.317)}
\gppoint{gp mark 0}{(4.441,3.096)}
\gppoint{gp mark 0}{(4.441,3.030)}
\gppoint{gp mark 0}{(4.441,3.096)}
\gppoint{gp mark 0}{(4.441,3.529)}
\gppoint{gp mark 0}{(4.441,3.096)}
\gppoint{gp mark 0}{(4.441,3.096)}
\gppoint{gp mark 0}{(4.441,2.696)}
\gppoint{gp mark 0}{(4.441,3.529)}
\gppoint{gp mark 0}{(4.441,3.157)}
\gppoint{gp mark 0}{(4.441,3.096)}
\gppoint{gp mark 0}{(4.441,3.096)}
\gppoint{gp mark 0}{(4.441,3.364)}
\gppoint{gp mark 0}{(4.441,3.096)}
\gppoint{gp mark 0}{(4.441,3.267)}
\gppoint{gp mark 0}{(4.522,3.267)}
\gppoint{gp mark 0}{(4.522,3.364)}
\gppoint{gp mark 0}{(4.522,3.600)}
\gppoint{gp mark 0}{(4.522,3.364)}
\gppoint{gp mark 0}{(4.522,3.529)}
\gppoint{gp mark 0}{(4.522,3.364)}
\gppoint{gp mark 0}{(4.522,3.267)}
\gppoint{gp mark 0}{(4.522,3.934)}
\gppoint{gp mark 0}{(4.522,3.364)}
\gppoint{gp mark 0}{(4.522,3.030)}
\gppoint{gp mark 0}{(4.522,3.364)}
\gppoint{gp mark 0}{(4.522,3.214)}
\gppoint{gp mark 0}{(4.522,3.214)}
\gppoint{gp mark 0}{(4.522,3.317)}
\gppoint{gp mark 0}{(4.522,3.529)}
\gppoint{gp mark 0}{(4.522,3.450)}
\gppoint{gp mark 0}{(4.522,2.958)}
\gppoint{gp mark 0}{(4.522,3.490)}
\gppoint{gp mark 0}{(4.522,3.030)}
\gppoint{gp mark 0}{(4.522,3.408)}
\gppoint{gp mark 0}{(4.522,3.030)}
\gppoint{gp mark 0}{(4.522,3.157)}
\gppoint{gp mark 0}{(4.522,3.214)}
\gppoint{gp mark 0}{(4.522,3.317)}
\gppoint{gp mark 0}{(4.522,2.880)}
\gppoint{gp mark 0}{(4.522,3.565)}
\gppoint{gp mark 0}{(4.522,3.565)}
\gppoint{gp mark 0}{(4.522,3.157)}
\gppoint{gp mark 0}{(4.522,2.880)}
\gppoint{gp mark 0}{(4.522,3.697)}
\gppoint{gp mark 0}{(4.522,3.317)}
\gppoint{gp mark 0}{(4.522,3.267)}
\gppoint{gp mark 0}{(4.522,3.157)}
\gppoint{gp mark 0}{(4.522,3.408)}
\gppoint{gp mark 0}{(4.522,3.214)}
\gppoint{gp mark 0}{(4.522,2.880)}
\gppoint{gp mark 0}{(4.522,3.267)}
\gppoint{gp mark 0}{(4.522,3.317)}
\gppoint{gp mark 0}{(4.522,3.214)}
\gppoint{gp mark 0}{(4.522,3.697)}
\gppoint{gp mark 0}{(4.522,3.490)}
\gppoint{gp mark 0}{(4.522,3.317)}
\gppoint{gp mark 0}{(4.522,2.958)}
\gppoint{gp mark 0}{(4.522,2.793)}
\gppoint{gp mark 0}{(4.522,2.793)}
\gppoint{gp mark 0}{(4.522,3.267)}
\gppoint{gp mark 0}{(4.522,3.096)}
\gppoint{gp mark 0}{(4.522,3.887)}
\gppoint{gp mark 0}{(4.522,3.157)}
\gppoint{gp mark 0}{(4.522,3.030)}
\gppoint{gp mark 0}{(4.522,3.634)}
\gppoint{gp mark 0}{(4.522,3.096)}
\gppoint{gp mark 0}{(4.522,3.600)}
\gppoint{gp mark 0}{(4.522,3.364)}
\gppoint{gp mark 0}{(4.522,3.267)}
\gppoint{gp mark 0}{(4.522,2.793)}
\gppoint{gp mark 0}{(4.522,2.880)}
\gppoint{gp mark 0}{(4.522,3.214)}
\gppoint{gp mark 0}{(4.522,3.364)}
\gppoint{gp mark 0}{(4.522,3.408)}
\gppoint{gp mark 0}{(4.522,3.490)}
\gppoint{gp mark 0}{(4.522,3.214)}
\gppoint{gp mark 0}{(4.522,2.880)}
\gppoint{gp mark 0}{(4.522,3.157)}
\gppoint{gp mark 0}{(4.522,3.408)}
\gppoint{gp mark 0}{(4.522,3.267)}
\gppoint{gp mark 0}{(4.522,3.317)}
\gppoint{gp mark 0}{(4.522,3.408)}
\gppoint{gp mark 0}{(4.522,2.958)}
\gppoint{gp mark 0}{(4.522,3.267)}
\gppoint{gp mark 0}{(4.522,3.214)}
\gppoint{gp mark 0}{(4.522,3.634)}
\gppoint{gp mark 0}{(4.522,3.157)}
\gppoint{gp mark 0}{(4.522,3.267)}
\gppoint{gp mark 0}{(4.522,3.408)}
\gppoint{gp mark 0}{(4.522,3.030)}
\gppoint{gp mark 0}{(4.522,3.364)}
\gppoint{gp mark 0}{(4.522,3.408)}
\gppoint{gp mark 0}{(4.522,3.096)}
\gppoint{gp mark 0}{(4.522,3.267)}
\gppoint{gp mark 0}{(4.522,3.364)}
\gppoint{gp mark 0}{(4.522,3.317)}
\gppoint{gp mark 0}{(4.522,3.096)}
\gppoint{gp mark 0}{(4.522,2.958)}
\gppoint{gp mark 0}{(4.522,3.408)}
\gppoint{gp mark 0}{(4.522,3.408)}
\gppoint{gp mark 0}{(4.522,4.153)}
\gppoint{gp mark 0}{(4.522,3.634)}
\gppoint{gp mark 0}{(4.522,3.364)}
\gppoint{gp mark 0}{(4.522,3.157)}
\gppoint{gp mark 0}{(4.522,3.666)}
\gppoint{gp mark 0}{(4.522,3.408)}
\gppoint{gp mark 0}{(4.522,3.214)}
\gppoint{gp mark 0}{(4.522,3.096)}
\gppoint{gp mark 0}{(4.522,3.214)}
\gppoint{gp mark 0}{(4.522,3.450)}
\gppoint{gp mark 0}{(4.522,3.364)}
\gppoint{gp mark 0}{(4.522,3.784)}
\gppoint{gp mark 0}{(4.522,3.450)}
\gppoint{gp mark 0}{(4.522,3.267)}
\gppoint{gp mark 0}{(4.522,3.030)}
\gppoint{gp mark 0}{(4.522,3.666)}
\gppoint{gp mark 0}{(4.522,3.030)}
\gppoint{gp mark 0}{(4.522,3.364)}
\gppoint{gp mark 0}{(4.522,3.157)}
\gppoint{gp mark 0}{(4.522,3.490)}
\gppoint{gp mark 0}{(4.522,2.586)}
\gppoint{gp mark 0}{(4.522,3.317)}
\gppoint{gp mark 0}{(4.522,3.490)}
\gppoint{gp mark 0}{(4.522,3.267)}
\gppoint{gp mark 0}{(4.522,3.157)}
\gppoint{gp mark 0}{(4.522,3.784)}
\gppoint{gp mark 0}{(4.522,3.600)}
\gppoint{gp mark 0}{(4.522,3.364)}
\gppoint{gp mark 0}{(4.522,3.096)}
\gppoint{gp mark 0}{(4.522,3.096)}
\gppoint{gp mark 0}{(4.522,3.565)}
\gppoint{gp mark 0}{(4.522,3.096)}
\gppoint{gp mark 0}{(4.522,3.408)}
\gppoint{gp mark 0}{(4.522,3.317)}
\gppoint{gp mark 0}{(4.522,3.364)}
\gppoint{gp mark 0}{(4.522,3.408)}
\gppoint{gp mark 0}{(4.522,3.157)}
\gppoint{gp mark 0}{(4.522,3.565)}
\gppoint{gp mark 0}{(4.522,3.565)}
\gppoint{gp mark 0}{(4.522,3.565)}
\gppoint{gp mark 0}{(4.522,3.364)}
\gppoint{gp mark 0}{(4.522,3.364)}
\gppoint{gp mark 0}{(4.522,3.490)}
\gppoint{gp mark 0}{(4.522,3.157)}
\gppoint{gp mark 0}{(4.522,3.030)}
\gppoint{gp mark 0}{(4.522,3.096)}
\gppoint{gp mark 0}{(4.522,3.214)}
\gppoint{gp mark 0}{(4.522,3.096)}
\gppoint{gp mark 0}{(4.522,3.490)}
\gppoint{gp mark 0}{(4.522,3.214)}
\gppoint{gp mark 0}{(4.522,3.096)}
\gppoint{gp mark 0}{(4.522,3.317)}
\gppoint{gp mark 0}{(4.522,3.157)}
\gppoint{gp mark 0}{(4.522,3.317)}
\gppoint{gp mark 0}{(4.522,3.214)}
\gppoint{gp mark 0}{(4.522,3.364)}
\gppoint{gp mark 0}{(4.522,3.408)}
\gppoint{gp mark 0}{(4.522,3.450)}
\gppoint{gp mark 0}{(4.522,3.317)}
\gppoint{gp mark 0}{(4.522,3.408)}
\gppoint{gp mark 0}{(4.522,3.634)}
\gppoint{gp mark 0}{(4.522,3.408)}
\gppoint{gp mark 0}{(4.522,3.634)}
\gppoint{gp mark 0}{(4.522,3.408)}
\gppoint{gp mark 0}{(4.522,3.450)}
\gppoint{gp mark 0}{(4.522,3.267)}
\gppoint{gp mark 0}{(4.522,3.317)}
\gppoint{gp mark 0}{(4.522,3.096)}
\gppoint{gp mark 0}{(4.522,3.408)}
\gppoint{gp mark 0}{(4.522,3.408)}
\gppoint{gp mark 0}{(4.522,3.450)}
\gppoint{gp mark 0}{(4.522,3.267)}
\gppoint{gp mark 0}{(4.522,3.214)}
\gppoint{gp mark 0}{(4.522,3.096)}
\gppoint{gp mark 0}{(4.522,3.364)}
\gppoint{gp mark 0}{(4.522,3.267)}
\gppoint{gp mark 0}{(4.522,3.364)}
\gppoint{gp mark 0}{(4.522,3.317)}
\gppoint{gp mark 0}{(4.522,3.157)}
\gppoint{gp mark 0}{(4.522,3.408)}
\gppoint{gp mark 0}{(4.522,3.267)}
\gppoint{gp mark 0}{(4.522,3.756)}
\gppoint{gp mark 0}{(4.522,3.317)}
\gppoint{gp mark 0}{(4.522,2.793)}
\gppoint{gp mark 0}{(4.522,3.214)}
\gppoint{gp mark 0}{(4.522,3.408)}
\gppoint{gp mark 0}{(4.522,3.529)}
\gppoint{gp mark 0}{(4.522,3.490)}
\gppoint{gp mark 0}{(4.522,3.096)}
\gppoint{gp mark 0}{(4.522,3.490)}
\gppoint{gp mark 0}{(4.522,3.529)}
\gppoint{gp mark 0}{(4.522,2.880)}
\gppoint{gp mark 0}{(4.522,3.784)}
\gppoint{gp mark 0}{(4.522,3.784)}
\gppoint{gp mark 0}{(4.522,3.364)}
\gppoint{gp mark 0}{(4.522,2.958)}
\gppoint{gp mark 0}{(4.522,3.490)}
\gppoint{gp mark 0}{(4.522,3.450)}
\gppoint{gp mark 0}{(4.522,3.317)}
\gppoint{gp mark 0}{(4.522,3.408)}
\gppoint{gp mark 0}{(4.522,3.157)}
\gppoint{gp mark 0}{(4.522,3.490)}
\gppoint{gp mark 0}{(4.522,2.958)}
\gppoint{gp mark 0}{(4.522,3.634)}
\gppoint{gp mark 0}{(4.522,3.450)}
\gppoint{gp mark 0}{(4.522,3.490)}
\gppoint{gp mark 0}{(4.522,3.784)}
\gppoint{gp mark 0}{(4.522,3.837)}
\gppoint{gp mark 0}{(4.522,3.408)}
\gppoint{gp mark 0}{(4.522,3.157)}
\gppoint{gp mark 0}{(4.522,2.793)}
\gppoint{gp mark 0}{(4.522,3.811)}
\gppoint{gp mark 0}{(4.522,3.364)}
\gppoint{gp mark 0}{(4.522,2.880)}
\gppoint{gp mark 0}{(4.522,3.317)}
\gppoint{gp mark 0}{(4.522,3.267)}
\gppoint{gp mark 0}{(4.522,3.214)}
\gppoint{gp mark 0}{(4.522,3.450)}
\gppoint{gp mark 0}{(4.522,3.364)}
\gppoint{gp mark 0}{(4.522,3.267)}
\gppoint{gp mark 0}{(4.522,3.157)}
\gppoint{gp mark 0}{(4.522,3.408)}
\gppoint{gp mark 0}{(4.522,3.157)}
\gppoint{gp mark 0}{(4.522,3.756)}
\gppoint{gp mark 0}{(4.522,3.490)}
\gppoint{gp mark 0}{(4.522,3.600)}
\gppoint{gp mark 0}{(4.522,3.157)}
\gppoint{gp mark 0}{(4.522,3.408)}
\gppoint{gp mark 0}{(4.522,3.490)}
\gppoint{gp mark 0}{(4.522,3.529)}
\gppoint{gp mark 0}{(4.522,3.317)}
\gppoint{gp mark 0}{(4.522,3.529)}
\gppoint{gp mark 0}{(4.522,3.529)}
\gppoint{gp mark 0}{(4.522,2.958)}
\gppoint{gp mark 0}{(4.522,3.364)}
\gppoint{gp mark 0}{(4.522,3.450)}
\gppoint{gp mark 0}{(4.522,3.364)}
\gppoint{gp mark 0}{(4.522,3.364)}
\gppoint{gp mark 0}{(4.522,3.267)}
\gppoint{gp mark 0}{(4.522,3.450)}
\gppoint{gp mark 0}{(4.522,3.450)}
\gppoint{gp mark 0}{(4.522,2.880)}
\gppoint{gp mark 0}{(4.522,3.096)}
\gppoint{gp mark 0}{(4.522,3.317)}
\gppoint{gp mark 0}{(4.522,3.214)}
\gppoint{gp mark 0}{(4.522,3.157)}
\gppoint{gp mark 0}{(4.522,3.267)}
\gppoint{gp mark 0}{(4.522,3.450)}
\gppoint{gp mark 0}{(4.522,3.364)}
\gppoint{gp mark 0}{(4.522,3.317)}
\gppoint{gp mark 0}{(4.522,2.793)}
\gppoint{gp mark 0}{(4.522,3.408)}
\gppoint{gp mark 0}{(4.522,3.267)}
\gppoint{gp mark 0}{(4.522,3.450)}
\gppoint{gp mark 0}{(4.522,3.157)}
\gppoint{gp mark 0}{(4.522,3.756)}
\gppoint{gp mark 0}{(4.522,3.490)}
\gppoint{gp mark 0}{(4.522,3.157)}
\gppoint{gp mark 0}{(4.522,3.697)}
\gppoint{gp mark 0}{(4.522,3.096)}
\gppoint{gp mark 0}{(4.522,3.317)}
\gppoint{gp mark 0}{(4.522,3.030)}
\gppoint{gp mark 0}{(4.522,3.811)}
\gppoint{gp mark 0}{(4.522,3.408)}
\gppoint{gp mark 0}{(4.522,3.030)}
\gppoint{gp mark 0}{(4.522,3.157)}
\gppoint{gp mark 0}{(4.522,3.600)}
\gppoint{gp mark 0}{(4.522,3.450)}
\gppoint{gp mark 0}{(4.522,3.450)}
\gppoint{gp mark 0}{(4.522,3.727)}
\gppoint{gp mark 0}{(4.522,3.784)}
\gppoint{gp mark 0}{(4.522,3.317)}
\gppoint{gp mark 0}{(4.522,3.450)}
\gppoint{gp mark 0}{(4.522,3.529)}
\gppoint{gp mark 0}{(4.522,3.408)}
\gppoint{gp mark 0}{(4.522,3.317)}
\gppoint{gp mark 0}{(4.522,3.096)}
\gppoint{gp mark 0}{(4.522,3.317)}
\gppoint{gp mark 0}{(4.522,3.450)}
\gppoint{gp mark 0}{(4.522,3.364)}
\gppoint{gp mark 0}{(4.522,3.267)}
\gppoint{gp mark 0}{(4.522,3.450)}
\gppoint{gp mark 0}{(4.522,3.784)}
\gppoint{gp mark 0}{(4.522,3.214)}
\gppoint{gp mark 0}{(4.522,3.317)}
\gppoint{gp mark 0}{(4.522,3.529)}
\gppoint{gp mark 0}{(4.522,3.317)}
\gppoint{gp mark 0}{(4.522,3.565)}
\gppoint{gp mark 0}{(4.522,3.364)}
\gppoint{gp mark 0}{(4.522,3.408)}
\gppoint{gp mark 0}{(4.522,3.157)}
\gppoint{gp mark 0}{(4.522,3.267)}
\gppoint{gp mark 0}{(4.522,3.784)}
\gppoint{gp mark 0}{(4.522,4.283)}
\gppoint{gp mark 0}{(4.522,3.408)}
\gppoint{gp mark 0}{(4.522,3.267)}
\gppoint{gp mark 0}{(4.522,3.267)}
\gppoint{gp mark 0}{(4.522,3.756)}
\gppoint{gp mark 0}{(4.522,2.958)}
\gppoint{gp mark 0}{(4.522,3.364)}
\gppoint{gp mark 0}{(4.522,3.030)}
\gppoint{gp mark 0}{(4.522,3.317)}
\gppoint{gp mark 0}{(4.522,3.317)}
\gppoint{gp mark 0}{(4.522,3.666)}
\gppoint{gp mark 0}{(4.522,3.157)}
\gppoint{gp mark 0}{(4.522,3.634)}
\gppoint{gp mark 0}{(4.522,3.364)}
\gppoint{gp mark 0}{(4.522,3.634)}
\gppoint{gp mark 0}{(4.522,3.490)}
\gppoint{gp mark 0}{(4.522,3.364)}
\gppoint{gp mark 0}{(4.522,3.317)}
\gppoint{gp mark 0}{(4.522,3.450)}
\gppoint{gp mark 0}{(4.522,3.490)}
\gppoint{gp mark 0}{(4.522,3.490)}
\gppoint{gp mark 0}{(4.522,3.364)}
\gppoint{gp mark 0}{(4.522,3.317)}
\gppoint{gp mark 0}{(4.522,3.364)}
\gppoint{gp mark 0}{(4.522,3.727)}
\gppoint{gp mark 0}{(4.522,3.267)}
\gppoint{gp mark 0}{(4.522,3.364)}
\gppoint{gp mark 0}{(4.522,3.408)}
\gppoint{gp mark 0}{(4.522,3.408)}
\gppoint{gp mark 0}{(4.522,3.911)}
\gppoint{gp mark 0}{(4.522,3.364)}
\gppoint{gp mark 0}{(4.522,3.096)}
\gppoint{gp mark 0}{(4.522,3.756)}
\gppoint{gp mark 0}{(4.522,3.096)}
\gppoint{gp mark 0}{(4.522,3.267)}
\gppoint{gp mark 0}{(4.522,3.408)}
\gppoint{gp mark 0}{(4.522,3.408)}
\gppoint{gp mark 0}{(4.522,3.096)}
\gppoint{gp mark 0}{(4.522,3.096)}
\gppoint{gp mark 0}{(4.522,3.030)}
\gppoint{gp mark 0}{(4.522,3.267)}
\gppoint{gp mark 0}{(4.522,3.450)}
\gppoint{gp mark 0}{(4.522,2.880)}
\gppoint{gp mark 0}{(4.522,3.096)}
\gppoint{gp mark 0}{(4.522,3.214)}
\gppoint{gp mark 0}{(4.522,3.030)}
\gppoint{gp mark 0}{(4.522,3.317)}
\gppoint{gp mark 0}{(4.522,3.267)}
\gppoint{gp mark 0}{(4.522,3.317)}
\gppoint{gp mark 0}{(4.522,3.030)}
\gppoint{gp mark 0}{(4.522,3.096)}
\gppoint{gp mark 0}{(4.522,3.600)}
\gppoint{gp mark 0}{(4.522,3.490)}
\gppoint{gp mark 0}{(4.522,3.408)}
\gppoint{gp mark 0}{(4.522,2.958)}
\gppoint{gp mark 0}{(4.522,3.267)}
\gppoint{gp mark 0}{(4.522,3.634)}
\gppoint{gp mark 0}{(4.522,2.880)}
\gppoint{gp mark 0}{(4.522,3.529)}
\gppoint{gp mark 0}{(4.522,3.214)}
\gppoint{gp mark 0}{(4.522,3.408)}
\gppoint{gp mark 0}{(4.522,3.408)}
\gppoint{gp mark 0}{(4.522,3.450)}
\gppoint{gp mark 0}{(4.522,3.634)}
\gppoint{gp mark 0}{(4.522,3.214)}
\gppoint{gp mark 0}{(4.522,3.214)}
\gppoint{gp mark 0}{(4.522,2.880)}
\gppoint{gp mark 0}{(4.522,3.317)}
\gppoint{gp mark 0}{(4.522,3.096)}
\gppoint{gp mark 0}{(4.522,3.450)}
\gppoint{gp mark 0}{(4.522,3.697)}
\gppoint{gp mark 0}{(4.522,3.096)}
\gppoint{gp mark 0}{(4.522,2.696)}
\gppoint{gp mark 0}{(4.522,3.157)}
\gppoint{gp mark 0}{(4.522,3.157)}
\gppoint{gp mark 0}{(4.522,3.529)}
\gppoint{gp mark 0}{(4.522,3.317)}
\gppoint{gp mark 0}{(4.522,3.364)}
\gppoint{gp mark 0}{(4.522,3.490)}
\gppoint{gp mark 0}{(4.522,3.096)}
\gppoint{gp mark 0}{(4.522,3.490)}
\gppoint{gp mark 0}{(4.522,3.666)}
\gppoint{gp mark 0}{(4.522,3.364)}
\gppoint{gp mark 0}{(4.522,3.450)}
\gppoint{gp mark 0}{(4.522,3.490)}
\gppoint{gp mark 0}{(4.522,3.697)}
\gppoint{gp mark 0}{(4.522,3.450)}
\gppoint{gp mark 0}{(4.522,3.408)}
\gppoint{gp mark 0}{(4.522,3.030)}
\gppoint{gp mark 0}{(4.522,3.267)}
\gppoint{gp mark 0}{(4.522,3.157)}
\gppoint{gp mark 0}{(4.522,3.214)}
\gppoint{gp mark 0}{(4.522,3.450)}
\gppoint{gp mark 0}{(4.522,3.529)}
\gppoint{gp mark 0}{(4.522,2.958)}
\gppoint{gp mark 0}{(4.522,3.317)}
\gppoint{gp mark 0}{(4.522,3.214)}
\gppoint{gp mark 0}{(4.522,3.317)}
\gppoint{gp mark 0}{(4.522,3.214)}
\gppoint{gp mark 0}{(4.522,2.958)}
\gppoint{gp mark 0}{(4.522,3.490)}
\gppoint{gp mark 0}{(4.522,3.408)}
\gppoint{gp mark 0}{(4.522,3.317)}
\gppoint{gp mark 0}{(4.522,3.157)}
\gppoint{gp mark 0}{(4.522,2.958)}
\gppoint{gp mark 0}{(4.522,3.214)}
\gppoint{gp mark 0}{(4.522,3.408)}
\gppoint{gp mark 0}{(4.522,3.490)}
\gppoint{gp mark 0}{(4.522,3.030)}
\gppoint{gp mark 0}{(4.522,3.030)}
\gppoint{gp mark 0}{(4.522,3.408)}
\gppoint{gp mark 0}{(4.522,3.267)}
\gppoint{gp mark 0}{(4.522,3.364)}
\gppoint{gp mark 0}{(4.522,3.364)}
\gppoint{gp mark 0}{(4.522,3.214)}
\gppoint{gp mark 0}{(4.522,3.317)}
\gppoint{gp mark 0}{(4.522,3.529)}
\gppoint{gp mark 0}{(4.522,3.697)}
\gppoint{gp mark 0}{(4.522,3.317)}
\gppoint{gp mark 0}{(4.522,3.634)}
\gppoint{gp mark 0}{(4.522,3.490)}
\gppoint{gp mark 0}{(4.522,3.634)}
\gppoint{gp mark 0}{(4.522,3.490)}
\gppoint{gp mark 0}{(4.522,3.030)}
\gppoint{gp mark 0}{(4.522,3.030)}
\gppoint{gp mark 0}{(4.522,3.364)}
\gppoint{gp mark 0}{(4.522,3.727)}
\gppoint{gp mark 0}{(4.522,3.887)}
\gppoint{gp mark 0}{(4.522,3.157)}
\gppoint{gp mark 0}{(4.522,3.490)}
\gppoint{gp mark 0}{(4.522,3.634)}
\gppoint{gp mark 0}{(4.522,3.214)}
\gppoint{gp mark 0}{(4.522,3.450)}
\gppoint{gp mark 0}{(4.522,3.600)}
\gppoint{gp mark 0}{(4.522,3.490)}
\gppoint{gp mark 0}{(4.522,3.267)}
\gppoint{gp mark 0}{(4.522,3.214)}
\gppoint{gp mark 0}{(4.522,3.317)}
\gppoint{gp mark 0}{(4.522,3.317)}
\gppoint{gp mark 0}{(4.522,2.793)}
\gppoint{gp mark 0}{(4.522,3.317)}
\gppoint{gp mark 0}{(4.522,3.408)}
\gppoint{gp mark 0}{(4.522,3.408)}
\gppoint{gp mark 0}{(4.522,3.666)}
\gppoint{gp mark 0}{(4.522,3.214)}
\gppoint{gp mark 0}{(4.522,3.214)}
\gppoint{gp mark 0}{(4.522,3.030)}
\gppoint{gp mark 0}{(4.522,3.157)}
\gppoint{gp mark 0}{(4.522,3.267)}
\gppoint{gp mark 0}{(4.522,3.887)}
\gppoint{gp mark 0}{(4.522,3.450)}
\gppoint{gp mark 0}{(4.522,3.490)}
\gppoint{gp mark 0}{(4.522,2.958)}
\gppoint{gp mark 0}{(4.522,3.267)}
\gppoint{gp mark 0}{(4.522,3.267)}
\gppoint{gp mark 0}{(4.522,3.267)}
\gppoint{gp mark 0}{(4.522,3.756)}
\gppoint{gp mark 0}{(4.522,3.214)}
\gppoint{gp mark 0}{(4.522,3.490)}
\gppoint{gp mark 0}{(4.522,3.267)}
\gppoint{gp mark 0}{(4.522,3.408)}
\gppoint{gp mark 0}{(4.522,3.317)}
\gppoint{gp mark 0}{(4.522,3.450)}
\gppoint{gp mark 0}{(4.522,3.490)}
\gppoint{gp mark 0}{(4.522,3.096)}
\gppoint{gp mark 0}{(4.522,2.958)}
\gppoint{gp mark 0}{(4.522,3.214)}
\gppoint{gp mark 0}{(4.522,3.214)}
\gppoint{gp mark 0}{(4.522,3.862)}
\gppoint{gp mark 0}{(4.522,3.214)}
\gppoint{gp mark 0}{(4.522,3.364)}
\gppoint{gp mark 0}{(4.522,2.958)}
\gppoint{gp mark 0}{(4.522,3.450)}
\gppoint{gp mark 0}{(4.522,3.634)}
\gppoint{gp mark 0}{(4.522,3.096)}
\gppoint{gp mark 0}{(4.522,4.420)}
\gppoint{gp mark 0}{(4.522,3.727)}
\gppoint{gp mark 0}{(4.522,3.317)}
\gppoint{gp mark 0}{(4.522,3.267)}
\gppoint{gp mark 0}{(4.522,3.490)}
\gppoint{gp mark 0}{(4.522,3.450)}
\gppoint{gp mark 0}{(4.522,3.157)}
\gppoint{gp mark 0}{(4.522,2.958)}
\gppoint{gp mark 0}{(4.522,3.267)}
\gppoint{gp mark 0}{(4.598,3.529)}
\gppoint{gp mark 0}{(4.598,3.450)}
\gppoint{gp mark 0}{(4.598,3.666)}
\gppoint{gp mark 0}{(4.598,3.267)}
\gppoint{gp mark 0}{(4.598,3.317)}
\gppoint{gp mark 0}{(4.598,3.317)}
\gppoint{gp mark 0}{(4.598,3.267)}
\gppoint{gp mark 0}{(4.598,3.317)}
\gppoint{gp mark 0}{(4.598,3.317)}
\gppoint{gp mark 0}{(4.598,3.317)}
\gppoint{gp mark 0}{(4.598,3.450)}
\gppoint{gp mark 0}{(4.598,2.880)}
\gppoint{gp mark 0}{(4.598,3.408)}
\gppoint{gp mark 0}{(4.598,3.317)}
\gppoint{gp mark 0}{(4.598,3.030)}
\gppoint{gp mark 0}{(4.598,3.317)}
\gppoint{gp mark 0}{(4.598,3.317)}
\gppoint{gp mark 0}{(4.598,3.317)}
\gppoint{gp mark 0}{(4.598,3.317)}
\gppoint{gp mark 0}{(4.598,3.096)}
\gppoint{gp mark 0}{(4.598,3.317)}
\gppoint{gp mark 0}{(4.598,3.317)}
\gppoint{gp mark 0}{(4.598,3.157)}
\gppoint{gp mark 0}{(4.598,3.317)}
\gppoint{gp mark 0}{(4.598,3.157)}
\gppoint{gp mark 0}{(4.598,3.317)}
\gppoint{gp mark 0}{(4.598,3.317)}
\gppoint{gp mark 0}{(4.598,3.697)}
\gppoint{gp mark 0}{(4.598,3.490)}
\gppoint{gp mark 0}{(4.598,3.490)}
\gppoint{gp mark 0}{(4.598,3.317)}
\gppoint{gp mark 0}{(4.598,3.317)}
\gppoint{gp mark 0}{(4.598,3.317)}
\gppoint{gp mark 0}{(4.598,3.317)}
\gppoint{gp mark 0}{(4.598,3.450)}
\gppoint{gp mark 0}{(4.598,3.267)}
\gppoint{gp mark 0}{(4.598,3.934)}
\gppoint{gp mark 0}{(4.598,2.958)}
\gppoint{gp mark 0}{(4.598,3.408)}
\gppoint{gp mark 0}{(4.598,3.214)}
\gppoint{gp mark 0}{(4.598,3.157)}
\gppoint{gp mark 0}{(4.598,3.096)}
\gppoint{gp mark 0}{(4.598,3.317)}
\gppoint{gp mark 0}{(4.598,3.267)}
\gppoint{gp mark 0}{(4.598,3.666)}
\gppoint{gp mark 0}{(4.598,3.317)}
\gppoint{gp mark 0}{(4.598,3.317)}
\gppoint{gp mark 0}{(4.598,3.634)}
\gppoint{gp mark 0}{(4.598,3.317)}
\gppoint{gp mark 0}{(4.598,3.317)}
\gppoint{gp mark 0}{(4.598,3.317)}
\gppoint{gp mark 0}{(4.598,3.408)}
\gppoint{gp mark 0}{(4.598,3.317)}
\gppoint{gp mark 0}{(4.598,3.490)}
\gppoint{gp mark 0}{(4.598,3.408)}
\gppoint{gp mark 0}{(4.598,3.408)}
\gppoint{gp mark 0}{(4.598,3.934)}
\gppoint{gp mark 0}{(4.598,3.529)}
\gppoint{gp mark 0}{(4.598,3.784)}
\gppoint{gp mark 0}{(4.598,3.364)}
\gppoint{gp mark 0}{(4.598,3.529)}
\gppoint{gp mark 0}{(4.598,3.490)}
\gppoint{gp mark 0}{(4.598,3.529)}
\gppoint{gp mark 0}{(4.598,3.634)}
\gppoint{gp mark 0}{(4.598,3.529)}
\gppoint{gp mark 0}{(4.598,3.317)}
\gppoint{gp mark 0}{(4.598,3.490)}
\gppoint{gp mark 0}{(4.598,3.450)}
\gppoint{gp mark 0}{(4.598,3.490)}
\gppoint{gp mark 0}{(4.598,3.267)}
\gppoint{gp mark 0}{(4.598,3.666)}
\gppoint{gp mark 0}{(4.598,3.317)}
\gppoint{gp mark 0}{(4.598,3.490)}
\gppoint{gp mark 0}{(4.598,3.317)}
\gppoint{gp mark 0}{(4.598,3.214)}
\gppoint{gp mark 0}{(4.598,3.267)}
\gppoint{gp mark 0}{(4.598,3.490)}
\gppoint{gp mark 0}{(4.598,3.096)}
\gppoint{gp mark 0}{(4.598,3.214)}
\gppoint{gp mark 0}{(4.598,4.061)}
\gppoint{gp mark 0}{(4.598,3.317)}
\gppoint{gp mark 0}{(4.598,3.267)}
\gppoint{gp mark 0}{(4.598,3.529)}
\gppoint{gp mark 0}{(4.598,3.784)}
\gppoint{gp mark 0}{(4.598,3.157)}
\gppoint{gp mark 0}{(4.598,3.214)}
\gppoint{gp mark 0}{(4.598,3.157)}
\gppoint{gp mark 0}{(4.598,3.565)}
\gppoint{gp mark 0}{(4.598,3.600)}
\gppoint{gp mark 0}{(4.598,3.267)}
\gppoint{gp mark 0}{(4.598,3.267)}
\gppoint{gp mark 0}{(4.598,3.267)}
\gppoint{gp mark 0}{(4.598,3.364)}
\gppoint{gp mark 0}{(4.598,3.490)}
\gppoint{gp mark 0}{(4.598,3.529)}
\gppoint{gp mark 0}{(4.598,3.157)}
\gppoint{gp mark 0}{(4.598,3.157)}
\gppoint{gp mark 0}{(4.598,3.267)}
\gppoint{gp mark 0}{(4.598,3.364)}
\gppoint{gp mark 0}{(4.598,3.565)}
\gppoint{gp mark 0}{(4.598,3.364)}
\gppoint{gp mark 0}{(4.598,3.214)}
\gppoint{gp mark 0}{(4.598,3.490)}
\gppoint{gp mark 0}{(4.598,3.157)}
\gppoint{gp mark 0}{(4.598,3.157)}
\gppoint{gp mark 0}{(4.598,3.408)}
\gppoint{gp mark 0}{(4.598,3.490)}
\gppoint{gp mark 0}{(4.598,3.214)}
\gppoint{gp mark 0}{(4.598,3.214)}
\gppoint{gp mark 0}{(4.598,3.634)}
\gppoint{gp mark 0}{(4.598,3.565)}
\gppoint{gp mark 0}{(4.598,3.837)}
\gppoint{gp mark 0}{(4.598,3.529)}
\gppoint{gp mark 0}{(4.598,3.450)}
\gppoint{gp mark 0}{(4.598,3.600)}
\gppoint{gp mark 0}{(4.598,3.529)}
\gppoint{gp mark 0}{(4.598,3.364)}
\gppoint{gp mark 0}{(4.598,3.600)}
\gppoint{gp mark 0}{(4.598,3.214)}
\gppoint{gp mark 0}{(4.598,3.364)}
\gppoint{gp mark 0}{(4.598,3.490)}
\gppoint{gp mark 0}{(4.598,3.364)}
\gppoint{gp mark 0}{(4.598,3.364)}
\gppoint{gp mark 0}{(4.598,3.565)}
\gppoint{gp mark 0}{(4.598,3.450)}
\gppoint{gp mark 0}{(4.598,3.565)}
\gppoint{gp mark 0}{(4.598,3.096)}
\gppoint{gp mark 0}{(4.598,3.408)}
\gppoint{gp mark 0}{(4.598,3.529)}
\gppoint{gp mark 0}{(4.598,3.364)}
\gppoint{gp mark 0}{(4.598,3.157)}
\gppoint{gp mark 0}{(4.598,3.634)}
\gppoint{gp mark 0}{(4.598,3.408)}
\gppoint{gp mark 0}{(4.598,3.450)}
\gppoint{gp mark 0}{(4.598,3.364)}
\gppoint{gp mark 0}{(4.598,3.364)}
\gppoint{gp mark 0}{(4.598,3.634)}
\gppoint{gp mark 0}{(4.598,3.214)}
\gppoint{gp mark 0}{(4.598,2.958)}
\gppoint{gp mark 0}{(4.598,3.364)}
\gppoint{gp mark 0}{(4.598,3.450)}
\gppoint{gp mark 0}{(4.598,3.364)}
\gppoint{gp mark 0}{(4.598,3.529)}
\gppoint{gp mark 0}{(4.598,3.364)}
\gppoint{gp mark 0}{(4.598,3.030)}
\gppoint{gp mark 0}{(4.598,2.958)}
\gppoint{gp mark 0}{(4.598,3.565)}
\gppoint{gp mark 0}{(4.598,3.157)}
\gppoint{gp mark 0}{(4.598,3.490)}
\gppoint{gp mark 0}{(4.598,3.364)}
\gppoint{gp mark 0}{(4.598,3.157)}
\gppoint{gp mark 0}{(4.598,3.364)}
\gppoint{gp mark 0}{(4.598,3.600)}
\gppoint{gp mark 0}{(4.598,3.490)}
\gppoint{gp mark 0}{(4.598,3.214)}
\gppoint{gp mark 0}{(4.598,3.030)}
\gppoint{gp mark 0}{(4.598,3.214)}
\gppoint{gp mark 0}{(4.598,3.364)}
\gppoint{gp mark 0}{(4.598,3.364)}
\gppoint{gp mark 0}{(4.598,3.666)}
\gppoint{gp mark 0}{(4.598,3.214)}
\gppoint{gp mark 0}{(4.598,3.096)}
\gppoint{gp mark 0}{(4.598,3.697)}
\gppoint{gp mark 0}{(4.598,3.565)}
\gppoint{gp mark 0}{(4.598,3.364)}
\gppoint{gp mark 0}{(4.598,3.666)}
\gppoint{gp mark 0}{(4.598,3.837)}
\gppoint{gp mark 0}{(4.598,3.490)}
\gppoint{gp mark 0}{(4.598,3.096)}
\gppoint{gp mark 0}{(4.598,3.214)}
\gppoint{gp mark 0}{(4.598,3.697)}
\gppoint{gp mark 0}{(4.598,3.267)}
\gppoint{gp mark 0}{(4.598,3.934)}
\gppoint{gp mark 0}{(4.598,3.267)}
\gppoint{gp mark 0}{(4.598,3.364)}
\gppoint{gp mark 0}{(4.598,3.529)}
\gppoint{gp mark 0}{(4.598,3.364)}
\gppoint{gp mark 0}{(4.598,3.666)}
\gppoint{gp mark 0}{(4.598,3.600)}
\gppoint{gp mark 0}{(4.598,3.450)}
\gppoint{gp mark 0}{(4.598,3.364)}
\gppoint{gp mark 0}{(4.598,3.837)}
\gppoint{gp mark 0}{(4.598,3.157)}
\gppoint{gp mark 0}{(4.598,3.317)}
\gppoint{gp mark 0}{(4.598,3.408)}
\gppoint{gp mark 0}{(4.598,3.364)}
\gppoint{gp mark 0}{(4.598,2.958)}
\gppoint{gp mark 0}{(4.598,3.666)}
\gppoint{gp mark 0}{(4.598,3.267)}
\gppoint{gp mark 0}{(4.598,3.450)}
\gppoint{gp mark 0}{(4.598,3.267)}
\gppoint{gp mark 0}{(4.598,3.697)}
\gppoint{gp mark 0}{(4.598,3.565)}
\gppoint{gp mark 0}{(4.598,3.267)}
\gppoint{gp mark 0}{(4.598,3.157)}
\gppoint{gp mark 0}{(4.598,3.317)}
\gppoint{gp mark 0}{(4.598,3.450)}
\gppoint{gp mark 0}{(4.598,3.600)}
\gppoint{gp mark 0}{(4.598,3.408)}
\gppoint{gp mark 0}{(4.598,3.666)}
\gppoint{gp mark 0}{(4.598,3.157)}
\gppoint{gp mark 0}{(4.598,3.529)}
\gppoint{gp mark 0}{(4.598,3.030)}
\gppoint{gp mark 0}{(4.598,2.958)}
\gppoint{gp mark 0}{(4.598,3.529)}
\gppoint{gp mark 0}{(4.598,3.157)}
\gppoint{gp mark 0}{(4.598,3.666)}
\gppoint{gp mark 0}{(4.598,3.096)}
\gppoint{gp mark 0}{(4.598,3.634)}
\gppoint{gp mark 0}{(4.598,3.634)}
\gppoint{gp mark 0}{(4.598,3.267)}
\gppoint{gp mark 0}{(4.598,4.118)}
\gppoint{gp mark 0}{(4.598,3.364)}
\gppoint{gp mark 0}{(4.598,3.096)}
\gppoint{gp mark 0}{(4.598,3.490)}
\gppoint{gp mark 0}{(4.598,2.880)}
\gppoint{gp mark 0}{(4.598,3.529)}
\gppoint{gp mark 0}{(4.598,3.096)}
\gppoint{gp mark 0}{(4.598,3.317)}
\gppoint{gp mark 0}{(4.598,3.529)}
\gppoint{gp mark 0}{(4.598,3.364)}
\gppoint{gp mark 0}{(4.598,3.030)}
\gppoint{gp mark 0}{(4.598,3.529)}
\gppoint{gp mark 0}{(4.598,3.408)}
\gppoint{gp mark 0}{(4.598,3.490)}
\gppoint{gp mark 0}{(4.598,3.600)}
\gppoint{gp mark 0}{(4.598,3.600)}
\gppoint{gp mark 0}{(4.598,3.450)}
\gppoint{gp mark 0}{(4.598,3.364)}
\gppoint{gp mark 0}{(4.598,3.364)}
\gppoint{gp mark 0}{(4.598,3.317)}
\gppoint{gp mark 0}{(4.598,3.529)}
\gppoint{gp mark 0}{(4.598,3.214)}
\gppoint{gp mark 0}{(4.598,3.408)}
\gppoint{gp mark 0}{(4.598,3.267)}
\gppoint{gp mark 0}{(4.598,3.697)}
\gppoint{gp mark 0}{(4.598,3.490)}
\gppoint{gp mark 0}{(4.598,2.958)}
\gppoint{gp mark 0}{(4.598,3.157)}
\gppoint{gp mark 0}{(4.598,2.958)}
\gppoint{gp mark 0}{(4.598,3.529)}
\gppoint{gp mark 0}{(4.598,3.490)}
\gppoint{gp mark 0}{(4.598,3.408)}
\gppoint{gp mark 0}{(4.598,3.267)}
\gppoint{gp mark 0}{(4.598,3.267)}
\gppoint{gp mark 0}{(4.598,3.408)}
\gppoint{gp mark 0}{(4.598,3.317)}
\gppoint{gp mark 0}{(4.598,3.364)}
\gppoint{gp mark 0}{(4.598,3.096)}
\gppoint{gp mark 0}{(4.598,3.267)}
\gppoint{gp mark 0}{(4.598,3.408)}
\gppoint{gp mark 0}{(4.598,3.784)}
\gppoint{gp mark 0}{(4.598,3.887)}
\gppoint{gp mark 0}{(4.598,3.157)}
\gppoint{gp mark 0}{(4.598,3.267)}
\gppoint{gp mark 0}{(4.598,2.793)}
\gppoint{gp mark 0}{(4.598,3.267)}
\gppoint{gp mark 0}{(4.598,2.958)}
\gppoint{gp mark 0}{(4.598,3.317)}
\gppoint{gp mark 0}{(4.598,3.364)}
\gppoint{gp mark 0}{(4.598,3.490)}
\gppoint{gp mark 0}{(4.598,3.317)}
\gppoint{gp mark 0}{(4.598,3.490)}
\gppoint{gp mark 0}{(4.598,3.450)}
\gppoint{gp mark 0}{(4.598,3.408)}
\gppoint{gp mark 0}{(4.598,3.096)}
\gppoint{gp mark 0}{(4.598,3.096)}
\gppoint{gp mark 0}{(4.598,3.529)}
\gppoint{gp mark 0}{(4.598,2.880)}
\gppoint{gp mark 0}{(4.598,3.727)}
\gppoint{gp mark 0}{(4.598,3.096)}
\gppoint{gp mark 0}{(4.598,3.450)}
\gppoint{gp mark 0}{(4.598,3.529)}
\gppoint{gp mark 0}{(4.598,3.529)}
\gppoint{gp mark 0}{(4.598,3.600)}
\gppoint{gp mark 0}{(4.598,3.030)}
\gppoint{gp mark 0}{(4.598,3.267)}
\gppoint{gp mark 0}{(4.598,3.030)}
\gppoint{gp mark 0}{(4.598,3.490)}
\gppoint{gp mark 0}{(4.598,3.634)}
\gppoint{gp mark 0}{(4.598,2.958)}
\gppoint{gp mark 0}{(4.598,3.317)}
\gppoint{gp mark 0}{(4.598,3.157)}
\gppoint{gp mark 0}{(4.598,2.958)}
\gppoint{gp mark 0}{(4.598,3.317)}
\gppoint{gp mark 0}{(4.598,3.666)}
\gppoint{gp mark 0}{(4.598,3.267)}
\gppoint{gp mark 0}{(4.598,3.490)}
\gppoint{gp mark 0}{(4.598,3.450)}
\gppoint{gp mark 0}{(4.598,3.364)}
\gppoint{gp mark 0}{(4.598,3.214)}
\gppoint{gp mark 0}{(4.598,3.267)}
\gppoint{gp mark 0}{(4.598,3.490)}
\gppoint{gp mark 0}{(4.598,3.450)}
\gppoint{gp mark 0}{(4.598,3.364)}
\gppoint{gp mark 0}{(4.598,2.958)}
\gppoint{gp mark 0}{(4.598,3.267)}
\gppoint{gp mark 0}{(4.598,3.634)}
\gppoint{gp mark 0}{(4.598,3.634)}
\gppoint{gp mark 0}{(4.598,3.317)}
\gppoint{gp mark 0}{(4.598,2.958)}
\gppoint{gp mark 0}{(4.598,3.529)}
\gppoint{gp mark 0}{(4.598,3.364)}
\gppoint{gp mark 0}{(4.598,4.171)}
\gppoint{gp mark 0}{(4.598,3.490)}
\gppoint{gp mark 0}{(4.598,3.317)}
\gppoint{gp mark 0}{(4.598,3.364)}
\gppoint{gp mark 0}{(4.598,3.811)}
\gppoint{gp mark 0}{(4.598,3.697)}
\gppoint{gp mark 0}{(4.598,3.450)}
\gppoint{gp mark 0}{(4.598,3.529)}
\gppoint{gp mark 0}{(4.598,3.450)}
\gppoint{gp mark 0}{(4.598,3.490)}
\gppoint{gp mark 0}{(4.598,3.364)}
\gppoint{gp mark 0}{(4.598,3.450)}
\gppoint{gp mark 0}{(4.598,3.666)}
\gppoint{gp mark 0}{(4.598,3.096)}
\gppoint{gp mark 0}{(4.598,3.450)}
\gppoint{gp mark 0}{(4.598,3.490)}
\gppoint{gp mark 0}{(4.598,3.408)}
\gppoint{gp mark 0}{(4.598,3.364)}
\gppoint{gp mark 0}{(4.598,3.408)}
\gppoint{gp mark 0}{(4.598,3.317)}
\gppoint{gp mark 0}{(4.598,3.811)}
\gppoint{gp mark 0}{(4.598,3.030)}
\gppoint{gp mark 0}{(4.598,3.408)}
\gppoint{gp mark 0}{(4.598,3.450)}
\gppoint{gp mark 0}{(4.598,3.450)}
\gppoint{gp mark 0}{(4.598,3.214)}
\gppoint{gp mark 0}{(4.598,3.030)}
\gppoint{gp mark 0}{(4.598,3.214)}
\gppoint{gp mark 0}{(4.598,3.096)}
\gppoint{gp mark 0}{(4.598,3.317)}
\gppoint{gp mark 0}{(4.598,3.364)}
\gppoint{gp mark 0}{(4.598,3.727)}
\gppoint{gp mark 0}{(4.598,3.267)}
\gppoint{gp mark 0}{(4.598,3.565)}
\gppoint{gp mark 0}{(4.598,3.364)}
\gppoint{gp mark 0}{(4.598,3.957)}
\gppoint{gp mark 0}{(4.598,2.958)}
\gppoint{gp mark 0}{(4.598,3.565)}
\gppoint{gp mark 0}{(4.598,3.450)}
\gppoint{gp mark 0}{(4.598,3.697)}
\gppoint{gp mark 0}{(4.598,3.697)}
\gppoint{gp mark 0}{(4.598,3.450)}
\gppoint{gp mark 0}{(4.598,3.490)}
\gppoint{gp mark 0}{(4.598,3.529)}
\gppoint{gp mark 0}{(4.598,3.408)}
\gppoint{gp mark 0}{(4.598,3.408)}
\gppoint{gp mark 0}{(4.598,3.364)}
\gppoint{gp mark 0}{(4.598,3.450)}
\gppoint{gp mark 0}{(4.598,3.030)}
\gppoint{gp mark 0}{(4.598,3.565)}
\gppoint{gp mark 0}{(4.598,3.030)}
\gppoint{gp mark 0}{(4.598,3.811)}
\gppoint{gp mark 0}{(4.598,3.490)}
\gppoint{gp mark 0}{(4.598,3.697)}
\gppoint{gp mark 0}{(4.598,3.267)}
\gppoint{gp mark 0}{(4.598,3.267)}
\gppoint{gp mark 0}{(4.598,3.490)}
\gppoint{gp mark 0}{(4.598,3.565)}
\gppoint{gp mark 0}{(4.598,3.030)}
\gppoint{gp mark 0}{(4.598,3.837)}
\gppoint{gp mark 0}{(4.598,3.666)}
\gppoint{gp mark 0}{(4.598,3.862)}
\gppoint{gp mark 0}{(4.598,3.030)}
\gppoint{gp mark 0}{(4.598,3.214)}
\gppoint{gp mark 0}{(4.598,3.157)}
\gppoint{gp mark 0}{(4.598,3.450)}
\gppoint{gp mark 0}{(4.598,4.118)}
\gppoint{gp mark 0}{(4.598,4.118)}
\gppoint{gp mark 0}{(4.598,3.666)}
\gppoint{gp mark 0}{(4.598,3.317)}
\gppoint{gp mark 0}{(4.598,3.267)}
\gppoint{gp mark 0}{(4.598,3.364)}
\gppoint{gp mark 0}{(4.598,3.157)}
\gppoint{gp mark 0}{(4.598,3.096)}
\gppoint{gp mark 0}{(4.598,3.317)}
\gppoint{gp mark 0}{(4.598,3.600)}
\gppoint{gp mark 0}{(4.598,2.880)}
\gppoint{gp mark 0}{(4.598,3.030)}
\gppoint{gp mark 0}{(4.598,3.214)}
\gppoint{gp mark 0}{(4.598,3.030)}
\gppoint{gp mark 0}{(4.598,3.529)}
\gppoint{gp mark 0}{(4.598,3.408)}
\gppoint{gp mark 0}{(4.598,2.958)}
\gppoint{gp mark 0}{(4.598,3.565)}
\gppoint{gp mark 0}{(4.598,3.364)}
\gppoint{gp mark 0}{(4.598,3.408)}
\gppoint{gp mark 0}{(4.598,2.958)}
\gppoint{gp mark 0}{(4.598,3.529)}
\gppoint{gp mark 0}{(4.598,3.317)}
\gppoint{gp mark 0}{(4.598,3.565)}
\gppoint{gp mark 0}{(4.598,3.756)}
\gppoint{gp mark 0}{(4.598,3.317)}
\gppoint{gp mark 0}{(4.598,3.408)}
\gppoint{gp mark 0}{(4.598,3.030)}
\gppoint{gp mark 0}{(4.598,3.030)}
\gppoint{gp mark 0}{(4.598,3.317)}
\gppoint{gp mark 0}{(4.598,3.600)}
\gppoint{gp mark 0}{(4.598,3.490)}
\gppoint{gp mark 0}{(4.598,3.267)}
\gppoint{gp mark 0}{(4.598,3.565)}
\gppoint{gp mark 0}{(4.598,3.317)}
\gppoint{gp mark 0}{(4.598,3.666)}
\gppoint{gp mark 0}{(4.598,3.811)}
\gppoint{gp mark 0}{(4.598,3.157)}
\gppoint{gp mark 0}{(4.598,3.364)}
\gppoint{gp mark 0}{(4.598,3.157)}
\gppoint{gp mark 0}{(4.598,3.666)}
\gppoint{gp mark 0}{(4.598,3.634)}
\gppoint{gp mark 0}{(4.598,2.958)}
\gppoint{gp mark 0}{(4.598,3.317)}
\gppoint{gp mark 0}{(4.598,3.450)}
\gppoint{gp mark 0}{(4.598,2.958)}
\gppoint{gp mark 0}{(4.598,3.096)}
\gppoint{gp mark 0}{(4.598,3.600)}
\gppoint{gp mark 0}{(4.598,2.958)}
\gppoint{gp mark 0}{(4.598,3.490)}
\gppoint{gp mark 0}{(4.598,3.450)}
\gppoint{gp mark 0}{(4.598,3.267)}
\gppoint{gp mark 0}{(4.598,3.600)}
\gppoint{gp mark 0}{(4.598,3.450)}
\gppoint{gp mark 0}{(4.598,3.666)}
\gppoint{gp mark 0}{(4.598,3.157)}
\gppoint{gp mark 0}{(4.670,3.267)}
\gppoint{gp mark 0}{(4.670,3.364)}
\gppoint{gp mark 0}{(4.670,3.565)}
\gppoint{gp mark 0}{(4.670,2.793)}
\gppoint{gp mark 0}{(4.670,3.317)}
\gppoint{gp mark 0}{(4.670,3.317)}
\gppoint{gp mark 0}{(4.670,3.317)}
\gppoint{gp mark 0}{(4.670,3.317)}
\gppoint{gp mark 0}{(4.670,4.153)}
\gppoint{gp mark 0}{(4.670,3.317)}
\gppoint{gp mark 0}{(4.670,3.565)}
\gppoint{gp mark 0}{(4.670,3.096)}
\gppoint{gp mark 0}{(4.670,3.565)}
\gppoint{gp mark 0}{(4.670,3.317)}
\gppoint{gp mark 0}{(4.670,3.565)}
\gppoint{gp mark 0}{(4.670,3.634)}
\gppoint{gp mark 0}{(4.670,3.096)}
\gppoint{gp mark 0}{(4.670,3.317)}
\gppoint{gp mark 0}{(4.670,3.157)}
\gppoint{gp mark 0}{(4.670,3.600)}
\gppoint{gp mark 0}{(4.670,3.030)}
\gppoint{gp mark 0}{(4.670,3.666)}
\gppoint{gp mark 0}{(4.670,3.214)}
\gppoint{gp mark 0}{(4.670,3.030)}
\gppoint{gp mark 0}{(4.670,3.529)}
\gppoint{gp mark 0}{(4.670,3.408)}
\gppoint{gp mark 0}{(4.670,3.634)}
\gppoint{gp mark 0}{(4.670,3.267)}
\gppoint{gp mark 0}{(4.670,3.317)}
\gppoint{gp mark 0}{(4.670,3.784)}
\gppoint{gp mark 0}{(4.670,3.364)}
\gppoint{gp mark 0}{(4.670,3.364)}
\gppoint{gp mark 0}{(4.670,2.880)}
\gppoint{gp mark 0}{(4.670,3.666)}
\gppoint{gp mark 0}{(4.670,3.408)}
\gppoint{gp mark 0}{(4.670,3.157)}
\gppoint{gp mark 0}{(4.670,3.408)}
\gppoint{gp mark 0}{(4.670,3.450)}
\gppoint{gp mark 0}{(4.670,3.887)}
\gppoint{gp mark 0}{(4.670,3.317)}
\gppoint{gp mark 0}{(4.670,3.408)}
\gppoint{gp mark 0}{(4.670,3.450)}
\gppoint{gp mark 0}{(4.670,3.600)}
\gppoint{gp mark 0}{(4.670,3.317)}
\gppoint{gp mark 0}{(4.670,3.096)}
\gppoint{gp mark 0}{(4.670,3.317)}
\gppoint{gp mark 0}{(4.670,3.364)}
\gppoint{gp mark 0}{(4.670,3.811)}
\gppoint{gp mark 0}{(4.670,3.317)}
\gppoint{gp mark 0}{(4.670,3.157)}
\gppoint{gp mark 0}{(4.670,3.157)}
\gppoint{gp mark 0}{(4.670,3.634)}
\gppoint{gp mark 0}{(4.670,3.529)}
\gppoint{gp mark 0}{(4.670,3.408)}
\gppoint{gp mark 0}{(4.670,3.784)}
\gppoint{gp mark 0}{(4.670,3.214)}
\gppoint{gp mark 0}{(4.670,3.565)}
\gppoint{gp mark 0}{(4.670,3.317)}
\gppoint{gp mark 0}{(4.670,3.267)}
\gppoint{gp mark 0}{(4.670,2.793)}
\gppoint{gp mark 0}{(4.670,3.364)}
\gppoint{gp mark 0}{(4.670,2.958)}
\gppoint{gp mark 0}{(4.670,3.666)}
\gppoint{gp mark 0}{(4.670,3.157)}
\gppoint{gp mark 0}{(4.670,3.529)}
\gppoint{gp mark 0}{(4.670,3.697)}
\gppoint{gp mark 0}{(4.670,3.214)}
\gppoint{gp mark 0}{(4.670,3.408)}
\gppoint{gp mark 0}{(4.670,3.267)}
\gppoint{gp mark 0}{(4.670,3.214)}
\gppoint{gp mark 0}{(4.670,3.490)}
\gppoint{gp mark 0}{(4.670,3.450)}
\gppoint{gp mark 0}{(4.670,3.697)}
\gppoint{gp mark 0}{(4.670,2.958)}
\gppoint{gp mark 0}{(4.670,3.408)}
\gppoint{gp mark 0}{(4.670,3.727)}
\gppoint{gp mark 0}{(4.670,3.408)}
\gppoint{gp mark 0}{(4.670,3.408)}
\gppoint{gp mark 0}{(4.670,3.157)}
\gppoint{gp mark 0}{(4.670,3.450)}
\gppoint{gp mark 0}{(4.670,3.030)}
\gppoint{gp mark 0}{(4.670,3.214)}
\gppoint{gp mark 0}{(4.670,3.214)}
\gppoint{gp mark 0}{(4.670,3.450)}
\gppoint{gp mark 0}{(4.670,3.634)}
\gppoint{gp mark 0}{(4.670,3.214)}
\gppoint{gp mark 0}{(4.670,3.450)}
\gppoint{gp mark 0}{(4.670,3.214)}
\gppoint{gp mark 0}{(4.670,3.214)}
\gppoint{gp mark 0}{(4.670,3.214)}
\gppoint{gp mark 0}{(4.670,3.317)}
\gppoint{gp mark 0}{(4.670,3.317)}
\gppoint{gp mark 0}{(4.670,3.862)}
\gppoint{gp mark 0}{(4.670,3.214)}
\gppoint{gp mark 0}{(4.670,3.317)}
\gppoint{gp mark 0}{(4.670,3.317)}
\gppoint{gp mark 0}{(4.670,3.450)}
\gppoint{gp mark 0}{(4.670,3.317)}
\gppoint{gp mark 0}{(4.670,3.634)}
\gppoint{gp mark 0}{(4.670,3.267)}
\gppoint{gp mark 0}{(4.670,3.529)}
\gppoint{gp mark 0}{(4.670,3.364)}
\gppoint{gp mark 0}{(4.670,3.450)}
\gppoint{gp mark 0}{(4.670,3.317)}
\gppoint{gp mark 0}{(4.670,3.214)}
\gppoint{gp mark 0}{(4.670,3.450)}
\gppoint{gp mark 0}{(4.670,2.958)}
\gppoint{gp mark 0}{(4.670,3.317)}
\gppoint{gp mark 0}{(4.670,3.529)}
\gppoint{gp mark 0}{(4.670,3.317)}
\gppoint{gp mark 0}{(4.670,3.030)}
\gppoint{gp mark 0}{(4.670,3.666)}
\gppoint{gp mark 0}{(4.670,3.887)}
\gppoint{gp mark 0}{(4.670,3.364)}
\gppoint{gp mark 0}{(4.670,3.267)}
\gppoint{gp mark 0}{(4.670,3.408)}
\gppoint{gp mark 0}{(4.670,3.408)}
\gppoint{gp mark 0}{(4.670,3.364)}
\gppoint{gp mark 0}{(4.670,3.666)}
\gppoint{gp mark 0}{(4.670,3.317)}
\gppoint{gp mark 0}{(4.670,3.450)}
\gppoint{gp mark 0}{(4.670,3.666)}
\gppoint{gp mark 0}{(4.670,3.317)}
\gppoint{gp mark 0}{(4.670,3.811)}
\gppoint{gp mark 0}{(4.670,3.317)}
\gppoint{gp mark 0}{(4.670,3.317)}
\gppoint{gp mark 0}{(4.670,3.267)}
\gppoint{gp mark 0}{(4.670,3.317)}
\gppoint{gp mark 0}{(4.670,3.317)}
\gppoint{gp mark 0}{(4.670,3.450)}
\gppoint{gp mark 0}{(4.670,3.529)}
\gppoint{gp mark 0}{(4.670,2.880)}
\gppoint{gp mark 0}{(4.670,3.317)}
\gppoint{gp mark 0}{(4.670,3.634)}
\gppoint{gp mark 0}{(4.670,3.096)}
\gppoint{gp mark 0}{(4.670,3.408)}
\gppoint{gp mark 0}{(4.670,3.666)}
\gppoint{gp mark 0}{(4.670,3.408)}
\gppoint{gp mark 0}{(4.670,3.600)}
\gppoint{gp mark 0}{(4.670,3.529)}
\gppoint{gp mark 0}{(4.670,3.666)}
\gppoint{gp mark 0}{(4.670,3.364)}
\gppoint{gp mark 0}{(4.670,3.214)}
\gppoint{gp mark 0}{(4.670,3.030)}
\gppoint{gp mark 0}{(4.670,3.267)}
\gppoint{gp mark 0}{(4.670,3.214)}
\gppoint{gp mark 0}{(4.670,3.214)}
\gppoint{gp mark 0}{(4.670,3.565)}
\gppoint{gp mark 0}{(4.670,3.490)}
\gppoint{gp mark 0}{(4.670,3.317)}
\gppoint{gp mark 0}{(4.670,3.267)}
\gppoint{gp mark 0}{(4.670,3.364)}
\gppoint{gp mark 0}{(4.670,3.634)}
\gppoint{gp mark 0}{(4.670,3.317)}
\gppoint{gp mark 0}{(4.670,3.600)}
\gppoint{gp mark 0}{(4.670,3.911)}
\gppoint{gp mark 0}{(4.670,3.364)}
\gppoint{gp mark 0}{(4.670,3.529)}
\gppoint{gp mark 0}{(4.670,3.408)}
\gppoint{gp mark 0}{(4.670,3.214)}
\gppoint{gp mark 0}{(4.670,3.267)}
\gppoint{gp mark 0}{(4.670,3.490)}
\gppoint{gp mark 0}{(4.670,2.880)}
\gppoint{gp mark 0}{(4.670,3.666)}
\gppoint{gp mark 0}{(4.670,3.490)}
\gppoint{gp mark 0}{(4.670,3.317)}
\gppoint{gp mark 0}{(4.670,3.317)}
\gppoint{gp mark 0}{(4.670,4.000)}
\gppoint{gp mark 0}{(4.670,3.529)}
\gppoint{gp mark 0}{(4.670,3.317)}
\gppoint{gp mark 0}{(4.670,3.529)}
\gppoint{gp mark 0}{(4.670,3.214)}
\gppoint{gp mark 0}{(4.670,3.317)}
\gppoint{gp mark 0}{(4.670,3.157)}
\gppoint{gp mark 0}{(4.670,3.214)}
\gppoint{gp mark 0}{(4.670,3.408)}
\gppoint{gp mark 0}{(4.670,3.317)}
\gppoint{gp mark 0}{(4.670,3.364)}
\gppoint{gp mark 0}{(4.670,3.364)}
\gppoint{gp mark 0}{(4.670,3.267)}
\gppoint{gp mark 0}{(4.670,3.157)}
\gppoint{gp mark 0}{(4.670,3.030)}
\gppoint{gp mark 0}{(4.670,3.756)}
\gppoint{gp mark 0}{(4.670,3.317)}
\gppoint{gp mark 0}{(4.670,3.450)}
\gppoint{gp mark 0}{(4.670,3.214)}
\gppoint{gp mark 0}{(4.670,3.529)}
\gppoint{gp mark 0}{(4.670,3.157)}
\gppoint{gp mark 0}{(4.670,3.408)}
\gppoint{gp mark 0}{(4.670,3.364)}
\gppoint{gp mark 0}{(4.670,3.727)}
\gppoint{gp mark 0}{(4.670,3.157)}
\gppoint{gp mark 0}{(4.670,3.408)}
\gppoint{gp mark 0}{(4.670,3.317)}
\gppoint{gp mark 0}{(4.670,3.490)}
\gppoint{gp mark 0}{(4.670,3.317)}
\gppoint{gp mark 0}{(4.670,3.214)}
\gppoint{gp mark 0}{(4.670,3.214)}
\gppoint{gp mark 0}{(4.670,3.529)}
\gppoint{gp mark 0}{(4.670,3.317)}
\gppoint{gp mark 0}{(4.670,3.364)}
\gppoint{gp mark 0}{(4.670,3.565)}
\gppoint{gp mark 0}{(4.670,3.214)}
\gppoint{gp mark 0}{(4.670,3.756)}
\gppoint{gp mark 0}{(4.670,3.756)}
\gppoint{gp mark 0}{(4.670,3.364)}
\gppoint{gp mark 0}{(4.670,3.490)}
\gppoint{gp mark 0}{(4.670,3.364)}
\gppoint{gp mark 0}{(4.670,3.157)}
\gppoint{gp mark 0}{(4.670,3.490)}
\gppoint{gp mark 0}{(4.670,2.793)}
\gppoint{gp mark 0}{(4.670,3.529)}
\gppoint{gp mark 0}{(4.670,3.317)}
\gppoint{gp mark 0}{(4.670,3.214)}
\gppoint{gp mark 0}{(4.670,3.214)}
\gppoint{gp mark 0}{(4.670,3.267)}
\gppoint{gp mark 0}{(4.670,3.364)}
\gppoint{gp mark 0}{(4.670,3.529)}
\gppoint{gp mark 0}{(4.670,3.364)}
\gppoint{gp mark 0}{(4.670,3.267)}
\gppoint{gp mark 0}{(4.670,3.756)}
\gppoint{gp mark 0}{(4.670,3.364)}
\gppoint{gp mark 0}{(4.670,3.267)}
\gppoint{gp mark 0}{(4.670,3.364)}
\gppoint{gp mark 0}{(4.670,3.214)}
\gppoint{gp mark 0}{(4.670,3.565)}
\gppoint{gp mark 0}{(4.670,2.880)}
\gppoint{gp mark 0}{(4.670,3.408)}
\gppoint{gp mark 0}{(4.670,4.021)}
\gppoint{gp mark 0}{(4.670,3.317)}
\gppoint{gp mark 0}{(4.670,3.317)}
\gppoint{gp mark 0}{(4.670,3.364)}
\gppoint{gp mark 0}{(4.670,3.408)}
\gppoint{gp mark 0}{(4.670,3.364)}
\gppoint{gp mark 0}{(4.670,3.529)}
\gppoint{gp mark 0}{(4.670,3.697)}
\gppoint{gp mark 0}{(4.670,3.408)}
\gppoint{gp mark 0}{(4.670,3.490)}
\gppoint{gp mark 0}{(4.670,3.529)}
\gppoint{gp mark 0}{(4.670,3.408)}
\gppoint{gp mark 0}{(4.670,3.529)}
\gppoint{gp mark 0}{(4.670,3.214)}
\gppoint{gp mark 0}{(4.670,3.364)}
\gppoint{gp mark 0}{(4.670,3.317)}
\gppoint{gp mark 0}{(4.670,3.267)}
\gppoint{gp mark 0}{(4.670,3.214)}
\gppoint{gp mark 0}{(4.670,3.317)}
\gppoint{gp mark 0}{(4.670,3.317)}
\gppoint{gp mark 0}{(4.670,3.408)}
\gppoint{gp mark 0}{(4.670,3.214)}
\gppoint{gp mark 0}{(4.670,3.450)}
\gppoint{gp mark 0}{(4.670,3.450)}
\gppoint{gp mark 0}{(4.670,3.450)}
\gppoint{gp mark 0}{(4.670,3.364)}
\gppoint{gp mark 0}{(4.670,3.666)}
\gppoint{gp mark 0}{(4.670,3.697)}
\gppoint{gp mark 0}{(4.670,3.450)}
\gppoint{gp mark 0}{(4.670,3.666)}
\gppoint{gp mark 0}{(4.670,3.317)}
\gppoint{gp mark 0}{(4.670,3.317)}
\gppoint{gp mark 0}{(4.670,3.317)}
\gppoint{gp mark 0}{(4.670,3.317)}
\gppoint{gp mark 0}{(4.670,3.214)}
\gppoint{gp mark 0}{(4.670,3.317)}
\gppoint{gp mark 0}{(4.670,3.267)}
\gppoint{gp mark 0}{(4.670,2.958)}
\gppoint{gp mark 0}{(4.670,3.317)}
\gppoint{gp mark 0}{(4.670,3.727)}
\gppoint{gp mark 0}{(4.670,3.317)}
\gppoint{gp mark 0}{(4.670,3.364)}
\gppoint{gp mark 0}{(4.670,3.727)}
\gppoint{gp mark 0}{(4.670,3.317)}
\gppoint{gp mark 0}{(4.670,3.157)}
\gppoint{gp mark 0}{(4.670,3.934)}
\gppoint{gp mark 0}{(4.670,3.364)}
\gppoint{gp mark 0}{(4.670,3.267)}
\gppoint{gp mark 0}{(4.670,3.214)}
\gppoint{gp mark 0}{(4.670,3.450)}
\gppoint{gp mark 0}{(4.670,3.214)}
\gppoint{gp mark 0}{(4.670,3.317)}
\gppoint{gp mark 0}{(4.670,3.317)}
\gppoint{gp mark 0}{(4.670,3.317)}
\gppoint{gp mark 0}{(4.670,3.408)}
\gppoint{gp mark 0}{(4.670,3.887)}
\gppoint{gp mark 0}{(4.670,3.267)}
\gppoint{gp mark 0}{(4.670,3.364)}
\gppoint{gp mark 0}{(4.670,3.408)}
\gppoint{gp mark 0}{(4.670,3.600)}
\gppoint{gp mark 0}{(4.670,3.408)}
\gppoint{gp mark 0}{(4.670,3.408)}
\gppoint{gp mark 0}{(4.670,3.214)}
\gppoint{gp mark 0}{(4.670,3.214)}
\gppoint{gp mark 0}{(4.670,3.887)}
\gppoint{gp mark 0}{(4.670,3.529)}
\gppoint{gp mark 0}{(4.670,3.267)}
\gppoint{gp mark 0}{(4.670,3.214)}
\gppoint{gp mark 0}{(4.670,2.958)}
\gppoint{gp mark 0}{(4.670,3.529)}
\gppoint{gp mark 0}{(4.670,3.450)}
\gppoint{gp mark 0}{(4.670,3.450)}
\gppoint{gp mark 0}{(4.670,3.811)}
\gppoint{gp mark 0}{(4.670,3.529)}
\gppoint{gp mark 0}{(4.670,3.157)}
\gppoint{gp mark 0}{(4.670,3.600)}
\gppoint{gp mark 0}{(4.670,3.529)}
\gppoint{gp mark 0}{(4.670,3.565)}
\gppoint{gp mark 0}{(4.670,2.958)}
\gppoint{gp mark 0}{(4.670,3.267)}
\gppoint{gp mark 0}{(4.670,3.096)}
\gppoint{gp mark 0}{(4.670,3.862)}
\gppoint{gp mark 0}{(4.670,3.600)}
\gppoint{gp mark 0}{(4.670,3.490)}
\gppoint{gp mark 0}{(4.670,3.364)}
\gppoint{gp mark 0}{(4.670,3.634)}
\gppoint{gp mark 0}{(4.670,3.267)}
\gppoint{gp mark 0}{(4.670,3.490)}
\gppoint{gp mark 0}{(4.670,3.600)}
\gppoint{gp mark 0}{(4.670,3.214)}
\gppoint{gp mark 0}{(4.670,3.565)}
\gppoint{gp mark 0}{(4.670,3.364)}
\gppoint{gp mark 0}{(4.670,3.157)}
\gppoint{gp mark 0}{(4.670,3.450)}
\gppoint{gp mark 0}{(4.670,3.408)}
\gppoint{gp mark 0}{(4.670,3.408)}
\gppoint{gp mark 0}{(4.670,3.408)}
\gppoint{gp mark 0}{(4.670,4.368)}
\gppoint{gp mark 0}{(4.670,3.408)}
\gppoint{gp mark 0}{(4.670,3.408)}
\gppoint{gp mark 0}{(4.670,3.408)}
\gppoint{gp mark 0}{(4.670,3.214)}
\gppoint{gp mark 0}{(4.670,4.354)}
\gppoint{gp mark 0}{(4.670,3.214)}
\gppoint{gp mark 0}{(4.670,3.727)}
\gppoint{gp mark 0}{(4.670,3.214)}
\gppoint{gp mark 0}{(4.670,3.490)}
\gppoint{gp mark 0}{(4.670,3.756)}
\gppoint{gp mark 0}{(4.670,3.529)}
\gppoint{gp mark 0}{(4.670,3.490)}
\gppoint{gp mark 0}{(4.670,3.666)}
\gppoint{gp mark 0}{(4.670,3.666)}
\gppoint{gp mark 0}{(4.670,3.214)}
\gppoint{gp mark 0}{(4.670,2.958)}
\gppoint{gp mark 0}{(4.670,3.529)}
\gppoint{gp mark 0}{(4.670,3.364)}
\gppoint{gp mark 0}{(4.670,3.364)}
\gppoint{gp mark 0}{(4.670,3.490)}
\gppoint{gp mark 0}{(4.670,3.529)}
\gppoint{gp mark 0}{(4.670,3.529)}
\gppoint{gp mark 0}{(4.670,3.529)}
\gppoint{gp mark 0}{(4.670,3.267)}
\gppoint{gp mark 0}{(4.670,3.529)}
\gppoint{gp mark 0}{(4.670,3.317)}
\gppoint{gp mark 0}{(4.670,3.529)}
\gppoint{gp mark 0}{(4.670,3.634)}
\gppoint{gp mark 0}{(4.670,3.317)}
\gppoint{gp mark 0}{(4.670,3.317)}
\gppoint{gp mark 0}{(4.670,3.529)}
\gppoint{gp mark 0}{(4.670,3.529)}
\gppoint{gp mark 0}{(4.670,4.000)}
\gppoint{gp mark 0}{(4.670,2.696)}
\gppoint{gp mark 0}{(4.670,3.727)}
\gppoint{gp mark 0}{(4.670,3.529)}
\gppoint{gp mark 0}{(4.670,3.529)}
\gppoint{gp mark 0}{(4.670,3.030)}
\gppoint{gp mark 0}{(4.670,3.157)}
\gppoint{gp mark 0}{(4.670,3.529)}
\gppoint{gp mark 0}{(4.670,3.529)}
\gppoint{gp mark 0}{(4.670,3.529)}
\gppoint{gp mark 0}{(4.670,3.529)}
\gppoint{gp mark 0}{(4.670,3.364)}
\gppoint{gp mark 0}{(4.670,3.529)}
\gppoint{gp mark 0}{(4.670,3.529)}
\gppoint{gp mark 0}{(4.670,3.529)}
\gppoint{gp mark 0}{(4.670,3.529)}
\gppoint{gp mark 0}{(4.670,3.529)}
\gppoint{gp mark 0}{(4.670,3.030)}
\gppoint{gp mark 0}{(4.670,3.096)}
\gppoint{gp mark 0}{(4.670,3.096)}
\gppoint{gp mark 0}{(4.670,4.061)}
\gppoint{gp mark 0}{(4.670,3.490)}
\gppoint{gp mark 0}{(4.670,3.214)}
\gppoint{gp mark 0}{(4.670,3.529)}
\gppoint{gp mark 0}{(4.670,4.080)}
\gppoint{gp mark 0}{(4.670,3.666)}
\gppoint{gp mark 0}{(4.670,3.317)}
\gppoint{gp mark 0}{(4.670,3.887)}
\gppoint{gp mark 0}{(4.670,3.096)}
\gppoint{gp mark 0}{(4.670,3.317)}
\gppoint{gp mark 0}{(4.670,3.450)}
\gppoint{gp mark 0}{(4.670,3.450)}
\gppoint{gp mark 0}{(4.670,3.529)}
\gppoint{gp mark 0}{(4.670,3.529)}
\gppoint{gp mark 0}{(4.670,3.450)}
\gppoint{gp mark 0}{(4.670,3.096)}
\gppoint{gp mark 0}{(4.670,3.450)}
\gppoint{gp mark 0}{(4.670,3.214)}
\gppoint{gp mark 0}{(4.670,3.364)}
\gppoint{gp mark 0}{(4.670,3.364)}
\gppoint{gp mark 0}{(4.670,3.267)}
\gppoint{gp mark 0}{(4.670,3.317)}
\gppoint{gp mark 0}{(4.670,3.450)}
\gppoint{gp mark 0}{(4.670,3.837)}
\gppoint{gp mark 0}{(4.670,3.157)}
\gppoint{gp mark 0}{(4.670,3.837)}
\gppoint{gp mark 0}{(4.670,3.565)}
\gppoint{gp mark 0}{(4.670,4.021)}
\gppoint{gp mark 0}{(4.670,3.214)}
\gppoint{gp mark 0}{(4.670,3.096)}
\gppoint{gp mark 0}{(4.670,3.529)}
\gppoint{gp mark 0}{(4.670,3.529)}
\gppoint{gp mark 0}{(4.670,3.267)}
\gppoint{gp mark 0}{(4.670,3.267)}
\gppoint{gp mark 0}{(4.670,3.408)}
\gppoint{gp mark 0}{(4.670,3.634)}
\gppoint{gp mark 0}{(4.670,3.529)}
\gppoint{gp mark 0}{(4.670,3.634)}
\gppoint{gp mark 0}{(4.670,2.880)}
\gppoint{gp mark 0}{(4.670,3.600)}
\gppoint{gp mark 0}{(4.670,3.490)}
\gppoint{gp mark 0}{(4.670,3.529)}
\gppoint{gp mark 0}{(4.670,3.490)}
\gppoint{gp mark 0}{(4.670,3.317)}
\gppoint{gp mark 0}{(4.670,2.880)}
\gppoint{gp mark 0}{(4.670,3.450)}
\gppoint{gp mark 0}{(4.670,3.529)}
\gppoint{gp mark 0}{(4.670,3.529)}
\gppoint{gp mark 0}{(4.670,3.529)}
\gppoint{gp mark 0}{(4.670,3.529)}
\gppoint{gp mark 0}{(4.670,3.450)}
\gppoint{gp mark 0}{(4.670,3.600)}
\gppoint{gp mark 0}{(4.670,3.450)}
\gppoint{gp mark 0}{(4.670,3.565)}
\gppoint{gp mark 0}{(4.670,3.529)}
\gppoint{gp mark 0}{(4.670,3.408)}
\gppoint{gp mark 0}{(4.670,2.958)}
\gppoint{gp mark 0}{(4.670,3.565)}
\gppoint{gp mark 0}{(4.670,3.529)}
\gppoint{gp mark 0}{(4.670,3.529)}
\gppoint{gp mark 0}{(4.670,3.408)}
\gppoint{gp mark 0}{(4.670,3.529)}
\gppoint{gp mark 0}{(4.670,3.697)}
\gppoint{gp mark 0}{(4.670,3.364)}
\gppoint{gp mark 0}{(4.670,3.096)}
\gppoint{gp mark 0}{(4.670,3.529)}
\gppoint{gp mark 0}{(4.670,3.267)}
\gppoint{gp mark 0}{(4.670,3.267)}
\gppoint{gp mark 0}{(4.670,2.880)}
\gppoint{gp mark 0}{(4.670,3.317)}
\gppoint{gp mark 0}{(4.670,3.214)}
\gppoint{gp mark 0}{(4.670,3.267)}
\gppoint{gp mark 0}{(4.670,3.697)}
\gppoint{gp mark 0}{(4.670,3.267)}
\gppoint{gp mark 0}{(4.670,3.756)}
\gppoint{gp mark 0}{(4.670,3.565)}
\gppoint{gp mark 0}{(4.670,3.267)}
\gppoint{gp mark 0}{(4.670,2.586)}
\gppoint{gp mark 0}{(4.670,3.408)}
\gppoint{gp mark 0}{(4.670,4.237)}
\gppoint{gp mark 0}{(4.670,3.408)}
\gppoint{gp mark 0}{(4.670,3.490)}
\gppoint{gp mark 0}{(4.738,3.490)}
\gppoint{gp mark 0}{(4.738,3.600)}
\gppoint{gp mark 0}{(4.738,3.934)}
\gppoint{gp mark 0}{(4.738,3.529)}
\gppoint{gp mark 0}{(4.738,3.450)}
\gppoint{gp mark 0}{(4.738,3.317)}
\gppoint{gp mark 0}{(4.738,3.450)}
\gppoint{gp mark 0}{(4.738,3.408)}
\gppoint{gp mark 0}{(4.738,3.364)}
\gppoint{gp mark 0}{(4.738,3.364)}
\gppoint{gp mark 0}{(4.738,3.811)}
\gppoint{gp mark 0}{(4.738,3.364)}
\gppoint{gp mark 0}{(4.738,3.157)}
\gppoint{gp mark 0}{(4.738,3.529)}
\gppoint{gp mark 0}{(4.738,3.214)}
\gppoint{gp mark 0}{(4.738,2.958)}
\gppoint{gp mark 0}{(4.738,3.157)}
\gppoint{gp mark 0}{(4.738,3.157)}
\gppoint{gp mark 0}{(4.738,3.408)}
\gppoint{gp mark 0}{(4.738,3.364)}
\gppoint{gp mark 0}{(4.738,3.600)}
\gppoint{gp mark 0}{(4.738,3.887)}
\gppoint{gp mark 0}{(4.738,3.267)}
\gppoint{gp mark 0}{(4.738,3.887)}
\gppoint{gp mark 0}{(4.738,3.887)}
\gppoint{gp mark 0}{(4.738,3.364)}
\gppoint{gp mark 0}{(4.738,3.529)}
\gppoint{gp mark 0}{(4.738,3.529)}
\gppoint{gp mark 0}{(4.738,3.364)}
\gppoint{gp mark 0}{(4.738,3.529)}
\gppoint{gp mark 0}{(4.738,3.450)}
\gppoint{gp mark 0}{(4.738,3.490)}
\gppoint{gp mark 0}{(4.738,3.408)}
\gppoint{gp mark 0}{(4.738,3.408)}
\gppoint{gp mark 0}{(4.738,3.600)}
\gppoint{gp mark 0}{(4.738,3.600)}
\gppoint{gp mark 0}{(4.738,3.364)}
\gppoint{gp mark 0}{(4.738,3.450)}
\gppoint{gp mark 0}{(4.738,3.666)}
\gppoint{gp mark 0}{(4.738,3.408)}
\gppoint{gp mark 0}{(4.738,3.214)}
\gppoint{gp mark 0}{(4.738,3.408)}
\gppoint{gp mark 0}{(4.738,3.157)}
\gppoint{gp mark 0}{(4.738,3.565)}
\gppoint{gp mark 0}{(4.738,3.600)}
\gppoint{gp mark 0}{(4.738,3.364)}
\gppoint{gp mark 0}{(4.738,3.600)}
\gppoint{gp mark 0}{(4.738,3.727)}
\gppoint{gp mark 0}{(4.738,3.565)}
\gppoint{gp mark 0}{(4.738,3.214)}
\gppoint{gp mark 0}{(4.738,2.793)}
\gppoint{gp mark 0}{(4.738,3.811)}
\gppoint{gp mark 0}{(4.738,3.450)}
\gppoint{gp mark 0}{(4.738,3.450)}
\gppoint{gp mark 0}{(4.738,3.096)}
\gppoint{gp mark 0}{(4.738,3.450)}
\gppoint{gp mark 0}{(4.738,3.529)}
\gppoint{gp mark 0}{(4.738,3.600)}
\gppoint{gp mark 0}{(4.738,3.157)}
\gppoint{gp mark 0}{(4.738,3.887)}
\gppoint{gp mark 0}{(4.738,3.600)}
\gppoint{gp mark 0}{(4.738,3.317)}
\gppoint{gp mark 0}{(4.738,3.317)}
\gppoint{gp mark 0}{(4.738,3.697)}
\gppoint{gp mark 0}{(4.738,3.157)}
\gppoint{gp mark 0}{(4.738,3.317)}
\gppoint{gp mark 0}{(4.738,3.096)}
\gppoint{gp mark 0}{(4.738,3.490)}
\gppoint{gp mark 0}{(4.738,3.214)}
\gppoint{gp mark 0}{(4.738,3.364)}
\gppoint{gp mark 0}{(4.738,3.490)}
\gppoint{gp mark 0}{(4.738,3.529)}
\gppoint{gp mark 0}{(4.738,3.887)}
\gppoint{gp mark 0}{(4.738,3.364)}
\gppoint{gp mark 0}{(4.738,3.267)}
\gppoint{gp mark 0}{(4.738,3.408)}
\gppoint{gp mark 0}{(4.738,3.408)}
\gppoint{gp mark 0}{(4.738,3.364)}
\gppoint{gp mark 0}{(4.738,3.529)}
\gppoint{gp mark 0}{(4.738,3.666)}
\gppoint{gp mark 0}{(4.738,4.041)}
\gppoint{gp mark 0}{(4.738,3.157)}
\gppoint{gp mark 0}{(4.738,3.364)}
\gppoint{gp mark 0}{(4.738,3.364)}
\gppoint{gp mark 0}{(4.738,3.267)}
\gppoint{gp mark 0}{(4.738,2.958)}
\gppoint{gp mark 0}{(4.738,3.317)}
\gppoint{gp mark 0}{(4.738,3.600)}
\gppoint{gp mark 0}{(4.738,3.214)}
\gppoint{gp mark 0}{(4.738,3.600)}
\gppoint{gp mark 0}{(4.738,3.364)}
\gppoint{gp mark 0}{(4.738,3.666)}
\gppoint{gp mark 0}{(4.738,3.317)}
\gppoint{gp mark 0}{(4.738,3.214)}
\gppoint{gp mark 0}{(4.738,3.450)}
\gppoint{gp mark 0}{(4.738,3.267)}
\gppoint{gp mark 0}{(4.738,3.214)}
\gppoint{gp mark 0}{(4.738,3.634)}
\gppoint{gp mark 0}{(4.738,3.096)}
\gppoint{gp mark 0}{(4.738,3.157)}
\gppoint{gp mark 0}{(4.738,3.529)}
\gppoint{gp mark 0}{(4.738,3.529)}
\gppoint{gp mark 0}{(4.738,3.529)}
\gppoint{gp mark 0}{(4.738,3.490)}
\gppoint{gp mark 0}{(4.738,4.061)}
\gppoint{gp mark 0}{(4.738,3.030)}
\gppoint{gp mark 0}{(4.738,3.666)}
\gppoint{gp mark 0}{(4.738,4.021)}
\gppoint{gp mark 0}{(4.738,3.490)}
\gppoint{gp mark 0}{(4.738,3.634)}
\gppoint{gp mark 0}{(4.738,3.408)}
\gppoint{gp mark 0}{(4.738,3.317)}
\gppoint{gp mark 0}{(4.738,3.408)}
\gppoint{gp mark 0}{(4.738,3.408)}
\gppoint{gp mark 0}{(4.738,3.267)}
\gppoint{gp mark 0}{(4.738,3.408)}
\gppoint{gp mark 0}{(4.738,3.529)}
\gppoint{gp mark 0}{(4.738,3.529)}
\gppoint{gp mark 0}{(4.738,3.157)}
\gppoint{gp mark 0}{(4.738,3.529)}
\gppoint{gp mark 0}{(4.738,3.096)}
\gppoint{gp mark 0}{(4.738,3.450)}
\gppoint{gp mark 0}{(4.738,3.030)}
\gppoint{gp mark 0}{(4.738,3.529)}
\gppoint{gp mark 0}{(4.738,3.267)}
\gppoint{gp mark 0}{(4.738,3.887)}
\gppoint{gp mark 0}{(4.738,3.529)}
\gppoint{gp mark 0}{(4.738,3.666)}
\gppoint{gp mark 0}{(4.738,3.267)}
\gppoint{gp mark 0}{(4.738,3.957)}
\gppoint{gp mark 0}{(4.738,3.408)}
\gppoint{gp mark 0}{(4.738,3.837)}
\gppoint{gp mark 0}{(4.738,3.267)}
\gppoint{gp mark 0}{(4.738,3.666)}
\gppoint{gp mark 0}{(4.738,3.267)}
\gppoint{gp mark 0}{(4.738,3.317)}
\gppoint{gp mark 0}{(4.738,3.267)}
\gppoint{gp mark 0}{(4.738,3.267)}
\gppoint{gp mark 0}{(4.738,3.727)}
\gppoint{gp mark 0}{(4.738,3.096)}
\gppoint{gp mark 0}{(4.738,3.157)}
\gppoint{gp mark 0}{(4.738,3.317)}
\gppoint{gp mark 0}{(4.738,3.450)}
\gppoint{gp mark 0}{(4.738,3.490)}
\gppoint{gp mark 0}{(4.738,3.887)}
\gppoint{gp mark 0}{(4.738,3.214)}
\gppoint{gp mark 0}{(4.738,3.096)}
\gppoint{gp mark 0}{(4.738,3.600)}
\gppoint{gp mark 0}{(4.738,3.490)}
\gppoint{gp mark 0}{(4.738,3.450)}
\gppoint{gp mark 0}{(4.738,3.408)}
\gppoint{gp mark 0}{(4.738,3.565)}
\gppoint{gp mark 0}{(4.738,3.267)}
\gppoint{gp mark 0}{(4.738,3.529)}
\gppoint{gp mark 0}{(4.738,3.214)}
\gppoint{gp mark 0}{(4.738,3.490)}
\gppoint{gp mark 0}{(4.738,3.887)}
\gppoint{gp mark 0}{(4.738,3.490)}
\gppoint{gp mark 0}{(4.738,3.490)}
\gppoint{gp mark 0}{(4.738,3.887)}
\gppoint{gp mark 0}{(4.738,3.157)}
\gppoint{gp mark 0}{(4.738,3.934)}
\gppoint{gp mark 0}{(4.738,3.529)}
\gppoint{gp mark 0}{(4.738,3.529)}
\gppoint{gp mark 0}{(4.738,3.529)}
\gppoint{gp mark 0}{(4.738,3.450)}
\gppoint{gp mark 0}{(4.738,3.862)}
\gppoint{gp mark 0}{(4.738,3.450)}
\gppoint{gp mark 0}{(4.738,3.096)}
\gppoint{gp mark 0}{(4.738,3.364)}
\gppoint{gp mark 0}{(4.738,3.837)}
\gppoint{gp mark 0}{(4.738,3.666)}
\gppoint{gp mark 0}{(4.738,3.267)}
\gppoint{gp mark 0}{(4.738,3.529)}
\gppoint{gp mark 0}{(4.738,3.957)}
\gppoint{gp mark 0}{(4.738,3.214)}
\gppoint{gp mark 0}{(4.738,3.600)}
\gppoint{gp mark 0}{(4.738,3.214)}
\gppoint{gp mark 0}{(4.738,3.214)}
\gppoint{gp mark 0}{(4.738,3.214)}
\gppoint{gp mark 0}{(4.738,3.214)}
\gppoint{gp mark 0}{(4.738,3.727)}
\gppoint{gp mark 0}{(4.738,3.214)}
\gppoint{gp mark 0}{(4.738,3.979)}
\gppoint{gp mark 0}{(4.738,3.364)}
\gppoint{gp mark 0}{(4.738,3.364)}
\gppoint{gp mark 0}{(4.738,3.364)}
\gppoint{gp mark 0}{(4.738,3.364)}
\gppoint{gp mark 0}{(4.738,3.364)}
\gppoint{gp mark 0}{(4.738,3.364)}
\gppoint{gp mark 0}{(4.738,3.364)}
\gppoint{gp mark 0}{(4.738,2.793)}
\gppoint{gp mark 0}{(4.738,3.157)}
\gppoint{gp mark 0}{(4.738,3.529)}
\gppoint{gp mark 0}{(4.738,3.529)}
\gppoint{gp mark 0}{(4.738,3.490)}
\gppoint{gp mark 0}{(4.738,3.697)}
\gppoint{gp mark 0}{(4.738,3.529)}
\gppoint{gp mark 0}{(4.738,3.862)}
\gppoint{gp mark 0}{(4.738,3.634)}
\gppoint{gp mark 0}{(4.738,3.565)}
\gppoint{gp mark 0}{(4.738,3.565)}
\gppoint{gp mark 0}{(4.738,3.529)}
\gppoint{gp mark 0}{(4.738,3.529)}
\gppoint{gp mark 0}{(4.738,3.267)}
\gppoint{gp mark 0}{(4.738,3.565)}
\gppoint{gp mark 0}{(4.738,3.727)}
\gppoint{gp mark 0}{(4.738,3.408)}
\gppoint{gp mark 0}{(4.738,3.697)}
\gppoint{gp mark 0}{(4.738,3.317)}
\gppoint{gp mark 0}{(4.738,3.450)}
\gppoint{gp mark 0}{(4.738,3.911)}
\gppoint{gp mark 0}{(4.738,3.911)}
\gppoint{gp mark 0}{(4.738,3.214)}
\gppoint{gp mark 0}{(4.738,3.364)}
\gppoint{gp mark 0}{(4.738,3.565)}
\gppoint{gp mark 0}{(4.738,3.887)}
\gppoint{gp mark 0}{(4.738,3.317)}
\gppoint{gp mark 0}{(4.738,3.600)}
\gppoint{gp mark 0}{(4.738,3.887)}
\gppoint{gp mark 0}{(4.738,3.408)}
\gppoint{gp mark 0}{(4.738,3.157)}
\gppoint{gp mark 0}{(4.738,3.030)}
\gppoint{gp mark 0}{(4.738,3.887)}
\gppoint{gp mark 0}{(4.738,3.529)}
\gppoint{gp mark 0}{(4.738,3.529)}
\gppoint{gp mark 0}{(4.738,3.600)}
\gppoint{gp mark 0}{(4.738,3.214)}
\gppoint{gp mark 0}{(4.738,3.364)}
\gppoint{gp mark 0}{(4.738,3.529)}
\gppoint{gp mark 0}{(4.738,3.490)}
\gppoint{gp mark 0}{(4.738,3.565)}
\gppoint{gp mark 0}{(4.738,3.529)}
\gppoint{gp mark 0}{(4.738,3.450)}
\gppoint{gp mark 0}{(4.738,3.756)}
\gppoint{gp mark 0}{(4.738,3.666)}
\gppoint{gp mark 0}{(4.738,3.408)}
\gppoint{gp mark 0}{(4.738,3.267)}
\gppoint{gp mark 0}{(4.738,3.784)}
\gppoint{gp mark 0}{(4.738,3.214)}
\gppoint{gp mark 0}{(4.738,3.565)}
\gppoint{gp mark 0}{(4.738,3.317)}
\gppoint{gp mark 0}{(4.738,3.490)}
\gppoint{gp mark 0}{(4.738,3.565)}
\gppoint{gp mark 0}{(4.738,3.697)}
\gppoint{gp mark 0}{(4.738,3.317)}
\gppoint{gp mark 0}{(4.738,3.317)}
\gppoint{gp mark 0}{(4.738,3.727)}
\gppoint{gp mark 0}{(4.738,3.408)}
\gppoint{gp mark 0}{(4.738,3.666)}
\gppoint{gp mark 0}{(4.738,3.267)}
\gppoint{gp mark 0}{(4.738,3.565)}
\gppoint{gp mark 0}{(4.738,3.364)}
\gppoint{gp mark 0}{(4.738,3.529)}
\gppoint{gp mark 0}{(4.738,3.784)}
\gppoint{gp mark 0}{(4.738,3.565)}
\gppoint{gp mark 0}{(4.738,3.565)}
\gppoint{gp mark 0}{(4.738,3.784)}
\gppoint{gp mark 0}{(4.738,3.450)}
\gppoint{gp mark 0}{(4.738,3.529)}
\gppoint{gp mark 0}{(4.738,3.634)}
\gppoint{gp mark 0}{(4.738,3.565)}
\gppoint{gp mark 0}{(4.738,3.529)}
\gppoint{gp mark 0}{(4.738,3.214)}
\gppoint{gp mark 0}{(4.738,3.697)}
\gppoint{gp mark 0}{(4.738,3.727)}
\gppoint{gp mark 0}{(4.738,3.214)}
\gppoint{gp mark 0}{(4.738,3.317)}
\gppoint{gp mark 0}{(4.738,3.450)}
\gppoint{gp mark 0}{(4.738,3.887)}
\gppoint{gp mark 0}{(4.738,3.529)}
\gppoint{gp mark 0}{(4.738,3.529)}
\gppoint{gp mark 0}{(4.738,3.450)}
\gppoint{gp mark 0}{(4.738,4.061)}
\gppoint{gp mark 0}{(4.738,3.529)}
\gppoint{gp mark 0}{(4.738,3.565)}
\gppoint{gp mark 0}{(4.738,3.450)}
\gppoint{gp mark 0}{(4.738,3.317)}
\gppoint{gp mark 0}{(4.738,3.364)}
\gppoint{gp mark 0}{(4.738,3.364)}
\gppoint{gp mark 0}{(4.738,3.490)}
\gppoint{gp mark 0}{(4.738,3.529)}
\gppoint{gp mark 0}{(4.738,3.364)}
\gppoint{gp mark 0}{(4.738,3.811)}
\gppoint{gp mark 0}{(4.738,3.408)}
\gppoint{gp mark 0}{(4.738,3.784)}
\gppoint{gp mark 0}{(4.738,3.666)}
\gppoint{gp mark 0}{(4.738,3.529)}
\gppoint{gp mark 0}{(4.738,3.030)}
\gppoint{gp mark 0}{(4.738,3.784)}
\gppoint{gp mark 0}{(4.738,3.214)}
\gppoint{gp mark 0}{(4.738,3.600)}
\gppoint{gp mark 0}{(4.738,3.697)}
\gppoint{gp mark 0}{(4.738,3.317)}
\gppoint{gp mark 0}{(4.738,3.157)}
\gppoint{gp mark 0}{(4.738,3.490)}
\gppoint{gp mark 0}{(4.738,3.697)}
\gppoint{gp mark 0}{(4.738,3.450)}
\gppoint{gp mark 0}{(4.738,3.756)}
\gppoint{gp mark 0}{(4.738,3.408)}
\gppoint{gp mark 0}{(4.738,3.490)}
\gppoint{gp mark 0}{(4.738,3.697)}
\gppoint{gp mark 0}{(4.738,3.317)}
\gppoint{gp mark 0}{(4.738,3.408)}
\gppoint{gp mark 0}{(4.738,3.529)}
\gppoint{gp mark 0}{(4.738,3.934)}
\gppoint{gp mark 0}{(4.738,3.157)}
\gppoint{gp mark 0}{(4.738,3.450)}
\gppoint{gp mark 0}{(4.738,3.837)}
\gppoint{gp mark 0}{(4.738,3.450)}
\gppoint{gp mark 0}{(4.738,3.267)}
\gppoint{gp mark 0}{(4.738,3.565)}
\gppoint{gp mark 0}{(4.738,3.317)}
\gppoint{gp mark 0}{(4.738,3.490)}
\gppoint{gp mark 0}{(4.738,3.887)}
\gppoint{gp mark 0}{(4.738,3.450)}
\gppoint{gp mark 0}{(4.738,3.317)}
\gppoint{gp mark 0}{(4.738,3.317)}
\gppoint{gp mark 0}{(4.738,3.408)}
\gppoint{gp mark 0}{(4.738,3.490)}
\gppoint{gp mark 0}{(4.738,3.364)}
\gppoint{gp mark 0}{(4.738,3.529)}
\gppoint{gp mark 0}{(4.738,3.408)}
\gppoint{gp mark 0}{(4.738,3.317)}
\gppoint{gp mark 0}{(4.738,3.408)}
\gppoint{gp mark 0}{(4.738,3.600)}
\gppoint{gp mark 0}{(4.738,3.267)}
\gppoint{gp mark 0}{(4.738,3.727)}
\gppoint{gp mark 0}{(4.738,3.408)}
\gppoint{gp mark 0}{(4.738,3.450)}
\gppoint{gp mark 0}{(4.738,3.490)}
\gppoint{gp mark 0}{(4.738,3.862)}
\gppoint{gp mark 0}{(4.738,3.862)}
\gppoint{gp mark 0}{(4.738,3.096)}
\gppoint{gp mark 0}{(4.738,3.837)}
\gppoint{gp mark 0}{(4.738,3.267)}
\gppoint{gp mark 0}{(4.738,3.450)}
\gppoint{gp mark 0}{(4.738,3.214)}
\gppoint{gp mark 0}{(4.738,3.450)}
\gppoint{gp mark 0}{(4.738,3.634)}
\gppoint{gp mark 0}{(4.738,3.450)}
\gppoint{gp mark 0}{(4.738,3.450)}
\gppoint{gp mark 0}{(4.738,3.979)}
\gppoint{gp mark 0}{(4.738,3.529)}
\gppoint{gp mark 0}{(4.738,3.450)}
\gppoint{gp mark 0}{(4.738,3.157)}
\gppoint{gp mark 0}{(4.738,3.979)}
\gppoint{gp mark 0}{(4.738,3.727)}
\gppoint{gp mark 0}{(4.738,3.934)}
\gppoint{gp mark 0}{(4.738,3.157)}
\gppoint{gp mark 0}{(4.738,3.957)}
\gppoint{gp mark 0}{(4.738,3.157)}
\gppoint{gp mark 0}{(4.738,3.600)}
\gppoint{gp mark 0}{(4.738,3.267)}
\gppoint{gp mark 0}{(4.738,3.408)}
\gppoint{gp mark 0}{(4.738,3.862)}
\gppoint{gp mark 0}{(4.738,3.317)}
\gppoint{gp mark 0}{(4.738,3.600)}
\gppoint{gp mark 0}{(4.738,4.252)}
\gppoint{gp mark 0}{(4.738,4.000)}
\gppoint{gp mark 0}{(4.738,2.880)}
\gppoint{gp mark 0}{(4.738,3.408)}
\gppoint{gp mark 0}{(4.738,3.811)}
\gppoint{gp mark 0}{(4.738,3.887)}
\gppoint{gp mark 0}{(4.738,3.490)}
\gppoint{gp mark 0}{(4.738,3.408)}
\gppoint{gp mark 0}{(4.738,3.666)}
\gppoint{gp mark 0}{(4.738,3.811)}
\gppoint{gp mark 0}{(4.738,3.666)}
\gppoint{gp mark 0}{(4.738,3.634)}
\gppoint{gp mark 0}{(4.738,3.267)}
\gppoint{gp mark 0}{(4.738,3.862)}
\gppoint{gp mark 0}{(4.738,3.565)}
\gppoint{gp mark 0}{(4.738,3.450)}
\gppoint{gp mark 0}{(4.738,3.490)}
\gppoint{gp mark 0}{(4.738,3.529)}
\gppoint{gp mark 0}{(4.738,3.490)}
\gppoint{gp mark 0}{(4.738,3.697)}
\gppoint{gp mark 0}{(4.738,3.157)}
\gppoint{gp mark 0}{(4.738,3.529)}
\gppoint{gp mark 0}{(4.738,3.756)}
\gppoint{gp mark 0}{(4.738,3.364)}
\gppoint{gp mark 0}{(4.802,4.268)}
\gppoint{gp mark 0}{(4.802,3.364)}
\gppoint{gp mark 0}{(4.802,4.021)}
\gppoint{gp mark 0}{(4.802,3.979)}
\gppoint{gp mark 0}{(4.802,3.634)}
\gppoint{gp mark 0}{(4.802,3.317)}
\gppoint{gp mark 0}{(4.802,3.408)}
\gppoint{gp mark 0}{(4.802,3.267)}
\gppoint{gp mark 0}{(4.802,3.727)}
\gppoint{gp mark 0}{(4.802,3.450)}
\gppoint{gp mark 0}{(4.802,3.267)}
\gppoint{gp mark 0}{(4.802,3.600)}
\gppoint{gp mark 0}{(4.802,3.666)}
\gppoint{gp mark 0}{(4.802,3.317)}
\gppoint{gp mark 0}{(4.802,3.862)}
\gppoint{gp mark 0}{(4.802,3.450)}
\gppoint{gp mark 0}{(4.802,3.565)}
\gppoint{gp mark 0}{(4.802,4.445)}
\gppoint{gp mark 0}{(4.802,3.408)}
\gppoint{gp mark 0}{(4.802,3.862)}
\gppoint{gp mark 0}{(4.802,3.979)}
\gppoint{gp mark 0}{(4.802,3.600)}
\gppoint{gp mark 0}{(4.802,4.080)}
\gppoint{gp mark 0}{(4.802,3.565)}
\gppoint{gp mark 0}{(4.802,3.529)}
\gppoint{gp mark 0}{(4.802,3.490)}
\gppoint{gp mark 0}{(4.802,3.450)}
\gppoint{gp mark 0}{(4.802,3.565)}
\gppoint{gp mark 0}{(4.802,4.118)}
\gppoint{gp mark 0}{(4.802,3.450)}
\gppoint{gp mark 0}{(4.802,3.600)}
\gppoint{gp mark 0}{(4.802,3.811)}
\gppoint{gp mark 0}{(4.802,3.529)}
\gppoint{gp mark 0}{(4.802,3.837)}
\gppoint{gp mark 0}{(4.802,3.529)}
\gppoint{gp mark 0}{(4.802,4.237)}
\gppoint{gp mark 0}{(4.802,3.096)}
\gppoint{gp mark 0}{(4.802,3.756)}
\gppoint{gp mark 0}{(4.802,3.317)}
\gppoint{gp mark 0}{(4.802,3.634)}
\gppoint{gp mark 0}{(4.802,3.666)}
\gppoint{gp mark 0}{(4.802,3.666)}
\gppoint{gp mark 0}{(4.802,3.317)}
\gppoint{gp mark 0}{(4.802,3.364)}
\gppoint{gp mark 0}{(4.802,3.565)}
\gppoint{gp mark 0}{(4.802,3.157)}
\gppoint{gp mark 0}{(4.802,3.529)}
\gppoint{gp mark 0}{(4.802,3.666)}
\gppoint{gp mark 0}{(4.802,3.364)}
\gppoint{gp mark 0}{(4.802,3.784)}
\gppoint{gp mark 0}{(4.802,3.565)}
\gppoint{gp mark 0}{(4.802,3.756)}
\gppoint{gp mark 0}{(4.802,3.364)}
\gppoint{gp mark 0}{(4.802,3.450)}
\gppoint{gp mark 0}{(4.802,3.784)}
\gppoint{gp mark 0}{(4.802,3.364)}
\gppoint{gp mark 0}{(4.802,3.565)}
\gppoint{gp mark 0}{(4.802,3.565)}
\gppoint{gp mark 0}{(4.802,3.450)}
\gppoint{gp mark 0}{(4.802,3.267)}
\gppoint{gp mark 0}{(4.802,3.317)}
\gppoint{gp mark 0}{(4.802,3.408)}
\gppoint{gp mark 0}{(4.802,3.408)}
\gppoint{gp mark 0}{(4.802,3.408)}
\gppoint{gp mark 0}{(4.802,3.666)}
\gppoint{gp mark 0}{(4.802,3.317)}
\gppoint{gp mark 0}{(4.802,3.565)}
\gppoint{gp mark 0}{(4.802,3.934)}
\gppoint{gp mark 0}{(4.802,3.450)}
\gppoint{gp mark 0}{(4.802,3.634)}
\gppoint{gp mark 0}{(4.802,2.958)}
\gppoint{gp mark 0}{(4.802,3.600)}
\gppoint{gp mark 0}{(4.802,3.565)}
\gppoint{gp mark 0}{(4.802,3.756)}
\gppoint{gp mark 0}{(4.802,3.565)}
\gppoint{gp mark 0}{(4.802,3.030)}
\gppoint{gp mark 0}{(4.802,3.030)}
\gppoint{gp mark 0}{(4.802,3.697)}
\gppoint{gp mark 0}{(4.802,3.450)}
\gppoint{gp mark 0}{(4.802,3.565)}
\gppoint{gp mark 0}{(4.802,3.450)}
\gppoint{gp mark 0}{(4.802,3.408)}
\gppoint{gp mark 0}{(4.802,3.565)}
\gppoint{gp mark 0}{(4.802,3.214)}
\gppoint{gp mark 0}{(4.802,3.450)}
\gppoint{gp mark 0}{(4.802,3.600)}
\gppoint{gp mark 0}{(4.802,3.565)}
\gppoint{gp mark 0}{(4.802,3.529)}
\gppoint{gp mark 0}{(4.802,3.030)}
\gppoint{gp mark 0}{(4.802,3.450)}
\gppoint{gp mark 0}{(4.802,3.214)}
\gppoint{gp mark 0}{(4.802,3.697)}
\gppoint{gp mark 0}{(4.802,3.727)}
\gppoint{gp mark 0}{(4.802,3.565)}
\gppoint{gp mark 0}{(4.802,3.565)}
\gppoint{gp mark 0}{(4.802,3.565)}
\gppoint{gp mark 0}{(4.802,3.450)}
\gppoint{gp mark 0}{(4.802,3.450)}
\gppoint{gp mark 0}{(4.802,3.565)}
\gppoint{gp mark 0}{(4.802,3.565)}
\gppoint{gp mark 0}{(4.802,3.634)}
\gppoint{gp mark 0}{(4.802,3.565)}
\gppoint{gp mark 0}{(4.802,3.450)}
\gppoint{gp mark 0}{(4.802,3.565)}
\gppoint{gp mark 0}{(4.802,3.600)}
\gppoint{gp mark 0}{(4.802,3.811)}
\gppoint{gp mark 0}{(4.802,3.565)}
\gppoint{gp mark 0}{(4.802,3.784)}
\gppoint{gp mark 0}{(4.802,3.666)}
\gppoint{gp mark 0}{(4.802,3.784)}
\gppoint{gp mark 0}{(4.802,3.600)}
\gppoint{gp mark 0}{(4.802,3.267)}
\gppoint{gp mark 0}{(4.802,3.408)}
\gppoint{gp mark 0}{(4.802,3.529)}
\gppoint{gp mark 0}{(4.802,3.408)}
\gppoint{gp mark 0}{(4.802,3.565)}
\gppoint{gp mark 0}{(4.802,3.490)}
\gppoint{gp mark 0}{(4.802,3.364)}
\gppoint{gp mark 0}{(4.802,3.600)}
\gppoint{gp mark 0}{(4.802,3.490)}
\gppoint{gp mark 0}{(4.802,3.911)}
\gppoint{gp mark 0}{(4.802,3.811)}
\gppoint{gp mark 0}{(4.802,3.364)}
\gppoint{gp mark 0}{(4.802,3.565)}
\gppoint{gp mark 0}{(4.802,3.600)}
\gppoint{gp mark 0}{(4.802,3.565)}
\gppoint{gp mark 0}{(4.802,3.450)}
\gppoint{gp mark 0}{(4.802,3.450)}
\gppoint{gp mark 0}{(4.802,3.565)}
\gppoint{gp mark 0}{(4.802,3.784)}
\gppoint{gp mark 0}{(4.802,3.450)}
\gppoint{gp mark 0}{(4.802,3.030)}
\gppoint{gp mark 0}{(4.802,3.600)}
\gppoint{gp mark 0}{(4.802,3.565)}
\gppoint{gp mark 0}{(4.802,3.565)}
\gppoint{gp mark 0}{(4.802,3.450)}
\gppoint{gp mark 0}{(4.802,3.490)}
\gppoint{gp mark 0}{(4.802,3.317)}
\gppoint{gp mark 0}{(4.802,3.565)}
\gppoint{gp mark 0}{(4.802,3.565)}
\gppoint{gp mark 0}{(4.802,3.565)}
\gppoint{gp mark 0}{(4.802,3.096)}
\gppoint{gp mark 0}{(4.802,3.565)}
\gppoint{gp mark 0}{(4.802,3.450)}
\gppoint{gp mark 0}{(4.802,3.600)}
\gppoint{gp mark 0}{(4.802,3.490)}
\gppoint{gp mark 0}{(4.802,3.030)}
\gppoint{gp mark 0}{(4.802,3.030)}
\gppoint{gp mark 0}{(4.802,3.862)}
\gppoint{gp mark 0}{(4.802,3.565)}
\gppoint{gp mark 0}{(4.802,3.565)}
\gppoint{gp mark 0}{(4.802,3.565)}
\gppoint{gp mark 0}{(4.802,3.565)}
\gppoint{gp mark 0}{(4.802,3.756)}
\gppoint{gp mark 0}{(4.802,3.600)}
\gppoint{gp mark 0}{(4.802,3.408)}
\gppoint{gp mark 0}{(4.802,3.096)}
\gppoint{gp mark 0}{(4.802,3.600)}
\gppoint{gp mark 0}{(4.802,3.030)}
\gppoint{gp mark 0}{(4.802,3.727)}
\gppoint{gp mark 0}{(4.802,3.408)}
\gppoint{gp mark 0}{(4.802,3.408)}
\gppoint{gp mark 0}{(4.802,3.214)}
\gppoint{gp mark 0}{(4.802,3.529)}
\gppoint{gp mark 0}{(4.802,3.634)}
\gppoint{gp mark 0}{(4.802,3.030)}
\gppoint{gp mark 0}{(4.802,3.450)}
\gppoint{gp mark 0}{(4.802,3.364)}
\gppoint{gp mark 0}{(4.802,3.267)}
\gppoint{gp mark 0}{(4.802,3.267)}
\gppoint{gp mark 0}{(4.802,4.099)}
\gppoint{gp mark 0}{(4.802,3.727)}
\gppoint{gp mark 0}{(4.802,3.267)}
\gppoint{gp mark 0}{(4.802,3.267)}
\gppoint{gp mark 0}{(4.802,4.021)}
\gppoint{gp mark 0}{(4.802,3.317)}
\gppoint{gp mark 0}{(4.802,3.529)}
\gppoint{gp mark 0}{(4.802,3.529)}
\gppoint{gp mark 0}{(4.802,3.565)}
\gppoint{gp mark 0}{(4.802,3.600)}
\gppoint{gp mark 0}{(4.802,3.408)}
\gppoint{gp mark 0}{(4.802,3.408)}
\gppoint{gp mark 0}{(4.802,3.267)}
\gppoint{gp mark 0}{(4.802,3.565)}
\gppoint{gp mark 0}{(4.802,3.600)}
\gppoint{gp mark 0}{(4.802,3.666)}
\gppoint{gp mark 0}{(4.802,3.666)}
\gppoint{gp mark 0}{(4.802,3.634)}
\gppoint{gp mark 0}{(4.802,3.408)}
\gppoint{gp mark 0}{(4.802,3.934)}
\gppoint{gp mark 0}{(4.802,3.756)}
\gppoint{gp mark 0}{(4.802,3.490)}
\gppoint{gp mark 0}{(4.802,3.450)}
\gppoint{gp mark 0}{(4.802,2.793)}
\gppoint{gp mark 0}{(4.802,3.490)}
\gppoint{gp mark 0}{(4.802,3.364)}
\gppoint{gp mark 0}{(4.802,3.666)}
\gppoint{gp mark 0}{(4.802,3.030)}
\gppoint{gp mark 0}{(4.802,3.666)}
\gppoint{gp mark 0}{(4.802,3.490)}
\gppoint{gp mark 0}{(4.802,3.450)}
\gppoint{gp mark 0}{(4.802,3.490)}
\gppoint{gp mark 0}{(4.802,3.364)}
\gppoint{gp mark 0}{(4.802,3.756)}
\gppoint{gp mark 0}{(4.802,3.490)}
\gppoint{gp mark 0}{(4.802,4.041)}
\gppoint{gp mark 0}{(4.802,3.490)}
\gppoint{gp mark 0}{(4.802,3.756)}
\gppoint{gp mark 0}{(4.802,3.811)}
\gppoint{gp mark 0}{(4.802,3.364)}
\gppoint{gp mark 0}{(4.802,4.041)}
\gppoint{gp mark 0}{(4.802,3.450)}
\gppoint{gp mark 0}{(4.802,3.408)}
\gppoint{gp mark 0}{(4.802,2.880)}
\gppoint{gp mark 0}{(4.802,3.862)}
\gppoint{gp mark 0}{(4.802,3.811)}
\gppoint{gp mark 0}{(4.802,3.490)}
\gppoint{gp mark 0}{(4.802,3.214)}
\gppoint{gp mark 0}{(4.802,2.793)}
\gppoint{gp mark 0}{(4.802,3.408)}
\gppoint{gp mark 0}{(4.802,3.666)}
\gppoint{gp mark 0}{(4.802,3.666)}
\gppoint{gp mark 0}{(4.802,3.450)}
\gppoint{gp mark 0}{(4.802,4.171)}
\gppoint{gp mark 0}{(4.802,3.784)}
\gppoint{gp mark 0}{(4.802,3.887)}
\gppoint{gp mark 0}{(4.802,4.041)}
\gppoint{gp mark 0}{(4.802,3.317)}
\gppoint{gp mark 0}{(4.802,3.267)}
\gppoint{gp mark 0}{(4.802,3.600)}
\gppoint{gp mark 0}{(4.802,3.030)}
\gppoint{gp mark 0}{(4.802,3.214)}
\gppoint{gp mark 0}{(4.802,3.666)}
\gppoint{gp mark 0}{(4.802,3.157)}
\gppoint{gp mark 0}{(4.802,4.041)}
\gppoint{gp mark 0}{(4.802,3.697)}
\gppoint{gp mark 0}{(4.802,4.171)}
\gppoint{gp mark 0}{(4.802,3.666)}
\gppoint{gp mark 0}{(4.802,3.756)}
\gppoint{gp mark 0}{(4.802,3.756)}
\gppoint{gp mark 0}{(4.802,3.030)}
\gppoint{gp mark 0}{(4.802,3.529)}
\gppoint{gp mark 0}{(4.802,3.157)}
\gppoint{gp mark 0}{(4.802,3.096)}
\gppoint{gp mark 0}{(4.802,3.957)}
\gppoint{gp mark 0}{(4.802,3.030)}
\gppoint{gp mark 0}{(4.802,3.756)}
\gppoint{gp mark 0}{(4.802,3.529)}
\gppoint{gp mark 0}{(4.802,3.529)}
\gppoint{gp mark 0}{(4.802,3.030)}
\gppoint{gp mark 0}{(4.802,3.364)}
\gppoint{gp mark 0}{(4.802,3.317)}
\gppoint{gp mark 0}{(4.802,3.408)}
\gppoint{gp mark 0}{(4.802,3.565)}
\gppoint{gp mark 0}{(4.802,3.096)}
\gppoint{gp mark 0}{(4.802,3.756)}
\gppoint{gp mark 0}{(4.802,3.565)}
\gppoint{gp mark 0}{(4.802,3.565)}
\gppoint{gp mark 0}{(4.802,3.911)}
\gppoint{gp mark 0}{(4.802,3.096)}
\gppoint{gp mark 0}{(4.802,3.565)}
\gppoint{gp mark 0}{(4.802,3.529)}
\gppoint{gp mark 0}{(4.802,3.600)}
\gppoint{gp mark 0}{(4.802,3.267)}
\gppoint{gp mark 0}{(4.802,3.756)}
\gppoint{gp mark 0}{(4.802,3.408)}
\gppoint{gp mark 0}{(4.802,4.080)}
\gppoint{gp mark 0}{(4.802,3.634)}
\gppoint{gp mark 0}{(4.802,3.267)}
\gppoint{gp mark 0}{(4.802,3.600)}
\gppoint{gp mark 0}{(4.802,3.529)}
\gppoint{gp mark 0}{(4.802,3.529)}
\gppoint{gp mark 0}{(4.802,3.214)}
\gppoint{gp mark 0}{(4.802,3.450)}
\gppoint{gp mark 0}{(4.802,3.214)}
\gppoint{gp mark 0}{(4.802,3.911)}
\gppoint{gp mark 0}{(4.802,3.979)}
\gppoint{gp mark 0}{(4.802,3.634)}
\gppoint{gp mark 0}{(4.802,3.317)}
\gppoint{gp mark 0}{(4.802,3.529)}
\gppoint{gp mark 0}{(4.802,3.096)}
\gppoint{gp mark 0}{(4.802,3.784)}
\gppoint{gp mark 0}{(4.802,3.317)}
\gppoint{gp mark 0}{(4.802,3.490)}
\gppoint{gp mark 0}{(4.802,4.041)}
\gppoint{gp mark 0}{(4.802,3.979)}
\gppoint{gp mark 0}{(4.802,3.565)}
\gppoint{gp mark 0}{(4.802,3.408)}
\gppoint{gp mark 0}{(4.802,3.784)}
\gppoint{gp mark 0}{(4.802,3.727)}
\gppoint{gp mark 0}{(4.802,3.911)}
\gppoint{gp mark 0}{(4.802,3.214)}
\gppoint{gp mark 0}{(4.802,3.364)}
\gppoint{gp mark 0}{(4.802,3.364)}
\gppoint{gp mark 0}{(4.802,3.096)}
\gppoint{gp mark 0}{(4.802,3.408)}
\gppoint{gp mark 0}{(4.802,3.697)}
\gppoint{gp mark 0}{(4.802,3.408)}
\gppoint{gp mark 0}{(4.802,3.267)}
\gppoint{gp mark 0}{(4.802,3.490)}
\gppoint{gp mark 0}{(4.802,3.317)}
\gppoint{gp mark 0}{(4.802,3.317)}
\gppoint{gp mark 0}{(4.802,3.450)}
\gppoint{gp mark 0}{(4.802,3.450)}
\gppoint{gp mark 0}{(4.802,3.529)}
\gppoint{gp mark 0}{(4.802,3.756)}
\gppoint{gp mark 0}{(4.802,2.880)}
\gppoint{gp mark 0}{(4.802,3.030)}
\gppoint{gp mark 0}{(4.862,3.490)}
\gppoint{gp mark 0}{(4.862,3.450)}
\gppoint{gp mark 0}{(4.862,3.634)}
\gppoint{gp mark 0}{(4.862,3.634)}
\gppoint{gp mark 0}{(4.862,3.450)}
\gppoint{gp mark 0}{(4.862,3.666)}
\gppoint{gp mark 0}{(4.862,4.188)}
\gppoint{gp mark 0}{(4.862,3.837)}
\gppoint{gp mark 0}{(4.862,3.364)}
\gppoint{gp mark 0}{(4.862,3.214)}
\gppoint{gp mark 0}{(4.862,3.450)}
\gppoint{gp mark 0}{(4.862,4.283)}
\gppoint{gp mark 0}{(4.862,3.490)}
\gppoint{gp mark 0}{(4.862,3.490)}
\gppoint{gp mark 0}{(4.862,3.784)}
\gppoint{gp mark 0}{(4.862,3.756)}
\gppoint{gp mark 0}{(4.862,3.364)}
\gppoint{gp mark 0}{(4.862,3.600)}
\gppoint{gp mark 0}{(4.862,3.600)}
\gppoint{gp mark 0}{(4.862,3.634)}
\gppoint{gp mark 0}{(4.862,3.364)}
\gppoint{gp mark 0}{(4.862,3.364)}
\gppoint{gp mark 0}{(4.862,3.666)}
\gppoint{gp mark 0}{(4.862,3.784)}
\gppoint{gp mark 0}{(4.862,3.364)}
\gppoint{gp mark 0}{(4.862,3.157)}
\gppoint{gp mark 0}{(4.862,3.634)}
\gppoint{gp mark 0}{(4.862,3.364)}
\gppoint{gp mark 0}{(4.862,3.811)}
\gppoint{gp mark 0}{(4.862,4.171)}
\gppoint{gp mark 0}{(4.862,3.214)}
\gppoint{gp mark 0}{(4.862,3.529)}
\gppoint{gp mark 0}{(4.862,3.408)}
\gppoint{gp mark 0}{(4.862,3.756)}
\gppoint{gp mark 0}{(4.862,3.529)}
\gppoint{gp mark 0}{(4.862,3.529)}
\gppoint{gp mark 0}{(4.862,3.364)}
\gppoint{gp mark 0}{(4.862,3.096)}
\gppoint{gp mark 0}{(4.862,3.267)}
\gppoint{gp mark 0}{(4.862,4.000)}
\gppoint{gp mark 0}{(4.862,3.529)}
\gppoint{gp mark 0}{(4.862,3.490)}
\gppoint{gp mark 0}{(4.862,3.666)}
\gppoint{gp mark 0}{(4.862,3.957)}
\gppoint{gp mark 0}{(4.862,3.697)}
\gppoint{gp mark 0}{(4.862,3.450)}
\gppoint{gp mark 0}{(4.862,3.267)}
\gppoint{gp mark 0}{(4.862,3.529)}
\gppoint{gp mark 0}{(4.862,3.030)}
\gppoint{gp mark 0}{(4.862,3.214)}
\gppoint{gp mark 0}{(4.862,3.666)}
\gppoint{gp mark 0}{(4.862,4.341)}
\gppoint{gp mark 0}{(4.862,3.529)}
\gppoint{gp mark 0}{(4.862,3.634)}
\gppoint{gp mark 0}{(4.862,3.666)}
\gppoint{gp mark 0}{(4.862,3.529)}
\gppoint{gp mark 0}{(4.862,3.364)}
\gppoint{gp mark 0}{(4.862,4.061)}
\gppoint{gp mark 0}{(4.862,3.600)}
\gppoint{gp mark 0}{(4.862,3.408)}
\gppoint{gp mark 0}{(4.862,3.565)}
\gppoint{gp mark 0}{(4.862,3.408)}
\gppoint{gp mark 0}{(4.862,3.267)}
\gppoint{gp mark 0}{(4.862,3.490)}
\gppoint{gp mark 0}{(4.862,3.317)}
\gppoint{gp mark 0}{(4.862,3.565)}
\gppoint{gp mark 0}{(4.862,3.727)}
\gppoint{gp mark 0}{(4.862,3.666)}
\gppoint{gp mark 0}{(4.862,3.450)}
\gppoint{gp mark 0}{(4.862,3.408)}
\gppoint{gp mark 0}{(4.862,3.364)}
\gppoint{gp mark 0}{(4.862,3.934)}
\gppoint{gp mark 0}{(4.862,3.408)}
\gppoint{gp mark 0}{(4.862,3.600)}
\gppoint{gp mark 0}{(4.862,4.080)}
\gppoint{gp mark 0}{(4.862,3.529)}
\gppoint{gp mark 0}{(4.862,4.080)}
\gppoint{gp mark 0}{(4.862,3.408)}
\gppoint{gp mark 0}{(4.862,3.784)}
\gppoint{gp mark 0}{(4.862,3.634)}
\gppoint{gp mark 0}{(4.862,3.756)}
\gppoint{gp mark 0}{(4.862,3.600)}
\gppoint{gp mark 0}{(4.862,3.490)}
\gppoint{gp mark 0}{(4.862,3.450)}
\gppoint{gp mark 0}{(4.862,3.811)}
\gppoint{gp mark 0}{(4.862,4.283)}
\gppoint{gp mark 0}{(4.862,3.408)}
\gppoint{gp mark 0}{(4.862,3.811)}
\gppoint{gp mark 0}{(4.862,3.450)}
\gppoint{gp mark 0}{(4.862,3.979)}
\gppoint{gp mark 0}{(4.862,3.214)}
\gppoint{gp mark 0}{(4.862,3.529)}
\gppoint{gp mark 0}{(4.862,3.837)}
\gppoint{gp mark 0}{(4.862,4.252)}
\gppoint{gp mark 0}{(4.862,3.811)}
\gppoint{gp mark 0}{(4.862,3.666)}
\gppoint{gp mark 0}{(4.862,3.529)}
\gppoint{gp mark 0}{(4.862,3.529)}
\gppoint{gp mark 0}{(4.862,3.634)}
\gppoint{gp mark 0}{(4.862,3.600)}
\gppoint{gp mark 0}{(4.862,3.490)}
\gppoint{gp mark 0}{(4.862,3.450)}
\gppoint{gp mark 0}{(4.862,3.529)}
\gppoint{gp mark 0}{(4.862,3.408)}
\gppoint{gp mark 0}{(4.862,3.529)}
\gppoint{gp mark 0}{(4.862,3.490)}
\gppoint{gp mark 0}{(4.862,3.727)}
\gppoint{gp mark 0}{(4.862,3.565)}
\gppoint{gp mark 0}{(4.862,3.529)}
\gppoint{gp mark 0}{(4.862,3.666)}
\gppoint{gp mark 0}{(4.862,3.600)}
\gppoint{gp mark 0}{(4.862,3.529)}
\gppoint{gp mark 0}{(4.862,2.958)}
\gppoint{gp mark 0}{(4.862,3.490)}
\gppoint{gp mark 0}{(4.862,3.600)}
\gppoint{gp mark 0}{(4.862,3.490)}
\gppoint{gp mark 0}{(4.862,3.214)}
\gppoint{gp mark 0}{(4.862,3.565)}
\gppoint{gp mark 0}{(4.862,3.529)}
\gppoint{gp mark 0}{(4.862,3.490)}
\gppoint{gp mark 0}{(4.862,3.600)}
\gppoint{gp mark 0}{(4.862,3.565)}
\gppoint{gp mark 0}{(4.862,3.727)}
\gppoint{gp mark 0}{(4.862,3.529)}
\gppoint{gp mark 0}{(4.862,3.565)}
\gppoint{gp mark 0}{(4.862,3.450)}
\gppoint{gp mark 0}{(4.862,3.862)}
\gppoint{gp mark 0}{(4.862,3.490)}
\gppoint{gp mark 0}{(4.862,3.317)}
\gppoint{gp mark 0}{(4.862,3.634)}
\gppoint{gp mark 0}{(4.862,3.529)}
\gppoint{gp mark 0}{(4.862,3.267)}
\gppoint{gp mark 0}{(4.862,3.364)}
\gppoint{gp mark 0}{(4.862,3.565)}
\gppoint{gp mark 0}{(4.862,3.450)}
\gppoint{gp mark 0}{(4.862,3.727)}
\gppoint{gp mark 0}{(4.862,3.600)}
\gppoint{gp mark 0}{(4.862,3.450)}
\gppoint{gp mark 0}{(4.862,4.061)}
\gppoint{gp mark 0}{(4.862,3.600)}
\gppoint{gp mark 0}{(4.862,3.450)}
\gppoint{gp mark 0}{(4.862,3.634)}
\gppoint{gp mark 0}{(4.862,3.600)}
\gppoint{gp mark 0}{(4.862,3.490)}
\gppoint{gp mark 0}{(4.862,3.600)}
\gppoint{gp mark 0}{(4.862,3.490)}
\gppoint{gp mark 0}{(4.862,3.600)}
\gppoint{gp mark 0}{(4.862,3.267)}
\gppoint{gp mark 0}{(4.862,3.364)}
\gppoint{gp mark 0}{(4.862,3.697)}
\gppoint{gp mark 0}{(4.862,3.784)}
\gppoint{gp mark 0}{(4.862,3.030)}
\gppoint{gp mark 0}{(4.862,3.529)}
\gppoint{gp mark 0}{(4.862,3.600)}
\gppoint{gp mark 0}{(4.862,3.756)}
\gppoint{gp mark 0}{(4.862,4.061)}
\gppoint{gp mark 0}{(4.862,3.600)}
\gppoint{gp mark 0}{(4.862,3.600)}
\gppoint{gp mark 0}{(4.862,4.041)}
\gppoint{gp mark 0}{(4.862,3.364)}
\gppoint{gp mark 0}{(4.862,3.957)}
\gppoint{gp mark 0}{(4.862,3.490)}
\gppoint{gp mark 0}{(4.862,3.214)}
\gppoint{gp mark 0}{(4.862,3.565)}
\gppoint{gp mark 0}{(4.862,4.204)}
\gppoint{gp mark 0}{(4.862,3.529)}
\gppoint{gp mark 0}{(4.862,3.529)}
\gppoint{gp mark 0}{(4.862,4.118)}
\gppoint{gp mark 0}{(4.862,3.317)}
\gppoint{gp mark 0}{(4.862,3.697)}
\gppoint{gp mark 0}{(4.862,3.600)}
\gppoint{gp mark 0}{(4.862,3.529)}
\gppoint{gp mark 0}{(4.862,3.756)}
\gppoint{gp mark 0}{(4.862,3.756)}
\gppoint{gp mark 0}{(4.862,4.021)}
\gppoint{gp mark 0}{(4.862,3.450)}
\gppoint{gp mark 0}{(4.862,3.096)}
\gppoint{gp mark 0}{(4.862,3.600)}
\gppoint{gp mark 0}{(4.862,3.600)}
\gppoint{gp mark 0}{(4.862,3.756)}
\gppoint{gp mark 0}{(4.862,4.204)}
\gppoint{gp mark 0}{(4.862,3.267)}
\gppoint{gp mark 0}{(4.862,3.096)}
\gppoint{gp mark 0}{(4.862,3.600)}
\gppoint{gp mark 0}{(4.862,3.096)}
\gppoint{gp mark 0}{(4.862,3.529)}
\gppoint{gp mark 0}{(4.862,3.565)}
\gppoint{gp mark 0}{(4.862,3.600)}
\gppoint{gp mark 0}{(4.862,3.490)}
\gppoint{gp mark 0}{(4.862,3.887)}
\gppoint{gp mark 0}{(4.862,3.957)}
\gppoint{gp mark 0}{(4.862,3.529)}
\gppoint{gp mark 0}{(4.862,3.727)}
\gppoint{gp mark 0}{(4.862,3.529)}
\gppoint{gp mark 0}{(4.862,3.529)}
\gppoint{gp mark 0}{(4.862,3.600)}
\gppoint{gp mark 0}{(4.862,3.934)}
\gppoint{gp mark 0}{(4.862,4.204)}
\gppoint{gp mark 0}{(4.862,3.529)}
\gppoint{gp mark 0}{(4.862,3.934)}
\gppoint{gp mark 0}{(4.862,3.267)}
\gppoint{gp mark 0}{(4.862,3.887)}
\gppoint{gp mark 0}{(4.862,3.317)}
\gppoint{gp mark 0}{(4.862,3.727)}
\gppoint{gp mark 0}{(4.862,3.634)}
\gppoint{gp mark 0}{(4.862,3.364)}
\gppoint{gp mark 0}{(4.862,3.317)}
\gppoint{gp mark 0}{(4.862,3.600)}
\gppoint{gp mark 0}{(4.862,3.529)}
\gppoint{gp mark 0}{(4.862,3.490)}
\gppoint{gp mark 0}{(4.862,3.529)}
\gppoint{gp mark 0}{(4.862,3.267)}
\gppoint{gp mark 0}{(4.862,3.600)}
\gppoint{gp mark 0}{(4.862,3.862)}
\gppoint{gp mark 0}{(4.862,3.267)}
\gppoint{gp mark 0}{(4.862,3.634)}
\gppoint{gp mark 0}{(4.862,3.408)}
\gppoint{gp mark 0}{(4.862,3.267)}
\gppoint{gp mark 0}{(4.862,3.364)}
\gppoint{gp mark 0}{(4.862,4.204)}
\gppoint{gp mark 0}{(4.862,3.600)}
\gppoint{gp mark 0}{(4.862,3.157)}
\gppoint{gp mark 0}{(4.862,3.529)}
\gppoint{gp mark 0}{(4.862,4.204)}
\gppoint{gp mark 0}{(4.862,3.450)}
\gppoint{gp mark 0}{(4.862,3.529)}
\gppoint{gp mark 0}{(4.862,3.600)}
\gppoint{gp mark 0}{(4.862,3.317)}
\gppoint{gp mark 0}{(4.862,3.697)}
\gppoint{gp mark 0}{(4.862,3.490)}
\gppoint{gp mark 0}{(4.862,3.911)}
\gppoint{gp mark 0}{(4.862,4.204)}
\gppoint{gp mark 0}{(4.862,4.204)}
\gppoint{gp mark 0}{(4.862,4.204)}
\gppoint{gp mark 0}{(4.862,3.529)}
\gppoint{gp mark 0}{(4.862,3.317)}
\gppoint{gp mark 0}{(4.862,3.600)}
\gppoint{gp mark 0}{(4.862,4.204)}
\gppoint{gp mark 0}{(4.862,4.268)}
\gppoint{gp mark 0}{(4.862,3.529)}
\gppoint{gp mark 0}{(4.862,4.061)}
\gppoint{gp mark 0}{(4.862,4.204)}
\gppoint{gp mark 0}{(4.862,3.784)}
\gppoint{gp mark 0}{(4.862,3.364)}
\gppoint{gp mark 0}{(4.862,3.600)}
\gppoint{gp mark 0}{(4.862,3.529)}
\gppoint{gp mark 0}{(4.862,3.634)}
\gppoint{gp mark 0}{(4.862,3.490)}
\gppoint{gp mark 0}{(4.862,3.727)}
\gppoint{gp mark 0}{(4.862,3.727)}
\gppoint{gp mark 0}{(4.862,3.364)}
\gppoint{gp mark 0}{(4.862,3.697)}
\gppoint{gp mark 0}{(4.862,3.317)}
\gppoint{gp mark 0}{(4.862,3.600)}
\gppoint{gp mark 0}{(4.862,3.450)}
\gppoint{gp mark 0}{(4.862,3.600)}
\gppoint{gp mark 0}{(4.862,3.408)}
\gppoint{gp mark 0}{(4.862,3.697)}
\gppoint{gp mark 0}{(4.862,3.600)}
\gppoint{gp mark 0}{(4.862,3.600)}
\gppoint{gp mark 0}{(4.862,3.600)}
\gppoint{gp mark 0}{(4.862,3.634)}
\gppoint{gp mark 0}{(4.862,3.634)}
\gppoint{gp mark 0}{(4.862,3.784)}
\gppoint{gp mark 0}{(4.862,3.529)}
\gppoint{gp mark 0}{(4.862,3.529)}
\gppoint{gp mark 0}{(4.862,3.727)}
\gppoint{gp mark 0}{(4.862,3.634)}
\gppoint{gp mark 0}{(4.862,3.634)}
\gppoint{gp mark 0}{(4.862,3.600)}
\gppoint{gp mark 0}{(4.862,3.364)}
\gppoint{gp mark 0}{(4.862,3.634)}
\gppoint{gp mark 0}{(4.862,3.634)}
\gppoint{gp mark 0}{(4.862,3.529)}
\gppoint{gp mark 0}{(4.862,3.600)}
\gppoint{gp mark 0}{(4.862,3.408)}
\gppoint{gp mark 0}{(4.862,3.600)}
\gppoint{gp mark 0}{(4.862,3.600)}
\gppoint{gp mark 0}{(4.862,4.000)}
\gppoint{gp mark 0}{(4.862,3.600)}
\gppoint{gp mark 0}{(4.862,3.529)}
\gppoint{gp mark 0}{(4.862,3.565)}
\gppoint{gp mark 0}{(4.862,3.030)}
\gppoint{gp mark 0}{(4.862,3.887)}
\gppoint{gp mark 0}{(4.862,3.565)}
\gppoint{gp mark 0}{(4.862,3.600)}
\gppoint{gp mark 0}{(4.862,3.600)}
\gppoint{gp mark 0}{(4.862,3.784)}
\gppoint{gp mark 0}{(4.862,3.529)}
\gppoint{gp mark 0}{(4.862,3.096)}
\gppoint{gp mark 0}{(4.862,3.887)}
\gppoint{gp mark 0}{(4.862,3.666)}
\gppoint{gp mark 0}{(4.862,3.911)}
\gppoint{gp mark 0}{(4.862,3.364)}
\gppoint{gp mark 0}{(4.862,3.490)}
\gppoint{gp mark 0}{(4.862,3.887)}
\gppoint{gp mark 0}{(4.862,3.600)}
\gppoint{gp mark 0}{(4.862,3.911)}
\gppoint{gp mark 0}{(4.862,3.911)}
\gppoint{gp mark 0}{(4.862,3.529)}
\gppoint{gp mark 0}{(4.862,3.600)}
\gppoint{gp mark 0}{(4.862,3.529)}
\gppoint{gp mark 0}{(4.862,3.600)}
\gppoint{gp mark 0}{(4.862,3.600)}
\gppoint{gp mark 0}{(4.862,3.837)}
\gppoint{gp mark 0}{(4.862,3.157)}
\gppoint{gp mark 0}{(4.862,3.565)}
\gppoint{gp mark 0}{(4.862,3.600)}
\gppoint{gp mark 0}{(4.862,3.784)}
\gppoint{gp mark 0}{(4.862,3.911)}
\gppoint{gp mark 0}{(4.862,3.600)}
\gppoint{gp mark 0}{(4.862,3.529)}
\gppoint{gp mark 0}{(4.862,3.862)}
\gppoint{gp mark 0}{(4.862,3.364)}
\gppoint{gp mark 0}{(4.862,3.600)}
\gppoint{gp mark 0}{(4.862,3.450)}
\gppoint{gp mark 0}{(4.862,3.600)}
\gppoint{gp mark 0}{(4.862,3.214)}
\gppoint{gp mark 0}{(4.862,3.727)}
\gppoint{gp mark 0}{(4.920,3.450)}
\gppoint{gp mark 0}{(4.920,3.267)}
\gppoint{gp mark 0}{(4.920,3.697)}
\gppoint{gp mark 0}{(4.920,3.600)}
\gppoint{gp mark 0}{(4.920,3.490)}
\gppoint{gp mark 0}{(4.920,3.565)}
\gppoint{gp mark 0}{(4.920,3.697)}
\gppoint{gp mark 0}{(4.920,3.979)}
\gppoint{gp mark 0}{(4.920,4.221)}
\gppoint{gp mark 0}{(4.920,3.666)}
\gppoint{gp mark 0}{(4.920,3.267)}
\gppoint{gp mark 0}{(4.920,3.364)}
\gppoint{gp mark 0}{(4.920,2.958)}
\gppoint{gp mark 0}{(4.920,3.666)}
\gppoint{gp mark 0}{(4.920,3.600)}
\gppoint{gp mark 0}{(4.920,4.171)}
\gppoint{gp mark 0}{(4.920,3.408)}
\gppoint{gp mark 0}{(4.920,3.911)}
\gppoint{gp mark 0}{(4.920,3.157)}
\gppoint{gp mark 0}{(4.920,3.529)}
\gppoint{gp mark 0}{(4.920,4.516)}
\gppoint{gp mark 0}{(4.920,3.697)}
\gppoint{gp mark 0}{(4.920,3.157)}
\gppoint{gp mark 0}{(4.920,3.490)}
\gppoint{gp mark 0}{(4.920,3.697)}
\gppoint{gp mark 0}{(4.920,3.756)}
\gppoint{gp mark 0}{(4.920,4.061)}
\gppoint{gp mark 0}{(4.920,3.364)}
\gppoint{gp mark 0}{(4.920,3.450)}
\gppoint{gp mark 0}{(4.920,3.600)}
\gppoint{gp mark 0}{(4.920,4.021)}
\gppoint{gp mark 0}{(4.920,3.600)}
\gppoint{gp mark 0}{(4.920,3.317)}
\gppoint{gp mark 0}{(4.920,3.600)}
\gppoint{gp mark 0}{(4.920,3.450)}
\gppoint{gp mark 0}{(4.920,3.490)}
\gppoint{gp mark 0}{(4.920,3.666)}
\gppoint{gp mark 0}{(4.920,3.634)}
\gppoint{gp mark 0}{(4.920,3.490)}
\gppoint{gp mark 0}{(4.920,3.600)}
\gppoint{gp mark 0}{(4.920,3.666)}
\gppoint{gp mark 0}{(4.920,3.364)}
\gppoint{gp mark 0}{(4.920,3.157)}
\gppoint{gp mark 0}{(4.920,3.030)}
\gppoint{gp mark 0}{(4.920,3.862)}
\gppoint{gp mark 0}{(4.920,3.364)}
\gppoint{gp mark 0}{(4.920,3.837)}
\gppoint{gp mark 0}{(4.920,3.837)}
\gppoint{gp mark 0}{(4.920,3.450)}
\gppoint{gp mark 0}{(4.920,3.364)}
\gppoint{gp mark 0}{(4.920,3.364)}
\gppoint{gp mark 0}{(4.920,3.634)}
\gppoint{gp mark 0}{(4.920,3.666)}
\gppoint{gp mark 0}{(4.920,3.490)}
\gppoint{gp mark 0}{(4.920,4.021)}
\gppoint{gp mark 0}{(4.920,3.862)}
\gppoint{gp mark 0}{(4.920,3.450)}
\gppoint{gp mark 0}{(4.920,3.408)}
\gppoint{gp mark 0}{(4.920,3.408)}
\gppoint{gp mark 0}{(4.920,3.450)}
\gppoint{gp mark 0}{(4.920,3.634)}
\gppoint{gp mark 0}{(4.920,3.529)}
\gppoint{gp mark 0}{(4.920,3.214)}
\gppoint{gp mark 0}{(4.920,3.030)}
\gppoint{gp mark 0}{(4.920,4.237)}
\gppoint{gp mark 0}{(4.920,3.490)}
\gppoint{gp mark 0}{(4.920,3.214)}
\gppoint{gp mark 0}{(4.920,3.666)}
\gppoint{gp mark 0}{(4.920,3.634)}
\gppoint{gp mark 0}{(4.920,3.317)}
\gppoint{gp mark 0}{(4.920,3.408)}
\gppoint{gp mark 0}{(4.920,3.784)}
\gppoint{gp mark 0}{(4.920,3.565)}
\gppoint{gp mark 0}{(4.920,3.911)}
\gppoint{gp mark 0}{(4.920,3.267)}
\gppoint{gp mark 0}{(4.920,3.862)}
\gppoint{gp mark 0}{(4.920,3.600)}
\gppoint{gp mark 0}{(4.920,3.697)}
\gppoint{gp mark 0}{(4.920,3.666)}
\gppoint{gp mark 0}{(4.920,3.600)}
\gppoint{gp mark 0}{(4.920,3.408)}
\gppoint{gp mark 0}{(4.920,3.600)}
\gppoint{gp mark 0}{(4.920,3.600)}
\gppoint{gp mark 0}{(4.920,3.666)}
\gppoint{gp mark 0}{(4.920,3.979)}
\gppoint{gp mark 0}{(4.920,3.862)}
\gppoint{gp mark 0}{(4.920,3.529)}
\gppoint{gp mark 0}{(4.920,3.267)}
\gppoint{gp mark 0}{(4.920,3.634)}
\gppoint{gp mark 0}{(4.920,4.041)}
\gppoint{gp mark 0}{(4.920,3.811)}
\gppoint{gp mark 0}{(4.920,3.837)}
\gppoint{gp mark 0}{(4.920,3.490)}
\gppoint{gp mark 0}{(4.920,3.529)}
\gppoint{gp mark 0}{(4.920,3.408)}
\gppoint{gp mark 0}{(4.920,3.529)}
\gppoint{gp mark 0}{(4.920,3.862)}
\gppoint{gp mark 0}{(4.920,3.408)}
\gppoint{gp mark 0}{(4.920,3.666)}
\gppoint{gp mark 0}{(4.920,3.600)}
\gppoint{gp mark 0}{(4.920,3.634)}
\gppoint{gp mark 0}{(4.920,3.811)}
\gppoint{gp mark 0}{(4.920,3.529)}
\gppoint{gp mark 0}{(4.920,3.490)}
\gppoint{gp mark 0}{(4.920,3.727)}
\gppoint{gp mark 0}{(4.920,3.529)}
\gppoint{gp mark 0}{(4.920,3.408)}
\gppoint{gp mark 0}{(4.920,3.490)}
\gppoint{gp mark 0}{(4.920,3.364)}
\gppoint{gp mark 0}{(4.920,3.490)}
\gppoint{gp mark 0}{(4.920,3.697)}
\gppoint{gp mark 0}{(4.920,3.784)}
\gppoint{gp mark 0}{(4.920,3.529)}
\gppoint{gp mark 0}{(4.920,3.490)}
\gppoint{gp mark 0}{(4.920,3.887)}
\gppoint{gp mark 0}{(4.920,3.490)}
\gppoint{gp mark 0}{(4.920,3.450)}
\gppoint{gp mark 0}{(4.920,3.490)}
\gppoint{gp mark 0}{(4.920,3.811)}
\gppoint{gp mark 0}{(4.920,3.600)}
\gppoint{gp mark 0}{(4.920,3.408)}
\gppoint{gp mark 0}{(4.920,3.214)}
\gppoint{gp mark 0}{(4.920,3.317)}
\gppoint{gp mark 0}{(4.920,4.000)}
\gppoint{gp mark 0}{(4.920,3.267)}
\gppoint{gp mark 0}{(4.920,3.862)}
\gppoint{gp mark 0}{(4.920,3.697)}
\gppoint{gp mark 0}{(4.920,3.697)}
\gppoint{gp mark 0}{(4.920,3.157)}
\gppoint{gp mark 0}{(4.920,3.565)}
\gppoint{gp mark 0}{(4.920,3.529)}
\gppoint{gp mark 0}{(4.920,3.490)}
\gppoint{gp mark 0}{(4.920,3.565)}
\gppoint{gp mark 0}{(4.920,3.697)}
\gppoint{gp mark 0}{(4.920,3.450)}
\gppoint{gp mark 0}{(4.920,3.408)}
\gppoint{gp mark 0}{(4.920,4.341)}
\gppoint{gp mark 0}{(4.920,3.600)}
\gppoint{gp mark 0}{(4.920,3.697)}
\gppoint{gp mark 0}{(4.920,3.811)}
\gppoint{gp mark 0}{(4.920,3.756)}
\gppoint{gp mark 0}{(4.920,3.756)}
\gppoint{gp mark 0}{(4.920,3.634)}
\gppoint{gp mark 0}{(4.920,3.697)}
\gppoint{gp mark 0}{(4.920,3.490)}
\gppoint{gp mark 0}{(4.920,3.697)}
\gppoint{gp mark 0}{(4.920,3.634)}
\gppoint{gp mark 0}{(4.920,3.408)}
\gppoint{gp mark 0}{(4.920,3.214)}
\gppoint{gp mark 0}{(4.920,3.450)}
\gppoint{gp mark 0}{(4.920,3.634)}
\gppoint{gp mark 0}{(4.920,3.911)}
\gppoint{gp mark 0}{(4.920,3.565)}
\gppoint{gp mark 0}{(4.920,3.450)}
\gppoint{gp mark 0}{(4.920,3.565)}
\gppoint{gp mark 0}{(4.920,4.080)}
\gppoint{gp mark 0}{(4.920,3.634)}
\gppoint{gp mark 0}{(4.920,4.061)}
\gppoint{gp mark 0}{(4.920,3.450)}
\gppoint{gp mark 0}{(4.920,3.267)}
\gppoint{gp mark 0}{(4.920,3.697)}
\gppoint{gp mark 0}{(4.920,3.565)}
\gppoint{gp mark 0}{(4.920,3.911)}
\gppoint{gp mark 0}{(4.920,4.221)}
\gppoint{gp mark 0}{(4.920,3.979)}
\gppoint{gp mark 0}{(4.920,3.862)}
\gppoint{gp mark 0}{(4.920,3.784)}
\gppoint{gp mark 0}{(4.920,3.096)}
\gppoint{gp mark 0}{(4.920,3.784)}
\gppoint{gp mark 0}{(4.920,3.887)}
\gppoint{gp mark 0}{(4.920,3.529)}
\gppoint{gp mark 0}{(4.920,3.887)}
\gppoint{gp mark 0}{(4.920,3.267)}
\gppoint{gp mark 0}{(4.920,3.600)}
\gppoint{gp mark 0}{(4.920,3.666)}
\gppoint{gp mark 0}{(4.920,3.756)}
\gppoint{gp mark 0}{(4.920,3.784)}
\gppoint{gp mark 0}{(4.920,3.364)}
\gppoint{gp mark 0}{(4.920,3.317)}
\gppoint{gp mark 0}{(4.920,3.756)}
\gppoint{gp mark 0}{(4.920,3.600)}
\gppoint{gp mark 0}{(4.920,3.756)}
\gppoint{gp mark 0}{(4.920,3.529)}
\gppoint{gp mark 0}{(4.920,3.450)}
\gppoint{gp mark 0}{(4.920,3.727)}
\gppoint{gp mark 0}{(4.920,3.600)}
\gppoint{gp mark 0}{(4.920,3.756)}
\gppoint{gp mark 0}{(4.920,3.529)}
\gppoint{gp mark 0}{(4.920,3.529)}
\gppoint{gp mark 0}{(4.920,3.157)}
\gppoint{gp mark 0}{(4.920,3.634)}
\gppoint{gp mark 0}{(4.920,3.490)}
\gppoint{gp mark 0}{(4.920,3.364)}
\gppoint{gp mark 0}{(4.920,3.600)}
\gppoint{gp mark 0}{(4.920,3.934)}
\gppoint{gp mark 0}{(4.920,4.136)}
\gppoint{gp mark 0}{(4.920,3.634)}
\gppoint{gp mark 0}{(4.920,3.697)}
\gppoint{gp mark 0}{(4.920,3.666)}
\gppoint{gp mark 0}{(4.920,3.408)}
\gppoint{gp mark 0}{(4.920,3.096)}
\gppoint{gp mark 0}{(4.920,3.727)}
\gppoint{gp mark 0}{(4.920,3.529)}
\gppoint{gp mark 0}{(4.920,3.600)}
\gppoint{gp mark 0}{(4.920,3.634)}
\gppoint{gp mark 0}{(4.920,3.784)}
\gppoint{gp mark 0}{(4.920,3.887)}
\gppoint{gp mark 0}{(4.920,3.634)}
\gppoint{gp mark 0}{(4.920,3.529)}
\gppoint{gp mark 0}{(4.920,3.529)}
\gppoint{gp mark 0}{(4.920,3.600)}
\gppoint{gp mark 0}{(4.920,3.600)}
\gppoint{gp mark 0}{(4.920,3.600)}
\gppoint{gp mark 0}{(4.920,3.529)}
\gppoint{gp mark 0}{(4.920,3.529)}
\gppoint{gp mark 0}{(4.920,3.600)}
\gppoint{gp mark 0}{(4.920,3.784)}
\gppoint{gp mark 0}{(4.920,3.697)}
\gppoint{gp mark 0}{(4.920,3.727)}
\gppoint{gp mark 0}{(4.920,3.565)}
\gppoint{gp mark 0}{(4.920,3.666)}
\gppoint{gp mark 0}{(4.920,3.408)}
\gppoint{gp mark 0}{(4.920,3.600)}
\gppoint{gp mark 0}{(4.920,3.490)}
\gppoint{gp mark 0}{(4.920,3.490)}
\gppoint{gp mark 0}{(4.920,3.157)}
\gppoint{gp mark 0}{(4.920,3.666)}
\gppoint{gp mark 0}{(4.920,3.267)}
\gppoint{gp mark 0}{(4.920,3.317)}
\gppoint{gp mark 0}{(4.920,3.529)}
\gppoint{gp mark 0}{(4.920,3.600)}
\gppoint{gp mark 0}{(4.920,3.784)}
\gppoint{gp mark 0}{(4.920,3.862)}
\gppoint{gp mark 0}{(4.920,4.136)}
\gppoint{gp mark 0}{(4.920,3.666)}
\gppoint{gp mark 0}{(4.920,3.666)}
\gppoint{gp mark 0}{(4.920,3.634)}
\gppoint{gp mark 0}{(4.920,3.529)}
\gppoint{gp mark 0}{(4.920,3.565)}
\gppoint{gp mark 0}{(4.920,3.634)}
\gppoint{gp mark 0}{(4.920,3.214)}
\gppoint{gp mark 0}{(4.920,3.490)}
\gppoint{gp mark 0}{(4.920,3.727)}
\gppoint{gp mark 0}{(4.920,3.957)}
\gppoint{gp mark 0}{(4.920,3.565)}
\gppoint{gp mark 0}{(4.920,4.080)}
\gppoint{gp mark 0}{(4.920,3.529)}
\gppoint{gp mark 0}{(4.920,3.697)}
\gppoint{gp mark 0}{(4.920,3.565)}
\gppoint{gp mark 0}{(4.920,4.136)}
\gppoint{gp mark 0}{(4.920,4.041)}
\gppoint{gp mark 0}{(4.920,3.450)}
\gppoint{gp mark 0}{(4.920,3.634)}
\gppoint{gp mark 0}{(4.920,3.214)}
\gppoint{gp mark 0}{(4.920,3.697)}
\gppoint{gp mark 0}{(4.975,3.490)}
\gppoint{gp mark 0}{(4.975,3.529)}
\gppoint{gp mark 0}{(4.975,3.727)}
\gppoint{gp mark 0}{(4.975,3.979)}
\gppoint{gp mark 0}{(4.975,3.565)}
\gppoint{gp mark 0}{(4.975,4.000)}
\gppoint{gp mark 0}{(4.975,4.041)}
\gppoint{gp mark 0}{(4.975,3.666)}
\gppoint{gp mark 0}{(4.975,4.312)}
\gppoint{gp mark 0}{(4.975,4.204)}
\gppoint{gp mark 0}{(4.975,3.565)}
\gppoint{gp mark 0}{(4.975,3.529)}
\gppoint{gp mark 0}{(4.975,3.030)}
\gppoint{gp mark 0}{(4.975,3.887)}
\gppoint{gp mark 0}{(4.975,3.490)}
\gppoint{gp mark 0}{(4.975,3.957)}
\gppoint{gp mark 0}{(4.975,3.666)}
\gppoint{gp mark 0}{(4.975,3.408)}
\gppoint{gp mark 0}{(4.975,4.221)}
\gppoint{gp mark 0}{(4.975,3.600)}
\gppoint{gp mark 0}{(4.975,3.450)}
\gppoint{gp mark 0}{(4.975,3.634)}
\gppoint{gp mark 0}{(4.975,3.565)}
\gppoint{gp mark 0}{(4.975,3.214)}
\gppoint{gp mark 0}{(4.975,4.204)}
\gppoint{gp mark 0}{(4.975,3.666)}
\gppoint{gp mark 0}{(4.975,3.634)}
\gppoint{gp mark 0}{(4.975,3.565)}
\gppoint{gp mark 0}{(4.975,3.600)}
\gppoint{gp mark 0}{(4.975,3.887)}
\gppoint{gp mark 0}{(4.975,3.565)}
\gppoint{gp mark 0}{(4.975,3.756)}
\gppoint{gp mark 0}{(4.975,3.600)}
\gppoint{gp mark 0}{(4.975,3.697)}
\gppoint{gp mark 0}{(4.975,3.600)}
\gppoint{gp mark 0}{(4.975,3.634)}
\gppoint{gp mark 0}{(4.975,3.490)}
\gppoint{gp mark 0}{(4.975,3.634)}
\gppoint{gp mark 0}{(4.975,3.408)}
\gppoint{gp mark 0}{(4.975,3.837)}
\gppoint{gp mark 0}{(4.975,3.634)}
\gppoint{gp mark 0}{(4.975,3.727)}
\gppoint{gp mark 0}{(4.975,3.565)}
\gppoint{gp mark 0}{(4.975,4.204)}
\gppoint{gp mark 0}{(4.975,3.317)}
\gppoint{gp mark 0}{(4.975,3.600)}
\gppoint{gp mark 0}{(4.975,3.756)}
\gppoint{gp mark 0}{(4.975,3.697)}
\gppoint{gp mark 0}{(4.975,3.934)}
\gppoint{gp mark 0}{(4.975,3.727)}
\gppoint{gp mark 0}{(4.975,4.021)}
\gppoint{gp mark 0}{(4.975,3.666)}
\gppoint{gp mark 0}{(4.975,3.529)}
\gppoint{gp mark 0}{(4.975,3.634)}
\gppoint{gp mark 0}{(4.975,3.727)}
\gppoint{gp mark 0}{(4.975,3.565)}
\gppoint{gp mark 0}{(4.975,4.204)}
\gppoint{gp mark 0}{(4.975,3.756)}
\gppoint{gp mark 0}{(4.975,3.837)}
\gppoint{gp mark 0}{(4.975,3.565)}
\gppoint{gp mark 0}{(4.975,3.887)}
\gppoint{gp mark 0}{(4.975,3.727)}
\gppoint{gp mark 0}{(4.975,3.634)}
\gppoint{gp mark 0}{(4.975,3.634)}
\gppoint{gp mark 0}{(4.975,3.697)}
\gppoint{gp mark 0}{(4.975,3.529)}
\gppoint{gp mark 0}{(4.975,4.000)}
\gppoint{gp mark 0}{(4.975,3.317)}
\gppoint{gp mark 0}{(4.975,3.666)}
\gppoint{gp mark 0}{(4.975,3.450)}
\gppoint{gp mark 0}{(4.975,3.756)}
\gppoint{gp mark 0}{(4.975,4.000)}
\gppoint{gp mark 0}{(4.975,3.030)}
\gppoint{gp mark 0}{(4.975,3.450)}
\gppoint{gp mark 0}{(4.975,3.317)}
\gppoint{gp mark 0}{(4.975,3.565)}
\gppoint{gp mark 0}{(4.975,3.727)}
\gppoint{gp mark 0}{(4.975,4.000)}
\gppoint{gp mark 0}{(4.975,3.934)}
\gppoint{gp mark 0}{(4.975,3.811)}
\gppoint{gp mark 0}{(4.975,3.450)}
\gppoint{gp mark 0}{(4.975,4.099)}
\gppoint{gp mark 0}{(4.975,3.862)}
\gppoint{gp mark 0}{(4.975,3.267)}
\gppoint{gp mark 0}{(4.975,3.450)}
\gppoint{gp mark 0}{(4.975,3.408)}
\gppoint{gp mark 0}{(4.975,3.756)}
\gppoint{gp mark 0}{(4.975,3.756)}
\gppoint{gp mark 0}{(4.975,3.634)}
\gppoint{gp mark 0}{(4.975,3.756)}
\gppoint{gp mark 0}{(4.975,3.450)}
\gppoint{gp mark 0}{(4.975,3.756)}
\gppoint{gp mark 0}{(4.975,3.784)}
\gppoint{gp mark 0}{(4.975,3.727)}
\gppoint{gp mark 0}{(4.975,3.600)}
\gppoint{gp mark 0}{(4.975,3.529)}
\gppoint{gp mark 0}{(4.975,3.364)}
\gppoint{gp mark 0}{(4.975,3.529)}
\gppoint{gp mark 0}{(4.975,3.600)}
\gppoint{gp mark 0}{(4.975,3.364)}
\gppoint{gp mark 0}{(4.975,3.450)}
\gppoint{gp mark 0}{(4.975,3.837)}
\gppoint{gp mark 0}{(4.975,3.784)}
\gppoint{gp mark 0}{(4.975,3.666)}
\gppoint{gp mark 0}{(4.975,3.529)}
\gppoint{gp mark 0}{(4.975,3.697)}
\gppoint{gp mark 0}{(4.975,3.634)}
\gppoint{gp mark 0}{(4.975,3.267)}
\gppoint{gp mark 0}{(4.975,4.099)}
\gppoint{gp mark 0}{(4.975,3.529)}
\gppoint{gp mark 0}{(4.975,3.634)}
\gppoint{gp mark 0}{(4.975,3.490)}
\gppoint{gp mark 0}{(4.975,4.252)}
\gppoint{gp mark 0}{(4.975,3.634)}
\gppoint{gp mark 0}{(4.975,3.565)}
\gppoint{gp mark 0}{(4.975,3.450)}
\gppoint{gp mark 0}{(4.975,3.030)}
\gppoint{gp mark 0}{(4.975,3.911)}
\gppoint{gp mark 0}{(4.975,3.697)}
\gppoint{gp mark 0}{(4.975,3.756)}
\gppoint{gp mark 0}{(4.975,3.317)}
\gppoint{gp mark 0}{(4.975,4.000)}
\gppoint{gp mark 0}{(4.975,3.697)}
\gppoint{gp mark 0}{(4.975,3.565)}
\gppoint{gp mark 0}{(4.975,3.727)}
\gppoint{gp mark 0}{(4.975,3.887)}
\gppoint{gp mark 0}{(4.975,3.600)}
\gppoint{gp mark 0}{(4.975,4.000)}
\gppoint{gp mark 0}{(4.975,3.450)}
\gppoint{gp mark 0}{(4.975,3.450)}
\gppoint{gp mark 0}{(4.975,3.267)}
\gppoint{gp mark 0}{(4.975,4.283)}
\gppoint{gp mark 0}{(4.975,3.490)}
\gppoint{gp mark 0}{(4.975,3.529)}
\gppoint{gp mark 0}{(4.975,3.408)}
\gppoint{gp mark 0}{(4.975,3.157)}
\gppoint{gp mark 0}{(4.975,3.697)}
\gppoint{gp mark 0}{(4.975,3.784)}
\gppoint{gp mark 0}{(4.975,3.634)}
\gppoint{gp mark 0}{(4.975,3.364)}
\gppoint{gp mark 0}{(4.975,3.408)}
\gppoint{gp mark 0}{(4.975,4.080)}
\gppoint{gp mark 0}{(4.975,3.666)}
\gppoint{gp mark 0}{(4.975,3.267)}
\gppoint{gp mark 0}{(4.975,3.565)}
\gppoint{gp mark 0}{(4.975,3.267)}
\gppoint{gp mark 0}{(4.975,3.529)}
\gppoint{gp mark 0}{(4.975,3.666)}
\gppoint{gp mark 0}{(4.975,4.204)}
\gppoint{gp mark 0}{(4.975,3.727)}
\gppoint{gp mark 0}{(4.975,3.267)}
\gppoint{gp mark 0}{(4.975,3.529)}
\gppoint{gp mark 0}{(4.975,3.756)}
\gppoint{gp mark 0}{(4.975,3.727)}
\gppoint{gp mark 0}{(4.975,3.529)}
\gppoint{gp mark 0}{(4.975,3.529)}
\gppoint{gp mark 0}{(4.975,3.697)}
\gppoint{gp mark 0}{(4.975,3.096)}
\gppoint{gp mark 0}{(4.975,3.756)}
\gppoint{gp mark 0}{(4.975,3.450)}
\gppoint{gp mark 0}{(4.975,3.267)}
\gppoint{gp mark 0}{(4.975,3.364)}
\gppoint{gp mark 0}{(4.975,3.634)}
\gppoint{gp mark 0}{(4.975,3.565)}
\gppoint{gp mark 0}{(4.975,3.727)}
\gppoint{gp mark 0}{(4.975,3.529)}
\gppoint{gp mark 0}{(4.975,3.756)}
\gppoint{gp mark 0}{(4.975,3.784)}
\gppoint{gp mark 0}{(4.975,4.298)}
\gppoint{gp mark 0}{(4.975,3.634)}
\gppoint{gp mark 0}{(4.975,3.529)}
\gppoint{gp mark 0}{(4.975,3.666)}
\gppoint{gp mark 0}{(4.975,3.408)}
\gppoint{gp mark 0}{(4.975,3.756)}
\gppoint{gp mark 0}{(4.975,3.450)}
\gppoint{gp mark 0}{(4.975,3.600)}
\gppoint{gp mark 0}{(4.975,3.756)}
\gppoint{gp mark 0}{(4.975,3.862)}
\gppoint{gp mark 0}{(4.975,3.450)}
\gppoint{gp mark 0}{(4.975,3.862)}
\gppoint{gp mark 0}{(4.975,3.887)}
\gppoint{gp mark 0}{(4.975,3.697)}
\gppoint{gp mark 0}{(4.975,3.756)}
\gppoint{gp mark 0}{(4.975,3.862)}
\gppoint{gp mark 0}{(4.975,4.354)}
\gppoint{gp mark 0}{(4.975,3.529)}
\gppoint{gp mark 0}{(4.975,3.529)}
\gppoint{gp mark 0}{(4.975,3.911)}
\gppoint{gp mark 0}{(4.975,3.727)}
\gppoint{gp mark 0}{(4.975,3.364)}
\gppoint{gp mark 0}{(4.975,3.934)}
\gppoint{gp mark 0}{(4.975,3.887)}
\gppoint{gp mark 0}{(4.975,3.911)}
\gppoint{gp mark 0}{(4.975,3.408)}
\gppoint{gp mark 0}{(4.975,3.727)}
\gppoint{gp mark 0}{(4.975,3.979)}
\gppoint{gp mark 0}{(4.975,3.727)}
\gppoint{gp mark 0}{(4.975,3.529)}
\gppoint{gp mark 0}{(4.975,3.490)}
\gppoint{gp mark 0}{(4.975,3.565)}
\gppoint{gp mark 0}{(4.975,3.697)}
\gppoint{gp mark 0}{(4.975,3.408)}
\gppoint{gp mark 0}{(4.975,3.565)}
\gppoint{gp mark 0}{(4.975,3.934)}
\gppoint{gp mark 0}{(4.975,3.727)}
\gppoint{gp mark 0}{(4.975,3.529)}
\gppoint{gp mark 0}{(4.975,3.634)}
\gppoint{gp mark 0}{(4.975,3.666)}
\gppoint{gp mark 0}{(4.975,3.756)}
\gppoint{gp mark 0}{(4.975,3.634)}
\gppoint{gp mark 0}{(4.975,4.021)}
\gppoint{gp mark 0}{(4.975,3.565)}
\gppoint{gp mark 0}{(4.975,3.727)}
\gppoint{gp mark 0}{(4.975,3.600)}
\gppoint{gp mark 0}{(4.975,4.000)}
\gppoint{gp mark 0}{(4.975,3.600)}
\gppoint{gp mark 0}{(4.975,4.041)}
\gppoint{gp mark 0}{(4.975,3.862)}
\gppoint{gp mark 0}{(4.975,3.600)}
\gppoint{gp mark 0}{(4.975,3.979)}
\gppoint{gp mark 0}{(4.975,3.697)}
\gppoint{gp mark 0}{(4.975,4.469)}
\gppoint{gp mark 0}{(4.975,3.490)}
\gppoint{gp mark 0}{(4.975,3.490)}
\gppoint{gp mark 0}{(4.975,4.000)}
\gppoint{gp mark 0}{(4.975,3.666)}
\gppoint{gp mark 0}{(4.975,3.911)}
\gppoint{gp mark 0}{(4.975,3.634)}
\gppoint{gp mark 0}{(4.975,3.565)}
\gppoint{gp mark 0}{(4.975,3.666)}
\gppoint{gp mark 0}{(4.975,3.666)}
\gppoint{gp mark 0}{(4.975,3.911)}
\gppoint{gp mark 0}{(4.975,3.490)}
\gppoint{gp mark 0}{(4.975,3.600)}
\gppoint{gp mark 0}{(4.975,3.911)}
\gppoint{gp mark 0}{(4.975,3.450)}
\gppoint{gp mark 0}{(4.975,3.600)}
\gppoint{gp mark 0}{(4.975,3.565)}
\gppoint{gp mark 0}{(4.975,3.317)}
\gppoint{gp mark 0}{(4.975,4.000)}
\gppoint{gp mark 0}{(4.975,3.364)}
\gppoint{gp mark 0}{(4.975,3.911)}
\gppoint{gp mark 0}{(4.975,3.565)}
\gppoint{gp mark 0}{(4.975,3.450)}
\gppoint{gp mark 0}{(4.975,3.934)}
\gppoint{gp mark 0}{(4.975,3.267)}
\gppoint{gp mark 0}{(4.975,3.727)}
\gppoint{gp mark 0}{(4.975,3.317)}
\gppoint{gp mark 0}{(4.975,3.666)}
\gppoint{gp mark 0}{(4.975,3.837)}
\gppoint{gp mark 0}{(4.975,3.408)}
\gppoint{gp mark 0}{(4.975,3.634)}
\gppoint{gp mark 0}{(4.975,3.450)}
\gppoint{gp mark 0}{(4.975,3.565)}
\gppoint{gp mark 0}{(4.975,3.666)}
\gppoint{gp mark 0}{(4.975,3.634)}
\gppoint{gp mark 0}{(4.975,3.887)}
\gppoint{gp mark 0}{(4.975,3.756)}
\gppoint{gp mark 0}{(4.975,3.727)}
\gppoint{gp mark 0}{(4.975,3.364)}
\gppoint{gp mark 0}{(4.975,3.727)}
\gppoint{gp mark 0}{(4.975,3.364)}
\gppoint{gp mark 0}{(4.975,3.267)}
\gppoint{gp mark 0}{(4.975,3.565)}
\gppoint{gp mark 0}{(4.975,3.529)}
\gppoint{gp mark 0}{(4.975,3.529)}
\gppoint{gp mark 0}{(4.975,3.364)}
\gppoint{gp mark 0}{(4.975,3.030)}
\gppoint{gp mark 0}{(4.975,3.096)}
\gppoint{gp mark 0}{(4.975,3.862)}
\gppoint{gp mark 0}{(4.975,3.157)}
\gppoint{gp mark 0}{(4.975,3.317)}
\gppoint{gp mark 0}{(4.975,3.756)}
\gppoint{gp mark 0}{(4.975,3.490)}
\gppoint{gp mark 0}{(4.975,3.934)}
\gppoint{gp mark 0}{(4.975,3.408)}
\gppoint{gp mark 0}{(4.975,3.529)}
\gppoint{gp mark 0}{(4.975,3.634)}
\gppoint{gp mark 0}{(4.975,3.408)}
\gppoint{gp mark 0}{(4.975,3.666)}
\gppoint{gp mark 0}{(4.975,3.634)}
\gppoint{gp mark 0}{(4.975,3.529)}
\gppoint{gp mark 0}{(4.975,3.364)}
\gppoint{gp mark 0}{(5.028,3.934)}
\gppoint{gp mark 0}{(5.028,3.957)}
\gppoint{gp mark 0}{(5.028,3.862)}
\gppoint{gp mark 0}{(5.028,3.697)}
\gppoint{gp mark 0}{(5.028,3.887)}
\gppoint{gp mark 0}{(5.028,3.634)}
\gppoint{gp mark 0}{(5.028,3.697)}
\gppoint{gp mark 0}{(5.028,4.581)}
\gppoint{gp mark 0}{(5.028,3.934)}
\gppoint{gp mark 0}{(5.028,3.697)}
\gppoint{gp mark 0}{(5.028,4.099)}
\gppoint{gp mark 0}{(5.028,3.565)}
\gppoint{gp mark 0}{(5.028,3.666)}
\gppoint{gp mark 0}{(5.028,3.450)}
\gppoint{gp mark 0}{(5.028,3.565)}
\gppoint{gp mark 0}{(5.028,3.756)}
\gppoint{gp mark 0}{(5.028,3.727)}
\gppoint{gp mark 0}{(5.028,3.727)}
\gppoint{gp mark 0}{(5.028,3.727)}
\gppoint{gp mark 0}{(5.028,4.136)}
\gppoint{gp mark 0}{(5.028,4.041)}
\gppoint{gp mark 0}{(5.028,4.041)}
\gppoint{gp mark 0}{(5.028,3.529)}
\gppoint{gp mark 0}{(5.028,3.811)}
\gppoint{gp mark 0}{(5.028,4.080)}
\gppoint{gp mark 0}{(5.028,4.080)}
\gppoint{gp mark 0}{(5.028,3.756)}
\gppoint{gp mark 0}{(5.028,3.450)}
\gppoint{gp mark 0}{(5.028,3.887)}
\gppoint{gp mark 0}{(5.028,3.811)}
\gppoint{gp mark 0}{(5.028,3.634)}
\gppoint{gp mark 0}{(5.028,3.490)}
\gppoint{gp mark 0}{(5.028,3.600)}
\gppoint{gp mark 0}{(5.028,3.600)}
\gppoint{gp mark 0}{(5.028,3.529)}
\gppoint{gp mark 0}{(5.028,3.600)}
\gppoint{gp mark 0}{(5.028,3.666)}
\gppoint{gp mark 0}{(5.028,4.368)}
\gppoint{gp mark 0}{(5.028,3.529)}
\gppoint{gp mark 0}{(5.028,3.784)}
\gppoint{gp mark 0}{(5.028,3.887)}
\gppoint{gp mark 0}{(5.028,3.756)}
\gppoint{gp mark 0}{(5.028,3.214)}
\gppoint{gp mark 0}{(5.028,3.784)}
\gppoint{gp mark 0}{(5.028,4.221)}
\gppoint{gp mark 0}{(5.028,3.934)}
\gppoint{gp mark 0}{(5.028,3.934)}
\gppoint{gp mark 0}{(5.028,3.979)}
\gppoint{gp mark 0}{(5.028,3.911)}
\gppoint{gp mark 0}{(5.028,3.697)}
\gppoint{gp mark 0}{(5.028,3.529)}
\gppoint{gp mark 0}{(5.028,3.490)}
\gppoint{gp mark 0}{(5.028,3.529)}
\gppoint{gp mark 0}{(5.028,3.697)}
\gppoint{gp mark 0}{(5.028,3.979)}
\gppoint{gp mark 0}{(5.028,4.493)}
\gppoint{gp mark 0}{(5.028,3.490)}
\gppoint{gp mark 0}{(5.028,4.136)}
\gppoint{gp mark 0}{(5.028,3.317)}
\gppoint{gp mark 0}{(5.028,3.600)}
\gppoint{gp mark 0}{(5.028,3.214)}
\gppoint{gp mark 0}{(5.028,4.021)}
\gppoint{gp mark 0}{(5.028,4.080)}
\gppoint{gp mark 0}{(5.028,3.490)}
\gppoint{gp mark 0}{(5.028,4.000)}
\gppoint{gp mark 0}{(5.028,3.529)}
\gppoint{gp mark 0}{(5.028,3.408)}
\gppoint{gp mark 0}{(5.028,3.634)}
\gppoint{gp mark 0}{(5.028,3.157)}
\gppoint{gp mark 0}{(5.028,3.697)}
\gppoint{gp mark 0}{(5.028,3.727)}
\gppoint{gp mark 0}{(5.028,3.756)}
\gppoint{gp mark 0}{(5.028,3.600)}
\gppoint{gp mark 0}{(5.028,3.697)}
\gppoint{gp mark 0}{(5.028,3.666)}
\gppoint{gp mark 0}{(5.028,3.634)}
\gppoint{gp mark 0}{(5.028,3.837)}
\gppoint{gp mark 0}{(5.028,3.811)}
\gppoint{gp mark 0}{(5.028,3.697)}
\gppoint{gp mark 0}{(5.028,3.979)}
\gppoint{gp mark 0}{(5.028,3.317)}
\gppoint{gp mark 0}{(5.028,3.450)}
\gppoint{gp mark 0}{(5.028,3.408)}
\gppoint{gp mark 0}{(5.028,3.214)}
\gppoint{gp mark 0}{(5.028,3.634)}
\gppoint{gp mark 0}{(5.028,3.267)}
\gppoint{gp mark 0}{(5.028,3.490)}
\gppoint{gp mark 0}{(5.028,3.317)}
\gppoint{gp mark 0}{(5.028,3.911)}
\gppoint{gp mark 0}{(5.028,3.317)}
\gppoint{gp mark 0}{(5.028,3.317)}
\gppoint{gp mark 0}{(5.028,3.529)}
\gppoint{gp mark 0}{(5.028,3.317)}
\gppoint{gp mark 0}{(5.028,3.634)}
\gppoint{gp mark 0}{(5.028,3.784)}
\gppoint{gp mark 0}{(5.028,3.934)}
\gppoint{gp mark 0}{(5.028,3.634)}
\gppoint{gp mark 0}{(5.028,3.600)}
\gppoint{gp mark 0}{(5.028,3.408)}
\gppoint{gp mark 0}{(5.028,3.529)}
\gppoint{gp mark 0}{(5.028,4.237)}
\gppoint{gp mark 0}{(5.028,3.727)}
\gppoint{gp mark 0}{(5.028,3.756)}
\gppoint{gp mark 0}{(5.028,3.565)}
\gppoint{gp mark 0}{(5.028,3.364)}
\gppoint{gp mark 0}{(5.028,3.634)}
\gppoint{gp mark 0}{(5.028,3.634)}
\gppoint{gp mark 0}{(5.028,3.317)}
\gppoint{gp mark 0}{(5.028,3.697)}
\gppoint{gp mark 0}{(5.028,3.784)}
\gppoint{gp mark 0}{(5.028,3.697)}
\gppoint{gp mark 0}{(5.028,3.697)}
\gppoint{gp mark 0}{(5.028,3.600)}
\gppoint{gp mark 0}{(5.028,3.697)}
\gppoint{gp mark 0}{(5.028,3.697)}
\gppoint{gp mark 0}{(5.028,3.317)}
\gppoint{gp mark 0}{(5.028,3.666)}
\gppoint{gp mark 0}{(5.028,3.887)}
\gppoint{gp mark 0}{(5.028,3.634)}
\gppoint{gp mark 0}{(5.028,3.317)}
\gppoint{gp mark 0}{(5.028,3.862)}
\gppoint{gp mark 0}{(5.028,3.666)}
\gppoint{gp mark 0}{(5.028,3.811)}
\gppoint{gp mark 0}{(5.028,3.600)}
\gppoint{gp mark 0}{(5.028,3.565)}
\gppoint{gp mark 0}{(5.028,3.364)}
\gppoint{gp mark 0}{(5.028,3.450)}
\gppoint{gp mark 0}{(5.028,3.634)}
\gppoint{gp mark 0}{(5.028,3.529)}
\gppoint{gp mark 0}{(5.028,3.727)}
\gppoint{gp mark 0}{(5.028,3.364)}
\gppoint{gp mark 0}{(5.028,3.727)}
\gppoint{gp mark 0}{(5.028,3.666)}
\gppoint{gp mark 0}{(5.028,3.887)}
\gppoint{gp mark 0}{(5.028,3.600)}
\gppoint{gp mark 0}{(5.028,3.784)}
\gppoint{gp mark 0}{(5.028,3.862)}
\gppoint{gp mark 0}{(5.028,3.214)}
\gppoint{gp mark 0}{(5.028,3.837)}
\gppoint{gp mark 0}{(5.028,3.862)}
\gppoint{gp mark 0}{(5.028,3.317)}
\gppoint{gp mark 0}{(5.028,3.784)}
\gppoint{gp mark 0}{(5.028,3.634)}
\gppoint{gp mark 0}{(5.028,3.756)}
\gppoint{gp mark 0}{(5.028,3.529)}
\gppoint{gp mark 0}{(5.028,3.727)}
\gppoint{gp mark 0}{(5.028,3.450)}
\gppoint{gp mark 0}{(5.028,4.298)}
\gppoint{gp mark 0}{(5.028,3.490)}
\gppoint{gp mark 0}{(5.028,3.600)}
\gppoint{gp mark 0}{(5.028,4.080)}
\gppoint{gp mark 0}{(5.028,3.756)}
\gppoint{gp mark 0}{(5.028,4.080)}
\gppoint{gp mark 0}{(5.028,3.697)}
\gppoint{gp mark 0}{(5.028,3.600)}
\gppoint{gp mark 0}{(5.028,3.756)}
\gppoint{gp mark 0}{(5.028,4.221)}
\gppoint{gp mark 0}{(5.028,3.811)}
\gppoint{gp mark 0}{(5.028,3.364)}
\gppoint{gp mark 0}{(5.028,3.666)}
\gppoint{gp mark 0}{(5.028,3.634)}
\gppoint{gp mark 0}{(5.028,3.450)}
\gppoint{gp mark 0}{(5.028,3.784)}
\gppoint{gp mark 0}{(5.028,3.408)}
\gppoint{gp mark 0}{(5.028,3.565)}
\gppoint{gp mark 0}{(5.028,3.490)}
\gppoint{gp mark 0}{(5.028,3.756)}
\gppoint{gp mark 0}{(5.028,3.727)}
\gppoint{gp mark 0}{(5.028,3.565)}
\gppoint{gp mark 0}{(5.028,3.784)}
\gppoint{gp mark 0}{(5.028,3.030)}
\gppoint{gp mark 0}{(5.028,3.317)}
\gppoint{gp mark 0}{(5.028,3.697)}
\gppoint{gp mark 0}{(5.028,3.862)}
\gppoint{gp mark 0}{(5.028,3.887)}
\gppoint{gp mark 0}{(5.028,3.565)}
\gppoint{gp mark 0}{(5.028,3.317)}
\gppoint{gp mark 0}{(5.028,3.600)}
\gppoint{gp mark 0}{(5.028,3.529)}
\gppoint{gp mark 0}{(5.028,3.450)}
\gppoint{gp mark 0}{(5.028,3.529)}
\gppoint{gp mark 0}{(5.028,3.634)}
\gppoint{gp mark 0}{(5.028,3.600)}
\gppoint{gp mark 0}{(5.028,3.490)}
\gppoint{gp mark 0}{(5.028,3.490)}
\gppoint{gp mark 0}{(5.028,3.666)}
\gppoint{gp mark 0}{(5.028,4.221)}
\gppoint{gp mark 0}{(5.028,3.727)}
\gppoint{gp mark 0}{(5.028,3.600)}
\gppoint{gp mark 0}{(5.028,3.600)}
\gppoint{gp mark 0}{(5.028,3.862)}
\gppoint{gp mark 0}{(5.028,3.600)}
\gppoint{gp mark 0}{(5.028,3.887)}
\gppoint{gp mark 0}{(5.028,3.317)}
\gppoint{gp mark 0}{(5.028,3.408)}
\gppoint{gp mark 0}{(5.028,3.634)}
\gppoint{gp mark 0}{(5.028,3.565)}
\gppoint{gp mark 0}{(5.028,3.727)}
\gppoint{gp mark 0}{(5.028,3.697)}
\gppoint{gp mark 0}{(5.028,3.450)}
\gppoint{gp mark 0}{(5.028,3.267)}
\gppoint{gp mark 0}{(5.028,3.565)}
\gppoint{gp mark 0}{(5.028,3.317)}
\gppoint{gp mark 0}{(5.028,3.666)}
\gppoint{gp mark 0}{(5.028,3.666)}
\gppoint{gp mark 0}{(5.028,3.490)}
\gppoint{gp mark 0}{(5.028,3.727)}
\gppoint{gp mark 0}{(5.028,3.666)}
\gppoint{gp mark 0}{(5.028,3.862)}
\gppoint{gp mark 0}{(5.028,3.600)}
\gppoint{gp mark 0}{(5.028,3.727)}
\gppoint{gp mark 0}{(5.028,2.958)}
\gppoint{gp mark 0}{(5.028,3.600)}
\gppoint{gp mark 0}{(5.028,3.030)}
\gppoint{gp mark 0}{(5.028,3.697)}
\gppoint{gp mark 0}{(5.028,3.727)}
\gppoint{gp mark 0}{(5.028,3.862)}
\gppoint{gp mark 0}{(5.028,3.666)}
\gppoint{gp mark 0}{(5.028,3.887)}
\gppoint{gp mark 0}{(5.028,3.697)}
\gppoint{gp mark 0}{(5.028,3.934)}
\gppoint{gp mark 0}{(5.028,3.934)}
\gppoint{gp mark 0}{(5.028,3.697)}
\gppoint{gp mark 0}{(5.028,3.862)}
\gppoint{gp mark 0}{(5.028,3.887)}
\gppoint{gp mark 0}{(5.028,3.784)}
\gppoint{gp mark 0}{(5.028,3.756)}
\gppoint{gp mark 0}{(5.028,3.756)}
\gppoint{gp mark 0}{(5.028,3.934)}
\gppoint{gp mark 0}{(5.028,3.490)}
\gppoint{gp mark 0}{(5.028,3.697)}
\gppoint{gp mark 0}{(5.028,3.408)}
\gppoint{gp mark 0}{(5.028,4.080)}
\gppoint{gp mark 0}{(5.028,3.934)}
\gppoint{gp mark 0}{(5.028,3.934)}
\gppoint{gp mark 0}{(5.028,3.634)}
\gppoint{gp mark 0}{(5.028,3.450)}
\gppoint{gp mark 0}{(5.028,3.600)}
\gppoint{gp mark 0}{(5.028,3.862)}
\gppoint{gp mark 0}{(5.028,3.490)}
\gppoint{gp mark 0}{(5.028,3.756)}
\gppoint{gp mark 0}{(5.028,3.364)}
\gppoint{gp mark 0}{(5.028,3.697)}
\gppoint{gp mark 0}{(5.028,3.666)}
\gppoint{gp mark 0}{(5.028,3.862)}
\gppoint{gp mark 0}{(5.028,4.221)}
\gppoint{gp mark 0}{(5.028,3.811)}
\gppoint{gp mark 0}{(5.028,4.021)}
\gppoint{gp mark 0}{(5.028,3.030)}
\gppoint{gp mark 0}{(5.028,3.214)}
\gppoint{gp mark 0}{(5.028,3.837)}
\gppoint{gp mark 0}{(5.028,3.450)}
\gppoint{gp mark 0}{(5.028,3.811)}
\gppoint{gp mark 0}{(5.028,3.727)}
\gppoint{gp mark 0}{(5.028,3.529)}
\gppoint{gp mark 0}{(5.028,3.727)}
\gppoint{gp mark 0}{(5.028,3.529)}
\gppoint{gp mark 0}{(5.028,3.727)}
\gppoint{gp mark 0}{(5.078,3.030)}
\gppoint{gp mark 0}{(5.078,3.634)}
\gppoint{gp mark 0}{(5.078,3.727)}
\gppoint{gp mark 0}{(5.078,3.697)}
\gppoint{gp mark 0}{(5.078,3.727)}
\gppoint{gp mark 0}{(5.078,3.565)}
\gppoint{gp mark 0}{(5.078,4.099)}
\gppoint{gp mark 0}{(5.078,3.811)}
\gppoint{gp mark 0}{(5.078,3.529)}
\gppoint{gp mark 0}{(5.078,3.697)}
\gppoint{gp mark 0}{(5.078,3.450)}
\gppoint{gp mark 0}{(5.078,3.727)}
\gppoint{gp mark 0}{(5.078,3.784)}
\gppoint{gp mark 0}{(5.078,3.862)}
\gppoint{gp mark 0}{(5.078,3.408)}
\gppoint{gp mark 0}{(5.078,3.756)}
\gppoint{gp mark 0}{(5.078,3.634)}
\gppoint{gp mark 0}{(5.078,3.756)}
\gppoint{gp mark 0}{(5.078,3.600)}
\gppoint{gp mark 0}{(5.078,3.666)}
\gppoint{gp mark 0}{(5.078,3.600)}
\gppoint{gp mark 0}{(5.078,3.957)}
\gppoint{gp mark 0}{(5.078,3.634)}
\gppoint{gp mark 0}{(5.078,3.784)}
\gppoint{gp mark 0}{(5.078,3.697)}
\gppoint{gp mark 0}{(5.078,3.837)}
\gppoint{gp mark 0}{(5.078,3.600)}
\gppoint{gp mark 0}{(5.078,3.565)}
\gppoint{gp mark 0}{(5.078,3.697)}
\gppoint{gp mark 0}{(5.078,3.317)}
\gppoint{gp mark 0}{(5.078,3.727)}
\gppoint{gp mark 0}{(5.078,3.600)}
\gppoint{gp mark 0}{(5.078,3.811)}
\gppoint{gp mark 0}{(5.078,3.490)}
\gppoint{gp mark 0}{(5.078,3.697)}
\gppoint{gp mark 0}{(5.078,3.666)}
\gppoint{gp mark 0}{(5.078,3.887)}
\gppoint{gp mark 0}{(5.078,4.080)}
\gppoint{gp mark 0}{(5.078,3.862)}
\gppoint{gp mark 0}{(5.078,4.061)}
\gppoint{gp mark 0}{(5.078,3.887)}
\gppoint{gp mark 0}{(5.078,3.697)}
\gppoint{gp mark 0}{(5.078,3.600)}
\gppoint{gp mark 0}{(5.078,3.529)}
\gppoint{gp mark 0}{(5.078,3.450)}
\gppoint{gp mark 0}{(5.078,3.408)}
\gppoint{gp mark 0}{(5.078,3.267)}
\gppoint{gp mark 0}{(5.078,3.756)}
\gppoint{gp mark 0}{(5.078,3.784)}
\gppoint{gp mark 0}{(5.078,4.080)}
\gppoint{gp mark 0}{(5.078,3.756)}
\gppoint{gp mark 0}{(5.078,4.021)}
\gppoint{gp mark 0}{(5.078,3.697)}
\gppoint{gp mark 0}{(5.078,4.099)}
\gppoint{gp mark 0}{(5.078,3.600)}
\gppoint{gp mark 0}{(5.078,3.490)}
\gppoint{gp mark 0}{(5.078,4.204)}
\gppoint{gp mark 0}{(5.078,3.784)}
\gppoint{gp mark 0}{(5.078,3.565)}
\gppoint{gp mark 0}{(5.078,3.784)}
\gppoint{gp mark 0}{(5.078,3.600)}
\gppoint{gp mark 0}{(5.078,3.756)}
\gppoint{gp mark 0}{(5.078,3.408)}
\gppoint{gp mark 0}{(5.078,3.811)}
\gppoint{gp mark 0}{(5.078,3.634)}
\gppoint{gp mark 0}{(5.078,3.934)}
\gppoint{gp mark 0}{(5.078,3.756)}
\gppoint{gp mark 0}{(5.078,3.811)}
\gppoint{gp mark 0}{(5.078,3.811)}
\gppoint{gp mark 0}{(5.078,3.756)}
\gppoint{gp mark 0}{(5.078,3.600)}
\gppoint{gp mark 0}{(5.078,3.565)}
\gppoint{gp mark 0}{(5.078,3.837)}
\gppoint{gp mark 0}{(5.078,3.634)}
\gppoint{gp mark 0}{(5.078,3.408)}
\gppoint{gp mark 0}{(5.078,3.600)}
\gppoint{gp mark 0}{(5.078,3.862)}
\gppoint{gp mark 0}{(5.078,3.408)}
\gppoint{gp mark 0}{(5.078,3.600)}
\gppoint{gp mark 0}{(5.078,3.756)}
\gppoint{gp mark 0}{(5.078,3.697)}
\gppoint{gp mark 0}{(5.078,3.697)}
\gppoint{gp mark 0}{(5.078,3.697)}
\gppoint{gp mark 0}{(5.078,3.600)}
\gppoint{gp mark 0}{(5.078,3.862)}
\gppoint{gp mark 0}{(5.078,3.837)}
\gppoint{gp mark 0}{(5.078,3.666)}
\gppoint{gp mark 0}{(5.078,4.061)}
\gppoint{gp mark 0}{(5.078,3.634)}
\gppoint{gp mark 0}{(5.078,4.080)}
\gppoint{gp mark 0}{(5.078,4.298)}
\gppoint{gp mark 0}{(5.078,3.837)}
\gppoint{gp mark 0}{(5.078,3.837)}
\gppoint{gp mark 0}{(5.078,3.784)}
\gppoint{gp mark 0}{(5.078,3.600)}
\gppoint{gp mark 0}{(5.078,3.697)}
\gppoint{gp mark 0}{(5.078,4.221)}
\gppoint{gp mark 0}{(5.078,3.756)}
\gppoint{gp mark 0}{(5.078,3.600)}
\gppoint{gp mark 0}{(5.078,4.000)}
\gppoint{gp mark 0}{(5.078,3.529)}
\gppoint{gp mark 0}{(5.078,3.697)}
\gppoint{gp mark 0}{(5.078,3.529)}
\gppoint{gp mark 0}{(5.078,3.565)}
\gppoint{gp mark 0}{(5.078,3.529)}
\gppoint{gp mark 0}{(5.078,4.099)}
\gppoint{gp mark 0}{(5.078,3.666)}
\gppoint{gp mark 0}{(5.078,3.811)}
\gppoint{gp mark 0}{(5.078,4.021)}
\gppoint{gp mark 0}{(5.078,3.600)}
\gppoint{gp mark 0}{(5.078,3.811)}
\gppoint{gp mark 0}{(5.078,3.911)}
\gppoint{gp mark 0}{(5.078,3.837)}
\gppoint{gp mark 0}{(5.078,3.565)}
\gppoint{gp mark 0}{(5.078,3.408)}
\gppoint{gp mark 0}{(5.078,3.600)}
\gppoint{gp mark 0}{(5.078,4.516)}
\gppoint{gp mark 0}{(5.078,3.911)}
\gppoint{gp mark 0}{(5.078,3.911)}
\gppoint{gp mark 0}{(5.078,3.911)}
\gppoint{gp mark 0}{(5.078,3.565)}
\gppoint{gp mark 0}{(5.078,3.600)}
\gppoint{gp mark 0}{(5.078,4.000)}
\gppoint{gp mark 0}{(5.078,3.887)}
\gppoint{gp mark 0}{(5.078,3.784)}
\gppoint{gp mark 0}{(5.078,3.811)}
\gppoint{gp mark 0}{(5.078,3.979)}
\gppoint{gp mark 0}{(5.078,3.979)}
\gppoint{gp mark 0}{(5.078,3.727)}
\gppoint{gp mark 0}{(5.078,3.565)}
\gppoint{gp mark 0}{(5.078,3.634)}
\gppoint{gp mark 0}{(5.078,3.784)}
\gppoint{gp mark 0}{(5.078,3.862)}
\gppoint{gp mark 0}{(5.078,3.666)}
\gppoint{gp mark 0}{(5.078,3.784)}
\gppoint{gp mark 0}{(5.078,4.171)}
\gppoint{gp mark 0}{(5.078,3.364)}
\gppoint{gp mark 0}{(5.078,3.934)}
\gppoint{gp mark 0}{(5.078,3.666)}
\gppoint{gp mark 0}{(5.078,3.811)}
\gppoint{gp mark 0}{(5.078,3.697)}
\gppoint{gp mark 0}{(5.078,3.634)}
\gppoint{gp mark 0}{(5.078,3.634)}
\gppoint{gp mark 0}{(5.078,3.727)}
\gppoint{gp mark 0}{(5.078,3.666)}
\gppoint{gp mark 0}{(5.078,3.666)}
\gppoint{gp mark 0}{(5.078,3.600)}
\gppoint{gp mark 0}{(5.078,3.784)}
\gppoint{gp mark 0}{(5.078,3.727)}
\gppoint{gp mark 0}{(5.078,3.666)}
\gppoint{gp mark 0}{(5.078,3.565)}
\gppoint{gp mark 0}{(5.078,3.529)}
\gppoint{gp mark 0}{(5.078,3.666)}
\gppoint{gp mark 0}{(5.078,3.490)}
\gppoint{gp mark 0}{(5.078,3.490)}
\gppoint{gp mark 0}{(5.078,3.490)}
\gppoint{gp mark 0}{(5.078,3.934)}
\gppoint{gp mark 0}{(5.078,3.490)}
\gppoint{gp mark 0}{(5.078,4.080)}
\gppoint{gp mark 0}{(5.078,3.784)}
\gppoint{gp mark 0}{(5.078,3.811)}
\gppoint{gp mark 0}{(5.078,3.490)}
\gppoint{gp mark 0}{(5.078,3.600)}
\gppoint{gp mark 0}{(5.078,3.727)}
\gppoint{gp mark 0}{(5.078,3.697)}
\gppoint{gp mark 0}{(5.078,3.529)}
\gppoint{gp mark 0}{(5.078,3.697)}
\gppoint{gp mark 0}{(5.078,3.096)}
\gppoint{gp mark 0}{(5.078,3.862)}
\gppoint{gp mark 0}{(5.078,3.529)}
\gppoint{gp mark 0}{(5.078,4.021)}
\gppoint{gp mark 0}{(5.078,3.756)}
\gppoint{gp mark 0}{(5.078,4.136)}
\gppoint{gp mark 0}{(5.078,3.666)}
\gppoint{gp mark 0}{(5.078,3.887)}
\gppoint{gp mark 0}{(5.078,3.697)}
\gppoint{gp mark 0}{(5.078,3.697)}
\gppoint{gp mark 0}{(5.078,4.298)}
\gppoint{gp mark 0}{(5.078,3.364)}
\gppoint{gp mark 0}{(5.078,3.784)}
\gppoint{gp mark 0}{(5.078,3.600)}
\gppoint{gp mark 0}{(5.078,3.600)}
\gppoint{gp mark 0}{(5.078,3.756)}
\gppoint{gp mark 0}{(5.078,3.565)}
\gppoint{gp mark 0}{(5.078,3.157)}
\gppoint{gp mark 0}{(5.078,3.727)}
\gppoint{gp mark 0}{(5.078,4.538)}
\gppoint{gp mark 0}{(5.078,3.565)}
\gppoint{gp mark 0}{(5.078,3.600)}
\gppoint{gp mark 0}{(5.078,3.634)}
\gppoint{gp mark 0}{(5.078,3.600)}
\gppoint{gp mark 0}{(5.078,3.450)}
\gppoint{gp mark 0}{(5.078,3.600)}
\gppoint{gp mark 0}{(5.078,3.697)}
\gppoint{gp mark 0}{(5.078,3.957)}
\gppoint{gp mark 0}{(5.078,3.490)}
\gppoint{gp mark 0}{(5.078,4.268)}
\gppoint{gp mark 0}{(5.078,3.565)}
\gppoint{gp mark 0}{(5.078,3.634)}
\gppoint{gp mark 0}{(5.078,3.666)}
\gppoint{gp mark 0}{(5.078,3.565)}
\gppoint{gp mark 0}{(5.078,3.756)}
\gppoint{gp mark 0}{(5.078,3.600)}
\gppoint{gp mark 0}{(5.078,3.634)}
\gppoint{gp mark 0}{(5.078,4.354)}
\gppoint{gp mark 0}{(5.078,3.666)}
\gppoint{gp mark 0}{(5.078,3.837)}
\gppoint{gp mark 0}{(5.078,3.756)}
\gppoint{gp mark 0}{(5.078,3.634)}
\gppoint{gp mark 0}{(5.078,3.837)}
\gppoint{gp mark 0}{(5.078,3.784)}
\gppoint{gp mark 0}{(5.078,3.317)}
\gppoint{gp mark 0}{(5.078,3.697)}
\gppoint{gp mark 0}{(5.078,3.784)}
\gppoint{gp mark 0}{(5.078,3.634)}
\gppoint{gp mark 0}{(5.078,3.529)}
\gppoint{gp mark 0}{(5.078,3.634)}
\gppoint{gp mark 0}{(5.078,3.565)}
\gppoint{gp mark 0}{(5.078,3.811)}
\gppoint{gp mark 0}{(5.078,3.634)}
\gppoint{gp mark 0}{(5.078,3.529)}
\gppoint{gp mark 0}{(5.078,3.600)}
\gppoint{gp mark 0}{(5.078,3.267)}
\gppoint{gp mark 0}{(5.078,3.756)}
\gppoint{gp mark 0}{(5.078,3.957)}
\gppoint{gp mark 0}{(5.078,3.634)}
\gppoint{gp mark 0}{(5.078,3.565)}
\gppoint{gp mark 0}{(5.078,3.666)}
\gppoint{gp mark 0}{(5.078,3.666)}
\gppoint{gp mark 0}{(5.078,3.267)}
\gppoint{gp mark 0}{(5.078,3.490)}
\gppoint{gp mark 0}{(5.078,3.666)}
\gppoint{gp mark 0}{(5.078,3.408)}
\gppoint{gp mark 0}{(5.078,3.600)}
\gppoint{gp mark 0}{(5.078,3.837)}
\gppoint{gp mark 0}{(5.078,3.634)}
\gppoint{gp mark 0}{(5.078,3.887)}
\gppoint{gp mark 0}{(5.078,3.887)}
\gppoint{gp mark 0}{(5.078,3.756)}
\gppoint{gp mark 0}{(5.078,3.811)}
\gppoint{gp mark 0}{(5.078,3.887)}
\gppoint{gp mark 0}{(5.078,4.080)}
\gppoint{gp mark 0}{(5.078,4.670)}
\gppoint{gp mark 0}{(5.078,3.666)}
\gppoint{gp mark 0}{(5.078,3.565)}
\gppoint{gp mark 0}{(5.078,3.666)}
\gppoint{gp mark 0}{(5.078,3.634)}
\gppoint{gp mark 0}{(5.078,3.811)}
\gppoint{gp mark 0}{(5.078,3.600)}
\gppoint{gp mark 0}{(5.078,3.666)}
\gppoint{gp mark 0}{(5.078,3.565)}
\gppoint{gp mark 0}{(5.078,3.727)}
\gppoint{gp mark 0}{(5.078,3.565)}
\gppoint{gp mark 0}{(5.078,3.784)}
\gppoint{gp mark 0}{(5.078,3.811)}
\gppoint{gp mark 0}{(5.078,3.697)}
\gppoint{gp mark 0}{(5.078,3.096)}
\gppoint{gp mark 0}{(5.078,3.979)}
\gppoint{gp mark 0}{(5.078,4.021)}
\gppoint{gp mark 0}{(5.078,3.364)}
\gppoint{gp mark 0}{(5.078,3.600)}
\gppoint{gp mark 0}{(5.078,3.364)}
\gppoint{gp mark 0}{(5.078,4.153)}
\gppoint{gp mark 0}{(5.078,3.811)}
\gppoint{gp mark 0}{(5.078,4.136)}
\gppoint{gp mark 0}{(5.078,3.697)}
\gppoint{gp mark 0}{(5.126,3.666)}
\gppoint{gp mark 0}{(5.126,4.153)}
\gppoint{gp mark 0}{(5.126,3.634)}
\gppoint{gp mark 0}{(5.126,3.666)}
\gppoint{gp mark 0}{(5.126,3.837)}
\gppoint{gp mark 0}{(5.126,3.490)}
\gppoint{gp mark 0}{(5.126,3.600)}
\gppoint{gp mark 0}{(5.126,3.837)}
\gppoint{gp mark 0}{(5.126,3.317)}
\gppoint{gp mark 0}{(5.126,3.666)}
\gppoint{gp mark 0}{(5.126,3.600)}
\gppoint{gp mark 0}{(5.126,3.756)}
\gppoint{gp mark 0}{(5.126,3.364)}
\gppoint{gp mark 0}{(5.126,3.364)}
\gppoint{gp mark 0}{(5.126,3.697)}
\gppoint{gp mark 0}{(5.126,3.934)}
\gppoint{gp mark 0}{(5.126,3.450)}
\gppoint{gp mark 0}{(5.126,3.317)}
\gppoint{gp mark 0}{(5.126,4.408)}
\gppoint{gp mark 0}{(5.126,3.634)}
\gppoint{gp mark 0}{(5.126,3.756)}
\gppoint{gp mark 0}{(5.126,3.697)}
\gppoint{gp mark 0}{(5.126,3.727)}
\gppoint{gp mark 0}{(5.126,4.061)}
\gppoint{gp mark 0}{(5.126,3.317)}
\gppoint{gp mark 0}{(5.126,3.267)}
\gppoint{gp mark 0}{(5.126,3.666)}
\gppoint{gp mark 0}{(5.126,3.756)}
\gppoint{gp mark 0}{(5.126,3.756)}
\gppoint{gp mark 0}{(5.126,3.811)}
\gppoint{gp mark 0}{(5.126,3.934)}
\gppoint{gp mark 0}{(5.126,3.364)}
\gppoint{gp mark 0}{(5.126,4.061)}
\gppoint{gp mark 0}{(5.126,3.727)}
\gppoint{gp mark 0}{(5.126,3.267)}
\gppoint{gp mark 0}{(5.126,4.204)}
\gppoint{gp mark 0}{(5.126,3.450)}
\gppoint{gp mark 0}{(5.126,3.490)}
\gppoint{gp mark 0}{(5.126,3.911)}
\gppoint{gp mark 0}{(5.126,4.118)}
\gppoint{gp mark 0}{(5.126,3.727)}
\gppoint{gp mark 0}{(5.126,4.641)}
\gppoint{gp mark 0}{(5.126,3.565)}
\gppoint{gp mark 0}{(5.126,3.784)}
\gppoint{gp mark 0}{(5.126,3.364)}
\gppoint{gp mark 0}{(5.126,3.837)}
\gppoint{gp mark 0}{(5.126,3.697)}
\gppoint{gp mark 0}{(5.126,3.666)}
\gppoint{gp mark 0}{(5.126,3.727)}
\gppoint{gp mark 0}{(5.126,3.408)}
\gppoint{gp mark 0}{(5.126,3.727)}
\gppoint{gp mark 0}{(5.126,3.979)}
\gppoint{gp mark 0}{(5.126,3.979)}
\gppoint{gp mark 0}{(5.126,3.490)}
\gppoint{gp mark 0}{(5.126,3.837)}
\gppoint{gp mark 0}{(5.126,3.957)}
\gppoint{gp mark 0}{(5.126,3.666)}
\gppoint{gp mark 0}{(5.126,3.666)}
\gppoint{gp mark 0}{(5.126,3.666)}
\gppoint{gp mark 0}{(5.126,3.756)}
\gppoint{gp mark 0}{(5.126,4.153)}
\gppoint{gp mark 0}{(5.126,3.756)}
\gppoint{gp mark 0}{(5.126,3.979)}
\gppoint{gp mark 0}{(5.126,3.727)}
\gppoint{gp mark 0}{(5.126,3.634)}
\gppoint{gp mark 0}{(5.126,3.450)}
\gppoint{gp mark 0}{(5.126,3.490)}
\gppoint{gp mark 0}{(5.126,3.979)}
\gppoint{gp mark 0}{(5.126,3.862)}
\gppoint{gp mark 0}{(5.126,3.811)}
\gppoint{gp mark 0}{(5.126,3.697)}
\gppoint{gp mark 0}{(5.126,3.634)}
\gppoint{gp mark 0}{(5.126,4.021)}
\gppoint{gp mark 0}{(5.126,3.666)}
\gppoint{gp mark 0}{(5.126,3.666)}
\gppoint{gp mark 0}{(5.126,3.727)}
\gppoint{gp mark 0}{(5.126,4.171)}
\gppoint{gp mark 0}{(5.126,4.408)}
\gppoint{gp mark 0}{(5.126,3.634)}
\gppoint{gp mark 0}{(5.126,3.600)}
\gppoint{gp mark 0}{(5.126,3.837)}
\gppoint{gp mark 0}{(5.126,3.727)}
\gppoint{gp mark 0}{(5.126,3.837)}
\gppoint{gp mark 0}{(5.126,3.666)}
\gppoint{gp mark 0}{(5.126,3.811)}
\gppoint{gp mark 0}{(5.126,4.268)}
\gppoint{gp mark 0}{(5.126,3.811)}
\gppoint{gp mark 0}{(5.126,3.317)}
\gppoint{gp mark 0}{(5.126,3.811)}
\gppoint{gp mark 0}{(5.126,3.634)}
\gppoint{gp mark 0}{(5.126,3.697)}
\gppoint{gp mark 0}{(5.126,3.565)}
\gppoint{gp mark 0}{(5.126,3.811)}
\gppoint{gp mark 0}{(5.126,3.666)}
\gppoint{gp mark 0}{(5.126,3.811)}
\gppoint{gp mark 0}{(5.126,3.756)}
\gppoint{gp mark 0}{(5.126,3.811)}
\gppoint{gp mark 0}{(5.126,3.634)}
\gppoint{gp mark 0}{(5.126,3.600)}
\gppoint{gp mark 0}{(5.126,3.887)}
\gppoint{gp mark 0}{(5.126,4.408)}
\gppoint{gp mark 0}{(5.126,3.157)}
\gppoint{gp mark 0}{(5.126,3.911)}
\gppoint{gp mark 0}{(5.126,3.837)}
\gppoint{gp mark 0}{(5.126,3.157)}
\gppoint{gp mark 0}{(5.126,3.634)}
\gppoint{gp mark 0}{(5.126,3.756)}
\gppoint{gp mark 0}{(5.126,3.408)}
\gppoint{gp mark 0}{(5.126,3.911)}
\gppoint{gp mark 0}{(5.126,4.099)}
\gppoint{gp mark 0}{(5.126,3.784)}
\gppoint{gp mark 0}{(5.126,3.408)}
\gppoint{gp mark 0}{(5.126,3.837)}
\gppoint{gp mark 0}{(5.126,3.157)}
\gppoint{gp mark 0}{(5.126,3.756)}
\gppoint{gp mark 0}{(5.126,3.157)}
\gppoint{gp mark 0}{(5.126,3.317)}
\gppoint{gp mark 0}{(5.126,3.697)}
\gppoint{gp mark 0}{(5.126,4.408)}
\gppoint{gp mark 0}{(5.126,3.697)}
\gppoint{gp mark 0}{(5.126,3.364)}
\gppoint{gp mark 0}{(5.126,3.666)}
\gppoint{gp mark 0}{(5.126,4.041)}
\gppoint{gp mark 0}{(5.126,3.666)}
\gppoint{gp mark 0}{(5.126,3.490)}
\gppoint{gp mark 0}{(5.126,3.862)}
\gppoint{gp mark 0}{(5.126,3.634)}
\gppoint{gp mark 0}{(5.126,3.887)}
\gppoint{gp mark 0}{(5.126,4.298)}
\gppoint{gp mark 0}{(5.126,3.096)}
\gppoint{gp mark 0}{(5.126,3.784)}
\gppoint{gp mark 0}{(5.126,3.267)}
\gppoint{gp mark 0}{(5.126,3.666)}
\gppoint{gp mark 0}{(5.126,3.837)}
\gppoint{gp mark 0}{(5.126,3.756)}
\gppoint{gp mark 0}{(5.126,3.408)}
\gppoint{gp mark 0}{(5.126,3.529)}
\gppoint{gp mark 0}{(5.126,3.666)}
\gppoint{gp mark 0}{(5.126,3.529)}
\gppoint{gp mark 0}{(5.126,3.529)}
\gppoint{gp mark 0}{(5.126,3.697)}
\gppoint{gp mark 0}{(5.126,3.697)}
\gppoint{gp mark 0}{(5.126,3.697)}
\gppoint{gp mark 0}{(5.126,3.697)}
\gppoint{gp mark 0}{(5.126,3.887)}
\gppoint{gp mark 0}{(5.126,3.811)}
\gppoint{gp mark 0}{(5.126,3.529)}
\gppoint{gp mark 0}{(5.126,3.490)}
\gppoint{gp mark 0}{(5.126,3.666)}
\gppoint{gp mark 0}{(5.126,3.756)}
\gppoint{gp mark 0}{(5.126,4.283)}
\gppoint{gp mark 0}{(5.126,3.697)}
\gppoint{gp mark 0}{(5.126,4.061)}
\gppoint{gp mark 0}{(5.126,3.267)}
\gppoint{gp mark 0}{(5.126,3.811)}
\gppoint{gp mark 0}{(5.126,3.911)}
\gppoint{gp mark 0}{(5.126,3.979)}
\gppoint{gp mark 0}{(5.126,3.979)}
\gppoint{gp mark 0}{(5.126,3.957)}
\gppoint{gp mark 0}{(5.126,3.364)}
\gppoint{gp mark 0}{(5.126,3.756)}
\gppoint{gp mark 0}{(5.126,3.934)}
\gppoint{gp mark 0}{(5.126,3.727)}
\gppoint{gp mark 0}{(5.126,4.000)}
\gppoint{gp mark 0}{(5.126,3.634)}
\gppoint{gp mark 0}{(5.126,3.837)}
\gppoint{gp mark 0}{(5.126,3.837)}
\gppoint{gp mark 0}{(5.126,3.934)}
\gppoint{gp mark 0}{(5.126,3.529)}
\gppoint{gp mark 0}{(5.126,3.529)}
\gppoint{gp mark 0}{(5.126,3.934)}
\gppoint{gp mark 0}{(5.126,3.756)}
\gppoint{gp mark 0}{(5.126,3.364)}
\gppoint{gp mark 0}{(5.126,4.000)}
\gppoint{gp mark 0}{(5.126,4.237)}
\gppoint{gp mark 0}{(5.126,3.634)}
\gppoint{gp mark 0}{(5.126,3.756)}
\gppoint{gp mark 0}{(5.126,3.450)}
\gppoint{gp mark 0}{(5.126,3.408)}
\gppoint{gp mark 0}{(5.126,3.634)}
\gppoint{gp mark 0}{(5.126,3.565)}
\gppoint{gp mark 0}{(5.126,3.529)}
\gppoint{gp mark 0}{(5.126,3.756)}
\gppoint{gp mark 0}{(5.126,3.811)}
\gppoint{gp mark 0}{(5.126,3.784)}
\gppoint{gp mark 0}{(5.126,4.641)}
\gppoint{gp mark 0}{(5.126,3.490)}
\gppoint{gp mark 0}{(5.126,3.697)}
\gppoint{gp mark 0}{(5.126,3.727)}
\gppoint{gp mark 0}{(5.126,3.666)}
\gppoint{gp mark 0}{(5.126,3.756)}
\gppoint{gp mark 0}{(5.126,3.529)}
\gppoint{gp mark 0}{(5.126,3.756)}
\gppoint{gp mark 0}{(5.126,3.450)}
\gppoint{gp mark 0}{(5.126,3.811)}
\gppoint{gp mark 0}{(5.126,3.911)}
\gppoint{gp mark 0}{(5.126,4.445)}
\gppoint{gp mark 0}{(5.126,3.756)}
\gppoint{gp mark 0}{(5.126,3.811)}
\gppoint{gp mark 0}{(5.126,3.811)}
\gppoint{gp mark 0}{(5.126,3.811)}
\gppoint{gp mark 0}{(5.126,3.756)}
\gppoint{gp mark 0}{(5.126,3.911)}
\gppoint{gp mark 0}{(5.126,3.784)}
\gppoint{gp mark 0}{(5.126,3.887)}
\gppoint{gp mark 0}{(5.126,3.811)}
\gppoint{gp mark 0}{(5.126,3.811)}
\gppoint{gp mark 0}{(5.126,3.490)}
\gppoint{gp mark 0}{(5.126,3.862)}
\gppoint{gp mark 0}{(5.126,3.811)}
\gppoint{gp mark 0}{(5.126,3.697)}
\gppoint{gp mark 0}{(5.126,4.153)}
\gppoint{gp mark 0}{(5.126,4.041)}
\gppoint{gp mark 0}{(5.126,3.096)}
\gppoint{gp mark 0}{(5.126,4.080)}
\gppoint{gp mark 0}{(5.126,3.666)}
\gppoint{gp mark 0}{(5.126,3.727)}
\gppoint{gp mark 0}{(5.126,4.283)}
\gppoint{gp mark 0}{(5.126,3.666)}
\gppoint{gp mark 0}{(5.126,3.727)}
\gppoint{gp mark 0}{(5.126,3.634)}
\gppoint{gp mark 0}{(5.126,3.697)}
\gppoint{gp mark 0}{(5.126,3.634)}
\gppoint{gp mark 0}{(5.126,3.666)}
\gppoint{gp mark 0}{(5.126,3.565)}
\gppoint{gp mark 0}{(5.126,3.697)}
\gppoint{gp mark 0}{(5.126,3.490)}
\gppoint{gp mark 0}{(5.126,4.000)}
\gppoint{gp mark 0}{(5.126,3.837)}
\gppoint{gp mark 0}{(5.126,3.600)}
\gppoint{gp mark 0}{(5.126,3.756)}
\gppoint{gp mark 0}{(5.126,4.591)}
\gppoint{gp mark 0}{(5.126,3.600)}
\gppoint{gp mark 0}{(5.126,4.000)}
\gppoint{gp mark 0}{(5.126,3.317)}
\gppoint{gp mark 0}{(5.126,3.267)}
\gppoint{gp mark 0}{(5.126,3.934)}
\gppoint{gp mark 0}{(5.126,3.784)}
\gppoint{gp mark 0}{(5.126,4.118)}
\gppoint{gp mark 0}{(5.126,3.666)}
\gppoint{gp mark 0}{(5.126,3.666)}
\gppoint{gp mark 0}{(5.126,4.000)}
\gppoint{gp mark 0}{(5.126,3.979)}
\gppoint{gp mark 0}{(5.126,4.408)}
\gppoint{gp mark 0}{(5.126,3.837)}
\gppoint{gp mark 0}{(5.126,3.565)}
\gppoint{gp mark 0}{(5.126,3.666)}
\gppoint{gp mark 0}{(5.126,3.697)}
\gppoint{gp mark 0}{(5.126,3.811)}
\gppoint{gp mark 0}{(5.126,3.157)}
\gppoint{gp mark 0}{(5.126,3.811)}
\gppoint{gp mark 0}{(5.126,3.727)}
\gppoint{gp mark 0}{(5.126,3.784)}
\gppoint{gp mark 0}{(5.126,3.317)}
\gppoint{gp mark 0}{(5.126,3.565)}
\gppoint{gp mark 0}{(5.126,3.408)}
\gppoint{gp mark 0}{(5.126,4.080)}
\gppoint{gp mark 0}{(5.126,3.697)}
\gppoint{gp mark 0}{(5.173,3.887)}
\gppoint{gp mark 0}{(5.173,3.756)}
\gppoint{gp mark 0}{(5.173,4.188)}
\gppoint{gp mark 0}{(5.173,3.490)}
\gppoint{gp mark 0}{(5.173,3.957)}
\gppoint{gp mark 0}{(5.173,3.727)}
\gppoint{gp mark 0}{(5.173,3.862)}
\gppoint{gp mark 0}{(5.173,4.021)}
\gppoint{gp mark 0}{(5.173,4.041)}
\gppoint{gp mark 0}{(5.173,3.565)}
\gppoint{gp mark 0}{(5.173,4.061)}
\gppoint{gp mark 0}{(5.173,4.041)}
\gppoint{gp mark 0}{(5.173,4.188)}
\gppoint{gp mark 0}{(5.173,3.862)}
\gppoint{gp mark 0}{(5.173,3.727)}
\gppoint{gp mark 0}{(5.173,4.136)}
\gppoint{gp mark 0}{(5.173,3.934)}
\gppoint{gp mark 0}{(5.173,3.727)}
\gppoint{gp mark 0}{(5.173,3.634)}
\gppoint{gp mark 0}{(5.173,4.188)}
\gppoint{gp mark 0}{(5.173,4.188)}
\gppoint{gp mark 0}{(5.173,4.221)}
\gppoint{gp mark 0}{(5.173,3.666)}
\gppoint{gp mark 0}{(5.173,3.634)}
\gppoint{gp mark 0}{(5.173,3.565)}
\gppoint{gp mark 0}{(5.173,3.837)}
\gppoint{gp mark 0}{(5.173,3.529)}
\gppoint{gp mark 0}{(5.173,3.697)}
\gppoint{gp mark 0}{(5.173,3.697)}
\gppoint{gp mark 0}{(5.173,3.634)}
\gppoint{gp mark 0}{(5.173,3.887)}
\gppoint{gp mark 0}{(5.173,3.756)}
\gppoint{gp mark 0}{(5.173,3.634)}
\gppoint{gp mark 0}{(5.173,4.368)}
\gppoint{gp mark 0}{(5.173,3.529)}
\gppoint{gp mark 0}{(5.173,4.099)}
\gppoint{gp mark 0}{(5.173,3.666)}
\gppoint{gp mark 0}{(5.173,3.784)}
\gppoint{gp mark 0}{(5.173,3.727)}
\gppoint{gp mark 0}{(5.173,3.634)}
\gppoint{gp mark 0}{(5.173,3.666)}
\gppoint{gp mark 0}{(5.173,3.887)}
\gppoint{gp mark 0}{(5.173,3.727)}
\gppoint{gp mark 0}{(5.173,3.784)}
\gppoint{gp mark 0}{(5.173,4.041)}
\gppoint{gp mark 0}{(5.173,3.317)}
\gppoint{gp mark 0}{(5.173,4.080)}
\gppoint{gp mark 0}{(5.173,3.666)}
\gppoint{gp mark 0}{(5.173,3.214)}
\gppoint{gp mark 0}{(5.173,3.911)}
\gppoint{gp mark 0}{(5.173,3.811)}
\gppoint{gp mark 0}{(5.173,3.887)}
\gppoint{gp mark 0}{(5.173,4.080)}
\gppoint{gp mark 0}{(5.173,4.188)}
\gppoint{gp mark 0}{(5.173,3.837)}
\gppoint{gp mark 0}{(5.173,3.727)}
\gppoint{gp mark 0}{(5.173,3.634)}
\gppoint{gp mark 0}{(5.173,3.934)}
\gppoint{gp mark 0}{(5.173,3.887)}
\gppoint{gp mark 0}{(5.173,3.756)}
\gppoint{gp mark 0}{(5.173,3.811)}
\gppoint{gp mark 0}{(5.173,3.934)}
\gppoint{gp mark 0}{(5.173,3.837)}
\gppoint{gp mark 0}{(5.173,3.862)}
\gppoint{gp mark 0}{(5.173,3.837)}
\gppoint{gp mark 0}{(5.173,4.188)}
\gppoint{gp mark 0}{(5.173,3.634)}
\gppoint{gp mark 0}{(5.173,3.934)}
\gppoint{gp mark 0}{(5.173,3.887)}
\gppoint{gp mark 0}{(5.173,3.666)}
\gppoint{gp mark 0}{(5.173,4.601)}
\gppoint{gp mark 0}{(5.173,3.862)}
\gppoint{gp mark 0}{(5.173,3.600)}
\gppoint{gp mark 0}{(5.173,3.979)}
\gppoint{gp mark 0}{(5.173,3.267)}
\gppoint{gp mark 0}{(5.173,4.188)}
\gppoint{gp mark 0}{(5.173,4.341)}
\gppoint{gp mark 0}{(5.173,4.341)}
\gppoint{gp mark 0}{(5.173,4.188)}
\gppoint{gp mark 0}{(5.173,3.811)}
\gppoint{gp mark 0}{(5.173,3.934)}
\gppoint{gp mark 0}{(5.173,3.600)}
\gppoint{gp mark 0}{(5.173,3.727)}
\gppoint{gp mark 0}{(5.173,4.041)}
\gppoint{gp mark 0}{(5.173,3.756)}
\gppoint{gp mark 0}{(5.173,3.784)}
\gppoint{gp mark 0}{(5.173,4.188)}
\gppoint{gp mark 0}{(5.173,4.080)}
\gppoint{gp mark 0}{(5.173,4.080)}
\gppoint{gp mark 0}{(5.173,3.862)}
\gppoint{gp mark 0}{(5.173,3.784)}
\gppoint{gp mark 0}{(5.173,3.727)}
\gppoint{gp mark 0}{(5.173,3.784)}
\gppoint{gp mark 0}{(5.173,4.171)}
\gppoint{gp mark 0}{(5.173,3.784)}
\gppoint{gp mark 0}{(5.173,3.697)}
\gppoint{gp mark 0}{(5.173,3.697)}
\gppoint{gp mark 0}{(5.173,3.756)}
\gppoint{gp mark 0}{(5.173,3.784)}
\gppoint{gp mark 0}{(5.173,3.934)}
\gppoint{gp mark 0}{(5.173,3.634)}
\gppoint{gp mark 0}{(5.173,3.784)}
\gppoint{gp mark 0}{(5.173,4.188)}
\gppoint{gp mark 0}{(5.173,3.862)}
\gppoint{gp mark 0}{(5.173,3.727)}
\gppoint{gp mark 0}{(5.173,3.666)}
\gppoint{gp mark 0}{(5.173,3.634)}
\gppoint{gp mark 0}{(5.173,3.634)}
\gppoint{gp mark 0}{(5.173,3.784)}
\gppoint{gp mark 0}{(5.173,3.957)}
\gppoint{gp mark 0}{(5.173,3.837)}
\gppoint{gp mark 0}{(5.173,3.666)}
\gppoint{gp mark 0}{(5.173,3.957)}
\gppoint{gp mark 0}{(5.173,3.666)}
\gppoint{gp mark 0}{(5.173,3.784)}
\gppoint{gp mark 0}{(5.173,3.979)}
\gppoint{gp mark 0}{(5.173,3.666)}
\gppoint{gp mark 0}{(5.173,3.727)}
\gppoint{gp mark 0}{(5.173,3.697)}
\gppoint{gp mark 0}{(5.173,3.697)}
\gppoint{gp mark 0}{(5.173,3.811)}
\gppoint{gp mark 0}{(5.173,4.188)}
\gppoint{gp mark 0}{(5.173,3.697)}
\gppoint{gp mark 0}{(5.173,3.697)}
\gppoint{gp mark 0}{(5.173,3.837)}
\gppoint{gp mark 0}{(5.173,3.957)}
\gppoint{gp mark 0}{(5.173,4.080)}
\gppoint{gp mark 0}{(5.173,4.041)}
\gppoint{gp mark 0}{(5.173,3.565)}
\gppoint{gp mark 0}{(5.173,3.529)}
\gppoint{gp mark 0}{(5.173,3.697)}
\gppoint{gp mark 0}{(5.173,3.934)}
\gppoint{gp mark 0}{(5.173,4.237)}
\gppoint{gp mark 0}{(5.173,4.061)}
\gppoint{gp mark 0}{(5.173,4.237)}
\gppoint{gp mark 0}{(5.173,3.784)}
\gppoint{gp mark 0}{(5.173,4.354)}
\gppoint{gp mark 0}{(5.173,3.490)}
\gppoint{gp mark 0}{(5.173,3.911)}
\gppoint{gp mark 0}{(5.173,3.784)}
\gppoint{gp mark 0}{(5.173,3.887)}
\gppoint{gp mark 0}{(5.173,3.666)}
\gppoint{gp mark 0}{(5.173,3.756)}
\gppoint{gp mark 0}{(5.173,3.837)}
\gppoint{gp mark 0}{(5.173,3.697)}
\gppoint{gp mark 0}{(5.173,4.188)}
\gppoint{gp mark 0}{(5.173,3.837)}
\gppoint{gp mark 0}{(5.173,3.666)}
\gppoint{gp mark 0}{(5.173,3.697)}
\gppoint{gp mark 0}{(5.173,3.450)}
\gppoint{gp mark 0}{(5.173,4.000)}
\gppoint{gp mark 0}{(5.173,3.490)}
\gppoint{gp mark 0}{(5.173,3.666)}
\gppoint{gp mark 0}{(5.173,3.979)}
\gppoint{gp mark 0}{(5.173,3.565)}
\gppoint{gp mark 0}{(5.173,3.784)}
\gppoint{gp mark 0}{(5.173,3.565)}
\gppoint{gp mark 0}{(5.173,3.784)}
\gppoint{gp mark 0}{(5.173,3.666)}
\gppoint{gp mark 0}{(5.173,4.041)}
\gppoint{gp mark 0}{(5.173,4.041)}
\gppoint{gp mark 0}{(5.173,4.041)}
\gppoint{gp mark 0}{(5.173,3.957)}
\gppoint{gp mark 0}{(5.173,3.756)}
\gppoint{gp mark 0}{(5.173,4.000)}
\gppoint{gp mark 0}{(5.173,3.979)}
\gppoint{gp mark 0}{(5.173,3.911)}
\gppoint{gp mark 0}{(5.173,4.099)}
\gppoint{gp mark 0}{(5.173,4.621)}
\gppoint{gp mark 0}{(5.173,3.837)}
\gppoint{gp mark 0}{(5.173,3.837)}
\gppoint{gp mark 0}{(5.173,4.000)}
\gppoint{gp mark 0}{(5.173,3.565)}
\gppoint{gp mark 0}{(5.173,3.756)}
\gppoint{gp mark 0}{(5.173,3.634)}
\gppoint{gp mark 0}{(5.173,4.153)}
\gppoint{gp mark 0}{(5.173,3.666)}
\gppoint{gp mark 0}{(5.173,3.957)}
\gppoint{gp mark 0}{(5.173,3.666)}
\gppoint{gp mark 0}{(5.173,4.080)}
\gppoint{gp mark 0}{(5.173,3.408)}
\gppoint{gp mark 0}{(5.173,4.080)}
\gppoint{gp mark 0}{(5.173,3.529)}
\gppoint{gp mark 0}{(5.173,3.957)}
\gppoint{gp mark 0}{(5.173,3.565)}
\gppoint{gp mark 0}{(5.173,3.529)}
\gppoint{gp mark 0}{(5.173,4.080)}
\gppoint{gp mark 0}{(5.173,3.600)}
\gppoint{gp mark 0}{(5.173,3.600)}
\gppoint{gp mark 0}{(5.173,3.756)}
\gppoint{gp mark 0}{(5.173,3.697)}
\gppoint{gp mark 0}{(5.173,3.600)}
\gppoint{gp mark 0}{(5.173,3.666)}
\gppoint{gp mark 0}{(5.173,3.666)}
\gppoint{gp mark 0}{(5.173,3.450)}
\gppoint{gp mark 0}{(5.173,3.634)}
\gppoint{gp mark 0}{(5.173,3.529)}
\gppoint{gp mark 0}{(5.173,3.784)}
\gppoint{gp mark 0}{(5.173,3.727)}
\gppoint{gp mark 0}{(5.173,3.957)}
\gppoint{gp mark 0}{(5.173,3.697)}
\gppoint{gp mark 0}{(5.173,3.727)}
\gppoint{gp mark 0}{(5.173,3.756)}
\gppoint{gp mark 0}{(5.173,3.600)}
\gppoint{gp mark 0}{(5.173,3.565)}
\gppoint{gp mark 0}{(5.173,3.697)}
\gppoint{gp mark 0}{(5.173,3.600)}
\gppoint{gp mark 0}{(5.173,3.666)}
\gppoint{gp mark 0}{(5.173,3.408)}
\gppoint{gp mark 0}{(5.173,3.979)}
\gppoint{gp mark 0}{(5.173,3.887)}
\gppoint{gp mark 0}{(5.173,3.565)}
\gppoint{gp mark 0}{(5.173,3.784)}
\gppoint{gp mark 0}{(5.173,4.283)}
\gppoint{gp mark 0}{(5.173,4.283)}
\gppoint{gp mark 0}{(5.173,3.934)}
\gppoint{gp mark 0}{(5.173,3.837)}
\gppoint{gp mark 0}{(5.173,3.408)}
\gppoint{gp mark 0}{(5.173,3.697)}
\gppoint{gp mark 0}{(5.173,3.837)}
\gppoint{gp mark 0}{(5.173,3.697)}
\gppoint{gp mark 0}{(5.173,4.283)}
\gppoint{gp mark 0}{(5.173,3.811)}
\gppoint{gp mark 0}{(5.173,3.529)}
\gppoint{gp mark 0}{(5.173,3.784)}
\gppoint{gp mark 0}{(5.173,3.490)}
\gppoint{gp mark 0}{(5.173,3.784)}
\gppoint{gp mark 0}{(5.173,3.887)}
\gppoint{gp mark 0}{(5.173,3.697)}
\gppoint{gp mark 0}{(5.173,4.204)}
\gppoint{gp mark 0}{(5.173,3.957)}
\gppoint{gp mark 0}{(5.173,4.099)}
\gppoint{gp mark 0}{(5.173,3.837)}
\gppoint{gp mark 0}{(5.173,4.080)}
\gppoint{gp mark 0}{(5.217,3.364)}
\gppoint{gp mark 0}{(5.217,3.364)}
\gppoint{gp mark 0}{(5.217,3.697)}
\gppoint{gp mark 0}{(5.217,3.364)}
\gppoint{gp mark 0}{(5.217,3.364)}
\gppoint{gp mark 0}{(5.217,3.727)}
\gppoint{gp mark 0}{(5.217,3.862)}
\gppoint{gp mark 0}{(5.217,3.727)}
\gppoint{gp mark 0}{(5.217,3.450)}
\gppoint{gp mark 0}{(5.217,3.364)}
\gppoint{gp mark 0}{(5.217,3.364)}
\gppoint{gp mark 0}{(5.217,3.364)}
\gppoint{gp mark 0}{(5.217,3.529)}
\gppoint{gp mark 0}{(5.217,3.364)}
\gppoint{gp mark 0}{(5.217,3.450)}
\gppoint{gp mark 0}{(5.217,3.364)}
\gppoint{gp mark 0}{(5.217,3.727)}
\gppoint{gp mark 0}{(5.217,3.529)}
\gppoint{gp mark 0}{(5.217,3.600)}
\gppoint{gp mark 0}{(5.217,3.756)}
\gppoint{gp mark 0}{(5.217,3.317)}
\gppoint{gp mark 0}{(5.217,4.099)}
\gppoint{gp mark 0}{(5.217,3.697)}
\gppoint{gp mark 0}{(5.217,3.450)}
\gppoint{gp mark 0}{(5.217,4.188)}
\gppoint{gp mark 0}{(5.217,3.887)}
\gppoint{gp mark 0}{(5.217,3.811)}
\gppoint{gp mark 0}{(5.217,3.862)}
\gppoint{gp mark 0}{(5.217,3.862)}
\gppoint{gp mark 0}{(5.217,3.529)}
\gppoint{gp mark 0}{(5.217,4.118)}
\gppoint{gp mark 0}{(5.217,4.118)}
\gppoint{gp mark 0}{(5.217,3.957)}
\gppoint{gp mark 0}{(5.217,3.957)}
\gppoint{gp mark 0}{(5.217,3.529)}
\gppoint{gp mark 0}{(5.217,3.934)}
\gppoint{gp mark 0}{(5.217,3.697)}
\gppoint{gp mark 0}{(5.217,3.887)}
\gppoint{gp mark 0}{(5.217,3.837)}
\gppoint{gp mark 0}{(5.217,3.934)}
\gppoint{gp mark 0}{(5.217,3.600)}
\gppoint{gp mark 0}{(5.217,4.041)}
\gppoint{gp mark 0}{(5.217,3.756)}
\gppoint{gp mark 0}{(5.217,3.934)}
\gppoint{gp mark 0}{(5.217,3.934)}
\gppoint{gp mark 0}{(5.217,3.727)}
\gppoint{gp mark 0}{(5.217,3.490)}
\gppoint{gp mark 0}{(5.217,3.096)}
\gppoint{gp mark 0}{(5.217,3.784)}
\gppoint{gp mark 0}{(5.217,3.529)}
\gppoint{gp mark 0}{(5.217,3.784)}
\gppoint{gp mark 0}{(5.217,4.171)}
\gppoint{gp mark 0}{(5.217,4.118)}
\gppoint{gp mark 0}{(5.217,4.136)}
\gppoint{gp mark 0}{(5.217,3.727)}
\gppoint{gp mark 0}{(5.217,3.697)}
\gppoint{gp mark 0}{(5.217,3.784)}
\gppoint{gp mark 0}{(5.217,3.634)}
\gppoint{gp mark 0}{(5.217,3.565)}
\gppoint{gp mark 0}{(5.217,3.837)}
\gppoint{gp mark 0}{(5.217,3.784)}
\gppoint{gp mark 0}{(5.217,3.490)}
\gppoint{gp mark 0}{(5.217,3.697)}
\gppoint{gp mark 0}{(5.217,3.784)}
\gppoint{gp mark 0}{(5.217,3.408)}
\gppoint{gp mark 0}{(5.217,3.756)}
\gppoint{gp mark 0}{(5.217,3.634)}
\gppoint{gp mark 0}{(5.217,3.784)}
\gppoint{gp mark 0}{(5.217,3.666)}
\gppoint{gp mark 0}{(5.217,3.837)}
\gppoint{gp mark 0}{(5.217,3.756)}
\gppoint{gp mark 0}{(5.217,3.697)}
\gppoint{gp mark 0}{(5.217,3.697)}
\gppoint{gp mark 0}{(5.217,3.934)}
\gppoint{gp mark 0}{(5.217,4.000)}
\gppoint{gp mark 0}{(5.217,3.214)}
\gppoint{gp mark 0}{(5.217,3.697)}
\gppoint{gp mark 0}{(5.217,3.862)}
\gppoint{gp mark 0}{(5.217,3.887)}
\gppoint{gp mark 0}{(5.217,3.697)}
\gppoint{gp mark 0}{(5.217,3.727)}
\gppoint{gp mark 0}{(5.217,3.529)}
\gppoint{gp mark 0}{(5.217,4.021)}
\gppoint{gp mark 0}{(5.217,3.600)}
\gppoint{gp mark 0}{(5.217,3.727)}
\gppoint{gp mark 0}{(5.217,4.080)}
\gppoint{gp mark 0}{(5.217,3.697)}
\gppoint{gp mark 0}{(5.217,3.666)}
\gppoint{gp mark 0}{(5.217,3.727)}
\gppoint{gp mark 0}{(5.217,3.666)}
\gppoint{gp mark 0}{(5.217,3.697)}
\gppoint{gp mark 0}{(5.217,3.697)}
\gppoint{gp mark 0}{(5.217,3.634)}
\gppoint{gp mark 0}{(5.217,3.666)}
\gppoint{gp mark 0}{(5.217,4.420)}
\gppoint{gp mark 0}{(5.217,3.697)}
\gppoint{gp mark 0}{(5.217,3.727)}
\gppoint{gp mark 0}{(5.217,3.697)}
\gppoint{gp mark 0}{(5.217,3.727)}
\gppoint{gp mark 0}{(5.217,3.784)}
\gppoint{gp mark 0}{(5.217,3.756)}
\gppoint{gp mark 0}{(5.217,3.697)}
\gppoint{gp mark 0}{(5.217,3.911)}
\gppoint{gp mark 0}{(5.217,3.529)}
\gppoint{gp mark 0}{(5.217,3.529)}
\gppoint{gp mark 0}{(5.217,4.021)}
\gppoint{gp mark 0}{(5.217,3.600)}
\gppoint{gp mark 0}{(5.217,4.021)}
\gppoint{gp mark 0}{(5.217,3.600)}
\gppoint{gp mark 0}{(5.217,3.837)}
\gppoint{gp mark 0}{(5.217,3.957)}
\gppoint{gp mark 0}{(5.217,3.364)}
\gppoint{gp mark 0}{(5.217,3.600)}
\gppoint{gp mark 0}{(5.217,4.000)}
\gppoint{gp mark 0}{(5.217,3.697)}
\gppoint{gp mark 0}{(5.217,4.000)}
\gppoint{gp mark 0}{(5.217,3.934)}
\gppoint{gp mark 0}{(5.217,4.000)}
\gppoint{gp mark 0}{(5.217,3.784)}
\gppoint{gp mark 0}{(5.217,3.784)}
\gppoint{gp mark 0}{(5.217,3.666)}
\gppoint{gp mark 0}{(5.217,3.666)}
\gppoint{gp mark 0}{(5.217,4.080)}
\gppoint{gp mark 0}{(5.217,3.727)}
\gppoint{gp mark 0}{(5.217,3.634)}
\gppoint{gp mark 0}{(5.217,3.727)}
\gppoint{gp mark 0}{(5.217,4.000)}
\gppoint{gp mark 0}{(5.217,3.565)}
\gppoint{gp mark 0}{(5.217,3.490)}
\gppoint{gp mark 0}{(5.217,3.784)}
\gppoint{gp mark 0}{(5.217,3.697)}
\gppoint{gp mark 0}{(5.217,3.756)}
\gppoint{gp mark 0}{(5.217,3.934)}
\gppoint{gp mark 0}{(5.217,3.450)}
\gppoint{gp mark 0}{(5.217,3.756)}
\gppoint{gp mark 0}{(5.217,3.957)}
\gppoint{gp mark 0}{(5.217,3.450)}
\gppoint{gp mark 0}{(5.217,3.862)}
\gppoint{gp mark 0}{(5.217,3.862)}
\gppoint{gp mark 0}{(5.217,3.697)}
\gppoint{gp mark 0}{(5.217,3.490)}
\gppoint{gp mark 0}{(5.217,3.934)}
\gppoint{gp mark 0}{(5.217,4.171)}
\gppoint{gp mark 0}{(5.217,4.381)}
\gppoint{gp mark 0}{(5.217,4.021)}
\gppoint{gp mark 0}{(5.217,3.727)}
\gppoint{gp mark 0}{(5.217,3.529)}
\gppoint{gp mark 0}{(5.217,3.450)}
\gppoint{gp mark 0}{(5.217,3.565)}
\gppoint{gp mark 0}{(5.217,3.529)}
\gppoint{gp mark 0}{(5.217,3.934)}
\gppoint{gp mark 0}{(5.217,3.957)}
\gppoint{gp mark 0}{(5.217,4.000)}
\gppoint{gp mark 0}{(5.217,4.395)}
\gppoint{gp mark 0}{(5.217,3.364)}
\gppoint{gp mark 0}{(5.217,3.862)}
\gppoint{gp mark 0}{(5.217,3.887)}
\gppoint{gp mark 0}{(5.217,4.041)}
\gppoint{gp mark 0}{(5.217,3.408)}
\gppoint{gp mark 0}{(5.217,3.666)}
\gppoint{gp mark 0}{(5.217,3.811)}
\gppoint{gp mark 0}{(5.217,3.490)}
\gppoint{gp mark 0}{(5.217,3.811)}
\gppoint{gp mark 0}{(5.217,3.887)}
\gppoint{gp mark 0}{(5.217,3.666)}
\gppoint{gp mark 0}{(5.217,3.697)}
\gppoint{gp mark 0}{(5.217,3.862)}
\gppoint{gp mark 0}{(5.217,4.041)}
\gppoint{gp mark 0}{(5.217,3.756)}
\gppoint{gp mark 0}{(5.217,4.136)}
\gppoint{gp mark 0}{(5.217,4.136)}
\gppoint{gp mark 0}{(5.217,3.887)}
\gppoint{gp mark 0}{(5.217,3.727)}
\gppoint{gp mark 0}{(5.217,3.408)}
\gppoint{gp mark 0}{(5.217,3.666)}
\gppoint{gp mark 0}{(5.217,3.565)}
\gppoint{gp mark 0}{(5.217,3.666)}
\gppoint{gp mark 0}{(5.217,4.136)}
\gppoint{gp mark 0}{(5.217,4.697)}
\gppoint{gp mark 0}{(5.217,3.756)}
\gppoint{gp mark 0}{(5.217,3.727)}
\gppoint{gp mark 0}{(5.217,3.756)}
\gppoint{gp mark 0}{(5.217,3.600)}
\gppoint{gp mark 0}{(5.217,3.837)}
\gppoint{gp mark 0}{(5.217,4.041)}
\gppoint{gp mark 0}{(5.217,3.666)}
\gppoint{gp mark 0}{(5.217,4.408)}
\gppoint{gp mark 0}{(5.217,3.529)}
\gppoint{gp mark 0}{(5.217,3.957)}
\gppoint{gp mark 0}{(5.217,4.080)}
\gppoint{gp mark 0}{(5.217,3.600)}
\gppoint{gp mark 0}{(5.217,3.565)}
\gppoint{gp mark 0}{(5.217,3.096)}
\gppoint{gp mark 0}{(5.217,3.697)}
\gppoint{gp mark 0}{(5.217,3.811)}
\gppoint{gp mark 0}{(5.217,3.756)}
\gppoint{gp mark 0}{(5.217,3.837)}
\gppoint{gp mark 0}{(5.217,4.237)}
\gppoint{gp mark 0}{(5.217,3.267)}
\gppoint{gp mark 0}{(5.217,3.837)}
\gppoint{gp mark 0}{(5.217,3.811)}
\gppoint{gp mark 0}{(5.217,4.283)}
\gppoint{gp mark 0}{(5.217,3.756)}
\gppoint{gp mark 0}{(5.217,4.204)}
\gppoint{gp mark 0}{(5.217,3.666)}
\gppoint{gp mark 0}{(5.217,3.862)}
\gppoint{gp mark 0}{(5.217,3.811)}
\gppoint{gp mark 0}{(5.217,3.490)}
\gppoint{gp mark 0}{(5.217,4.041)}
\gppoint{gp mark 0}{(5.217,3.666)}
\gppoint{gp mark 0}{(5.217,3.811)}
\gppoint{gp mark 0}{(5.217,3.529)}
\gppoint{gp mark 0}{(5.217,4.679)}
\gppoint{gp mark 0}{(5.217,4.312)}
\gppoint{gp mark 0}{(5.217,4.041)}
\gppoint{gp mark 0}{(5.217,3.727)}
\gppoint{gp mark 0}{(5.217,3.490)}
\gppoint{gp mark 0}{(5.217,3.784)}
\gppoint{gp mark 0}{(5.217,3.529)}
\gppoint{gp mark 0}{(5.217,3.887)}
\gppoint{gp mark 0}{(5.217,3.565)}
\gppoint{gp mark 0}{(5.217,3.979)}
\gppoint{gp mark 0}{(5.217,3.756)}
\gppoint{gp mark 0}{(5.217,3.450)}
\gppoint{gp mark 0}{(5.217,3.756)}
\gppoint{gp mark 0}{(5.217,3.837)}
\gppoint{gp mark 0}{(5.217,3.634)}
\gppoint{gp mark 0}{(5.217,3.862)}
\gppoint{gp mark 0}{(5.217,3.697)}
\gppoint{gp mark 0}{(5.217,3.600)}
\gppoint{gp mark 0}{(5.217,3.529)}
\gppoint{gp mark 0}{(5.217,3.837)}
\gppoint{gp mark 0}{(5.217,3.887)}
\gppoint{gp mark 0}{(5.217,3.666)}
\gppoint{gp mark 0}{(5.217,3.317)}
\gppoint{gp mark 0}{(5.260,3.934)}
\gppoint{gp mark 0}{(5.260,3.450)}
\gppoint{gp mark 0}{(5.260,3.756)}
\gppoint{gp mark 0}{(5.260,3.756)}
\gppoint{gp mark 0}{(5.260,3.756)}
\gppoint{gp mark 0}{(5.260,3.565)}
\gppoint{gp mark 0}{(5.260,3.214)}
\gppoint{gp mark 0}{(5.260,3.697)}
\gppoint{gp mark 0}{(5.260,3.756)}
\gppoint{gp mark 0}{(5.260,4.188)}
\gppoint{gp mark 0}{(5.260,3.784)}
\gppoint{gp mark 0}{(5.260,4.000)}
\gppoint{gp mark 0}{(5.260,4.041)}
\gppoint{gp mark 0}{(5.260,3.756)}
\gppoint{gp mark 0}{(5.260,3.727)}
\gppoint{gp mark 0}{(5.260,3.697)}
\gppoint{gp mark 0}{(5.260,4.021)}
\gppoint{gp mark 0}{(5.260,3.408)}
\gppoint{gp mark 0}{(5.260,3.666)}
\gppoint{gp mark 0}{(5.260,3.666)}
\gppoint{gp mark 0}{(5.260,3.666)}
\gppoint{gp mark 0}{(5.260,3.784)}
\gppoint{gp mark 0}{(5.260,4.188)}
\gppoint{gp mark 0}{(5.260,3.214)}
\gppoint{gp mark 0}{(5.260,3.756)}
\gppoint{gp mark 0}{(5.260,4.268)}
\gppoint{gp mark 0}{(5.260,3.529)}
\gppoint{gp mark 0}{(5.260,4.000)}
\gppoint{gp mark 0}{(5.260,3.979)}
\gppoint{gp mark 0}{(5.260,3.697)}
\gppoint{gp mark 0}{(5.260,3.214)}
\gppoint{gp mark 0}{(5.260,4.153)}
\gppoint{gp mark 0}{(5.260,3.030)}
\gppoint{gp mark 0}{(5.260,3.450)}
\gppoint{gp mark 0}{(5.260,3.727)}
\gppoint{gp mark 0}{(5.260,3.450)}
\gppoint{gp mark 0}{(5.260,3.529)}
\gppoint{gp mark 0}{(5.260,4.021)}
\gppoint{gp mark 0}{(5.260,4.268)}
\gppoint{gp mark 0}{(5.260,4.660)}
\gppoint{gp mark 0}{(5.260,3.957)}
\gppoint{gp mark 0}{(5.260,3.911)}
\gppoint{gp mark 0}{(5.260,3.784)}
\gppoint{gp mark 0}{(5.260,3.697)}
\gppoint{gp mark 0}{(5.260,3.697)}
\gppoint{gp mark 0}{(5.260,3.756)}
\gppoint{gp mark 0}{(5.260,3.727)}
\gppoint{gp mark 0}{(5.260,3.727)}
\gppoint{gp mark 0}{(5.260,3.934)}
\gppoint{gp mark 0}{(5.260,3.934)}
\gppoint{gp mark 0}{(5.260,3.837)}
\gppoint{gp mark 0}{(5.260,4.021)}
\gppoint{gp mark 0}{(5.260,3.666)}
\gppoint{gp mark 0}{(5.260,3.697)}
\gppoint{gp mark 0}{(5.260,3.934)}
\gppoint{gp mark 0}{(5.260,3.957)}
\gppoint{gp mark 0}{(5.260,3.267)}
\gppoint{gp mark 0}{(5.260,3.784)}
\gppoint{gp mark 0}{(5.260,3.727)}
\gppoint{gp mark 0}{(5.260,3.697)}
\gppoint{gp mark 0}{(5.260,3.697)}
\gppoint{gp mark 0}{(5.260,3.600)}
\gppoint{gp mark 0}{(5.260,3.784)}
\gppoint{gp mark 0}{(5.260,4.327)}
\gppoint{gp mark 0}{(5.260,3.911)}
\gppoint{gp mark 0}{(5.260,3.911)}
\gppoint{gp mark 0}{(5.260,3.634)}
\gppoint{gp mark 0}{(5.260,3.634)}
\gppoint{gp mark 0}{(5.260,3.934)}
\gppoint{gp mark 0}{(5.260,3.634)}
\gppoint{gp mark 0}{(5.260,3.634)}
\gppoint{gp mark 0}{(5.260,3.634)}
\gppoint{gp mark 0}{(5.260,4.080)}
\gppoint{gp mark 0}{(5.260,3.911)}
\gppoint{gp mark 0}{(5.260,4.188)}
\gppoint{gp mark 0}{(5.260,3.811)}
\gppoint{gp mark 0}{(5.260,4.153)}
\gppoint{gp mark 0}{(5.260,3.727)}
\gppoint{gp mark 0}{(5.260,3.697)}
\gppoint{gp mark 0}{(5.260,3.634)}
\gppoint{gp mark 0}{(5.260,3.565)}
\gppoint{gp mark 0}{(5.260,3.784)}
\gppoint{gp mark 0}{(5.260,3.934)}
\gppoint{gp mark 0}{(5.260,4.041)}
\gppoint{gp mark 0}{(5.260,3.887)}
\gppoint{gp mark 0}{(5.260,3.634)}
\gppoint{gp mark 0}{(5.260,3.727)}
\gppoint{gp mark 0}{(5.260,3.957)}
\gppoint{gp mark 0}{(5.260,4.021)}
\gppoint{gp mark 0}{(5.260,3.756)}
\gppoint{gp mark 0}{(5.260,3.565)}
\gppoint{gp mark 0}{(5.260,3.979)}
\gppoint{gp mark 0}{(5.260,3.727)}
\gppoint{gp mark 0}{(5.260,3.727)}
\gppoint{gp mark 0}{(5.260,3.811)}
\gppoint{gp mark 0}{(5.260,3.811)}
\gppoint{gp mark 0}{(5.260,3.862)}
\gppoint{gp mark 0}{(5.260,3.450)}
\gppoint{gp mark 0}{(5.260,3.490)}
\gppoint{gp mark 0}{(5.260,3.811)}
\gppoint{gp mark 0}{(5.260,3.887)}
\gppoint{gp mark 0}{(5.260,3.756)}
\gppoint{gp mark 0}{(5.260,4.099)}
\gppoint{gp mark 0}{(5.260,3.934)}
\gppoint{gp mark 0}{(5.260,4.099)}
\gppoint{gp mark 0}{(5.260,4.099)}
\gppoint{gp mark 0}{(5.260,4.118)}
\gppoint{gp mark 0}{(5.260,3.862)}
\gppoint{gp mark 0}{(5.260,4.099)}
\gppoint{gp mark 0}{(5.260,4.099)}
\gppoint{gp mark 0}{(5.260,3.862)}
\gppoint{gp mark 0}{(5.260,3.634)}
\gppoint{gp mark 0}{(5.260,4.783)}
\gppoint{gp mark 0}{(5.260,4.099)}
\gppoint{gp mark 0}{(5.260,4.516)}
\gppoint{gp mark 0}{(5.260,3.666)}
\gppoint{gp mark 0}{(5.260,3.727)}
\gppoint{gp mark 0}{(5.260,3.837)}
\gppoint{gp mark 0}{(5.260,3.756)}
\gppoint{gp mark 0}{(5.260,4.368)}
\gppoint{gp mark 0}{(5.260,3.666)}
\gppoint{gp mark 0}{(5.260,4.221)}
\gppoint{gp mark 0}{(5.260,4.080)}
\gppoint{gp mark 0}{(5.260,3.450)}
\gppoint{gp mark 0}{(5.260,3.450)}
\gppoint{gp mark 0}{(5.260,3.887)}
\gppoint{gp mark 0}{(5.260,3.911)}
\gppoint{gp mark 0}{(5.260,3.756)}
\gppoint{gp mark 0}{(5.260,3.727)}
\gppoint{gp mark 0}{(5.260,4.171)}
\gppoint{gp mark 0}{(5.260,3.979)}
\gppoint{gp mark 0}{(5.260,3.911)}
\gppoint{gp mark 0}{(5.260,3.911)}
\gppoint{gp mark 0}{(5.260,3.911)}
\gppoint{gp mark 0}{(5.260,3.697)}
\gppoint{gp mark 0}{(5.260,3.784)}
\gppoint{gp mark 0}{(5.260,3.837)}
\gppoint{gp mark 0}{(5.260,3.979)}
\gppoint{gp mark 0}{(5.260,3.756)}
\gppoint{gp mark 0}{(5.260,3.934)}
\gppoint{gp mark 0}{(5.260,3.979)}
\gppoint{gp mark 0}{(5.260,3.634)}
\gppoint{gp mark 0}{(5.260,3.862)}
\gppoint{gp mark 0}{(5.260,3.837)}
\gppoint{gp mark 0}{(5.260,3.837)}
\gppoint{gp mark 0}{(5.260,3.727)}
\gppoint{gp mark 0}{(5.260,3.364)}
\gppoint{gp mark 0}{(5.260,3.364)}
\gppoint{gp mark 0}{(5.260,3.887)}
\gppoint{gp mark 0}{(5.260,3.697)}
\gppoint{gp mark 0}{(5.260,3.364)}
\gppoint{gp mark 0}{(5.260,3.811)}
\gppoint{gp mark 0}{(5.260,4.041)}
\gppoint{gp mark 0}{(5.260,3.811)}
\gppoint{gp mark 0}{(5.260,3.697)}
\gppoint{gp mark 0}{(5.260,4.312)}
\gppoint{gp mark 0}{(5.260,3.811)}
\gppoint{gp mark 0}{(5.260,3.784)}
\gppoint{gp mark 0}{(5.260,3.784)}
\gppoint{gp mark 0}{(5.260,4.099)}
\gppoint{gp mark 0}{(5.260,3.727)}
\gppoint{gp mark 0}{(5.260,3.317)}
\gppoint{gp mark 0}{(5.260,3.957)}
\gppoint{gp mark 0}{(5.260,4.298)}
\gppoint{gp mark 0}{(5.260,3.934)}
\gppoint{gp mark 0}{(5.260,3.634)}
\gppoint{gp mark 0}{(5.260,3.811)}
\gppoint{gp mark 0}{(5.260,4.221)}
\gppoint{gp mark 0}{(5.260,3.934)}
\gppoint{gp mark 0}{(5.260,3.364)}
\gppoint{gp mark 0}{(5.260,3.911)}
\gppoint{gp mark 0}{(5.260,3.600)}
\gppoint{gp mark 0}{(5.260,3.957)}
\gppoint{gp mark 0}{(5.260,3.979)}
\gppoint{gp mark 0}{(5.260,3.756)}
\gppoint{gp mark 0}{(5.260,3.911)}
\gppoint{gp mark 0}{(5.260,3.979)}
\gppoint{gp mark 0}{(5.260,3.634)}
\gppoint{gp mark 0}{(5.260,3.911)}
\gppoint{gp mark 0}{(5.260,3.784)}
\gppoint{gp mark 0}{(5.260,3.837)}
\gppoint{gp mark 0}{(5.260,3.697)}
\gppoint{gp mark 0}{(5.260,3.697)}
\gppoint{gp mark 0}{(5.260,3.756)}
\gppoint{gp mark 0}{(5.260,3.697)}
\gppoint{gp mark 0}{(5.260,3.911)}
\gppoint{gp mark 0}{(5.260,4.171)}
\gppoint{gp mark 0}{(5.260,3.887)}
\gppoint{gp mark 0}{(5.260,3.862)}
\gppoint{gp mark 0}{(5.260,3.697)}
\gppoint{gp mark 0}{(5.260,3.811)}
\gppoint{gp mark 0}{(5.260,3.600)}
\gppoint{gp mark 0}{(5.260,3.364)}
\gppoint{gp mark 0}{(5.260,3.837)}
\gppoint{gp mark 0}{(5.260,3.784)}
\gppoint{gp mark 0}{(5.260,3.666)}
\gppoint{gp mark 0}{(5.260,3.666)}
\gppoint{gp mark 0}{(5.260,3.811)}
\gppoint{gp mark 0}{(5.260,3.727)}
\gppoint{gp mark 0}{(5.260,4.041)}
\gppoint{gp mark 0}{(5.260,4.221)}
\gppoint{gp mark 0}{(5.260,3.727)}
\gppoint{gp mark 0}{(5.260,3.450)}
\gppoint{gp mark 0}{(5.260,3.979)}
\gppoint{gp mark 0}{(5.260,4.000)}
\gppoint{gp mark 0}{(5.260,3.911)}
\gppoint{gp mark 0}{(5.260,3.911)}
\gppoint{gp mark 0}{(5.260,3.490)}
\gppoint{gp mark 0}{(5.260,3.862)}
\gppoint{gp mark 0}{(5.260,3.666)}
\gppoint{gp mark 0}{(5.260,3.979)}
\gppoint{gp mark 0}{(5.260,3.887)}
\gppoint{gp mark 0}{(5.260,3.811)}
\gppoint{gp mark 0}{(5.302,4.136)}
\gppoint{gp mark 0}{(5.302,4.560)}
\gppoint{gp mark 0}{(5.302,3.911)}
\gppoint{gp mark 0}{(5.302,4.080)}
\gppoint{gp mark 0}{(5.302,3.697)}
\gppoint{gp mark 0}{(5.302,4.268)}
\gppoint{gp mark 0}{(5.302,3.811)}
\gppoint{gp mark 0}{(5.302,3.934)}
\gppoint{gp mark 0}{(5.302,3.811)}
\gppoint{gp mark 0}{(5.302,3.697)}
\gppoint{gp mark 0}{(5.302,3.697)}
\gppoint{gp mark 0}{(5.302,3.887)}
\gppoint{gp mark 0}{(5.302,4.612)}
\gppoint{gp mark 0}{(5.302,3.887)}
\gppoint{gp mark 0}{(5.302,3.934)}
\gppoint{gp mark 0}{(5.302,3.811)}
\gppoint{gp mark 0}{(5.302,3.887)}
\gppoint{gp mark 0}{(5.302,3.600)}
\gppoint{gp mark 0}{(5.302,4.099)}
\gppoint{gp mark 0}{(5.302,4.527)}
\gppoint{gp mark 0}{(5.302,3.490)}
\gppoint{gp mark 0}{(5.302,3.697)}
\gppoint{gp mark 0}{(5.302,3.979)}
\gppoint{gp mark 0}{(5.302,3.756)}
\gppoint{gp mark 0}{(5.302,4.153)}
\gppoint{gp mark 0}{(5.302,3.565)}
\gppoint{gp mark 0}{(5.302,3.697)}
\gppoint{gp mark 0}{(5.302,3.887)}
\gppoint{gp mark 0}{(5.302,3.784)}
\gppoint{gp mark 0}{(5.302,3.697)}
\gppoint{gp mark 0}{(5.302,3.756)}
\gppoint{gp mark 0}{(5.302,3.784)}
\gppoint{gp mark 0}{(5.302,4.252)}
\gppoint{gp mark 0}{(5.302,4.099)}
\gppoint{gp mark 0}{(5.302,3.887)}
\gppoint{gp mark 0}{(5.302,3.979)}
\gppoint{gp mark 0}{(5.302,3.490)}
\gppoint{gp mark 0}{(5.302,3.756)}
\gppoint{gp mark 0}{(5.302,3.634)}
\gppoint{gp mark 0}{(5.302,3.666)}
\gppoint{gp mark 0}{(5.302,4.000)}
\gppoint{gp mark 0}{(5.302,3.862)}
\gppoint{gp mark 0}{(5.302,4.061)}
\gppoint{gp mark 0}{(5.302,3.756)}
\gppoint{gp mark 0}{(5.302,3.934)}
\gppoint{gp mark 0}{(5.302,3.634)}
\gppoint{gp mark 0}{(5.302,3.862)}
\gppoint{gp mark 0}{(5.302,3.862)}
\gppoint{gp mark 0}{(5.302,3.811)}
\gppoint{gp mark 0}{(5.302,4.041)}
\gppoint{gp mark 0}{(5.302,4.283)}
\gppoint{gp mark 0}{(5.302,3.756)}
\gppoint{gp mark 0}{(5.302,4.041)}
\gppoint{gp mark 0}{(5.302,3.934)}
\gppoint{gp mark 0}{(5.302,3.911)}
\gppoint{gp mark 0}{(5.302,3.862)}
\gppoint{gp mark 0}{(5.302,3.862)}
\gppoint{gp mark 0}{(5.302,4.171)}
\gppoint{gp mark 0}{(5.302,4.041)}
\gppoint{gp mark 0}{(5.302,4.041)}
\gppoint{gp mark 0}{(5.302,3.887)}
\gppoint{gp mark 0}{(5.302,4.021)}
\gppoint{gp mark 0}{(5.302,3.887)}
\gppoint{gp mark 0}{(5.302,4.041)}
\gppoint{gp mark 0}{(5.302,3.911)}
\gppoint{gp mark 0}{(5.302,4.204)}
\gppoint{gp mark 0}{(5.302,4.041)}
\gppoint{gp mark 0}{(5.302,4.188)}
\gppoint{gp mark 0}{(5.302,3.887)}
\gppoint{gp mark 0}{(5.302,3.887)}
\gppoint{gp mark 0}{(5.302,3.887)}
\gppoint{gp mark 0}{(5.302,3.811)}
\gppoint{gp mark 0}{(5.302,3.979)}
\gppoint{gp mark 0}{(5.302,3.887)}
\gppoint{gp mark 0}{(5.302,3.450)}
\gppoint{gp mark 0}{(5.302,3.756)}
\gppoint{gp mark 0}{(5.302,3.634)}
\gppoint{gp mark 0}{(5.302,3.450)}
\gppoint{gp mark 0}{(5.302,3.811)}
\gppoint{gp mark 0}{(5.302,3.887)}
\gppoint{gp mark 0}{(5.302,3.784)}
\gppoint{gp mark 0}{(5.302,3.784)}
\gppoint{gp mark 0}{(5.302,4.118)}
\gppoint{gp mark 0}{(5.302,3.756)}
\gppoint{gp mark 0}{(5.302,4.283)}
\gppoint{gp mark 0}{(5.302,3.887)}
\gppoint{gp mark 0}{(5.302,3.784)}
\gppoint{gp mark 0}{(5.302,3.784)}
\gppoint{gp mark 0}{(5.302,4.041)}
\gppoint{gp mark 0}{(5.302,3.957)}
\gppoint{gp mark 0}{(5.302,3.934)}
\gppoint{gp mark 0}{(5.302,3.784)}
\gppoint{gp mark 0}{(5.302,4.021)}
\gppoint{gp mark 0}{(5.302,4.171)}
\gppoint{gp mark 0}{(5.302,4.118)}
\gppoint{gp mark 0}{(5.302,3.727)}
\gppoint{gp mark 0}{(5.302,4.118)}
\gppoint{gp mark 0}{(5.302,3.784)}
\gppoint{gp mark 0}{(5.302,3.634)}
\gppoint{gp mark 0}{(5.302,3.887)}
\gppoint{gp mark 0}{(5.302,3.862)}
\gppoint{gp mark 0}{(5.302,3.837)}
\gppoint{gp mark 0}{(5.302,3.727)}
\gppoint{gp mark 0}{(5.302,3.811)}
\gppoint{gp mark 0}{(5.302,4.061)}
\gppoint{gp mark 0}{(5.302,3.784)}
\gppoint{gp mark 0}{(5.302,4.237)}
\gppoint{gp mark 0}{(5.302,3.565)}
\gppoint{gp mark 0}{(5.302,3.490)}
\gppoint{gp mark 0}{(5.302,3.887)}
\gppoint{gp mark 0}{(5.302,3.784)}
\gppoint{gp mark 0}{(5.302,3.862)}
\gppoint{gp mark 0}{(5.302,3.837)}
\gppoint{gp mark 0}{(5.302,3.784)}
\gppoint{gp mark 0}{(5.302,4.080)}
\gppoint{gp mark 0}{(5.302,3.837)}
\gppoint{gp mark 0}{(5.302,3.887)}
\gppoint{gp mark 0}{(5.302,3.267)}
\gppoint{gp mark 0}{(5.302,4.283)}
\gppoint{gp mark 0}{(5.302,3.697)}
\gppoint{gp mark 0}{(5.302,3.784)}
\gppoint{gp mark 0}{(5.302,4.341)}
\gppoint{gp mark 0}{(5.302,3.697)}
\gppoint{gp mark 0}{(5.302,4.000)}
\gppoint{gp mark 0}{(5.302,3.934)}
\gppoint{gp mark 0}{(5.302,3.756)}
\gppoint{gp mark 0}{(5.302,3.666)}
\gppoint{gp mark 0}{(5.302,4.118)}
\gppoint{gp mark 0}{(5.302,4.000)}
\gppoint{gp mark 0}{(5.302,3.727)}
\gppoint{gp mark 0}{(5.302,4.080)}
\gppoint{gp mark 0}{(5.302,3.957)}
\gppoint{gp mark 0}{(5.302,3.490)}
\gppoint{gp mark 0}{(5.302,4.268)}
\gppoint{gp mark 0}{(5.302,3.979)}
\gppoint{gp mark 0}{(5.302,4.327)}
\gppoint{gp mark 0}{(5.302,3.600)}
\gppoint{gp mark 0}{(5.302,3.784)}
\gppoint{gp mark 0}{(5.302,3.450)}
\gppoint{gp mark 0}{(5.302,3.267)}
\gppoint{gp mark 0}{(5.302,3.811)}
\gppoint{gp mark 0}{(5.302,4.750)}
\gppoint{gp mark 0}{(5.302,3.911)}
\gppoint{gp mark 0}{(5.302,3.811)}
\gppoint{gp mark 0}{(5.302,3.911)}
\gppoint{gp mark 0}{(5.302,3.811)}
\gppoint{gp mark 0}{(5.302,3.634)}
\gppoint{gp mark 0}{(5.302,3.911)}
\gppoint{gp mark 0}{(5.302,3.634)}
\gppoint{gp mark 0}{(5.302,4.000)}
\gppoint{gp mark 0}{(5.302,3.837)}
\gppoint{gp mark 0}{(5.302,4.099)}
\gppoint{gp mark 0}{(5.302,4.327)}
\gppoint{gp mark 0}{(5.302,4.171)}
\gppoint{gp mark 0}{(5.302,4.080)}
\gppoint{gp mark 0}{(5.302,3.957)}
\gppoint{gp mark 0}{(5.302,4.341)}
\gppoint{gp mark 0}{(5.302,4.099)}
\gppoint{gp mark 0}{(5.302,3.837)}
\gppoint{gp mark 0}{(5.302,4.153)}
\gppoint{gp mark 0}{(5.302,4.341)}
\gppoint{gp mark 0}{(5.302,4.171)}
\gppoint{gp mark 0}{(5.302,3.727)}
\gppoint{gp mark 0}{(5.302,3.697)}
\gppoint{gp mark 0}{(5.302,3.727)}
\gppoint{gp mark 0}{(5.302,3.784)}
\gppoint{gp mark 0}{(5.302,3.756)}
\gppoint{gp mark 0}{(5.302,4.041)}
\gppoint{gp mark 0}{(5.302,3.600)}
\gppoint{gp mark 0}{(5.302,4.153)}
\gppoint{gp mark 0}{(5.302,4.493)}
\gppoint{gp mark 0}{(5.302,3.450)}
\gppoint{gp mark 0}{(5.302,3.811)}
\gppoint{gp mark 0}{(5.302,3.934)}
\gppoint{gp mark 0}{(5.302,3.887)}
\gppoint{gp mark 0}{(5.302,4.021)}
\gppoint{gp mark 0}{(5.302,4.581)}
\gppoint{gp mark 0}{(5.302,3.529)}
\gppoint{gp mark 0}{(5.302,3.887)}
\gppoint{gp mark 0}{(5.302,3.634)}
\gppoint{gp mark 0}{(5.302,4.021)}
\gppoint{gp mark 0}{(5.302,4.237)}
\gppoint{gp mark 0}{(5.302,4.237)}
\gppoint{gp mark 0}{(5.302,4.341)}
\gppoint{gp mark 0}{(5.302,4.153)}
\gppoint{gp mark 0}{(5.302,4.188)}
\gppoint{gp mark 0}{(5.302,4.298)}
\gppoint{gp mark 0}{(5.302,4.420)}
\gppoint{gp mark 0}{(5.302,3.837)}
\gppoint{gp mark 0}{(5.302,3.887)}
\gppoint{gp mark 0}{(5.302,3.934)}
\gppoint{gp mark 0}{(5.302,3.600)}
\gppoint{gp mark 0}{(5.302,3.911)}
\gppoint{gp mark 0}{(5.302,3.811)}
\gppoint{gp mark 0}{(5.342,3.756)}
\gppoint{gp mark 0}{(5.342,4.341)}
\gppoint{gp mark 0}{(5.342,4.099)}
\gppoint{gp mark 0}{(5.342,3.784)}
\gppoint{gp mark 0}{(5.342,3.837)}
\gppoint{gp mark 0}{(5.342,3.600)}
\gppoint{gp mark 0}{(5.342,3.837)}
\gppoint{gp mark 0}{(5.342,4.298)}
\gppoint{gp mark 0}{(5.342,4.188)}
\gppoint{gp mark 0}{(5.342,3.600)}
\gppoint{gp mark 0}{(5.342,4.118)}
\gppoint{gp mark 0}{(5.342,4.080)}
\gppoint{gp mark 0}{(5.342,3.837)}
\gppoint{gp mark 0}{(5.342,4.099)}
\gppoint{gp mark 0}{(5.342,3.214)}
\gppoint{gp mark 0}{(5.342,4.000)}
\gppoint{gp mark 0}{(5.342,3.837)}
\gppoint{gp mark 0}{(5.342,4.381)}
\gppoint{gp mark 0}{(5.342,3.862)}
\gppoint{gp mark 0}{(5.342,4.445)}
\gppoint{gp mark 0}{(5.342,4.298)}
\gppoint{gp mark 0}{(5.342,3.666)}
\gppoint{gp mark 0}{(5.342,4.298)}
\gppoint{gp mark 0}{(5.342,3.697)}
\gppoint{gp mark 0}{(5.342,4.268)}
\gppoint{gp mark 0}{(5.342,4.298)}
\gppoint{gp mark 0}{(5.342,3.934)}
\gppoint{gp mark 0}{(5.342,4.000)}
\gppoint{gp mark 0}{(5.342,3.957)}
\gppoint{gp mark 0}{(5.342,4.188)}
\gppoint{gp mark 0}{(5.342,3.697)}
\gppoint{gp mark 0}{(5.342,3.911)}
\gppoint{gp mark 0}{(5.342,4.252)}
\gppoint{gp mark 0}{(5.342,4.298)}
\gppoint{gp mark 0}{(5.342,3.979)}
\gppoint{gp mark 0}{(5.342,4.469)}
\gppoint{gp mark 0}{(5.342,3.030)}
\gppoint{gp mark 0}{(5.342,3.934)}
\gppoint{gp mark 0}{(5.342,3.784)}
\gppoint{gp mark 0}{(5.342,3.529)}
\gppoint{gp mark 0}{(5.342,3.979)}
\gppoint{gp mark 0}{(5.342,3.934)}
\gppoint{gp mark 0}{(5.342,3.811)}
\gppoint{gp mark 0}{(5.342,4.000)}
\gppoint{gp mark 0}{(5.342,4.715)}
\gppoint{gp mark 0}{(5.342,3.450)}
\gppoint{gp mark 0}{(5.342,4.312)}
\gppoint{gp mark 0}{(5.342,3.666)}
\gppoint{gp mark 0}{(5.342,3.837)}
\gppoint{gp mark 0}{(5.342,4.204)}
\gppoint{gp mark 0}{(5.342,3.666)}
\gppoint{gp mark 0}{(5.342,3.811)}
\gppoint{gp mark 0}{(5.342,4.312)}
\gppoint{gp mark 0}{(5.342,4.312)}
\gppoint{gp mark 0}{(5.342,3.957)}
\gppoint{gp mark 0}{(5.342,3.811)}
\gppoint{gp mark 0}{(5.342,3.887)}
\gppoint{gp mark 0}{(5.342,3.600)}
\gppoint{gp mark 0}{(5.342,3.862)}
\gppoint{gp mark 0}{(5.342,4.237)}
\gppoint{gp mark 0}{(5.342,3.811)}
\gppoint{gp mark 0}{(5.342,4.021)}
\gppoint{gp mark 0}{(5.342,4.171)}
\gppoint{gp mark 0}{(5.342,3.756)}
\gppoint{gp mark 0}{(5.342,4.000)}
\gppoint{gp mark 0}{(5.342,4.061)}
\gppoint{gp mark 0}{(5.342,4.080)}
\gppoint{gp mark 0}{(5.342,3.887)}
\gppoint{gp mark 0}{(5.342,4.312)}
\gppoint{gp mark 0}{(5.342,3.887)}
\gppoint{gp mark 0}{(5.342,3.756)}
\gppoint{gp mark 0}{(5.342,3.887)}
\gppoint{gp mark 0}{(5.342,3.364)}
\gppoint{gp mark 0}{(5.342,3.957)}
\gppoint{gp mark 0}{(5.342,3.979)}
\gppoint{gp mark 0}{(5.342,4.041)}
\gppoint{gp mark 0}{(5.342,3.811)}
\gppoint{gp mark 0}{(5.342,3.862)}
\gppoint{gp mark 0}{(5.342,3.837)}
\gppoint{gp mark 0}{(5.342,3.911)}
\gppoint{gp mark 0}{(5.342,3.727)}
\gppoint{gp mark 0}{(5.342,4.268)}
\gppoint{gp mark 0}{(5.342,3.666)}
\gppoint{gp mark 0}{(5.342,3.756)}
\gppoint{gp mark 0}{(5.342,3.600)}
\gppoint{gp mark 0}{(5.342,3.490)}
\gppoint{gp mark 0}{(5.342,3.756)}
\gppoint{gp mark 0}{(5.342,3.811)}
\gppoint{gp mark 0}{(5.342,3.979)}
\gppoint{gp mark 0}{(5.342,3.837)}
\gppoint{gp mark 0}{(5.342,3.756)}
\gppoint{gp mark 0}{(5.342,3.934)}
\gppoint{gp mark 0}{(5.342,3.811)}
\gppoint{gp mark 0}{(5.342,3.837)}
\gppoint{gp mark 0}{(5.342,3.756)}
\gppoint{gp mark 0}{(5.342,3.934)}
\gppoint{gp mark 0}{(5.342,3.490)}
\gppoint{gp mark 0}{(5.342,4.021)}
\gppoint{gp mark 0}{(5.342,4.000)}
\gppoint{gp mark 0}{(5.342,4.000)}
\gppoint{gp mark 0}{(5.342,3.887)}
\gppoint{gp mark 0}{(5.342,3.490)}
\gppoint{gp mark 0}{(5.342,3.887)}
\gppoint{gp mark 0}{(5.342,4.041)}
\gppoint{gp mark 0}{(5.342,4.268)}
\gppoint{gp mark 0}{(5.342,3.887)}
\gppoint{gp mark 0}{(5.342,3.887)}
\gppoint{gp mark 0}{(5.342,3.529)}
\gppoint{gp mark 0}{(5.342,3.957)}
\gppoint{gp mark 0}{(5.342,4.099)}
\gppoint{gp mark 0}{(5.342,3.979)}
\gppoint{gp mark 0}{(5.342,3.727)}
\gppoint{gp mark 0}{(5.342,3.666)}
\gppoint{gp mark 0}{(5.342,3.887)}
\gppoint{gp mark 0}{(5.342,3.911)}
\gppoint{gp mark 0}{(5.342,3.756)}
\gppoint{gp mark 0}{(5.342,4.080)}
\gppoint{gp mark 0}{(5.342,3.911)}
\gppoint{gp mark 0}{(5.342,3.529)}
\gppoint{gp mark 0}{(5.342,3.784)}
\gppoint{gp mark 0}{(5.342,3.887)}
\gppoint{gp mark 0}{(5.342,3.408)}
\gppoint{gp mark 0}{(5.342,3.364)}
\gppoint{gp mark 0}{(5.342,3.811)}
\gppoint{gp mark 0}{(5.342,3.634)}
\gppoint{gp mark 0}{(5.342,3.784)}
\gppoint{gp mark 0}{(5.342,3.756)}
\gppoint{gp mark 0}{(5.342,4.061)}
\gppoint{gp mark 0}{(5.342,3.862)}
\gppoint{gp mark 0}{(5.342,3.862)}
\gppoint{gp mark 0}{(5.342,4.021)}
\gppoint{gp mark 0}{(5.342,3.666)}
\gppoint{gp mark 0}{(5.342,3.600)}
\gppoint{gp mark 0}{(5.342,3.911)}
\gppoint{gp mark 0}{(5.342,3.911)}
\gppoint{gp mark 0}{(5.342,3.634)}
\gppoint{gp mark 0}{(5.342,3.784)}
\gppoint{gp mark 0}{(5.342,4.041)}
\gppoint{gp mark 0}{(5.342,4.061)}
\gppoint{gp mark 0}{(5.342,3.408)}
\gppoint{gp mark 0}{(5.342,3.957)}
\gppoint{gp mark 0}{(5.342,4.327)}
\gppoint{gp mark 0}{(5.342,4.000)}
\gppoint{gp mark 0}{(5.342,4.136)}
\gppoint{gp mark 0}{(5.342,3.862)}
\gppoint{gp mark 0}{(5.342,3.634)}
\gppoint{gp mark 0}{(5.342,3.634)}
\gppoint{gp mark 0}{(5.342,3.756)}
\gppoint{gp mark 0}{(5.342,3.666)}
\gppoint{gp mark 0}{(5.342,3.784)}
\gppoint{gp mark 0}{(5.342,3.934)}
\gppoint{gp mark 0}{(5.342,3.979)}
\gppoint{gp mark 0}{(5.342,3.756)}
\gppoint{gp mark 0}{(5.342,3.887)}
\gppoint{gp mark 0}{(5.342,4.136)}
\gppoint{gp mark 0}{(5.342,4.153)}
\gppoint{gp mark 0}{(5.342,3.756)}
\gppoint{gp mark 0}{(5.342,4.445)}
\gppoint{gp mark 0}{(5.342,3.364)}
\gppoint{gp mark 0}{(5.342,3.157)}
\gppoint{gp mark 0}{(5.342,3.756)}
\gppoint{gp mark 0}{(5.342,4.838)}
\gppoint{gp mark 0}{(5.342,3.811)}
\gppoint{gp mark 0}{(5.342,3.887)}
\gppoint{gp mark 0}{(5.342,4.153)}
\gppoint{gp mark 0}{(5.342,4.221)}
\gppoint{gp mark 0}{(5.342,3.364)}
\gppoint{gp mark 0}{(5.342,3.634)}
\gppoint{gp mark 0}{(5.342,3.727)}
\gppoint{gp mark 0}{(5.342,3.979)}
\gppoint{gp mark 0}{(5.342,3.697)}
\gppoint{gp mark 0}{(5.342,4.080)}
\gppoint{gp mark 0}{(5.342,4.136)}
\gppoint{gp mark 0}{(5.342,4.136)}
\gppoint{gp mark 0}{(5.342,3.450)}
\gppoint{gp mark 0}{(5.342,3.934)}
\gppoint{gp mark 0}{(5.342,3.784)}
\gppoint{gp mark 0}{(5.342,4.000)}
\gppoint{gp mark 0}{(5.342,4.080)}
\gppoint{gp mark 0}{(5.342,3.911)}
\gppoint{gp mark 0}{(5.342,4.080)}
\gppoint{gp mark 0}{(5.342,3.727)}
\gppoint{gp mark 0}{(5.342,3.837)}
\gppoint{gp mark 0}{(5.342,3.565)}
\gppoint{gp mark 0}{(5.342,3.697)}
\gppoint{gp mark 0}{(5.342,4.118)}
\gppoint{gp mark 0}{(5.342,3.934)}
\gppoint{gp mark 0}{(5.342,3.666)}
\gppoint{gp mark 0}{(5.342,4.000)}
\gppoint{gp mark 0}{(5.342,3.727)}
\gppoint{gp mark 0}{(5.342,3.784)}
\gppoint{gp mark 0}{(5.381,3.957)}
\gppoint{gp mark 0}{(5.381,3.565)}
\gppoint{gp mark 0}{(5.381,4.061)}
\gppoint{gp mark 0}{(5.381,4.061)}
\gppoint{gp mark 0}{(5.381,3.934)}
\gppoint{gp mark 0}{(5.381,3.957)}
\gppoint{gp mark 0}{(5.381,3.565)}
\gppoint{gp mark 0}{(5.381,4.000)}
\gppoint{gp mark 0}{(5.381,3.911)}
\gppoint{gp mark 0}{(5.381,4.021)}
\gppoint{gp mark 0}{(5.381,4.000)}
\gppoint{gp mark 0}{(5.381,3.634)}
\gppoint{gp mark 0}{(5.381,3.666)}
\gppoint{gp mark 0}{(5.381,4.099)}
\gppoint{gp mark 0}{(5.381,4.327)}
\gppoint{gp mark 0}{(5.381,3.784)}
\gppoint{gp mark 0}{(5.381,3.811)}
\gppoint{gp mark 0}{(5.381,3.811)}
\gppoint{gp mark 0}{(5.381,4.061)}
\gppoint{gp mark 0}{(5.381,3.934)}
\gppoint{gp mark 0}{(5.381,4.061)}
\gppoint{gp mark 0}{(5.381,3.784)}
\gppoint{gp mark 0}{(5.381,4.252)}
\gppoint{gp mark 0}{(5.381,3.837)}
\gppoint{gp mark 0}{(5.381,4.153)}
\gppoint{gp mark 0}{(5.381,4.000)}
\gppoint{gp mark 0}{(5.381,3.887)}
\gppoint{gp mark 0}{(5.381,4.021)}
\gppoint{gp mark 0}{(5.381,3.837)}
\gppoint{gp mark 0}{(5.381,4.000)}
\gppoint{gp mark 0}{(5.381,4.000)}
\gppoint{gp mark 0}{(5.381,4.000)}
\gppoint{gp mark 0}{(5.381,4.171)}
\gppoint{gp mark 0}{(5.381,4.118)}
\gppoint{gp mark 0}{(5.381,4.000)}
\gppoint{gp mark 0}{(5.381,3.811)}
\gppoint{gp mark 0}{(5.381,4.118)}
\gppoint{gp mark 0}{(5.381,4.252)}
\gppoint{gp mark 0}{(5.381,3.666)}
\gppoint{gp mark 0}{(5.381,3.784)}
\gppoint{gp mark 0}{(5.381,3.600)}
\gppoint{gp mark 0}{(5.381,3.979)}
\gppoint{gp mark 0}{(5.381,3.837)}
\gppoint{gp mark 0}{(5.381,4.041)}
\gppoint{gp mark 0}{(5.381,3.600)}
\gppoint{gp mark 0}{(5.381,4.061)}
\gppoint{gp mark 0}{(5.381,3.837)}
\gppoint{gp mark 0}{(5.381,4.041)}
\gppoint{gp mark 0}{(5.381,4.153)}
\gppoint{gp mark 0}{(5.381,4.041)}
\gppoint{gp mark 0}{(5.381,4.136)}
\gppoint{gp mark 0}{(5.381,4.080)}
\gppoint{gp mark 0}{(5.381,3.811)}
\gppoint{gp mark 0}{(5.381,3.565)}
\gppoint{gp mark 0}{(5.381,3.934)}
\gppoint{gp mark 0}{(5.381,3.911)}
\gppoint{gp mark 0}{(5.381,3.934)}
\gppoint{gp mark 0}{(5.381,3.911)}
\gppoint{gp mark 0}{(5.381,3.979)}
\gppoint{gp mark 0}{(5.381,3.784)}
\gppoint{gp mark 0}{(5.381,3.565)}
\gppoint{gp mark 0}{(5.381,3.957)}
\gppoint{gp mark 0}{(5.381,4.041)}
\gppoint{gp mark 0}{(5.381,3.957)}
\gppoint{gp mark 0}{(5.381,3.862)}
\gppoint{gp mark 0}{(5.381,3.979)}
\gppoint{gp mark 0}{(5.381,3.784)}
\gppoint{gp mark 0}{(5.381,4.118)}
\gppoint{gp mark 0}{(5.381,3.934)}
\gppoint{gp mark 0}{(5.381,3.666)}
\gppoint{gp mark 0}{(5.381,4.252)}
\gppoint{gp mark 0}{(5.381,3.030)}
\gppoint{gp mark 0}{(5.381,3.979)}
\gppoint{gp mark 0}{(5.381,3.784)}
\gppoint{gp mark 0}{(5.381,3.957)}
\gppoint{gp mark 0}{(5.381,4.000)}
\gppoint{gp mark 0}{(5.381,3.862)}
\gppoint{gp mark 0}{(5.381,3.267)}
\gppoint{gp mark 0}{(5.381,3.934)}
\gppoint{gp mark 0}{(5.381,4.000)}
\gppoint{gp mark 0}{(5.381,4.021)}
\gppoint{gp mark 0}{(5.381,3.408)}
\gppoint{gp mark 0}{(5.381,4.118)}
\gppoint{gp mark 0}{(5.381,3.600)}
\gppoint{gp mark 0}{(5.381,3.666)}
\gppoint{gp mark 0}{(5.381,3.697)}
\gppoint{gp mark 0}{(5.381,3.911)}
\gppoint{gp mark 0}{(5.381,3.837)}
\gppoint{gp mark 0}{(5.381,4.354)}
\gppoint{gp mark 0}{(5.381,4.252)}
\gppoint{gp mark 0}{(5.381,3.490)}
\gppoint{gp mark 0}{(5.381,3.862)}
\gppoint{gp mark 0}{(5.381,3.837)}
\gppoint{gp mark 0}{(5.381,4.136)}
\gppoint{gp mark 0}{(5.381,3.934)}
\gppoint{gp mark 0}{(5.381,4.000)}
\gppoint{gp mark 0}{(5.381,3.811)}
\gppoint{gp mark 0}{(5.381,3.727)}
\gppoint{gp mark 0}{(5.381,4.080)}
\gppoint{gp mark 0}{(5.381,3.979)}
\gppoint{gp mark 0}{(5.381,3.911)}
\gppoint{gp mark 0}{(5.381,3.862)}
\gppoint{gp mark 0}{(5.381,4.099)}
\gppoint{gp mark 0}{(5.381,3.934)}
\gppoint{gp mark 0}{(5.381,3.697)}
\gppoint{gp mark 0}{(5.381,4.354)}
\gppoint{gp mark 0}{(5.381,4.118)}
\gppoint{gp mark 0}{(5.381,3.862)}
\gppoint{gp mark 0}{(5.381,3.837)}
\gppoint{gp mark 0}{(5.381,4.237)}
\gppoint{gp mark 0}{(5.381,3.811)}
\gppoint{gp mark 0}{(5.381,3.756)}
\gppoint{gp mark 0}{(5.381,3.934)}
\gppoint{gp mark 0}{(5.381,3.837)}
\gppoint{gp mark 0}{(5.381,3.957)}
\gppoint{gp mark 0}{(5.381,3.811)}
\gppoint{gp mark 0}{(5.381,3.811)}
\gppoint{gp mark 0}{(5.381,3.317)}
\gppoint{gp mark 0}{(5.381,4.041)}
\gppoint{gp mark 0}{(5.381,4.000)}
\gppoint{gp mark 0}{(5.381,3.756)}
\gppoint{gp mark 0}{(5.381,3.697)}
\gppoint{gp mark 0}{(5.381,3.911)}
\gppoint{gp mark 0}{(5.381,3.214)}
\gppoint{gp mark 0}{(5.381,4.204)}
\gppoint{gp mark 0}{(5.381,3.634)}
\gppoint{gp mark 0}{(5.381,4.061)}
\gppoint{gp mark 0}{(5.381,3.811)}
\gppoint{gp mark 0}{(5.381,4.188)}
\gppoint{gp mark 0}{(5.381,3.811)}
\gppoint{gp mark 0}{(5.381,4.171)}
\gppoint{gp mark 0}{(5.381,3.934)}
\gppoint{gp mark 0}{(5.381,3.911)}
\gppoint{gp mark 0}{(5.381,4.408)}
\gppoint{gp mark 0}{(5.381,4.136)}
\gppoint{gp mark 0}{(5.381,3.727)}
\gppoint{gp mark 0}{(5.381,3.911)}
\gppoint{gp mark 0}{(5.381,3.887)}
\gppoint{gp mark 0}{(5.381,3.911)}
\gppoint{gp mark 0}{(5.381,3.911)}
\gppoint{gp mark 0}{(5.381,4.118)}
\gppoint{gp mark 0}{(5.381,3.957)}
\gppoint{gp mark 0}{(5.381,4.204)}
\gppoint{gp mark 0}{(5.381,3.811)}
\gppoint{gp mark 0}{(5.381,4.118)}
\gppoint{gp mark 0}{(5.381,3.979)}
\gppoint{gp mark 0}{(5.381,3.727)}
\gppoint{gp mark 0}{(5.381,3.911)}
\gppoint{gp mark 0}{(5.381,3.934)}
\gppoint{gp mark 0}{(5.381,4.469)}
\gppoint{gp mark 0}{(5.381,4.061)}
\gppoint{gp mark 0}{(5.381,3.887)}
\gppoint{gp mark 0}{(5.381,3.811)}
\gppoint{gp mark 0}{(5.381,4.171)}
\gppoint{gp mark 0}{(5.381,4.591)}
\gppoint{gp mark 0}{(5.381,3.837)}
\gppoint{gp mark 0}{(5.381,3.957)}
\gppoint{gp mark 0}{(5.381,3.697)}
\gppoint{gp mark 0}{(5.381,3.837)}
\gppoint{gp mark 0}{(5.381,3.862)}
\gppoint{gp mark 0}{(5.381,3.214)}
\gppoint{gp mark 0}{(5.381,4.171)}
\gppoint{gp mark 0}{(5.381,3.887)}
\gppoint{gp mark 0}{(5.381,4.000)}
\gppoint{gp mark 0}{(5.381,3.934)}
\gppoint{gp mark 0}{(5.418,3.529)}
\gppoint{gp mark 0}{(5.418,3.756)}
\gppoint{gp mark 0}{(5.418,3.887)}
\gppoint{gp mark 0}{(5.418,4.368)}
\gppoint{gp mark 0}{(5.418,4.395)}
\gppoint{gp mark 0}{(5.418,3.756)}
\gppoint{gp mark 0}{(5.418,4.395)}
\gppoint{gp mark 0}{(5.418,3.862)}
\gppoint{gp mark 0}{(5.418,3.784)}
\gppoint{gp mark 0}{(5.418,3.862)}
\gppoint{gp mark 0}{(5.418,3.862)}
\gppoint{gp mark 0}{(5.418,3.934)}
\gppoint{gp mark 0}{(5.418,3.727)}
\gppoint{gp mark 0}{(5.418,3.862)}
\gppoint{gp mark 0}{(5.418,3.887)}
\gppoint{gp mark 0}{(5.418,4.252)}
\gppoint{gp mark 0}{(5.418,4.237)}
\gppoint{gp mark 0}{(5.418,3.634)}
\gppoint{gp mark 0}{(5.418,3.784)}
\gppoint{gp mark 0}{(5.418,3.979)}
\gppoint{gp mark 0}{(5.418,3.565)}
\gppoint{gp mark 0}{(5.418,3.979)}
\gppoint{gp mark 0}{(5.418,3.934)}
\gppoint{gp mark 0}{(5.418,4.080)}
\gppoint{gp mark 0}{(5.418,4.327)}
\gppoint{gp mark 0}{(5.418,4.341)}
\gppoint{gp mark 0}{(5.418,4.080)}
\gppoint{gp mark 0}{(5.418,4.080)}
\gppoint{gp mark 0}{(5.418,4.021)}
\gppoint{gp mark 0}{(5.418,4.041)}
\gppoint{gp mark 0}{(5.418,3.600)}
\gppoint{gp mark 0}{(5.418,4.080)}
\gppoint{gp mark 0}{(5.418,4.000)}
\gppoint{gp mark 0}{(5.418,3.634)}
\gppoint{gp mark 0}{(5.418,4.153)}
\gppoint{gp mark 0}{(5.418,4.000)}
\gppoint{gp mark 0}{(5.418,4.171)}
\gppoint{gp mark 0}{(5.418,3.979)}
\gppoint{gp mark 0}{(5.418,4.775)}
\gppoint{gp mark 0}{(5.418,3.887)}
\gppoint{gp mark 0}{(5.418,3.837)}
\gppoint{gp mark 0}{(5.418,3.666)}
\gppoint{gp mark 0}{(5.418,3.934)}
\gppoint{gp mark 0}{(5.418,4.457)}
\gppoint{gp mark 0}{(5.418,4.099)}
\gppoint{gp mark 0}{(5.418,4.767)}
\gppoint{gp mark 0}{(5.418,4.433)}
\gppoint{gp mark 0}{(5.418,4.204)}
\gppoint{gp mark 0}{(5.418,4.312)}
\gppoint{gp mark 0}{(5.418,3.756)}
\gppoint{gp mark 0}{(5.418,3.957)}
\gppoint{gp mark 0}{(5.418,3.634)}
\gppoint{gp mark 0}{(5.418,4.298)}
\gppoint{gp mark 0}{(5.418,4.283)}
\gppoint{gp mark 0}{(5.418,4.021)}
\gppoint{gp mark 0}{(5.418,3.979)}
\gppoint{gp mark 0}{(5.418,3.811)}
\gppoint{gp mark 0}{(5.418,3.490)}
\gppoint{gp mark 0}{(5.418,4.021)}
\gppoint{gp mark 0}{(5.418,4.041)}
\gppoint{gp mark 0}{(5.418,3.634)}
\gppoint{gp mark 0}{(5.418,3.529)}
\gppoint{gp mark 0}{(5.418,3.756)}
\gppoint{gp mark 0}{(5.418,3.756)}
\gppoint{gp mark 0}{(5.418,4.445)}
\gppoint{gp mark 0}{(5.418,3.666)}
\gppoint{gp mark 0}{(5.418,4.041)}
\gppoint{gp mark 0}{(5.418,3.979)}
\gppoint{gp mark 0}{(5.418,4.041)}
\gppoint{gp mark 0}{(5.418,4.283)}
\gppoint{gp mark 0}{(5.418,3.979)}
\gppoint{gp mark 0}{(5.418,3.957)}
\gppoint{gp mark 0}{(5.418,3.697)}
\gppoint{gp mark 0}{(5.418,4.237)}
\gppoint{gp mark 0}{(5.418,3.837)}
\gppoint{gp mark 0}{(5.418,4.061)}
\gppoint{gp mark 0}{(5.418,4.204)}
\gppoint{gp mark 0}{(5.418,3.979)}
\gppoint{gp mark 0}{(5.418,3.666)}
\gppoint{gp mark 0}{(5.418,3.450)}
\gppoint{gp mark 0}{(5.418,4.204)}
\gppoint{gp mark 0}{(5.418,3.911)}
\gppoint{gp mark 0}{(5.418,3.450)}
\gppoint{gp mark 0}{(5.418,3.784)}
\gppoint{gp mark 0}{(5.418,4.481)}
\gppoint{gp mark 0}{(5.418,3.862)}
\gppoint{gp mark 0}{(5.418,3.934)}
\gppoint{gp mark 0}{(5.418,3.634)}
\gppoint{gp mark 0}{(5.418,3.811)}
\gppoint{gp mark 0}{(5.418,4.221)}
\gppoint{gp mark 0}{(5.418,3.811)}
\gppoint{gp mark 0}{(5.418,4.237)}
\gppoint{gp mark 0}{(5.418,3.862)}
\gppoint{gp mark 0}{(5.418,3.811)}
\gppoint{gp mark 0}{(5.418,3.979)}
\gppoint{gp mark 0}{(5.418,3.364)}
\gppoint{gp mark 0}{(5.418,3.364)}
\gppoint{gp mark 0}{(5.418,4.061)}
\gppoint{gp mark 0}{(5.418,3.911)}
\gppoint{gp mark 0}{(5.418,4.252)}
\gppoint{gp mark 0}{(5.418,4.327)}
\gppoint{gp mark 0}{(5.418,4.061)}
\gppoint{gp mark 0}{(5.418,3.214)}
\gppoint{gp mark 0}{(5.418,4.099)}
\gppoint{gp mark 0}{(5.418,4.268)}
\gppoint{gp mark 0}{(5.418,3.529)}
\gppoint{gp mark 0}{(5.418,3.862)}
\gppoint{gp mark 0}{(5.418,4.268)}
\gppoint{gp mark 0}{(5.418,4.000)}
\gppoint{gp mark 0}{(5.418,4.061)}
\gppoint{gp mark 0}{(5.418,4.000)}
\gppoint{gp mark 0}{(5.418,3.634)}
\gppoint{gp mark 0}{(5.418,3.697)}
\gppoint{gp mark 0}{(5.418,4.000)}
\gppoint{gp mark 0}{(5.418,3.862)}
\gppoint{gp mark 0}{(5.418,3.957)}
\gppoint{gp mark 0}{(5.418,3.811)}
\gppoint{gp mark 0}{(5.418,4.204)}
\gppoint{gp mark 0}{(5.418,3.979)}
\gppoint{gp mark 0}{(5.418,3.811)}
\gppoint{gp mark 0}{(5.418,3.529)}
\gppoint{gp mark 0}{(5.418,3.784)}
\gppoint{gp mark 0}{(5.418,3.727)}
\gppoint{gp mark 0}{(5.418,3.934)}
\gppoint{gp mark 0}{(5.418,4.204)}
\gppoint{gp mark 0}{(5.418,3.911)}
\gppoint{gp mark 0}{(5.418,3.887)}
\gppoint{gp mark 0}{(5.418,3.784)}
\gppoint{gp mark 0}{(5.418,3.756)}
\gppoint{gp mark 0}{(5.418,3.887)}
\gppoint{gp mark 0}{(5.418,3.862)}
\gppoint{gp mark 0}{(5.418,3.811)}
\gppoint{gp mark 0}{(5.418,3.727)}
\gppoint{gp mark 0}{(5.418,4.420)}
\gppoint{gp mark 0}{(5.418,3.565)}
\gppoint{gp mark 0}{(5.418,3.862)}
\gppoint{gp mark 0}{(5.418,4.099)}
\gppoint{gp mark 0}{(5.418,3.979)}
\gppoint{gp mark 0}{(5.418,3.837)}
\gppoint{gp mark 0}{(5.418,3.837)}
\gppoint{gp mark 0}{(5.418,4.061)}
\gppoint{gp mark 0}{(5.418,3.862)}
\gppoint{gp mark 0}{(5.418,3.697)}
\gppoint{gp mark 0}{(5.418,4.099)}
\gppoint{gp mark 0}{(5.454,3.756)}
\gppoint{gp mark 0}{(5.454,4.041)}
\gppoint{gp mark 0}{(5.454,4.136)}
\gppoint{gp mark 0}{(5.454,3.887)}
\gppoint{gp mark 0}{(5.454,3.811)}
\gppoint{gp mark 0}{(5.454,3.934)}
\gppoint{gp mark 0}{(5.454,4.118)}
\gppoint{gp mark 0}{(5.454,3.600)}
\gppoint{gp mark 0}{(5.454,3.934)}
\gppoint{gp mark 0}{(5.454,3.727)}
\gppoint{gp mark 0}{(5.454,3.784)}
\gppoint{gp mark 0}{(5.454,3.979)}
\gppoint{gp mark 0}{(5.454,3.934)}
\gppoint{gp mark 0}{(5.454,3.837)}
\gppoint{gp mark 0}{(5.454,3.934)}
\gppoint{gp mark 0}{(5.454,3.727)}
\gppoint{gp mark 0}{(5.454,4.099)}
\gppoint{gp mark 0}{(5.454,4.395)}
\gppoint{gp mark 0}{(5.454,4.099)}
\gppoint{gp mark 0}{(5.454,3.784)}
\gppoint{gp mark 0}{(5.454,4.041)}
\gppoint{gp mark 0}{(5.454,4.395)}
\gppoint{gp mark 0}{(5.454,3.911)}
\gppoint{gp mark 0}{(5.454,4.099)}
\gppoint{gp mark 0}{(5.454,3.837)}
\gppoint{gp mark 0}{(5.454,3.784)}
\gppoint{gp mark 0}{(5.454,4.099)}
\gppoint{gp mark 0}{(5.454,3.887)}
\gppoint{gp mark 0}{(5.454,3.957)}
\gppoint{gp mark 0}{(5.454,3.837)}
\gppoint{gp mark 0}{(5.454,4.118)}
\gppoint{gp mark 0}{(5.454,4.188)}
\gppoint{gp mark 0}{(5.454,3.811)}
\gppoint{gp mark 0}{(5.454,4.171)}
\gppoint{gp mark 0}{(5.454,3.957)}
\gppoint{gp mark 0}{(5.454,4.099)}
\gppoint{gp mark 0}{(5.454,4.445)}
\gppoint{gp mark 0}{(5.454,3.911)}
\gppoint{gp mark 0}{(5.454,4.041)}
\gppoint{gp mark 0}{(5.454,3.911)}
\gppoint{gp mark 0}{(5.454,4.252)}
\gppoint{gp mark 0}{(5.454,4.457)}
\gppoint{gp mark 0}{(5.454,3.811)}
\gppoint{gp mark 0}{(5.454,3.911)}
\gppoint{gp mark 0}{(5.454,4.118)}
\gppoint{gp mark 0}{(5.454,4.000)}
\gppoint{gp mark 0}{(5.454,4.136)}
\gppoint{gp mark 0}{(5.454,3.811)}
\gppoint{gp mark 0}{(5.454,3.887)}
\gppoint{gp mark 0}{(5.454,3.727)}
\gppoint{gp mark 0}{(5.454,4.021)}
\gppoint{gp mark 0}{(5.454,3.666)}
\gppoint{gp mark 0}{(5.454,4.298)}
\gppoint{gp mark 0}{(5.454,4.204)}
\gppoint{gp mark 0}{(5.454,4.153)}
\gppoint{gp mark 0}{(5.454,4.237)}
\gppoint{gp mark 0}{(5.454,4.481)}
\gppoint{gp mark 0}{(5.454,4.445)}
\gppoint{gp mark 0}{(5.454,3.565)}
\gppoint{gp mark 0}{(5.454,4.631)}
\gppoint{gp mark 0}{(5.454,3.979)}
\gppoint{gp mark 0}{(5.454,3.490)}
\gppoint{gp mark 0}{(5.454,3.957)}
\gppoint{gp mark 0}{(5.454,3.911)}
\gppoint{gp mark 0}{(5.454,4.021)}
\gppoint{gp mark 0}{(5.454,4.312)}
\gppoint{gp mark 0}{(5.454,4.061)}
\gppoint{gp mark 0}{(5.454,3.911)}
\gppoint{gp mark 0}{(5.454,4.549)}
\gppoint{gp mark 0}{(5.454,3.811)}
\gppoint{gp mark 0}{(5.454,3.756)}
\gppoint{gp mark 0}{(5.454,4.061)}
\gppoint{gp mark 0}{(5.454,3.911)}
\gppoint{gp mark 0}{(5.454,4.000)}
\gppoint{gp mark 0}{(5.454,3.911)}
\gppoint{gp mark 0}{(5.454,3.784)}
\gppoint{gp mark 0}{(5.454,3.957)}
\gppoint{gp mark 0}{(5.454,3.934)}
\gppoint{gp mark 0}{(5.454,4.000)}
\gppoint{gp mark 0}{(5.454,3.911)}
\gppoint{gp mark 0}{(5.454,4.118)}
\gppoint{gp mark 0}{(5.454,3.727)}
\gppoint{gp mark 0}{(5.454,4.000)}
\gppoint{gp mark 0}{(5.454,3.957)}
\gppoint{gp mark 0}{(5.454,4.268)}
\gppoint{gp mark 0}{(5.454,3.600)}
\gppoint{gp mark 0}{(5.454,4.153)}
\gppoint{gp mark 0}{(5.454,3.837)}
\gppoint{gp mark 0}{(5.454,3.934)}
\gppoint{gp mark 0}{(5.454,4.395)}
\gppoint{gp mark 0}{(5.454,4.000)}
\gppoint{gp mark 0}{(5.454,3.837)}
\gppoint{gp mark 0}{(5.454,4.298)}
\gppoint{gp mark 0}{(5.454,4.021)}
\gppoint{gp mark 0}{(5.454,4.395)}
\gppoint{gp mark 0}{(5.454,3.727)}
\gppoint{gp mark 0}{(5.454,3.666)}
\gppoint{gp mark 0}{(5.454,3.911)}
\gppoint{gp mark 0}{(5.454,4.283)}
\gppoint{gp mark 0}{(5.454,3.887)}
\gppoint{gp mark 0}{(5.454,4.118)}
\gppoint{gp mark 0}{(5.454,3.911)}
\gppoint{gp mark 0}{(5.454,4.298)}
\gppoint{gp mark 0}{(5.454,3.811)}
\gppoint{gp mark 0}{(5.454,4.000)}
\gppoint{gp mark 0}{(5.454,4.549)}
\gppoint{gp mark 0}{(5.454,3.727)}
\gppoint{gp mark 0}{(5.454,4.000)}
\gppoint{gp mark 0}{(5.454,3.811)}
\gppoint{gp mark 0}{(5.454,3.634)}
\gppoint{gp mark 0}{(5.454,4.000)}
\gppoint{gp mark 0}{(5.454,3.600)}
\gppoint{gp mark 0}{(5.454,3.934)}
\gppoint{gp mark 0}{(5.454,3.756)}
\gppoint{gp mark 0}{(5.454,3.697)}
\gppoint{gp mark 0}{(5.454,3.756)}
\gppoint{gp mark 0}{(5.454,3.837)}
\gppoint{gp mark 0}{(5.454,4.549)}
\gppoint{gp mark 0}{(5.454,3.862)}
\gppoint{gp mark 0}{(5.454,3.600)}
\gppoint{gp mark 0}{(5.454,4.041)}
\gppoint{gp mark 0}{(5.454,3.600)}
\gppoint{gp mark 0}{(5.454,3.157)}
\gppoint{gp mark 0}{(5.454,3.811)}
\gppoint{gp mark 0}{(5.454,3.727)}
\gppoint{gp mark 0}{(5.454,4.153)}
\gppoint{gp mark 0}{(5.454,4.099)}
\gppoint{gp mark 0}{(5.454,3.957)}
\gppoint{gp mark 0}{(5.454,4.061)}
\gppoint{gp mark 0}{(5.454,4.252)}
\gppoint{gp mark 0}{(5.454,3.756)}
\gppoint{gp mark 0}{(5.454,3.490)}
\gppoint{gp mark 0}{(5.454,3.887)}
\gppoint{gp mark 0}{(5.454,4.153)}
\gppoint{gp mark 0}{(5.454,3.862)}
\gppoint{gp mark 0}{(5.454,3.911)}
\gppoint{gp mark 0}{(5.454,4.136)}
\gppoint{gp mark 0}{(5.454,4.153)}
\gppoint{gp mark 0}{(5.454,4.204)}
\gppoint{gp mark 0}{(5.454,3.697)}
\gppoint{gp mark 0}{(5.454,4.252)}
\gppoint{gp mark 0}{(5.454,4.298)}
\gppoint{gp mark 0}{(5.454,4.298)}
\gppoint{gp mark 0}{(5.454,3.957)}
\gppoint{gp mark 0}{(5.454,3.979)}
\gppoint{gp mark 0}{(5.454,4.252)}
\gppoint{gp mark 0}{(5.454,3.862)}
\gppoint{gp mark 0}{(5.454,4.021)}
\gppoint{gp mark 0}{(5.454,4.298)}
\gppoint{gp mark 0}{(5.454,3.634)}
\gppoint{gp mark 0}{(5.454,4.021)}
\gppoint{gp mark 0}{(5.454,3.911)}
\gppoint{gp mark 0}{(5.454,3.911)}
\gppoint{gp mark 0}{(5.454,3.697)}
\gppoint{gp mark 0}{(5.454,3.837)}
\gppoint{gp mark 0}{(5.454,3.811)}
\gppoint{gp mark 0}{(5.454,4.061)}
\gppoint{gp mark 0}{(5.454,4.118)}
\gppoint{gp mark 0}{(5.454,4.000)}
\gppoint{gp mark 0}{(5.454,4.041)}
\gppoint{gp mark 0}{(5.454,4.171)}
\gppoint{gp mark 0}{(5.454,4.080)}
\gppoint{gp mark 0}{(5.454,3.979)}
\gppoint{gp mark 0}{(5.454,3.697)}
\gppoint{gp mark 0}{(5.454,3.957)}
\gppoint{gp mark 0}{(5.454,3.911)}
\gppoint{gp mark 0}{(5.454,4.549)}
\gppoint{gp mark 0}{(5.454,3.911)}
\gppoint{gp mark 0}{(5.454,2.958)}
\gppoint{gp mark 0}{(5.454,3.214)}
\gppoint{gp mark 0}{(5.454,4.221)}
\gppoint{gp mark 0}{(5.454,3.911)}
\gppoint{gp mark 0}{(5.454,4.395)}
\gppoint{gp mark 0}{(5.490,4.041)}
\gppoint{gp mark 0}{(5.490,3.934)}
\gppoint{gp mark 0}{(5.490,3.811)}
\gppoint{gp mark 0}{(5.490,3.887)}
\gppoint{gp mark 0}{(5.490,3.666)}
\gppoint{gp mark 0}{(5.490,4.252)}
\gppoint{gp mark 0}{(5.490,4.204)}
\gppoint{gp mark 0}{(5.490,4.493)}
\gppoint{gp mark 0}{(5.490,3.837)}
\gppoint{gp mark 0}{(5.490,4.061)}
\gppoint{gp mark 0}{(5.490,4.221)}
\gppoint{gp mark 0}{(5.490,4.021)}
\gppoint{gp mark 0}{(5.490,4.099)}
\gppoint{gp mark 0}{(5.490,3.565)}
\gppoint{gp mark 0}{(5.490,3.666)}
\gppoint{gp mark 0}{(5.490,3.911)}
\gppoint{gp mark 0}{(5.490,4.000)}
\gppoint{gp mark 0}{(5.490,4.041)}
\gppoint{gp mark 0}{(5.490,4.080)}
\gppoint{gp mark 0}{(5.490,3.887)}
\gppoint{gp mark 0}{(5.490,3.979)}
\gppoint{gp mark 0}{(5.490,4.221)}
\gppoint{gp mark 0}{(5.490,3.784)}
\gppoint{gp mark 0}{(5.490,4.381)}
\gppoint{gp mark 0}{(5.490,4.041)}
\gppoint{gp mark 0}{(5.490,3.957)}
\gppoint{gp mark 0}{(5.490,4.221)}
\gppoint{gp mark 0}{(5.490,3.490)}
\gppoint{gp mark 0}{(5.490,3.837)}
\gppoint{gp mark 0}{(5.490,4.000)}
\gppoint{gp mark 0}{(5.490,4.000)}
\gppoint{gp mark 0}{(5.490,3.862)}
\gppoint{gp mark 0}{(5.490,4.171)}
\gppoint{gp mark 0}{(5.490,3.784)}
\gppoint{gp mark 0}{(5.490,4.341)}
\gppoint{gp mark 0}{(5.490,3.784)}
\gppoint{gp mark 0}{(5.490,3.697)}
\gppoint{gp mark 0}{(5.490,3.957)}
\gppoint{gp mark 0}{(5.490,4.815)}
\gppoint{gp mark 0}{(5.490,4.153)}
\gppoint{gp mark 0}{(5.490,3.450)}
\gppoint{gp mark 0}{(5.490,3.887)}
\gppoint{gp mark 0}{(5.490,3.600)}
\gppoint{gp mark 0}{(5.490,4.061)}
\gppoint{gp mark 0}{(5.490,4.268)}
\gppoint{gp mark 0}{(5.490,4.021)}
\gppoint{gp mark 0}{(5.490,3.666)}
\gppoint{gp mark 0}{(5.490,4.021)}
\gppoint{gp mark 0}{(5.490,4.298)}
\gppoint{gp mark 0}{(5.490,4.408)}
\gppoint{gp mark 0}{(5.490,4.000)}
\gppoint{gp mark 0}{(5.490,4.000)}
\gppoint{gp mark 0}{(5.490,3.957)}
\gppoint{gp mark 0}{(5.490,4.204)}
\gppoint{gp mark 0}{(5.490,4.457)}
\gppoint{gp mark 0}{(5.490,3.862)}
\gppoint{gp mark 0}{(5.490,3.364)}
\gppoint{gp mark 0}{(5.490,3.862)}
\gppoint{gp mark 0}{(5.490,3.837)}
\gppoint{gp mark 0}{(5.490,4.171)}
\gppoint{gp mark 0}{(5.490,3.697)}
\gppoint{gp mark 0}{(5.490,4.469)}
\gppoint{gp mark 0}{(5.490,4.118)}
\gppoint{gp mark 0}{(5.490,4.268)}
\gppoint{gp mark 0}{(5.490,3.490)}
\gppoint{gp mark 0}{(5.490,3.727)}
\gppoint{gp mark 0}{(5.490,3.837)}
\gppoint{gp mark 0}{(5.490,4.395)}
\gppoint{gp mark 0}{(5.490,3.979)}
\gppoint{gp mark 0}{(5.490,3.934)}
\gppoint{gp mark 0}{(5.490,3.837)}
\gppoint{gp mark 0}{(5.490,3.862)}
\gppoint{gp mark 0}{(5.490,4.080)}
\gppoint{gp mark 0}{(5.490,3.911)}
\gppoint{gp mark 0}{(5.490,4.433)}
\gppoint{gp mark 0}{(5.490,3.784)}
\gppoint{gp mark 0}{(5.490,3.979)}
\gppoint{gp mark 0}{(5.490,4.118)}
\gppoint{gp mark 0}{(5.490,4.041)}
\gppoint{gp mark 0}{(5.490,4.395)}
\gppoint{gp mark 0}{(5.490,3.979)}
\gppoint{gp mark 0}{(5.490,4.312)}
\gppoint{gp mark 0}{(5.490,4.327)}
\gppoint{gp mark 0}{(5.490,4.252)}
\gppoint{gp mark 0}{(5.490,4.651)}
\gppoint{gp mark 0}{(5.490,4.153)}
\gppoint{gp mark 0}{(5.490,3.911)}
\gppoint{gp mark 0}{(5.490,3.979)}
\gppoint{gp mark 0}{(5.490,3.979)}
\gppoint{gp mark 0}{(5.490,3.911)}
\gppoint{gp mark 0}{(5.490,3.317)}
\gppoint{gp mark 0}{(5.490,4.000)}
\gppoint{gp mark 0}{(5.490,4.041)}
\gppoint{gp mark 0}{(5.490,3.784)}
\gppoint{gp mark 0}{(5.490,4.118)}
\gppoint{gp mark 0}{(5.490,4.153)}
\gppoint{gp mark 0}{(5.490,3.727)}
\gppoint{gp mark 0}{(5.490,4.041)}
\gppoint{gp mark 0}{(5.490,4.118)}
\gppoint{gp mark 0}{(5.490,3.811)}
\gppoint{gp mark 0}{(5.490,4.408)}
\gppoint{gp mark 0}{(5.490,3.957)}
\gppoint{gp mark 0}{(5.490,3.565)}
\gppoint{gp mark 0}{(5.490,4.221)}
\gppoint{gp mark 0}{(5.490,3.811)}
\gppoint{gp mark 0}{(5.490,3.756)}
\gppoint{gp mark 0}{(5.490,4.171)}
\gppoint{gp mark 0}{(5.490,4.099)}
\gppoint{gp mark 0}{(5.490,3.979)}
\gppoint{gp mark 0}{(5.490,3.979)}
\gppoint{gp mark 0}{(5.490,3.887)}
\gppoint{gp mark 0}{(5.490,4.080)}
\gppoint{gp mark 0}{(5.490,4.408)}
\gppoint{gp mark 0}{(5.490,4.354)}
\gppoint{gp mark 0}{(5.490,4.354)}
\gppoint{gp mark 0}{(5.490,3.565)}
\gppoint{gp mark 0}{(5.490,4.481)}
\gppoint{gp mark 0}{(5.490,3.727)}
\gppoint{gp mark 0}{(5.490,4.000)}
\gppoint{gp mark 0}{(5.490,3.837)}
\gppoint{gp mark 0}{(5.490,4.000)}
\gppoint{gp mark 0}{(5.490,4.204)}
\gppoint{gp mark 0}{(5.490,4.099)}
\gppoint{gp mark 0}{(5.490,3.957)}
\gppoint{gp mark 0}{(5.490,4.153)}
\gppoint{gp mark 0}{(5.490,4.549)}
\gppoint{gp mark 0}{(5.490,4.153)}
\gppoint{gp mark 0}{(5.490,3.837)}
\gppoint{gp mark 0}{(5.490,4.061)}
\gppoint{gp mark 0}{(5.490,4.252)}
\gppoint{gp mark 0}{(5.490,3.450)}
\gppoint{gp mark 0}{(5.490,4.118)}
\gppoint{gp mark 0}{(5.490,3.837)}
\gppoint{gp mark 0}{(5.524,3.934)}
\gppoint{gp mark 0}{(5.524,3.727)}
\gppoint{gp mark 0}{(5.524,3.934)}
\gppoint{gp mark 0}{(5.524,3.934)}
\gppoint{gp mark 0}{(5.524,3.934)}
\gppoint{gp mark 0}{(5.524,4.061)}
\gppoint{gp mark 0}{(5.524,3.784)}
\gppoint{gp mark 0}{(5.524,3.979)}
\gppoint{gp mark 0}{(5.524,4.041)}
\gppoint{gp mark 0}{(5.524,3.837)}
\gppoint{gp mark 0}{(5.524,3.837)}
\gppoint{gp mark 0}{(5.524,3.837)}
\gppoint{gp mark 0}{(5.524,4.171)}
\gppoint{gp mark 0}{(5.524,3.784)}
\gppoint{gp mark 0}{(5.524,4.080)}
\gppoint{gp mark 0}{(5.524,4.283)}
\gppoint{gp mark 0}{(5.524,4.171)}
\gppoint{gp mark 0}{(5.524,4.252)}
\gppoint{gp mark 0}{(5.524,3.979)}
\gppoint{gp mark 0}{(5.524,4.118)}
\gppoint{gp mark 0}{(5.524,4.188)}
\gppoint{gp mark 0}{(5.524,4.457)}
\gppoint{gp mark 0}{(5.524,4.136)}
\gppoint{gp mark 0}{(5.524,4.381)}
\gppoint{gp mark 0}{(5.524,3.979)}
\gppoint{gp mark 0}{(5.524,3.811)}
\gppoint{gp mark 0}{(5.524,3.957)}
\gppoint{gp mark 0}{(5.524,4.099)}
\gppoint{gp mark 0}{(5.524,4.469)}
\gppoint{gp mark 0}{(5.524,3.862)}
\gppoint{gp mark 0}{(5.524,3.862)}
\gppoint{gp mark 0}{(5.524,4.021)}
\gppoint{gp mark 0}{(5.524,3.862)}
\gppoint{gp mark 0}{(5.524,4.298)}
\gppoint{gp mark 0}{(5.524,3.934)}
\gppoint{gp mark 0}{(5.524,4.354)}
\gppoint{gp mark 0}{(5.524,4.457)}
\gppoint{gp mark 0}{(5.524,4.433)}
\gppoint{gp mark 0}{(5.524,4.000)}
\gppoint{gp mark 0}{(5.524,3.837)}
\gppoint{gp mark 0}{(5.524,4.283)}
\gppoint{gp mark 0}{(5.524,4.341)}
\gppoint{gp mark 0}{(5.524,3.030)}
\gppoint{gp mark 0}{(5.524,4.136)}
\gppoint{gp mark 0}{(5.524,3.934)}
\gppoint{gp mark 0}{(5.524,3.756)}
\gppoint{gp mark 0}{(5.524,4.188)}
\gppoint{gp mark 0}{(5.524,4.099)}
\gppoint{gp mark 0}{(5.524,3.887)}
\gppoint{gp mark 0}{(5.524,4.153)}
\gppoint{gp mark 0}{(5.524,3.934)}
\gppoint{gp mark 0}{(5.524,4.581)}
\gppoint{gp mark 0}{(5.524,3.756)}
\gppoint{gp mark 0}{(5.524,3.957)}
\gppoint{gp mark 0}{(5.524,3.887)}
\gppoint{gp mark 0}{(5.524,4.021)}
\gppoint{gp mark 0}{(5.524,3.756)}
\gppoint{gp mark 0}{(5.524,4.354)}
\gppoint{gp mark 0}{(5.524,3.957)}
\gppoint{gp mark 0}{(5.524,3.862)}
\gppoint{gp mark 0}{(5.524,4.298)}
\gppoint{gp mark 0}{(5.524,3.565)}
\gppoint{gp mark 0}{(5.524,3.784)}
\gppoint{gp mark 0}{(5.524,3.934)}
\gppoint{gp mark 0}{(5.524,4.298)}
\gppoint{gp mark 0}{(5.524,4.080)}
\gppoint{gp mark 0}{(5.524,4.791)}
\gppoint{gp mark 0}{(5.524,3.756)}
\gppoint{gp mark 0}{(5.524,4.171)}
\gppoint{gp mark 0}{(5.524,3.979)}
\gppoint{gp mark 0}{(5.524,4.327)}
\gppoint{gp mark 0}{(5.524,4.118)}
\gppoint{gp mark 0}{(5.524,4.061)}
\gppoint{gp mark 0}{(5.524,3.784)}
\gppoint{gp mark 0}{(5.524,4.171)}
\gppoint{gp mark 0}{(5.524,4.237)}
\gppoint{gp mark 0}{(5.524,4.041)}
\gppoint{gp mark 0}{(5.524,3.862)}
\gppoint{gp mark 0}{(5.524,4.420)}
\gppoint{gp mark 0}{(5.524,4.061)}
\gppoint{gp mark 0}{(5.524,3.957)}
\gppoint{gp mark 0}{(5.524,3.957)}
\gppoint{gp mark 0}{(5.524,3.979)}
\gppoint{gp mark 0}{(5.524,3.837)}
\gppoint{gp mark 0}{(5.524,4.188)}
\gppoint{gp mark 0}{(5.524,3.634)}
\gppoint{gp mark 0}{(5.524,4.153)}
\gppoint{gp mark 0}{(5.524,4.118)}
\gppoint{gp mark 0}{(5.524,3.979)}
\gppoint{gp mark 0}{(5.524,4.041)}
\gppoint{gp mark 0}{(5.524,3.979)}
\gppoint{gp mark 0}{(5.524,3.756)}
\gppoint{gp mark 0}{(5.524,4.099)}
\gppoint{gp mark 0}{(5.524,4.221)}
\gppoint{gp mark 0}{(5.524,3.634)}
\gppoint{gp mark 0}{(5.524,3.811)}
\gppoint{gp mark 0}{(5.524,4.041)}
\gppoint{gp mark 0}{(5.524,4.061)}
\gppoint{gp mark 0}{(5.524,3.862)}
\gppoint{gp mark 0}{(5.524,3.837)}
\gppoint{gp mark 0}{(5.524,3.697)}
\gppoint{gp mark 0}{(5.524,3.837)}
\gppoint{gp mark 0}{(5.524,3.784)}
\gppoint{gp mark 0}{(5.524,4.061)}
\gppoint{gp mark 0}{(5.524,4.080)}
\gppoint{gp mark 0}{(5.524,4.516)}
\gppoint{gp mark 0}{(5.524,3.811)}
\gppoint{gp mark 0}{(5.524,3.957)}
\gppoint{gp mark 0}{(5.524,4.041)}
\gppoint{gp mark 0}{(5.524,3.934)}
\gppoint{gp mark 0}{(5.524,3.887)}
\gppoint{gp mark 0}{(5.524,4.000)}
\gppoint{gp mark 0}{(5.524,3.957)}
\gppoint{gp mark 0}{(5.524,3.911)}
\gppoint{gp mark 0}{(5.524,3.408)}
\gppoint{gp mark 0}{(5.524,4.527)}
\gppoint{gp mark 0}{(5.524,3.837)}
\gppoint{gp mark 0}{(5.524,3.214)}
\gppoint{gp mark 0}{(5.524,3.634)}
\gppoint{gp mark 0}{(5.524,3.784)}
\gppoint{gp mark 0}{(5.524,3.887)}
\gppoint{gp mark 0}{(5.524,3.811)}
\gppoint{gp mark 0}{(5.524,3.756)}
\gppoint{gp mark 0}{(5.524,3.784)}
\gppoint{gp mark 0}{(5.524,3.784)}
\gppoint{gp mark 0}{(5.524,3.784)}
\gppoint{gp mark 0}{(5.524,4.000)}
\gppoint{gp mark 0}{(5.524,4.099)}
\gppoint{gp mark 0}{(5.524,4.000)}
\gppoint{gp mark 0}{(5.524,3.811)}
\gppoint{gp mark 0}{(5.524,4.136)}
\gppoint{gp mark 0}{(5.524,3.837)}
\gppoint{gp mark 0}{(5.524,3.727)}
\gppoint{gp mark 0}{(5.524,3.697)}
\gppoint{gp mark 0}{(5.524,4.237)}
\gppoint{gp mark 0}{(5.524,4.237)}
\gppoint{gp mark 0}{(5.524,4.000)}
\gppoint{gp mark 0}{(5.524,4.021)}
\gppoint{gp mark 0}{(5.524,4.341)}
\gppoint{gp mark 0}{(5.524,3.837)}
\gppoint{gp mark 0}{(5.557,3.811)}
\gppoint{gp mark 0}{(5.557,4.481)}
\gppoint{gp mark 0}{(5.557,4.516)}
\gppoint{gp mark 0}{(5.557,3.957)}
\gppoint{gp mark 0}{(5.557,3.957)}
\gppoint{gp mark 0}{(5.557,3.811)}
\gppoint{gp mark 0}{(5.557,4.381)}
\gppoint{gp mark 0}{(5.557,3.811)}
\gppoint{gp mark 0}{(5.557,4.118)}
\gppoint{gp mark 0}{(5.557,4.758)}
\gppoint{gp mark 0}{(5.557,4.395)}
\gppoint{gp mark 0}{(5.557,3.862)}
\gppoint{gp mark 0}{(5.557,3.862)}
\gppoint{gp mark 0}{(5.557,3.837)}
\gppoint{gp mark 0}{(5.557,4.118)}
\gppoint{gp mark 0}{(5.557,4.080)}
\gppoint{gp mark 0}{(5.557,4.153)}
\gppoint{gp mark 0}{(5.557,4.136)}
\gppoint{gp mark 0}{(5.557,4.268)}
\gppoint{gp mark 0}{(5.557,4.000)}
\gppoint{gp mark 0}{(5.557,4.327)}
\gppoint{gp mark 0}{(5.557,4.041)}
\gppoint{gp mark 0}{(5.557,3.979)}
\gppoint{gp mark 0}{(5.557,4.204)}
\gppoint{gp mark 0}{(5.557,4.420)}
\gppoint{gp mark 0}{(5.557,4.420)}
\gppoint{gp mark 0}{(5.557,3.957)}
\gppoint{gp mark 0}{(5.557,4.395)}
\gppoint{gp mark 0}{(5.557,3.756)}
\gppoint{gp mark 0}{(5.557,3.364)}
\gppoint{gp mark 0}{(5.557,4.136)}
\gppoint{gp mark 0}{(5.557,4.268)}
\gppoint{gp mark 0}{(5.557,3.887)}
\gppoint{gp mark 0}{(5.557,4.041)}
\gppoint{gp mark 0}{(5.557,3.756)}
\gppoint{gp mark 0}{(5.557,3.887)}
\gppoint{gp mark 0}{(5.557,4.136)}
\gppoint{gp mark 0}{(5.557,4.118)}
\gppoint{gp mark 0}{(5.557,3.811)}
\gppoint{gp mark 0}{(5.557,3.957)}
\gppoint{gp mark 0}{(5.557,4.221)}
\gppoint{gp mark 0}{(5.557,3.934)}
\gppoint{gp mark 0}{(5.557,4.188)}
\gppoint{gp mark 0}{(5.557,3.979)}
\gppoint{gp mark 0}{(5.557,3.957)}
\gppoint{gp mark 0}{(5.557,3.911)}
\gppoint{gp mark 0}{(5.557,3.862)}
\gppoint{gp mark 0}{(5.557,3.979)}
\gppoint{gp mark 0}{(5.557,4.099)}
\gppoint{gp mark 0}{(5.557,3.565)}
\gppoint{gp mark 0}{(5.557,3.811)}
\gppoint{gp mark 0}{(5.557,4.099)}
\gppoint{gp mark 0}{(5.557,4.493)}
\gppoint{gp mark 0}{(5.557,3.979)}
\gppoint{gp mark 0}{(5.557,3.756)}
\gppoint{gp mark 0}{(5.557,3.697)}
\gppoint{gp mark 0}{(5.557,4.268)}
\gppoint{gp mark 0}{(5.557,4.099)}
\gppoint{gp mark 0}{(5.557,4.204)}
\gppoint{gp mark 0}{(5.557,4.118)}
\gppoint{gp mark 0}{(5.557,3.887)}
\gppoint{gp mark 0}{(5.557,4.118)}
\gppoint{gp mark 0}{(5.557,3.862)}
\gppoint{gp mark 0}{(5.557,4.000)}
\gppoint{gp mark 0}{(5.557,3.887)}
\gppoint{gp mark 0}{(5.557,4.021)}
\gppoint{gp mark 0}{(5.557,3.887)}
\gppoint{gp mark 0}{(5.557,4.136)}
\gppoint{gp mark 0}{(5.557,3.862)}
\gppoint{gp mark 0}{(5.557,4.445)}
\gppoint{gp mark 0}{(5.557,4.118)}
\gppoint{gp mark 0}{(5.557,3.600)}
\gppoint{gp mark 0}{(5.557,3.727)}
\gppoint{gp mark 0}{(5.557,3.887)}
\gppoint{gp mark 0}{(5.557,4.327)}
\gppoint{gp mark 0}{(5.557,4.000)}
\gppoint{gp mark 0}{(5.557,4.021)}
\gppoint{gp mark 0}{(5.557,3.957)}
\gppoint{gp mark 0}{(5.557,4.861)}
\gppoint{gp mark 0}{(5.557,4.153)}
\gppoint{gp mark 0}{(5.557,3.934)}
\gppoint{gp mark 0}{(5.557,4.099)}
\gppoint{gp mark 0}{(5.557,3.756)}
\gppoint{gp mark 0}{(5.557,3.837)}
\gppoint{gp mark 0}{(5.557,4.204)}
\gppoint{gp mark 0}{(5.557,4.061)}
\gppoint{gp mark 0}{(5.557,3.957)}
\gppoint{gp mark 0}{(5.557,3.811)}
\gppoint{gp mark 0}{(5.557,3.811)}
\gppoint{gp mark 0}{(5.557,4.368)}
\gppoint{gp mark 0}{(5.557,4.312)}
\gppoint{gp mark 0}{(5.557,4.204)}
\gppoint{gp mark 0}{(5.557,4.061)}
\gppoint{gp mark 0}{(5.557,3.957)}
\gppoint{gp mark 0}{(5.557,3.979)}
\gppoint{gp mark 0}{(5.557,3.887)}
\gppoint{gp mark 0}{(5.557,3.811)}
\gppoint{gp mark 0}{(5.557,4.204)}
\gppoint{gp mark 0}{(5.557,4.368)}
\gppoint{gp mark 0}{(5.557,4.408)}
\gppoint{gp mark 0}{(5.557,3.634)}
\gppoint{gp mark 0}{(5.557,4.136)}
\gppoint{gp mark 0}{(5.557,3.756)}
\gppoint{gp mark 0}{(5.557,3.600)}
\gppoint{gp mark 0}{(5.557,4.268)}
\gppoint{gp mark 0}{(5.557,3.811)}
\gppoint{gp mark 0}{(5.557,4.080)}
\gppoint{gp mark 0}{(5.557,4.221)}
\gppoint{gp mark 0}{(5.557,3.887)}
\gppoint{gp mark 0}{(5.557,3.727)}
\gppoint{gp mark 0}{(5.557,4.080)}
\gppoint{gp mark 0}{(5.557,4.021)}
\gppoint{gp mark 0}{(5.557,3.565)}
\gppoint{gp mark 0}{(5.557,3.934)}
\gppoint{gp mark 0}{(5.557,4.000)}
\gppoint{gp mark 0}{(5.557,4.457)}
\gppoint{gp mark 0}{(5.557,4.041)}
\gppoint{gp mark 0}{(5.557,3.811)}
\gppoint{gp mark 0}{(5.557,4.118)}
\gppoint{gp mark 0}{(5.557,4.118)}
\gppoint{gp mark 0}{(5.557,3.811)}
\gppoint{gp mark 0}{(5.557,3.957)}
\gppoint{gp mark 0}{(5.557,4.021)}
\gppoint{gp mark 0}{(5.557,4.118)}
\gppoint{gp mark 0}{(5.557,4.204)}
\gppoint{gp mark 0}{(5.557,4.061)}
\gppoint{gp mark 0}{(5.557,4.204)}
\gppoint{gp mark 0}{(5.557,3.862)}
\gppoint{gp mark 0}{(5.557,4.021)}
\gppoint{gp mark 0}{(5.590,4.099)}
\gppoint{gp mark 0}{(5.590,3.934)}
\gppoint{gp mark 0}{(5.590,3.784)}
\gppoint{gp mark 0}{(5.590,3.934)}
\gppoint{gp mark 0}{(5.590,4.099)}
\gppoint{gp mark 0}{(5.590,3.934)}
\gppoint{gp mark 0}{(5.590,4.061)}
\gppoint{gp mark 0}{(5.590,3.887)}
\gppoint{gp mark 0}{(5.590,3.811)}
\gppoint{gp mark 0}{(5.590,4.041)}
\gppoint{gp mark 0}{(5.590,3.979)}
\gppoint{gp mark 0}{(5.590,3.957)}
\gppoint{gp mark 0}{(5.590,3.408)}
\gppoint{gp mark 0}{(5.590,3.979)}
\gppoint{gp mark 0}{(5.590,3.957)}
\gppoint{gp mark 0}{(5.590,3.887)}
\gppoint{gp mark 0}{(5.590,4.204)}
\gppoint{gp mark 0}{(5.590,4.237)}
\gppoint{gp mark 0}{(5.590,4.000)}
\gppoint{gp mark 0}{(5.590,3.811)}
\gppoint{gp mark 0}{(5.590,3.887)}
\gppoint{gp mark 0}{(5.590,3.934)}
\gppoint{gp mark 0}{(5.590,4.621)}
\gppoint{gp mark 0}{(5.590,4.621)}
\gppoint{gp mark 0}{(5.590,4.099)}
\gppoint{gp mark 0}{(5.590,4.118)}
\gppoint{gp mark 0}{(5.590,3.934)}
\gppoint{gp mark 0}{(5.590,3.727)}
\gppoint{gp mark 0}{(5.590,4.061)}
\gppoint{gp mark 0}{(5.590,4.298)}
\gppoint{gp mark 0}{(5.590,4.298)}
\gppoint{gp mark 0}{(5.590,3.979)}
\gppoint{gp mark 0}{(5.590,4.283)}
\gppoint{gp mark 0}{(5.590,4.171)}
\gppoint{gp mark 0}{(5.590,4.041)}
\gppoint{gp mark 0}{(5.590,4.000)}
\gppoint{gp mark 0}{(5.590,3.979)}
\gppoint{gp mark 0}{(5.590,3.727)}
\gppoint{gp mark 0}{(5.590,4.171)}
\gppoint{gp mark 0}{(5.590,4.080)}
\gppoint{gp mark 0}{(5.590,3.862)}
\gppoint{gp mark 0}{(5.590,4.171)}
\gppoint{gp mark 0}{(5.590,4.153)}
\gppoint{gp mark 0}{(5.590,3.911)}
\gppoint{gp mark 0}{(5.590,4.433)}
\gppoint{gp mark 0}{(5.590,3.811)}
\gppoint{gp mark 0}{(5.590,4.268)}
\gppoint{gp mark 0}{(5.590,4.000)}
\gppoint{gp mark 0}{(5.590,3.979)}
\gppoint{gp mark 0}{(5.590,4.021)}
\gppoint{gp mark 0}{(5.590,3.811)}
\gppoint{gp mark 0}{(5.590,4.099)}
\gppoint{gp mark 0}{(5.590,4.237)}
\gppoint{gp mark 0}{(5.590,3.979)}
\gppoint{gp mark 0}{(5.590,3.979)}
\gppoint{gp mark 0}{(5.590,3.911)}
\gppoint{gp mark 0}{(5.590,3.957)}
\gppoint{gp mark 0}{(5.590,4.457)}
\gppoint{gp mark 0}{(5.590,3.911)}
\gppoint{gp mark 0}{(5.590,3.934)}
\gppoint{gp mark 0}{(5.590,4.204)}
\gppoint{gp mark 0}{(5.590,3.979)}
\gppoint{gp mark 0}{(5.590,3.934)}
\gppoint{gp mark 0}{(5.590,3.934)}
\gppoint{gp mark 0}{(5.590,3.979)}
\gppoint{gp mark 0}{(5.590,4.368)}
\gppoint{gp mark 0}{(5.590,4.118)}
\gppoint{gp mark 0}{(5.590,4.651)}
\gppoint{gp mark 0}{(5.590,4.153)}
\gppoint{gp mark 0}{(5.590,4.000)}
\gppoint{gp mark 0}{(5.590,3.837)}
\gppoint{gp mark 0}{(5.590,3.979)}
\gppoint{gp mark 0}{(5.590,3.450)}
\gppoint{gp mark 0}{(5.590,4.061)}
\gppoint{gp mark 0}{(5.590,4.799)}
\gppoint{gp mark 0}{(5.590,4.021)}
\gppoint{gp mark 0}{(5.590,4.080)}
\gppoint{gp mark 0}{(5.590,3.934)}
\gppoint{gp mark 0}{(5.590,4.445)}
\gppoint{gp mark 0}{(5.590,3.911)}
\gppoint{gp mark 0}{(5.590,4.000)}
\gppoint{gp mark 0}{(5.590,3.911)}
\gppoint{gp mark 0}{(5.590,4.457)}
\gppoint{gp mark 0}{(5.590,3.727)}
\gppoint{gp mark 0}{(5.590,4.395)}
\gppoint{gp mark 0}{(5.590,4.341)}
\gppoint{gp mark 0}{(5.590,4.469)}
\gppoint{gp mark 0}{(5.590,4.341)}
\gppoint{gp mark 0}{(5.590,4.312)}
\gppoint{gp mark 0}{(5.590,4.268)}
\gppoint{gp mark 0}{(5.590,4.099)}
\gppoint{gp mark 0}{(5.590,4.171)}
\gppoint{gp mark 0}{(5.590,3.784)}
\gppoint{gp mark 0}{(5.590,4.136)}
\gppoint{gp mark 0}{(5.590,3.979)}
\gppoint{gp mark 0}{(5.590,4.204)}
\gppoint{gp mark 0}{(5.590,3.979)}
\gppoint{gp mark 0}{(5.590,4.099)}
\gppoint{gp mark 0}{(5.590,3.862)}
\gppoint{gp mark 0}{(5.590,3.979)}
\gppoint{gp mark 0}{(5.590,4.021)}
\gppoint{gp mark 0}{(5.590,4.061)}
\gppoint{gp mark 0}{(5.590,4.000)}
\gppoint{gp mark 0}{(5.590,4.099)}
\gppoint{gp mark 0}{(5.590,4.080)}
\gppoint{gp mark 0}{(5.590,4.061)}
\gppoint{gp mark 0}{(5.590,3.934)}
\gppoint{gp mark 0}{(5.590,4.099)}
\gppoint{gp mark 0}{(5.590,3.784)}
\gppoint{gp mark 0}{(5.590,3.934)}
\gppoint{gp mark 0}{(5.590,3.784)}
\gppoint{gp mark 0}{(5.590,4.221)}
\gppoint{gp mark 0}{(5.590,4.099)}
\gppoint{gp mark 0}{(5.590,4.408)}
\gppoint{gp mark 0}{(5.590,3.666)}
\gppoint{gp mark 0}{(5.590,4.118)}
\gppoint{gp mark 0}{(5.590,3.784)}
\gppoint{gp mark 0}{(5.590,4.099)}
\gppoint{gp mark 0}{(5.590,4.061)}
\gppoint{gp mark 0}{(5.590,3.979)}
\gppoint{gp mark 0}{(5.590,3.887)}
\gppoint{gp mark 0}{(5.590,4.171)}
\gppoint{gp mark 0}{(5.590,3.934)}
\gppoint{gp mark 0}{(5.590,4.099)}
\gppoint{gp mark 0}{(5.590,4.341)}
\gppoint{gp mark 0}{(5.590,4.061)}
\gppoint{gp mark 0}{(5.590,3.862)}
\gppoint{gp mark 0}{(5.590,4.041)}
\gppoint{gp mark 0}{(5.621,3.600)}
\gppoint{gp mark 0}{(5.621,4.327)}
\gppoint{gp mark 0}{(5.621,4.171)}
\gppoint{gp mark 0}{(5.621,4.041)}
\gppoint{gp mark 0}{(5.621,4.298)}
\gppoint{gp mark 0}{(5.621,4.000)}
\gppoint{gp mark 0}{(5.621,3.634)}
\gppoint{gp mark 0}{(5.621,3.811)}
\gppoint{gp mark 0}{(5.621,4.171)}
\gppoint{gp mark 0}{(5.621,4.395)}
\gppoint{gp mark 0}{(5.621,4.099)}
\gppoint{gp mark 0}{(5.621,4.283)}
\gppoint{gp mark 0}{(5.621,4.171)}
\gppoint{gp mark 0}{(5.621,4.481)}
\gppoint{gp mark 0}{(5.621,4.204)}
\gppoint{gp mark 0}{(5.621,4.021)}
\gppoint{gp mark 0}{(5.621,3.979)}
\gppoint{gp mark 0}{(5.621,4.118)}
\gppoint{gp mark 0}{(5.621,4.327)}
\gppoint{gp mark 0}{(5.621,4.099)}
\gppoint{gp mark 0}{(5.621,4.080)}
\gppoint{gp mark 0}{(5.621,4.341)}
\gppoint{gp mark 0}{(5.621,3.911)}
\gppoint{gp mark 0}{(5.621,3.837)}
\gppoint{gp mark 0}{(5.621,4.283)}
\gppoint{gp mark 0}{(5.621,3.911)}
\gppoint{gp mark 0}{(5.621,4.021)}
\gppoint{gp mark 0}{(5.621,4.061)}
\gppoint{gp mark 0}{(5.621,3.957)}
\gppoint{gp mark 0}{(5.621,4.118)}
\gppoint{gp mark 0}{(5.621,4.041)}
\gppoint{gp mark 0}{(5.621,3.784)}
\gppoint{gp mark 0}{(5.621,4.099)}
\gppoint{gp mark 0}{(5.621,3.979)}
\gppoint{gp mark 0}{(5.621,4.327)}
\gppoint{gp mark 0}{(5.621,4.420)}
\gppoint{gp mark 0}{(5.621,3.697)}
\gppoint{gp mark 0}{(5.621,4.237)}
\gppoint{gp mark 0}{(5.621,4.021)}
\gppoint{gp mark 0}{(5.621,4.080)}
\gppoint{gp mark 0}{(5.621,4.021)}
\gppoint{gp mark 0}{(5.621,3.934)}
\gppoint{gp mark 0}{(5.621,4.171)}
\gppoint{gp mark 0}{(5.621,3.957)}
\gppoint{gp mark 0}{(5.621,3.911)}
\gppoint{gp mark 0}{(5.621,3.934)}
\gppoint{gp mark 0}{(5.621,4.341)}
\gppoint{gp mark 0}{(5.621,4.268)}
\gppoint{gp mark 0}{(5.621,4.237)}
\gppoint{gp mark 0}{(5.621,4.153)}
\gppoint{gp mark 0}{(5.621,4.237)}
\gppoint{gp mark 0}{(5.621,4.099)}
\gppoint{gp mark 0}{(5.621,3.911)}
\gppoint{gp mark 0}{(5.621,4.153)}
\gppoint{gp mark 0}{(5.621,3.887)}
\gppoint{gp mark 0}{(5.621,3.811)}
\gppoint{gp mark 0}{(5.621,3.911)}
\gppoint{gp mark 0}{(5.621,3.979)}
\gppoint{gp mark 0}{(5.621,3.887)}
\gppoint{gp mark 0}{(5.621,3.979)}
\gppoint{gp mark 0}{(5.621,4.099)}
\gppoint{gp mark 0}{(5.621,4.153)}
\gppoint{gp mark 0}{(5.621,4.118)}
\gppoint{gp mark 0}{(5.621,4.061)}
\gppoint{gp mark 0}{(5.621,4.312)}
\gppoint{gp mark 0}{(5.621,3.979)}
\gppoint{gp mark 0}{(5.621,3.979)}
\gppoint{gp mark 0}{(5.621,4.237)}
\gppoint{gp mark 0}{(5.621,4.080)}
\gppoint{gp mark 0}{(5.621,4.621)}
\gppoint{gp mark 0}{(5.621,3.862)}
\gppoint{gp mark 0}{(5.621,4.061)}
\gppoint{gp mark 0}{(5.621,3.911)}
\gppoint{gp mark 0}{(5.621,4.099)}
\gppoint{gp mark 0}{(5.621,3.934)}
\gppoint{gp mark 0}{(5.621,4.171)}
\gppoint{gp mark 0}{(5.621,4.021)}
\gppoint{gp mark 0}{(5.621,3.811)}
\gppoint{gp mark 0}{(5.621,4.549)}
\gppoint{gp mark 0}{(5.621,4.204)}
\gppoint{gp mark 0}{(5.621,4.000)}
\gppoint{gp mark 0}{(5.621,4.641)}
\gppoint{gp mark 0}{(5.621,4.327)}
\gppoint{gp mark 0}{(5.621,4.237)}
\gppoint{gp mark 0}{(5.621,3.756)}
\gppoint{gp mark 0}{(5.621,4.136)}
\gppoint{gp mark 0}{(5.621,3.756)}
\gppoint{gp mark 0}{(5.621,4.188)}
\gppoint{gp mark 0}{(5.621,4.549)}
\gppoint{gp mark 0}{(5.621,4.298)}
\gppoint{gp mark 0}{(5.621,3.837)}
\gppoint{gp mark 0}{(5.621,4.188)}
\gppoint{gp mark 0}{(5.621,4.395)}
\gppoint{gp mark 0}{(5.621,4.000)}
\gppoint{gp mark 0}{(5.621,4.188)}
\gppoint{gp mark 0}{(5.621,4.204)}
\gppoint{gp mark 0}{(5.621,3.957)}
\gppoint{gp mark 0}{(5.621,3.697)}
\gppoint{gp mark 0}{(5.621,4.041)}
\gppoint{gp mark 0}{(5.621,3.862)}
\gppoint{gp mark 0}{(5.621,4.327)}
\gppoint{gp mark 0}{(5.621,4.631)}
\gppoint{gp mark 0}{(5.621,3.862)}
\gppoint{gp mark 0}{(5.621,4.368)}
\gppoint{gp mark 0}{(5.621,4.099)}
\gppoint{gp mark 0}{(5.621,4.327)}
\gppoint{gp mark 0}{(5.621,3.862)}
\gppoint{gp mark 0}{(5.621,3.887)}
\gppoint{gp mark 0}{(5.621,4.153)}
\gppoint{gp mark 0}{(5.621,4.188)}
\gppoint{gp mark 0}{(5.621,4.283)}
\gppoint{gp mark 0}{(5.621,3.957)}
\gppoint{gp mark 0}{(5.621,4.099)}
\gppoint{gp mark 0}{(5.621,4.237)}
\gppoint{gp mark 0}{(5.621,4.204)}
\gppoint{gp mark 0}{(5.621,3.862)}
\gppoint{gp mark 0}{(5.621,3.934)}
\gppoint{gp mark 0}{(5.621,4.741)}
\gppoint{gp mark 0}{(5.621,4.136)}
\gppoint{gp mark 0}{(5.621,4.118)}
\gppoint{gp mark 0}{(5.621,4.408)}
\gppoint{gp mark 0}{(5.621,4.021)}
\gppoint{gp mark 0}{(5.621,4.118)}
\gppoint{gp mark 0}{(5.621,4.252)}
\gppoint{gp mark 0}{(5.621,4.021)}
\gppoint{gp mark 0}{(5.621,3.600)}
\gppoint{gp mark 0}{(5.621,3.911)}
\gppoint{gp mark 0}{(5.621,4.327)}
\gppoint{gp mark 0}{(5.621,3.727)}
\gppoint{gp mark 0}{(5.621,4.136)}
\gppoint{gp mark 0}{(5.621,4.268)}
\gppoint{gp mark 0}{(5.621,3.957)}
\gppoint{gp mark 0}{(5.621,4.188)}
\gppoint{gp mark 0}{(5.621,4.237)}
\gppoint{gp mark 0}{(5.652,3.911)}
\gppoint{gp mark 0}{(5.652,4.408)}
\gppoint{gp mark 0}{(5.652,4.188)}
\gppoint{gp mark 0}{(5.652,4.469)}
\gppoint{gp mark 0}{(5.652,4.381)}
\gppoint{gp mark 0}{(5.652,4.099)}
\gppoint{gp mark 0}{(5.652,4.080)}
\gppoint{gp mark 0}{(5.652,4.000)}
\gppoint{gp mark 0}{(5.652,4.021)}
\gppoint{gp mark 0}{(5.652,4.000)}
\gppoint{gp mark 0}{(5.652,3.979)}
\gppoint{gp mark 0}{(5.652,4.000)}
\gppoint{gp mark 0}{(5.652,4.283)}
\gppoint{gp mark 0}{(5.652,3.634)}
\gppoint{gp mark 0}{(5.652,4.118)}
\gppoint{gp mark 0}{(5.652,4.099)}
\gppoint{gp mark 0}{(5.652,4.252)}
\gppoint{gp mark 0}{(5.652,4.283)}
\gppoint{gp mark 0}{(5.652,4.481)}
\gppoint{gp mark 0}{(5.652,4.021)}
\gppoint{gp mark 0}{(5.652,4.237)}
\gppoint{gp mark 0}{(5.652,3.887)}
\gppoint{gp mark 0}{(5.652,4.252)}
\gppoint{gp mark 0}{(5.652,3.979)}
\gppoint{gp mark 0}{(5.652,3.666)}
\gppoint{gp mark 0}{(5.652,4.445)}
\gppoint{gp mark 0}{(5.652,4.221)}
\gppoint{gp mark 0}{(5.652,4.312)}
\gppoint{gp mark 0}{(5.652,3.756)}
\gppoint{gp mark 0}{(5.652,4.021)}
\gppoint{gp mark 0}{(5.652,4.252)}
\gppoint{gp mark 0}{(5.652,3.756)}
\gppoint{gp mark 0}{(5.652,4.221)}
\gppoint{gp mark 0}{(5.652,4.283)}
\gppoint{gp mark 0}{(5.652,4.327)}
\gppoint{gp mark 0}{(5.652,4.021)}
\gppoint{gp mark 0}{(5.652,4.041)}
\gppoint{gp mark 0}{(5.652,3.697)}
\gppoint{gp mark 0}{(5.652,3.811)}
\gppoint{gp mark 0}{(5.652,4.171)}
\gppoint{gp mark 0}{(5.652,4.381)}
\gppoint{gp mark 0}{(5.652,4.061)}
\gppoint{gp mark 0}{(5.652,3.727)}
\gppoint{gp mark 0}{(5.652,3.837)}
\gppoint{gp mark 0}{(5.652,4.099)}
\gppoint{gp mark 0}{(5.652,4.368)}
\gppoint{gp mark 0}{(5.652,3.957)}
\gppoint{gp mark 0}{(5.652,3.957)}
\gppoint{gp mark 0}{(5.652,4.221)}
\gppoint{gp mark 0}{(5.652,4.204)}
\gppoint{gp mark 0}{(5.652,4.504)}
\gppoint{gp mark 0}{(5.652,4.136)}
\gppoint{gp mark 0}{(5.652,4.268)}
\gppoint{gp mark 0}{(5.652,4.252)}
\gppoint{gp mark 0}{(5.652,4.283)}
\gppoint{gp mark 0}{(5.652,4.420)}
\gppoint{gp mark 0}{(5.652,4.000)}
\gppoint{gp mark 0}{(5.652,4.381)}
\gppoint{gp mark 0}{(5.652,4.283)}
\gppoint{gp mark 0}{(5.652,4.000)}
\gppoint{gp mark 0}{(5.652,4.830)}
\gppoint{gp mark 0}{(5.652,4.651)}
\gppoint{gp mark 0}{(5.652,4.000)}
\gppoint{gp mark 0}{(5.652,4.252)}
\gppoint{gp mark 0}{(5.652,4.204)}
\gppoint{gp mark 0}{(5.652,4.188)}
\gppoint{gp mark 0}{(5.652,3.979)}
\gppoint{gp mark 0}{(5.652,4.631)}
\gppoint{gp mark 0}{(5.652,4.469)}
\gppoint{gp mark 0}{(5.652,4.312)}
\gppoint{gp mark 0}{(5.652,4.268)}
\gppoint{gp mark 0}{(5.652,4.538)}
\gppoint{gp mark 0}{(5.652,4.560)}
\gppoint{gp mark 0}{(5.652,4.395)}
\gppoint{gp mark 0}{(5.652,4.080)}
\gppoint{gp mark 0}{(5.652,4.706)}
\gppoint{gp mark 0}{(5.652,4.171)}
\gppoint{gp mark 0}{(5.652,4.560)}
\gppoint{gp mark 0}{(5.652,4.433)}
\gppoint{gp mark 0}{(5.652,4.395)}
\gppoint{gp mark 0}{(5.652,3.600)}
\gppoint{gp mark 0}{(5.652,4.560)}
\gppoint{gp mark 0}{(5.652,3.697)}
\gppoint{gp mark 0}{(5.652,4.021)}
\gppoint{gp mark 0}{(5.652,4.021)}
\gppoint{gp mark 0}{(5.652,3.666)}
\gppoint{gp mark 0}{(5.652,4.041)}
\gppoint{gp mark 0}{(5.652,4.061)}
\gppoint{gp mark 0}{(5.652,3.784)}
\gppoint{gp mark 0}{(5.652,4.433)}
\gppoint{gp mark 0}{(5.652,4.433)}
\gppoint{gp mark 0}{(5.652,4.283)}
\gppoint{gp mark 0}{(5.652,4.188)}
\gppoint{gp mark 0}{(5.652,4.268)}
\gppoint{gp mark 0}{(5.652,3.957)}
\gppoint{gp mark 0}{(5.652,3.957)}
\gppoint{gp mark 0}{(5.652,3.911)}
\gppoint{gp mark 0}{(5.652,3.911)}
\gppoint{gp mark 0}{(5.652,3.957)}
\gppoint{gp mark 0}{(5.652,3.784)}
\gppoint{gp mark 0}{(5.652,4.118)}
\gppoint{gp mark 0}{(5.652,3.697)}
\gppoint{gp mark 0}{(5.652,3.862)}
\gppoint{gp mark 0}{(5.652,4.080)}
\gppoint{gp mark 0}{(5.652,4.591)}
\gppoint{gp mark 0}{(5.652,4.830)}
\gppoint{gp mark 0}{(5.652,3.756)}
\gppoint{gp mark 0}{(5.652,3.887)}
\gppoint{gp mark 0}{(5.652,3.450)}
\gppoint{gp mark 0}{(5.652,4.312)}
\gppoint{gp mark 0}{(5.652,4.237)}
\gppoint{gp mark 0}{(5.652,4.021)}
\gppoint{gp mark 0}{(5.652,4.188)}
\gppoint{gp mark 0}{(5.652,3.837)}
\gppoint{gp mark 0}{(5.652,4.493)}
\gppoint{gp mark 0}{(5.652,4.080)}
\gppoint{gp mark 0}{(5.652,4.171)}
\gppoint{gp mark 0}{(5.652,4.188)}
\gppoint{gp mark 0}{(5.652,4.504)}
\gppoint{gp mark 0}{(5.652,4.041)}
\gppoint{gp mark 0}{(5.652,4.153)}
\gppoint{gp mark 0}{(5.652,4.099)}
\gppoint{gp mark 0}{(5.652,3.911)}
\gppoint{gp mark 0}{(5.652,4.298)}
\gppoint{gp mark 0}{(5.652,4.204)}
\gppoint{gp mark 0}{(5.652,4.171)}
\gppoint{gp mark 0}{(5.652,4.021)}
\gppoint{gp mark 0}{(5.652,4.670)}
\gppoint{gp mark 0}{(5.652,4.283)}
\gppoint{gp mark 0}{(5.652,4.136)}
\gppoint{gp mark 0}{(5.652,4.118)}
\gppoint{gp mark 0}{(5.652,3.911)}
\gppoint{gp mark 0}{(5.652,4.237)}
\gppoint{gp mark 0}{(5.652,4.252)}
\gppoint{gp mark 0}{(5.652,4.061)}
\gppoint{gp mark 0}{(5.652,4.061)}
\gppoint{gp mark 0}{(5.652,4.136)}
\gppoint{gp mark 0}{(5.652,4.283)}
\gppoint{gp mark 0}{(5.652,3.957)}
\gppoint{gp mark 0}{(5.652,4.706)}
\gppoint{gp mark 0}{(5.652,4.268)}
\gppoint{gp mark 0}{(5.652,3.697)}
\gppoint{gp mark 0}{(5.652,4.021)}
\gppoint{gp mark 0}{(5.652,4.204)}
\gppoint{gp mark 0}{(5.652,4.118)}
\gppoint{gp mark 0}{(5.652,3.784)}
\gppoint{gp mark 0}{(5.682,3.756)}
\gppoint{gp mark 0}{(5.682,3.887)}
\gppoint{gp mark 0}{(5.682,4.188)}
\gppoint{gp mark 0}{(5.682,3.756)}
\gppoint{gp mark 0}{(5.682,3.837)}
\gppoint{gp mark 0}{(5.682,4.457)}
\gppoint{gp mark 0}{(5.682,4.298)}
\gppoint{gp mark 0}{(5.682,4.080)}
\gppoint{gp mark 0}{(5.682,3.666)}
\gppoint{gp mark 0}{(5.682,4.080)}
\gppoint{gp mark 0}{(5.682,4.188)}
\gppoint{gp mark 0}{(5.682,3.957)}
\gppoint{gp mark 0}{(5.682,4.118)}
\gppoint{gp mark 0}{(5.682,4.171)}
\gppoint{gp mark 0}{(5.682,4.000)}
\gppoint{gp mark 0}{(5.682,3.934)}
\gppoint{gp mark 0}{(5.682,4.080)}
\gppoint{gp mark 0}{(5.682,3.887)}
\gppoint{gp mark 0}{(5.682,4.171)}
\gppoint{gp mark 0}{(5.682,3.811)}
\gppoint{gp mark 0}{(5.682,3.862)}
\gppoint{gp mark 0}{(5.682,4.706)}
\gppoint{gp mark 0}{(5.682,3.979)}
\gppoint{gp mark 0}{(5.682,4.204)}
\gppoint{gp mark 0}{(5.682,4.041)}
\gppoint{gp mark 0}{(5.682,4.188)}
\gppoint{gp mark 0}{(5.682,4.000)}
\gppoint{gp mark 0}{(5.682,4.041)}
\gppoint{gp mark 0}{(5.682,4.021)}
\gppoint{gp mark 0}{(5.682,3.887)}
\gppoint{gp mark 0}{(5.682,4.021)}
\gppoint{gp mark 0}{(5.682,4.516)}
\gppoint{gp mark 0}{(5.682,4.136)}
\gppoint{gp mark 0}{(5.682,4.252)}
\gppoint{gp mark 0}{(5.682,4.188)}
\gppoint{gp mark 0}{(5.682,4.000)}
\gppoint{gp mark 0}{(5.682,4.118)}
\gppoint{gp mark 0}{(5.682,4.153)}
\gppoint{gp mark 0}{(5.682,4.252)}
\gppoint{gp mark 0}{(5.682,4.099)}
\gppoint{gp mark 0}{(5.682,4.237)}
\gppoint{gp mark 0}{(5.682,3.911)}
\gppoint{gp mark 0}{(5.682,4.204)}
\gppoint{gp mark 0}{(5.682,4.204)}
\gppoint{gp mark 0}{(5.682,3.811)}
\gppoint{gp mark 0}{(5.682,4.188)}
\gppoint{gp mark 0}{(5.682,3.957)}
\gppoint{gp mark 0}{(5.682,4.021)}
\gppoint{gp mark 0}{(5.682,4.099)}
\gppoint{gp mark 0}{(5.682,4.341)}
\gppoint{gp mark 0}{(5.682,4.171)}
\gppoint{gp mark 0}{(5.682,3.727)}
\gppoint{gp mark 0}{(5.682,3.887)}
\gppoint{gp mark 0}{(5.682,3.529)}
\gppoint{gp mark 0}{(5.682,3.756)}
\gppoint{gp mark 0}{(5.682,3.957)}
\gppoint{gp mark 0}{(5.682,4.136)}
\gppoint{gp mark 0}{(5.682,4.080)}
\gppoint{gp mark 0}{(5.682,3.784)}
\gppoint{gp mark 0}{(5.682,4.381)}
\gppoint{gp mark 0}{(5.682,4.457)}
\gppoint{gp mark 0}{(5.682,4.061)}
\gppoint{gp mark 0}{(5.682,4.621)}
\gppoint{gp mark 0}{(5.682,3.979)}
\gppoint{gp mark 0}{(5.682,4.549)}
\gppoint{gp mark 0}{(5.682,3.957)}
\gppoint{gp mark 0}{(5.682,4.204)}
\gppoint{gp mark 0}{(5.682,4.493)}
\gppoint{gp mark 0}{(5.682,4.204)}
\gppoint{gp mark 0}{(5.682,3.756)}
\gppoint{gp mark 0}{(5.682,4.283)}
\gppoint{gp mark 0}{(5.682,4.298)}
\gppoint{gp mark 0}{(5.682,4.591)}
\gppoint{gp mark 0}{(5.682,4.099)}
\gppoint{gp mark 0}{(5.682,3.979)}
\gppoint{gp mark 0}{(5.682,4.368)}
\gppoint{gp mark 0}{(5.682,3.911)}
\gppoint{gp mark 0}{(5.682,4.061)}
\gppoint{gp mark 0}{(5.682,3.957)}
\gppoint{gp mark 0}{(5.682,4.354)}
\gppoint{gp mark 0}{(5.682,3.862)}
\gppoint{gp mark 0}{(5.682,3.811)}
\gppoint{gp mark 0}{(5.682,4.188)}
\gppoint{gp mark 0}{(5.682,4.021)}
\gppoint{gp mark 0}{(5.682,3.887)}
\gppoint{gp mark 0}{(5.682,4.204)}
\gppoint{gp mark 0}{(5.682,4.549)}
\gppoint{gp mark 0}{(5.682,4.061)}
\gppoint{gp mark 0}{(5.682,4.381)}
\gppoint{gp mark 0}{(5.682,4.312)}
\gppoint{gp mark 0}{(5.682,4.354)}
\gppoint{gp mark 0}{(5.682,4.204)}
\gppoint{gp mark 0}{(5.682,4.099)}
\gppoint{gp mark 0}{(5.682,4.516)}
\gppoint{gp mark 0}{(5.682,4.118)}
\gppoint{gp mark 0}{(5.682,4.099)}
\gppoint{gp mark 0}{(5.682,4.327)}
\gppoint{gp mark 0}{(5.682,4.252)}
\gppoint{gp mark 0}{(5.682,4.420)}
\gppoint{gp mark 0}{(5.682,3.979)}
\gppoint{gp mark 0}{(5.682,4.268)}
\gppoint{gp mark 0}{(5.682,4.061)}
\gppoint{gp mark 0}{(5.682,4.099)}
\gppoint{gp mark 0}{(5.682,4.481)}
\gppoint{gp mark 0}{(5.682,4.021)}
\gppoint{gp mark 0}{(5.682,4.493)}
\gppoint{gp mark 0}{(5.682,4.021)}
\gppoint{gp mark 0}{(5.682,4.493)}
\gppoint{gp mark 0}{(5.682,4.381)}
\gppoint{gp mark 0}{(5.682,4.581)}
\gppoint{gp mark 0}{(5.682,3.934)}
\gppoint{gp mark 0}{(5.682,4.252)}
\gppoint{gp mark 0}{(5.682,3.634)}
\gppoint{gp mark 0}{(5.682,4.750)}
\gppoint{gp mark 0}{(5.682,4.021)}
\gppoint{gp mark 0}{(5.682,4.153)}
\gppoint{gp mark 0}{(5.682,4.021)}
\gppoint{gp mark 0}{(5.682,4.099)}
\gppoint{gp mark 0}{(5.682,4.171)}
\gppoint{gp mark 0}{(5.682,4.204)}
\gppoint{gp mark 0}{(5.682,3.887)}
\gppoint{gp mark 0}{(5.682,4.283)}
\gppoint{gp mark 0}{(5.682,4.493)}
\gppoint{gp mark 0}{(5.682,3.887)}
\gppoint{gp mark 0}{(5.682,3.727)}
\gppoint{gp mark 0}{(5.711,4.080)}
\gppoint{gp mark 0}{(5.711,3.979)}
\gppoint{gp mark 0}{(5.711,3.934)}
\gppoint{gp mark 0}{(5.711,4.000)}
\gppoint{gp mark 0}{(5.711,4.354)}
\gppoint{gp mark 0}{(5.711,4.283)}
\gppoint{gp mark 0}{(5.711,4.021)}
\gppoint{gp mark 0}{(5.711,4.136)}
\gppoint{gp mark 0}{(5.711,3.957)}
\gppoint{gp mark 0}{(5.711,4.469)}
\gppoint{gp mark 0}{(5.711,4.312)}
\gppoint{gp mark 0}{(5.711,4.000)}
\gppoint{gp mark 0}{(5.711,3.957)}
\gppoint{gp mark 0}{(5.711,4.679)}
\gppoint{gp mark 0}{(5.711,3.837)}
\gppoint{gp mark 0}{(5.711,5.114)}
\gppoint{gp mark 0}{(5.711,4.136)}
\gppoint{gp mark 0}{(5.711,4.021)}
\gppoint{gp mark 0}{(5.711,4.204)}
\gppoint{gp mark 0}{(5.711,3.934)}
\gppoint{gp mark 0}{(5.711,4.041)}
\gppoint{gp mark 0}{(5.711,3.911)}
\gppoint{gp mark 0}{(5.711,3.957)}
\gppoint{gp mark 0}{(5.711,4.000)}
\gppoint{gp mark 0}{(5.711,4.041)}
\gppoint{gp mark 0}{(5.711,4.221)}
\gppoint{gp mark 0}{(5.711,4.621)}
\gppoint{gp mark 0}{(5.711,3.979)}
\gppoint{gp mark 0}{(5.711,4.354)}
\gppoint{gp mark 0}{(5.711,4.283)}
\gppoint{gp mark 0}{(5.711,4.354)}
\gppoint{gp mark 0}{(5.711,3.979)}
\gppoint{gp mark 0}{(5.711,4.000)}
\gppoint{gp mark 0}{(5.711,4.171)}
\gppoint{gp mark 0}{(5.711,4.283)}
\gppoint{gp mark 0}{(5.711,4.838)}
\gppoint{gp mark 0}{(5.711,4.268)}
\gppoint{gp mark 0}{(5.711,4.268)}
\gppoint{gp mark 0}{(5.711,3.697)}
\gppoint{gp mark 0}{(5.711,4.237)}
\gppoint{gp mark 0}{(5.711,4.268)}
\gppoint{gp mark 0}{(5.711,3.979)}
\gppoint{gp mark 0}{(5.711,3.697)}
\gppoint{gp mark 0}{(5.711,3.934)}
\gppoint{gp mark 0}{(5.711,4.000)}
\gppoint{gp mark 0}{(5.711,4.327)}
\gppoint{gp mark 0}{(5.711,4.354)}
\gppoint{gp mark 0}{(5.711,4.136)}
\gppoint{gp mark 0}{(5.711,4.136)}
\gppoint{gp mark 0}{(5.711,4.061)}
\gppoint{gp mark 0}{(5.711,4.312)}
\gppoint{gp mark 0}{(5.711,4.237)}
\gppoint{gp mark 0}{(5.711,4.136)}
\gppoint{gp mark 0}{(5.711,4.570)}
\gppoint{gp mark 0}{(5.711,4.846)}
\gppoint{gp mark 0}{(5.711,4.099)}
\gppoint{gp mark 0}{(5.711,4.368)}
\gppoint{gp mark 0}{(5.711,4.000)}
\gppoint{gp mark 0}{(5.711,4.408)}
\gppoint{gp mark 0}{(5.711,4.000)}
\gppoint{gp mark 0}{(5.711,4.252)}
\gppoint{gp mark 0}{(5.711,4.171)}
\gppoint{gp mark 0}{(5.711,4.312)}
\gppoint{gp mark 0}{(5.711,4.807)}
\gppoint{gp mark 0}{(5.711,4.221)}
\gppoint{gp mark 0}{(5.711,4.538)}
\gppoint{gp mark 0}{(5.711,4.041)}
\gppoint{gp mark 0}{(5.711,4.080)}
\gppoint{gp mark 0}{(5.711,4.041)}
\gppoint{gp mark 0}{(5.711,3.911)}
\gppoint{gp mark 0}{(5.711,3.565)}
\gppoint{gp mark 0}{(5.711,3.934)}
\gppoint{gp mark 0}{(5.711,4.283)}
\gppoint{gp mark 0}{(5.711,4.298)}
\gppoint{gp mark 0}{(5.711,4.312)}
\gppoint{gp mark 0}{(5.711,3.565)}
\gppoint{gp mark 0}{(5.711,3.666)}
\gppoint{gp mark 0}{(5.711,3.979)}
\gppoint{gp mark 0}{(5.711,4.136)}
\gppoint{gp mark 0}{(5.711,4.298)}
\gppoint{gp mark 0}{(5.711,3.911)}
\gppoint{gp mark 0}{(5.711,4.381)}
\gppoint{gp mark 0}{(5.711,3.911)}
\gppoint{gp mark 0}{(5.711,4.252)}
\gppoint{gp mark 0}{(5.711,4.433)}
\gppoint{gp mark 0}{(5.711,4.204)}
\gppoint{gp mark 0}{(5.711,4.000)}
\gppoint{gp mark 0}{(5.711,4.118)}
\gppoint{gp mark 0}{(5.711,4.221)}
\gppoint{gp mark 0}{(5.711,4.457)}
\gppoint{gp mark 0}{(5.711,4.381)}
\gppoint{gp mark 0}{(5.711,4.420)}
\gppoint{gp mark 0}{(5.711,4.268)}
\gppoint{gp mark 0}{(5.711,4.237)}
\gppoint{gp mark 0}{(5.711,4.670)}
\gppoint{gp mark 0}{(5.711,4.268)}
\gppoint{gp mark 0}{(5.711,4.268)}
\gppoint{gp mark 0}{(5.711,4.252)}
\gppoint{gp mark 0}{(5.711,4.341)}
\gppoint{gp mark 0}{(5.711,3.887)}
\gppoint{gp mark 0}{(5.711,3.887)}
\gppoint{gp mark 0}{(5.711,4.188)}
\gppoint{gp mark 0}{(5.711,3.979)}
\gppoint{gp mark 0}{(5.711,3.957)}
\gppoint{gp mark 0}{(5.711,3.600)}
\gppoint{gp mark 0}{(5.711,4.298)}
\gppoint{gp mark 0}{(5.711,4.298)}
\gppoint{gp mark 0}{(5.711,4.283)}
\gppoint{gp mark 0}{(5.711,4.775)}
\gppoint{gp mark 0}{(5.711,4.965)}
\gppoint{gp mark 0}{(5.711,4.327)}
\gppoint{gp mark 0}{(5.711,4.021)}
\gppoint{gp mark 0}{(5.711,4.171)}
\gppoint{gp mark 0}{(5.711,4.221)}
\gppoint{gp mark 0}{(5.711,4.118)}
\gppoint{gp mark 0}{(5.711,4.136)}
\gppoint{gp mark 0}{(5.711,3.565)}
\gppoint{gp mark 0}{(5.711,4.408)}
\gppoint{gp mark 0}{(5.711,4.381)}
\gppoint{gp mark 0}{(5.711,3.887)}
\gppoint{gp mark 0}{(5.711,4.080)}
\gppoint{gp mark 0}{(5.711,4.118)}
\gppoint{gp mark 0}{(5.711,3.911)}
\gppoint{gp mark 0}{(5.711,3.887)}
\gppoint{gp mark 0}{(5.711,3.811)}
\gppoint{gp mark 0}{(5.740,4.080)}
\gppoint{gp mark 0}{(5.740,3.666)}
\gppoint{gp mark 0}{(5.740,4.153)}
\gppoint{gp mark 0}{(5.740,3.957)}
\gppoint{gp mark 0}{(5.740,3.811)}
\gppoint{gp mark 0}{(5.740,4.041)}
\gppoint{gp mark 0}{(5.740,4.237)}
\gppoint{gp mark 0}{(5.740,3.957)}
\gppoint{gp mark 0}{(5.740,4.715)}
\gppoint{gp mark 0}{(5.740,4.469)}
\gppoint{gp mark 0}{(5.740,3.957)}
\gppoint{gp mark 0}{(5.740,4.153)}
\gppoint{gp mark 0}{(5.740,4.153)}
\gppoint{gp mark 0}{(5.740,4.080)}
\gppoint{gp mark 0}{(5.740,4.188)}
\gppoint{gp mark 0}{(5.740,4.136)}
\gppoint{gp mark 0}{(5.740,4.000)}
\gppoint{gp mark 0}{(5.740,4.312)}
\gppoint{gp mark 0}{(5.740,4.153)}
\gppoint{gp mark 0}{(5.740,4.153)}
\gppoint{gp mark 0}{(5.740,3.697)}
\gppoint{gp mark 0}{(5.740,4.591)}
\gppoint{gp mark 0}{(5.740,4.298)}
\gppoint{gp mark 0}{(5.740,4.408)}
\gppoint{gp mark 0}{(5.740,3.697)}
\gppoint{gp mark 0}{(5.740,3.697)}
\gppoint{gp mark 0}{(5.740,3.911)}
\gppoint{gp mark 0}{(5.740,3.979)}
\gppoint{gp mark 0}{(5.740,3.887)}
\gppoint{gp mark 0}{(5.740,4.758)}
\gppoint{gp mark 0}{(5.740,4.136)}
\gppoint{gp mark 0}{(5.740,3.957)}
\gppoint{gp mark 0}{(5.740,3.957)}
\gppoint{gp mark 0}{(5.740,4.118)}
\gppoint{gp mark 0}{(5.740,4.153)}
\gppoint{gp mark 0}{(5.740,4.493)}
\gppoint{gp mark 0}{(5.740,4.188)}
\gppoint{gp mark 0}{(5.740,4.504)}
\gppoint{gp mark 0}{(5.740,4.938)}
\gppoint{gp mark 0}{(5.740,3.934)}
\gppoint{gp mark 0}{(5.740,4.815)}
\gppoint{gp mark 0}{(5.740,3.979)}
\gppoint{gp mark 0}{(5.740,4.493)}
\gppoint{gp mark 0}{(5.740,4.021)}
\gppoint{gp mark 0}{(5.740,4.911)}
\gppoint{gp mark 0}{(5.740,4.099)}
\gppoint{gp mark 0}{(5.740,4.252)}
\gppoint{gp mark 0}{(5.740,4.061)}
\gppoint{gp mark 0}{(5.740,4.395)}
\gppoint{gp mark 0}{(5.740,3.979)}
\gppoint{gp mark 0}{(5.740,3.957)}
\gppoint{gp mark 0}{(5.740,4.041)}
\gppoint{gp mark 0}{(5.740,3.957)}
\gppoint{gp mark 0}{(5.740,4.221)}
\gppoint{gp mark 0}{(5.740,4.021)}
\gppoint{gp mark 0}{(5.740,4.493)}
\gppoint{gp mark 0}{(5.740,4.469)}
\gppoint{gp mark 0}{(5.740,4.237)}
\gppoint{gp mark 0}{(5.740,4.688)}
\gppoint{gp mark 0}{(5.740,4.080)}
\gppoint{gp mark 0}{(5.740,4.061)}
\gppoint{gp mark 0}{(5.740,4.153)}
\gppoint{gp mark 0}{(5.740,4.268)}
\gppoint{gp mark 0}{(5.740,4.021)}
\gppoint{gp mark 0}{(5.740,3.957)}
\gppoint{gp mark 0}{(5.740,4.188)}
\gppoint{gp mark 0}{(5.740,4.697)}
\gppoint{gp mark 0}{(5.740,4.268)}
\gppoint{gp mark 0}{(5.740,4.118)}
\gppoint{gp mark 0}{(5.740,4.118)}
\gppoint{gp mark 0}{(5.740,4.221)}
\gppoint{gp mark 0}{(5.740,4.171)}
\gppoint{gp mark 0}{(5.740,3.756)}
\gppoint{gp mark 0}{(5.740,4.268)}
\gppoint{gp mark 0}{(5.740,4.327)}
\gppoint{gp mark 0}{(5.740,4.041)}
\gppoint{gp mark 0}{(5.740,4.080)}
\gppoint{gp mark 0}{(5.740,3.887)}
\gppoint{gp mark 0}{(5.740,3.979)}
\gppoint{gp mark 0}{(5.740,4.651)}
\gppoint{gp mark 0}{(5.740,4.298)}
\gppoint{gp mark 0}{(5.740,4.221)}
\gppoint{gp mark 0}{(5.740,3.979)}
\gppoint{gp mark 0}{(5.740,4.171)}
\gppoint{gp mark 0}{(5.740,4.283)}
\gppoint{gp mark 0}{(5.740,4.268)}
\gppoint{gp mark 0}{(5.740,4.395)}
\gppoint{gp mark 0}{(5.740,4.000)}
\gppoint{gp mark 0}{(5.740,4.061)}
\gppoint{gp mark 0}{(5.740,4.136)}
\gppoint{gp mark 0}{(5.740,4.041)}
\gppoint{gp mark 0}{(5.740,3.837)}
\gppoint{gp mark 0}{(5.740,4.204)}
\gppoint{gp mark 0}{(5.740,4.252)}
\gppoint{gp mark 0}{(5.740,4.493)}
\gppoint{gp mark 0}{(5.740,4.080)}
\gppoint{gp mark 0}{(5.740,3.979)}
\gppoint{gp mark 0}{(5.740,4.061)}
\gppoint{gp mark 0}{(5.740,4.395)}
\gppoint{gp mark 0}{(5.740,4.204)}
\gppoint{gp mark 0}{(5.740,4.433)}
\gppoint{gp mark 0}{(5.740,3.957)}
\gppoint{gp mark 0}{(5.740,4.298)}
\gppoint{gp mark 0}{(5.740,4.767)}
\gppoint{gp mark 0}{(5.740,4.099)}
\gppoint{gp mark 0}{(5.740,4.080)}
\gppoint{gp mark 0}{(5.740,3.979)}
\gppoint{gp mark 0}{(5.740,4.408)}
\gppoint{gp mark 0}{(5.740,4.516)}
\gppoint{gp mark 0}{(5.740,4.252)}
\gppoint{gp mark 0}{(5.740,3.911)}
\gppoint{gp mark 0}{(5.740,3.837)}
\gppoint{gp mark 0}{(5.740,4.268)}
\gppoint{gp mark 0}{(5.740,4.099)}
\gppoint{gp mark 0}{(5.740,4.570)}
\gppoint{gp mark 0}{(5.740,3.957)}
\gppoint{gp mark 0}{(5.740,4.570)}
\gppoint{gp mark 0}{(5.740,4.237)}
\gppoint{gp mark 0}{(5.740,3.030)}
\gppoint{gp mark 0}{(5.740,4.136)}
\gppoint{gp mark 0}{(5.740,4.221)}
\gppoint{gp mark 0}{(5.740,3.837)}
\gppoint{gp mark 0}{(5.740,4.136)}
\gppoint{gp mark 0}{(5.740,4.221)}
\gppoint{gp mark 0}{(5.740,4.601)}
\gppoint{gp mark 0}{(5.767,4.118)}
\gppoint{gp mark 0}{(5.767,4.298)}
\gppoint{gp mark 0}{(5.767,4.706)}
\gppoint{gp mark 0}{(5.767,4.283)}
\gppoint{gp mark 0}{(5.767,4.445)}
\gppoint{gp mark 0}{(5.767,4.341)}
\gppoint{gp mark 0}{(5.767,4.204)}
\gppoint{gp mark 0}{(5.767,4.493)}
\gppoint{gp mark 0}{(5.767,4.237)}
\gppoint{gp mark 0}{(5.767,4.252)}
\gppoint{gp mark 0}{(5.767,4.041)}
\gppoint{gp mark 0}{(5.767,4.204)}
\gppoint{gp mark 0}{(5.767,3.957)}
\gppoint{gp mark 0}{(5.767,4.861)}
\gppoint{gp mark 0}{(5.767,3.934)}
\gppoint{gp mark 0}{(5.767,3.979)}
\gppoint{gp mark 0}{(5.767,4.221)}
\gppoint{gp mark 0}{(5.767,4.221)}
\gppoint{gp mark 0}{(5.767,4.341)}
\gppoint{gp mark 0}{(5.767,4.283)}
\gppoint{gp mark 0}{(5.767,4.237)}
\gppoint{gp mark 0}{(5.767,4.420)}
\gppoint{gp mark 0}{(5.767,4.601)}
\gppoint{gp mark 0}{(5.767,4.268)}
\gppoint{gp mark 0}{(5.767,4.504)}
\gppoint{gp mark 0}{(5.767,4.021)}
\gppoint{gp mark 0}{(5.767,4.408)}
\gppoint{gp mark 0}{(5.767,4.298)}
\gppoint{gp mark 0}{(5.767,4.061)}
\gppoint{gp mark 0}{(5.767,3.957)}
\gppoint{gp mark 0}{(5.767,3.887)}
\gppoint{gp mark 0}{(5.767,4.171)}
\gppoint{gp mark 0}{(5.767,4.099)}
\gppoint{gp mark 0}{(5.767,4.660)}
\gppoint{gp mark 0}{(5.767,4.237)}
\gppoint{gp mark 0}{(5.767,4.153)}
\gppoint{gp mark 0}{(5.767,4.775)}
\gppoint{gp mark 0}{(5.767,4.118)}
\gppoint{gp mark 0}{(5.767,3.450)}
\gppoint{gp mark 0}{(5.767,5.167)}
\gppoint{gp mark 0}{(5.767,3.911)}
\gppoint{gp mark 0}{(5.767,4.493)}
\gppoint{gp mark 0}{(5.767,4.099)}
\gppoint{gp mark 0}{(5.767,4.354)}
\gppoint{gp mark 0}{(5.767,4.354)}
\gppoint{gp mark 0}{(5.767,4.327)}
\gppoint{gp mark 0}{(5.767,4.341)}
\gppoint{gp mark 0}{(5.767,4.221)}
\gppoint{gp mark 0}{(5.767,4.612)}
\gppoint{gp mark 0}{(5.767,4.099)}
\gppoint{gp mark 0}{(5.767,4.118)}
\gppoint{gp mark 0}{(5.767,4.651)}
\gppoint{gp mark 0}{(5.767,3.727)}
\gppoint{gp mark 0}{(5.767,4.268)}
\gppoint{gp mark 0}{(5.767,3.887)}
\gppoint{gp mark 0}{(5.767,3.756)}
\gppoint{gp mark 0}{(5.767,4.041)}
\gppoint{gp mark 0}{(5.767,3.756)}
\gppoint{gp mark 0}{(5.767,4.061)}
\gppoint{gp mark 0}{(5.767,4.341)}
\gppoint{gp mark 0}{(5.767,4.354)}
\gppoint{gp mark 0}{(5.767,4.041)}
\gppoint{gp mark 0}{(5.767,4.041)}
\gppoint{gp mark 0}{(5.767,4.204)}
\gppoint{gp mark 0}{(5.767,4.188)}
\gppoint{gp mark 0}{(5.767,3.934)}
\gppoint{gp mark 0}{(5.767,4.136)}
\gppoint{gp mark 0}{(5.767,3.887)}
\gppoint{gp mark 0}{(5.767,4.118)}
\gppoint{gp mark 0}{(5.767,4.118)}
\gppoint{gp mark 0}{(5.767,4.153)}
\gppoint{gp mark 0}{(5.767,4.631)}
\gppoint{gp mark 0}{(5.767,3.979)}
\gppoint{gp mark 0}{(5.767,4.237)}
\gppoint{gp mark 0}{(5.767,4.099)}
\gppoint{gp mark 0}{(5.767,4.327)}
\gppoint{gp mark 0}{(5.767,4.433)}
\gppoint{gp mark 0}{(5.767,4.237)}
\gppoint{gp mark 0}{(5.767,4.136)}
\gppoint{gp mark 0}{(5.767,4.268)}
\gppoint{gp mark 0}{(5.767,3.887)}
\gppoint{gp mark 0}{(5.767,4.204)}
\gppoint{gp mark 0}{(5.767,4.327)}
\gppoint{gp mark 0}{(5.767,3.887)}
\gppoint{gp mark 0}{(5.767,3.979)}
\gppoint{gp mark 0}{(5.767,3.979)}
\gppoint{gp mark 0}{(5.767,3.214)}
\gppoint{gp mark 0}{(5.767,4.118)}
\gppoint{gp mark 0}{(5.767,4.853)}
\gppoint{gp mark 0}{(5.767,4.188)}
\gppoint{gp mark 0}{(5.767,4.527)}
\gppoint{gp mark 0}{(5.767,3.911)}
\gppoint{gp mark 0}{(5.767,4.395)}
\gppoint{gp mark 0}{(5.767,4.221)}
\gppoint{gp mark 0}{(5.767,4.136)}
\gppoint{gp mark 0}{(5.767,4.268)}
\gppoint{gp mark 0}{(5.767,3.756)}
\gppoint{gp mark 0}{(5.767,3.887)}
\gppoint{gp mark 0}{(5.767,4.041)}
\gppoint{gp mark 0}{(5.767,4.153)}
\gppoint{gp mark 0}{(5.767,4.312)}
\gppoint{gp mark 0}{(5.767,4.136)}
\gppoint{gp mark 0}{(5.767,4.381)}
\gppoint{gp mark 0}{(5.767,4.188)}
\gppoint{gp mark 0}{(5.767,4.312)}
\gppoint{gp mark 0}{(5.767,4.118)}
\gppoint{gp mark 0}{(5.767,4.171)}
\gppoint{gp mark 0}{(5.767,3.887)}
\gppoint{gp mark 0}{(5.767,3.979)}
\gppoint{gp mark 0}{(5.767,4.298)}
\gppoint{gp mark 0}{(5.767,4.118)}
\gppoint{gp mark 0}{(5.795,4.516)}
\gppoint{gp mark 0}{(5.795,4.136)}
\gppoint{gp mark 0}{(5.795,3.887)}
\gppoint{gp mark 0}{(5.795,4.341)}
\gppoint{gp mark 0}{(5.795,4.204)}
\gppoint{gp mark 0}{(5.795,4.516)}
\gppoint{gp mark 0}{(5.795,4.041)}
\gppoint{gp mark 0}{(5.795,4.252)}
\gppoint{gp mark 0}{(5.795,4.298)}
\gppoint{gp mark 0}{(5.795,4.000)}
\gppoint{gp mark 0}{(5.795,4.000)}
\gppoint{gp mark 0}{(5.795,4.153)}
\gppoint{gp mark 0}{(5.795,4.381)}
\gppoint{gp mark 0}{(5.795,4.118)}
\gppoint{gp mark 0}{(5.795,4.457)}
\gppoint{gp mark 0}{(5.795,4.268)}
\gppoint{gp mark 0}{(5.795,4.171)}
\gppoint{gp mark 0}{(5.795,4.118)}
\gppoint{gp mark 0}{(5.795,4.298)}
\gppoint{gp mark 0}{(5.795,4.368)}
\gppoint{gp mark 0}{(5.795,4.420)}
\gppoint{gp mark 0}{(5.795,4.153)}
\gppoint{gp mark 0}{(5.795,4.041)}
\gppoint{gp mark 0}{(5.795,4.171)}
\gppoint{gp mark 0}{(5.795,3.911)}
\gppoint{gp mark 0}{(5.795,4.641)}
\gppoint{gp mark 0}{(5.795,3.911)}
\gppoint{gp mark 0}{(5.795,4.875)}
\gppoint{gp mark 0}{(5.795,4.221)}
\gppoint{gp mark 0}{(5.795,3.727)}
\gppoint{gp mark 0}{(5.795,4.252)}
\gppoint{gp mark 0}{(5.795,4.188)}
\gppoint{gp mark 0}{(5.795,4.237)}
\gppoint{gp mark 0}{(5.795,4.457)}
\gppoint{gp mark 0}{(5.795,4.830)}
\gppoint{gp mark 0}{(5.795,4.549)}
\gppoint{gp mark 0}{(5.795,4.041)}
\gppoint{gp mark 0}{(5.795,3.979)}
\gppoint{gp mark 0}{(5.795,3.666)}
\gppoint{gp mark 0}{(5.795,4.408)}
\gppoint{gp mark 0}{(5.795,4.298)}
\gppoint{gp mark 0}{(5.795,3.756)}
\gppoint{gp mark 0}{(5.795,4.395)}
\gppoint{gp mark 0}{(5.795,4.697)}
\gppoint{gp mark 0}{(5.795,4.252)}
\gppoint{gp mark 0}{(5.795,4.283)}
\gppoint{gp mark 0}{(5.795,3.784)}
\gppoint{gp mark 0}{(5.795,4.631)}
\gppoint{gp mark 0}{(5.795,4.252)}
\gppoint{gp mark 0}{(5.795,3.957)}
\gppoint{gp mark 0}{(5.795,4.061)}
\gppoint{gp mark 0}{(5.795,3.887)}
\gppoint{gp mark 0}{(5.795,4.408)}
\gppoint{gp mark 0}{(5.795,4.368)}
\gppoint{gp mark 0}{(5.795,3.837)}
\gppoint{gp mark 0}{(5.795,4.204)}
\gppoint{gp mark 0}{(5.795,4.312)}
\gppoint{gp mark 0}{(5.795,4.516)}
\gppoint{gp mark 0}{(5.795,4.099)}
\gppoint{gp mark 0}{(5.795,4.298)}
\gppoint{gp mark 0}{(5.795,4.298)}
\gppoint{gp mark 0}{(5.795,3.811)}
\gppoint{gp mark 0}{(5.795,4.080)}
\gppoint{gp mark 0}{(5.795,4.715)}
\gppoint{gp mark 0}{(5.795,4.041)}
\gppoint{gp mark 0}{(5.795,4.481)}
\gppoint{gp mark 0}{(5.795,3.957)}
\gppoint{gp mark 0}{(5.795,4.641)}
\gppoint{gp mark 0}{(5.795,4.188)}
\gppoint{gp mark 0}{(5.795,4.354)}
\gppoint{gp mark 0}{(5.795,3.030)}
\gppoint{gp mark 0}{(5.795,3.837)}
\gppoint{gp mark 0}{(5.795,4.153)}
\gppoint{gp mark 0}{(5.795,4.153)}
\gppoint{gp mark 0}{(5.795,4.457)}
\gppoint{gp mark 0}{(5.795,3.727)}
\gppoint{gp mark 0}{(5.795,4.000)}
\gppoint{gp mark 0}{(5.795,3.030)}
\gppoint{gp mark 0}{(5.795,3.837)}
\gppoint{gp mark 0}{(5.795,4.000)}
\gppoint{gp mark 0}{(5.795,4.911)}
\gppoint{gp mark 0}{(5.795,4.481)}
\gppoint{gp mark 0}{(5.795,4.911)}
\gppoint{gp mark 0}{(5.795,4.516)}
\gppoint{gp mark 0}{(5.795,3.934)}
\gppoint{gp mark 0}{(5.795,3.957)}
\gppoint{gp mark 0}{(5.795,4.000)}
\gppoint{gp mark 0}{(5.795,4.000)}
\gppoint{gp mark 0}{(5.795,4.433)}
\gppoint{gp mark 0}{(5.795,4.660)}
\gppoint{gp mark 0}{(5.795,4.153)}
\gppoint{gp mark 0}{(5.795,4.153)}
\gppoint{gp mark 0}{(5.795,4.021)}
\gppoint{gp mark 0}{(5.795,4.221)}
\gppoint{gp mark 0}{(5.795,4.153)}
\gppoint{gp mark 0}{(5.795,4.911)}
\gppoint{gp mark 0}{(5.795,4.252)}
\gppoint{gp mark 0}{(5.795,4.591)}
\gppoint{gp mark 0}{(5.795,4.433)}
\gppoint{gp mark 0}{(5.795,4.237)}
\gppoint{gp mark 0}{(5.795,3.979)}
\gppoint{gp mark 0}{(5.795,4.252)}
\gppoint{gp mark 0}{(5.821,4.516)}
\gppoint{gp mark 0}{(5.821,4.846)}
\gppoint{gp mark 0}{(5.821,4.298)}
\gppoint{gp mark 0}{(5.821,4.237)}
\gppoint{gp mark 0}{(5.821,4.560)}
\gppoint{gp mark 0}{(5.821,3.634)}
\gppoint{gp mark 0}{(5.821,4.327)}
\gppoint{gp mark 0}{(5.821,4.041)}
\gppoint{gp mark 0}{(5.821,4.481)}
\gppoint{gp mark 0}{(5.821,4.298)}
\gppoint{gp mark 0}{(5.821,4.904)}
\gppoint{gp mark 0}{(5.821,4.445)}
\gppoint{gp mark 0}{(5.821,4.641)}
\gppoint{gp mark 0}{(5.821,3.862)}
\gppoint{gp mark 0}{(5.821,4.171)}
\gppoint{gp mark 0}{(5.821,3.811)}
\gppoint{gp mark 0}{(5.821,4.298)}
\gppoint{gp mark 0}{(5.821,4.099)}
\gppoint{gp mark 0}{(5.821,3.957)}
\gppoint{gp mark 0}{(5.821,3.957)}
\gppoint{gp mark 0}{(5.821,4.000)}
\gppoint{gp mark 0}{(5.821,4.099)}
\gppoint{gp mark 0}{(5.821,4.000)}
\gppoint{gp mark 0}{(5.821,4.312)}
\gppoint{gp mark 0}{(5.821,4.099)}
\gppoint{gp mark 0}{(5.821,4.061)}
\gppoint{gp mark 0}{(5.821,4.670)}
\gppoint{gp mark 0}{(5.821,4.252)}
\gppoint{gp mark 0}{(5.821,4.237)}
\gppoint{gp mark 0}{(5.821,4.408)}
\gppoint{gp mark 0}{(5.821,4.560)}
\gppoint{gp mark 0}{(5.821,4.061)}
\gppoint{gp mark 0}{(5.821,4.000)}
\gppoint{gp mark 0}{(5.821,3.887)}
\gppoint{gp mark 0}{(5.821,4.570)}
\gppoint{gp mark 0}{(5.821,4.724)}
\gppoint{gp mark 0}{(5.821,4.252)}
\gppoint{gp mark 0}{(5.821,4.041)}
\gppoint{gp mark 0}{(5.821,4.041)}
\gppoint{gp mark 0}{(5.821,4.354)}
\gppoint{gp mark 0}{(5.821,4.136)}
\gppoint{gp mark 0}{(5.821,4.118)}
\gppoint{gp mark 0}{(5.821,4.815)}
\gppoint{gp mark 0}{(5.821,4.408)}
\gppoint{gp mark 0}{(5.821,3.934)}
\gppoint{gp mark 0}{(5.821,4.252)}
\gppoint{gp mark 0}{(5.821,3.911)}
\gppoint{gp mark 0}{(5.821,4.327)}
\gppoint{gp mark 0}{(5.821,4.420)}
\gppoint{gp mark 0}{(5.821,4.283)}
\gppoint{gp mark 0}{(5.821,4.136)}
\gppoint{gp mark 0}{(5.821,4.283)}
\gppoint{gp mark 0}{(5.821,4.118)}
\gppoint{gp mark 0}{(5.821,4.341)}
\gppoint{gp mark 0}{(5.821,4.368)}
\gppoint{gp mark 0}{(5.821,3.811)}
\gppoint{gp mark 0}{(5.821,4.080)}
\gppoint{gp mark 0}{(5.821,4.221)}
\gppoint{gp mark 0}{(5.821,4.188)}
\gppoint{gp mark 0}{(5.821,4.807)}
\gppoint{gp mark 0}{(5.821,3.756)}
\gppoint{gp mark 0}{(5.821,4.171)}
\gppoint{gp mark 0}{(5.821,4.395)}
\gppoint{gp mark 0}{(5.821,4.118)}
\gppoint{gp mark 0}{(5.821,3.784)}
\gppoint{gp mark 0}{(5.821,4.481)}
\gppoint{gp mark 0}{(5.821,4.327)}
\gppoint{gp mark 0}{(5.821,4.136)}
\gppoint{gp mark 0}{(5.821,4.354)}
\gppoint{gp mark 0}{(5.821,3.979)}
\gppoint{gp mark 0}{(5.821,4.408)}
\gppoint{gp mark 0}{(5.821,4.469)}
\gppoint{gp mark 0}{(5.821,4.000)}
\gppoint{gp mark 0}{(5.821,4.021)}
\gppoint{gp mark 0}{(5.821,4.327)}
\gppoint{gp mark 0}{(5.821,3.979)}
\gppoint{gp mark 0}{(5.821,4.741)}
\gppoint{gp mark 0}{(5.821,4.457)}
\gppoint{gp mark 0}{(5.821,4.570)}
\gppoint{gp mark 0}{(5.821,4.252)}
\gppoint{gp mark 0}{(5.821,4.061)}
\gppoint{gp mark 0}{(5.821,4.823)}
\gppoint{gp mark 0}{(5.821,4.741)}
\gppoint{gp mark 0}{(5.821,4.504)}
\gppoint{gp mark 0}{(5.821,4.298)}
\gppoint{gp mark 0}{(5.821,4.504)}
\gppoint{gp mark 0}{(5.821,4.341)}
\gppoint{gp mark 0}{(5.821,4.341)}
\gppoint{gp mark 0}{(5.821,4.153)}
\gppoint{gp mark 0}{(5.821,4.823)}
\gppoint{gp mark 0}{(5.821,4.080)}
\gppoint{gp mark 0}{(5.847,4.433)}
\gppoint{gp mark 0}{(5.847,4.136)}
\gppoint{gp mark 0}{(5.847,4.481)}
\gppoint{gp mark 0}{(5.847,4.136)}
\gppoint{gp mark 0}{(5.847,3.862)}
\gppoint{gp mark 0}{(5.847,4.298)}
\gppoint{gp mark 0}{(5.847,4.136)}
\gppoint{gp mark 0}{(5.847,4.775)}
\gppoint{gp mark 0}{(5.847,4.171)}
\gppoint{gp mark 0}{(5.847,4.327)}
\gppoint{gp mark 0}{(5.847,3.979)}
\gppoint{gp mark 0}{(5.847,4.099)}
\gppoint{gp mark 0}{(5.847,4.420)}
\gppoint{gp mark 0}{(5.847,3.979)}
\gppoint{gp mark 0}{(5.847,3.756)}
\gppoint{gp mark 0}{(5.847,4.268)}
\gppoint{gp mark 0}{(5.847,3.934)}
\gppoint{gp mark 0}{(5.847,4.527)}
\gppoint{gp mark 0}{(5.847,4.445)}
\gppoint{gp mark 0}{(5.847,4.445)}
\gppoint{gp mark 0}{(5.847,3.979)}
\gppoint{gp mark 0}{(5.847,4.237)}
\gppoint{gp mark 0}{(5.847,4.268)}
\gppoint{gp mark 0}{(5.847,3.911)}
\gppoint{gp mark 0}{(5.847,4.237)}
\gppoint{gp mark 0}{(5.847,4.420)}
\gppoint{gp mark 0}{(5.847,4.153)}
\gppoint{gp mark 0}{(5.847,4.136)}
\gppoint{gp mark 0}{(5.847,4.883)}
\gppoint{gp mark 0}{(5.847,4.395)}
\gppoint{gp mark 0}{(5.847,3.957)}
\gppoint{gp mark 0}{(5.847,4.697)}
\gppoint{gp mark 0}{(5.847,4.136)}
\gppoint{gp mark 0}{(5.847,4.312)}
\gppoint{gp mark 0}{(5.847,4.099)}
\gppoint{gp mark 0}{(5.847,4.354)}
\gppoint{gp mark 0}{(5.847,4.061)}
\gppoint{gp mark 0}{(5.847,4.516)}
\gppoint{gp mark 0}{(5.847,3.934)}
\gppoint{gp mark 0}{(5.847,4.171)}
\gppoint{gp mark 0}{(5.847,4.368)}
\gppoint{gp mark 0}{(5.847,4.368)}
\gppoint{gp mark 0}{(5.847,4.591)}
\gppoint{gp mark 0}{(5.847,4.395)}
\gppoint{gp mark 0}{(5.847,4.445)}
\gppoint{gp mark 0}{(5.847,4.368)}
\gppoint{gp mark 0}{(5.847,4.354)}
\gppoint{gp mark 0}{(5.847,4.516)}
\gppoint{gp mark 0}{(5.847,3.887)}
\gppoint{gp mark 0}{(5.847,4.080)}
\gppoint{gp mark 0}{(5.847,4.591)}
\gppoint{gp mark 0}{(5.847,4.591)}
\gppoint{gp mark 0}{(5.847,4.298)}
\gppoint{gp mark 0}{(5.847,4.481)}
\gppoint{gp mark 0}{(5.847,4.591)}
\gppoint{gp mark 0}{(5.847,4.153)}
\gppoint{gp mark 0}{(5.847,4.283)}
\gppoint{gp mark 0}{(5.847,4.312)}
\gppoint{gp mark 0}{(5.847,4.368)}
\gppoint{gp mark 0}{(5.847,4.408)}
\gppoint{gp mark 0}{(5.847,4.368)}
\gppoint{gp mark 0}{(5.847,4.368)}
\gppoint{gp mark 0}{(5.847,4.237)}
\gppoint{gp mark 0}{(5.847,4.171)}
\gppoint{gp mark 0}{(5.847,4.080)}
\gppoint{gp mark 0}{(5.847,4.136)}
\gppoint{gp mark 0}{(5.847,4.204)}
\gppoint{gp mark 0}{(5.847,4.099)}
\gppoint{gp mark 0}{(5.847,4.341)}
\gppoint{gp mark 0}{(5.847,4.188)}
\gppoint{gp mark 0}{(5.847,4.481)}
\gppoint{gp mark 0}{(5.847,3.979)}
\gppoint{gp mark 0}{(5.847,4.368)}
\gppoint{gp mark 0}{(5.847,4.368)}
\gppoint{gp mark 0}{(5.847,4.368)}
\gppoint{gp mark 0}{(5.847,4.136)}
\gppoint{gp mark 0}{(5.847,4.368)}
\gppoint{gp mark 0}{(5.847,4.368)}
\gppoint{gp mark 0}{(5.847,4.368)}
\gppoint{gp mark 0}{(5.847,4.080)}
\gppoint{gp mark 0}{(5.847,4.846)}
\gppoint{gp mark 0}{(5.847,3.934)}
\gppoint{gp mark 0}{(5.847,4.171)}
\gppoint{gp mark 0}{(5.847,4.000)}
\gppoint{gp mark 0}{(5.847,4.733)}
\gppoint{gp mark 0}{(5.847,3.934)}
\gppoint{gp mark 0}{(5.847,3.934)}
\gppoint{gp mark 0}{(5.847,4.099)}
\gppoint{gp mark 0}{(5.847,4.591)}
\gppoint{gp mark 0}{(5.847,3.934)}
\gppoint{gp mark 0}{(5.847,4.395)}
\gppoint{gp mark 0}{(5.847,3.934)}
\gppoint{gp mark 0}{(5.847,4.188)}
\gppoint{gp mark 0}{(5.847,4.570)}
\gppoint{gp mark 0}{(5.847,3.934)}
\gppoint{gp mark 0}{(5.847,4.591)}
\gppoint{gp mark 0}{(5.847,4.268)}
\gppoint{gp mark 0}{(5.847,4.549)}
\gppoint{gp mark 0}{(5.847,4.591)}
\gppoint{gp mark 0}{(5.847,4.408)}
\gppoint{gp mark 0}{(5.847,4.368)}
\gppoint{gp mark 0}{(5.847,4.298)}
\gppoint{gp mark 0}{(5.847,4.041)}
\gppoint{gp mark 0}{(5.847,4.153)}
\gppoint{gp mark 0}{(5.847,4.591)}
\gppoint{gp mark 0}{(5.847,4.838)}
\gppoint{gp mark 0}{(5.847,4.925)}
\gppoint{gp mark 0}{(5.847,4.221)}
\gppoint{gp mark 0}{(5.847,4.080)}
\gppoint{gp mark 0}{(5.847,4.237)}
\gppoint{gp mark 0}{(5.847,4.188)}
\gppoint{gp mark 0}{(5.847,4.368)}
\gppoint{gp mark 0}{(5.847,4.312)}
\gppoint{gp mark 0}{(5.873,3.934)}
\gppoint{gp mark 0}{(5.873,4.420)}
\gppoint{gp mark 0}{(5.873,4.021)}
\gppoint{gp mark 0}{(5.873,4.733)}
\gppoint{gp mark 0}{(5.873,4.724)}
\gppoint{gp mark 0}{(5.873,5.202)}
\gppoint{gp mark 0}{(5.873,4.381)}
\gppoint{gp mark 0}{(5.873,4.958)}
\gppoint{gp mark 0}{(5.873,4.268)}
\gppoint{gp mark 0}{(5.873,4.080)}
\gppoint{gp mark 0}{(5.873,4.204)}
\gppoint{gp mark 0}{(5.873,4.041)}
\gppoint{gp mark 0}{(5.873,3.811)}
\gppoint{gp mark 0}{(5.873,4.298)}
\gppoint{gp mark 0}{(5.873,4.445)}
\gppoint{gp mark 0}{(5.873,4.153)}
\gppoint{gp mark 0}{(5.873,4.516)}
\gppoint{gp mark 0}{(5.873,4.445)}
\gppoint{gp mark 0}{(5.873,4.433)}
\gppoint{gp mark 0}{(5.873,4.715)}
\gppoint{gp mark 0}{(5.873,4.516)}
\gppoint{gp mark 0}{(5.873,4.724)}
\gppoint{gp mark 0}{(5.873,4.724)}
\gppoint{gp mark 0}{(5.873,4.433)}
\gppoint{gp mark 0}{(5.873,4.925)}
\gppoint{gp mark 0}{(5.873,4.237)}
\gppoint{gp mark 0}{(5.873,4.445)}
\gppoint{gp mark 0}{(5.873,4.080)}
\gppoint{gp mark 0}{(5.873,4.504)}
\gppoint{gp mark 0}{(5.873,3.979)}
\gppoint{gp mark 0}{(5.873,4.041)}
\gppoint{gp mark 0}{(5.873,4.118)}
\gppoint{gp mark 0}{(5.873,4.136)}
\gppoint{gp mark 0}{(5.873,4.679)}
\gppoint{gp mark 0}{(5.873,4.298)}
\gppoint{gp mark 0}{(5.873,4.481)}
\gppoint{gp mark 0}{(5.873,4.298)}
\gppoint{gp mark 0}{(5.873,4.312)}
\gppoint{gp mark 0}{(5.873,4.327)}
\gppoint{gp mark 0}{(5.873,4.527)}
\gppoint{gp mark 0}{(5.873,4.697)}
\gppoint{gp mark 0}{(5.873,4.298)}
\gppoint{gp mark 0}{(5.873,4.061)}
\gppoint{gp mark 0}{(5.873,3.408)}
\gppoint{gp mark 0}{(5.873,4.080)}
\gppoint{gp mark 0}{(5.873,4.354)}
\gppoint{gp mark 0}{(5.873,5.081)}
\gppoint{gp mark 0}{(5.873,4.237)}
\gppoint{gp mark 0}{(5.873,4.099)}
\gppoint{gp mark 0}{(5.873,4.341)}
\gppoint{gp mark 0}{(5.873,4.080)}
\gppoint{gp mark 0}{(5.873,4.188)}
\gppoint{gp mark 0}{(5.873,3.756)}
\gppoint{gp mark 0}{(5.873,4.527)}
\gppoint{gp mark 0}{(5.873,4.354)}
\gppoint{gp mark 0}{(5.873,4.283)}
\gppoint{gp mark 0}{(5.873,3.957)}
\gppoint{gp mark 0}{(5.873,4.041)}
\gppoint{gp mark 0}{(5.873,4.457)}
\gppoint{gp mark 0}{(5.873,4.527)}
\gppoint{gp mark 0}{(5.873,4.204)}
\gppoint{gp mark 0}{(5.873,3.911)}
\gppoint{gp mark 0}{(5.873,4.381)}
\gppoint{gp mark 0}{(5.873,4.221)}
\gppoint{gp mark 0}{(5.873,4.118)}
\gppoint{gp mark 0}{(5.873,4.080)}
\gppoint{gp mark 0}{(5.873,4.354)}
\gppoint{gp mark 0}{(5.873,3.784)}
\gppoint{gp mark 0}{(5.873,4.283)}
\gppoint{gp mark 0}{(5.873,4.221)}
\gppoint{gp mark 0}{(5.873,4.041)}
\gppoint{gp mark 0}{(5.873,4.171)}
\gppoint{gp mark 0}{(5.873,4.612)}
\gppoint{gp mark 0}{(5.873,4.327)}
\gppoint{gp mark 0}{(5.873,3.957)}
\gppoint{gp mark 0}{(5.873,4.381)}
\gppoint{gp mark 0}{(5.873,4.408)}
\gppoint{gp mark 0}{(5.873,4.641)}
\gppoint{gp mark 0}{(5.873,4.354)}
\gppoint{gp mark 0}{(5.873,4.136)}
\gppoint{gp mark 0}{(5.873,4.312)}
\gppoint{gp mark 0}{(5.873,4.368)}
\gppoint{gp mark 0}{(5.873,4.631)}
\gppoint{gp mark 0}{(5.873,4.445)}
\gppoint{gp mark 0}{(5.873,4.061)}
\gppoint{gp mark 0}{(5.898,4.021)}
\gppoint{gp mark 0}{(5.898,3.979)}
\gppoint{gp mark 0}{(5.898,3.811)}
\gppoint{gp mark 0}{(5.898,4.697)}
\gppoint{gp mark 0}{(5.898,4.527)}
\gppoint{gp mark 0}{(5.898,4.136)}
\gppoint{gp mark 0}{(5.898,4.381)}
\gppoint{gp mark 0}{(5.898,4.469)}
\gppoint{gp mark 0}{(5.898,3.727)}
\gppoint{gp mark 0}{(5.898,4.631)}
\gppoint{gp mark 0}{(5.898,4.651)}
\gppoint{gp mark 0}{(5.898,4.298)}
\gppoint{gp mark 0}{(5.898,4.237)}
\gppoint{gp mark 0}{(5.898,4.457)}
\gppoint{gp mark 0}{(5.898,3.862)}
\gppoint{gp mark 0}{(5.898,4.171)}
\gppoint{gp mark 0}{(5.898,4.481)}
\gppoint{gp mark 0}{(5.898,4.527)}
\gppoint{gp mark 0}{(5.898,3.979)}
\gppoint{gp mark 0}{(5.898,3.837)}
\gppoint{gp mark 0}{(5.898,4.621)}
\gppoint{gp mark 0}{(5.898,4.136)}
\gppoint{gp mark 0}{(5.898,4.688)}
\gppoint{gp mark 0}{(5.898,4.838)}
\gppoint{gp mark 0}{(5.898,3.934)}
\gppoint{gp mark 0}{(5.898,4.679)}
\gppoint{gp mark 0}{(5.898,4.395)}
\gppoint{gp mark 0}{(5.898,3.979)}
\gppoint{gp mark 0}{(5.898,4.327)}
\gppoint{gp mark 0}{(5.898,4.171)}
\gppoint{gp mark 0}{(5.898,4.153)}
\gppoint{gp mark 0}{(5.898,4.861)}
\gppoint{gp mark 0}{(5.898,3.979)}
\gppoint{gp mark 0}{(5.898,4.171)}
\gppoint{gp mark 0}{(5.898,4.354)}
\gppoint{gp mark 0}{(5.898,4.221)}
\gppoint{gp mark 0}{(5.898,4.799)}
\gppoint{gp mark 0}{(5.898,4.153)}
\gppoint{gp mark 0}{(5.898,4.767)}
\gppoint{gp mark 0}{(5.898,4.368)}
\gppoint{gp mark 0}{(5.898,4.153)}
\gppoint{gp mark 0}{(5.898,4.298)}
\gppoint{gp mark 0}{(5.898,4.298)}
\gppoint{gp mark 0}{(5.898,4.368)}
\gppoint{gp mark 0}{(5.898,4.118)}
\gppoint{gp mark 0}{(5.898,4.631)}
\gppoint{gp mark 0}{(5.898,4.651)}
\gppoint{gp mark 0}{(5.898,4.061)}
\gppoint{gp mark 0}{(5.898,3.911)}
\gppoint{gp mark 0}{(5.898,4.631)}
\gppoint{gp mark 0}{(5.898,4.481)}
\gppoint{gp mark 0}{(5.898,4.601)}
\gppoint{gp mark 0}{(5.898,4.268)}
\gppoint{gp mark 0}{(5.898,4.612)}
\gppoint{gp mark 0}{(5.898,4.221)}
\gppoint{gp mark 0}{(5.898,4.354)}
\gppoint{gp mark 0}{(5.898,3.957)}
\gppoint{gp mark 0}{(5.898,4.408)}
\gppoint{gp mark 0}{(5.898,4.481)}
\gppoint{gp mark 0}{(5.898,4.298)}
\gppoint{gp mark 0}{(5.898,4.041)}
\gppoint{gp mark 0}{(5.898,4.420)}
\gppoint{gp mark 0}{(5.898,4.651)}
\gppoint{gp mark 0}{(5.898,4.651)}
\gppoint{gp mark 0}{(5.898,4.767)}
\gppoint{gp mark 0}{(5.898,4.252)}
\gppoint{gp mark 0}{(5.898,4.237)}
\gppoint{gp mark 0}{(5.898,4.171)}
\gppoint{gp mark 0}{(5.898,4.312)}
\gppoint{gp mark 0}{(5.898,4.420)}
\gppoint{gp mark 0}{(5.898,4.354)}
\gppoint{gp mark 0}{(5.898,4.237)}
\gppoint{gp mark 0}{(5.898,4.283)}
\gppoint{gp mark 0}{(5.898,4.368)}
\gppoint{gp mark 0}{(5.898,4.706)}
\gppoint{gp mark 0}{(5.898,3.837)}
\gppoint{gp mark 0}{(5.898,4.395)}
\gppoint{gp mark 0}{(5.898,4.481)}
\gppoint{gp mark 0}{(5.898,4.118)}
\gppoint{gp mark 0}{(5.898,4.341)}
\gppoint{gp mark 0}{(5.898,4.041)}
\gppoint{gp mark 0}{(5.898,3.727)}
\gppoint{gp mark 0}{(5.898,4.099)}
\gppoint{gp mark 0}{(5.898,4.099)}
\gppoint{gp mark 0}{(5.898,4.268)}
\gppoint{gp mark 0}{(5.898,3.887)}
\gppoint{gp mark 0}{(5.898,3.600)}
\gppoint{gp mark 0}{(5.922,4.395)}
\gppoint{gp mark 0}{(5.922,4.136)}
\gppoint{gp mark 0}{(5.922,4.733)}
\gppoint{gp mark 0}{(5.922,4.252)}
\gppoint{gp mark 0}{(5.922,4.021)}
\gppoint{gp mark 0}{(5.922,4.327)}
\gppoint{gp mark 0}{(5.922,4.061)}
\gppoint{gp mark 0}{(5.922,4.733)}
\gppoint{gp mark 0}{(5.922,4.298)}
\gppoint{gp mark 0}{(5.922,4.188)}
\gppoint{gp mark 0}{(5.922,4.237)}
\gppoint{gp mark 0}{(5.922,4.204)}
\gppoint{gp mark 0}{(5.922,4.381)}
\gppoint{gp mark 0}{(5.922,3.811)}
\gppoint{gp mark 0}{(5.922,4.527)}
\gppoint{gp mark 0}{(5.922,4.469)}
\gppoint{gp mark 0}{(5.922,4.298)}
\gppoint{gp mark 0}{(5.922,4.688)}
\gppoint{gp mark 0}{(5.922,4.481)}
\gppoint{gp mark 0}{(5.922,4.204)}
\gppoint{gp mark 0}{(5.922,4.312)}
\gppoint{gp mark 0}{(5.922,4.381)}
\gppoint{gp mark 0}{(5.922,4.268)}
\gppoint{gp mark 0}{(5.922,4.697)}
\gppoint{gp mark 0}{(5.922,4.918)}
\gppoint{gp mark 0}{(5.922,4.341)}
\gppoint{gp mark 0}{(5.922,4.368)}
\gppoint{gp mark 0}{(5.922,4.457)}
\gppoint{gp mark 0}{(5.922,4.041)}
\gppoint{gp mark 0}{(5.922,4.171)}
\gppoint{gp mark 0}{(5.922,4.433)}
\gppoint{gp mark 0}{(5.922,3.979)}
\gppoint{gp mark 0}{(5.922,4.061)}
\gppoint{gp mark 0}{(5.922,4.283)}
\gppoint{gp mark 0}{(5.922,4.433)}
\gppoint{gp mark 0}{(5.922,4.268)}
\gppoint{gp mark 0}{(5.922,4.283)}
\gppoint{gp mark 0}{(5.922,3.934)}
\gppoint{gp mark 0}{(5.922,4.875)}
\gppoint{gp mark 0}{(5.922,4.688)}
\gppoint{gp mark 0}{(5.922,3.811)}
\gppoint{gp mark 0}{(5.922,4.327)}
\gppoint{gp mark 0}{(5.922,4.153)}
\gppoint{gp mark 0}{(5.922,4.268)}
\gppoint{gp mark 0}{(5.922,4.549)}
\gppoint{gp mark 0}{(5.922,4.153)}
\gppoint{gp mark 0}{(5.922,4.972)}
\gppoint{gp mark 0}{(5.922,4.268)}
\gppoint{gp mark 0}{(5.922,4.433)}
\gppoint{gp mark 0}{(5.922,3.837)}
\gppoint{gp mark 0}{(5.922,4.868)}
\gppoint{gp mark 0}{(5.922,3.979)}
\gppoint{gp mark 0}{(5.922,4.861)}
\gppoint{gp mark 0}{(5.922,4.469)}
\gppoint{gp mark 0}{(5.922,3.979)}
\gppoint{gp mark 0}{(5.922,4.283)}
\gppoint{gp mark 0}{(5.922,4.204)}
\gppoint{gp mark 0}{(5.922,4.408)}
\gppoint{gp mark 0}{(5.922,4.312)}
\gppoint{gp mark 0}{(5.922,3.887)}
\gppoint{gp mark 0}{(5.922,4.758)}
\gppoint{gp mark 0}{(5.922,3.934)}
\gppoint{gp mark 0}{(5.922,4.061)}
\gppoint{gp mark 0}{(5.922,4.549)}
\gppoint{gp mark 0}{(5.922,4.080)}
\gppoint{gp mark 0}{(5.922,4.099)}
\gppoint{gp mark 0}{(5.922,4.601)}
\gppoint{gp mark 0}{(5.922,4.621)}
\gppoint{gp mark 0}{(5.922,4.171)}
\gppoint{gp mark 0}{(5.922,4.204)}
\gppoint{gp mark 0}{(5.922,4.560)}
\gppoint{gp mark 0}{(5.922,4.469)}
\gppoint{gp mark 0}{(5.922,4.601)}
\gppoint{gp mark 0}{(5.946,4.268)}
\gppoint{gp mark 0}{(5.946,4.061)}
\gppoint{gp mark 0}{(5.946,4.591)}
\gppoint{gp mark 0}{(5.946,4.153)}
\gppoint{gp mark 0}{(5.946,4.118)}
\gppoint{gp mark 0}{(5.946,4.237)}
\gppoint{gp mark 0}{(5.946,4.538)}
\gppoint{gp mark 0}{(5.946,4.283)}
\gppoint{gp mark 0}{(5.946,4.171)}
\gppoint{gp mark 0}{(5.946,4.381)}
\gppoint{gp mark 0}{(5.946,4.298)}
\gppoint{gp mark 0}{(5.946,4.560)}
\gppoint{gp mark 0}{(5.946,4.221)}
\gppoint{gp mark 0}{(5.946,4.298)}
\gppoint{gp mark 0}{(5.946,4.420)}
\gppoint{gp mark 0}{(5.946,4.527)}
\gppoint{gp mark 0}{(5.946,4.469)}
\gppoint{gp mark 0}{(5.946,4.381)}
\gppoint{gp mark 0}{(5.946,4.237)}
\gppoint{gp mark 0}{(5.946,4.381)}
\gppoint{gp mark 0}{(5.946,4.237)}
\gppoint{gp mark 0}{(5.946,4.925)}
\gppoint{gp mark 0}{(5.946,4.395)}
\gppoint{gp mark 0}{(5.946,3.666)}
\gppoint{gp mark 0}{(5.946,4.368)}
\gppoint{gp mark 0}{(5.946,4.395)}
\gppoint{gp mark 0}{(5.946,4.408)}
\gppoint{gp mark 0}{(5.946,4.061)}
\gppoint{gp mark 0}{(5.946,4.601)}
\gppoint{gp mark 0}{(5.946,4.697)}
\gppoint{gp mark 0}{(5.946,4.621)}
\gppoint{gp mark 0}{(5.946,4.354)}
\gppoint{gp mark 0}{(5.946,4.327)}
\gppoint{gp mark 0}{(5.946,3.030)}
\gppoint{gp mark 0}{(5.946,4.136)}
\gppoint{gp mark 0}{(5.946,4.099)}
\gppoint{gp mark 0}{(5.946,4.481)}
\gppoint{gp mark 0}{(5.946,4.204)}
\gppoint{gp mark 0}{(5.946,4.312)}
\gppoint{gp mark 0}{(5.946,4.118)}
\gppoint{gp mark 0}{(5.946,4.298)}
\gppoint{gp mark 0}{(5.946,4.204)}
\gppoint{gp mark 0}{(5.946,4.204)}
\gppoint{gp mark 0}{(5.946,4.188)}
\gppoint{gp mark 0}{(5.946,4.715)}
\gppoint{gp mark 0}{(5.946,4.395)}
\gppoint{gp mark 0}{(5.946,4.445)}
\gppoint{gp mark 0}{(5.946,4.080)}
\gppoint{gp mark 0}{(5.946,4.516)}
\gppoint{gp mark 0}{(5.946,4.612)}
\gppoint{gp mark 0}{(5.946,4.118)}
\gppoint{gp mark 0}{(5.946,4.504)}
\gppoint{gp mark 0}{(5.946,3.784)}
\gppoint{gp mark 0}{(5.946,4.252)}
\gppoint{gp mark 0}{(5.946,4.298)}
\gppoint{gp mark 0}{(5.946,3.979)}
\gppoint{gp mark 0}{(5.946,4.171)}
\gppoint{gp mark 0}{(5.946,4.153)}
\gppoint{gp mark 0}{(5.946,4.631)}
\gppoint{gp mark 0}{(5.946,4.341)}
\gppoint{gp mark 0}{(5.946,4.395)}
\gppoint{gp mark 0}{(5.946,4.136)}
\gppoint{gp mark 0}{(5.946,4.298)}
\gppoint{gp mark 0}{(5.946,3.666)}
\gppoint{gp mark 0}{(5.946,4.118)}
\gppoint{gp mark 0}{(5.946,4.204)}
\gppoint{gp mark 0}{(5.946,4.298)}
\gppoint{gp mark 0}{(5.946,3.811)}
\gppoint{gp mark 0}{(5.946,4.204)}
\gppoint{gp mark 0}{(5.946,4.481)}
\gppoint{gp mark 0}{(5.946,4.469)}
\gppoint{gp mark 0}{(5.946,4.118)}
\gppoint{gp mark 0}{(5.946,4.268)}
\gppoint{gp mark 0}{(5.946,4.171)}
\gppoint{gp mark 0}{(5.946,4.527)}
\gppoint{gp mark 0}{(5.946,4.420)}
\gppoint{gp mark 0}{(5.946,4.457)}
\gppoint{gp mark 0}{(5.946,3.811)}
\gppoint{gp mark 0}{(5.946,3.957)}
\gppoint{gp mark 0}{(5.946,4.368)}
\gppoint{gp mark 0}{(5.946,4.099)}
\gppoint{gp mark 0}{(5.946,3.979)}
\gppoint{gp mark 0}{(5.946,4.381)}
\gppoint{gp mark 0}{(5.946,4.883)}
\gppoint{gp mark 0}{(5.946,4.549)}
\gppoint{gp mark 0}{(5.969,4.527)}
\gppoint{gp mark 0}{(5.969,4.570)}
\gppoint{gp mark 0}{(5.969,4.153)}
\gppoint{gp mark 0}{(5.969,4.504)}
\gppoint{gp mark 0}{(5.969,4.221)}
\gppoint{gp mark 0}{(5.969,4.679)}
\gppoint{gp mark 0}{(5.969,4.560)}
\gppoint{gp mark 0}{(5.969,4.312)}
\gppoint{gp mark 0}{(5.969,4.457)}
\gppoint{gp mark 0}{(5.969,4.527)}
\gppoint{gp mark 0}{(5.969,4.965)}
\gppoint{gp mark 0}{(5.969,4.221)}
\gppoint{gp mark 0}{(5.969,4.312)}
\gppoint{gp mark 0}{(5.969,4.354)}
\gppoint{gp mark 0}{(5.969,4.445)}
\gppoint{gp mark 0}{(5.969,4.341)}
\gppoint{gp mark 0}{(5.969,4.153)}
\gppoint{gp mark 0}{(5.969,4.153)}
\gppoint{gp mark 0}{(5.969,4.621)}
\gppoint{gp mark 0}{(5.969,4.493)}
\gppoint{gp mark 0}{(5.969,4.136)}
\gppoint{gp mark 0}{(5.969,5.003)}
\gppoint{gp mark 0}{(5.969,4.457)}
\gppoint{gp mark 0}{(5.969,4.601)}
\gppoint{gp mark 0}{(5.969,4.354)}
\gppoint{gp mark 0}{(5.969,4.252)}
\gppoint{gp mark 0}{(5.969,5.016)}
\gppoint{gp mark 0}{(5.969,4.341)}
\gppoint{gp mark 0}{(5.969,4.527)}
\gppoint{gp mark 0}{(5.969,4.601)}
\gppoint{gp mark 0}{(5.969,2.880)}
\gppoint{gp mark 0}{(5.969,3.934)}
\gppoint{gp mark 0}{(5.969,3.934)}
\gppoint{gp mark 0}{(5.969,4.283)}
\gppoint{gp mark 0}{(5.969,4.493)}
\gppoint{gp mark 0}{(5.969,4.000)}
\gppoint{gp mark 0}{(5.969,4.420)}
\gppoint{gp mark 0}{(5.969,4.601)}
\gppoint{gp mark 0}{(5.969,4.601)}
\gppoint{gp mark 0}{(5.969,4.136)}
\gppoint{gp mark 0}{(5.969,4.758)}
\gppoint{gp mark 0}{(5.969,4.697)}
\gppoint{gp mark 0}{(5.969,3.030)}
\gppoint{gp mark 0}{(5.969,4.237)}
\gppoint{gp mark 0}{(5.969,4.670)}
\gppoint{gp mark 0}{(5.969,3.565)}
\gppoint{gp mark 0}{(5.969,4.341)}
\gppoint{gp mark 0}{(5.969,4.783)}
\gppoint{gp mark 0}{(5.969,4.670)}
\gppoint{gp mark 0}{(5.969,4.136)}
\gppoint{gp mark 0}{(5.969,4.061)}
\gppoint{gp mark 0}{(5.969,4.457)}
\gppoint{gp mark 0}{(5.969,4.327)}
\gppoint{gp mark 0}{(5.969,4.679)}
\gppoint{gp mark 0}{(5.969,4.354)}
\gppoint{gp mark 0}{(5.969,4.341)}
\gppoint{gp mark 0}{(5.969,4.433)}
\gppoint{gp mark 0}{(5.969,4.493)}
\gppoint{gp mark 0}{(5.969,4.767)}
\gppoint{gp mark 0}{(5.969,4.538)}
\gppoint{gp mark 0}{(5.969,4.312)}
\gppoint{gp mark 0}{(5.969,4.631)}
\gppoint{gp mark 0}{(5.969,4.252)}
\gppoint{gp mark 0}{(5.969,4.420)}
\gppoint{gp mark 0}{(5.969,4.041)}
\gppoint{gp mark 0}{(5.969,4.724)}
\gppoint{gp mark 0}{(5.969,4.631)}
\gppoint{gp mark 0}{(5.969,4.341)}
\gppoint{gp mark 0}{(5.969,4.457)}
\gppoint{gp mark 0}{(5.969,4.136)}
\gppoint{gp mark 0}{(5.969,4.171)}
\gppoint{gp mark 0}{(5.969,3.862)}
\gppoint{gp mark 0}{(5.969,4.445)}
\gppoint{gp mark 0}{(5.969,4.312)}
\gppoint{gp mark 0}{(5.969,4.136)}
\gppoint{gp mark 0}{(5.969,4.204)}
\gppoint{gp mark 0}{(5.969,4.327)}
\gppoint{gp mark 0}{(5.969,4.408)}
\gppoint{gp mark 0}{(5.969,4.670)}
\gppoint{gp mark 0}{(5.969,3.934)}
\gppoint{gp mark 0}{(5.969,4.188)}
\gppoint{gp mark 0}{(5.969,4.021)}
\gppoint{gp mark 0}{(5.969,4.420)}
\gppoint{gp mark 0}{(5.992,4.767)}
\gppoint{gp mark 0}{(5.992,4.516)}
\gppoint{gp mark 0}{(5.992,4.153)}
\gppoint{gp mark 0}{(5.992,4.408)}
\gppoint{gp mark 0}{(5.992,4.000)}
\gppoint{gp mark 0}{(5.992,4.237)}
\gppoint{gp mark 0}{(5.992,4.538)}
\gppoint{gp mark 0}{(5.992,4.268)}
\gppoint{gp mark 0}{(5.992,4.368)}
\gppoint{gp mark 0}{(5.992,4.268)}
\gppoint{gp mark 0}{(5.992,4.395)}
\gppoint{gp mark 0}{(5.992,4.420)}
\gppoint{gp mark 0}{(5.992,4.268)}
\gppoint{gp mark 0}{(5.992,4.000)}
\gppoint{gp mark 0}{(5.992,4.312)}
\gppoint{gp mark 0}{(5.992,4.724)}
\gppoint{gp mark 0}{(5.992,4.283)}
\gppoint{gp mark 0}{(5.992,4.368)}
\gppoint{gp mark 0}{(5.992,4.118)}
\gppoint{gp mark 0}{(5.992,4.061)}
\gppoint{gp mark 0}{(5.992,4.341)}
\gppoint{gp mark 0}{(5.992,4.433)}
\gppoint{gp mark 0}{(5.992,4.631)}
\gppoint{gp mark 0}{(5.992,4.395)}
\gppoint{gp mark 0}{(5.992,4.237)}
\gppoint{gp mark 0}{(5.992,4.341)}
\gppoint{gp mark 0}{(5.992,3.887)}
\gppoint{gp mark 0}{(5.992,4.237)}
\gppoint{gp mark 0}{(5.992,4.420)}
\gppoint{gp mark 0}{(5.992,4.118)}
\gppoint{gp mark 0}{(5.992,4.171)}
\gppoint{gp mark 0}{(5.992,4.327)}
\gppoint{gp mark 0}{(5.992,4.283)}
\gppoint{gp mark 0}{(5.992,4.327)}
\gppoint{gp mark 0}{(5.992,4.445)}
\gppoint{gp mark 0}{(5.992,4.041)}
\gppoint{gp mark 0}{(5.992,4.327)}
\gppoint{gp mark 0}{(5.992,4.118)}
\gppoint{gp mark 0}{(5.992,4.493)}
\gppoint{gp mark 0}{(5.992,4.312)}
\gppoint{gp mark 0}{(5.992,4.381)}
\gppoint{gp mark 0}{(5.992,4.549)}
\gppoint{gp mark 0}{(5.992,4.420)}
\gppoint{gp mark 0}{(5.992,3.979)}
\gppoint{gp mark 0}{(5.992,4.433)}
\gppoint{gp mark 0}{(5.992,4.549)}
\gppoint{gp mark 0}{(5.992,4.395)}
\gppoint{gp mark 0}{(5.992,4.252)}
\gppoint{gp mark 0}{(5.992,3.565)}
\gppoint{gp mark 0}{(5.992,4.171)}
\gppoint{gp mark 0}{(5.992,4.312)}
\gppoint{gp mark 0}{(5.992,4.457)}
\gppoint{gp mark 0}{(5.992,4.679)}
\gppoint{gp mark 0}{(5.992,4.283)}
\gppoint{gp mark 0}{(5.992,4.188)}
\gppoint{gp mark 0}{(5.992,4.433)}
\gppoint{gp mark 0}{(5.992,3.979)}
\gppoint{gp mark 0}{(5.992,4.395)}
\gppoint{gp mark 0}{(5.992,4.527)}
\gppoint{gp mark 0}{(5.992,4.570)}
\gppoint{gp mark 0}{(5.992,4.283)}
\gppoint{gp mark 0}{(5.992,4.283)}
\gppoint{gp mark 0}{(5.992,4.420)}
\gppoint{gp mark 0}{(6.015,4.099)}
\gppoint{gp mark 0}{(6.015,4.670)}
\gppoint{gp mark 0}{(6.015,4.341)}
\gppoint{gp mark 0}{(6.015,3.784)}
\gppoint{gp mark 0}{(6.015,4.549)}
\gppoint{gp mark 0}{(6.015,4.527)}
\gppoint{gp mark 0}{(6.015,5.022)}
\gppoint{gp mark 0}{(6.015,4.252)}
\gppoint{gp mark 0}{(6.015,4.061)}
\gppoint{gp mark 0}{(6.015,4.118)}
\gppoint{gp mark 0}{(6.015,5.058)}
\gppoint{gp mark 0}{(6.015,4.549)}
\gppoint{gp mark 0}{(6.015,4.570)}
\gppoint{gp mark 0}{(6.015,4.188)}
\gppoint{gp mark 0}{(6.015,4.153)}
\gppoint{gp mark 0}{(6.015,4.381)}
\gppoint{gp mark 0}{(6.015,4.516)}
\gppoint{gp mark 0}{(6.015,4.516)}
\gppoint{gp mark 0}{(6.015,4.118)}
\gppoint{gp mark 0}{(6.015,4.815)}
\gppoint{gp mark 0}{(6.015,4.395)}
\gppoint{gp mark 0}{(6.015,4.868)}
\gppoint{gp mark 0}{(6.015,4.188)}
\gppoint{gp mark 0}{(6.015,4.312)}
\gppoint{gp mark 0}{(6.015,4.868)}
\gppoint{gp mark 0}{(6.015,4.570)}
\gppoint{gp mark 0}{(6.015,4.312)}
\gppoint{gp mark 0}{(6.015,3.957)}
\gppoint{gp mark 0}{(6.015,4.204)}
\gppoint{gp mark 0}{(6.015,4.252)}
\gppoint{gp mark 0}{(6.015,4.268)}
\gppoint{gp mark 0}{(6.015,4.570)}
\gppoint{gp mark 0}{(6.015,4.408)}
\gppoint{gp mark 0}{(6.015,4.221)}
\gppoint{gp mark 0}{(6.015,4.153)}
\gppoint{gp mark 0}{(6.015,4.445)}
\gppoint{gp mark 0}{(6.015,4.136)}
\gppoint{gp mark 0}{(6.015,4.925)}
\gppoint{gp mark 0}{(6.015,4.136)}
\gppoint{gp mark 0}{(6.015,4.688)}
\gppoint{gp mark 0}{(6.015,4.171)}
\gppoint{gp mark 0}{(6.015,4.445)}
\gppoint{gp mark 0}{(6.015,4.469)}
\gppoint{gp mark 0}{(6.015,4.327)}
\gppoint{gp mark 0}{(6.015,4.268)}
\gppoint{gp mark 0}{(6.015,4.688)}
\gppoint{gp mark 0}{(6.015,4.420)}
\gppoint{gp mark 0}{(6.015,3.887)}
\gppoint{gp mark 0}{(6.015,4.516)}
\gppoint{gp mark 0}{(6.015,4.080)}
\gppoint{gp mark 0}{(6.015,4.408)}
\gppoint{gp mark 0}{(6.015,4.283)}
\gppoint{gp mark 0}{(6.015,4.268)}
\gppoint{gp mark 0}{(6.015,4.021)}
\gppoint{gp mark 0}{(6.015,4.136)}
\gppoint{gp mark 0}{(6.015,4.268)}
\gppoint{gp mark 0}{(6.015,4.354)}
\gppoint{gp mark 0}{(6.015,4.538)}
\gppoint{gp mark 0}{(6.015,4.381)}
\gppoint{gp mark 0}{(6.015,4.395)}
\gppoint{gp mark 0}{(6.015,4.591)}
\gppoint{gp mark 0}{(6.015,4.381)}
\gppoint{gp mark 0}{(6.015,4.581)}
\gppoint{gp mark 0}{(6.015,4.560)}
\gppoint{gp mark 0}{(6.015,4.061)}
\gppoint{gp mark 0}{(6.015,4.457)}
\gppoint{gp mark 0}{(6.015,4.268)}
\gppoint{gp mark 0}{(6.015,4.433)}
\gppoint{gp mark 0}{(6.015,4.560)}
\gppoint{gp mark 0}{(6.015,4.581)}
\gppoint{gp mark 0}{(6.015,4.560)}
\gppoint{gp mark 0}{(6.015,4.171)}
\gppoint{gp mark 0}{(6.015,4.341)}
\gppoint{gp mark 0}{(6.015,4.469)}
\gppoint{gp mark 0}{(6.015,4.504)}
\gppoint{gp mark 0}{(6.015,4.099)}
\gppoint{gp mark 0}{(6.015,4.327)}
\gppoint{gp mark 0}{(6.015,4.679)}
\gppoint{gp mark 0}{(6.015,4.354)}
\gppoint{gp mark 0}{(6.037,4.268)}
\gppoint{gp mark 0}{(6.037,4.188)}
\gppoint{gp mark 0}{(6.037,4.457)}
\gppoint{gp mark 0}{(6.037,4.420)}
\gppoint{gp mark 0}{(6.037,4.268)}
\gppoint{gp mark 0}{(6.037,3.887)}
\gppoint{gp mark 0}{(6.037,4.268)}
\gppoint{gp mark 0}{(6.037,3.934)}
\gppoint{gp mark 0}{(6.037,4.381)}
\gppoint{gp mark 0}{(6.037,4.591)}
\gppoint{gp mark 0}{(6.037,4.341)}
\gppoint{gp mark 0}{(6.037,4.381)}
\gppoint{gp mark 0}{(6.037,4.381)}
\gppoint{gp mark 0}{(6.037,4.283)}
\gppoint{gp mark 0}{(6.037,4.327)}
\gppoint{gp mark 0}{(6.037,4.381)}
\gppoint{gp mark 0}{(6.037,4.204)}
\gppoint{gp mark 0}{(6.037,4.381)}
\gppoint{gp mark 0}{(6.037,4.549)}
\gppoint{gp mark 0}{(6.037,4.408)}
\gppoint{gp mark 0}{(6.037,5.010)}
\gppoint{gp mark 0}{(6.037,4.591)}
\gppoint{gp mark 0}{(6.037,4.601)}
\gppoint{gp mark 0}{(6.037,4.312)}
\gppoint{gp mark 0}{(6.037,4.188)}
\gppoint{gp mark 0}{(6.037,4.237)}
\gppoint{gp mark 0}{(6.037,4.381)}
\gppoint{gp mark 0}{(6.037,4.984)}
\gppoint{gp mark 0}{(6.037,4.904)}
\gppoint{gp mark 0}{(6.037,4.433)}
\gppoint{gp mark 0}{(6.037,4.252)}
\gppoint{gp mark 0}{(6.037,4.527)}
\gppoint{gp mark 0}{(6.037,4.188)}
\gppoint{gp mark 0}{(6.037,3.934)}
\gppoint{gp mark 0}{(6.037,4.420)}
\gppoint{gp mark 0}{(6.037,4.408)}
\gppoint{gp mark 0}{(6.037,4.268)}
\gppoint{gp mark 0}{(6.037,4.252)}
\gppoint{gp mark 0}{(6.037,4.188)}
\gppoint{gp mark 0}{(6.037,4.099)}
\gppoint{gp mark 0}{(6.037,4.601)}
\gppoint{gp mark 0}{(6.037,4.118)}
\gppoint{gp mark 0}{(6.037,4.481)}
\gppoint{gp mark 0}{(6.037,4.846)}
\gppoint{gp mark 0}{(6.037,4.651)}
\gppoint{gp mark 0}{(6.037,4.516)}
\gppoint{gp mark 0}{(6.037,4.268)}
\gppoint{gp mark 0}{(6.037,4.570)}
\gppoint{gp mark 0}{(6.037,4.341)}
\gppoint{gp mark 0}{(6.037,4.679)}
\gppoint{gp mark 0}{(6.037,4.469)}
\gppoint{gp mark 0}{(6.037,4.445)}
\gppoint{gp mark 0}{(6.037,4.327)}
\gppoint{gp mark 0}{(6.037,4.570)}
\gppoint{gp mark 0}{(6.037,4.268)}
\gppoint{gp mark 0}{(6.037,4.641)}
\gppoint{gp mark 0}{(6.037,4.153)}
\gppoint{gp mark 0}{(6.037,3.887)}
\gppoint{gp mark 0}{(6.037,4.783)}
\gppoint{gp mark 0}{(6.037,4.268)}
\gppoint{gp mark 0}{(6.037,4.697)}
\gppoint{gp mark 0}{(6.037,4.481)}
\gppoint{gp mark 0}{(6.037,4.791)}
\gppoint{gp mark 0}{(6.037,4.660)}
\gppoint{gp mark 0}{(6.037,4.420)}
\gppoint{gp mark 0}{(6.037,4.268)}
\gppoint{gp mark 0}{(6.037,4.136)}
\gppoint{gp mark 0}{(6.037,4.570)}
\gppoint{gp mark 0}{(6.037,4.283)}
\gppoint{gp mark 0}{(6.037,4.538)}
\gppoint{gp mark 0}{(6.059,4.799)}
\gppoint{gp mark 0}{(6.059,4.420)}
\gppoint{gp mark 0}{(6.059,3.697)}
\gppoint{gp mark 0}{(6.059,4.997)}
\gppoint{gp mark 0}{(6.059,4.327)}
\gppoint{gp mark 0}{(6.059,4.807)}
\gppoint{gp mark 0}{(6.059,4.298)}
\gppoint{gp mark 0}{(6.059,4.581)}
\gppoint{gp mark 0}{(6.059,4.298)}
\gppoint{gp mark 0}{(6.059,4.861)}
\gppoint{gp mark 0}{(6.059,4.408)}
\gppoint{gp mark 0}{(6.059,4.368)}
\gppoint{gp mark 0}{(6.059,4.853)}
\gppoint{gp mark 0}{(6.059,4.408)}
\gppoint{gp mark 0}{(6.059,4.445)}
\gppoint{gp mark 0}{(6.059,4.221)}
\gppoint{gp mark 0}{(6.059,4.560)}
\gppoint{gp mark 0}{(6.059,4.750)}
\gppoint{gp mark 0}{(6.059,4.679)}
\gppoint{gp mark 0}{(6.059,4.612)}
\gppoint{gp mark 0}{(6.059,4.527)}
\gppoint{gp mark 0}{(6.059,4.188)}
\gppoint{gp mark 0}{(6.059,4.327)}
\gppoint{gp mark 0}{(6.059,4.549)}
\gppoint{gp mark 0}{(6.059,4.298)}
\gppoint{gp mark 0}{(6.059,4.549)}
\gppoint{gp mark 0}{(6.059,4.268)}
\gppoint{gp mark 0}{(6.059,4.312)}
\gppoint{gp mark 0}{(6.059,4.204)}
\gppoint{gp mark 0}{(6.059,4.000)}
\gppoint{gp mark 0}{(6.059,4.252)}
\gppoint{gp mark 0}{(6.059,4.469)}
\gppoint{gp mark 0}{(6.059,4.420)}
\gppoint{gp mark 0}{(6.059,4.298)}
\gppoint{gp mark 0}{(6.059,4.481)}
\gppoint{gp mark 0}{(6.059,4.204)}
\gppoint{gp mark 0}{(6.059,5.058)}
\gppoint{gp mark 0}{(6.059,4.591)}
\gppoint{gp mark 0}{(6.059,4.237)}
\gppoint{gp mark 0}{(6.059,4.204)}
\gppoint{gp mark 0}{(6.059,4.153)}
\gppoint{gp mark 0}{(6.059,4.153)}
\gppoint{gp mark 0}{(6.059,4.911)}
\gppoint{gp mark 0}{(6.059,4.061)}
\gppoint{gp mark 0}{(6.059,3.957)}
\gppoint{gp mark 0}{(6.059,4.504)}
\gppoint{gp mark 0}{(6.059,4.381)}
\gppoint{gp mark 0}{(6.059,4.601)}
\gppoint{gp mark 0}{(6.059,4.420)}
\gppoint{gp mark 0}{(6.059,4.612)}
\gppoint{gp mark 0}{(6.059,4.621)}
\gppoint{gp mark 0}{(6.059,4.268)}
\gppoint{gp mark 0}{(6.059,4.099)}
\gppoint{gp mark 0}{(6.059,4.538)}
\gppoint{gp mark 0}{(6.059,4.457)}
\gppoint{gp mark 0}{(6.059,4.153)}
\gppoint{gp mark 0}{(6.059,4.445)}
\gppoint{gp mark 0}{(6.059,4.327)}
\gppoint{gp mark 0}{(6.059,4.815)}
\gppoint{gp mark 0}{(6.059,4.527)}
\gppoint{gp mark 0}{(6.059,4.298)}
\gppoint{gp mark 0}{(6.059,4.204)}
\gppoint{gp mark 0}{(6.059,4.298)}
\gppoint{gp mark 0}{(6.059,4.897)}
\gppoint{gp mark 0}{(6.059,4.341)}
\gppoint{gp mark 0}{(6.059,4.021)}
\gppoint{gp mark 0}{(6.059,4.481)}
\gppoint{gp mark 0}{(6.059,4.868)}
\gppoint{gp mark 0}{(6.059,4.381)}
\gppoint{gp mark 0}{(6.059,4.136)}
\gppoint{gp mark 0}{(6.059,4.733)}
\gppoint{gp mark 0}{(6.059,4.395)}
\gppoint{gp mark 0}{(6.059,4.581)}
\gppoint{gp mark 0}{(6.059,4.733)}
\gppoint{gp mark 0}{(6.059,4.341)}
\gppoint{gp mark 0}{(6.059,4.341)}
\gppoint{gp mark 0}{(6.059,4.679)}
\gppoint{gp mark 0}{(6.080,4.493)}
\gppoint{gp mark 0}{(6.080,4.504)}
\gppoint{gp mark 0}{(6.080,4.445)}
\gppoint{gp mark 0}{(6.080,4.252)}
\gppoint{gp mark 0}{(6.080,4.061)}
\gppoint{gp mark 0}{(6.080,4.952)}
\gppoint{gp mark 0}{(6.080,4.601)}
\gppoint{gp mark 0}{(6.080,4.268)}
\gppoint{gp mark 0}{(6.080,4.408)}
\gppoint{gp mark 0}{(6.080,4.252)}
\gppoint{gp mark 0}{(6.080,4.504)}
\gppoint{gp mark 0}{(6.080,4.457)}
\gppoint{gp mark 0}{(6.080,4.252)}
\gppoint{gp mark 0}{(6.080,4.420)}
\gppoint{gp mark 0}{(6.080,4.327)}
\gppoint{gp mark 0}{(6.080,4.741)}
\gppoint{gp mark 0}{(6.080,4.591)}
\gppoint{gp mark 0}{(6.080,4.298)}
\gppoint{gp mark 0}{(6.080,4.283)}
\gppoint{gp mark 0}{(6.080,5.081)}
\gppoint{gp mark 0}{(6.080,4.549)}
\gppoint{gp mark 0}{(6.080,4.252)}
\gppoint{gp mark 0}{(6.080,4.601)}
\gppoint{gp mark 0}{(6.080,4.341)}
\gppoint{gp mark 0}{(6.080,4.408)}
\gppoint{gp mark 0}{(6.080,4.688)}
\gppoint{gp mark 0}{(6.080,4.188)}
\gppoint{gp mark 0}{(6.080,4.469)}
\gppoint{gp mark 0}{(6.080,4.570)}
\gppoint{gp mark 0}{(6.080,3.957)}
\gppoint{gp mark 0}{(6.080,4.868)}
\gppoint{gp mark 0}{(6.080,4.868)}
\gppoint{gp mark 0}{(6.080,4.171)}
\gppoint{gp mark 0}{(6.080,4.395)}
\gppoint{gp mark 0}{(6.080,4.080)}
\gppoint{gp mark 0}{(6.080,4.283)}
\gppoint{gp mark 0}{(6.080,4.481)}
\gppoint{gp mark 0}{(6.080,4.252)}
\gppoint{gp mark 0}{(6.080,4.846)}
\gppoint{gp mark 0}{(6.080,4.469)}
\gppoint{gp mark 0}{(6.080,4.846)}
\gppoint{gp mark 0}{(6.080,4.171)}
\gppoint{gp mark 0}{(6.080,4.080)}
\gppoint{gp mark 0}{(6.080,4.221)}
\gppoint{gp mark 0}{(6.080,4.153)}
\gppoint{gp mark 0}{(6.080,3.979)}
\gppoint{gp mark 0}{(6.080,4.688)}
\gppoint{gp mark 0}{(6.080,4.041)}
\gppoint{gp mark 0}{(6.080,4.381)}
\gppoint{gp mark 0}{(6.080,4.601)}
\gppoint{gp mark 0}{(6.080,4.433)}
\gppoint{gp mark 0}{(6.080,4.136)}
\gppoint{gp mark 0}{(6.080,4.395)}
\gppoint{gp mark 0}{(6.080,4.312)}
\gppoint{gp mark 0}{(6.080,4.395)}
\gppoint{gp mark 0}{(6.080,4.153)}
\gppoint{gp mark 0}{(6.080,4.188)}
\gppoint{gp mark 0}{(6.080,4.080)}
\gppoint{gp mark 0}{(6.080,4.204)}
\gppoint{gp mark 0}{(6.080,4.327)}
\gppoint{gp mark 0}{(6.080,4.171)}
\gppoint{gp mark 0}{(6.080,4.445)}
\gppoint{gp mark 0}{(6.101,3.957)}
\gppoint{gp mark 0}{(6.101,4.237)}
\gppoint{gp mark 0}{(6.101,4.341)}
\gppoint{gp mark 0}{(6.101,4.651)}
\gppoint{gp mark 0}{(6.101,4.651)}
\gppoint{gp mark 0}{(6.101,4.591)}
\gppoint{gp mark 0}{(6.101,4.679)}
\gppoint{gp mark 0}{(6.101,4.775)}
\gppoint{gp mark 0}{(6.101,4.570)}
\gppoint{gp mark 0}{(6.101,4.830)}
\gppoint{gp mark 0}{(6.101,4.395)}
\gppoint{gp mark 0}{(6.101,4.298)}
\gppoint{gp mark 0}{(6.101,4.516)}
\gppoint{gp mark 0}{(6.101,4.099)}
\gppoint{gp mark 0}{(6.101,4.433)}
\gppoint{gp mark 0}{(6.101,4.538)}
\gppoint{gp mark 0}{(6.101,4.581)}
\gppoint{gp mark 0}{(6.101,4.312)}
\gppoint{gp mark 0}{(6.101,4.621)}
\gppoint{gp mark 0}{(6.101,4.641)}
\gppoint{gp mark 0}{(6.101,3.979)}
\gppoint{gp mark 0}{(6.101,4.733)}
\gppoint{gp mark 0}{(6.101,4.481)}
\gppoint{gp mark 0}{(6.101,4.651)}
\gppoint{gp mark 0}{(6.101,4.670)}
\gppoint{gp mark 0}{(6.101,4.591)}
\gppoint{gp mark 0}{(6.101,4.972)}
\gppoint{gp mark 0}{(6.101,4.641)}
\gppoint{gp mark 0}{(6.101,4.354)}
\gppoint{gp mark 0}{(6.101,4.283)}
\gppoint{gp mark 0}{(6.101,4.408)}
\gppoint{gp mark 0}{(6.101,4.815)}
\gppoint{gp mark 0}{(6.101,4.408)}
\gppoint{gp mark 0}{(6.101,4.670)}
\gppoint{gp mark 0}{(6.101,4.651)}
\gppoint{gp mark 0}{(6.101,4.341)}
\gppoint{gp mark 0}{(6.101,4.457)}
\gppoint{gp mark 0}{(6.101,4.368)}
\gppoint{gp mark 0}{(6.101,4.945)}
\gppoint{gp mark 0}{(6.101,4.433)}
\gppoint{gp mark 0}{(6.101,4.481)}
\gppoint{gp mark 0}{(6.101,4.897)}
\gppoint{gp mark 0}{(6.101,4.481)}
\gppoint{gp mark 0}{(6.101,4.327)}
\gppoint{gp mark 0}{(6.101,4.252)}
\gppoint{gp mark 0}{(6.101,3.979)}
\gppoint{gp mark 0}{(6.101,4.972)}
\gppoint{gp mark 0}{(6.101,4.118)}
\gppoint{gp mark 0}{(6.101,4.651)}
\gppoint{gp mark 0}{(6.101,4.368)}
\gppoint{gp mark 0}{(6.101,4.481)}
\gppoint{gp mark 0}{(6.101,3.030)}
\gppoint{gp mark 0}{(6.101,3.030)}
\gppoint{gp mark 0}{(6.101,4.527)}
\gppoint{gp mark 0}{(6.101,5.167)}
\gppoint{gp mark 0}{(6.101,4.283)}
\gppoint{gp mark 0}{(6.101,4.493)}
\gppoint{gp mark 0}{(6.101,4.538)}
\gppoint{gp mark 0}{(6.101,4.651)}
\gppoint{gp mark 0}{(6.101,4.136)}
\gppoint{gp mark 0}{(6.101,4.481)}
\gppoint{gp mark 0}{(6.101,4.527)}
\gppoint{gp mark 0}{(6.101,4.591)}
\gppoint{gp mark 0}{(6.101,4.469)}
\gppoint{gp mark 0}{(6.101,4.283)}
\gppoint{gp mark 0}{(6.101,4.298)}
\gppoint{gp mark 0}{(6.101,4.420)}
\gppoint{gp mark 0}{(6.121,4.612)}
\gppoint{gp mark 0}{(6.121,4.830)}
\gppoint{gp mark 0}{(6.121,4.171)}
\gppoint{gp mark 0}{(6.121,4.570)}
\gppoint{gp mark 0}{(6.121,5.114)}
\gppoint{gp mark 0}{(6.121,4.408)}
\gppoint{gp mark 0}{(6.121,4.783)}
\gppoint{gp mark 0}{(6.121,4.433)}
\gppoint{gp mark 0}{(6.121,4.641)}
\gppoint{gp mark 0}{(6.121,4.965)}
\gppoint{gp mark 0}{(6.121,5.281)}
\gppoint{gp mark 0}{(6.121,4.527)}
\gppoint{gp mark 0}{(6.121,4.527)}
\gppoint{gp mark 0}{(6.121,4.660)}
\gppoint{gp mark 0}{(6.121,4.041)}
\gppoint{gp mark 0}{(6.121,4.651)}
\gppoint{gp mark 0}{(6.121,4.679)}
\gppoint{gp mark 0}{(6.121,4.504)}
\gppoint{gp mark 0}{(6.121,4.724)}
\gppoint{gp mark 0}{(6.121,4.041)}
\gppoint{gp mark 0}{(6.121,4.221)}
\gppoint{gp mark 0}{(6.121,4.493)}
\gppoint{gp mark 0}{(6.121,4.099)}
\gppoint{gp mark 0}{(6.121,4.549)}
\gppoint{gp mark 0}{(6.121,4.932)}
\gppoint{gp mark 0}{(6.121,4.457)}
\gppoint{gp mark 0}{(6.121,4.897)}
\gppoint{gp mark 0}{(6.121,4.099)}
\gppoint{gp mark 0}{(6.121,4.904)}
\gppoint{gp mark 0}{(6.121,4.560)}
\gppoint{gp mark 0}{(6.121,4.504)}
\gppoint{gp mark 0}{(6.121,4.861)}
\gppoint{gp mark 0}{(6.121,4.612)}
\gppoint{gp mark 0}{(6.121,4.457)}
\gppoint{gp mark 0}{(6.121,4.660)}
\gppoint{gp mark 0}{(6.121,4.433)}
\gppoint{gp mark 0}{(6.121,4.327)}
\gppoint{gp mark 0}{(6.121,4.457)}
\gppoint{gp mark 0}{(6.121,4.481)}
\gppoint{gp mark 0}{(6.121,4.861)}
\gppoint{gp mark 0}{(6.121,4.516)}
\gppoint{gp mark 0}{(6.121,4.516)}
\gppoint{gp mark 0}{(6.121,4.516)}
\gppoint{gp mark 0}{(6.121,4.750)}
\gppoint{gp mark 0}{(6.121,4.560)}
\gppoint{gp mark 0}{(6.121,4.631)}
\gppoint{gp mark 0}{(6.121,4.504)}
\gppoint{gp mark 0}{(6.121,4.791)}
\gppoint{gp mark 0}{(6.121,4.591)}
\gppoint{gp mark 0}{(6.121,4.433)}
\gppoint{gp mark 0}{(6.121,4.395)}
\gppoint{gp mark 0}{(6.121,4.750)}
\gppoint{gp mark 0}{(6.121,4.283)}
\gppoint{gp mark 0}{(6.121,4.591)}
\gppoint{gp mark 0}{(6.121,4.984)}
\gppoint{gp mark 0}{(6.121,4.538)}
\gppoint{gp mark 0}{(6.121,3.979)}
\gppoint{gp mark 0}{(6.121,4.591)}
\gppoint{gp mark 0}{(6.121,4.481)}
\gppoint{gp mark 0}{(6.121,4.591)}
\gppoint{gp mark 0}{(6.121,4.327)}
\gppoint{gp mark 0}{(6.121,4.395)}
\gppoint{gp mark 0}{(6.121,4.341)}
\gppoint{gp mark 0}{(6.121,4.395)}
\gppoint{gp mark 0}{(6.121,4.221)}
\gppoint{gp mark 0}{(6.121,4.538)}
\gppoint{gp mark 0}{(6.121,4.395)}
\gppoint{gp mark 0}{(6.121,4.538)}
\gppoint{gp mark 0}{(6.142,4.679)}
\gppoint{gp mark 0}{(6.142,4.549)}
\gppoint{gp mark 0}{(6.142,4.660)}
\gppoint{gp mark 0}{(6.142,4.395)}
\gppoint{gp mark 0}{(6.142,4.706)}
\gppoint{gp mark 0}{(6.142,4.395)}
\gppoint{gp mark 0}{(6.142,4.433)}
\gppoint{gp mark 0}{(6.142,4.697)}
\gppoint{gp mark 0}{(6.142,4.679)}
\gppoint{gp mark 0}{(6.142,4.538)}
\gppoint{gp mark 0}{(6.142,4.538)}
\gppoint{gp mark 0}{(6.142,4.327)}
\gppoint{gp mark 0}{(6.142,4.775)}
\gppoint{gp mark 0}{(6.142,4.679)}
\gppoint{gp mark 0}{(6.142,4.021)}
\gppoint{gp mark 0}{(6.142,4.481)}
\gppoint{gp mark 0}{(6.142,4.395)}
\gppoint{gp mark 0}{(6.142,4.897)}
\gppoint{gp mark 0}{(6.142,4.601)}
\gppoint{gp mark 0}{(6.142,4.688)}
\gppoint{gp mark 0}{(6.142,4.188)}
\gppoint{gp mark 0}{(6.142,4.298)}
\gppoint{gp mark 0}{(6.142,4.298)}
\gppoint{gp mark 0}{(6.142,4.298)}
\gppoint{gp mark 0}{(6.142,4.408)}
\gppoint{gp mark 0}{(6.142,4.327)}
\gppoint{gp mark 0}{(6.142,4.775)}
\gppoint{gp mark 0}{(6.142,4.631)}
\gppoint{gp mark 0}{(6.142,4.504)}
\gppoint{gp mark 0}{(6.142,4.897)}
\gppoint{gp mark 0}{(6.142,4.997)}
\gppoint{gp mark 0}{(6.142,4.791)}
\gppoint{gp mark 0}{(6.142,4.651)}
\gppoint{gp mark 0}{(6.142,4.538)}
\gppoint{gp mark 0}{(6.142,4.000)}
\gppoint{gp mark 0}{(6.142,4.724)}
\gppoint{gp mark 0}{(6.142,4.516)}
\gppoint{gp mark 0}{(6.142,4.612)}
\gppoint{gp mark 0}{(6.142,4.549)}
\gppoint{gp mark 0}{(6.142,4.021)}
\gppoint{gp mark 0}{(6.142,4.883)}
\gppoint{gp mark 0}{(6.142,4.469)}
\gppoint{gp mark 0}{(6.142,4.395)}
\gppoint{gp mark 0}{(6.142,4.136)}
\gppoint{gp mark 0}{(6.142,4.312)}
\gppoint{gp mark 0}{(6.142,4.493)}
\gppoint{gp mark 0}{(6.142,4.697)}
\gppoint{gp mark 0}{(6.142,4.724)}
\gppoint{gp mark 0}{(6.142,4.408)}
\gppoint{gp mark 0}{(6.142,4.395)}
\gppoint{gp mark 0}{(6.142,4.298)}
\gppoint{gp mark 0}{(6.142,4.136)}
\gppoint{gp mark 0}{(6.142,4.395)}
\gppoint{gp mark 0}{(6.142,4.846)}
\gppoint{gp mark 0}{(6.142,4.581)}
\gppoint{gp mark 0}{(6.142,4.204)}
\gppoint{gp mark 0}{(6.142,4.408)}
\gppoint{gp mark 0}{(6.142,4.918)}
\gppoint{gp mark 0}{(6.142,4.679)}
\gppoint{gp mark 0}{(6.142,4.767)}
\gppoint{gp mark 0}{(6.142,4.252)}
\gppoint{gp mark 0}{(6.142,4.504)}
\gppoint{gp mark 0}{(6.142,4.758)}
\gppoint{gp mark 0}{(6.142,4.171)}
\gppoint{gp mark 0}{(6.142,4.670)}
\gppoint{gp mark 0}{(6.142,4.433)}
\gppoint{gp mark 0}{(6.142,3.887)}
\gppoint{gp mark 0}{(6.142,4.341)}
\gppoint{gp mark 0}{(6.142,4.469)}
\gppoint{gp mark 0}{(6.142,4.560)}
\gppoint{gp mark 0}{(6.142,4.538)}
\gppoint{gp mark 0}{(6.142,4.341)}
\gppoint{gp mark 0}{(6.142,4.560)}
\gppoint{gp mark 0}{(6.142,4.368)}
\gppoint{gp mark 0}{(6.142,4.601)}
\gppoint{gp mark 0}{(6.142,4.741)}
\gppoint{gp mark 0}{(6.142,5.268)}
\gppoint{gp mark 0}{(6.161,4.861)}
\gppoint{gp mark 0}{(6.161,4.354)}
\gppoint{gp mark 0}{(6.161,4.538)}
\gppoint{gp mark 0}{(6.161,4.354)}
\gppoint{gp mark 0}{(6.161,4.312)}
\gppoint{gp mark 0}{(6.161,4.758)}
\gppoint{gp mark 0}{(6.161,4.469)}
\gppoint{gp mark 0}{(6.161,4.381)}
\gppoint{gp mark 0}{(6.161,4.612)}
\gppoint{gp mark 0}{(6.161,4.298)}
\gppoint{gp mark 0}{(6.161,5.063)}
\gppoint{gp mark 0}{(6.161,4.469)}
\gppoint{gp mark 0}{(6.161,4.188)}
\gppoint{gp mark 0}{(6.161,4.504)}
\gppoint{gp mark 0}{(6.161,4.504)}
\gppoint{gp mark 0}{(6.161,4.221)}
\gppoint{gp mark 0}{(6.161,4.298)}
\gppoint{gp mark 0}{(6.161,4.679)}
\gppoint{gp mark 0}{(6.161,4.581)}
\gppoint{gp mark 0}{(6.161,4.381)}
\gppoint{gp mark 0}{(6.161,4.679)}
\gppoint{gp mark 0}{(6.161,4.445)}
\gppoint{gp mark 0}{(6.161,4.601)}
\gppoint{gp mark 0}{(6.161,4.445)}
\gppoint{gp mark 0}{(6.161,4.679)}
\gppoint{gp mark 0}{(6.161,4.679)}
\gppoint{gp mark 0}{(6.161,4.527)}
\gppoint{gp mark 0}{(6.161,4.741)}
\gppoint{gp mark 0}{(6.161,4.381)}
\gppoint{gp mark 0}{(6.161,4.420)}
\gppoint{gp mark 0}{(6.161,4.118)}
\gppoint{gp mark 0}{(6.161,4.221)}
\gppoint{gp mark 0}{(6.161,4.188)}
\gppoint{gp mark 0}{(6.161,4.188)}
\gppoint{gp mark 0}{(6.161,4.252)}
\gppoint{gp mark 0}{(6.161,4.741)}
\gppoint{gp mark 0}{(6.161,4.750)}
\gppoint{gp mark 0}{(6.161,4.560)}
\gppoint{gp mark 0}{(6.161,4.670)}
\gppoint{gp mark 0}{(6.161,4.601)}
\gppoint{gp mark 0}{(6.161,4.853)}
\gppoint{gp mark 0}{(6.161,4.493)}
\gppoint{gp mark 0}{(6.161,4.354)}
\gppoint{gp mark 0}{(6.161,4.221)}
\gppoint{gp mark 0}{(6.161,4.706)}
\gppoint{gp mark 0}{(6.161,4.631)}
\gppoint{gp mark 0}{(6.161,4.799)}
\gppoint{gp mark 0}{(6.161,4.670)}
\gppoint{gp mark 0}{(6.161,4.846)}
\gppoint{gp mark 0}{(6.161,4.368)}
\gppoint{gp mark 0}{(6.161,4.368)}
\gppoint{gp mark 0}{(6.161,4.688)}
\gppoint{gp mark 0}{(6.161,4.368)}
\gppoint{gp mark 0}{(6.161,4.697)}
\gppoint{gp mark 0}{(6.161,4.368)}
\gppoint{gp mark 0}{(6.161,4.621)}
\gppoint{gp mark 0}{(6.161,4.354)}
\gppoint{gp mark 0}{(6.161,4.591)}
\gppoint{gp mark 0}{(6.161,4.504)}
\gppoint{gp mark 0}{(6.161,4.570)}
\gppoint{gp mark 0}{(6.161,4.504)}
\gppoint{gp mark 0}{(6.161,4.641)}
\gppoint{gp mark 0}{(6.161,4.549)}
\gppoint{gp mark 0}{(6.161,4.830)}
\gppoint{gp mark 0}{(6.181,4.733)}
\gppoint{gp mark 0}{(6.181,3.364)}
\gppoint{gp mark 0}{(6.181,4.631)}
\gppoint{gp mark 0}{(6.181,4.252)}
\gppoint{gp mark 0}{(6.181,4.327)}
\gppoint{gp mark 0}{(6.181,4.581)}
\gppoint{gp mark 0}{(6.181,4.457)}
\gppoint{gp mark 0}{(6.181,4.516)}
\gppoint{gp mark 0}{(6.181,4.457)}
\gppoint{gp mark 0}{(6.181,4.504)}
\gppoint{gp mark 0}{(6.181,4.549)}
\gppoint{gp mark 0}{(6.181,4.697)}
\gppoint{gp mark 0}{(6.181,4.538)}
\gppoint{gp mark 0}{(6.181,5.052)}
\gppoint{gp mark 0}{(6.181,5.075)}
\gppoint{gp mark 0}{(6.181,4.283)}
\gppoint{gp mark 0}{(6.181,4.670)}
\gppoint{gp mark 0}{(6.181,4.631)}
\gppoint{gp mark 0}{(6.181,4.420)}
\gppoint{gp mark 0}{(6.181,4.697)}
\gppoint{gp mark 0}{(6.181,4.457)}
\gppoint{gp mark 0}{(6.181,4.631)}
\gppoint{gp mark 0}{(6.181,4.504)}
\gppoint{gp mark 0}{(6.181,4.697)}
\gppoint{gp mark 0}{(6.181,4.549)}
\gppoint{gp mark 0}{(6.181,5.263)}
\gppoint{gp mark 0}{(6.181,4.697)}
\gppoint{gp mark 0}{(6.181,4.591)}
\gppoint{gp mark 0}{(6.181,4.408)}
\gppoint{gp mark 0}{(6.181,4.697)}
\gppoint{gp mark 0}{(6.181,4.612)}
\gppoint{gp mark 0}{(6.181,3.784)}
\gppoint{gp mark 0}{(6.181,3.957)}
\gppoint{gp mark 0}{(6.181,4.591)}
\gppoint{gp mark 0}{(6.181,4.911)}
\gppoint{gp mark 0}{(6.181,4.549)}
\gppoint{gp mark 0}{(6.181,4.697)}
\gppoint{gp mark 0}{(6.181,4.925)}
\gppoint{gp mark 0}{(6.181,4.237)}
\gppoint{gp mark 0}{(6.181,5.151)}
\gppoint{gp mark 0}{(6.181,4.341)}
\gppoint{gp mark 0}{(6.181,4.724)}
\gppoint{gp mark 0}{(6.181,4.204)}
\gppoint{gp mark 0}{(6.181,5.052)}
\gppoint{gp mark 0}{(6.181,4.697)}
\gppoint{gp mark 0}{(6.181,4.021)}
\gppoint{gp mark 0}{(6.181,4.758)}
\gppoint{gp mark 0}{(6.181,4.420)}
\gppoint{gp mark 0}{(6.181,4.493)}
\gppoint{gp mark 0}{(6.181,4.651)}
\gppoint{gp mark 0}{(6.181,4.660)}
\gppoint{gp mark 0}{(6.181,4.660)}
\gppoint{gp mark 0}{(6.181,4.641)}
\gppoint{gp mark 0}{(6.181,4.408)}
\gppoint{gp mark 0}{(6.181,4.312)}
\gppoint{gp mark 0}{(6.181,4.341)}
\gppoint{gp mark 0}{(6.181,4.395)}
\gppoint{gp mark 0}{(6.181,4.283)}
\gppoint{gp mark 0}{(6.181,4.560)}
\gppoint{gp mark 0}{(6.181,4.153)}
\gppoint{gp mark 0}{(6.181,4.298)}
\gppoint{gp mark 0}{(6.181,4.341)}
\gppoint{gp mark 0}{(6.181,4.298)}
\gppoint{gp mark 0}{(6.181,4.420)}
\gppoint{gp mark 0}{(6.181,4.897)}
\gppoint{gp mark 0}{(6.181,4.327)}
\gppoint{gp mark 0}{(6.181,4.621)}
\gppoint{gp mark 0}{(6.181,4.118)}
\gppoint{gp mark 0}{(6.200,4.204)}
\gppoint{gp mark 0}{(6.200,4.679)}
\gppoint{gp mark 0}{(6.200,3.887)}
\gppoint{gp mark 0}{(6.200,4.381)}
\gppoint{gp mark 0}{(6.200,4.327)}
\gppoint{gp mark 0}{(6.200,4.750)}
\gppoint{gp mark 0}{(6.200,4.408)}
\gppoint{gp mark 0}{(6.200,4.601)}
\gppoint{gp mark 0}{(6.200,4.283)}
\gppoint{gp mark 0}{(6.200,4.641)}
\gppoint{gp mark 0}{(6.200,4.807)}
\gppoint{gp mark 0}{(6.200,4.621)}
\gppoint{gp mark 0}{(6.200,4.381)}
\gppoint{gp mark 0}{(6.200,4.381)}
\gppoint{gp mark 0}{(6.200,4.791)}
\gppoint{gp mark 0}{(6.200,4.354)}
\gppoint{gp mark 0}{(6.200,4.697)}
\gppoint{gp mark 0}{(6.200,4.381)}
\gppoint{gp mark 0}{(6.200,4.741)}
\gppoint{gp mark 0}{(6.200,4.420)}
\gppoint{gp mark 0}{(6.200,4.283)}
\gppoint{gp mark 0}{(6.200,4.445)}
\gppoint{gp mark 0}{(6.200,4.660)}
\gppoint{gp mark 0}{(6.200,4.433)}
\gppoint{gp mark 0}{(6.200,4.538)}
\gppoint{gp mark 0}{(6.200,4.952)}
\gppoint{gp mark 0}{(6.200,4.237)}
\gppoint{gp mark 0}{(6.200,4.783)}
\gppoint{gp mark 0}{(6.200,5.162)}
\gppoint{gp mark 0}{(6.200,4.846)}
\gppoint{gp mark 0}{(6.200,4.830)}
\gppoint{gp mark 0}{(6.200,4.861)}
\gppoint{gp mark 0}{(6.200,4.679)}
\gppoint{gp mark 0}{(6.200,4.298)}
\gppoint{gp mark 0}{(6.200,4.688)}
\gppoint{gp mark 0}{(6.200,4.469)}
\gppoint{gp mark 0}{(6.200,4.527)}
\gppoint{gp mark 0}{(6.200,4.918)}
\gppoint{gp mark 0}{(6.200,4.549)}
\gppoint{gp mark 0}{(6.200,4.853)}
\gppoint{gp mark 0}{(6.200,4.560)}
\gppoint{gp mark 0}{(6.200,4.481)}
\gppoint{gp mark 0}{(6.200,4.641)}
\gppoint{gp mark 0}{(6.200,4.298)}
\gppoint{gp mark 0}{(6.200,4.395)}
\gppoint{gp mark 0}{(6.200,4.570)}
\gppoint{gp mark 0}{(6.200,4.136)}
\gppoint{gp mark 0}{(6.200,4.433)}
\gppoint{gp mark 0}{(6.200,4.368)}
\gppoint{gp mark 0}{(6.200,4.724)}
\gppoint{gp mark 0}{(6.200,4.368)}
\gppoint{gp mark 0}{(6.200,4.368)}
\gppoint{gp mark 0}{(6.200,4.368)}
\gppoint{gp mark 0}{(6.200,4.560)}
\gppoint{gp mark 0}{(6.200,4.395)}
\gppoint{gp mark 0}{(6.200,3.030)}
\gppoint{gp mark 0}{(6.200,4.591)}
\gppoint{gp mark 0}{(6.200,4.136)}
\gppoint{gp mark 0}{(6.200,4.408)}
\gppoint{gp mark 0}{(6.200,4.984)}
\gppoint{gp mark 0}{(6.200,4.660)}
\gppoint{gp mark 0}{(6.200,4.972)}
\gppoint{gp mark 0}{(6.200,4.758)}
\gppoint{gp mark 0}{(6.200,4.641)}
\gppoint{gp mark 0}{(6.200,4.527)}
\gppoint{gp mark 0}{(6.200,4.445)}
\gppoint{gp mark 0}{(6.200,4.660)}
\gppoint{gp mark 0}{(6.200,4.395)}
\gppoint{gp mark 0}{(6.200,4.830)}
\gppoint{gp mark 0}{(6.200,4.830)}
\gppoint{gp mark 0}{(6.200,4.538)}
\gppoint{gp mark 0}{(6.219,4.538)}
\gppoint{gp mark 0}{(6.219,4.354)}
\gppoint{gp mark 0}{(6.219,4.099)}
\gppoint{gp mark 0}{(6.219,4.758)}
\gppoint{gp mark 0}{(6.219,5.003)}
\gppoint{gp mark 0}{(6.219,4.549)}
\gppoint{gp mark 0}{(6.219,4.341)}
\gppoint{gp mark 0}{(6.219,4.750)}
\gppoint{gp mark 0}{(6.219,4.688)}
\gppoint{gp mark 0}{(6.219,4.504)}
\gppoint{gp mark 0}{(6.219,4.469)}
\gppoint{gp mark 0}{(6.219,4.549)}
\gppoint{gp mark 0}{(6.219,4.221)}
\gppoint{gp mark 0}{(6.219,4.688)}
\gppoint{gp mark 0}{(6.219,4.354)}
\gppoint{gp mark 0}{(6.219,4.591)}
\gppoint{gp mark 0}{(6.219,4.641)}
\gppoint{gp mark 0}{(6.219,4.420)}
\gppoint{gp mark 0}{(6.219,4.783)}
\gppoint{gp mark 0}{(6.219,4.612)}
\gppoint{gp mark 0}{(6.219,4.504)}
\gppoint{gp mark 0}{(6.219,4.750)}
\gppoint{gp mark 0}{(6.219,4.591)}
\gppoint{gp mark 0}{(6.219,4.799)}
\gppoint{gp mark 0}{(6.219,4.493)}
\gppoint{gp mark 0}{(6.219,4.538)}
\gppoint{gp mark 0}{(6.219,4.724)}
\gppoint{gp mark 0}{(6.219,3.887)}
\gppoint{gp mark 0}{(6.219,4.660)}
\gppoint{gp mark 0}{(6.219,5.069)}
\gppoint{gp mark 0}{(6.219,4.188)}
\gppoint{gp mark 0}{(6.219,4.408)}
\gppoint{gp mark 0}{(6.219,4.493)}
\gppoint{gp mark 0}{(6.219,4.868)}
\gppoint{gp mark 0}{(6.219,4.890)}
\gppoint{gp mark 0}{(6.219,4.341)}
\gppoint{gp mark 0}{(6.219,4.724)}
\gppoint{gp mark 0}{(6.219,4.581)}
\gppoint{gp mark 0}{(6.219,4.469)}
\gppoint{gp mark 0}{(6.219,4.283)}
\gppoint{gp mark 0}{(6.219,4.932)}
\gppoint{gp mark 0}{(6.219,4.395)}
\gppoint{gp mark 0}{(6.219,4.538)}
\gppoint{gp mark 0}{(6.219,4.984)}
\gppoint{gp mark 0}{(6.219,4.660)}
\gppoint{gp mark 0}{(6.219,4.938)}
\gppoint{gp mark 0}{(6.219,4.469)}
\gppoint{gp mark 0}{(6.219,4.354)}
\gppoint{gp mark 0}{(6.219,4.670)}
\gppoint{gp mark 0}{(6.219,4.136)}
\gppoint{gp mark 0}{(6.219,4.830)}
\gppoint{gp mark 0}{(6.219,4.408)}
\gppoint{gp mark 0}{(6.219,4.493)}
\gppoint{gp mark 0}{(6.219,4.493)}
\gppoint{gp mark 0}{(6.219,4.395)}
\gppoint{gp mark 0}{(6.219,4.408)}
\gppoint{gp mark 0}{(6.219,4.660)}
\gppoint{gp mark 0}{(6.238,4.581)}
\gppoint{gp mark 0}{(6.238,4.741)}
\gppoint{gp mark 0}{(6.238,4.469)}
\gppoint{gp mark 0}{(6.238,4.204)}
\gppoint{gp mark 0}{(6.238,4.846)}
\gppoint{gp mark 0}{(6.238,4.516)}
\gppoint{gp mark 0}{(6.238,4.420)}
\gppoint{gp mark 0}{(6.238,4.493)}
\gppoint{gp mark 0}{(6.238,5.182)}
\gppoint{gp mark 0}{(6.238,4.741)}
\gppoint{gp mark 0}{(6.238,4.875)}
\gppoint{gp mark 0}{(6.238,4.457)}
\gppoint{gp mark 0}{(6.238,4.381)}
\gppoint{gp mark 0}{(6.238,4.679)}
\gppoint{gp mark 0}{(6.238,4.775)}
\gppoint{gp mark 0}{(6.238,4.783)}
\gppoint{gp mark 0}{(6.238,4.715)}
\gppoint{gp mark 0}{(6.238,4.570)}
\gppoint{gp mark 0}{(6.238,4.381)}
\gppoint{gp mark 0}{(6.238,4.875)}
\gppoint{gp mark 0}{(6.238,4.457)}
\gppoint{gp mark 0}{(6.238,4.631)}
\gppoint{gp mark 0}{(6.238,4.861)}
\gppoint{gp mark 0}{(6.238,4.670)}
\gppoint{gp mark 0}{(6.238,4.354)}
\gppoint{gp mark 0}{(6.238,4.601)}
\gppoint{gp mark 0}{(6.238,4.612)}
\gppoint{gp mark 0}{(6.238,4.670)}
\gppoint{gp mark 0}{(6.238,4.799)}
\gppoint{gp mark 0}{(6.238,4.651)}
\gppoint{gp mark 0}{(6.238,4.445)}
\gppoint{gp mark 0}{(6.238,4.660)}
\gppoint{gp mark 0}{(6.238,4.457)}
\gppoint{gp mark 0}{(6.238,4.660)}
\gppoint{gp mark 0}{(6.238,4.706)}
\gppoint{gp mark 0}{(6.238,4.560)}
\gppoint{gp mark 0}{(6.238,5.022)}
\gppoint{gp mark 0}{(6.238,4.897)}
\gppoint{gp mark 0}{(6.238,4.327)}
\gppoint{gp mark 0}{(6.238,4.679)}
\gppoint{gp mark 0}{(6.238,4.890)}
\gppoint{gp mark 0}{(6.238,4.457)}
\gppoint{gp mark 0}{(6.238,4.581)}
\gppoint{gp mark 0}{(6.238,4.021)}
\gppoint{gp mark 0}{(6.238,4.516)}
\gppoint{gp mark 0}{(6.238,4.560)}
\gppoint{gp mark 0}{(6.238,4.612)}
\gppoint{gp mark 0}{(6.238,4.457)}
\gppoint{gp mark 0}{(6.238,4.767)}
\gppoint{gp mark 0}{(6.238,4.283)}
\gppoint{gp mark 0}{(6.238,4.846)}
\gppoint{gp mark 0}{(6.238,4.799)}
\gppoint{gp mark 0}{(6.238,5.069)}
\gppoint{gp mark 0}{(6.256,4.527)}
\gppoint{gp mark 0}{(6.256,4.758)}
\gppoint{gp mark 0}{(6.256,4.204)}
\gppoint{gp mark 0}{(6.256,4.601)}
\gppoint{gp mark 0}{(6.256,4.354)}
\gppoint{gp mark 0}{(6.256,4.204)}
\gppoint{gp mark 0}{(6.256,4.775)}
\gppoint{gp mark 0}{(6.256,4.516)}
\gppoint{gp mark 0}{(6.256,4.481)}
\gppoint{gp mark 0}{(6.256,4.715)}
\gppoint{gp mark 0}{(6.256,4.493)}
\gppoint{gp mark 0}{(6.256,4.327)}
\gppoint{gp mark 0}{(6.256,4.469)}
\gppoint{gp mark 0}{(6.256,4.538)}
\gppoint{gp mark 0}{(6.256,4.204)}
\gppoint{gp mark 0}{(6.256,5.182)}
\gppoint{gp mark 0}{(6.256,4.651)}
\gppoint{gp mark 0}{(6.256,4.581)}
\gppoint{gp mark 0}{(6.256,4.516)}
\gppoint{gp mark 0}{(6.256,4.538)}
\gppoint{gp mark 0}{(6.256,4.538)}
\gppoint{gp mark 0}{(6.256,4.733)}
\gppoint{gp mark 0}{(6.256,4.252)}
\gppoint{gp mark 0}{(6.256,4.697)}
\gppoint{gp mark 0}{(6.256,4.641)}
\gppoint{gp mark 0}{(6.256,4.433)}
\gppoint{gp mark 0}{(6.256,4.527)}
\gppoint{gp mark 0}{(6.256,4.679)}
\gppoint{gp mark 0}{(6.256,4.493)}
\gppoint{gp mark 0}{(6.256,4.420)}
\gppoint{gp mark 0}{(6.256,4.408)}
\gppoint{gp mark 0}{(6.256,4.538)}
\gppoint{gp mark 0}{(6.256,4.469)}
\gppoint{gp mark 0}{(6.256,4.783)}
\gppoint{gp mark 0}{(6.256,4.136)}
\gppoint{gp mark 0}{(6.256,4.433)}
\gppoint{gp mark 0}{(6.256,4.445)}
\gppoint{gp mark 0}{(6.256,4.408)}
\gppoint{gp mark 0}{(6.256,4.445)}
\gppoint{gp mark 0}{(6.256,4.783)}
\gppoint{gp mark 0}{(6.256,4.978)}
\gppoint{gp mark 0}{(6.256,4.420)}
\gppoint{gp mark 0}{(6.256,4.724)}
\gppoint{gp mark 0}{(6.256,4.601)}
\gppoint{gp mark 0}{(6.256,4.581)}
\gppoint{gp mark 0}{(6.256,4.750)}
\gppoint{gp mark 0}{(6.256,4.631)}
\gppoint{gp mark 0}{(6.256,4.697)}
\gppoint{gp mark 0}{(6.256,4.368)}
\gppoint{gp mark 0}{(6.256,4.433)}
\gppoint{gp mark 0}{(6.256,4.581)}
\gppoint{gp mark 0}{(6.274,4.381)}
\gppoint{gp mark 0}{(6.274,4.560)}
\gppoint{gp mark 0}{(6.274,4.493)}
\gppoint{gp mark 0}{(6.274,4.252)}
\gppoint{gp mark 0}{(6.274,4.445)}
\gppoint{gp mark 0}{(6.274,5.022)}
\gppoint{gp mark 0}{(6.274,4.815)}
\gppoint{gp mark 0}{(6.274,4.445)}
\gppoint{gp mark 0}{(6.274,4.221)}
\gppoint{gp mark 0}{(6.274,4.395)}
\gppoint{gp mark 0}{(6.274,4.549)}
\gppoint{gp mark 0}{(6.274,4.381)}
\gppoint{gp mark 0}{(6.274,4.527)}
\gppoint{gp mark 0}{(6.274,4.688)}
\gppoint{gp mark 0}{(6.274,4.481)}
\gppoint{gp mark 0}{(6.274,4.601)}
\gppoint{gp mark 0}{(6.274,4.791)}
\gppoint{gp mark 0}{(6.274,4.136)}
\gppoint{gp mark 0}{(6.274,4.612)}
\gppoint{gp mark 0}{(6.274,4.538)}
\gppoint{gp mark 0}{(6.274,4.853)}
\gppoint{gp mark 0}{(6.274,4.527)}
\gppoint{gp mark 0}{(6.274,4.221)}
\gppoint{gp mark 0}{(6.274,4.838)}
\gppoint{gp mark 0}{(6.274,4.433)}
\gppoint{gp mark 0}{(6.274,4.298)}
\gppoint{gp mark 0}{(6.274,4.621)}
\gppoint{gp mark 0}{(6.274,4.591)}
\gppoint{gp mark 0}{(6.274,3.811)}
\gppoint{gp mark 0}{(6.274,4.527)}
\gppoint{gp mark 0}{(6.274,4.493)}
\gppoint{gp mark 0}{(6.274,4.875)}
\gppoint{gp mark 0}{(6.274,4.420)}
\gppoint{gp mark 0}{(6.274,4.493)}
\gppoint{gp mark 0}{(6.274,4.581)}
\gppoint{gp mark 0}{(6.274,4.875)}
\gppoint{gp mark 0}{(6.274,4.724)}
\gppoint{gp mark 0}{(6.274,4.715)}
\gppoint{gp mark 0}{(6.274,4.527)}
\gppoint{gp mark 0}{(6.274,4.516)}
\gppoint{gp mark 0}{(6.274,4.538)}
\gppoint{gp mark 0}{(6.274,4.783)}
\gppoint{gp mark 0}{(6.274,5.040)}
\gppoint{gp mark 0}{(6.274,4.433)}
\gppoint{gp mark 0}{(6.274,4.341)}
\gppoint{gp mark 0}{(6.274,4.420)}
\gppoint{gp mark 0}{(6.274,4.268)}
\gppoint{gp mark 0}{(6.274,4.715)}
\gppoint{gp mark 0}{(6.274,4.601)}
\gppoint{gp mark 0}{(6.274,4.354)}
\gppoint{gp mark 0}{(6.274,4.153)}
\gppoint{gp mark 0}{(6.292,4.581)}
\gppoint{gp mark 0}{(6.292,4.775)}
\gppoint{gp mark 0}{(6.292,4.516)}
\gppoint{gp mark 0}{(6.292,3.862)}
\gppoint{gp mark 0}{(6.292,4.298)}
\gppoint{gp mark 0}{(6.292,4.868)}
\gppoint{gp mark 0}{(6.292,4.171)}
\gppoint{gp mark 0}{(6.292,4.706)}
\gppoint{gp mark 0}{(6.292,4.823)}
\gppoint{gp mark 0}{(6.292,4.621)}
\gppoint{gp mark 0}{(6.292,4.381)}
\gppoint{gp mark 0}{(6.292,4.457)}
\gppoint{gp mark 0}{(6.292,4.783)}
\gppoint{gp mark 0}{(6.292,4.381)}
\gppoint{gp mark 0}{(6.292,4.354)}
\gppoint{gp mark 0}{(6.292,3.784)}
\gppoint{gp mark 0}{(6.292,4.775)}
\gppoint{gp mark 0}{(6.292,4.420)}
\gppoint{gp mark 0}{(6.292,4.775)}
\gppoint{gp mark 0}{(6.292,4.481)}
\gppoint{gp mark 0}{(6.292,4.612)}
\gppoint{gp mark 0}{(6.292,4.750)}
\gppoint{gp mark 0}{(6.292,4.651)}
\gppoint{gp mark 0}{(6.292,4.945)}
\gppoint{gp mark 0}{(6.292,4.791)}
\gppoint{gp mark 0}{(6.292,4.591)}
\gppoint{gp mark 0}{(6.292,4.252)}
\gppoint{gp mark 0}{(6.292,4.420)}
\gppoint{gp mark 0}{(6.292,4.724)}
\gppoint{gp mark 0}{(6.292,4.612)}
\gppoint{gp mark 0}{(6.292,3.784)}
\gppoint{gp mark 0}{(6.292,3.862)}
\gppoint{gp mark 0}{(6.292,4.715)}
\gppoint{gp mark 0}{(6.292,4.621)}
\gppoint{gp mark 0}{(6.292,4.560)}
\gppoint{gp mark 0}{(6.292,4.733)}
\gppoint{gp mark 0}{(6.292,4.890)}
\gppoint{gp mark 0}{(6.292,4.853)}
\gppoint{gp mark 0}{(6.292,4.457)}
\gppoint{gp mark 0}{(6.292,4.516)}
\gppoint{gp mark 0}{(6.292,4.560)}
\gppoint{gp mark 0}{(6.292,4.354)}
\gppoint{gp mark 0}{(6.292,4.830)}
\gppoint{gp mark 0}{(6.292,4.408)}
\gppoint{gp mark 0}{(6.292,4.679)}
\gppoint{gp mark 0}{(6.292,4.679)}
\gppoint{gp mark 0}{(6.292,4.846)}
\gppoint{gp mark 0}{(6.292,5.034)}
\gppoint{gp mark 0}{(6.292,4.381)}
\gppoint{gp mark 0}{(6.292,4.733)}
\gppoint{gp mark 0}{(6.292,4.433)}
\gppoint{gp mark 0}{(6.292,4.890)}
\gppoint{gp mark 0}{(6.292,4.354)}
\gppoint{gp mark 0}{(6.292,4.861)}
\gppoint{gp mark 0}{(6.292,4.381)}
\gppoint{gp mark 0}{(6.292,4.775)}
\gppoint{gp mark 0}{(6.292,4.706)}
\gppoint{gp mark 0}{(6.292,4.715)}
\gppoint{gp mark 0}{(6.292,4.581)}
\gppoint{gp mark 0}{(6.292,4.493)}
\gppoint{gp mark 0}{(6.309,4.807)}
\gppoint{gp mark 0}{(6.309,4.457)}
\gppoint{gp mark 0}{(6.309,4.758)}
\gppoint{gp mark 0}{(6.309,4.188)}
\gppoint{gp mark 0}{(6.309,4.952)}
\gppoint{gp mark 0}{(6.309,4.469)}
\gppoint{gp mark 0}{(6.309,4.679)}
\gppoint{gp mark 0}{(6.309,4.204)}
\gppoint{gp mark 0}{(6.309,4.457)}
\gppoint{gp mark 0}{(6.309,4.799)}
\gppoint{gp mark 0}{(6.309,4.549)}
\gppoint{gp mark 0}{(6.309,4.890)}
\gppoint{gp mark 0}{(6.309,4.591)}
\gppoint{gp mark 0}{(6.309,4.516)}
\gppoint{gp mark 0}{(6.309,4.830)}
\gppoint{gp mark 0}{(6.309,4.758)}
\gppoint{gp mark 0}{(6.309,4.516)}
\gppoint{gp mark 0}{(6.309,5.063)}
\gppoint{gp mark 0}{(6.309,4.601)}
\gppoint{gp mark 0}{(6.309,4.341)}
\gppoint{gp mark 0}{(6.309,4.651)}
\gppoint{gp mark 0}{(6.309,4.457)}
\gppoint{gp mark 0}{(6.309,4.549)}
\gppoint{gp mark 0}{(6.309,4.457)}
\gppoint{gp mark 0}{(6.309,4.493)}
\gppoint{gp mark 0}{(6.309,5.370)}
\gppoint{gp mark 0}{(6.309,5.146)}
\gppoint{gp mark 0}{(6.309,4.457)}
\gppoint{gp mark 0}{(6.309,4.581)}
\gppoint{gp mark 0}{(6.309,4.493)}
\gppoint{gp mark 0}{(6.309,4.457)}
\gppoint{gp mark 0}{(6.309,4.457)}
\gppoint{gp mark 0}{(6.309,4.469)}
\gppoint{gp mark 0}{(6.309,5.016)}
\gppoint{gp mark 0}{(6.309,4.621)}
\gppoint{gp mark 0}{(6.309,5.259)}
\gppoint{gp mark 0}{(6.309,4.775)}
\gppoint{gp mark 0}{(6.309,4.621)}
\gppoint{gp mark 0}{(6.309,4.783)}
\gppoint{gp mark 0}{(6.309,4.875)}
\gppoint{gp mark 0}{(6.309,4.783)}
\gppoint{gp mark 0}{(6.309,4.621)}
\gppoint{gp mark 0}{(6.309,4.783)}
\gppoint{gp mark 0}{(6.309,4.549)}
\gppoint{gp mark 0}{(6.309,4.457)}
\gppoint{gp mark 0}{(6.309,4.612)}
\gppoint{gp mark 0}{(6.309,4.823)}
\gppoint{gp mark 0}{(6.309,4.706)}
\gppoint{gp mark 0}{(6.309,4.767)}
\gppoint{gp mark 0}{(6.309,4.697)}
\gppoint{gp mark 0}{(6.309,4.651)}
\gppoint{gp mark 0}{(6.309,4.679)}
\gppoint{gp mark 0}{(6.309,4.612)}
\gppoint{gp mark 0}{(6.309,5.034)}
\gppoint{gp mark 0}{(6.309,4.791)}
\gppoint{gp mark 0}{(6.309,4.750)}
\gppoint{gp mark 0}{(6.327,4.660)}
\gppoint{gp mark 0}{(6.327,4.268)}
\gppoint{gp mark 0}{(6.327,4.171)}
\gppoint{gp mark 0}{(6.327,4.621)}
\gppoint{gp mark 0}{(6.327,4.420)}
\gppoint{gp mark 0}{(6.327,4.516)}
\gppoint{gp mark 0}{(6.327,4.504)}
\gppoint{gp mark 0}{(6.327,4.327)}
\gppoint{gp mark 0}{(6.327,4.679)}
\gppoint{gp mark 0}{(6.327,4.972)}
\gppoint{gp mark 0}{(6.327,4.591)}
\gppoint{gp mark 0}{(6.327,4.493)}
\gppoint{gp mark 0}{(6.327,4.433)}
\gppoint{gp mark 0}{(6.327,4.538)}
\gppoint{gp mark 0}{(6.327,4.570)}
\gppoint{gp mark 0}{(6.327,4.965)}
\gppoint{gp mark 0}{(6.327,4.581)}
\gppoint{gp mark 0}{(6.327,4.715)}
\gppoint{gp mark 0}{(6.327,4.733)}
\gppoint{gp mark 0}{(6.327,4.621)}
\gppoint{gp mark 0}{(6.327,5.040)}
\gppoint{gp mark 0}{(6.327,4.354)}
\gppoint{gp mark 0}{(6.327,4.493)}
\gppoint{gp mark 0}{(6.327,5.075)}
\gppoint{gp mark 0}{(6.327,4.516)}
\gppoint{gp mark 0}{(6.327,4.791)}
\gppoint{gp mark 0}{(6.327,4.830)}
\gppoint{gp mark 0}{(6.327,4.481)}
\gppoint{gp mark 0}{(6.327,4.612)}
\gppoint{gp mark 0}{(6.327,4.846)}
\gppoint{gp mark 0}{(6.327,4.890)}
\gppoint{gp mark 0}{(6.327,4.775)}
\gppoint{gp mark 0}{(6.327,4.791)}
\gppoint{gp mark 0}{(6.327,4.469)}
\gppoint{gp mark 0}{(6.327,4.327)}
\gppoint{gp mark 0}{(6.327,4.791)}
\gppoint{gp mark 0}{(6.327,4.420)}
\gppoint{gp mark 0}{(6.327,4.904)}
\gppoint{gp mark 0}{(6.327,4.527)}
\gppoint{gp mark 0}{(6.327,5.052)}
\gppoint{gp mark 0}{(6.327,4.631)}
\gppoint{gp mark 0}{(6.327,4.775)}
\gppoint{gp mark 0}{(6.327,4.341)}
\gppoint{gp mark 0}{(6.327,5.151)}
\gppoint{gp mark 0}{(6.327,4.516)}
\gppoint{gp mark 0}{(6.327,4.631)}
\gppoint{gp mark 0}{(6.327,4.783)}
\gppoint{gp mark 0}{(6.327,4.767)}
\gppoint{gp mark 0}{(6.327,4.527)}
\gppoint{gp mark 0}{(6.327,4.457)}
\gppoint{gp mark 0}{(6.327,4.581)}
\gppoint{gp mark 0}{(6.327,4.838)}
\gppoint{gp mark 0}{(6.327,4.420)}
\gppoint{gp mark 0}{(6.344,4.457)}
\gppoint{gp mark 0}{(6.344,4.660)}
\gppoint{gp mark 0}{(6.344,4.395)}
\gppoint{gp mark 0}{(6.344,4.312)}
\gppoint{gp mark 0}{(6.344,4.504)}
\gppoint{gp mark 0}{(6.344,5.086)}
\gppoint{gp mark 0}{(6.344,4.504)}
\gppoint{gp mark 0}{(6.344,4.775)}
\gppoint{gp mark 0}{(6.344,4.733)}
\gppoint{gp mark 0}{(6.344,4.815)}
\gppoint{gp mark 0}{(6.344,4.823)}
\gppoint{gp mark 0}{(6.344,4.395)}
\gppoint{gp mark 0}{(6.344,4.741)}
\gppoint{gp mark 0}{(6.344,4.408)}
\gppoint{gp mark 0}{(6.344,4.641)}
\gppoint{gp mark 0}{(6.344,4.733)}
\gppoint{gp mark 0}{(6.344,4.481)}
\gppoint{gp mark 0}{(6.344,4.641)}
\gppoint{gp mark 0}{(6.344,4.408)}
\gppoint{gp mark 0}{(6.344,4.641)}
\gppoint{gp mark 0}{(6.344,4.527)}
\gppoint{gp mark 0}{(6.344,4.938)}
\gppoint{gp mark 0}{(6.344,4.631)}
\gppoint{gp mark 0}{(6.344,4.861)}
\gppoint{gp mark 0}{(6.344,4.883)}
\gppoint{gp mark 0}{(6.344,5.034)}
\gppoint{gp mark 0}{(6.344,4.883)}
\gppoint{gp mark 0}{(6.344,4.395)}
\gppoint{gp mark 0}{(6.344,5.086)}
\gppoint{gp mark 0}{(6.344,4.641)}
\gppoint{gp mark 0}{(6.344,4.457)}
\gppoint{gp mark 0}{(6.344,4.815)}
\gppoint{gp mark 0}{(6.344,4.395)}
\gppoint{gp mark 0}{(6.344,4.758)}
\gppoint{gp mark 0}{(6.344,4.631)}
\gppoint{gp mark 0}{(6.344,4.861)}
\gppoint{gp mark 0}{(6.344,4.591)}
\gppoint{gp mark 0}{(6.344,4.641)}
\gppoint{gp mark 0}{(6.344,4.283)}
\gppoint{gp mark 0}{(6.344,4.527)}
\gppoint{gp mark 0}{(6.344,4.861)}
\gppoint{gp mark 0}{(6.344,5.003)}
\gppoint{gp mark 0}{(6.344,4.549)}
\gppoint{gp mark 0}{(6.344,4.670)}
\gppoint{gp mark 0}{(6.344,4.538)}
\gppoint{gp mark 0}{(6.344,4.549)}
\gppoint{gp mark 0}{(6.344,4.938)}
\gppoint{gp mark 0}{(6.344,4.791)}
\gppoint{gp mark 0}{(6.344,4.715)}
\gppoint{gp mark 0}{(6.344,4.504)}
\gppoint{gp mark 0}{(6.344,4.298)}
\gppoint{gp mark 0}{(6.344,4.938)}
\gppoint{gp mark 0}{(6.344,4.938)}
\gppoint{gp mark 0}{(6.344,4.750)}
\gppoint{gp mark 0}{(6.344,4.890)}
\gppoint{gp mark 0}{(6.344,4.504)}
\gppoint{gp mark 0}{(6.344,4.706)}
\gppoint{gp mark 0}{(6.344,4.312)}
\gppoint{gp mark 0}{(6.344,4.706)}
\gppoint{gp mark 0}{(6.344,4.560)}
\gppoint{gp mark 0}{(6.344,4.368)}
\gppoint{gp mark 0}{(6.360,4.978)}
\gppoint{gp mark 0}{(6.360,4.469)}
\gppoint{gp mark 0}{(6.360,4.252)}
\gppoint{gp mark 0}{(6.360,4.733)}
\gppoint{gp mark 0}{(6.360,5.046)}
\gppoint{gp mark 0}{(6.360,4.868)}
\gppoint{gp mark 0}{(6.360,4.469)}
\gppoint{gp mark 0}{(6.360,4.706)}
\gppoint{gp mark 0}{(6.360,4.601)}
\gppoint{gp mark 0}{(6.360,4.445)}
\gppoint{gp mark 0}{(6.360,4.868)}
\gppoint{gp mark 0}{(6.360,4.775)}
\gppoint{gp mark 0}{(6.360,4.733)}
\gppoint{gp mark 0}{(6.360,4.341)}
\gppoint{gp mark 0}{(6.360,4.641)}
\gppoint{gp mark 0}{(6.360,4.354)}
\gppoint{gp mark 0}{(6.360,4.041)}
\gppoint{gp mark 0}{(6.360,5.028)}
\gppoint{gp mark 0}{(6.360,4.883)}
\gppoint{gp mark 0}{(6.360,4.341)}
\gppoint{gp mark 0}{(6.360,4.631)}
\gppoint{gp mark 0}{(6.360,4.670)}
\gppoint{gp mark 0}{(6.360,4.706)}
\gppoint{gp mark 0}{(6.360,4.641)}
\gppoint{gp mark 0}{(6.360,4.815)}
\gppoint{gp mark 0}{(6.360,4.327)}
\gppoint{gp mark 0}{(6.360,4.493)}
\gppoint{gp mark 0}{(6.360,4.612)}
\gppoint{gp mark 0}{(6.360,4.341)}
\gppoint{gp mark 0}{(6.360,4.621)}
\gppoint{gp mark 0}{(6.360,4.791)}
\gppoint{gp mark 0}{(6.360,4.298)}
\gppoint{gp mark 0}{(6.360,4.838)}
\gppoint{gp mark 0}{(6.360,4.918)}
\gppoint{gp mark 0}{(6.360,4.493)}
\gppoint{gp mark 0}{(6.360,4.312)}
\gppoint{gp mark 0}{(6.360,5.146)}
\gppoint{gp mark 0}{(6.360,4.341)}
\gppoint{gp mark 0}{(6.360,4.978)}
\gppoint{gp mark 0}{(6.360,4.252)}
\gppoint{gp mark 0}{(6.360,4.679)}
\gppoint{gp mark 0}{(6.360,4.697)}
\gppoint{gp mark 0}{(6.360,4.252)}
\gppoint{gp mark 0}{(6.360,4.868)}
\gppoint{gp mark 0}{(6.360,4.651)}
\gppoint{gp mark 0}{(6.360,4.660)}
\gppoint{gp mark 0}{(6.377,5.075)}
\gppoint{gp mark 0}{(6.377,4.890)}
\gppoint{gp mark 0}{(6.377,4.000)}
\gppoint{gp mark 0}{(6.377,4.457)}
\gppoint{gp mark 0}{(6.377,4.861)}
\gppoint{gp mark 0}{(6.377,4.791)}
\gppoint{gp mark 0}{(6.377,4.481)}
\gppoint{gp mark 0}{(6.377,4.724)}
\gppoint{gp mark 0}{(6.377,3.934)}
\gppoint{gp mark 0}{(6.377,4.312)}
\gppoint{gp mark 0}{(6.377,5.016)}
\gppoint{gp mark 0}{(6.377,4.932)}
\gppoint{gp mark 0}{(6.377,4.621)}
\gppoint{gp mark 0}{(6.377,4.688)}
\gppoint{gp mark 0}{(6.377,4.697)}
\gppoint{gp mark 0}{(6.377,4.481)}
\gppoint{gp mark 0}{(6.377,4.991)}
\gppoint{gp mark 0}{(6.377,5.097)}
\gppoint{gp mark 0}{(6.377,4.918)}
\gppoint{gp mark 0}{(6.377,4.838)}
\gppoint{gp mark 0}{(6.377,4.581)}
\gppoint{gp mark 0}{(6.377,4.932)}
\gppoint{gp mark 0}{(6.377,4.581)}
\gppoint{gp mark 0}{(6.377,4.741)}
\gppoint{gp mark 0}{(6.377,4.283)}
\gppoint{gp mark 0}{(6.377,4.670)}
\gppoint{gp mark 0}{(6.377,4.791)}
\gppoint{gp mark 0}{(6.377,5.016)}
\gppoint{gp mark 0}{(6.377,4.791)}
\gppoint{gp mark 0}{(6.377,4.791)}
\gppoint{gp mark 0}{(6.377,4.861)}
\gppoint{gp mark 0}{(6.377,4.791)}
\gppoint{gp mark 0}{(6.377,4.767)}
\gppoint{gp mark 0}{(6.377,4.724)}
\gppoint{gp mark 0}{(6.377,4.733)}
\gppoint{gp mark 0}{(6.377,4.395)}
\gppoint{gp mark 0}{(6.377,5.172)}
\gppoint{gp mark 0}{(6.377,4.791)}
\gppoint{gp mark 0}{(6.377,4.791)}
\gppoint{gp mark 0}{(6.377,4.621)}
\gppoint{gp mark 0}{(6.377,4.395)}
\gppoint{gp mark 0}{(6.377,4.327)}
\gppoint{gp mark 0}{(6.377,4.791)}
\gppoint{gp mark 0}{(6.377,4.741)}
\gppoint{gp mark 0}{(6.377,4.420)}
\gppoint{gp mark 0}{(6.377,4.679)}
\gppoint{gp mark 0}{(6.377,4.932)}
\gppoint{gp mark 0}{(6.377,4.516)}
\gppoint{gp mark 0}{(6.377,4.395)}
\gppoint{gp mark 0}{(6.377,4.136)}
\gppoint{gp mark 0}{(6.377,4.853)}
\gppoint{gp mark 0}{(6.377,5.003)}
\gppoint{gp mark 0}{(6.377,4.750)}
\gppoint{gp mark 0}{(6.377,4.481)}
\gppoint{gp mark 0}{(6.377,4.504)}
\gppoint{gp mark 0}{(6.377,5.003)}
\gppoint{gp mark 0}{(6.377,4.660)}
\gppoint{gp mark 0}{(6.377,4.883)}
\gppoint{gp mark 0}{(6.377,4.861)}
\gppoint{gp mark 0}{(6.377,5.052)}
\gppoint{gp mark 0}{(6.377,4.724)}
\gppoint{gp mark 0}{(6.377,4.516)}
\gppoint{gp mark 0}{(6.377,4.516)}
\gppoint{gp mark 0}{(6.393,4.799)}
\gppoint{gp mark 0}{(6.393,4.697)}
\gppoint{gp mark 0}{(6.393,4.560)}
\gppoint{gp mark 0}{(6.393,4.538)}
\gppoint{gp mark 0}{(6.393,4.538)}
\gppoint{gp mark 0}{(6.393,4.171)}
\gppoint{gp mark 0}{(6.393,4.481)}
\gppoint{gp mark 0}{(6.393,4.688)}
\gppoint{gp mark 0}{(6.393,4.724)}
\gppoint{gp mark 0}{(6.393,4.298)}
\gppoint{gp mark 0}{(6.393,4.298)}
\gppoint{gp mark 0}{(6.393,4.741)}
\gppoint{gp mark 0}{(6.393,4.741)}
\gppoint{gp mark 0}{(6.393,4.368)}
\gppoint{gp mark 0}{(6.393,4.688)}
\gppoint{gp mark 0}{(6.393,4.493)}
\gppoint{gp mark 0}{(6.393,4.570)}
\gppoint{gp mark 0}{(6.393,4.549)}
\gppoint{gp mark 0}{(6.393,4.733)}
\gppoint{gp mark 0}{(6.393,4.670)}
\gppoint{gp mark 0}{(6.393,4.651)}
\gppoint{gp mark 0}{(6.393,4.591)}
\gppoint{gp mark 0}{(6.393,5.324)}
\gppoint{gp mark 0}{(6.393,4.538)}
\gppoint{gp mark 0}{(6.393,4.469)}
\gppoint{gp mark 0}{(6.393,4.997)}
\gppoint{gp mark 0}{(6.393,4.549)}
\gppoint{gp mark 0}{(6.393,4.591)}
\gppoint{gp mark 0}{(6.393,4.952)}
\gppoint{gp mark 0}{(6.393,4.838)}
\gppoint{gp mark 0}{(6.393,4.697)}
\gppoint{gp mark 0}{(6.393,4.830)}
\gppoint{gp mark 0}{(6.393,4.783)}
\gppoint{gp mark 0}{(6.393,4.706)}
\gppoint{gp mark 0}{(6.393,4.791)}
\gppoint{gp mark 0}{(6.393,4.799)}
\gppoint{gp mark 0}{(6.393,4.688)}
\gppoint{gp mark 0}{(6.393,4.783)}
\gppoint{gp mark 0}{(6.393,4.283)}
\gppoint{gp mark 0}{(6.393,4.945)}
\gppoint{gp mark 0}{(6.393,4.733)}
\gppoint{gp mark 0}{(6.393,4.651)}
\gppoint{gp mark 0}{(6.393,4.641)}
\gppoint{gp mark 0}{(6.393,4.733)}
\gppoint{gp mark 0}{(6.393,4.433)}
\gppoint{gp mark 0}{(6.393,4.758)}
\gppoint{gp mark 0}{(6.393,4.481)}
\gppoint{gp mark 0}{(6.409,4.853)}
\gppoint{gp mark 0}{(6.409,4.875)}
\gppoint{gp mark 0}{(6.409,4.853)}
\gppoint{gp mark 0}{(6.409,3.911)}
\gppoint{gp mark 0}{(6.409,4.697)}
\gppoint{gp mark 0}{(6.409,4.775)}
\gppoint{gp mark 0}{(6.409,4.651)}
\gppoint{gp mark 0}{(6.409,4.252)}
\gppoint{gp mark 0}{(6.409,5.119)}
\gppoint{gp mark 0}{(6.409,4.807)}
\gppoint{gp mark 0}{(6.409,5.337)}
\gppoint{gp mark 0}{(6.409,4.312)}
\gppoint{gp mark 0}{(6.409,4.991)}
\gppoint{gp mark 0}{(6.409,5.119)}
\gppoint{gp mark 0}{(6.409,4.433)}
\gppoint{gp mark 0}{(6.409,4.445)}
\gppoint{gp mark 0}{(6.409,4.758)}
\gppoint{gp mark 0}{(6.409,3.911)}
\gppoint{gp mark 0}{(6.409,4.846)}
\gppoint{gp mark 0}{(6.409,4.188)}
\gppoint{gp mark 0}{(6.409,5.052)}
\gppoint{gp mark 0}{(6.409,4.549)}
\gppoint{gp mark 0}{(6.409,5.345)}
\gppoint{gp mark 0}{(6.409,5.221)}
\gppoint{gp mark 0}{(6.409,5.119)}
\gppoint{gp mark 0}{(6.409,5.103)}
\gppoint{gp mark 0}{(6.409,4.724)}
\gppoint{gp mark 0}{(6.409,4.641)}
\gppoint{gp mark 0}{(6.409,4.549)}
\gppoint{gp mark 0}{(6.409,4.815)}
\gppoint{gp mark 0}{(6.409,4.952)}
\gppoint{gp mark 0}{(6.409,4.283)}
\gppoint{gp mark 0}{(6.409,3.811)}
\gppoint{gp mark 0}{(6.409,4.504)}
\gppoint{gp mark 0}{(6.409,4.341)}
\gppoint{gp mark 0}{(6.409,4.846)}
\gppoint{gp mark 0}{(6.409,4.549)}
\gppoint{gp mark 0}{(6.409,4.984)}
\gppoint{gp mark 0}{(6.409,4.621)}
\gppoint{gp mark 0}{(6.409,4.706)}
\gppoint{gp mark 0}{(6.409,4.538)}
\gppoint{gp mark 0}{(6.409,4.469)}
\gppoint{gp mark 0}{(6.409,4.997)}
\gppoint{gp mark 0}{(6.409,4.504)}
\gppoint{gp mark 0}{(6.409,4.312)}
\gppoint{gp mark 0}{(6.409,4.581)}
\gppoint{gp mark 0}{(6.409,4.021)}
\gppoint{gp mark 0}{(6.409,4.997)}
\gppoint{gp mark 0}{(6.409,4.706)}
\gppoint{gp mark 0}{(6.409,5.366)}
\gppoint{gp mark 0}{(6.409,4.938)}
\gppoint{gp mark 0}{(6.409,4.549)}
\gppoint{gp mark 0}{(6.425,4.560)}
\gppoint{gp mark 0}{(6.425,4.651)}
\gppoint{gp mark 0}{(6.425,4.549)}
\gppoint{gp mark 0}{(6.425,4.252)}
\gppoint{gp mark 0}{(6.425,4.481)}
\gppoint{gp mark 0}{(6.425,4.000)}
\gppoint{gp mark 0}{(6.425,4.457)}
\gppoint{gp mark 0}{(6.425,4.775)}
\gppoint{gp mark 0}{(6.425,4.504)}
\gppoint{gp mark 0}{(6.425,4.697)}
\gppoint{gp mark 0}{(6.425,4.354)}
\gppoint{gp mark 0}{(6.425,4.560)}
\gppoint{gp mark 0}{(6.425,4.327)}
\gppoint{gp mark 0}{(6.425,4.733)}
\gppoint{gp mark 0}{(6.425,4.883)}
\gppoint{gp mark 0}{(6.425,4.268)}
\gppoint{gp mark 0}{(6.425,4.469)}
\gppoint{gp mark 0}{(6.425,4.688)}
\gppoint{gp mark 0}{(6.425,5.146)}
\gppoint{gp mark 0}{(6.425,5.202)}
\gppoint{gp mark 0}{(6.425,4.890)}
\gppoint{gp mark 0}{(6.425,5.081)}
\gppoint{gp mark 0}{(6.425,4.679)}
\gppoint{gp mark 0}{(6.425,4.697)}
\gppoint{gp mark 0}{(6.425,4.354)}
\gppoint{gp mark 0}{(6.425,4.670)}
\gppoint{gp mark 0}{(6.425,5.075)}
\gppoint{gp mark 0}{(6.425,5.574)}
\gppoint{gp mark 0}{(6.425,4.791)}
\gppoint{gp mark 0}{(6.425,4.984)}
\gppoint{gp mark 0}{(6.425,4.327)}
\gppoint{gp mark 0}{(6.425,4.237)}
\gppoint{gp mark 0}{(6.425,4.978)}
\gppoint{gp mark 0}{(6.425,4.298)}
\gppoint{gp mark 0}{(6.425,4.890)}
\gppoint{gp mark 0}{(6.425,4.527)}
\gppoint{gp mark 0}{(6.425,5.146)}
\gppoint{gp mark 0}{(6.425,4.679)}
\gppoint{gp mark 0}{(6.425,4.631)}
\gppoint{gp mark 0}{(6.441,4.204)}
\gppoint{gp mark 0}{(6.441,4.715)}
\gppoint{gp mark 0}{(6.441,4.469)}
\gppoint{gp mark 0}{(6.441,4.469)}
\gppoint{gp mark 0}{(6.441,4.549)}
\gppoint{gp mark 0}{(6.441,4.724)}
\gppoint{gp mark 0}{(6.441,4.570)}
\gppoint{gp mark 0}{(6.441,4.204)}
\gppoint{gp mark 0}{(6.441,4.875)}
\gppoint{gp mark 0}{(6.441,5.114)}
\gppoint{gp mark 0}{(6.441,4.724)}
\gppoint{gp mark 0}{(6.441,4.381)}
\gppoint{gp mark 0}{(6.441,4.612)}
\gppoint{gp mark 0}{(6.441,4.660)}
\gppoint{gp mark 0}{(6.441,4.791)}
\gppoint{gp mark 0}{(6.441,4.204)}
\gppoint{gp mark 0}{(6.441,4.612)}
\gppoint{gp mark 0}{(6.441,4.612)}
\gppoint{gp mark 0}{(6.441,4.972)}
\gppoint{gp mark 0}{(6.441,4.204)}
\gppoint{gp mark 0}{(6.441,4.395)}
\gppoint{gp mark 0}{(6.441,4.807)}
\gppoint{gp mark 0}{(6.441,5.022)}
\gppoint{gp mark 0}{(6.441,5.003)}
\gppoint{gp mark 0}{(6.441,4.724)}
\gppoint{gp mark 0}{(6.441,4.204)}
\gppoint{gp mark 0}{(6.441,4.984)}
\gppoint{gp mark 0}{(6.441,4.601)}
\gppoint{gp mark 0}{(6.441,4.807)}
\gppoint{gp mark 0}{(6.441,4.204)}
\gppoint{gp mark 0}{(6.441,4.433)}
\gppoint{gp mark 0}{(6.441,4.715)}
\gppoint{gp mark 0}{(6.441,4.733)}
\gppoint{gp mark 0}{(6.441,4.724)}
\gppoint{gp mark 0}{(6.441,4.706)}
\gppoint{gp mark 0}{(6.441,4.204)}
\gppoint{gp mark 0}{(6.441,4.775)}
\gppoint{gp mark 0}{(6.441,4.204)}
\gppoint{gp mark 0}{(6.441,4.204)}
\gppoint{gp mark 0}{(6.441,4.204)}
\gppoint{gp mark 0}{(6.441,4.799)}
\gppoint{gp mark 0}{(6.441,4.204)}
\gppoint{gp mark 0}{(6.441,4.741)}
\gppoint{gp mark 0}{(6.441,4.904)}
\gppoint{gp mark 0}{(6.441,4.221)}
\gppoint{gp mark 0}{(6.441,4.861)}
\gppoint{gp mark 0}{(6.441,4.381)}
\gppoint{gp mark 0}{(6.441,4.925)}
\gppoint{gp mark 0}{(6.441,4.838)}
\gppoint{gp mark 0}{(6.456,4.660)}
\gppoint{gp mark 0}{(6.456,5.216)}
\gppoint{gp mark 0}{(6.456,5.156)}
\gppoint{gp mark 0}{(6.456,4.601)}
\gppoint{gp mark 0}{(6.456,4.890)}
\gppoint{gp mark 0}{(6.456,5.022)}
\gppoint{gp mark 0}{(6.456,4.918)}
\gppoint{gp mark 0}{(6.456,4.660)}
\gppoint{gp mark 0}{(6.456,4.965)}
\gppoint{gp mark 0}{(6.456,4.204)}
\gppoint{gp mark 0}{(6.456,4.911)}
\gppoint{gp mark 0}{(6.456,4.204)}
\gppoint{gp mark 0}{(6.456,4.433)}
\gppoint{gp mark 0}{(6.456,5.114)}
\gppoint{gp mark 0}{(6.456,4.875)}
\gppoint{gp mark 0}{(6.456,4.679)}
\gppoint{gp mark 0}{(6.456,4.688)}
\gppoint{gp mark 0}{(6.456,4.560)}
\gppoint{gp mark 0}{(6.456,4.651)}
\gppoint{gp mark 0}{(6.456,4.061)}
\gppoint{gp mark 0}{(6.456,4.591)}
\gppoint{gp mark 0}{(6.456,4.875)}
\gppoint{gp mark 0}{(6.456,4.791)}
\gppoint{gp mark 0}{(6.456,5.221)}
\gppoint{gp mark 0}{(6.456,5.167)}
\gppoint{gp mark 0}{(6.456,4.978)}
\gppoint{gp mark 0}{(6.456,4.204)}
\gppoint{gp mark 0}{(6.456,4.283)}
\gppoint{gp mark 0}{(6.456,4.706)}
\gppoint{gp mark 0}{(6.456,4.670)}
\gppoint{gp mark 0}{(6.456,4.853)}
\gppoint{gp mark 0}{(6.456,4.204)}
\gppoint{gp mark 0}{(6.456,4.312)}
\gppoint{gp mark 0}{(6.456,4.799)}
\gppoint{gp mark 0}{(6.456,4.204)}
\gppoint{gp mark 0}{(6.456,4.612)}
\gppoint{gp mark 0}{(6.456,5.192)}
\gppoint{gp mark 0}{(6.456,4.641)}
\gppoint{gp mark 0}{(6.456,4.911)}
\gppoint{gp mark 0}{(6.456,4.204)}
\gppoint{gp mark 0}{(6.456,4.408)}
\gppoint{gp mark 0}{(6.456,4.938)}
\gppoint{gp mark 0}{(6.456,4.549)}
\gppoint{gp mark 0}{(6.456,4.791)}
\gppoint{gp mark 0}{(6.472,5.337)}
\gppoint{gp mark 0}{(6.472,5.003)}
\gppoint{gp mark 0}{(6.472,4.791)}
\gppoint{gp mark 0}{(6.472,5.135)}
\gppoint{gp mark 0}{(6.472,4.861)}
\gppoint{gp mark 0}{(6.472,4.733)}
\gppoint{gp mark 0}{(6.472,4.204)}
\gppoint{gp mark 0}{(6.472,4.904)}
\gppoint{gp mark 0}{(6.472,4.651)}
\gppoint{gp mark 0}{(6.472,4.631)}
\gppoint{gp mark 0}{(6.472,4.641)}
\gppoint{gp mark 0}{(6.472,4.932)}
\gppoint{gp mark 0}{(6.472,4.861)}
\gppoint{gp mark 0}{(6.472,5.103)}
\gppoint{gp mark 0}{(6.472,4.538)}
\gppoint{gp mark 0}{(6.472,4.791)}
\gppoint{gp mark 0}{(6.472,4.204)}
\gppoint{gp mark 0}{(6.472,4.750)}
\gppoint{gp mark 0}{(6.472,4.204)}
\gppoint{gp mark 0}{(6.472,4.853)}
\gppoint{gp mark 0}{(6.472,5.063)}
\gppoint{gp mark 0}{(6.472,4.883)}
\gppoint{gp mark 0}{(6.472,4.204)}
\gppoint{gp mark 0}{(6.472,4.733)}
\gppoint{gp mark 0}{(6.472,5.268)}
\gppoint{gp mark 0}{(6.472,4.904)}
\gppoint{gp mark 0}{(6.472,4.791)}
\gppoint{gp mark 0}{(6.472,4.733)}
\gppoint{gp mark 0}{(6.472,4.591)}
\gppoint{gp mark 0}{(6.472,4.504)}
\gppoint{gp mark 0}{(6.472,4.938)}
\gppoint{gp mark 0}{(6.472,4.733)}
\gppoint{gp mark 0}{(6.472,4.538)}
\gppoint{gp mark 0}{(6.487,4.481)}
\gppoint{gp mark 0}{(6.487,4.612)}
\gppoint{gp mark 0}{(6.487,5.040)}
\gppoint{gp mark 0}{(6.487,4.493)}
\gppoint{gp mark 0}{(6.487,4.853)}
\gppoint{gp mark 0}{(6.487,4.601)}
\gppoint{gp mark 0}{(6.487,4.750)}
\gppoint{gp mark 0}{(6.487,4.904)}
\gppoint{gp mark 0}{(6.487,4.958)}
\gppoint{gp mark 0}{(6.487,4.890)}
\gppoint{gp mark 0}{(6.487,4.741)}
\gppoint{gp mark 0}{(6.487,4.890)}
\gppoint{gp mark 0}{(6.487,5.058)}
\gppoint{gp mark 0}{(6.487,4.733)}
\gppoint{gp mark 0}{(6.487,4.853)}
\gppoint{gp mark 0}{(6.487,4.890)}
\gppoint{gp mark 0}{(6.487,4.368)}
\gppoint{gp mark 0}{(6.487,4.538)}
\gppoint{gp mark 0}{(6.487,4.341)}
\gppoint{gp mark 0}{(6.487,4.890)}
\gppoint{gp mark 0}{(6.487,4.911)}
\gppoint{gp mark 0}{(6.487,4.767)}
\gppoint{gp mark 0}{(6.487,4.591)}
\gppoint{gp mark 0}{(6.487,4.883)}
\gppoint{gp mark 0}{(6.487,4.591)}
\gppoint{gp mark 0}{(6.487,4.965)}
\gppoint{gp mark 0}{(6.487,4.897)}
\gppoint{gp mark 0}{(6.487,4.481)}
\gppoint{gp mark 0}{(6.487,4.758)}
\gppoint{gp mark 0}{(6.487,4.904)}
\gppoint{gp mark 0}{(6.487,4.925)}
\gppoint{gp mark 0}{(6.487,4.938)}
\gppoint{gp mark 0}{(6.487,4.660)}
\gppoint{gp mark 0}{(6.487,4.767)}
\gppoint{gp mark 0}{(6.502,5.478)}
\gppoint{gp mark 0}{(6.502,4.807)}
\gppoint{gp mark 0}{(6.502,4.493)}
\gppoint{gp mark 0}{(6.502,4.706)}
\gppoint{gp mark 0}{(6.502,4.706)}
\gppoint{gp mark 0}{(6.502,4.538)}
\gppoint{gp mark 0}{(6.502,4.724)}
\gppoint{gp mark 0}{(6.502,5.119)}
\gppoint{gp mark 0}{(6.502,5.040)}
\gppoint{gp mark 0}{(6.502,5.119)}
\gppoint{gp mark 0}{(6.502,4.679)}
\gppoint{gp mark 0}{(6.502,4.741)}
\gppoint{gp mark 0}{(6.502,5.086)}
\gppoint{gp mark 0}{(6.502,4.741)}
\gppoint{gp mark 0}{(6.502,4.549)}
\gppoint{gp mark 0}{(6.502,4.741)}
\gppoint{gp mark 0}{(6.502,5.040)}
\gppoint{gp mark 0}{(6.502,4.775)}
\gppoint{gp mark 0}{(6.502,4.118)}
\gppoint{gp mark 0}{(6.502,4.581)}
\gppoint{gp mark 0}{(6.502,4.750)}
\gppoint{gp mark 0}{(6.502,4.783)}
\gppoint{gp mark 0}{(6.502,4.853)}
\gppoint{gp mark 0}{(6.502,4.560)}
\gppoint{gp mark 0}{(6.502,5.431)}
\gppoint{gp mark 0}{(6.502,4.897)}
\gppoint{gp mark 0}{(6.502,4.853)}
\gppoint{gp mark 0}{(6.502,4.670)}
\gppoint{gp mark 0}{(6.502,4.368)}
\gppoint{gp mark 0}{(6.502,4.853)}
\gppoint{gp mark 0}{(6.502,4.775)}
\gppoint{gp mark 0}{(6.502,4.911)}
\gppoint{gp mark 0}{(6.502,4.621)}
\gppoint{gp mark 0}{(6.502,4.688)}
\gppoint{gp mark 0}{(6.502,4.688)}
\gppoint{gp mark 0}{(6.502,4.823)}
\gppoint{gp mark 0}{(6.502,5.075)}
\gppoint{gp mark 0}{(6.502,5.075)}
\gppoint{gp mark 0}{(6.502,4.581)}
\gppoint{gp mark 0}{(6.502,4.868)}
\gppoint{gp mark 0}{(6.502,5.010)}
\gppoint{gp mark 0}{(6.502,4.706)}
\gppoint{gp mark 0}{(6.502,4.679)}
\gppoint{gp mark 0}{(6.502,4.670)}
\gppoint{gp mark 0}{(6.502,4.469)}
\gppoint{gp mark 0}{(6.502,4.868)}
\gppoint{gp mark 0}{(6.502,4.581)}
\gppoint{gp mark 0}{(6.502,4.715)}
\gppoint{gp mark 0}{(6.502,4.932)}
\gppoint{gp mark 0}{(6.502,5.010)}
\gppoint{gp mark 0}{(6.502,4.268)}
\gppoint{gp mark 0}{(6.502,4.846)}
\gppoint{gp mark 0}{(6.502,4.408)}
\gppoint{gp mark 0}{(6.516,4.904)}
\gppoint{gp mark 0}{(6.516,5.046)}
\gppoint{gp mark 0}{(6.516,4.815)}
\gppoint{gp mark 0}{(6.516,4.591)}
\gppoint{gp mark 0}{(6.516,4.469)}
\gppoint{gp mark 0}{(6.516,4.312)}
\gppoint{gp mark 0}{(6.516,5.108)}
\gppoint{gp mark 0}{(6.516,4.560)}
\gppoint{gp mark 0}{(6.516,4.670)}
\gppoint{gp mark 0}{(6.516,4.312)}
\gppoint{gp mark 0}{(6.516,4.433)}
\gppoint{gp mark 0}{(6.516,4.420)}
\gppoint{gp mark 0}{(6.516,4.945)}
\gppoint{gp mark 0}{(6.516,5.097)}
\gppoint{gp mark 0}{(6.516,4.715)}
\gppoint{gp mark 0}{(6.516,5.108)}
\gppoint{gp mark 0}{(6.516,4.204)}
\gppoint{gp mark 0}{(6.516,4.570)}
\gppoint{gp mark 0}{(6.516,4.341)}
\gppoint{gp mark 0}{(6.516,4.823)}
\gppoint{gp mark 0}{(6.516,4.883)}
\gppoint{gp mark 0}{(6.516,4.897)}
\gppoint{gp mark 0}{(6.516,5.028)}
\gppoint{gp mark 0}{(6.516,4.549)}
\gppoint{gp mark 0}{(6.516,4.791)}
\gppoint{gp mark 0}{(6.516,4.570)}
\gppoint{gp mark 0}{(6.516,5.046)}
\gppoint{gp mark 0}{(6.516,4.560)}
\gppoint{gp mark 0}{(6.516,4.651)}
\gppoint{gp mark 0}{(6.516,4.750)}
\gppoint{gp mark 0}{(6.516,4.925)}
\gppoint{gp mark 0}{(6.516,5.167)}
\gppoint{gp mark 0}{(6.516,4.750)}
\gppoint{gp mark 0}{(6.516,4.099)}
\gppoint{gp mark 0}{(6.516,4.978)}
\gppoint{gp mark 0}{(6.516,4.570)}
\gppoint{gp mark 0}{(6.516,4.706)}
\gppoint{gp mark 0}{(6.516,5.135)}
\gppoint{gp mark 0}{(6.516,4.670)}
\gppoint{gp mark 0}{(6.516,4.641)}
\gppoint{gp mark 0}{(6.516,4.846)}
\gppoint{gp mark 0}{(6.516,4.741)}
\gppoint{gp mark 0}{(6.516,4.897)}
\gppoint{gp mark 0}{(6.516,5.040)}
\gppoint{gp mark 0}{(6.516,4.504)}
\gppoint{gp mark 0}{(6.516,4.538)}
\gppoint{gp mark 0}{(6.516,4.570)}
\gppoint{gp mark 0}{(6.531,5.034)}
\gppoint{gp mark 0}{(6.531,4.516)}
\gppoint{gp mark 0}{(6.531,5.108)}
\gppoint{gp mark 0}{(6.531,4.469)}
\gppoint{gp mark 0}{(6.531,4.750)}
\gppoint{gp mark 0}{(6.531,4.469)}
\gppoint{gp mark 0}{(6.531,4.978)}
\gppoint{gp mark 0}{(6.531,4.715)}
\gppoint{gp mark 0}{(6.531,4.868)}
\gppoint{gp mark 0}{(6.531,4.697)}
\gppoint{gp mark 0}{(6.531,5.177)}
\gppoint{gp mark 0}{(6.531,4.991)}
\gppoint{gp mark 0}{(6.531,5.659)}
\gppoint{gp mark 0}{(6.531,4.938)}
\gppoint{gp mark 0}{(6.531,4.420)}
\gppoint{gp mark 0}{(6.531,4.408)}
\gppoint{gp mark 0}{(6.531,4.612)}
\gppoint{gp mark 0}{(6.531,4.925)}
\gppoint{gp mark 0}{(6.531,5.034)}
\gppoint{gp mark 0}{(6.531,4.741)}
\gppoint{gp mark 0}{(6.531,4.697)}
\gppoint{gp mark 0}{(6.531,5.103)}
\gppoint{gp mark 0}{(6.531,4.493)}
\gppoint{gp mark 0}{(6.531,4.268)}
\gppoint{gp mark 0}{(6.531,4.570)}
\gppoint{gp mark 0}{(6.531,4.991)}
\gppoint{gp mark 0}{(6.531,4.791)}
\gppoint{gp mark 0}{(6.531,5.290)}
\gppoint{gp mark 0}{(6.531,4.868)}
\gppoint{gp mark 0}{(6.531,4.601)}
\gppoint{gp mark 0}{(6.531,5.028)}
\gppoint{gp mark 0}{(6.531,4.846)}
\gppoint{gp mark 0}{(6.531,4.516)}
\gppoint{gp mark 0}{(6.531,4.868)}
\gppoint{gp mark 0}{(6.531,4.697)}
\gppoint{gp mark 0}{(6.531,4.846)}
\gppoint{gp mark 0}{(6.531,5.097)}
\gppoint{gp mark 0}{(6.531,5.108)}
\gppoint{gp mark 0}{(6.531,4.733)}
\gppoint{gp mark 0}{(6.531,4.846)}
\gppoint{gp mark 0}{(6.545,4.504)}
\gppoint{gp mark 0}{(6.545,4.911)}
\gppoint{gp mark 0}{(6.545,4.651)}
\gppoint{gp mark 0}{(6.545,5.235)}
\gppoint{gp mark 0}{(6.545,5.156)}
\gppoint{gp mark 0}{(6.545,5.022)}
\gppoint{gp mark 0}{(6.545,4.268)}
\gppoint{gp mark 0}{(6.545,4.481)}
\gppoint{gp mark 0}{(6.545,4.581)}
\gppoint{gp mark 0}{(6.545,5.259)}
\gppoint{gp mark 0}{(6.545,4.741)}
\gppoint{gp mark 0}{(6.545,4.612)}
\gppoint{gp mark 0}{(6.545,5.492)}
\gppoint{gp mark 0}{(6.545,5.058)}
\gppoint{gp mark 0}{(6.545,5.028)}
\gppoint{gp mark 0}{(6.545,4.767)}
\gppoint{gp mark 0}{(6.545,5.022)}
\gppoint{gp mark 0}{(6.545,4.481)}
\gppoint{gp mark 0}{(6.545,5.028)}
\gppoint{gp mark 0}{(6.545,4.612)}
\gppoint{gp mark 0}{(6.545,5.063)}
\gppoint{gp mark 0}{(6.545,4.549)}
\gppoint{gp mark 0}{(6.545,4.457)}
\gppoint{gp mark 0}{(6.545,4.958)}
\gppoint{gp mark 0}{(6.545,5.063)}
\gppoint{gp mark 0}{(6.545,4.651)}
\gppoint{gp mark 0}{(6.545,5.146)}
\gppoint{gp mark 0}{(6.545,4.890)}
\gppoint{gp mark 0}{(6.545,4.651)}
\gppoint{gp mark 0}{(6.545,5.022)}
\gppoint{gp mark 0}{(6.545,5.249)}
\gppoint{gp mark 0}{(6.545,4.775)}
\gppoint{gp mark 0}{(6.545,5.058)}
\gppoint{gp mark 0}{(6.545,4.601)}
\gppoint{gp mark 0}{(6.545,4.758)}
\gppoint{gp mark 0}{(6.545,4.660)}
\gppoint{gp mark 0}{(6.545,4.750)}
\gppoint{gp mark 0}{(6.559,4.631)}
\gppoint{gp mark 0}{(6.559,4.527)}
\gppoint{gp mark 0}{(6.559,4.932)}
\gppoint{gp mark 0}{(6.559,4.807)}
\gppoint{gp mark 0}{(6.559,5.092)}
\gppoint{gp mark 0}{(6.559,4.420)}
\gppoint{gp mark 0}{(6.559,5.690)}
\gppoint{gp mark 0}{(6.559,4.853)}
\gppoint{gp mark 0}{(6.559,4.952)}
\gppoint{gp mark 0}{(6.559,4.631)}
\gppoint{gp mark 0}{(6.559,4.775)}
\gppoint{gp mark 0}{(6.559,5.040)}
\gppoint{gp mark 0}{(6.559,4.958)}
\gppoint{gp mark 0}{(6.559,4.601)}
\gppoint{gp mark 0}{(6.559,4.791)}
\gppoint{gp mark 0}{(6.559,4.570)}
\gppoint{gp mark 0}{(6.559,4.715)}
\gppoint{gp mark 0}{(6.559,4.433)}
\gppoint{gp mark 0}{(6.559,5.259)}
\gppoint{gp mark 0}{(6.559,4.549)}
\gppoint{gp mark 0}{(6.559,4.481)}
\gppoint{gp mark 0}{(6.559,4.799)}
\gppoint{gp mark 0}{(6.559,4.688)}
\gppoint{gp mark 0}{(6.559,5.207)}
\gppoint{gp mark 0}{(6.559,4.945)}
\gppoint{gp mark 0}{(6.559,4.984)}
\gppoint{gp mark 0}{(6.559,5.114)}
\gppoint{gp mark 0}{(6.559,4.918)}
\gppoint{gp mark 0}{(6.559,4.612)}
\gppoint{gp mark 0}{(6.559,4.724)}
\gppoint{gp mark 0}{(6.559,4.911)}
\gppoint{gp mark 0}{(6.559,4.549)}
\gppoint{gp mark 0}{(6.559,4.750)}
\gppoint{gp mark 0}{(6.559,4.868)}
\gppoint{gp mark 0}{(6.559,4.631)}
\gppoint{gp mark 0}{(6.559,4.724)}
\gppoint{gp mark 0}{(6.573,4.972)}
\gppoint{gp mark 0}{(6.573,4.660)}
\gppoint{gp mark 0}{(6.573,4.799)}
\gppoint{gp mark 0}{(6.573,4.298)}
\gppoint{gp mark 0}{(6.573,4.741)}
\gppoint{gp mark 0}{(6.573,4.972)}
\gppoint{gp mark 0}{(6.573,4.965)}
\gppoint{gp mark 0}{(6.573,4.767)}
\gppoint{gp mark 0}{(6.573,4.991)}
\gppoint{gp mark 0}{(6.573,4.457)}
\gppoint{gp mark 0}{(6.573,4.965)}
\gppoint{gp mark 0}{(6.573,5.151)}
\gppoint{gp mark 0}{(6.573,4.298)}
\gppoint{gp mark 0}{(6.573,4.750)}
\gppoint{gp mark 0}{(6.573,5.092)}
\gppoint{gp mark 0}{(6.573,4.815)}
\gppoint{gp mark 0}{(6.573,4.621)}
\gppoint{gp mark 0}{(6.573,5.141)}
\gppoint{gp mark 0}{(6.573,5.177)}
\gppoint{gp mark 0}{(6.573,4.883)}
\gppoint{gp mark 0}{(6.573,4.651)}
\gppoint{gp mark 0}{(6.573,4.815)}
\gppoint{gp mark 0}{(6.573,5.141)}
\gppoint{gp mark 0}{(6.573,4.838)}
\gppoint{gp mark 0}{(6.573,4.938)}
\gppoint{gp mark 0}{(6.573,4.651)}
\gppoint{gp mark 0}{(6.573,5.063)}
\gppoint{gp mark 0}{(6.573,4.945)}
\gppoint{gp mark 0}{(6.573,4.549)}
\gppoint{gp mark 0}{(6.573,4.724)}
\gppoint{gp mark 0}{(6.573,4.758)}
\gppoint{gp mark 0}{(6.587,4.925)}
\gppoint{gp mark 0}{(6.587,4.775)}
\gppoint{gp mark 0}{(6.587,4.791)}
\gppoint{gp mark 0}{(6.587,5.016)}
\gppoint{gp mark 0}{(6.587,5.010)}
\gppoint{gp mark 0}{(6.587,4.978)}
\gppoint{gp mark 0}{(6.587,4.775)}
\gppoint{gp mark 0}{(6.587,5.022)}
\gppoint{gp mark 0}{(6.587,5.378)}
\gppoint{gp mark 0}{(6.587,4.612)}
\gppoint{gp mark 0}{(6.587,4.853)}
\gppoint{gp mark 0}{(6.587,4.733)}
\gppoint{gp mark 0}{(6.587,4.853)}
\gppoint{gp mark 0}{(6.587,5.345)}
\gppoint{gp mark 0}{(6.587,4.932)}
\gppoint{gp mark 0}{(6.587,5.695)}
\gppoint{gp mark 0}{(6.587,5.337)}
\gppoint{gp mark 0}{(6.587,4.938)}
\gppoint{gp mark 0}{(6.587,4.938)}
\gppoint{gp mark 0}{(6.587,4.368)}
\gppoint{gp mark 0}{(6.587,4.527)}
\gppoint{gp mark 0}{(6.587,5.010)}
\gppoint{gp mark 0}{(6.587,5.108)}
\gppoint{gp mark 0}{(6.587,4.890)}
\gppoint{gp mark 0}{(6.587,4.807)}
\gppoint{gp mark 0}{(6.587,4.945)}
\gppoint{gp mark 0}{(6.587,4.445)}
\gppoint{gp mark 0}{(6.587,4.807)}
\gppoint{gp mark 0}{(6.587,4.651)}
\gppoint{gp mark 0}{(6.587,4.724)}
\gppoint{gp mark 0}{(6.587,5.022)}
\gppoint{gp mark 0}{(6.587,4.433)}
\gppoint{gp mark 0}{(6.587,5.187)}
\gppoint{gp mark 0}{(6.587,4.733)}
\gppoint{gp mark 0}{(6.587,4.904)}
\gppoint{gp mark 0}{(6.587,5.069)}
\gppoint{gp mark 0}{(6.587,4.984)}
\gppoint{gp mark 0}{(6.587,4.846)}
\gppoint{gp mark 0}{(6.587,4.660)}
\gppoint{gp mark 0}{(6.587,5.177)}
\gppoint{gp mark 0}{(6.587,4.838)}
\gppoint{gp mark 0}{(6.587,4.549)}
\gppoint{gp mark 0}{(6.587,4.767)}
\gppoint{gp mark 0}{(6.587,5.010)}
\gppoint{gp mark 0}{(6.601,5.010)}
\gppoint{gp mark 0}{(6.601,4.883)}
\gppoint{gp mark 0}{(6.601,4.868)}
\gppoint{gp mark 0}{(6.601,5.016)}
\gppoint{gp mark 0}{(6.601,4.952)}
\gppoint{gp mark 0}{(6.601,4.883)}
\gppoint{gp mark 0}{(6.601,4.758)}
\gppoint{gp mark 0}{(6.601,4.171)}
\gppoint{gp mark 0}{(6.601,5.197)}
\gppoint{gp mark 0}{(6.601,4.823)}
\gppoint{gp mark 0}{(6.601,4.815)}
\gppoint{gp mark 0}{(6.601,4.997)}
\gppoint{gp mark 0}{(6.601,4.783)}
\gppoint{gp mark 0}{(6.601,4.938)}
\gppoint{gp mark 0}{(6.601,4.945)}
\gppoint{gp mark 0}{(6.601,4.715)}
\gppoint{gp mark 0}{(6.601,4.679)}
\gppoint{gp mark 0}{(6.601,4.938)}
\gppoint{gp mark 0}{(6.601,4.538)}
\gppoint{gp mark 0}{(6.601,4.715)}
\gppoint{gp mark 0}{(6.601,4.688)}
\gppoint{gp mark 0}{(6.601,4.591)}
\gppoint{gp mark 0}{(6.601,4.538)}
\gppoint{gp mark 0}{(6.601,4.679)}
\gppoint{gp mark 0}{(6.601,4.381)}
\gppoint{gp mark 0}{(6.601,5.216)}
\gppoint{gp mark 0}{(6.601,4.354)}
\gppoint{gp mark 0}{(6.601,4.591)}
\gppoint{gp mark 0}{(6.601,4.807)}
\gppoint{gp mark 0}{(6.601,4.612)}
\gppoint{gp mark 0}{(6.601,4.861)}
\gppoint{gp mark 0}{(6.601,4.904)}
\gppoint{gp mark 0}{(6.601,4.581)}
\gppoint{gp mark 0}{(6.601,4.549)}
\gppoint{gp mark 0}{(6.601,5.086)}
\gppoint{gp mark 0}{(6.601,5.040)}
\gppoint{gp mark 0}{(6.601,4.457)}
\gppoint{gp mark 0}{(6.601,4.469)}
\gppoint{gp mark 0}{(6.601,5.092)}
\gppoint{gp mark 0}{(6.601,4.791)}
\gppoint{gp mark 0}{(6.614,5.040)}
\gppoint{gp mark 0}{(6.614,4.846)}
\gppoint{gp mark 0}{(6.614,4.697)}
\gppoint{gp mark 0}{(6.614,5.125)}
\gppoint{gp mark 0}{(6.614,4.972)}
\gppoint{gp mark 0}{(6.614,4.853)}
\gppoint{gp mark 0}{(6.614,4.911)}
\gppoint{gp mark 0}{(6.614,4.861)}
\gppoint{gp mark 0}{(6.614,4.783)}
\gppoint{gp mark 0}{(6.614,4.911)}
\gppoint{gp mark 0}{(6.614,4.408)}
\gppoint{gp mark 0}{(6.614,5.299)}
\gppoint{gp mark 0}{(6.614,5.125)}
\gppoint{gp mark 0}{(6.614,4.527)}
\gppoint{gp mark 0}{(6.614,4.715)}
\gppoint{gp mark 0}{(6.614,5.086)}
\gppoint{gp mark 0}{(6.614,4.875)}
\gppoint{gp mark 0}{(6.614,4.945)}
\gppoint{gp mark 0}{(6.614,4.823)}
\gppoint{gp mark 0}{(6.614,4.688)}
\gppoint{gp mark 0}{(6.614,4.868)}
\gppoint{gp mark 0}{(6.614,5.245)}
\gppoint{gp mark 0}{(6.614,4.395)}
\gppoint{gp mark 0}{(6.614,4.897)}
\gppoint{gp mark 0}{(6.614,4.791)}
\gppoint{gp mark 0}{(6.614,5.207)}
\gppoint{gp mark 0}{(6.614,4.846)}
\gppoint{gp mark 0}{(6.614,4.911)}
\gppoint{gp mark 0}{(6.614,4.861)}
\gppoint{gp mark 0}{(6.614,5.167)}
\gppoint{gp mark 0}{(6.614,4.395)}
\gppoint{gp mark 0}{(6.614,5.063)}
\gppoint{gp mark 0}{(6.614,4.612)}
\gppoint{gp mark 0}{(6.614,4.679)}
\gppoint{gp mark 0}{(6.614,4.875)}
\gppoint{gp mark 0}{(6.614,4.688)}
\gppoint{gp mark 0}{(6.628,5.046)}
\gppoint{gp mark 0}{(6.628,4.651)}
\gppoint{gp mark 0}{(6.628,4.978)}
\gppoint{gp mark 0}{(6.628,4.978)}
\gppoint{gp mark 0}{(6.628,4.978)}
\gppoint{gp mark 0}{(6.628,5.272)}
\gppoint{gp mark 0}{(6.628,4.978)}
\gppoint{gp mark 0}{(6.628,4.875)}
\gppoint{gp mark 0}{(6.628,4.978)}
\gppoint{gp mark 0}{(6.628,4.697)}
\gppoint{gp mark 0}{(6.628,4.750)}
\gppoint{gp mark 0}{(6.628,4.945)}
\gppoint{gp mark 0}{(6.628,4.741)}
\gppoint{gp mark 0}{(6.628,5.197)}
\gppoint{gp mark 0}{(6.628,4.581)}
\gppoint{gp mark 0}{(6.628,4.815)}
\gppoint{gp mark 0}{(6.628,4.965)}
\gppoint{gp mark 0}{(6.628,4.904)}
\gppoint{gp mark 0}{(6.628,4.445)}
\gppoint{gp mark 0}{(6.628,4.861)}
\gppoint{gp mark 0}{(6.628,4.984)}
\gppoint{gp mark 0}{(6.628,4.890)}
\gppoint{gp mark 0}{(6.628,4.846)}
\gppoint{gp mark 0}{(6.628,4.984)}
\gppoint{gp mark 0}{(6.628,4.904)}
\gppoint{gp mark 0}{(6.628,5.092)}
\gppoint{gp mark 0}{(6.628,4.904)}
\gppoint{gp mark 0}{(6.628,4.581)}
\gppoint{gp mark 0}{(6.628,4.679)}
\gppoint{gp mark 0}{(6.628,5.172)}
\gppoint{gp mark 0}{(6.628,5.478)}
\gppoint{gp mark 0}{(6.628,4.775)}
\gppoint{gp mark 0}{(6.628,4.741)}
\gppoint{gp mark 0}{(6.628,5.081)}
\gppoint{gp mark 0}{(6.628,5.135)}
\gppoint{gp mark 0}{(6.628,4.538)}
\gppoint{gp mark 0}{(6.628,4.481)}
\gppoint{gp mark 0}{(6.628,4.997)}
\gppoint{gp mark 0}{(6.641,4.938)}
\gppoint{gp mark 0}{(6.641,5.211)}
\gppoint{gp mark 0}{(6.641,5.092)}
\gppoint{gp mark 0}{(6.641,5.058)}
\gppoint{gp mark 0}{(6.641,4.591)}
\gppoint{gp mark 0}{(6.641,4.767)}
\gppoint{gp mark 0}{(6.641,4.938)}
\gppoint{gp mark 0}{(6.641,5.268)}
\gppoint{gp mark 0}{(6.641,4.883)}
\gppoint{gp mark 0}{(6.641,4.991)}
\gppoint{gp mark 0}{(6.641,4.354)}
\gppoint{gp mark 0}{(6.641,5.211)}
\gppoint{gp mark 0}{(6.641,4.591)}
\gppoint{gp mark 0}{(6.641,4.688)}
\gppoint{gp mark 0}{(6.641,4.890)}
\gppoint{gp mark 0}{(6.641,5.022)}
\gppoint{gp mark 0}{(6.641,4.395)}
\gppoint{gp mark 0}{(6.641,5.022)}
\gppoint{gp mark 0}{(6.641,4.991)}
\gppoint{gp mark 0}{(6.641,5.063)}
\gppoint{gp mark 0}{(6.641,5.075)}
\gppoint{gp mark 0}{(6.641,4.706)}
\gppoint{gp mark 0}{(6.641,5.211)}
\gppoint{gp mark 0}{(6.641,4.767)}
\gppoint{gp mark 0}{(6.641,4.408)}
\gppoint{gp mark 0}{(6.641,4.938)}
\gppoint{gp mark 0}{(6.641,4.679)}
\gppoint{gp mark 0}{(6.641,4.741)}
\gppoint{gp mark 0}{(6.641,4.918)}
\gppoint{gp mark 0}{(6.641,5.532)}
\gppoint{gp mark 0}{(6.641,5.316)}
\gppoint{gp mark 0}{(6.641,4.591)}
\gppoint{gp mark 0}{(6.641,5.366)}
\gppoint{gp mark 0}{(6.641,5.197)}
\gppoint{gp mark 0}{(6.641,4.560)}
\gppoint{gp mark 0}{(6.641,4.807)}
\gppoint{gp mark 0}{(6.641,4.846)}
\gppoint{gp mark 0}{(6.654,4.679)}
\gppoint{gp mark 0}{(6.654,4.631)}
\gppoint{gp mark 0}{(6.654,5.003)}
\gppoint{gp mark 0}{(6.654,5.374)}
\gppoint{gp mark 0}{(6.654,4.775)}
\gppoint{gp mark 0}{(6.654,5.022)}
\gppoint{gp mark 0}{(6.654,4.538)}
\gppoint{gp mark 0}{(6.654,4.799)}
\gppoint{gp mark 0}{(6.654,4.706)}
\gppoint{gp mark 0}{(6.654,4.538)}
\gppoint{gp mark 0}{(6.654,4.581)}
\gppoint{gp mark 0}{(6.654,5.075)}
\gppoint{gp mark 0}{(6.654,4.823)}
\gppoint{gp mark 0}{(6.654,4.904)}
\gppoint{gp mark 0}{(6.654,5.063)}
\gppoint{gp mark 0}{(6.654,5.141)}
\gppoint{gp mark 0}{(6.654,4.601)}
\gppoint{gp mark 0}{(6.654,4.904)}
\gppoint{gp mark 0}{(6.654,5.254)}
\gppoint{gp mark 0}{(6.654,4.868)}
\gppoint{gp mark 0}{(6.654,5.034)}
\gppoint{gp mark 0}{(6.654,4.733)}
\gppoint{gp mark 0}{(6.654,4.846)}
\gppoint{gp mark 0}{(6.654,4.823)}
\gppoint{gp mark 0}{(6.654,5.333)}
\gppoint{gp mark 0}{(6.654,4.758)}
\gppoint{gp mark 0}{(6.654,5.320)}
\gppoint{gp mark 0}{(6.654,4.904)}
\gppoint{gp mark 0}{(6.654,5.254)}
\gppoint{gp mark 0}{(6.654,4.846)}
\gppoint{gp mark 0}{(6.654,4.750)}
\gppoint{gp mark 0}{(6.654,5.135)}
\gppoint{gp mark 0}{(6.654,4.846)}
\gppoint{gp mark 0}{(6.654,4.783)}
\gppoint{gp mark 0}{(6.654,5.046)}
\gppoint{gp mark 0}{(6.654,4.823)}
\gppoint{gp mark 0}{(6.654,5.156)}
\gppoint{gp mark 0}{(6.654,4.846)}
\gppoint{gp mark 0}{(6.654,4.767)}
\gppoint{gp mark 0}{(6.667,5.010)}
\gppoint{gp mark 0}{(6.667,4.660)}
\gppoint{gp mark 0}{(6.667,4.724)}
\gppoint{gp mark 0}{(6.667,4.504)}
\gppoint{gp mark 0}{(6.667,5.281)}
\gppoint{gp mark 0}{(6.667,5.028)}
\gppoint{gp mark 0}{(6.667,4.897)}
\gppoint{gp mark 0}{(6.667,5.172)}
\gppoint{gp mark 0}{(6.667,4.641)}
\gppoint{gp mark 0}{(6.667,4.925)}
\gppoint{gp mark 0}{(6.667,4.733)}
\gppoint{gp mark 0}{(6.667,5.028)}
\gppoint{gp mark 0}{(6.667,4.890)}
\gppoint{gp mark 0}{(6.667,4.838)}
\gppoint{gp mark 0}{(6.667,4.612)}
\gppoint{gp mark 0}{(6.667,4.883)}
\gppoint{gp mark 0}{(6.667,4.846)}
\gppoint{gp mark 0}{(6.667,4.733)}
\gppoint{gp mark 0}{(6.667,4.621)}
\gppoint{gp mark 0}{(6.667,4.945)}
\gppoint{gp mark 0}{(6.667,5.187)}
\gppoint{gp mark 0}{(6.667,4.660)}
\gppoint{gp mark 0}{(6.667,4.965)}
\gppoint{gp mark 0}{(6.667,4.750)}
\gppoint{gp mark 0}{(6.667,4.830)}
\gppoint{gp mark 0}{(6.667,5.003)}
\gppoint{gp mark 0}{(6.667,4.601)}
\gppoint{gp mark 0}{(6.667,5.299)}
\gppoint{gp mark 0}{(6.667,5.075)}
\gppoint{gp mark 0}{(6.667,5.231)}
\gppoint{gp mark 0}{(6.667,5.221)}
\gppoint{gp mark 0}{(6.667,4.641)}
\gppoint{gp mark 0}{(6.667,4.883)}
\gppoint{gp mark 0}{(6.667,5.231)}
\gppoint{gp mark 0}{(6.667,4.868)}
\gppoint{gp mark 0}{(6.667,4.527)}
\gppoint{gp mark 0}{(6.667,5.003)}
\gppoint{gp mark 0}{(6.667,4.741)}
\gppoint{gp mark 0}{(6.667,4.925)}
\gppoint{gp mark 0}{(6.680,4.972)}
\gppoint{gp mark 0}{(6.680,4.883)}
\gppoint{gp mark 0}{(6.680,5.052)}
\gppoint{gp mark 0}{(6.680,4.991)}
\gppoint{gp mark 0}{(6.680,5.058)}
\gppoint{gp mark 0}{(6.680,4.758)}
\gppoint{gp mark 0}{(6.680,5.146)}
\gppoint{gp mark 0}{(6.680,4.679)}
\gppoint{gp mark 0}{(6.680,4.724)}
\gppoint{gp mark 0}{(6.680,5.034)}
\gppoint{gp mark 0}{(6.680,4.972)}
\gppoint{gp mark 0}{(6.680,5.207)}
\gppoint{gp mark 0}{(6.680,4.823)}
\gppoint{gp mark 0}{(6.680,5.075)}
\gppoint{gp mark 0}{(6.680,5.075)}
\gppoint{gp mark 0}{(6.680,4.791)}
\gppoint{gp mark 0}{(6.680,5.548)}
\gppoint{gp mark 0}{(6.680,4.591)}
\gppoint{gp mark 0}{(6.680,4.408)}
\gppoint{gp mark 0}{(6.680,4.868)}
\gppoint{gp mark 0}{(6.680,5.151)}
\gppoint{gp mark 0}{(6.680,4.958)}
\gppoint{gp mark 0}{(6.680,4.807)}
\gppoint{gp mark 0}{(6.680,5.003)}
\gppoint{gp mark 0}{(6.692,5.069)}
\gppoint{gp mark 0}{(6.692,5.370)}
\gppoint{gp mark 0}{(6.692,4.420)}
\gppoint{gp mark 0}{(6.692,5.125)}
\gppoint{gp mark 0}{(6.692,4.911)}
\gppoint{gp mark 0}{(6.692,5.063)}
\gppoint{gp mark 0}{(6.692,5.135)}
\gppoint{gp mark 0}{(6.692,4.984)}
\gppoint{gp mark 0}{(6.692,4.945)}
\gppoint{gp mark 0}{(6.692,4.925)}
\gppoint{gp mark 0}{(6.692,5.324)}
\gppoint{gp mark 0}{(6.692,5.254)}
\gppoint{gp mark 0}{(6.692,5.010)}
\gppoint{gp mark 0}{(6.692,4.861)}
\gppoint{gp mark 0}{(6.692,4.679)}
\gppoint{gp mark 0}{(6.692,4.911)}
\gppoint{gp mark 0}{(6.692,5.003)}
\gppoint{gp mark 0}{(6.692,5.003)}
\gppoint{gp mark 0}{(6.692,4.911)}
\gppoint{gp mark 0}{(6.692,5.316)}
\gppoint{gp mark 0}{(6.692,5.366)}
\gppoint{gp mark 0}{(6.692,4.938)}
\gppoint{gp mark 0}{(6.692,4.823)}
\gppoint{gp mark 0}{(6.692,4.621)}
\gppoint{gp mark 0}{(6.692,5.086)}
\gppoint{gp mark 0}{(6.692,4.791)}
\gppoint{gp mark 0}{(6.692,4.621)}
\gppoint{gp mark 0}{(6.692,4.767)}
\gppoint{gp mark 0}{(6.692,4.715)}
\gppoint{gp mark 0}{(6.692,5.197)}
\gppoint{gp mark 0}{(6.692,5.108)}
\gppoint{gp mark 0}{(6.692,5.852)}
\gppoint{gp mark 0}{(6.692,4.733)}
\gppoint{gp mark 0}{(6.692,5.086)}
\gppoint{gp mark 0}{(6.692,4.799)}
\gppoint{gp mark 0}{(6.692,5.086)}
\gppoint{gp mark 0}{(6.705,5.192)}
\gppoint{gp mark 0}{(6.705,5.187)}
\gppoint{gp mark 0}{(6.705,5.069)}
\gppoint{gp mark 0}{(6.705,4.883)}
\gppoint{gp mark 0}{(6.705,4.830)}
\gppoint{gp mark 0}{(6.705,4.621)}
\gppoint{gp mark 0}{(6.705,5.231)}
\gppoint{gp mark 0}{(6.705,4.830)}
\gppoint{gp mark 0}{(6.705,5.022)}
\gppoint{gp mark 0}{(6.705,5.075)}
\gppoint{gp mark 0}{(6.705,4.883)}
\gppoint{gp mark 0}{(6.705,4.997)}
\gppoint{gp mark 0}{(6.705,5.063)}
\gppoint{gp mark 0}{(6.705,5.187)}
\gppoint{gp mark 0}{(6.705,4.469)}
\gppoint{gp mark 0}{(6.705,5.187)}
\gppoint{gp mark 0}{(6.705,4.354)}
\gppoint{gp mark 0}{(6.705,4.897)}
\gppoint{gp mark 0}{(6.705,4.715)}
\gppoint{gp mark 0}{(6.705,4.815)}
\gppoint{gp mark 0}{(6.705,4.758)}
\gppoint{gp mark 0}{(6.705,4.838)}
\gppoint{gp mark 0}{(6.705,5.167)}
\gppoint{gp mark 0}{(6.705,5.187)}
\gppoint{gp mark 0}{(6.705,5.385)}
\gppoint{gp mark 0}{(6.705,4.715)}
\gppoint{gp mark 0}{(6.705,5.172)}
\gppoint{gp mark 0}{(6.705,4.724)}
\gppoint{gp mark 0}{(6.705,4.750)}
\gppoint{gp mark 0}{(6.705,4.775)}
\gppoint{gp mark 0}{(6.705,5.022)}
\gppoint{gp mark 0}{(6.705,4.991)}
\gppoint{gp mark 0}{(6.705,4.767)}
\gppoint{gp mark 0}{(6.705,4.807)}
\gppoint{gp mark 0}{(6.717,5.092)}
\gppoint{gp mark 0}{(6.717,5.259)}
\gppoint{gp mark 0}{(6.717,4.978)}
\gppoint{gp mark 0}{(6.717,5.092)}
\gppoint{gp mark 0}{(6.717,5.381)}
\gppoint{gp mark 0}{(6.717,5.092)}
\gppoint{gp mark 0}{(6.717,5.108)}
\gppoint{gp mark 0}{(6.717,5.092)}
\gppoint{gp mark 0}{(6.717,5.374)}
\gppoint{gp mark 0}{(6.717,5.254)}
\gppoint{gp mark 0}{(6.717,5.092)}
\gppoint{gp mark 0}{(6.717,4.601)}
\gppoint{gp mark 0}{(6.717,4.952)}
\gppoint{gp mark 0}{(6.717,5.092)}
\gppoint{gp mark 0}{(6.717,5.092)}
\gppoint{gp mark 0}{(6.717,5.092)}
\gppoint{gp mark 0}{(6.717,5.092)}
\gppoint{gp mark 0}{(6.717,5.063)}
\gppoint{gp mark 0}{(6.717,5.092)}
\gppoint{gp mark 0}{(6.717,5.092)}
\gppoint{gp mark 0}{(6.717,4.846)}
\gppoint{gp mark 0}{(6.717,5.022)}
\gppoint{gp mark 0}{(6.717,5.182)}
\gppoint{gp mark 0}{(6.717,4.875)}
\gppoint{gp mark 0}{(6.717,5.092)}
\gppoint{gp mark 0}{(6.717,4.904)}
\gppoint{gp mark 0}{(6.717,4.838)}
\gppoint{gp mark 0}{(6.717,5.034)}
\gppoint{gp mark 0}{(6.717,5.146)}
\gppoint{gp mark 0}{(6.717,5.092)}
\gppoint{gp mark 0}{(6.717,5.092)}
\gppoint{gp mark 0}{(6.717,5.092)}
\gppoint{gp mark 0}{(6.717,5.254)}
\gppoint{gp mark 0}{(6.717,5.092)}
\gppoint{gp mark 0}{(6.717,5.092)}
\gppoint{gp mark 0}{(6.717,5.092)}
\gppoint{gp mark 0}{(6.717,4.846)}
\gppoint{gp mark 0}{(6.717,5.092)}
\gppoint{gp mark 0}{(6.717,5.092)}
\gppoint{gp mark 0}{(6.717,4.327)}
\gppoint{gp mark 0}{(6.717,5.092)}
\gppoint{gp mark 0}{(6.717,5.075)}
\gppoint{gp mark 0}{(6.717,4.838)}
\gppoint{gp mark 0}{(6.717,5.221)}
\gppoint{gp mark 0}{(6.717,4.823)}
\gppoint{gp mark 0}{(6.717,4.670)}
\gppoint{gp mark 0}{(6.717,4.911)}
\gppoint{gp mark 0}{(6.717,5.058)}
\gppoint{gp mark 0}{(6.717,4.830)}
\gppoint{gp mark 0}{(6.717,5.092)}
\gppoint{gp mark 0}{(6.717,4.823)}
\gppoint{gp mark 0}{(6.717,5.058)}
\gppoint{gp mark 0}{(6.717,5.092)}
\gppoint{gp mark 0}{(6.717,5.092)}
\gppoint{gp mark 0}{(6.717,4.601)}
\gppoint{gp mark 0}{(6.729,4.904)}
\gppoint{gp mark 0}{(6.729,5.249)}
\gppoint{gp mark 0}{(6.729,5.058)}
\gppoint{gp mark 0}{(6.729,4.932)}
\gppoint{gp mark 0}{(6.729,5.046)}
\gppoint{gp mark 0}{(6.729,5.010)}
\gppoint{gp mark 0}{(6.729,5.086)}
\gppoint{gp mark 0}{(6.729,5.081)}
\gppoint{gp mark 0}{(6.729,4.767)}
\gppoint{gp mark 0}{(6.729,4.911)}
\gppoint{gp mark 0}{(6.729,4.724)}
\gppoint{gp mark 0}{(6.729,4.911)}
\gppoint{gp mark 0}{(6.729,4.911)}
\gppoint{gp mark 0}{(6.729,4.188)}
\gppoint{gp mark 0}{(6.729,4.997)}
\gppoint{gp mark 0}{(6.729,5.263)}
\gppoint{gp mark 0}{(6.729,4.911)}
\gppoint{gp mark 0}{(6.729,5.263)}
\gppoint{gp mark 0}{(6.729,4.911)}
\gppoint{gp mark 0}{(6.729,5.529)}
\gppoint{gp mark 0}{(6.729,4.741)}
\gppoint{gp mark 0}{(6.729,4.853)}
\gppoint{gp mark 0}{(6.729,5.207)}
\gppoint{gp mark 0}{(6.729,5.141)}
\gppoint{gp mark 0}{(6.729,5.216)}
\gppoint{gp mark 0}{(6.729,4.570)}
\gppoint{gp mark 0}{(6.729,5.052)}
\gppoint{gp mark 0}{(6.729,4.984)}
\gppoint{gp mark 0}{(6.729,5.108)}
\gppoint{gp mark 0}{(6.729,4.838)}
\gppoint{gp mark 0}{(6.729,5.374)}
\gppoint{gp mark 0}{(6.729,4.938)}
\gppoint{gp mark 0}{(6.729,5.172)}
\gppoint{gp mark 0}{(6.729,4.952)}
\gppoint{gp mark 0}{(6.742,4.767)}
\gppoint{gp mark 0}{(6.742,4.875)}
\gppoint{gp mark 0}{(6.742,4.670)}
\gppoint{gp mark 0}{(6.742,4.830)}
\gppoint{gp mark 0}{(6.742,6.240)}
\gppoint{gp mark 0}{(6.742,5.135)}
\gppoint{gp mark 0}{(6.742,4.861)}
\gppoint{gp mark 0}{(6.742,5.192)}
\gppoint{gp mark 0}{(6.742,4.670)}
\gppoint{gp mark 0}{(6.742,5.075)}
\gppoint{gp mark 0}{(6.742,4.612)}
\gppoint{gp mark 0}{(6.742,4.945)}
\gppoint{gp mark 0}{(6.742,5.897)}
\gppoint{gp mark 0}{(6.742,4.868)}
\gppoint{gp mark 0}{(6.742,5.162)}
\gppoint{gp mark 0}{(6.742,5.125)}
\gppoint{gp mark 0}{(6.742,5.130)}
\gppoint{gp mark 0}{(6.742,4.965)}
\gppoint{gp mark 0}{(6.742,4.733)}
\gppoint{gp mark 0}{(6.742,4.868)}
\gppoint{gp mark 0}{(6.742,4.958)}
\gppoint{gp mark 0}{(6.742,4.897)}
\gppoint{gp mark 0}{(6.742,5.046)}
\gppoint{gp mark 0}{(6.742,5.277)}
\gppoint{gp mark 0}{(6.742,4.952)}
\gppoint{gp mark 0}{(6.742,5.081)}
\gppoint{gp mark 0}{(6.742,5.263)}
\gppoint{gp mark 0}{(6.742,4.904)}
\gppoint{gp mark 0}{(6.742,4.715)}
\gppoint{gp mark 0}{(6.754,5.063)}
\gppoint{gp mark 0}{(6.754,4.925)}
\gppoint{gp mark 0}{(6.754,5.370)}
\gppoint{gp mark 0}{(6.754,5.156)}
\gppoint{gp mark 0}{(6.754,4.799)}
\gppoint{gp mark 0}{(6.754,5.235)}
\gppoint{gp mark 0}{(6.754,4.978)}
\gppoint{gp mark 0}{(6.754,4.697)}
\gppoint{gp mark 0}{(6.754,5.046)}
\gppoint{gp mark 0}{(6.754,4.601)}
\gppoint{gp mark 0}{(6.754,4.631)}
\gppoint{gp mark 0}{(6.754,4.750)}
\gppoint{gp mark 0}{(6.754,4.815)}
\gppoint{gp mark 0}{(6.754,5.097)}
\gppoint{gp mark 0}{(6.754,5.370)}
\gppoint{gp mark 0}{(6.754,4.938)}
\gppoint{gp mark 0}{(6.754,4.581)}
\gppoint{gp mark 0}{(6.754,5.345)}
\gppoint{gp mark 0}{(6.754,4.838)}
\gppoint{gp mark 0}{(6.754,4.904)}
\gppoint{gp mark 0}{(6.754,3.979)}
\gppoint{gp mark 0}{(6.754,4.911)}
\gppoint{gp mark 0}{(6.754,5.046)}
\gppoint{gp mark 0}{(6.754,4.861)}
\gppoint{gp mark 0}{(6.754,5.449)}
\gppoint{gp mark 0}{(6.754,4.853)}
\gppoint{gp mark 0}{(6.754,5.119)}
\gppoint{gp mark 0}{(6.754,4.904)}
\gppoint{gp mark 0}{(6.754,4.621)}
\gppoint{gp mark 0}{(6.765,5.010)}
\gppoint{gp mark 0}{(6.765,4.715)}
\gppoint{gp mark 0}{(6.765,5.172)}
\gppoint{gp mark 0}{(6.765,5.182)}
\gppoint{gp mark 0}{(6.765,5.058)}
\gppoint{gp mark 0}{(6.765,4.080)}
\gppoint{gp mark 0}{(6.765,4.972)}
\gppoint{gp mark 0}{(6.765,4.984)}
\gppoint{gp mark 0}{(6.765,5.245)}
\gppoint{gp mark 0}{(6.765,5.119)}
\gppoint{gp mark 0}{(6.765,4.958)}
\gppoint{gp mark 0}{(6.765,5.235)}
\gppoint{gp mark 0}{(6.765,5.141)}
\gppoint{gp mark 0}{(6.765,5.081)}
\gppoint{gp mark 0}{(6.765,4.875)}
\gppoint{gp mark 0}{(6.765,4.972)}
\gppoint{gp mark 0}{(6.765,5.286)}
\gppoint{gp mark 0}{(6.765,5.182)}
\gppoint{gp mark 0}{(6.765,5.097)}
\gppoint{gp mark 0}{(6.765,4.830)}
\gppoint{gp mark 0}{(6.765,4.875)}
\gppoint{gp mark 0}{(6.765,5.034)}
\gppoint{gp mark 0}{(6.765,5.092)}
\gppoint{gp mark 0}{(6.765,5.956)}
\gppoint{gp mark 0}{(6.765,5.135)}
\gppoint{gp mark 0}{(6.765,5.187)}
\gppoint{gp mark 0}{(6.765,4.758)}
\gppoint{gp mark 0}{(6.765,4.741)}
\gppoint{gp mark 0}{(6.765,5.130)}
\gppoint{gp mark 0}{(6.765,4.897)}
\gppoint{gp mark 0}{(6.765,5.141)}
\gppoint{gp mark 0}{(6.765,4.830)}
\gppoint{gp mark 0}{(6.777,4.952)}
\gppoint{gp mark 0}{(6.777,5.272)}
\gppoint{gp mark 0}{(6.777,4.911)}
\gppoint{gp mark 0}{(6.777,4.838)}
\gppoint{gp mark 0}{(6.777,4.750)}
\gppoint{gp mark 0}{(6.777,5.141)}
\gppoint{gp mark 0}{(6.777,4.846)}
\gppoint{gp mark 0}{(6.777,5.022)}
\gppoint{gp mark 0}{(6.777,4.807)}
\gppoint{gp mark 0}{(6.777,5.022)}
\gppoint{gp mark 0}{(6.777,5.156)}
\gppoint{gp mark 0}{(6.777,4.846)}
\gppoint{gp mark 0}{(6.777,5.016)}
\gppoint{gp mark 0}{(6.777,5.146)}
\gppoint{gp mark 0}{(6.777,4.688)}
\gppoint{gp mark 0}{(6.777,5.192)}
\gppoint{gp mark 0}{(6.777,5.416)}
\gppoint{gp mark 0}{(6.777,5.268)}
\gppoint{gp mark 0}{(6.777,4.651)}
\gppoint{gp mark 0}{(6.777,5.003)}
\gppoint{gp mark 0}{(6.777,5.268)}
\gppoint{gp mark 0}{(6.777,4.861)}
\gppoint{gp mark 0}{(6.777,5.052)}
\gppoint{gp mark 0}{(6.777,5.141)}
\gppoint{gp mark 0}{(6.777,4.791)}
\gppoint{gp mark 0}{(6.777,4.697)}
\gppoint{gp mark 0}{(6.777,4.815)}
\gppoint{gp mark 0}{(6.777,4.984)}
\gppoint{gp mark 0}{(6.789,4.897)}
\gppoint{gp mark 0}{(6.789,5.034)}
\gppoint{gp mark 0}{(6.789,5.324)}
\gppoint{gp mark 0}{(6.789,4.758)}
\gppoint{gp mark 0}{(6.789,5.345)}
\gppoint{gp mark 0}{(6.789,4.984)}
\gppoint{gp mark 0}{(6.789,4.679)}
\gppoint{gp mark 0}{(6.789,5.182)}
\gppoint{gp mark 0}{(6.789,5.063)}
\gppoint{gp mark 0}{(6.789,5.167)}
\gppoint{gp mark 0}{(6.789,4.601)}
\gppoint{gp mark 0}{(6.789,5.141)}
\gppoint{gp mark 0}{(6.789,5.202)}
\gppoint{gp mark 0}{(6.789,4.875)}
\gppoint{gp mark 0}{(6.789,4.925)}
\gppoint{gp mark 0}{(6.789,4.918)}
\gppoint{gp mark 0}{(6.789,5.281)}
\gppoint{gp mark 0}{(6.789,5.202)}
\gppoint{gp mark 0}{(6.789,4.918)}
\gppoint{gp mark 0}{(6.789,5.141)}
\gppoint{gp mark 0}{(6.789,5.202)}
\gppoint{gp mark 0}{(6.789,5.226)}
\gppoint{gp mark 0}{(6.789,5.130)}
\gppoint{gp mark 0}{(6.789,5.130)}
\gppoint{gp mark 0}{(6.789,4.875)}
\gppoint{gp mark 0}{(6.789,4.783)}
\gppoint{gp mark 0}{(6.789,5.034)}
\gppoint{gp mark 0}{(6.789,4.783)}
\gppoint{gp mark 0}{(6.789,5.058)}
\gppoint{gp mark 0}{(6.789,5.177)}
\gppoint{gp mark 0}{(6.789,5.353)}
\gppoint{gp mark 0}{(6.789,4.670)}
\gppoint{gp mark 0}{(6.789,4.875)}
\gppoint{gp mark 0}{(6.789,5.187)}
\gppoint{gp mark 0}{(6.789,4.750)}
\gppoint{gp mark 0}{(6.789,4.991)}
\gppoint{gp mark 0}{(6.789,4.733)}
\gppoint{gp mark 0}{(6.789,4.767)}
\gppoint{gp mark 0}{(6.789,5.034)}
\gppoint{gp mark 0}{(6.789,5.108)}
\gppoint{gp mark 0}{(6.789,5.146)}
\gppoint{gp mark 0}{(6.789,5.058)}
\gppoint{gp mark 0}{(6.800,4.932)}
\gppoint{gp mark 0}{(6.800,5.046)}
\gppoint{gp mark 0}{(6.800,5.156)}
\gppoint{gp mark 0}{(6.800,5.320)}
\gppoint{gp mark 0}{(6.800,5.362)}
\gppoint{gp mark 0}{(6.800,5.249)}
\gppoint{gp mark 0}{(6.800,4.724)}
\gppoint{gp mark 0}{(6.800,5.787)}
\gppoint{gp mark 0}{(6.800,4.861)}
\gppoint{gp mark 0}{(6.800,4.758)}
\gppoint{gp mark 0}{(6.800,4.952)}
\gppoint{gp mark 0}{(6.800,5.577)}
\gppoint{gp mark 0}{(6.800,5.299)}
\gppoint{gp mark 0}{(6.800,4.991)}
\gppoint{gp mark 0}{(6.800,5.416)}
\gppoint{gp mark 0}{(6.800,4.799)}
\gppoint{gp mark 0}{(6.800,4.354)}
\gppoint{gp mark 0}{(6.800,5.512)}
\gppoint{gp mark 0}{(6.800,5.711)}
\gppoint{gp mark 0}{(6.800,4.783)}
\gppoint{gp mark 0}{(6.800,4.932)}
\gppoint{gp mark 0}{(6.800,4.354)}
\gppoint{gp mark 0}{(6.800,4.952)}
\gppoint{gp mark 0}{(6.800,4.354)}
\gppoint{gp mark 0}{(6.800,4.621)}
\gppoint{gp mark 0}{(6.800,5.052)}
\gppoint{gp mark 0}{(6.800,4.706)}
\gppoint{gp mark 0}{(6.800,4.997)}
\gppoint{gp mark 0}{(6.800,4.641)}
\gppoint{gp mark 0}{(6.800,4.354)}
\gppoint{gp mark 0}{(6.800,4.570)}
\gppoint{gp mark 0}{(6.800,4.972)}
\gppoint{gp mark 0}{(6.800,4.706)}
\gppoint{gp mark 0}{(6.800,4.972)}
\gppoint{gp mark 0}{(6.800,5.114)}
\gppoint{gp mark 0}{(6.800,4.354)}
\gppoint{gp mark 0}{(6.800,5.086)}
\gppoint{gp mark 0}{(6.800,4.354)}
\gppoint{gp mark 0}{(6.800,4.875)}
\gppoint{gp mark 0}{(6.800,4.354)}
\gppoint{gp mark 0}{(6.800,4.445)}
\gppoint{gp mark 0}{(6.800,5.146)}
\gppoint{gp mark 0}{(6.800,4.952)}
\gppoint{gp mark 0}{(6.800,4.354)}
\gppoint{gp mark 0}{(6.800,4.354)}
\gppoint{gp mark 0}{(6.800,4.354)}
\gppoint{gp mark 0}{(6.800,4.724)}
\gppoint{gp mark 0}{(6.800,4.354)}
\gppoint{gp mark 0}{(6.800,4.354)}
\gppoint{gp mark 0}{(6.800,4.354)}
\gppoint{gp mark 0}{(6.800,5.202)}
\gppoint{gp mark 0}{(6.800,5.052)}
\gppoint{gp mark 0}{(6.812,4.651)}
\gppoint{gp mark 0}{(6.812,5.092)}
\gppoint{gp mark 0}{(6.812,5.268)}
\gppoint{gp mark 0}{(6.812,5.259)}
\gppoint{gp mark 0}{(6.812,4.890)}
\gppoint{gp mark 0}{(6.812,4.641)}
\gppoint{gp mark 0}{(6.812,5.316)}
\gppoint{gp mark 0}{(6.812,5.216)}
\gppoint{gp mark 0}{(6.812,4.938)}
\gppoint{gp mark 0}{(6.812,5.316)}
\gppoint{gp mark 0}{(6.812,5.263)}
\gppoint{gp mark 0}{(6.812,5.316)}
\gppoint{gp mark 0}{(6.812,5.108)}
\gppoint{gp mark 0}{(6.812,4.724)}
\gppoint{gp mark 0}{(6.812,5.022)}
\gppoint{gp mark 0}{(6.812,5.187)}
\gppoint{gp mark 0}{(6.812,4.715)}
\gppoint{gp mark 0}{(6.812,5.114)}
\gppoint{gp mark 0}{(6.812,4.741)}
\gppoint{gp mark 0}{(6.812,5.103)}
\gppoint{gp mark 0}{(6.812,5.046)}
\gppoint{gp mark 0}{(6.812,5.114)}
\gppoint{gp mark 0}{(6.812,5.010)}
\gppoint{gp mark 0}{(6.812,4.354)}
\gppoint{gp mark 0}{(6.812,5.040)}
\gppoint{gp mark 0}{(6.812,5.207)}
\gppoint{gp mark 0}{(6.812,4.952)}
\gppoint{gp mark 0}{(6.812,4.846)}
\gppoint{gp mark 0}{(6.812,4.783)}
\gppoint{gp mark 0}{(6.812,5.286)}
\gppoint{gp mark 0}{(6.812,5.108)}
\gppoint{gp mark 0}{(6.812,5.167)}
\gppoint{gp mark 0}{(6.812,4.783)}
\gppoint{gp mark 0}{(6.823,5.345)}
\gppoint{gp mark 0}{(6.823,4.875)}
\gppoint{gp mark 0}{(6.823,5.307)}
\gppoint{gp mark 0}{(6.823,4.688)}
\gppoint{gp mark 0}{(6.823,4.978)}
\gppoint{gp mark 0}{(6.823,4.952)}
\gppoint{gp mark 0}{(6.823,4.945)}
\gppoint{gp mark 0}{(6.823,4.660)}
\gppoint{gp mark 0}{(6.823,5.146)}
\gppoint{gp mark 0}{(6.823,5.177)}
\gppoint{gp mark 0}{(6.823,4.952)}
\gppoint{gp mark 0}{(6.823,4.932)}
\gppoint{gp mark 0}{(6.823,5.146)}
\gppoint{gp mark 0}{(6.823,5.130)}
\gppoint{gp mark 0}{(6.823,5.146)}
\gppoint{gp mark 0}{(6.823,5.240)}
\gppoint{gp mark 0}{(6.823,4.846)}
\gppoint{gp mark 0}{(6.823,5.281)}
\gppoint{gp mark 0}{(6.823,5.506)}
\gppoint{gp mark 0}{(6.823,4.799)}
\gppoint{gp mark 0}{(6.823,4.932)}
\gppoint{gp mark 0}{(6.823,5.146)}
\gppoint{gp mark 0}{(6.823,4.758)}
\gppoint{gp mark 0}{(6.823,5.146)}
\gppoint{gp mark 0}{(6.823,5.135)}
\gppoint{gp mark 0}{(6.823,4.958)}
\gppoint{gp mark 0}{(6.823,4.823)}
\gppoint{gp mark 0}{(6.823,4.799)}
\gppoint{gp mark 0}{(6.823,4.750)}
\gppoint{gp mark 0}{(6.834,4.991)}
\gppoint{gp mark 0}{(6.834,4.783)}
\gppoint{gp mark 0}{(6.834,5.187)}
\gppoint{gp mark 0}{(6.834,5.312)}
\gppoint{gp mark 0}{(6.834,4.861)}
\gppoint{gp mark 0}{(6.834,5.366)}
\gppoint{gp mark 0}{(6.834,4.984)}
\gppoint{gp mark 0}{(6.834,5.069)}
\gppoint{gp mark 0}{(6.834,5.187)}
\gppoint{gp mark 0}{(6.834,5.231)}
\gppoint{gp mark 0}{(6.834,5.022)}
\gppoint{gp mark 0}{(6.834,5.192)}
\gppoint{gp mark 0}{(6.834,5.381)}
\gppoint{gp mark 0}{(6.834,4.890)}
\gppoint{gp mark 0}{(6.834,4.861)}
\gppoint{gp mark 0}{(6.834,5.512)}
\gppoint{gp mark 0}{(6.834,5.177)}
\gppoint{gp mark 0}{(6.834,5.069)}
\gppoint{gp mark 0}{(6.834,4.312)}
\gppoint{gp mark 0}{(6.834,4.688)}
\gppoint{gp mark 0}{(6.834,5.108)}
\gppoint{gp mark 0}{(6.834,5.177)}
\gppoint{gp mark 0}{(6.834,4.938)}
\gppoint{gp mark 0}{(6.834,5.103)}
\gppoint{gp mark 0}{(6.834,5.069)}
\gppoint{gp mark 0}{(6.834,4.991)}
\gppoint{gp mark 0}{(6.834,5.046)}
\gppoint{gp mark 0}{(6.834,5.357)}
\gppoint{gp mark 0}{(6.834,4.911)}
\gppoint{gp mark 0}{(6.834,5.299)}
\gppoint{gp mark 0}{(6.834,5.141)}
\gppoint{gp mark 0}{(6.834,5.259)}
\gppoint{gp mark 0}{(6.834,4.945)}
\gppoint{gp mark 0}{(6.845,4.823)}
\gppoint{gp mark 0}{(6.845,5.235)}
\gppoint{gp mark 0}{(6.845,5.211)}
\gppoint{gp mark 0}{(6.845,5.022)}
\gppoint{gp mark 0}{(6.845,4.651)}
\gppoint{gp mark 0}{(6.845,5.167)}
\gppoint{gp mark 0}{(6.845,4.733)}
\gppoint{gp mark 0}{(6.845,4.911)}
\gppoint{gp mark 0}{(6.845,5.075)}
\gppoint{gp mark 0}{(6.845,4.853)}
\gppoint{gp mark 0}{(6.845,4.904)}
\gppoint{gp mark 0}{(6.845,4.958)}
\gppoint{gp mark 0}{(6.845,5.249)}
\gppoint{gp mark 0}{(6.845,4.581)}
\gppoint{gp mark 0}{(6.845,4.741)}
\gppoint{gp mark 0}{(6.845,4.631)}
\gppoint{gp mark 0}{(6.845,5.268)}
\gppoint{gp mark 0}{(6.845,5.081)}
\gppoint{gp mark 0}{(6.845,5.103)}
\gppoint{gp mark 0}{(6.845,4.830)}
\gppoint{gp mark 0}{(6.845,5.114)}
\gppoint{gp mark 0}{(6.845,4.408)}
\gppoint{gp mark 0}{(6.845,5.182)}
\gppoint{gp mark 0}{(6.845,5.052)}
\gppoint{gp mark 0}{(6.845,4.846)}
\gppoint{gp mark 0}{(6.845,5.156)}
\gppoint{gp mark 0}{(6.845,5.177)}
\gppoint{gp mark 0}{(6.845,5.324)}
\gppoint{gp mark 0}{(6.845,4.838)}
\gppoint{gp mark 0}{(6.845,4.868)}
\gppoint{gp mark 0}{(6.845,5.366)}
\gppoint{gp mark 0}{(6.845,5.010)}
\gppoint{gp mark 0}{(6.845,5.156)}
\gppoint{gp mark 0}{(6.856,5.108)}
\gppoint{gp mark 0}{(6.856,4.958)}
\gppoint{gp mark 0}{(6.856,4.861)}
\gppoint{gp mark 0}{(6.856,5.221)}
\gppoint{gp mark 0}{(6.856,4.978)}
\gppoint{gp mark 0}{(6.856,4.581)}
\gppoint{gp mark 0}{(6.856,4.670)}
\gppoint{gp mark 0}{(6.856,4.670)}
\gppoint{gp mark 0}{(6.856,5.197)}
\gppoint{gp mark 0}{(6.856,5.162)}
\gppoint{gp mark 0}{(6.856,5.063)}
\gppoint{gp mark 0}{(6.856,5.172)}
\gppoint{gp mark 0}{(6.856,5.130)}
\gppoint{gp mark 0}{(6.856,5.151)}
\gppoint{gp mark 0}{(6.856,5.182)}
\gppoint{gp mark 0}{(6.856,5.034)}
\gppoint{gp mark 0}{(6.856,4.972)}
\gppoint{gp mark 0}{(6.856,5.197)}
\gppoint{gp mark 0}{(6.856,4.724)}
\gppoint{gp mark 0}{(6.856,5.125)}
\gppoint{gp mark 0}{(6.856,5.366)}
\gppoint{gp mark 0}{(6.856,5.167)}
\gppoint{gp mark 0}{(6.856,5.303)}
\gppoint{gp mark 0}{(6.856,4.838)}
\gppoint{gp mark 0}{(6.856,5.240)}
\gppoint{gp mark 0}{(6.856,5.052)}
\gppoint{gp mark 0}{(6.856,4.958)}
\gppoint{gp mark 0}{(6.856,5.172)}
\gppoint{gp mark 0}{(6.867,4.868)}
\gppoint{gp mark 0}{(6.867,4.997)}
\gppoint{gp mark 0}{(6.867,5.028)}
\gppoint{gp mark 0}{(6.867,5.182)}
\gppoint{gp mark 0}{(6.867,4.938)}
\gppoint{gp mark 0}{(6.867,5.362)}
\gppoint{gp mark 0}{(6.867,5.427)}
\gppoint{gp mark 0}{(6.867,5.357)}
\gppoint{gp mark 0}{(6.867,5.028)}
\gppoint{gp mark 0}{(6.867,5.108)}
\gppoint{gp mark 0}{(6.867,5.235)}
\gppoint{gp mark 0}{(6.867,4.952)}
\gppoint{gp mark 0}{(6.867,5.010)}
\gppoint{gp mark 0}{(6.867,5.268)}
\gppoint{gp mark 0}{(6.867,5.069)}
\gppoint{gp mark 0}{(6.867,5.003)}
\gppoint{gp mark 0}{(6.867,5.016)}
\gppoint{gp mark 0}{(6.867,5.397)}
\gppoint{gp mark 0}{(6.867,5.574)}
\gppoint{gp mark 0}{(6.867,5.545)}
\gppoint{gp mark 0}{(6.867,5.069)}
\gppoint{gp mark 0}{(6.867,5.235)}
\gppoint{gp mark 0}{(6.878,5.040)}
\gppoint{gp mark 0}{(6.878,5.187)}
\gppoint{gp mark 0}{(6.878,4.823)}
\gppoint{gp mark 0}{(6.878,4.767)}
\gppoint{gp mark 0}{(6.878,5.097)}
\gppoint{gp mark 0}{(6.878,4.799)}
\gppoint{gp mark 0}{(6.878,4.724)}
\gppoint{gp mark 0}{(6.878,5.114)}
\gppoint{gp mark 0}{(6.878,5.075)}
\gppoint{gp mark 0}{(6.878,5.022)}
\gppoint{gp mark 0}{(6.878,5.187)}
\gppoint{gp mark 0}{(6.878,4.897)}
\gppoint{gp mark 0}{(6.878,4.830)}
\gppoint{gp mark 0}{(6.878,5.589)}
\gppoint{gp mark 0}{(6.878,5.028)}
\gppoint{gp mark 0}{(6.878,5.177)}
\gppoint{gp mark 0}{(6.878,5.182)}
\gppoint{gp mark 0}{(6.878,4.815)}
\gppoint{gp mark 0}{(6.878,5.114)}
\gppoint{gp mark 0}{(6.878,4.997)}
\gppoint{gp mark 0}{(6.878,5.108)}
\gppoint{gp mark 0}{(6.878,4.775)}
\gppoint{gp mark 0}{(6.878,5.028)}
\gppoint{gp mark 0}{(6.878,5.187)}
\gppoint{gp mark 0}{(6.878,5.333)}
\gppoint{gp mark 0}{(6.878,4.799)}
\gppoint{gp mark 0}{(6.889,5.522)}
\gppoint{gp mark 0}{(6.889,5.034)}
\gppoint{gp mark 0}{(6.889,5.125)}
\gppoint{gp mark 0}{(6.889,5.294)}
\gppoint{gp mark 0}{(6.889,4.925)}
\gppoint{gp mark 0}{(6.889,5.125)}
\gppoint{gp mark 0}{(6.889,5.187)}
\gppoint{gp mark 0}{(6.889,5.226)}
\gppoint{gp mark 0}{(6.889,5.240)}
\gppoint{gp mark 0}{(6.889,5.046)}
\gppoint{gp mark 0}{(6.889,4.341)}
\gppoint{gp mark 0}{(6.889,5.268)}
\gppoint{gp mark 0}{(6.889,5.320)}
\gppoint{gp mark 0}{(6.889,5.141)}
\gppoint{gp mark 0}{(6.889,4.972)}
\gppoint{gp mark 0}{(6.889,4.591)}
\gppoint{gp mark 0}{(6.889,5.446)}
\gppoint{gp mark 0}{(6.889,4.972)}
\gppoint{gp mark 0}{(6.889,5.177)}
\gppoint{gp mark 0}{(6.889,5.075)}
\gppoint{gp mark 0}{(6.889,4.433)}
\gppoint{gp mark 0}{(6.889,5.028)}
\gppoint{gp mark 0}{(6.889,5.290)}
\gppoint{gp mark 0}{(6.889,4.724)}
\gppoint{gp mark 0}{(6.889,4.904)}
\gppoint{gp mark 0}{(6.889,4.823)}
\gppoint{gp mark 0}{(6.889,5.167)}
\gppoint{gp mark 0}{(6.889,4.807)}
\gppoint{gp mark 0}{(6.889,5.299)}
\gppoint{gp mark 0}{(6.889,5.028)}
\gppoint{gp mark 0}{(6.889,5.063)}
\gppoint{gp mark 0}{(6.889,5.249)}
\gppoint{gp mark 0}{(6.899,5.177)}
\gppoint{gp mark 0}{(6.899,5.571)}
\gppoint{gp mark 0}{(6.899,5.226)}
\gppoint{gp mark 0}{(6.899,4.932)}
\gppoint{gp mark 0}{(6.899,4.783)}
\gppoint{gp mark 0}{(6.899,4.897)}
\gppoint{gp mark 0}{(6.899,5.187)}
\gppoint{gp mark 0}{(6.899,5.114)}
\gppoint{gp mark 0}{(6.899,5.046)}
\gppoint{gp mark 0}{(6.899,5.294)}
\gppoint{gp mark 0}{(6.899,5.187)}
\gppoint{gp mark 0}{(6.899,5.574)}
\gppoint{gp mark 0}{(6.899,5.010)}
\gppoint{gp mark 0}{(6.899,5.028)}
\gppoint{gp mark 0}{(6.899,5.187)}
\gppoint{gp mark 0}{(6.899,5.069)}
\gppoint{gp mark 0}{(6.899,5.397)}
\gppoint{gp mark 0}{(6.899,5.103)}
\gppoint{gp mark 0}{(6.899,5.097)}
\gppoint{gp mark 0}{(6.910,5.177)}
\gppoint{gp mark 0}{(6.910,4.853)}
\gppoint{gp mark 0}{(6.910,5.345)}
\gppoint{gp mark 0}{(6.910,5.187)}
\gppoint{gp mark 0}{(6.910,5.349)}
\gppoint{gp mark 0}{(6.910,4.706)}
\gppoint{gp mark 0}{(6.910,4.890)}
\gppoint{gp mark 0}{(6.910,5.016)}
\gppoint{gp mark 0}{(6.910,5.286)}
\gppoint{gp mark 0}{(6.910,4.984)}
\gppoint{gp mark 0}{(6.910,4.651)}
\gppoint{gp mark 0}{(6.910,4.991)}
\gppoint{gp mark 0}{(6.910,4.932)}
\gppoint{gp mark 0}{(6.910,5.034)}
\gppoint{gp mark 0}{(6.910,5.034)}
\gppoint{gp mark 0}{(6.910,5.130)}
\gppoint{gp mark 0}{(6.910,5.016)}
\gppoint{gp mark 0}{(6.910,5.345)}
\gppoint{gp mark 0}{(6.910,5.003)}
\gppoint{gp mark 0}{(6.910,4.846)}
\gppoint{gp mark 0}{(6.910,5.249)}
\gppoint{gp mark 0}{(6.910,4.846)}
\gppoint{gp mark 0}{(6.910,4.883)}
\gppoint{gp mark 0}{(6.910,4.911)}
\gppoint{gp mark 0}{(6.910,5.235)}
\gppoint{gp mark 0}{(6.910,5.349)}
\gppoint{gp mark 0}{(6.910,4.445)}
\gppoint{gp mark 0}{(6.910,5.097)}
\gppoint{gp mark 0}{(6.910,5.135)}
\gppoint{gp mark 0}{(6.920,4.783)}
\gppoint{gp mark 0}{(6.920,5.303)}
\gppoint{gp mark 0}{(6.920,5.329)}
\gppoint{gp mark 0}{(6.920,5.167)}
\gppoint{gp mark 0}{(6.920,4.715)}
\gppoint{gp mark 0}{(6.920,4.925)}
\gppoint{gp mark 0}{(6.920,5.381)}
\gppoint{gp mark 0}{(6.920,5.263)}
\gppoint{gp mark 0}{(6.920,4.984)}
\gppoint{gp mark 0}{(6.920,5.075)}
\gppoint{gp mark 0}{(6.920,4.925)}
\gppoint{gp mark 0}{(6.920,4.641)}
\gppoint{gp mark 0}{(6.920,5.040)}
\gppoint{gp mark 0}{(6.920,4.984)}
\gppoint{gp mark 0}{(6.920,5.156)}
\gppoint{gp mark 0}{(6.920,5.316)}
\gppoint{gp mark 0}{(6.920,5.254)}
\gppoint{gp mark 0}{(6.920,5.167)}
\gppoint{gp mark 0}{(6.920,4.938)}
\gppoint{gp mark 0}{(6.920,4.897)}
\gppoint{gp mark 0}{(6.920,5.381)}
\gppoint{gp mark 0}{(6.920,5.499)}
\gppoint{gp mark 0}{(6.920,5.324)}
\gppoint{gp mark 0}{(6.920,5.231)}
\gppoint{gp mark 0}{(6.920,4.688)}
\gppoint{gp mark 0}{(6.920,5.535)}
\gppoint{gp mark 0}{(6.920,5.167)}
\gppoint{gp mark 0}{(6.920,5.792)}
\gppoint{gp mark 0}{(6.920,5.412)}
\gppoint{gp mark 0}{(6.920,5.290)}
\gppoint{gp mark 0}{(6.920,5.167)}
\gppoint{gp mark 0}{(6.920,5.303)}
\gppoint{gp mark 0}{(6.920,5.303)}
\gppoint{gp mark 0}{(6.920,4.706)}
\gppoint{gp mark 0}{(6.920,4.932)}
\gppoint{gp mark 0}{(6.931,5.552)}
\gppoint{gp mark 0}{(6.931,4.911)}
\gppoint{gp mark 0}{(6.931,5.622)}
\gppoint{gp mark 0}{(6.931,5.416)}
\gppoint{gp mark 0}{(6.931,5.357)}
\gppoint{gp mark 0}{(6.931,5.040)}
\gppoint{gp mark 0}{(6.931,5.263)}
\gppoint{gp mark 0}{(6.931,4.938)}
\gppoint{gp mark 0}{(6.931,4.938)}
\gppoint{gp mark 0}{(6.931,5.333)}
\gppoint{gp mark 0}{(6.931,5.312)}
\gppoint{gp mark 0}{(6.931,4.958)}
\gppoint{gp mark 0}{(6.931,4.838)}
\gppoint{gp mark 0}{(6.931,5.303)}
\gppoint{gp mark 0}{(6.931,5.177)}
\gppoint{gp mark 0}{(6.931,5.312)}
\gppoint{gp mark 0}{(6.931,5.103)}
\gppoint{gp mark 0}{(6.931,4.925)}
\gppoint{gp mark 0}{(6.931,4.952)}
\gppoint{gp mark 0}{(6.931,5.114)}
\gppoint{gp mark 0}{(6.931,4.660)}
\gppoint{gp mark 0}{(6.931,5.303)}
\gppoint{gp mark 0}{(6.931,5.412)}
\gppoint{gp mark 0}{(6.931,5.442)}
\gppoint{gp mark 0}{(6.931,5.535)}
\gppoint{gp mark 0}{(6.931,5.207)}
\gppoint{gp mark 0}{(6.931,4.918)}
\gppoint{gp mark 0}{(6.931,5.086)}
\gppoint{gp mark 0}{(6.931,5.081)}
\gppoint{gp mark 0}{(6.931,5.192)}
\gppoint{gp mark 0}{(6.931,4.997)}
\gppoint{gp mark 0}{(6.941,4.925)}
\gppoint{gp mark 0}{(6.941,4.354)}
\gppoint{gp mark 0}{(6.941,5.263)}
\gppoint{gp mark 0}{(6.941,5.427)}
\gppoint{gp mark 0}{(6.941,4.991)}
\gppoint{gp mark 0}{(6.941,5.481)}
\gppoint{gp mark 0}{(6.941,4.978)}
\gppoint{gp mark 0}{(6.941,4.984)}
\gppoint{gp mark 0}{(6.941,4.997)}
\gppoint{gp mark 0}{(6.941,5.263)}
\gppoint{gp mark 0}{(6.941,5.034)}
\gppoint{gp mark 0}{(6.941,4.823)}
\gppoint{gp mark 0}{(6.941,5.307)}
\gppoint{gp mark 0}{(6.941,4.868)}
\gppoint{gp mark 0}{(6.941,5.146)}
\gppoint{gp mark 0}{(6.941,4.945)}
\gppoint{gp mark 0}{(6.941,4.897)}
\gppoint{gp mark 0}{(6.941,5.312)}
\gppoint{gp mark 0}{(6.941,4.972)}
\gppoint{gp mark 0}{(6.941,4.853)}
\gppoint{gp mark 0}{(6.941,5.162)}
\gppoint{gp mark 0}{(6.941,5.357)}
\gppoint{gp mark 0}{(6.941,5.416)}
\gppoint{gp mark 0}{(6.941,5.103)}
\gppoint{gp mark 0}{(6.941,5.003)}
\gppoint{gp mark 0}{(6.941,5.416)}
\gppoint{gp mark 0}{(6.941,4.823)}
\gppoint{gp mark 0}{(6.941,4.679)}
\gppoint{gp mark 0}{(6.941,5.416)}
\gppoint{gp mark 0}{(6.941,5.320)}
\gppoint{gp mark 0}{(6.941,4.815)}
\gppoint{gp mark 0}{(6.941,5.349)}
\gppoint{gp mark 0}{(6.941,4.972)}
\gppoint{gp mark 0}{(6.951,5.574)}
\gppoint{gp mark 0}{(6.951,4.660)}
\gppoint{gp mark 0}{(6.951,5.378)}
\gppoint{gp mark 0}{(6.951,5.162)}
\gppoint{gp mark 0}{(6.951,5.172)}
\gppoint{gp mark 0}{(6.951,5.341)}
\gppoint{gp mark 0}{(6.951,5.320)}
\gppoint{gp mark 0}{(6.951,5.519)}
\gppoint{gp mark 0}{(6.951,5.446)}
\gppoint{gp mark 0}{(6.951,5.341)}
\gppoint{gp mark 0}{(6.951,5.034)}
\gppoint{gp mark 0}{(6.951,4.651)}
\gppoint{gp mark 0}{(6.951,5.202)}
\gppoint{gp mark 0}{(6.951,5.081)}
\gppoint{gp mark 0}{(6.951,4.799)}
\gppoint{gp mark 0}{(6.951,4.938)}
\gppoint{gp mark 0}{(6.951,4.750)}
\gppoint{gp mark 0}{(6.951,5.103)}
\gppoint{gp mark 0}{(6.951,5.202)}
\gppoint{gp mark 0}{(6.961,5.651)}
\gppoint{gp mark 0}{(6.961,4.875)}
\gppoint{gp mark 0}{(6.961,4.991)}
\gppoint{gp mark 0}{(6.961,5.202)}
\gppoint{gp mark 0}{(6.961,4.861)}
\gppoint{gp mark 0}{(6.961,5.366)}
\gppoint{gp mark 0}{(6.961,4.861)}
\gppoint{gp mark 0}{(6.961,5.294)}
\gppoint{gp mark 0}{(6.961,5.409)}
\gppoint{gp mark 0}{(6.961,5.162)}
\gppoint{gp mark 0}{(6.961,4.697)}
\gppoint{gp mark 0}{(6.961,5.495)}
\gppoint{gp mark 0}{(6.961,5.010)}
\gppoint{gp mark 0}{(6.961,5.119)}
\gppoint{gp mark 0}{(6.961,5.216)}
\gppoint{gp mark 0}{(6.961,3.911)}
\gppoint{gp mark 0}{(6.961,5.488)}
\gppoint{gp mark 0}{(6.961,4.868)}
\gppoint{gp mark 0}{(6.961,5.601)}
\gppoint{gp mark 0}{(6.961,5.374)}
\gppoint{gp mark 0}{(6.961,5.378)}
\gppoint{gp mark 0}{(6.961,4.469)}
\gppoint{gp mark 0}{(6.961,5.108)}
\gppoint{gp mark 0}{(6.961,4.945)}
\gppoint{gp mark 0}{(6.961,5.187)}
\gppoint{gp mark 0}{(6.961,5.427)}
\gppoint{gp mark 0}{(6.961,5.162)}
\gppoint{gp mark 0}{(6.961,5.374)}
\gppoint{gp mark 0}{(6.961,5.412)}
\gppoint{gp mark 0}{(6.961,4.733)}
\gppoint{gp mark 0}{(6.971,5.040)}
\gppoint{gp mark 0}{(6.971,5.393)}
\gppoint{gp mark 0}{(6.971,5.075)}
\gppoint{gp mark 0}{(6.971,5.069)}
\gppoint{gp mark 0}{(6.971,5.259)}
\gppoint{gp mark 0}{(6.971,4.938)}
\gppoint{gp mark 0}{(6.971,4.838)}
\gppoint{gp mark 0}{(6.971,4.868)}
\gppoint{gp mark 0}{(6.971,5.075)}
\gppoint{gp mark 0}{(6.971,5.420)}
\gppoint{gp mark 0}{(6.971,5.427)}
\gppoint{gp mark 0}{(6.971,4.868)}
\gppoint{gp mark 0}{(6.971,5.397)}
\gppoint{gp mark 0}{(6.971,5.245)}
\gppoint{gp mark 0}{(6.971,5.103)}
\gppoint{gp mark 0}{(6.971,6.182)}
\gppoint{gp mark 0}{(6.971,5.114)}
\gppoint{gp mark 0}{(6.971,4.932)}
\gppoint{gp mark 0}{(6.971,5.435)}
\gppoint{gp mark 0}{(6.971,4.932)}
\gppoint{gp mark 0}{(6.971,4.846)}
\gppoint{gp mark 0}{(6.971,4.952)}
\gppoint{gp mark 0}{(6.971,4.853)}
\gppoint{gp mark 0}{(6.971,5.162)}
\gppoint{gp mark 0}{(6.971,5.010)}
\gppoint{gp mark 0}{(6.971,4.861)}
\gppoint{gp mark 0}{(6.981,5.349)}
\gppoint{gp mark 0}{(6.981,5.431)}
\gppoint{gp mark 0}{(6.981,5.370)}
\gppoint{gp mark 0}{(6.981,4.875)}
\gppoint{gp mark 0}{(6.981,5.125)}
\gppoint{gp mark 0}{(6.981,5.221)}
\gppoint{gp mark 0}{(6.981,4.830)}
\gppoint{gp mark 0}{(6.981,4.679)}
\gppoint{gp mark 0}{(6.981,4.965)}
\gppoint{gp mark 0}{(6.981,4.679)}
\gppoint{gp mark 0}{(6.981,5.613)}
\gppoint{gp mark 0}{(6.981,5.481)}
\gppoint{gp mark 0}{(6.981,5.378)}
\gppoint{gp mark 0}{(6.981,5.357)}
\gppoint{gp mark 0}{(6.981,5.378)}
\gppoint{gp mark 0}{(6.981,5.231)}
\gppoint{gp mark 0}{(6.981,5.022)}
\gppoint{gp mark 0}{(6.981,4.938)}
\gppoint{gp mark 0}{(6.981,4.911)}
\gppoint{gp mark 0}{(6.981,5.645)}
\gppoint{gp mark 0}{(6.981,5.509)}
\gppoint{gp mark 0}{(6.981,4.997)}
\gppoint{gp mark 0}{(6.981,5.097)}
\gppoint{gp mark 0}{(6.981,4.945)}
\gppoint{gp mark 0}{(6.981,5.249)}
\gppoint{gp mark 0}{(6.981,5.177)}
\gppoint{gp mark 0}{(6.981,5.003)}
\gppoint{gp mark 0}{(6.991,4.978)}
\gppoint{gp mark 0}{(6.991,5.197)}
\gppoint{gp mark 0}{(6.991,5.182)}
\gppoint{gp mark 0}{(6.991,5.130)}
\gppoint{gp mark 0}{(6.991,5.040)}
\gppoint{gp mark 0}{(6.991,5.221)}
\gppoint{gp mark 0}{(6.991,4.932)}
\gppoint{gp mark 0}{(6.991,5.567)}
\gppoint{gp mark 0}{(6.991,5.135)}
\gppoint{gp mark 0}{(6.991,5.722)}
\gppoint{gp mark 0}{(6.991,5.312)}
\gppoint{gp mark 0}{(6.991,4.853)}
\gppoint{gp mark 0}{(6.991,5.722)}
\gppoint{gp mark 0}{(6.991,5.146)}
\gppoint{gp mark 0}{(6.991,5.722)}
\gppoint{gp mark 0}{(6.991,5.119)}
\gppoint{gp mark 0}{(6.991,4.846)}
\gppoint{gp mark 0}{(6.991,4.938)}
\gppoint{gp mark 0}{(6.991,5.162)}
\gppoint{gp mark 0}{(6.991,4.918)}
\gppoint{gp mark 0}{(6.991,5.333)}
\gppoint{gp mark 0}{(6.991,5.197)}
\gppoint{gp mark 0}{(6.991,5.687)}
\gppoint{gp mark 0}{(6.991,5.135)}
\gppoint{gp mark 0}{(6.991,5.409)}
\gppoint{gp mark 0}{(6.991,5.312)}
\gppoint{gp mark 0}{(6.991,5.558)}
\gppoint{gp mark 0}{(6.991,5.303)}
\gppoint{gp mark 0}{(7.001,5.003)}
\gppoint{gp mark 0}{(7.001,5.182)}
\gppoint{gp mark 0}{(7.001,5.631)}
\gppoint{gp mark 0}{(7.001,5.108)}
\gppoint{gp mark 0}{(7.001,5.216)}
\gppoint{gp mark 0}{(7.001,5.329)}
\gppoint{gp mark 0}{(7.001,5.272)}
\gppoint{gp mark 0}{(7.001,5.182)}
\gppoint{gp mark 0}{(7.001,4.972)}
\gppoint{gp mark 0}{(7.001,5.182)}
\gppoint{gp mark 0}{(7.001,5.272)}
\gppoint{gp mark 0}{(7.001,5.182)}
\gppoint{gp mark 0}{(7.001,4.651)}
\gppoint{gp mark 0}{(7.001,5.125)}
\gppoint{gp mark 0}{(7.001,5.182)}
\gppoint{gp mark 0}{(7.001,5.125)}
\gppoint{gp mark 0}{(7.001,5.353)}
\gppoint{gp mark 0}{(7.001,5.216)}
\gppoint{gp mark 0}{(7.001,4.741)}
\gppoint{gp mark 0}{(7.001,5.182)}
\gppoint{gp mark 0}{(7.001,4.932)}
\gppoint{gp mark 0}{(7.001,5.182)}
\gppoint{gp mark 0}{(7.001,5.631)}
\gppoint{gp mark 0}{(7.001,5.207)}
\gppoint{gp mark 0}{(7.001,5.182)}
\gppoint{gp mark 0}{(7.001,5.182)}
\gppoint{gp mark 0}{(7.001,5.058)}
\gppoint{gp mark 0}{(7.001,5.385)}
\gppoint{gp mark 0}{(7.001,5.316)}
\gppoint{gp mark 0}{(7.001,5.034)}
\gppoint{gp mark 0}{(7.001,5.290)}
\gppoint{gp mark 0}{(7.001,5.046)}
\gppoint{gp mark 0}{(7.001,5.182)}
\gppoint{gp mark 0}{(7.001,5.424)}
\gppoint{gp mark 0}{(7.001,5.086)}
\gppoint{gp mark 0}{(7.001,4.984)}
\gppoint{gp mark 0}{(7.001,5.474)}
\gppoint{gp mark 0}{(7.001,5.393)}
\gppoint{gp mark 0}{(7.001,5.182)}
\gppoint{gp mark 0}{(7.001,5.474)}
\gppoint{gp mark 0}{(7.010,5.226)}
\gppoint{gp mark 0}{(7.010,5.245)}
\gppoint{gp mark 0}{(7.010,5.221)}
\gppoint{gp mark 0}{(7.010,4.890)}
\gppoint{gp mark 0}{(7.010,5.226)}
\gppoint{gp mark 0}{(7.010,5.424)}
\gppoint{gp mark 0}{(7.010,5.182)}
\gppoint{gp mark 0}{(7.010,4.688)}
\gppoint{gp mark 0}{(7.010,5.211)}
\gppoint{gp mark 0}{(7.010,5.119)}
\gppoint{gp mark 0}{(7.010,5.114)}
\gppoint{gp mark 0}{(7.010,5.316)}
\gppoint{gp mark 0}{(7.010,4.890)}
\gppoint{gp mark 0}{(7.010,5.272)}
\gppoint{gp mark 0}{(7.010,5.312)}
\gppoint{gp mark 0}{(7.010,4.890)}
\gppoint{gp mark 0}{(7.010,5.312)}
\gppoint{gp mark 0}{(7.010,4.911)}
\gppoint{gp mark 0}{(7.010,5.016)}
\gppoint{gp mark 0}{(7.010,5.231)}
\gppoint{gp mark 0}{(7.010,5.240)}
\gppoint{gp mark 0}{(7.010,5.616)}
\gppoint{gp mark 0}{(7.010,5.580)}
\gppoint{gp mark 0}{(7.010,4.733)}
\gppoint{gp mark 0}{(7.010,5.272)}
\gppoint{gp mark 0}{(7.020,5.211)}
\gppoint{gp mark 0}{(7.020,5.016)}
\gppoint{gp mark 0}{(7.020,5.081)}
\gppoint{gp mark 0}{(7.020,5.167)}
\gppoint{gp mark 0}{(7.020,5.146)}
\gppoint{gp mark 0}{(7.020,5.192)}
\gppoint{gp mark 0}{(7.020,5.211)}
\gppoint{gp mark 0}{(7.020,5.374)}
\gppoint{gp mark 0}{(7.020,5.052)}
\gppoint{gp mark 0}{(7.020,4.118)}
\gppoint{gp mark 0}{(7.020,5.211)}
\gppoint{gp mark 0}{(7.020,5.478)}
\gppoint{gp mark 0}{(7.020,4.799)}
\gppoint{gp mark 0}{(7.020,5.052)}
\gppoint{gp mark 0}{(7.020,3.364)}
\gppoint{gp mark 0}{(7.020,5.097)}
\gppoint{gp mark 0}{(7.020,5.345)}
\gppoint{gp mark 0}{(7.020,5.374)}
\gppoint{gp mark 0}{(7.020,5.548)}
\gppoint{gp mark 0}{(7.020,5.040)}
\gppoint{gp mark 0}{(7.020,4.925)}
\gppoint{gp mark 0}{(7.020,5.290)}
\gppoint{gp mark 0}{(7.020,6.629)}
\gppoint{gp mark 0}{(7.020,5.446)}
\gppoint{gp mark 0}{(7.020,5.345)}
\gppoint{gp mark 0}{(7.020,5.401)}
\gppoint{gp mark 0}{(7.029,5.385)}
\gppoint{gp mark 0}{(7.029,5.277)}
\gppoint{gp mark 0}{(7.029,5.316)}
\gppoint{gp mark 0}{(7.029,5.028)}
\gppoint{gp mark 0}{(7.029,5.221)}
\gppoint{gp mark 0}{(7.029,4.972)}
\gppoint{gp mark 0}{(7.029,5.221)}
\gppoint{gp mark 0}{(7.029,5.010)}
\gppoint{gp mark 0}{(7.029,5.075)}
\gppoint{gp mark 0}{(7.029,5.249)}
\gppoint{gp mark 0}{(7.029,5.046)}
\gppoint{gp mark 0}{(7.029,4.136)}
\gppoint{gp mark 0}{(7.029,5.397)}
\gppoint{gp mark 0}{(7.029,5.412)}
\gppoint{gp mark 0}{(7.029,5.034)}
\gppoint{gp mark 0}{(7.029,4.783)}
\gppoint{gp mark 0}{(7.029,5.324)}
\gppoint{gp mark 0}{(7.029,5.401)}
\gppoint{gp mark 0}{(7.029,5.281)}
\gppoint{gp mark 0}{(7.039,5.141)}
\gppoint{gp mark 0}{(7.039,5.254)}
\gppoint{gp mark 0}{(7.039,5.202)}
\gppoint{gp mark 0}{(7.039,5.366)}
\gppoint{gp mark 0}{(7.039,5.366)}
\gppoint{gp mark 0}{(7.039,5.254)}
\gppoint{gp mark 0}{(7.039,5.412)}
\gppoint{gp mark 0}{(7.039,5.103)}
\gppoint{gp mark 0}{(7.039,5.397)}
\gppoint{gp mark 0}{(7.039,5.254)}
\gppoint{gp mark 0}{(7.039,5.207)}
\gppoint{gp mark 0}{(7.039,5.172)}
\gppoint{gp mark 0}{(7.039,5.135)}
\gppoint{gp mark 0}{(7.039,5.312)}
\gppoint{gp mark 0}{(7.039,5.642)}
\gppoint{gp mark 0}{(7.039,5.431)}
\gppoint{gp mark 0}{(7.039,5.714)}
\gppoint{gp mark 0}{(7.039,5.659)}
\gppoint{gp mark 0}{(7.039,5.254)}
\gppoint{gp mark 0}{(7.039,5.378)}
\gppoint{gp mark 0}{(7.039,5.254)}
\gppoint{gp mark 0}{(7.039,5.086)}
\gppoint{gp mark 0}{(7.039,4.504)}
\gppoint{gp mark 0}{(7.039,5.058)}
\gppoint{gp mark 0}{(7.039,5.303)}
\gppoint{gp mark 0}{(7.039,5.529)}
\gppoint{gp mark 0}{(7.039,5.254)}
\gppoint{gp mark 0}{(7.039,5.254)}
\gppoint{gp mark 0}{(7.039,5.488)}
\gppoint{gp mark 0}{(7.039,5.254)}
\gppoint{gp mark 0}{(7.039,4.938)}
\gppoint{gp mark 0}{(7.039,4.952)}
\gppoint{gp mark 0}{(7.039,5.254)}
\gppoint{gp mark 0}{(7.039,4.984)}
\gppoint{gp mark 0}{(7.039,5.329)}
\gppoint{gp mark 0}{(7.039,4.904)}
\gppoint{gp mark 0}{(7.039,5.397)}
\gppoint{gp mark 0}{(7.039,5.281)}
\gppoint{gp mark 0}{(7.048,5.254)}
\gppoint{gp mark 0}{(7.048,4.952)}
\gppoint{gp mark 0}{(7.048,5.182)}
\gppoint{gp mark 0}{(7.048,5.081)}
\gppoint{gp mark 0}{(7.048,5.412)}
\gppoint{gp mark 0}{(7.048,5.097)}
\gppoint{gp mark 0}{(7.048,5.634)}
\gppoint{gp mark 0}{(7.048,5.058)}
\gppoint{gp mark 0}{(7.048,5.460)}
\gppoint{gp mark 0}{(7.048,4.897)}
\gppoint{gp mark 0}{(7.048,5.254)}
\gppoint{gp mark 0}{(7.048,5.460)}
\gppoint{gp mark 0}{(7.048,4.984)}
\gppoint{gp mark 0}{(7.048,5.211)}
\gppoint{gp mark 0}{(7.048,5.657)}
\gppoint{gp mark 0}{(7.048,5.349)}
\gppoint{gp mark 0}{(7.048,5.254)}
\gppoint{gp mark 0}{(7.048,4.897)}
\gppoint{gp mark 0}{(7.048,5.405)}
\gppoint{gp mark 0}{(7.048,5.182)}
\gppoint{gp mark 0}{(7.048,5.240)}
\gppoint{gp mark 0}{(7.048,5.254)}
\gppoint{gp mark 0}{(7.048,5.307)}
\gppoint{gp mark 0}{(7.048,5.577)}
\gppoint{gp mark 0}{(7.048,5.349)}
\gppoint{gp mark 0}{(7.048,5.254)}
\gppoint{gp mark 0}{(7.048,5.333)}
\gppoint{gp mark 0}{(7.048,4.741)}
\gppoint{gp mark 0}{(7.048,4.991)}
\gppoint{gp mark 0}{(7.048,5.130)}
\gppoint{gp mark 0}{(7.048,5.374)}
\gppoint{gp mark 0}{(7.048,5.431)}
\gppoint{gp mark 0}{(7.048,5.353)}
\gppoint{gp mark 0}{(7.057,4.958)}
\gppoint{gp mark 0}{(7.057,5.254)}
\gppoint{gp mark 0}{(7.057,5.259)}
\gppoint{gp mark 0}{(7.057,4.938)}
\gppoint{gp mark 0}{(7.057,4.911)}
\gppoint{gp mark 0}{(7.057,4.978)}
\gppoint{gp mark 0}{(7.057,4.549)}
\gppoint{gp mark 0}{(7.057,6.059)}
\gppoint{gp mark 0}{(7.057,5.420)}
\gppoint{gp mark 0}{(7.057,4.868)}
\gppoint{gp mark 0}{(7.057,5.665)}
\gppoint{gp mark 0}{(7.057,5.435)}
\gppoint{gp mark 0}{(7.057,5.665)}
\gppoint{gp mark 0}{(7.057,4.904)}
\gppoint{gp mark 0}{(7.057,6.059)}
\gppoint{gp mark 0}{(7.057,4.945)}
\gppoint{gp mark 0}{(7.057,5.519)}
\gppoint{gp mark 0}{(7.057,5.307)}
\gppoint{gp mark 0}{(7.057,5.312)}
\gppoint{gp mark 0}{(7.057,5.648)}
\gppoint{gp mark 0}{(7.057,5.254)}
\gppoint{gp mark 0}{(7.057,4.911)}
\gppoint{gp mark 0}{(7.057,5.103)}
\gppoint{gp mark 0}{(7.057,5.254)}
\gppoint{gp mark 0}{(7.057,5.752)}
\gppoint{gp mark 0}{(7.057,5.086)}
\gppoint{gp mark 0}{(7.057,4.724)}
\gppoint{gp mark 0}{(7.057,4.783)}
\gppoint{gp mark 0}{(7.057,4.978)}
\gppoint{gp mark 0}{(7.057,5.345)}
\gppoint{gp mark 0}{(7.057,5.277)}
\gppoint{gp mark 0}{(7.057,5.522)}
\gppoint{gp mark 0}{(7.067,5.634)}
\gppoint{gp mark 0}{(7.067,5.235)}
\gppoint{gp mark 0}{(7.067,5.634)}
\gppoint{gp mark 0}{(7.067,4.925)}
\gppoint{gp mark 0}{(7.067,5.634)}
\gppoint{gp mark 0}{(7.067,5.385)}
\gppoint{gp mark 0}{(7.067,5.431)}
\gppoint{gp mark 0}{(7.067,5.457)}
\gppoint{gp mark 0}{(7.067,5.634)}
\gppoint{gp mark 0}{(7.067,5.316)}
\gppoint{gp mark 0}{(7.067,5.634)}
\gppoint{gp mark 0}{(7.067,5.506)}
\gppoint{gp mark 0}{(7.067,5.506)}
\gppoint{gp mark 0}{(7.067,5.502)}
\gppoint{gp mark 0}{(7.067,5.333)}
\gppoint{gp mark 0}{(7.067,5.634)}
\gppoint{gp mark 0}{(7.067,5.034)}
\gppoint{gp mark 0}{(7.067,4.997)}
\gppoint{gp mark 0}{(7.067,5.362)}
\gppoint{gp mark 0}{(7.067,4.952)}
\gppoint{gp mark 0}{(7.067,5.634)}
\gppoint{gp mark 0}{(7.067,5.634)}
\gppoint{gp mark 0}{(7.067,5.290)}
\gppoint{gp mark 0}{(7.067,4.911)}
\gppoint{gp mark 0}{(7.067,5.634)}
\gppoint{gp mark 0}{(7.067,5.337)}
\gppoint{gp mark 0}{(7.067,5.412)}
\gppoint{gp mark 0}{(7.067,5.634)}
\gppoint{gp mark 0}{(7.067,5.634)}
\gppoint{gp mark 0}{(7.067,5.634)}
\gppoint{gp mark 0}{(7.067,5.634)}
\gppoint{gp mark 0}{(7.067,5.634)}
\gppoint{gp mark 0}{(7.067,4.875)}
\gppoint{gp mark 0}{(7.067,5.634)}
\gppoint{gp mark 0}{(7.067,5.634)}
\gppoint{gp mark 0}{(7.067,4.984)}
\gppoint{gp mark 0}{(7.067,5.634)}
\gppoint{gp mark 0}{(7.067,5.634)}
\gppoint{gp mark 0}{(7.067,5.634)}
\gppoint{gp mark 0}{(7.067,5.221)}
\gppoint{gp mark 0}{(7.067,5.182)}
\gppoint{gp mark 0}{(7.067,5.634)}
\gppoint{gp mark 0}{(7.067,5.634)}
\gppoint{gp mark 0}{(7.067,5.393)}
\gppoint{gp mark 0}{(7.067,5.634)}
\gppoint{gp mark 0}{(7.067,5.162)}
\gppoint{gp mark 0}{(7.067,5.634)}
\gppoint{gp mark 0}{(7.067,5.119)}
\gppoint{gp mark 0}{(7.067,5.075)}
\gppoint{gp mark 0}{(7.067,5.040)}
\gppoint{gp mark 0}{(7.076,5.187)}
\gppoint{gp mark 0}{(7.076,5.320)}
\gppoint{gp mark 0}{(7.076,5.719)}
\gppoint{gp mark 0}{(7.076,5.460)}
\gppoint{gp mark 0}{(7.076,5.197)}
\gppoint{gp mark 0}{(7.076,4.612)}
\gppoint{gp mark 0}{(7.076,4.612)}
\gppoint{gp mark 0}{(7.076,5.698)}
\gppoint{gp mark 0}{(7.076,5.135)}
\gppoint{gp mark 0}{(7.076,5.464)}
\gppoint{gp mark 0}{(7.076,5.125)}
\gppoint{gp mark 0}{(7.076,4.972)}
\gppoint{gp mark 0}{(7.076,5.081)}
\gppoint{gp mark 0}{(7.076,5.424)}
\gppoint{gp mark 0}{(7.076,5.719)}
\gppoint{gp mark 0}{(7.076,5.187)}
\gppoint{gp mark 0}{(7.076,5.245)}
\gppoint{gp mark 0}{(7.076,5.393)}
\gppoint{gp mark 0}{(7.076,5.226)}
\gppoint{gp mark 0}{(7.076,5.397)}
\gppoint{gp mark 0}{(7.076,5.182)}
\gppoint{gp mark 0}{(7.085,5.595)}
\gppoint{gp mark 0}{(7.085,5.601)}
\gppoint{gp mark 0}{(7.085,5.601)}
\gppoint{gp mark 0}{(7.085,5.601)}
\gppoint{gp mark 0}{(7.085,5.601)}
\gppoint{gp mark 0}{(7.085,5.601)}
\gppoint{gp mark 0}{(7.085,5.601)}
\gppoint{gp mark 0}{(7.085,5.601)}
\gppoint{gp mark 0}{(7.085,5.601)}
\gppoint{gp mark 0}{(7.085,5.601)}
\gppoint{gp mark 0}{(7.085,5.601)}
\gppoint{gp mark 0}{(7.085,4.997)}
\gppoint{gp mark 0}{(7.085,5.601)}
\gppoint{gp mark 0}{(7.085,5.601)}
\gppoint{gp mark 0}{(7.085,5.290)}
\gppoint{gp mark 0}{(7.085,5.601)}
\gppoint{gp mark 0}{(7.085,5.601)}
\gppoint{gp mark 0}{(7.085,4.997)}
\gppoint{gp mark 0}{(7.085,5.081)}
\gppoint{gp mark 0}{(7.085,5.290)}
\gppoint{gp mark 0}{(7.085,4.991)}
\gppoint{gp mark 0}{(7.085,4.932)}
\gppoint{gp mark 0}{(7.085,5.485)}
\gppoint{gp mark 0}{(7.085,5.207)}
\gppoint{gp mark 0}{(7.085,5.192)}
\gppoint{gp mark 0}{(7.085,5.464)}
\gppoint{gp mark 0}{(7.085,5.312)}
\gppoint{gp mark 0}{(7.085,5.405)}
\gppoint{gp mark 0}{(7.085,5.187)}
\gppoint{gp mark 0}{(7.085,4.741)}
\gppoint{gp mark 0}{(7.085,5.211)}
\gppoint{gp mark 0}{(7.085,5.316)}
\gppoint{gp mark 0}{(7.085,5.695)}
\gppoint{gp mark 0}{(7.085,5.254)}
\gppoint{gp mark 0}{(7.085,5.281)}
\gppoint{gp mark 0}{(7.085,4.791)}
\gppoint{gp mark 0}{(7.085,5.290)}
\gppoint{gp mark 0}{(7.085,5.453)}
\gppoint{gp mark 0}{(7.085,5.245)}
\gppoint{gp mark 0}{(7.085,5.601)}
\gppoint{gp mark 0}{(7.085,5.312)}
\gppoint{gp mark 0}{(7.094,5.722)}
\gppoint{gp mark 0}{(7.094,5.349)}
\gppoint{gp mark 0}{(7.094,5.595)}
\gppoint{gp mark 0}{(7.094,5.716)}
\gppoint{gp mark 0}{(7.094,5.362)}
\gppoint{gp mark 0}{(7.094,5.187)}
\gppoint{gp mark 0}{(7.094,5.216)}
\gppoint{gp mark 0}{(7.094,5.245)}
\gppoint{gp mark 0}{(7.094,5.125)}
\gppoint{gp mark 0}{(7.094,5.787)}
\gppoint{gp mark 0}{(7.094,5.329)}
\gppoint{gp mark 0}{(7.094,5.435)}
\gppoint{gp mark 0}{(7.094,5.558)}
\gppoint{gp mark 0}{(7.094,5.240)}
\gppoint{gp mark 0}{(7.094,5.416)}
\gppoint{gp mark 0}{(7.094,5.052)}
\gppoint{gp mark 0}{(7.094,5.254)}
\gppoint{gp mark 0}{(7.094,5.385)}
\gppoint{gp mark 0}{(7.094,4.830)}
\gppoint{gp mark 0}{(7.094,5.378)}
\gppoint{gp mark 0}{(7.094,5.679)}
\gppoint{gp mark 0}{(7.094,5.378)}
\gppoint{gp mark 0}{(7.094,5.366)}
\gppoint{gp mark 0}{(7.094,5.529)}
\gppoint{gp mark 0}{(7.094,5.735)}
\gppoint{gp mark 0}{(7.094,5.634)}
\gppoint{gp mark 0}{(7.094,5.172)}
\gppoint{gp mark 0}{(7.094,5.249)}
\gppoint{gp mark 0}{(7.094,4.815)}
\gppoint{gp mark 0}{(7.094,5.341)}
\gppoint{gp mark 0}{(7.094,4.883)}
\gppoint{gp mark 0}{(7.094,5.286)}
\gppoint{gp mark 0}{(7.094,5.460)}
\gppoint{gp mark 0}{(7.094,5.522)}
\gppoint{gp mark 0}{(7.103,4.958)}
\gppoint{gp mark 0}{(7.103,5.592)}
\gppoint{gp mark 0}{(7.103,5.645)}
\gppoint{gp mark 0}{(7.103,5.294)}
\gppoint{gp mark 0}{(7.103,5.349)}
\gppoint{gp mark 0}{(7.103,5.092)}
\gppoint{gp mark 0}{(7.103,5.119)}
\gppoint{gp mark 0}{(7.103,5.245)}
\gppoint{gp mark 0}{(7.103,5.532)}
\gppoint{gp mark 0}{(7.103,5.412)}
\gppoint{gp mark 0}{(7.103,5.294)}
\gppoint{gp mark 0}{(7.103,4.952)}
\gppoint{gp mark 0}{(7.103,4.965)}
\gppoint{gp mark 0}{(7.103,5.822)}
\gppoint{gp mark 0}{(7.103,4.952)}
\gppoint{gp mark 0}{(7.103,5.481)}
\gppoint{gp mark 0}{(7.103,5.324)}
\gppoint{gp mark 0}{(7.103,5.221)}
\gppoint{gp mark 0}{(7.103,5.197)}
\gppoint{gp mark 0}{(7.103,5.516)}
\gppoint{gp mark 0}{(7.103,5.385)}
\gppoint{gp mark 0}{(7.103,5.156)}
\gppoint{gp mark 0}{(7.103,5.156)}
\gppoint{gp mark 0}{(7.111,5.940)}
\gppoint{gp mark 0}{(7.111,5.665)}
\gppoint{gp mark 0}{(7.111,5.397)}
\gppoint{gp mark 0}{(7.111,5.177)}
\gppoint{gp mark 0}{(7.111,4.972)}
\gppoint{gp mark 0}{(7.111,5.449)}
\gppoint{gp mark 0}{(7.111,4.972)}
\gppoint{gp mark 0}{(7.111,5.249)}
\gppoint{gp mark 0}{(7.111,4.997)}
\gppoint{gp mark 0}{(7.111,5.221)}
\gppoint{gp mark 0}{(7.111,5.097)}
\gppoint{gp mark 0}{(7.111,5.069)}
\gppoint{gp mark 0}{(7.111,5.016)}
\gppoint{gp mark 0}{(7.111,4.978)}
\gppoint{gp mark 0}{(7.111,5.938)}
\gppoint{gp mark 0}{(7.111,5.682)}
\gppoint{gp mark 0}{(7.111,5.108)}
\gppoint{gp mark 0}{(7.120,5.167)}
\gppoint{gp mark 0}{(7.120,5.401)}
\gppoint{gp mark 0}{(7.120,5.401)}
\gppoint{gp mark 0}{(7.120,5.235)}
\gppoint{gp mark 0}{(7.120,4.897)}
\gppoint{gp mark 0}{(7.120,5.381)}
\gppoint{gp mark 0}{(7.120,5.416)}
\gppoint{gp mark 0}{(7.120,4.965)}
\gppoint{gp mark 0}{(7.120,5.401)}
\gppoint{gp mark 0}{(7.120,5.235)}
\gppoint{gp mark 0}{(7.120,5.642)}
\gppoint{gp mark 0}{(7.120,4.997)}
\gppoint{gp mark 0}{(7.120,5.329)}
\gppoint{gp mark 0}{(7.120,5.401)}
\gppoint{gp mark 0}{(7.120,5.555)}
\gppoint{gp mark 0}{(7.120,5.058)}
\gppoint{gp mark 0}{(7.120,5.211)}
\gppoint{gp mark 0}{(7.120,5.028)}
\gppoint{gp mark 0}{(7.120,5.211)}
\gppoint{gp mark 0}{(7.120,5.299)}
\gppoint{gp mark 0}{(7.120,5.312)}
\gppoint{gp mark 0}{(7.120,5.449)}
\gppoint{gp mark 0}{(7.120,5.286)}
\gppoint{gp mark 0}{(7.120,4.791)}
\gppoint{gp mark 0}{(7.120,5.103)}
\gppoint{gp mark 0}{(7.129,5.381)}
\gppoint{gp mark 0}{(7.129,5.075)}
\gppoint{gp mark 0}{(7.129,5.259)}
\gppoint{gp mark 0}{(7.129,5.103)}
\gppoint{gp mark 0}{(7.129,5.075)}
\gppoint{gp mark 0}{(7.129,5.449)}
\gppoint{gp mark 0}{(7.129,5.474)}
\gppoint{gp mark 0}{(7.129,5.442)}
\gppoint{gp mark 0}{(7.129,5.506)}
\gppoint{gp mark 0}{(7.129,4.706)}
\gppoint{gp mark 0}{(7.129,5.512)}
\gppoint{gp mark 0}{(7.129,5.162)}
\gppoint{gp mark 0}{(7.129,5.337)}
\gppoint{gp mark 0}{(7.129,5.294)}
\gppoint{gp mark 0}{(7.129,5.081)}
\gppoint{gp mark 0}{(7.129,4.783)}
\gppoint{gp mark 0}{(7.138,5.378)}
\gppoint{gp mark 0}{(7.138,5.564)}
\gppoint{gp mark 0}{(7.138,5.353)}
\gppoint{gp mark 0}{(7.138,5.552)}
\gppoint{gp mark 0}{(7.138,5.097)}
\gppoint{gp mark 0}{(7.138,5.564)}
\gppoint{gp mark 0}{(7.138,5.583)}
\gppoint{gp mark 0}{(7.138,5.307)}
\gppoint{gp mark 0}{(7.138,5.905)}
\gppoint{gp mark 0}{(7.138,5.574)}
\gppoint{gp mark 0}{(7.138,5.564)}
\gppoint{gp mark 0}{(7.138,5.081)}
\gppoint{gp mark 0}{(7.138,4.897)}
\gppoint{gp mark 0}{(7.138,5.081)}
\gppoint{gp mark 0}{(7.138,5.409)}
\gppoint{gp mark 0}{(7.138,5.815)}
\gppoint{gp mark 0}{(7.138,5.156)}
\gppoint{gp mark 0}{(7.138,5.081)}
\gppoint{gp mark 0}{(7.138,5.268)}
\gppoint{gp mark 0}{(7.138,5.187)}
\gppoint{gp mark 0}{(7.138,5.439)}
\gppoint{gp mark 0}{(7.138,5.119)}
\gppoint{gp mark 0}{(7.138,5.542)}
\gppoint{gp mark 0}{(7.138,5.478)}
\gppoint{gp mark 0}{(7.138,5.202)}
\gppoint{gp mark 0}{(7.138,5.092)}
\gppoint{gp mark 0}{(7.138,5.439)}
\gppoint{gp mark 0}{(7.138,4.660)}
\gppoint{gp mark 0}{(7.138,5.622)}
\gppoint{gp mark 0}{(7.138,5.240)}
\gppoint{gp mark 0}{(7.138,5.564)}
\gppoint{gp mark 0}{(7.138,5.564)}
\gppoint{gp mark 0}{(7.138,4.312)}
\gppoint{gp mark 0}{(7.146,5.286)}
\gppoint{gp mark 0}{(7.146,5.488)}
\gppoint{gp mark 0}{(7.146,5.567)}
\gppoint{gp mark 0}{(7.146,5.040)}
\gppoint{gp mark 0}{(7.146,5.052)}
\gppoint{gp mark 0}{(7.146,5.529)}
\gppoint{gp mark 0}{(7.146,5.235)}
\gppoint{gp mark 0}{(7.146,5.040)}
\gppoint{gp mark 0}{(7.146,4.984)}
\gppoint{gp mark 0}{(7.146,5.221)}
\gppoint{gp mark 0}{(7.146,5.409)}
\gppoint{gp mark 0}{(7.146,5.485)}
\gppoint{gp mark 0}{(7.146,5.187)}
\gppoint{gp mark 0}{(7.146,5.676)}
\gppoint{gp mark 0}{(7.146,5.075)}
\gppoint{gp mark 0}{(7.146,5.052)}
\gppoint{gp mark 0}{(7.146,5.676)}
\gppoint{gp mark 0}{(7.146,4.932)}
\gppoint{gp mark 0}{(7.146,5.063)}
\gppoint{gp mark 0}{(7.146,5.424)}
\gppoint{gp mark 0}{(7.146,4.612)}
\gppoint{gp mark 0}{(7.155,5.341)}
\gppoint{gp mark 0}{(7.155,5.197)}
\gppoint{gp mark 0}{(7.155,5.290)}
\gppoint{gp mark 0}{(7.155,5.341)}
\gppoint{gp mark 0}{(7.155,5.467)}
\gppoint{gp mark 0}{(7.155,5.187)}
\gppoint{gp mark 0}{(7.155,5.259)}
\gppoint{gp mark 0}{(7.155,5.022)}
\gppoint{gp mark 0}{(7.155,5.341)}
\gppoint{gp mark 0}{(7.155,5.312)}
\gppoint{gp mark 0}{(7.155,5.492)}
\gppoint{gp mark 0}{(7.155,5.567)}
\gppoint{gp mark 0}{(7.155,4.783)}
\gppoint{gp mark 0}{(7.155,5.114)}
\gppoint{gp mark 0}{(7.155,4.997)}
\gppoint{gp mark 0}{(7.155,4.997)}
\gppoint{gp mark 0}{(7.155,4.853)}
\gppoint{gp mark 0}{(7.155,5.299)}
\gppoint{gp mark 0}{(7.155,5.485)}
\gppoint{gp mark 0}{(7.155,5.125)}
\gppoint{gp mark 0}{(7.163,5.353)}
\gppoint{gp mark 0}{(7.163,5.393)}
\gppoint{gp mark 0}{(7.163,5.595)}
\gppoint{gp mark 0}{(7.163,5.216)}
\gppoint{gp mark 0}{(7.163,4.783)}
\gppoint{gp mark 0}{(7.163,5.820)}
\gppoint{gp mark 0}{(7.163,5.519)}
\gppoint{gp mark 0}{(7.163,5.651)}
\gppoint{gp mark 0}{(7.163,5.345)}
\gppoint{gp mark 0}{(7.163,4.823)}
\gppoint{gp mark 0}{(7.163,5.604)}
\gppoint{gp mark 0}{(7.163,5.010)}
\gppoint{gp mark 0}{(7.163,5.378)}
\gppoint{gp mark 0}{(7.163,5.286)}
\gppoint{gp mark 0}{(7.163,4.984)}
\gppoint{gp mark 0}{(7.163,5.362)}
\gppoint{gp mark 0}{(7.163,5.684)}
\gppoint{gp mark 0}{(7.163,5.197)}
\gppoint{gp mark 0}{(7.163,5.324)}
\gppoint{gp mark 0}{(7.163,5.385)}
\gppoint{gp mark 0}{(7.163,5.177)}
\gppoint{gp mark 0}{(7.163,5.345)}
\gppoint{gp mark 0}{(7.163,6.119)}
\gppoint{gp mark 0}{(7.172,5.409)}
\gppoint{gp mark 0}{(7.172,5.502)}
\gppoint{gp mark 0}{(7.172,5.539)}
\gppoint{gp mark 0}{(7.172,5.069)}
\gppoint{gp mark 0}{(7.172,5.684)}
\gppoint{gp mark 0}{(7.172,4.651)}
\gppoint{gp mark 0}{(7.172,5.303)}
\gppoint{gp mark 0}{(7.172,4.991)}
\gppoint{gp mark 0}{(7.172,5.092)}
\gppoint{gp mark 0}{(7.172,5.668)}
\gppoint{gp mark 0}{(7.172,5.349)}
\gppoint{gp mark 0}{(7.172,5.349)}
\gppoint{gp mark 0}{(7.172,5.294)}
\gppoint{gp mark 0}{(7.172,5.294)}
\gppoint{gp mark 0}{(7.180,5.187)}
\gppoint{gp mark 0}{(7.180,5.767)}
\gppoint{gp mark 0}{(7.180,4.775)}
\gppoint{gp mark 0}{(7.180,5.431)}
\gppoint{gp mark 0}{(7.180,5.034)}
\gppoint{gp mark 0}{(7.180,5.453)}
\gppoint{gp mark 0}{(7.180,5.767)}
\gppoint{gp mark 0}{(7.180,5.097)}
\gppoint{gp mark 0}{(7.180,4.830)}
\gppoint{gp mark 0}{(7.180,5.478)}
\gppoint{gp mark 0}{(7.180,5.374)}
\gppoint{gp mark 0}{(7.180,5.519)}
\gppoint{gp mark 0}{(7.180,5.668)}
\gppoint{gp mark 0}{(7.180,5.637)}
\gppoint{gp mark 0}{(7.180,4.830)}
\gppoint{gp mark 0}{(7.180,5.555)}
\gppoint{gp mark 0}{(7.180,4.846)}
\gppoint{gp mark 0}{(7.180,5.221)}
\gppoint{gp mark 0}{(7.180,5.474)}
\gppoint{gp mark 0}{(7.180,5.216)}
\gppoint{gp mark 0}{(7.180,5.114)}
\gppoint{gp mark 0}{(7.180,5.519)}
\gppoint{gp mark 0}{(7.180,5.162)}
\gppoint{gp mark 0}{(7.180,5.424)}
\gppoint{gp mark 0}{(7.180,5.427)}
\gppoint{gp mark 0}{(7.180,5.028)}
\gppoint{gp mark 0}{(7.180,5.272)}
\gppoint{gp mark 0}{(7.180,5.162)}
\gppoint{gp mark 0}{(7.188,5.329)}
\gppoint{gp mark 0}{(7.188,5.307)}
\gppoint{gp mark 0}{(7.188,5.277)}
\gppoint{gp mark 0}{(7.188,5.716)}
\gppoint{gp mark 0}{(7.188,5.471)}
\gppoint{gp mark 0}{(7.188,4.171)}
\gppoint{gp mark 0}{(7.188,5.046)}
\gppoint{gp mark 0}{(7.188,5.362)}
\gppoint{gp mark 0}{(7.188,5.349)}
\gppoint{gp mark 0}{(7.188,4.945)}
\gppoint{gp mark 0}{(7.188,5.069)}
\gppoint{gp mark 0}{(7.188,5.197)}
\gppoint{gp mark 0}{(7.188,5.548)}
\gppoint{gp mark 0}{(7.188,5.442)}
\gppoint{gp mark 0}{(7.188,5.063)}
\gppoint{gp mark 0}{(7.188,5.192)}
\gppoint{gp mark 0}{(7.188,5.616)}
\gppoint{gp mark 0}{(7.188,5.086)}
\gppoint{gp mark 0}{(7.197,5.642)}
\gppoint{gp mark 0}{(7.197,5.135)}
\gppoint{gp mark 0}{(7.197,5.151)}
\gppoint{gp mark 0}{(7.197,5.427)}
\gppoint{gp mark 0}{(7.197,5.760)}
\gppoint{gp mark 0}{(7.197,5.114)}
\gppoint{gp mark 0}{(7.197,5.580)}
\gppoint{gp mark 0}{(7.197,5.097)}
\gppoint{gp mark 0}{(7.197,5.412)}
\gppoint{gp mark 0}{(7.197,5.162)}
\gppoint{gp mark 0}{(7.197,5.333)}
\gppoint{gp mark 0}{(7.197,5.272)}
\gppoint{gp mark 0}{(7.197,4.791)}
\gppoint{gp mark 0}{(7.197,5.254)}
\gppoint{gp mark 0}{(7.197,5.366)}
\gppoint{gp mark 0}{(7.197,4.679)}
\gppoint{gp mark 0}{(7.197,4.830)}
\gppoint{gp mark 0}{(7.197,5.431)}
\gppoint{gp mark 0}{(7.197,5.312)}
\gppoint{gp mark 0}{(7.197,5.046)}
\gppoint{gp mark 0}{(7.205,5.464)}
\gppoint{gp mark 0}{(7.205,5.442)}
\gppoint{gp mark 0}{(7.205,5.439)}
\gppoint{gp mark 0}{(7.205,5.063)}
\gppoint{gp mark 0}{(7.205,5.427)}
\gppoint{gp mark 0}{(7.205,5.130)}
\gppoint{gp mark 0}{(7.205,5.539)}
\gppoint{gp mark 0}{(7.205,5.114)}
\gppoint{gp mark 0}{(7.205,5.040)}
\gppoint{gp mark 0}{(7.205,5.010)}
\gppoint{gp mark 0}{(7.205,5.307)}
\gppoint{gp mark 0}{(7.205,5.010)}
\gppoint{gp mark 0}{(7.205,5.457)}
\gppoint{gp mark 0}{(7.213,5.662)}
\gppoint{gp mark 0}{(7.213,5.687)}
\gppoint{gp mark 0}{(7.213,6.013)}
\gppoint{gp mark 0}{(7.213,5.245)}
\gppoint{gp mark 0}{(7.213,5.333)}
\gppoint{gp mark 0}{(7.213,5.378)}
\gppoint{gp mark 0}{(7.213,5.668)}
\gppoint{gp mark 0}{(7.213,5.574)}
\gppoint{gp mark 0}{(7.213,5.254)}
\gppoint{gp mark 0}{(7.213,5.141)}
\gppoint{gp mark 0}{(7.213,5.509)}
\gppoint{gp mark 0}{(7.213,5.598)}
\gppoint{gp mark 0}{(7.213,5.103)}
\gppoint{gp mark 0}{(7.213,5.401)}
\gppoint{gp mark 0}{(7.213,4.783)}
\gppoint{gp mark 0}{(7.213,5.192)}
\gppoint{gp mark 0}{(7.213,5.370)}
\gppoint{gp mark 0}{(7.213,5.130)}
\gppoint{gp mark 0}{(7.213,5.141)}
\gppoint{gp mark 0}{(7.213,5.182)}
\gppoint{gp mark 0}{(7.213,5.156)}
\gppoint{gp mark 0}{(7.213,5.333)}
\gppoint{gp mark 0}{(7.213,5.424)}
\gppoint{gp mark 0}{(7.213,5.294)}
\gppoint{gp mark 0}{(7.213,5.141)}
\gppoint{gp mark 0}{(7.213,5.843)}
\gppoint{gp mark 0}{(7.221,5.548)}
\gppoint{gp mark 0}{(7.221,5.548)}
\gppoint{gp mark 0}{(7.221,5.277)}
\gppoint{gp mark 0}{(7.221,5.495)}
\gppoint{gp mark 0}{(7.221,5.197)}
\gppoint{gp mark 0}{(7.221,5.226)}
\gppoint{gp mark 0}{(7.221,4.799)}
\gppoint{gp mark 0}{(7.221,5.058)}
\gppoint{gp mark 0}{(7.221,5.464)}
\gppoint{gp mark 0}{(7.221,5.601)}
\gppoint{gp mark 0}{(7.221,4.861)}
\gppoint{gp mark 0}{(7.221,5.207)}
\gppoint{gp mark 0}{(7.221,5.197)}
\gppoint{gp mark 0}{(7.221,4.660)}
\gppoint{gp mark 0}{(7.221,5.385)}
\gppoint{gp mark 0}{(7.221,5.595)}
\gppoint{gp mark 0}{(7.221,5.010)}
\gppoint{gp mark 0}{(7.221,5.381)}
\gppoint{gp mark 0}{(7.221,4.991)}
\gppoint{gp mark 0}{(7.221,5.353)}
\gppoint{gp mark 0}{(7.229,5.389)}
\gppoint{gp mark 0}{(7.229,6.974)}
\gppoint{gp mark 0}{(7.229,5.409)}
\gppoint{gp mark 0}{(7.229,5.097)}
\gppoint{gp mark 0}{(7.229,5.231)}
\gppoint{gp mark 0}{(7.229,5.075)}
\gppoint{gp mark 0}{(7.229,4.991)}
\gppoint{gp mark 0}{(7.229,5.125)}
\gppoint{gp mark 0}{(7.229,5.063)}
\gppoint{gp mark 0}{(7.229,5.495)}
\gppoint{gp mark 0}{(7.229,5.254)}
\gppoint{gp mark 0}{(7.229,5.307)}
\gppoint{gp mark 0}{(7.229,5.580)}
\gppoint{gp mark 0}{(7.237,5.016)}
\gppoint{gp mark 0}{(7.237,5.357)}
\gppoint{gp mark 0}{(7.237,5.345)}
\gppoint{gp mark 0}{(7.237,5.401)}
\gppoint{gp mark 0}{(7.237,5.478)}
\gppoint{gp mark 0}{(7.237,5.345)}
\gppoint{gp mark 0}{(7.237,5.719)}
\gppoint{gp mark 0}{(7.237,5.522)}
\gppoint{gp mark 0}{(7.237,5.182)}
\gppoint{gp mark 0}{(7.237,5.567)}
\gppoint{gp mark 0}{(7.237,5.345)}
\gppoint{gp mark 0}{(7.237,5.571)}
\gppoint{gp mark 0}{(7.237,5.353)}
\gppoint{gp mark 0}{(7.237,5.187)}
\gppoint{gp mark 0}{(7.237,5.416)}
\gppoint{gp mark 0}{(7.237,4.932)}
\gppoint{gp mark 0}{(7.237,5.522)}
\gppoint{gp mark 0}{(7.237,5.216)}
\gppoint{gp mark 0}{(7.237,5.069)}
\gppoint{gp mark 0}{(7.237,5.502)}
\gppoint{gp mark 0}{(7.237,4.997)}
\gppoint{gp mark 0}{(7.237,5.499)}
\gppoint{gp mark 0}{(7.237,5.509)}
\gppoint{gp mark 0}{(7.237,5.542)}
\gppoint{gp mark 0}{(7.245,5.268)}
\gppoint{gp mark 0}{(7.245,5.119)}
\gppoint{gp mark 0}{(7.245,5.092)}
\gppoint{gp mark 0}{(7.245,4.758)}
\gppoint{gp mark 0}{(7.245,5.389)}
\gppoint{gp mark 0}{(7.245,6.025)}
\gppoint{gp mark 0}{(7.245,4.767)}
\gppoint{gp mark 0}{(7.245,5.329)}
\gppoint{gp mark 0}{(7.245,4.775)}
\gppoint{gp mark 0}{(7.245,5.613)}
\gppoint{gp mark 0}{(7.245,4.741)}
\gppoint{gp mark 0}{(7.245,4.767)}
\gppoint{gp mark 0}{(7.245,5.254)}
\gppoint{gp mark 0}{(7.245,4.991)}
\gppoint{gp mark 0}{(7.245,5.146)}
\gppoint{gp mark 0}{(7.245,4.767)}
\gppoint{gp mark 0}{(7.245,4.945)}
\gppoint{gp mark 0}{(7.245,5.966)}
\gppoint{gp mark 0}{(7.245,4.945)}
\gppoint{gp mark 0}{(7.245,5.924)}
\gppoint{gp mark 0}{(7.245,5.294)}
\gppoint{gp mark 0}{(7.245,5.485)}
\gppoint{gp mark 0}{(7.245,5.532)}
\gppoint{gp mark 0}{(7.245,5.532)}
\gppoint{gp mark 0}{(7.245,5.532)}
\gppoint{gp mark 0}{(7.245,5.716)}
\gppoint{gp mark 0}{(7.245,5.187)}
\gppoint{gp mark 0}{(7.245,5.086)}
\gppoint{gp mark 0}{(7.245,4.978)}
\gppoint{gp mark 0}{(7.253,5.668)}
\gppoint{gp mark 0}{(7.253,5.558)}
\gppoint{gp mark 0}{(7.253,5.240)}
\gppoint{gp mark 0}{(7.253,4.911)}
\gppoint{gp mark 0}{(7.253,5.312)}
\gppoint{gp mark 0}{(7.253,5.069)}
\gppoint{gp mark 0}{(7.253,5.460)}
\gppoint{gp mark 0}{(7.253,5.130)}
\gppoint{gp mark 0}{(7.253,5.676)}
\gppoint{gp mark 0}{(7.253,5.401)}
\gppoint{gp mark 0}{(7.253,4.767)}
\gppoint{gp mark 0}{(7.253,5.374)}
\gppoint{gp mark 0}{(7.253,5.075)}
\gppoint{gp mark 0}{(7.253,5.529)}
\gppoint{gp mark 0}{(7.253,5.075)}
\gppoint{gp mark 0}{(7.253,5.119)}
\gppoint{gp mark 0}{(7.261,5.016)}
\gppoint{gp mark 0}{(7.261,5.216)}
\gppoint{gp mark 0}{(7.261,5.125)}
\gppoint{gp mark 0}{(7.261,4.767)}
\gppoint{gp mark 0}{(7.261,5.843)}
\gppoint{gp mark 0}{(7.261,5.424)}
\gppoint{gp mark 0}{(7.261,5.914)}
\gppoint{gp mark 0}{(7.261,5.341)}
\gppoint{gp mark 0}{(7.261,5.453)}
\gppoint{gp mark 0}{(7.261,5.495)}
\gppoint{gp mark 0}{(7.261,5.075)}
\gppoint{gp mark 0}{(7.261,5.086)}
\gppoint{gp mark 0}{(7.261,5.509)}
\gppoint{gp mark 0}{(7.261,5.539)}
\gppoint{gp mark 0}{(7.261,5.760)}
\gppoint{gp mark 0}{(7.268,5.125)}
\gppoint{gp mark 0}{(7.268,5.349)}
\gppoint{gp mark 0}{(7.268,5.595)}
\gppoint{gp mark 0}{(7.268,5.162)}
\gppoint{gp mark 0}{(7.268,5.177)}
\gppoint{gp mark 0}{(7.268,5.125)}
\gppoint{gp mark 0}{(7.268,5.135)}
\gppoint{gp mark 0}{(7.268,5.040)}
\gppoint{gp mark 0}{(7.268,5.125)}
\gppoint{gp mark 0}{(7.268,5.333)}
\gppoint{gp mark 0}{(7.268,5.320)}
\gppoint{gp mark 0}{(7.268,5.125)}
\gppoint{gp mark 0}{(7.268,6.412)}
\gppoint{gp mark 0}{(7.268,5.125)}
\gppoint{gp mark 0}{(7.268,5.827)}
\gppoint{gp mark 0}{(7.268,5.571)}
\gppoint{gp mark 0}{(7.268,5.125)}
\gppoint{gp mark 0}{(7.268,5.125)}
\gppoint{gp mark 0}{(7.268,5.888)}
\gppoint{gp mark 0}{(7.268,5.259)}
\gppoint{gp mark 0}{(7.268,5.249)}
\gppoint{gp mark 0}{(7.268,5.125)}
\gppoint{gp mark 0}{(7.276,5.235)}
\gppoint{gp mark 0}{(7.276,5.151)}
\gppoint{gp mark 0}{(7.276,5.467)}
\gppoint{gp mark 0}{(7.276,6.067)}
\gppoint{gp mark 0}{(7.276,5.156)}
\gppoint{gp mark 0}{(7.276,5.526)}
\gppoint{gp mark 0}{(7.276,5.103)}
\gppoint{gp mark 0}{(7.276,5.370)}
\gppoint{gp mark 0}{(7.276,5.231)}
\gppoint{gp mark 0}{(7.276,5.320)}
\gppoint{gp mark 0}{(7.276,5.464)}
\gppoint{gp mark 0}{(7.276,5.637)}
\gppoint{gp mark 0}{(7.276,4.861)}
\gppoint{gp mark 0}{(7.276,5.197)}
\gppoint{gp mark 0}{(7.276,5.151)}
\gppoint{gp mark 0}{(7.276,5.151)}
\gppoint{gp mark 0}{(7.276,5.034)}
\gppoint{gp mark 0}{(7.276,5.412)}
\gppoint{gp mark 0}{(7.276,5.216)}
\gppoint{gp mark 0}{(7.276,4.868)}
\gppoint{gp mark 0}{(7.276,5.286)}
\gppoint{gp mark 0}{(7.276,5.151)}
\gppoint{gp mark 0}{(7.276,5.172)}
\gppoint{gp mark 0}{(7.276,5.362)}
\gppoint{gp mark 0}{(7.276,5.467)}
\gppoint{gp mark 0}{(7.284,5.286)}
\gppoint{gp mark 0}{(7.284,5.003)}
\gppoint{gp mark 0}{(7.284,5.221)}
\gppoint{gp mark 0}{(7.284,5.226)}
\gppoint{gp mark 0}{(7.284,4.965)}
\gppoint{gp mark 0}{(7.284,5.453)}
\gppoint{gp mark 0}{(7.284,5.259)}
\gppoint{gp mark 0}{(7.284,5.420)}
\gppoint{gp mark 0}{(7.284,5.366)}
\gppoint{gp mark 0}{(7.284,4.783)}
\gppoint{gp mark 0}{(7.284,5.086)}
\gppoint{gp mark 0}{(7.284,5.460)}
\gppoint{gp mark 0}{(7.284,5.420)}
\gppoint{gp mark 0}{(7.284,5.752)}
\gppoint{gp mark 0}{(7.284,4.918)}
\gppoint{gp mark 0}{(7.284,5.254)}
\gppoint{gp mark 0}{(7.291,5.016)}
\gppoint{gp mark 0}{(7.291,4.767)}
\gppoint{gp mark 0}{(7.291,5.216)}
\gppoint{gp mark 0}{(7.291,5.495)}
\gppoint{gp mark 0}{(7.291,5.303)}
\gppoint{gp mark 0}{(7.291,5.467)}
\gppoint{gp mark 0}{(7.291,5.263)}
\gppoint{gp mark 0}{(7.291,4.838)}
\gppoint{gp mark 0}{(7.291,5.619)}
\gppoint{gp mark 0}{(7.291,5.478)}
\gppoint{gp mark 0}{(7.291,5.460)}
\gppoint{gp mark 0}{(7.291,5.167)}
\gppoint{gp mark 0}{(7.291,5.526)}
\gppoint{gp mark 0}{(7.291,5.366)}
\gppoint{gp mark 0}{(7.291,5.752)}
\gppoint{gp mark 0}{(7.291,5.478)}
\gppoint{gp mark 0}{(7.291,5.631)}
\gppoint{gp mark 0}{(7.291,5.130)}
\gppoint{gp mark 0}{(7.291,5.235)}
\gppoint{gp mark 0}{(7.291,5.460)}
\gppoint{gp mark 0}{(7.291,5.509)}
\gppoint{gp mark 0}{(7.291,5.167)}
\gppoint{gp mark 0}{(7.299,5.555)}
\gppoint{gp mark 0}{(7.299,5.519)}
\gppoint{gp mark 0}{(7.299,5.860)}
\gppoint{gp mark 0}{(7.299,5.235)}
\gppoint{gp mark 0}{(7.299,4.861)}
\gppoint{gp mark 0}{(7.299,4.938)}
\gppoint{gp mark 0}{(7.299,4.861)}
\gppoint{gp mark 0}{(7.299,5.333)}
\gppoint{gp mark 0}{(7.299,5.362)}
\gppoint{gp mark 0}{(7.299,5.245)}
\gppoint{gp mark 0}{(7.299,5.801)}
\gppoint{gp mark 0}{(7.299,5.307)}
\gppoint{gp mark 0}{(7.299,5.760)}
\gppoint{gp mark 0}{(7.299,5.299)}
\gppoint{gp mark 0}{(7.299,5.329)}
\gppoint{gp mark 0}{(7.299,5.509)}
\gppoint{gp mark 0}{(7.299,5.509)}
\gppoint{gp mark 0}{(7.299,5.052)}
\gppoint{gp mark 0}{(7.306,5.420)}
\gppoint{gp mark 0}{(7.306,5.427)}
\gppoint{gp mark 0}{(7.306,5.495)}
\gppoint{gp mark 0}{(7.306,5.341)}
\gppoint{gp mark 0}{(7.306,5.231)}
\gppoint{gp mark 0}{(7.306,4.875)}
\gppoint{gp mark 0}{(7.306,5.706)}
\gppoint{gp mark 0}{(7.306,5.542)}
\gppoint{gp mark 0}{(7.306,5.604)}
\gppoint{gp mark 0}{(7.306,5.401)}
\gppoint{gp mark 0}{(7.306,5.558)}
\gppoint{gp mark 0}{(7.314,4.904)}
\gppoint{gp mark 0}{(7.314,5.765)}
\gppoint{gp mark 0}{(7.314,5.519)}
\gppoint{gp mark 0}{(7.314,5.312)}
\gppoint{gp mark 0}{(7.314,4.997)}
\gppoint{gp mark 0}{(7.314,5.412)}
\gppoint{gp mark 0}{(7.314,5.478)}
\gppoint{gp mark 0}{(7.314,5.202)}
\gppoint{gp mark 0}{(7.314,5.240)}
\gppoint{gp mark 0}{(7.314,5.162)}
\gppoint{gp mark 0}{(7.314,5.460)}
\gppoint{gp mark 0}{(7.314,5.389)}
\gppoint{gp mark 0}{(7.314,5.586)}
\gppoint{gp mark 0}{(7.314,5.506)}
\gppoint{gp mark 0}{(7.314,5.141)}
\gppoint{gp mark 0}{(7.314,5.735)}
\gppoint{gp mark 0}{(7.314,5.787)}
\gppoint{gp mark 0}{(7.321,5.381)}
\gppoint{gp mark 0}{(7.321,5.673)}
\gppoint{gp mark 0}{(7.321,5.374)}
\gppoint{gp mark 0}{(7.321,4.911)}
\gppoint{gp mark 0}{(7.321,5.097)}
\gppoint{gp mark 0}{(7.321,5.114)}
\gppoint{gp mark 0}{(7.321,5.397)}
\gppoint{gp mark 0}{(7.321,5.389)}
\gppoint{gp mark 0}{(7.321,5.389)}
\gppoint{gp mark 0}{(7.321,5.299)}
\gppoint{gp mark 0}{(7.321,5.706)}
\gppoint{gp mark 0}{(7.321,5.221)}
\gppoint{gp mark 0}{(7.321,5.539)}
\gppoint{gp mark 0}{(7.321,5.714)}
\gppoint{gp mark 0}{(7.321,5.412)}
\gppoint{gp mark 0}{(7.321,5.389)}
\gppoint{gp mark 0}{(7.321,5.216)}
\gppoint{gp mark 0}{(7.321,4.679)}
\gppoint{gp mark 0}{(7.329,5.272)}
\gppoint{gp mark 0}{(7.329,5.690)}
\gppoint{gp mark 0}{(7.329,5.659)}
\gppoint{gp mark 0}{(7.329,5.519)}
\gppoint{gp mark 0}{(7.329,6.032)}
\gppoint{gp mark 0}{(7.329,5.481)}
\gppoint{gp mark 0}{(7.329,5.662)}
\gppoint{gp mark 0}{(7.329,4.823)}
\gppoint{gp mark 0}{(7.329,5.294)}
\gppoint{gp mark 0}{(7.329,5.834)}
\gppoint{gp mark 0}{(7.329,5.207)}
\gppoint{gp mark 0}{(7.329,5.770)}
\gppoint{gp mark 0}{(7.329,6.014)}
\gppoint{gp mark 0}{(7.329,5.815)}
\gppoint{gp mark 0}{(7.329,5.268)}
\gppoint{gp mark 0}{(7.329,5.858)}
\gppoint{gp mark 0}{(7.329,5.034)}
\gppoint{gp mark 0}{(7.336,5.659)}
\gppoint{gp mark 0}{(7.336,5.316)}
\gppoint{gp mark 0}{(7.336,5.141)}
\gppoint{gp mark 0}{(7.336,5.492)}
\gppoint{gp mark 0}{(7.336,5.092)}
\gppoint{gp mark 0}{(7.336,5.642)}
\gppoint{gp mark 0}{(7.336,5.378)}
\gppoint{gp mark 0}{(7.336,5.216)}
\gppoint{gp mark 0}{(7.336,5.119)}
\gppoint{gp mark 0}{(7.336,5.216)}
\gppoint{gp mark 0}{(7.336,5.770)}
\gppoint{gp mark 0}{(7.336,5.492)}
\gppoint{gp mark 0}{(7.336,5.401)}
\gppoint{gp mark 0}{(7.336,5.592)}
\gppoint{gp mark 0}{(7.336,5.401)}
\gppoint{gp mark 0}{(7.343,5.226)}
\gppoint{gp mark 0}{(7.343,6.029)}
\gppoint{gp mark 0}{(7.343,5.701)}
\gppoint{gp mark 0}{(7.343,5.752)}
\gppoint{gp mark 0}{(7.343,5.177)}
\gppoint{gp mark 0}{(7.343,5.474)}
\gppoint{gp mark 0}{(7.343,5.752)}
\gppoint{gp mark 0}{(7.343,4.972)}
\gppoint{gp mark 0}{(7.343,5.619)}
\gppoint{gp mark 0}{(7.343,5.847)}
\gppoint{gp mark 0}{(7.343,5.752)}
\gppoint{gp mark 0}{(7.343,5.485)}
\gppoint{gp mark 0}{(7.343,5.701)}
\gppoint{gp mark 0}{(7.343,6.032)}
\gppoint{gp mark 0}{(7.343,5.589)}
\gppoint{gp mark 0}{(7.350,5.202)}
\gppoint{gp mark 0}{(7.350,5.254)}
\gppoint{gp mark 0}{(7.350,5.495)}
\gppoint{gp mark 0}{(7.350,5.478)}
\gppoint{gp mark 0}{(7.350,5.172)}
\gppoint{gp mark 0}{(7.350,5.254)}
\gppoint{gp mark 0}{(7.350,5.307)}
\gppoint{gp mark 0}{(7.350,5.499)}
\gppoint{gp mark 0}{(7.350,5.706)}
\gppoint{gp mark 0}{(7.350,5.299)}
\gppoint{gp mark 0}{(7.350,5.724)}
\gppoint{gp mark 0}{(7.350,5.022)}
\gppoint{gp mark 0}{(7.350,5.125)}
\gppoint{gp mark 0}{(7.350,5.474)}
\gppoint{gp mark 0}{(7.350,5.552)}
\gppoint{gp mark 0}{(7.350,5.873)}
\gppoint{gp mark 0}{(7.350,5.299)}
\gppoint{gp mark 0}{(7.350,5.831)}
\gppoint{gp mark 0}{(7.350,5.512)}
\gppoint{gp mark 0}{(7.350,5.046)}
\gppoint{gp mark 0}{(7.350,5.221)}
\gppoint{gp mark 0}{(7.350,5.312)}
\gppoint{gp mark 0}{(7.350,5.366)}
\gppoint{gp mark 0}{(7.358,5.668)}
\gppoint{gp mark 0}{(7.358,5.545)}
\gppoint{gp mark 0}{(7.358,5.634)}
\gppoint{gp mark 0}{(7.358,5.651)}
\gppoint{gp mark 0}{(7.358,5.610)}
\gppoint{gp mark 0}{(7.358,5.760)}
\gppoint{gp mark 0}{(7.358,5.299)}
\gppoint{gp mark 0}{(7.358,5.495)}
\gppoint{gp mark 0}{(7.358,5.245)}
\gppoint{gp mark 0}{(7.358,5.740)}
\gppoint{gp mark 0}{(7.358,5.405)}
\gppoint{gp mark 0}{(7.358,5.948)}
\gppoint{gp mark 0}{(7.358,5.679)}
\gppoint{gp mark 0}{(7.358,5.385)}
\gppoint{gp mark 0}{(7.358,5.385)}
\gppoint{gp mark 0}{(7.358,5.022)}
\gppoint{gp mark 0}{(7.358,5.192)}
\gppoint{gp mark 0}{(7.358,5.435)}
\gppoint{gp mark 0}{(7.358,5.449)}
\gppoint{gp mark 0}{(7.358,5.792)}
\gppoint{gp mark 0}{(7.358,6.002)}
\gppoint{gp mark 0}{(7.358,5.349)}
\gppoint{gp mark 0}{(7.358,5.349)}
\gppoint{gp mark 0}{(7.358,5.286)}
\gppoint{gp mark 0}{(7.365,5.362)}
\gppoint{gp mark 0}{(7.365,5.409)}
\gppoint{gp mark 0}{(7.365,5.574)}
\gppoint{gp mark 0}{(7.365,5.086)}
\gppoint{gp mark 0}{(7.365,5.610)}
\gppoint{gp mark 0}{(7.365,5.141)}
\gppoint{gp mark 0}{(7.365,5.216)}
\gppoint{gp mark 0}{(7.365,5.081)}
\gppoint{gp mark 0}{(7.365,5.714)}
\gppoint{gp mark 0}{(7.365,5.936)}
\gppoint{gp mark 0}{(7.365,5.353)}
\gppoint{gp mark 0}{(7.365,5.207)}
\gppoint{gp mark 0}{(7.365,5.676)}
\gppoint{gp mark 0}{(7.365,5.564)}
\gppoint{gp mark 0}{(7.365,5.940)}
\gppoint{gp mark 0}{(7.365,5.263)}
\gppoint{gp mark 0}{(7.365,5.940)}
\gppoint{gp mark 0}{(7.365,5.337)}
\gppoint{gp mark 0}{(7.365,5.353)}
\gppoint{gp mark 0}{(7.365,5.676)}
\gppoint{gp mark 0}{(7.365,5.824)}
\gppoint{gp mark 0}{(7.365,5.464)}
\gppoint{gp mark 0}{(7.365,5.312)}
\gppoint{gp mark 0}{(7.365,6.551)}
\gppoint{gp mark 0}{(7.365,5.197)}
\gppoint{gp mark 0}{(7.372,5.610)}
\gppoint{gp mark 0}{(7.372,5.532)}
\gppoint{gp mark 0}{(7.372,4.549)}
\gppoint{gp mark 0}{(7.372,5.307)}
\gppoint{gp mark 0}{(7.372,5.268)}
\gppoint{gp mark 0}{(7.372,5.583)}
\gppoint{gp mark 0}{(7.372,5.119)}
\gppoint{gp mark 0}{(7.372,5.765)}
\gppoint{gp mark 0}{(7.372,5.645)}
\gppoint{gp mark 0}{(7.372,5.307)}
\gppoint{gp mark 0}{(7.372,5.446)}
\gppoint{gp mark 0}{(7.372,5.307)}
\gppoint{gp mark 0}{(7.372,5.801)}
\gppoint{gp mark 0}{(7.372,6.201)}
\gppoint{gp mark 0}{(7.372,5.202)}
\gppoint{gp mark 0}{(7.379,5.598)}
\gppoint{gp mark 0}{(7.379,5.539)}
\gppoint{gp mark 0}{(7.379,5.381)}
\gppoint{gp mark 0}{(7.379,5.564)}
\gppoint{gp mark 0}{(7.379,5.216)}
\gppoint{gp mark 0}{(7.379,5.711)}
\gppoint{gp mark 0}{(7.379,5.453)}
\gppoint{gp mark 0}{(7.379,5.637)}
\gppoint{gp mark 0}{(7.379,5.845)}
\gppoint{gp mark 0}{(7.379,5.453)}
\gppoint{gp mark 0}{(7.379,5.711)}
\gppoint{gp mark 0}{(7.379,4.945)}
\gppoint{gp mark 0}{(7.379,5.254)}
\gppoint{gp mark 0}{(7.379,5.854)}
\gppoint{gp mark 0}{(7.386,5.381)}
\gppoint{gp mark 0}{(7.386,5.880)}
\gppoint{gp mark 0}{(7.386,5.333)}
\gppoint{gp mark 0}{(7.386,5.799)}
\gppoint{gp mark 0}{(7.386,5.817)}
\gppoint{gp mark 0}{(7.386,4.481)}
\gppoint{gp mark 0}{(7.386,5.362)}
\gppoint{gp mark 0}{(7.386,5.393)}
\gppoint{gp mark 0}{(7.386,5.207)}
\gppoint{gp mark 0}{(7.386,5.207)}
\gppoint{gp mark 0}{(7.386,5.613)}
\gppoint{gp mark 0}{(7.386,5.207)}
\gppoint{gp mark 0}{(7.386,4.945)}
\gppoint{gp mark 0}{(7.386,5.378)}
\gppoint{gp mark 0}{(7.386,5.393)}
\gppoint{gp mark 0}{(7.386,5.679)}
\gppoint{gp mark 0}{(7.386,5.509)}
\gppoint{gp mark 0}{(7.386,5.003)}
\gppoint{gp mark 0}{(7.386,5.207)}
\gppoint{gp mark 0}{(7.386,5.277)}
\gppoint{gp mark 0}{(7.386,5.492)}
\gppoint{gp mark 0}{(7.386,5.366)}
\gppoint{gp mark 0}{(7.386,5.439)}
\gppoint{gp mark 0}{(7.386,5.207)}
\gppoint{gp mark 0}{(7.393,5.362)}
\gppoint{gp mark 0}{(7.393,5.750)}
\gppoint{gp mark 0}{(7.393,5.442)}
\gppoint{gp mark 0}{(7.393,5.312)}
\gppoint{gp mark 0}{(7.393,5.598)}
\gppoint{gp mark 0}{(7.393,5.843)}
\gppoint{gp mark 0}{(7.393,5.221)}
\gppoint{gp mark 0}{(7.393,5.028)}
\gppoint{gp mark 0}{(7.393,6.468)}
\gppoint{gp mark 0}{(7.393,5.750)}
\gppoint{gp mark 0}{(7.393,5.370)}
\gppoint{gp mark 0}{(7.393,5.703)}
\gppoint{gp mark 0}{(7.393,5.290)}
\gppoint{gp mark 0}{(7.393,5.706)}
\gppoint{gp mark 0}{(7.393,5.732)}
\gppoint{gp mark 0}{(7.393,5.750)}
\gppoint{gp mark 0}{(7.393,5.235)}
\gppoint{gp mark 0}{(7.393,5.353)}
\gppoint{gp mark 0}{(7.393,5.063)}
\gppoint{gp mark 0}{(7.393,5.474)}
\gppoint{gp mark 0}{(7.393,5.873)}
\gppoint{gp mark 0}{(7.393,5.750)}
\gppoint{gp mark 0}{(7.393,5.329)}
\gppoint{gp mark 0}{(7.393,5.679)}
\gppoint{gp mark 0}{(7.393,5.794)}
\gppoint{gp mark 0}{(7.393,5.750)}
\gppoint{gp mark 0}{(7.393,5.750)}
\gppoint{gp mark 0}{(7.400,5.808)}
\gppoint{gp mark 0}{(7.400,5.888)}
\gppoint{gp mark 0}{(7.400,5.075)}
\gppoint{gp mark 0}{(7.400,5.303)}
\gppoint{gp mark 0}{(7.400,6.130)}
\gppoint{gp mark 0}{(7.400,5.659)}
\gppoint{gp mark 0}{(7.400,5.598)}
\gppoint{gp mark 0}{(7.400,5.492)}
\gppoint{gp mark 0}{(7.400,5.263)}
\gppoint{gp mark 0}{(7.400,5.838)}
\gppoint{gp mark 0}{(7.400,5.762)}
\gppoint{gp mark 0}{(7.400,5.431)}
\gppoint{gp mark 0}{(7.400,5.631)}
\gppoint{gp mark 0}{(7.400,5.130)}
\gppoint{gp mark 0}{(7.400,5.216)}
\gppoint{gp mark 0}{(7.400,5.216)}
\gppoint{gp mark 0}{(7.407,5.747)}
\gppoint{gp mark 0}{(7.407,5.637)}
\gppoint{gp mark 0}{(7.407,5.981)}
\gppoint{gp mark 0}{(7.407,5.052)}
\gppoint{gp mark 0}{(7.407,5.245)}
\gppoint{gp mark 0}{(7.407,5.272)}
\gppoint{gp mark 0}{(7.407,5.357)}
\gppoint{gp mark 0}{(7.407,5.312)}
\gppoint{gp mark 0}{(7.407,5.478)}
\gppoint{gp mark 0}{(7.407,5.353)}
\gppoint{gp mark 0}{(7.407,5.897)}
\gppoint{gp mark 0}{(7.407,5.801)}
\gppoint{gp mark 0}{(7.407,4.904)}
\gppoint{gp mark 0}{(7.407,5.146)}
\gppoint{gp mark 0}{(7.407,5.135)}
\gppoint{gp mark 0}{(7.407,5.226)}
\gppoint{gp mark 0}{(7.407,5.845)}
\gppoint{gp mark 0}{(7.407,5.516)}
\gppoint{gp mark 0}{(7.407,5.249)}
\gppoint{gp mark 0}{(7.407,5.488)}
\gppoint{gp mark 0}{(7.407,5.727)}
\gppoint{gp mark 0}{(7.407,5.192)}
\gppoint{gp mark 0}{(7.407,6.059)}
\gppoint{gp mark 0}{(7.407,5.416)}
\gppoint{gp mark 0}{(7.407,5.092)}
\gppoint{gp mark 0}{(7.414,6.380)}
\gppoint{gp mark 0}{(7.414,5.827)}
\gppoint{gp mark 0}{(7.414,5.695)}
\gppoint{gp mark 0}{(7.414,5.801)}
\gppoint{gp mark 0}{(7.414,5.022)}
\gppoint{gp mark 0}{(7.414,5.843)}
\gppoint{gp mark 0}{(7.414,5.249)}
\gppoint{gp mark 0}{(7.414,5.586)}
\gppoint{gp mark 0}{(7.414,5.235)}
\gppoint{gp mark 0}{(7.414,4.984)}
\gppoint{gp mark 0}{(7.414,4.918)}
\gppoint{gp mark 0}{(7.414,4.791)}
\gppoint{gp mark 0}{(7.414,5.703)}
\gppoint{gp mark 0}{(7.414,5.827)}
\gppoint{gp mark 0}{(7.414,5.827)}
\gppoint{gp mark 0}{(7.414,5.141)}
\gppoint{gp mark 0}{(7.414,5.125)}
\gppoint{gp mark 0}{(7.414,5.374)}
\gppoint{gp mark 0}{(7.414,5.197)}
\gppoint{gp mark 0}{(7.414,5.226)}
\gppoint{gp mark 0}{(7.414,5.281)}
\gppoint{gp mark 0}{(7.414,5.827)}
\gppoint{gp mark 0}{(7.414,5.058)}
\gppoint{gp mark 0}{(7.414,5.374)}
\gppoint{gp mark 0}{(7.420,5.412)}
\gppoint{gp mark 0}{(7.420,5.424)}
\gppoint{gp mark 0}{(7.420,5.141)}
\gppoint{gp mark 0}{(7.420,5.405)}
\gppoint{gp mark 0}{(7.420,5.231)}
\gppoint{gp mark 0}{(7.420,5.370)}
\gppoint{gp mark 0}{(7.420,5.592)}
\gppoint{gp mark 0}{(7.420,5.172)}
\gppoint{gp mark 0}{(7.420,5.231)}
\gppoint{gp mark 0}{(7.420,5.370)}
\gppoint{gp mark 0}{(7.420,5.312)}
\gppoint{gp mark 0}{(7.420,5.281)}
\gppoint{gp mark 0}{(7.420,5.792)}
\gppoint{gp mark 0}{(7.420,5.648)}
\gppoint{gp mark 0}{(7.420,5.488)}
\gppoint{gp mark 0}{(7.420,5.259)}
\gppoint{gp mark 0}{(7.420,4.938)}
\gppoint{gp mark 0}{(7.420,5.478)}
\gppoint{gp mark 0}{(7.427,5.401)}
\gppoint{gp mark 0}{(7.427,5.393)}
\gppoint{gp mark 0}{(7.427,5.737)}
\gppoint{gp mark 0}{(7.427,5.709)}
\gppoint{gp mark 0}{(7.427,5.167)}
\gppoint{gp mark 0}{(7.427,5.135)}
\gppoint{gp mark 0}{(7.427,5.141)}
\gppoint{gp mark 0}{(7.427,5.668)}
\gppoint{gp mark 0}{(7.427,5.709)}
\gppoint{gp mark 0}{(7.427,5.362)}
\gppoint{gp mark 0}{(7.427,5.709)}
\gppoint{gp mark 0}{(7.427,5.709)}
\gppoint{gp mark 0}{(7.427,5.574)}
\gppoint{gp mark 0}{(7.427,5.679)}
\gppoint{gp mark 0}{(7.427,5.457)}
\gppoint{gp mark 0}{(7.427,5.973)}
\gppoint{gp mark 0}{(7.427,4.938)}
\gppoint{gp mark 0}{(7.427,5.263)}
\gppoint{gp mark 0}{(7.427,5.509)}
\gppoint{gp mark 0}{(7.427,5.397)}
\gppoint{gp mark 0}{(7.427,5.226)}
\gppoint{gp mark 0}{(7.427,5.192)}
\gppoint{gp mark 0}{(7.427,5.226)}
\gppoint{gp mark 0}{(7.427,5.772)}
\gppoint{gp mark 0}{(7.427,5.337)}
\gppoint{gp mark 0}{(7.434,5.046)}
\gppoint{gp mark 0}{(7.434,5.642)}
\gppoint{gp mark 0}{(7.434,5.420)}
\gppoint{gp mark 0}{(7.434,5.281)}
\gppoint{gp mark 0}{(7.434,5.245)}
\gppoint{gp mark 0}{(7.434,5.827)}
\gppoint{gp mark 0}{(7.434,5.034)}
\gppoint{gp mark 0}{(7.434,4.972)}
\gppoint{gp mark 0}{(7.434,5.695)}
\gppoint{gp mark 0}{(7.434,5.519)}
\gppoint{gp mark 0}{(7.434,5.601)}
\gppoint{gp mark 0}{(7.434,5.792)}
\gppoint{gp mark 0}{(7.434,5.665)}
\gppoint{gp mark 0}{(7.434,5.281)}
\gppoint{gp mark 0}{(7.434,5.385)}
\gppoint{gp mark 0}{(7.434,5.732)}
\gppoint{gp mark 0}{(7.434,5.799)}
\gppoint{gp mark 0}{(7.434,5.772)}
\gppoint{gp mark 0}{(7.434,5.362)}
\gppoint{gp mark 0}{(7.441,5.416)}
\gppoint{gp mark 0}{(7.441,5.545)}
\gppoint{gp mark 0}{(7.441,5.416)}
\gppoint{gp mark 0}{(7.441,5.416)}
\gppoint{gp mark 0}{(7.441,5.416)}
\gppoint{gp mark 0}{(7.441,5.416)}
\gppoint{gp mark 0}{(7.441,5.416)}
\gppoint{gp mark 0}{(7.441,5.757)}
\gppoint{gp mark 0}{(7.441,5.263)}
\gppoint{gp mark 0}{(7.441,5.416)}
\gppoint{gp mark 0}{(7.441,5.416)}
\gppoint{gp mark 0}{(7.441,5.416)}
\gppoint{gp mark 0}{(7.441,5.522)}
\gppoint{gp mark 0}{(7.441,5.416)}
\gppoint{gp mark 0}{(7.441,5.509)}
\gppoint{gp mark 0}{(7.441,5.416)}
\gppoint{gp mark 0}{(7.441,5.735)}
\gppoint{gp mark 0}{(7.441,5.416)}
\gppoint{gp mark 0}{(7.441,5.146)}
\gppoint{gp mark 0}{(7.441,5.416)}
\gppoint{gp mark 0}{(7.441,5.416)}
\gppoint{gp mark 0}{(7.441,5.156)}
\gppoint{gp mark 0}{(7.441,5.416)}
\gppoint{gp mark 0}{(7.441,5.416)}
\gppoint{gp mark 0}{(7.441,5.290)}
\gppoint{gp mark 0}{(7.441,5.416)}
\gppoint{gp mark 0}{(7.441,5.409)}
\gppoint{gp mark 0}{(7.441,5.416)}
\gppoint{gp mark 0}{(7.441,5.416)}
\gppoint{gp mark 0}{(7.441,5.416)}
\gppoint{gp mark 0}{(7.441,5.146)}
\gppoint{gp mark 0}{(7.447,5.457)}
\gppoint{gp mark 0}{(7.447,5.324)}
\gppoint{gp mark 0}{(7.447,4.984)}
\gppoint{gp mark 0}{(7.447,5.577)}
\gppoint{gp mark 0}{(7.447,5.586)}
\gppoint{gp mark 0}{(7.447,5.294)}
\gppoint{gp mark 0}{(7.447,5.662)}
\gppoint{gp mark 0}{(7.447,6.009)}
\gppoint{gp mark 0}{(7.447,5.409)}
\gppoint{gp mark 0}{(7.447,5.622)}
\gppoint{gp mark 0}{(7.447,5.860)}
\gppoint{gp mark 0}{(7.447,5.586)}
\gppoint{gp mark 0}{(7.447,5.420)}
\gppoint{gp mark 0}{(7.454,5.294)}
\gppoint{gp mark 0}{(7.454,5.824)}
\gppoint{gp mark 0}{(7.454,5.854)}
\gppoint{gp mark 0}{(7.454,5.824)}
\gppoint{gp mark 0}{(7.454,5.747)}
\gppoint{gp mark 0}{(7.454,5.506)}
\gppoint{gp mark 0}{(7.454,5.752)}
\gppoint{gp mark 0}{(7.454,5.424)}
\gppoint{gp mark 0}{(7.454,5.474)}
\gppoint{gp mark 0}{(7.454,5.815)}
\gppoint{gp mark 0}{(7.454,5.420)}
\gppoint{gp mark 0}{(7.454,6.002)}
\gppoint{gp mark 0}{(7.454,5.651)}
\gppoint{gp mark 0}{(7.454,5.381)}
\gppoint{gp mark 0}{(7.454,5.167)}
\gppoint{gp mark 0}{(7.454,5.409)}
\gppoint{gp mark 0}{(7.454,5.349)}
\gppoint{gp mark 0}{(7.454,5.586)}
\gppoint{gp mark 0}{(7.454,5.564)}
\gppoint{gp mark 0}{(7.454,4.883)}
\gppoint{gp mark 0}{(7.460,5.439)}
\gppoint{gp mark 0}{(7.460,5.211)}
\gppoint{gp mark 0}{(7.460,5.156)}
\gppoint{gp mark 0}{(7.460,5.899)}
\gppoint{gp mark 0}{(7.460,5.516)}
\gppoint{gp mark 0}{(7.460,5.151)}
\gppoint{gp mark 0}{(7.460,6.125)}
\gppoint{gp mark 0}{(7.460,5.058)}
\gppoint{gp mark 0}{(7.460,5.998)}
\gppoint{gp mark 0}{(7.460,5.580)}
\gppoint{gp mark 0}{(7.460,5.474)}
\gppoint{gp mark 0}{(7.460,5.595)}
\gppoint{gp mark 0}{(7.460,5.495)}
\gppoint{gp mark 0}{(7.460,6.091)}
\gppoint{gp mark 0}{(7.460,5.815)}
\gppoint{gp mark 0}{(7.460,5.307)}
\gppoint{gp mark 0}{(7.460,5.226)}
\gppoint{gp mark 0}{(7.460,5.506)}
\gppoint{gp mark 0}{(7.460,5.378)}
\gppoint{gp mark 0}{(7.460,5.366)}
\gppoint{gp mark 0}{(7.460,5.495)}
\gppoint{gp mark 0}{(7.460,5.202)}
\gppoint{gp mark 0}{(7.467,5.254)}
\gppoint{gp mark 0}{(7.467,5.235)}
\gppoint{gp mark 0}{(7.467,5.286)}
\gppoint{gp mark 0}{(7.467,5.345)}
\gppoint{gp mark 0}{(7.467,5.522)}
\gppoint{gp mark 0}{(7.467,5.449)}
\gppoint{gp mark 0}{(7.467,5.762)}
\gppoint{gp mark 0}{(7.467,5.942)}
\gppoint{gp mark 0}{(7.467,5.722)}
\gppoint{gp mark 0}{(7.467,5.439)}
\gppoint{gp mark 0}{(7.467,5.439)}
\gppoint{gp mark 0}{(7.467,5.442)}
\gppoint{gp mark 0}{(7.467,5.442)}
\gppoint{gp mark 0}{(7.467,6.366)}
\gppoint{gp mark 0}{(7.467,5.439)}
\gppoint{gp mark 0}{(7.467,5.362)}
\gppoint{gp mark 0}{(7.467,5.207)}
\gppoint{gp mark 0}{(7.467,5.393)}
\gppoint{gp mark 0}{(7.467,5.439)}
\gppoint{gp mark 0}{(7.467,5.659)}
\gppoint{gp mark 0}{(7.474,5.307)}
\gppoint{gp mark 0}{(7.474,5.378)}
\gppoint{gp mark 0}{(7.474,5.460)}
\gppoint{gp mark 0}{(7.474,5.801)}
\gppoint{gp mark 0}{(7.474,6.367)}
\gppoint{gp mark 0}{(7.474,5.668)}
\gppoint{gp mark 0}{(7.474,5.539)}
\gppoint{gp mark 0}{(7.474,5.604)}
\gppoint{gp mark 0}{(7.474,5.385)}
\gppoint{gp mark 0}{(7.474,5.767)}
\gppoint{gp mark 0}{(7.474,5.509)}
\gppoint{gp mark 0}{(7.474,5.177)}
\gppoint{gp mark 0}{(7.474,5.552)}
\gppoint{gp mark 0}{(7.474,5.294)}
\gppoint{gp mark 0}{(7.474,5.277)}
\gppoint{gp mark 0}{(7.474,5.259)}
\gppoint{gp mark 0}{(7.474,5.353)}
\gppoint{gp mark 0}{(7.480,5.829)}
\gppoint{gp mark 0}{(7.480,5.453)}
\gppoint{gp mark 0}{(7.480,5.449)}
\gppoint{gp mark 0}{(7.480,5.357)}
\gppoint{gp mark 0}{(7.480,5.888)}
\gppoint{gp mark 0}{(7.480,5.495)}
\gppoint{gp mark 0}{(7.480,5.119)}
\gppoint{gp mark 0}{(7.480,5.724)}
\gppoint{gp mark 0}{(7.480,5.701)}
\gppoint{gp mark 0}{(7.480,5.119)}
\gppoint{gp mark 0}{(7.480,5.747)}
\gppoint{gp mark 0}{(7.480,5.249)}
\gppoint{gp mark 0}{(7.480,5.416)}
\gppoint{gp mark 0}{(7.480,5.539)}
\gppoint{gp mark 0}{(7.480,5.259)}
\gppoint{gp mark 0}{(7.480,6.045)}
\gppoint{gp mark 0}{(7.480,5.571)}
\gppoint{gp mark 0}{(7.480,5.427)}
\gppoint{gp mark 0}{(7.480,5.772)}
\gppoint{gp mark 0}{(7.486,5.357)}
\gppoint{gp mark 0}{(7.486,5.290)}
\gppoint{gp mark 0}{(7.486,5.028)}
\gppoint{gp mark 0}{(7.486,5.840)}
\gppoint{gp mark 0}{(7.486,5.592)}
\gppoint{gp mark 0}{(7.486,5.777)}
\gppoint{gp mark 0}{(7.486,5.628)}
\gppoint{gp mark 0}{(7.486,5.416)}
\gppoint{gp mark 0}{(7.486,5.416)}
\gppoint{gp mark 0}{(7.486,5.516)}
\gppoint{gp mark 0}{(7.486,5.187)}
\gppoint{gp mark 0}{(7.486,5.860)}
\gppoint{gp mark 0}{(7.486,5.211)}
\gppoint{gp mark 0}{(7.486,5.854)}
\gppoint{gp mark 0}{(7.486,4.932)}
\gppoint{gp mark 0}{(7.486,5.893)}
\gppoint{gp mark 0}{(7.486,5.357)}
\gppoint{gp mark 0}{(7.486,5.595)}
\gppoint{gp mark 0}{(7.486,5.625)}
\gppoint{gp mark 0}{(7.486,5.385)}
\gppoint{gp mark 0}{(7.493,6.286)}
\gppoint{gp mark 0}{(7.493,5.637)}
\gppoint{gp mark 0}{(7.493,5.040)}
\gppoint{gp mark 0}{(7.493,5.226)}
\gppoint{gp mark 0}{(7.493,5.817)}
\gppoint{gp mark 0}{(7.493,5.366)}
\gppoint{gp mark 0}{(7.493,5.303)}
\gppoint{gp mark 0}{(7.493,5.840)}
\gppoint{gp mark 0}{(7.493,5.960)}
\gppoint{gp mark 0}{(7.493,5.420)}
\gppoint{gp mark 0}{(7.493,5.245)}
\gppoint{gp mark 0}{(7.493,5.948)}
\gppoint{gp mark 0}{(7.493,5.420)}
\gppoint{gp mark 0}{(7.493,5.926)}
\gppoint{gp mark 0}{(7.493,5.381)}
\gppoint{gp mark 0}{(7.493,5.286)}
\gppoint{gp mark 0}{(7.499,5.849)}
\gppoint{gp mark 0}{(7.499,5.580)}
\gppoint{gp mark 0}{(7.499,5.571)}
\gppoint{gp mark 0}{(7.499,5.389)}
\gppoint{gp mark 0}{(7.499,5.277)}
\gppoint{gp mark 0}{(7.499,5.824)}
\gppoint{gp mark 0}{(7.499,5.378)}
\gppoint{gp mark 0}{(7.499,5.502)}
\gppoint{gp mark 0}{(7.499,5.542)}
\gppoint{gp mark 0}{(7.499,5.555)}
\gppoint{gp mark 0}{(7.499,5.028)}
\gppoint{gp mark 0}{(7.499,5.880)}
\gppoint{gp mark 0}{(7.499,5.775)}
\gppoint{gp mark 0}{(7.499,5.750)}
\gppoint{gp mark 0}{(7.499,5.268)}
\gppoint{gp mark 0}{(7.499,5.820)}
\gppoint{gp mark 0}{(7.499,5.389)}
\gppoint{gp mark 0}{(7.506,5.799)}
\gppoint{gp mark 0}{(7.506,6.166)}
\gppoint{gp mark 0}{(7.506,5.092)}
\gppoint{gp mark 0}{(7.506,5.567)}
\gppoint{gp mark 0}{(7.506,5.777)}
\gppoint{gp mark 0}{(7.506,5.329)}
\gppoint{gp mark 0}{(7.506,5.499)}
\gppoint{gp mark 0}{(7.506,4.767)}
\gppoint{gp mark 0}{(7.506,5.378)}
\gppoint{gp mark 0}{(7.506,5.474)}
\gppoint{gp mark 0}{(7.506,5.167)}
\gppoint{gp mark 0}{(7.512,5.397)}
\gppoint{gp mark 0}{(7.512,5.772)}
\gppoint{gp mark 0}{(7.512,5.499)}
\gppoint{gp mark 0}{(7.512,5.320)}
\gppoint{gp mark 0}{(7.512,5.235)}
\gppoint{gp mark 0}{(7.512,5.481)}
\gppoint{gp mark 0}{(7.512,5.277)}
\gppoint{gp mark 0}{(7.512,5.381)}
\gppoint{gp mark 0}{(7.512,5.668)}
\gppoint{gp mark 0}{(7.512,5.329)}
\gppoint{gp mark 0}{(7.512,5.665)}
\gppoint{gp mark 0}{(7.512,5.860)}
\gppoint{gp mark 0}{(7.512,5.449)}
\gppoint{gp mark 0}{(7.512,5.453)}
\gppoint{gp mark 0}{(7.512,5.545)}
\gppoint{gp mark 0}{(7.512,5.046)}
\gppoint{gp mark 0}{(7.518,5.732)}
\gppoint{gp mark 0}{(7.518,5.457)}
\gppoint{gp mark 0}{(7.518,5.616)}
\gppoint{gp mark 0}{(7.518,5.693)}
\gppoint{gp mark 0}{(7.518,5.290)}
\gppoint{gp mark 0}{(7.518,5.385)}
\gppoint{gp mark 0}{(7.518,5.831)}
\gppoint{gp mark 0}{(7.518,5.676)}
\gppoint{gp mark 0}{(7.518,5.040)}
\gppoint{gp mark 0}{(7.518,5.729)}
\gppoint{gp mark 0}{(7.518,6.452)}
\gppoint{gp mark 0}{(7.518,6.452)}
\gppoint{gp mark 0}{(7.518,5.747)}
\gppoint{gp mark 0}{(7.518,5.389)}
\gppoint{gp mark 0}{(7.518,5.457)}
\gppoint{gp mark 0}{(7.518,6.452)}
\gppoint{gp mark 0}{(7.518,5.654)}
\gppoint{gp mark 0}{(7.518,6.187)}
\gppoint{gp mark 0}{(7.518,5.701)}
\gppoint{gp mark 0}{(7.518,5.824)}
\gppoint{gp mark 0}{(7.524,5.622)}
\gppoint{gp mark 0}{(7.524,5.272)}
\gppoint{gp mark 0}{(7.524,5.420)}
\gppoint{gp mark 0}{(7.524,5.737)}
\gppoint{gp mark 0}{(7.524,5.357)}
\gppoint{gp mark 0}{(7.524,5.709)}
\gppoint{gp mark 0}{(7.524,5.529)}
\gppoint{gp mark 0}{(7.524,5.610)}
\gppoint{gp mark 0}{(7.524,5.509)}
\gppoint{gp mark 0}{(7.524,5.752)}
\gppoint{gp mark 0}{(7.524,5.052)}
\gppoint{gp mark 0}{(7.524,5.737)}
\gppoint{gp mark 0}{(7.524,5.752)}
\gppoint{gp mark 0}{(7.524,5.522)}
\gppoint{gp mark 0}{(7.524,5.172)}
\gppoint{gp mark 0}{(7.524,5.424)}
\gppoint{gp mark 0}{(7.524,5.711)}
\gppoint{gp mark 0}{(7.524,5.849)}
\gppoint{gp mark 0}{(7.524,5.268)}
\gppoint{gp mark 0}{(7.524,5.405)}
\gppoint{gp mark 0}{(7.531,5.665)}
\gppoint{gp mark 0}{(7.531,5.693)}
\gppoint{gp mark 0}{(7.531,5.865)}
\gppoint{gp mark 0}{(7.531,5.471)}
\gppoint{gp mark 0}{(7.531,5.353)}
\gppoint{gp mark 0}{(7.531,5.381)}
\gppoint{gp mark 0}{(7.531,5.765)}
\gppoint{gp mark 0}{(7.531,5.312)}
\gppoint{gp mark 0}{(7.531,5.625)}
\gppoint{gp mark 0}{(7.531,5.637)}
\gppoint{gp mark 0}{(7.531,5.673)}
\gppoint{gp mark 0}{(7.531,5.416)}
\gppoint{gp mark 0}{(7.531,5.684)}
\gppoint{gp mark 0}{(7.531,5.268)}
\gppoint{gp mark 0}{(7.531,5.824)}
\gppoint{gp mark 0}{(7.531,5.542)}
\gppoint{gp mark 0}{(7.531,5.058)}
\gppoint{gp mark 0}{(7.531,5.259)}
\gppoint{gp mark 0}{(7.531,5.192)}
\gppoint{gp mark 0}{(7.531,5.966)}
\gppoint{gp mark 0}{(7.531,5.167)}
\gppoint{gp mark 0}{(7.531,5.192)}
\gppoint{gp mark 0}{(7.531,5.409)}
\gppoint{gp mark 0}{(7.537,5.320)}
\gppoint{gp mark 0}{(7.537,5.882)}
\gppoint{gp mark 0}{(7.537,5.598)}
\gppoint{gp mark 0}{(7.537,5.757)}
\gppoint{gp mark 0}{(7.537,5.571)}
\gppoint{gp mark 0}{(7.537,5.272)}
\gppoint{gp mark 0}{(7.537,5.405)}
\gppoint{gp mark 0}{(7.537,5.558)}
\gppoint{gp mark 0}{(7.537,5.737)}
\gppoint{gp mark 0}{(7.537,5.622)}
\gppoint{gp mark 0}{(7.537,5.634)}
\gppoint{gp mark 0}{(7.537,5.642)}
\gppoint{gp mark 0}{(7.537,5.787)}
\gppoint{gp mark 0}{(7.537,4.807)}
\gppoint{gp mark 0}{(7.537,5.930)}
\gppoint{gp mark 0}{(7.537,5.103)}
\gppoint{gp mark 0}{(7.537,5.427)}
\gppoint{gp mark 0}{(7.537,5.519)}
\gppoint{gp mark 0}{(7.537,5.687)}
\gppoint{gp mark 0}{(7.537,5.657)}
\gppoint{gp mark 0}{(7.537,5.698)}
\gppoint{gp mark 0}{(7.543,5.381)}
\gppoint{gp mark 0}{(7.543,5.796)}
\gppoint{gp mark 0}{(7.543,7.196)}
\gppoint{gp mark 0}{(7.543,5.762)}
\gppoint{gp mark 0}{(7.543,5.789)}
\gppoint{gp mark 0}{(7.543,5.789)}
\gppoint{gp mark 0}{(7.543,5.779)}
\gppoint{gp mark 0}{(7.543,5.616)}
\gppoint{gp mark 0}{(7.543,5.695)}
\gppoint{gp mark 0}{(7.543,5.405)}
\gppoint{gp mark 0}{(7.543,5.202)}
\gppoint{gp mark 0}{(7.543,5.952)}
\gppoint{gp mark 0}{(7.543,5.509)}
\gppoint{gp mark 0}{(7.543,5.362)}
\gppoint{gp mark 0}{(7.543,5.703)}
\gppoint{gp mark 0}{(7.543,5.548)}
\gppoint{gp mark 0}{(7.549,5.502)}
\gppoint{gp mark 0}{(7.549,5.645)}
\gppoint{gp mark 0}{(7.549,5.435)}
\gppoint{gp mark 0}{(7.549,5.464)}
\gppoint{gp mark 0}{(7.549,5.081)}
\gppoint{gp mark 0}{(7.549,5.888)}
\gppoint{gp mark 0}{(7.549,5.416)}
\gppoint{gp mark 0}{(7.549,5.389)}
\gppoint{gp mark 0}{(7.549,5.619)}
\gppoint{gp mark 0}{(7.549,5.249)}
\gppoint{gp mark 0}{(7.549,5.779)}
\gppoint{gp mark 0}{(7.549,5.701)}
\gppoint{gp mark 0}{(7.549,5.820)}
\gppoint{gp mark 0}{(7.549,4.868)}
\gppoint{gp mark 0}{(7.549,5.706)}
\gppoint{gp mark 0}{(7.555,5.464)}
\gppoint{gp mark 0}{(7.555,5.216)}
\gppoint{gp mark 0}{(7.555,5.303)}
\gppoint{gp mark 0}{(7.555,5.772)}
\gppoint{gp mark 0}{(7.555,5.571)}
\gppoint{gp mark 0}{(7.555,5.645)}
\gppoint{gp mark 0}{(7.555,5.081)}
\gppoint{gp mark 0}{(7.555,5.427)}
\gppoint{gp mark 0}{(7.555,4.984)}
\gppoint{gp mark 0}{(7.555,5.845)}
\gppoint{gp mark 0}{(7.555,6.602)}
\gppoint{gp mark 0}{(7.555,5.506)}
\gppoint{gp mark 0}{(7.555,5.654)}
\gppoint{gp mark 0}{(7.555,5.899)}
\gppoint{gp mark 0}{(7.555,5.495)}
\gppoint{gp mark 0}{(7.555,5.654)}
\gppoint{gp mark 0}{(7.561,5.442)}
\gppoint{gp mark 0}{(7.561,5.796)}
\gppoint{gp mark 0}{(7.561,5.580)}
\gppoint{gp mark 0}{(7.561,5.745)}
\gppoint{gp mark 0}{(7.561,6.023)}
\gppoint{gp mark 0}{(7.561,5.843)}
\gppoint{gp mark 0}{(7.561,5.854)}
\gppoint{gp mark 0}{(7.561,5.532)}
\gppoint{gp mark 0}{(7.561,5.607)}
\gppoint{gp mark 0}{(7.561,5.878)}
\gppoint{gp mark 0}{(7.561,5.789)}
\gppoint{gp mark 0}{(7.561,6.007)}
\gppoint{gp mark 0}{(7.561,5.631)}
\gppoint{gp mark 0}{(7.561,5.601)}
\gppoint{gp mark 0}{(7.561,5.499)}
\gppoint{gp mark 0}{(7.561,5.119)}
\gppoint{gp mark 0}{(7.567,6.029)}
\gppoint{gp mark 0}{(7.567,5.815)}
\gppoint{gp mark 0}{(7.567,5.845)}
\gppoint{gp mark 0}{(7.567,5.512)}
\gppoint{gp mark 0}{(7.567,5.446)}
\gppoint{gp mark 0}{(7.567,6.496)}
\gppoint{gp mark 0}{(7.567,6.025)}
\gppoint{gp mark 0}{(7.567,5.535)}
\gppoint{gp mark 0}{(7.573,5.752)}
\gppoint{gp mark 0}{(7.573,5.657)}
\gppoint{gp mark 0}{(7.573,5.393)}
\gppoint{gp mark 0}{(7.573,5.869)}
\gppoint{gp mark 0}{(7.573,5.640)}
\gppoint{gp mark 0}{(7.573,5.320)}
\gppoint{gp mark 0}{(7.573,6.078)}
\gppoint{gp mark 0}{(7.579,5.535)}
\gppoint{gp mark 0}{(7.579,5.119)}
\gppoint{gp mark 0}{(7.579,5.272)}
\gppoint{gp mark 0}{(7.579,5.535)}
\gppoint{gp mark 0}{(7.579,5.745)}
\gppoint{gp mark 0}{(7.579,5.097)}
\gppoint{gp mark 0}{(7.579,5.427)}
\gppoint{gp mark 0}{(7.579,5.695)}
\gppoint{gp mark 0}{(7.579,5.449)}
\gppoint{gp mark 0}{(7.579,5.424)}
\gppoint{gp mark 0}{(7.579,6.119)}
\gppoint{gp mark 0}{(7.579,5.272)}
\gppoint{gp mark 0}{(7.579,5.177)}
\gppoint{gp mark 0}{(7.579,5.535)}
\gppoint{gp mark 0}{(7.579,5.535)}
\gppoint{gp mark 0}{(7.579,5.727)}
\gppoint{gp mark 0}{(7.579,5.268)}
\gppoint{gp mark 0}{(7.579,5.535)}
\gppoint{gp mark 0}{(7.579,5.622)}
\gppoint{gp mark 0}{(7.585,5.719)}
\gppoint{gp mark 0}{(7.585,5.760)}
\gppoint{gp mark 0}{(7.585,5.485)}
\gppoint{gp mark 0}{(7.585,5.535)}
\gppoint{gp mark 0}{(7.585,5.485)}
\gppoint{gp mark 0}{(7.585,5.998)}
\gppoint{gp mark 0}{(7.585,5.703)}
\gppoint{gp mark 0}{(7.585,4.897)}
\gppoint{gp mark 0}{(7.585,5.679)}
\gppoint{gp mark 0}{(7.585,5.409)}
\gppoint{gp mark 0}{(7.585,5.580)}
\gppoint{gp mark 0}{(7.585,5.353)}
\gppoint{gp mark 0}{(7.585,5.729)}
\gppoint{gp mark 0}{(7.585,5.642)}
\gppoint{gp mark 0}{(7.585,5.509)}
\gppoint{gp mark 0}{(7.585,5.948)}
\gppoint{gp mark 0}{(7.591,5.942)}
\gppoint{gp mark 0}{(7.591,6.230)}
\gppoint{gp mark 0}{(7.591,5.263)}
\gppoint{gp mark 0}{(7.591,5.742)}
\gppoint{gp mark 0}{(7.591,5.682)}
\gppoint{gp mark 0}{(7.591,5.792)}
\gppoint{gp mark 0}{(7.591,5.631)}
\gppoint{gp mark 0}{(7.591,5.586)}
\gppoint{gp mark 0}{(7.591,5.876)}
\gppoint{gp mark 0}{(7.591,5.631)}
\gppoint{gp mark 0}{(7.591,5.474)}
\gppoint{gp mark 0}{(7.591,5.226)}
\gppoint{gp mark 0}{(7.591,5.341)}
\gppoint{gp mark 0}{(7.591,5.211)}
\gppoint{gp mark 0}{(7.591,5.737)}
\gppoint{gp mark 0}{(7.591,5.856)}
\gppoint{gp mark 0}{(7.597,5.981)}
\gppoint{gp mark 0}{(7.597,5.431)}
\gppoint{gp mark 0}{(7.597,5.226)}
\gppoint{gp mark 0}{(7.597,5.583)}
\gppoint{gp mark 0}{(7.597,6.069)}
\gppoint{gp mark 0}{(7.597,5.735)}
\gppoint{gp mark 0}{(7.597,5.765)}
\gppoint{gp mark 0}{(7.597,5.724)}
\gppoint{gp mark 0}{(7.597,6.023)}
\gppoint{gp mark 0}{(7.597,5.416)}
\gppoint{gp mark 0}{(7.597,5.732)}
\gppoint{gp mark 0}{(7.597,5.499)}
\gppoint{gp mark 0}{(7.603,5.564)}
\gppoint{gp mark 0}{(7.603,5.648)}
\gppoint{gp mark 0}{(7.603,6.127)}
\gppoint{gp mark 0}{(7.603,6.127)}
\gppoint{gp mark 0}{(7.603,5.235)}
\gppoint{gp mark 0}{(7.603,5.471)}
\gppoint{gp mark 0}{(7.603,6.127)}
\gppoint{gp mark 0}{(7.603,5.747)}
\gppoint{gp mark 0}{(7.603,5.519)}
\gppoint{gp mark 0}{(7.603,6.127)}
\gppoint{gp mark 0}{(7.603,5.642)}
\gppoint{gp mark 0}{(7.603,5.467)}
\gppoint{gp mark 0}{(7.603,6.127)}
\gppoint{gp mark 0}{(7.603,5.259)}
\gppoint{gp mark 0}{(7.603,5.412)}
\gppoint{gp mark 0}{(7.608,5.552)}
\gppoint{gp mark 0}{(7.608,5.737)}
\gppoint{gp mark 0}{(7.608,6.078)}
\gppoint{gp mark 0}{(7.608,5.977)}
\gppoint{gp mark 0}{(7.608,5.719)}
\gppoint{gp mark 0}{(7.608,5.467)}
\gppoint{gp mark 0}{(7.608,5.792)}
\gppoint{gp mark 0}{(7.608,5.831)}
\gppoint{gp mark 0}{(7.608,5.956)}
\gppoint{gp mark 0}{(7.608,5.086)}
\gppoint{gp mark 0}{(7.608,5.676)}
\gppoint{gp mark 0}{(7.608,5.854)}
\gppoint{gp mark 0}{(7.608,5.648)}
\gppoint{gp mark 0}{(7.608,5.312)}
\gppoint{gp mark 0}{(7.608,5.409)}
\gppoint{gp mark 0}{(7.608,5.545)}
\gppoint{gp mark 0}{(7.608,6.473)}
\gppoint{gp mark 0}{(7.608,5.852)}
\gppoint{gp mark 0}{(7.608,5.752)}
\gppoint{gp mark 0}{(7.608,6.018)}
\gppoint{gp mark 0}{(7.614,5.916)}
\gppoint{gp mark 0}{(7.614,5.654)}
\gppoint{gp mark 0}{(7.614,5.389)}
\gppoint{gp mark 0}{(7.614,5.796)}
\gppoint{gp mark 0}{(7.614,5.427)}
\gppoint{gp mark 0}{(7.614,5.586)}
\gppoint{gp mark 0}{(7.614,5.316)}
\gppoint{gp mark 0}{(7.614,5.545)}
\gppoint{gp mark 0}{(7.614,5.320)}
\gppoint{gp mark 0}{(7.614,5.393)}
\gppoint{gp mark 0}{(7.614,5.706)}
\gppoint{gp mark 0}{(7.614,5.740)}
\gppoint{gp mark 0}{(7.620,5.151)}
\gppoint{gp mark 0}{(7.620,5.975)}
\gppoint{gp mark 0}{(7.620,5.817)}
\gppoint{gp mark 0}{(7.620,5.659)}
\gppoint{gp mark 0}{(7.620,5.532)}
\gppoint{gp mark 0}{(7.620,5.162)}
\gppoint{gp mark 0}{(7.620,5.671)}
\gppoint{gp mark 0}{(7.620,5.794)}
\gppoint{gp mark 0}{(7.620,5.648)}
\gppoint{gp mark 0}{(7.620,5.662)}
\gppoint{gp mark 0}{(7.620,5.755)}
\gppoint{gp mark 0}{(7.620,5.555)}
\gppoint{gp mark 0}{(7.620,5.625)}
\gppoint{gp mark 0}{(7.620,5.580)}
\gppoint{gp mark 0}{(7.626,5.357)}
\gppoint{gp mark 0}{(7.626,5.969)}
\gppoint{gp mark 0}{(7.626,5.512)}
\gppoint{gp mark 0}{(7.626,6.130)}
\gppoint{gp mark 0}{(7.626,6.130)}
\gppoint{gp mark 0}{(7.626,4.918)}
\gppoint{gp mark 0}{(7.626,6.422)}
\gppoint{gp mark 0}{(7.626,5.628)}
\gppoint{gp mark 0}{(7.626,5.435)}
\gppoint{gp mark 0}{(7.626,5.574)}
\gppoint{gp mark 0}{(7.626,5.427)}
\gppoint{gp mark 0}{(7.626,5.412)}
\gppoint{gp mark 0}{(7.626,5.845)}
\gppoint{gp mark 0}{(7.631,5.709)}
\gppoint{gp mark 0}{(7.631,5.979)}
\gppoint{gp mark 0}{(7.631,5.724)}
\gppoint{gp mark 0}{(7.631,5.634)}
\gppoint{gp mark 0}{(7.631,5.679)}
\gppoint{gp mark 0}{(7.631,5.760)}
\gppoint{gp mark 0}{(7.631,5.645)}
\gppoint{gp mark 0}{(7.631,6.016)}
\gppoint{gp mark 0}{(7.631,5.836)}
\gppoint{gp mark 0}{(7.631,5.601)}
\gppoint{gp mark 0}{(7.631,5.813)}
\gppoint{gp mark 0}{(7.631,5.637)}
\gppoint{gp mark 0}{(7.631,6.081)}
\gppoint{gp mark 0}{(7.637,5.495)}
\gppoint{gp mark 0}{(7.637,5.682)}
\gppoint{gp mark 0}{(7.637,5.865)}
\gppoint{gp mark 0}{(7.637,5.374)}
\gppoint{gp mark 0}{(7.637,5.936)}
\gppoint{gp mark 0}{(7.637,5.682)}
\gppoint{gp mark 0}{(7.637,5.897)}
\gppoint{gp mark 0}{(7.637,5.374)}
\gppoint{gp mark 0}{(7.637,5.182)}
\gppoint{gp mark 0}{(7.637,4.868)}
\gppoint{gp mark 0}{(7.637,5.535)}
\gppoint{gp mark 0}{(7.637,5.616)}
\gppoint{gp mark 0}{(7.637,5.245)}
\gppoint{gp mark 0}{(7.637,5.485)}
\gppoint{gp mark 0}{(7.637,5.485)}
\gppoint{gp mark 0}{(7.637,5.952)}
\gppoint{gp mark 0}{(7.643,5.389)}
\gppoint{gp mark 0}{(7.643,5.724)}
\gppoint{gp mark 0}{(7.643,5.878)}
\gppoint{gp mark 0}{(7.643,5.574)}
\gppoint{gp mark 0}{(7.643,5.337)}
\gppoint{gp mark 0}{(7.643,5.272)}
\gppoint{gp mark 0}{(7.643,5.701)}
\gppoint{gp mark 0}{(7.643,5.485)}
\gppoint{gp mark 0}{(7.643,5.558)}
\gppoint{gp mark 0}{(7.643,5.485)}
\gppoint{gp mark 0}{(7.643,5.146)}
\gppoint{gp mark 0}{(7.643,5.737)}
\gppoint{gp mark 0}{(7.643,5.668)}
\gppoint{gp mark 0}{(7.643,6.023)}
\gppoint{gp mark 0}{(7.643,5.619)}
\gppoint{gp mark 0}{(7.648,5.952)}
\gppoint{gp mark 0}{(7.648,5.488)}
\gppoint{gp mark 0}{(7.648,6.135)}
\gppoint{gp mark 0}{(7.648,6.036)}
\gppoint{gp mark 0}{(7.648,5.772)}
\gppoint{gp mark 0}{(7.648,5.589)}
\gppoint{gp mark 0}{(7.648,5.752)}
\gppoint{gp mark 0}{(7.648,5.299)}
\gppoint{gp mark 0}{(7.648,5.613)}
\gppoint{gp mark 0}{(7.648,5.932)}
\gppoint{gp mark 0}{(7.648,5.272)}
\gppoint{gp mark 0}{(7.648,5.873)}
\gppoint{gp mark 0}{(7.648,5.435)}
\gppoint{gp mark 0}{(7.648,5.852)}
\gppoint{gp mark 0}{(7.648,5.616)}
\gppoint{gp mark 0}{(7.648,5.613)}
\gppoint{gp mark 0}{(7.648,6.094)}
\gppoint{gp mark 0}{(7.654,5.522)}
\gppoint{gp mark 0}{(7.654,5.522)}
\gppoint{gp mark 0}{(7.654,5.895)}
\gppoint{gp mark 0}{(7.654,5.789)}
\gppoint{gp mark 0}{(7.654,5.711)}
\gppoint{gp mark 0}{(7.654,5.789)}
\gppoint{gp mark 0}{(7.654,5.829)}
\gppoint{gp mark 0}{(7.654,5.393)}
\gppoint{gp mark 0}{(7.654,5.698)}
\gppoint{gp mark 0}{(7.654,5.914)}
\gppoint{gp mark 0}{(7.660,5.263)}
\gppoint{gp mark 0}{(7.660,5.337)}
\gppoint{gp mark 0}{(7.660,5.796)}
\gppoint{gp mark 0}{(7.660,5.389)}
\gppoint{gp mark 0}{(7.660,6.125)}
\gppoint{gp mark 0}{(7.660,6.027)}
\gppoint{gp mark 0}{(7.660,5.226)}
\gppoint{gp mark 0}{(7.660,5.294)}
\gppoint{gp mark 0}{(7.660,5.729)}
\gppoint{gp mark 0}{(7.660,5.796)}
\gppoint{gp mark 0}{(7.660,5.829)}
\gppoint{gp mark 0}{(7.665,5.634)}
\gppoint{gp mark 0}{(7.665,5.542)}
\gppoint{gp mark 0}{(7.665,6.091)}
\gppoint{gp mark 0}{(7.665,6.107)}
\gppoint{gp mark 0}{(7.665,6.104)}
\gppoint{gp mark 0}{(7.665,5.281)}
\gppoint{gp mark 0}{(7.665,5.899)}
\gppoint{gp mark 0}{(7.665,5.625)}
\gppoint{gp mark 0}{(7.671,5.992)}
\gppoint{gp mark 0}{(7.671,5.526)}
\gppoint{gp mark 0}{(7.671,5.506)}
\gppoint{gp mark 0}{(7.671,5.303)}
\gppoint{gp mark 0}{(7.671,5.810)}
\gppoint{gp mark 0}{(7.671,5.634)}
\gppoint{gp mark 0}{(7.671,5.956)}
\gppoint{gp mark 0}{(7.671,5.197)}
\gppoint{gp mark 0}{(7.671,5.529)}
\gppoint{gp mark 0}{(7.671,5.869)}
\gppoint{gp mark 0}{(7.671,5.808)}
\gppoint{gp mark 0}{(7.671,5.471)}
\gppoint{gp mark 0}{(7.671,5.687)}
\gppoint{gp mark 0}{(7.671,5.146)}
\gppoint{gp mark 0}{(7.671,5.529)}
\gppoint{gp mark 0}{(7.671,5.651)}
\gppoint{gp mark 0}{(7.676,5.512)}
\gppoint{gp mark 0}{(7.676,5.794)}
\gppoint{gp mark 0}{(7.676,5.586)}
\gppoint{gp mark 0}{(7.676,5.637)}
\gppoint{gp mark 0}{(7.676,5.651)}
\gppoint{gp mark 0}{(7.676,5.613)}
\gppoint{gp mark 0}{(7.676,5.784)}
\gppoint{gp mark 0}{(7.676,6.040)}
\gppoint{gp mark 0}{(7.676,5.277)}
\gppoint{gp mark 0}{(7.676,5.779)}
\gppoint{gp mark 0}{(7.676,5.831)}
\gppoint{gp mark 0}{(7.676,5.873)}
\gppoint{gp mark 0}{(7.676,5.545)}
\gppoint{gp mark 0}{(7.682,5.412)}
\gppoint{gp mark 0}{(7.682,5.610)}
\gppoint{gp mark 0}{(7.682,5.924)}
\gppoint{gp mark 0}{(7.682,5.393)}
\gppoint{gp mark 0}{(7.682,5.316)}
\gppoint{gp mark 0}{(7.687,5.706)}
\gppoint{gp mark 0}{(7.687,5.695)}
\gppoint{gp mark 0}{(7.687,5.775)}
\gppoint{gp mark 0}{(7.687,5.659)}
\gppoint{gp mark 0}{(7.687,5.249)}
\gppoint{gp mark 0}{(7.687,5.706)}
\gppoint{gp mark 0}{(7.687,5.706)}
\gppoint{gp mark 0}{(7.687,5.467)}
\gppoint{gp mark 0}{(7.687,5.907)}
\gppoint{gp mark 0}{(7.687,5.860)}
\gppoint{gp mark 0}{(7.692,6.200)}
\gppoint{gp mark 0}{(7.692,5.796)}
\gppoint{gp mark 0}{(7.692,6.027)}
\gppoint{gp mark 0}{(7.692,5.740)}
\gppoint{gp mark 0}{(7.692,5.370)}
\gppoint{gp mark 0}{(7.692,5.226)}
\gppoint{gp mark 0}{(7.692,5.586)}
\gppoint{gp mark 0}{(7.692,5.460)}
\gppoint{gp mark 0}{(7.692,5.625)}
\gppoint{gp mark 0}{(7.692,5.893)}
\gppoint{gp mark 0}{(7.692,5.622)}
\gppoint{gp mark 0}{(7.692,5.803)}
\gppoint{gp mark 0}{(7.692,5.506)}
\gppoint{gp mark 0}{(7.698,5.977)}
\gppoint{gp mark 0}{(7.698,5.385)}
\gppoint{gp mark 0}{(7.698,6.122)}
\gppoint{gp mark 0}{(7.698,6.122)}
\gppoint{gp mark 0}{(7.698,5.716)}
\gppoint{gp mark 0}{(7.698,5.552)}
\gppoint{gp mark 0}{(7.698,5.063)}
\gppoint{gp mark 0}{(7.698,5.711)}
\gppoint{gp mark 0}{(7.698,5.063)}
\gppoint{gp mark 0}{(7.698,5.409)}
\gppoint{gp mark 0}{(7.698,5.750)}
\gppoint{gp mark 0}{(7.698,5.268)}
\gppoint{gp mark 0}{(7.698,5.401)}
\gppoint{gp mark 0}{(7.698,5.665)}
\gppoint{gp mark 0}{(7.698,5.799)}
\gppoint{gp mark 0}{(7.703,5.416)}
\gppoint{gp mark 0}{(7.703,5.981)}
\gppoint{gp mark 0}{(7.703,5.847)}
\gppoint{gp mark 0}{(7.703,4.984)}
\gppoint{gp mark 0}{(7.703,5.849)}
\gppoint{gp mark 0}{(7.703,5.668)}
\gppoint{gp mark 0}{(7.703,5.787)}
\gppoint{gp mark 0}{(7.703,5.668)}
\gppoint{gp mark 0}{(7.703,6.712)}
\gppoint{gp mark 0}{(7.703,5.849)}
\gppoint{gp mark 0}{(7.703,5.545)}
\gppoint{gp mark 0}{(7.703,5.847)}
\gppoint{gp mark 0}{(7.703,5.499)}
\gppoint{gp mark 0}{(7.703,5.548)}
\gppoint{gp mark 0}{(7.703,5.847)}
\gppoint{gp mark 0}{(7.703,5.722)}
\gppoint{gp mark 0}{(7.703,5.847)}
\gppoint{gp mark 0}{(7.703,5.847)}
\gppoint{gp mark 0}{(7.708,6.011)}
\gppoint{gp mark 0}{(7.708,5.558)}
\gppoint{gp mark 0}{(7.708,5.880)}
\gppoint{gp mark 0}{(7.708,5.240)}
\gppoint{gp mark 0}{(7.708,6.064)}
\gppoint{gp mark 0}{(7.708,5.772)}
\gppoint{gp mark 0}{(7.708,5.912)}
\gppoint{gp mark 0}{(7.708,5.316)}
\gppoint{gp mark 0}{(7.708,5.891)}
\gppoint{gp mark 0}{(7.708,5.668)}
\gppoint{gp mark 0}{(7.708,5.316)}
\gppoint{gp mark 0}{(7.708,5.558)}
\gppoint{gp mark 0}{(7.708,6.050)}
\gppoint{gp mark 0}{(7.714,5.294)}
\gppoint{gp mark 0}{(7.714,6.120)}
\gppoint{gp mark 0}{(7.714,5.192)}
\gppoint{gp mark 0}{(7.714,5.737)}
\gppoint{gp mark 0}{(7.714,5.822)}
\gppoint{gp mark 0}{(7.714,5.762)}
\gppoint{gp mark 0}{(7.714,5.613)}
\gppoint{gp mark 0}{(7.714,5.499)}
\gppoint{gp mark 0}{(7.714,5.526)}
\gppoint{gp mark 0}{(7.714,5.873)}
\gppoint{gp mark 0}{(7.714,6.032)}
\gppoint{gp mark 0}{(7.714,5.580)}
\gppoint{gp mark 0}{(7.719,6.043)}
\gppoint{gp mark 0}{(7.719,5.405)}
\gppoint{gp mark 0}{(7.719,5.583)}
\gppoint{gp mark 0}{(7.719,5.564)}
\gppoint{gp mark 0}{(7.719,5.063)}
\gppoint{gp mark 0}{(7.719,6.003)}
\gppoint{gp mark 0}{(7.719,6.323)}
\gppoint{gp mark 0}{(7.719,5.895)}
\gppoint{gp mark 0}{(7.719,5.583)}
\gppoint{gp mark 0}{(7.719,5.645)}
\gppoint{gp mark 0}{(7.719,5.488)}
\gppoint{gp mark 0}{(7.719,5.863)}
\gppoint{gp mark 0}{(7.724,5.777)}
\gppoint{gp mark 0}{(7.724,5.654)}
\gppoint{gp mark 0}{(7.724,5.464)}
\gppoint{gp mark 0}{(7.730,6.089)}
\gppoint{gp mark 0}{(7.730,5.948)}
\gppoint{gp mark 0}{(7.730,5.772)}
\gppoint{gp mark 0}{(7.730,6.023)}
\gppoint{gp mark 0}{(7.730,5.604)}
\gppoint{gp mark 0}{(7.730,5.897)}
\gppoint{gp mark 0}{(7.730,5.981)}
\gppoint{gp mark 0}{(7.735,5.240)}
\gppoint{gp mark 0}{(7.735,5.385)}
\gppoint{gp mark 0}{(7.735,5.381)}
\gppoint{gp mark 0}{(7.735,5.687)}
\gppoint{gp mark 0}{(7.735,5.642)}
\gppoint{gp mark 0}{(7.735,5.442)}
\gppoint{gp mark 0}{(7.735,5.561)}
\gppoint{gp mark 0}{(7.735,5.640)}
\gppoint{gp mark 0}{(7.735,5.240)}
\gppoint{gp mark 0}{(7.735,6.041)}
\gppoint{gp mark 0}{(7.740,5.467)}
\gppoint{gp mark 0}{(7.740,5.539)}
\gppoint{gp mark 0}{(7.740,5.616)}
\gppoint{gp mark 0}{(7.740,5.474)}
\gppoint{gp mark 0}{(7.740,5.535)}
\gppoint{gp mark 0}{(7.740,5.378)}
\gppoint{gp mark 0}{(7.740,5.662)}
\gppoint{gp mark 0}{(7.745,5.604)}
\gppoint{gp mark 0}{(7.745,5.651)}
\gppoint{gp mark 0}{(7.745,5.539)}
\gppoint{gp mark 0}{(7.745,5.969)}
\gppoint{gp mark 0}{(7.745,5.474)}
\gppoint{gp mark 0}{(7.745,5.613)}
\gppoint{gp mark 0}{(7.745,5.671)}
\gppoint{gp mark 0}{(7.745,5.665)}
\gppoint{gp mark 0}{(7.745,5.509)}
\gppoint{gp mark 0}{(7.745,5.420)}
\gppoint{gp mark 0}{(7.745,5.259)}
\gppoint{gp mark 0}{(7.745,5.245)}
\gppoint{gp mark 0}{(7.750,5.567)}
\gppoint{gp mark 0}{(7.750,6.062)}
\gppoint{gp mark 0}{(7.750,6.204)}
\gppoint{gp mark 0}{(7.750,5.662)}
\gppoint{gp mark 0}{(7.750,5.698)}
\gppoint{gp mark 0}{(7.750,5.481)}
\gppoint{gp mark 0}{(7.750,5.449)}
\gppoint{gp mark 0}{(7.755,5.453)}
\gppoint{gp mark 0}{(7.755,6.016)}
\gppoint{gp mark 0}{(7.755,6.334)}
\gppoint{gp mark 0}{(7.755,5.539)}
\gppoint{gp mark 0}{(7.755,6.079)}
\gppoint{gp mark 0}{(7.761,5.880)}
\gppoint{gp mark 0}{(7.761,6.256)}
\gppoint{gp mark 0}{(7.761,5.320)}
\gppoint{gp mark 0}{(7.761,5.542)}
\gppoint{gp mark 0}{(7.761,5.799)}
\gppoint{gp mark 0}{(7.761,5.401)}
\gppoint{gp mark 0}{(7.761,5.574)}
\gppoint{gp mark 0}{(7.761,5.485)}
\gppoint{gp mark 0}{(7.761,5.847)}
\gppoint{gp mark 0}{(7.766,6.047)}
\gppoint{gp mark 0}{(7.766,5.577)}
\gppoint{gp mark 0}{(7.766,5.944)}
\gppoint{gp mark 0}{(7.766,5.488)}
\gppoint{gp mark 0}{(7.766,5.886)}
\gppoint{gp mark 0}{(7.766,6.016)}
\gppoint{gp mark 0}{(7.766,5.747)}
\gppoint{gp mark 0}{(7.771,5.409)}
\gppoint{gp mark 0}{(7.771,5.488)}
\gppoint{gp mark 0}{(7.771,5.580)}
\gppoint{gp mark 0}{(7.771,5.499)}
\gppoint{gp mark 0}{(7.771,5.932)}
\gppoint{gp mark 0}{(7.771,5.108)}
\gppoint{gp mark 0}{(7.771,5.878)}
\gppoint{gp mark 0}{(7.771,5.701)}
\gppoint{gp mark 0}{(7.771,5.836)}
\gppoint{gp mark 0}{(7.771,5.938)}
\gppoint{gp mark 0}{(7.776,5.272)}
\gppoint{gp mark 0}{(7.776,5.485)}
\gppoint{gp mark 0}{(7.776,5.625)}
\gppoint{gp mark 0}{(7.776,5.752)}
\gppoint{gp mark 0}{(7.781,5.824)}
\gppoint{gp mark 0}{(7.781,5.983)}
\gppoint{gp mark 0}{(7.781,5.616)}
\gppoint{gp mark 0}{(7.781,6.047)}
\gppoint{gp mark 0}{(7.781,6.096)}
\gppoint{gp mark 0}{(7.781,5.558)}
\gppoint{gp mark 0}{(7.781,5.558)}
\gppoint{gp mark 0}{(7.781,5.865)}
\gppoint{gp mark 0}{(7.781,5.628)}
\gppoint{gp mark 0}{(7.781,5.865)}
\gppoint{gp mark 0}{(7.786,5.453)}
\gppoint{gp mark 0}{(7.786,5.849)}
\gppoint{gp mark 0}{(7.786,5.657)}
\gppoint{gp mark 0}{(7.786,5.777)}
\gppoint{gp mark 0}{(7.786,5.548)}
\gppoint{gp mark 0}{(7.786,5.824)}
\gppoint{gp mark 0}{(7.786,6.096)}
\gppoint{gp mark 0}{(7.786,5.755)}
\gppoint{gp mark 0}{(7.786,5.747)}
\gppoint{gp mark 0}{(7.786,6.124)}
\gppoint{gp mark 0}{(7.791,5.424)}
\gppoint{gp mark 0}{(7.791,5.779)}
\gppoint{gp mark 0}{(7.791,5.824)}
\gppoint{gp mark 0}{(7.791,5.583)}
\gppoint{gp mark 0}{(7.791,5.834)}
\gppoint{gp mark 0}{(7.791,6.029)}
\gppoint{gp mark 0}{(7.796,5.457)}
\gppoint{gp mark 0}{(7.796,5.762)}
\gppoint{gp mark 0}{(7.796,6.275)}
\gppoint{gp mark 0}{(7.796,5.838)}
\gppoint{gp mark 0}{(7.796,5.378)}
\gppoint{gp mark 0}{(7.796,5.555)}
\gppoint{gp mark 0}{(7.796,5.052)}
\gppoint{gp mark 0}{(7.796,5.729)}
\gppoint{gp mark 0}{(7.801,5.642)}
\gppoint{gp mark 0}{(7.801,5.235)}
\gppoint{gp mark 0}{(7.801,5.378)}
\gppoint{gp mark 0}{(7.801,5.595)}
\gppoint{gp mark 0}{(7.806,5.580)}
\gppoint{gp mark 0}{(7.806,5.435)}
\gppoint{gp mark 0}{(7.806,5.930)}
\gppoint{gp mark 0}{(7.806,5.662)}
\gppoint{gp mark 0}{(7.806,5.979)}
\gppoint{gp mark 0}{(7.806,5.820)}
\gppoint{gp mark 0}{(7.806,5.762)}
\gppoint{gp mark 0}{(7.810,5.950)}
\gppoint{gp mark 0}{(7.810,5.815)}
\gppoint{gp mark 0}{(7.810,5.701)}
\gppoint{gp mark 0}{(7.810,5.481)}
\gppoint{gp mark 0}{(7.810,5.640)}
\gppoint{gp mark 0}{(7.810,5.567)}
\gppoint{gp mark 0}{(7.815,5.813)}
\gppoint{gp mark 0}{(7.815,5.516)}
\gppoint{gp mark 0}{(7.815,6.286)}
\gppoint{gp mark 0}{(7.820,5.640)}
\gppoint{gp mark 0}{(7.820,5.808)}
\gppoint{gp mark 0}{(7.820,6.050)}
\gppoint{gp mark 0}{(7.820,6.043)}
\gppoint{gp mark 0}{(7.820,6.380)}
\gppoint{gp mark 0}{(7.820,5.727)}
\gppoint{gp mark 0}{(7.825,5.990)}
\gppoint{gp mark 0}{(7.825,5.750)}
\gppoint{gp mark 0}{(7.825,5.735)}
\gppoint{gp mark 0}{(7.825,5.574)}
\gppoint{gp mark 0}{(7.830,5.903)}
\gppoint{gp mark 0}{(7.830,5.645)}
\gppoint{gp mark 0}{(7.830,5.522)}
\gppoint{gp mark 0}{(7.830,6.164)}
\gppoint{gp mark 0}{(7.830,5.431)}
\gppoint{gp mark 0}{(7.835,5.512)}
\gppoint{gp mark 0}{(7.835,5.737)}
\gppoint{gp mark 0}{(7.835,5.362)}
\gppoint{gp mark 0}{(7.835,5.893)}
\gppoint{gp mark 0}{(7.835,5.453)}
\gppoint{gp mark 0}{(7.835,6.052)}
\gppoint{gp mark 0}{(7.839,5.981)}
\gppoint{gp mark 0}{(7.839,5.389)}
\gppoint{gp mark 0}{(7.839,5.439)}
\gppoint{gp mark 0}{(7.839,5.574)}
\gppoint{gp mark 0}{(7.839,5.542)}
\gppoint{gp mark 0}{(7.839,5.916)}
\gppoint{gp mark 0}{(7.839,5.637)}
\gppoint{gp mark 0}{(7.844,5.574)}
\gppoint{gp mark 0}{(7.844,5.353)}
\gppoint{gp mark 0}{(7.844,5.719)}
\gppoint{gp mark 0}{(7.844,5.924)}
\gppoint{gp mark 0}{(7.844,5.950)}
\gppoint{gp mark 0}{(7.849,5.312)}
\gppoint{gp mark 0}{(7.849,5.651)}
\gppoint{gp mark 0}{(7.849,5.695)}
\gppoint{gp mark 0}{(7.849,5.732)}
\gppoint{gp mark 0}{(7.854,5.460)}
\gppoint{gp mark 0}{(7.854,5.729)}
\gppoint{gp mark 0}{(7.854,6.101)}
\gppoint{gp mark 0}{(7.854,5.701)}
\gppoint{gp mark 0}{(7.858,5.849)}
\gppoint{gp mark 0}{(7.858,5.865)}
\gppoint{gp mark 0}{(7.858,5.474)}
\gppoint{gp mark 0}{(7.863,5.909)}
\gppoint{gp mark 0}{(7.863,5.453)}
\gppoint{gp mark 0}{(7.863,5.916)}
\gppoint{gp mark 0}{(7.863,6.224)}
\gppoint{gp mark 0}{(7.863,5.737)}
\gppoint{gp mark 0}{(7.863,5.509)}
\gppoint{gp mark 0}{(7.868,5.668)}
\gppoint{gp mark 0}{(7.868,5.668)}
\gppoint{gp mark 0}{(7.868,5.668)}
\gppoint{gp mark 0}{(7.868,5.668)}
\gppoint{gp mark 0}{(7.868,5.950)}
\gppoint{gp mark 0}{(7.868,5.668)}
\gppoint{gp mark 0}{(7.868,5.435)}
\gppoint{gp mark 0}{(7.872,6.163)}
\gppoint{gp mark 0}{(7.872,5.682)}
\gppoint{gp mark 0}{(7.872,5.907)}
\gppoint{gp mark 0}{(7.872,6.114)}
\gppoint{gp mark 0}{(7.872,6.155)}
\gppoint{gp mark 0}{(7.872,5.690)}
\gppoint{gp mark 0}{(7.872,5.946)}
\gppoint{gp mark 0}{(7.872,5.709)}
\gppoint{gp mark 0}{(7.872,5.580)}
\gppoint{gp mark 0}{(7.872,5.856)}
\gppoint{gp mark 0}{(7.872,6.084)}
\gppoint{gp mark 0}{(7.877,6.088)}
\gppoint{gp mark 0}{(7.877,5.555)}
\gppoint{gp mark 0}{(7.877,5.698)}
\gppoint{gp mark 0}{(7.882,5.616)}
\gppoint{gp mark 0}{(7.882,5.457)}
\gppoint{gp mark 0}{(7.882,5.535)}
\gppoint{gp mark 0}{(7.882,5.824)}
\gppoint{gp mark 0}{(7.886,5.442)}
\gppoint{gp mark 0}{(7.886,5.845)}
\gppoint{gp mark 0}{(7.886,6.169)}
\gppoint{gp mark 0}{(7.886,5.992)}
\gppoint{gp mark 0}{(7.886,5.966)}
\gppoint{gp mark 0}{(7.891,5.676)}
\gppoint{gp mark 0}{(7.891,5.601)}
\gppoint{gp mark 0}{(7.891,5.488)}
\gppoint{gp mark 0}{(7.891,5.177)}
\gppoint{gp mark 0}{(7.891,5.424)}
\gppoint{gp mark 0}{(7.891,6.099)}
\gppoint{gp mark 0}{(7.895,5.485)}
\gppoint{gp mark 0}{(7.900,5.849)}
\gppoint{gp mark 0}{(7.900,5.827)}
\gppoint{gp mark 0}{(7.900,5.519)}
\gppoint{gp mark 0}{(7.900,5.665)}
\gppoint{gp mark 0}{(7.900,5.737)}
\gppoint{gp mark 0}{(7.904,5.711)}
\gppoint{gp mark 0}{(7.904,5.934)}
\gppoint{gp mark 0}{(7.904,5.467)}
\gppoint{gp mark 0}{(7.904,5.467)}
\gppoint{gp mark 0}{(7.904,5.944)}
\gppoint{gp mark 0}{(7.909,5.492)}
\gppoint{gp mark 0}{(7.909,5.719)}
\gppoint{gp mark 0}{(7.909,5.657)}
\gppoint{gp mark 0}{(7.909,5.495)}
\gppoint{gp mark 0}{(7.909,5.156)}
\gppoint{gp mark 0}{(7.909,6.128)}
\gppoint{gp mark 0}{(7.909,6.132)}
\gppoint{gp mark 0}{(7.913,5.836)}
\gppoint{gp mark 0}{(7.913,5.571)}
\gppoint{gp mark 0}{(7.913,5.824)}
\gppoint{gp mark 0}{(7.913,5.519)}
\gppoint{gp mark 0}{(7.913,5.574)}
\gppoint{gp mark 0}{(7.918,5.499)}
\gppoint{gp mark 0}{(7.918,5.427)}
\gppoint{gp mark 0}{(7.918,5.162)}
\gppoint{gp mark 0}{(7.918,5.852)}
\gppoint{gp mark 0}{(7.918,5.994)}
\gppoint{gp mark 0}{(7.918,5.806)}
\gppoint{gp mark 0}{(7.918,5.886)}
\gppoint{gp mark 0}{(7.918,5.852)}
\gppoint{gp mark 0}{(7.922,5.162)}
\gppoint{gp mark 0}{(7.922,5.467)}
\gppoint{gp mark 0}{(7.922,5.467)}
\gppoint{gp mark 0}{(7.922,6.319)}
\gppoint{gp mark 0}{(7.922,6.013)}
\gppoint{gp mark 0}{(7.922,5.467)}
\gppoint{gp mark 0}{(7.922,6.128)}
\gppoint{gp mark 0}{(7.922,5.757)}
\gppoint{gp mark 0}{(7.927,5.616)}
\gppoint{gp mark 0}{(7.927,5.703)}
\gppoint{gp mark 0}{(7.927,5.439)}
\gppoint{gp mark 0}{(7.927,5.703)}
\gppoint{gp mark 0}{(7.927,5.740)}
\gppoint{gp mark 0}{(7.927,5.409)}
\gppoint{gp mark 0}{(7.927,5.742)}
\gppoint{gp mark 0}{(7.927,6.057)}
\gppoint{gp mark 0}{(7.927,6.141)}
\gppoint{gp mark 0}{(7.927,6.193)}
\gppoint{gp mark 0}{(7.931,5.873)}
\gppoint{gp mark 0}{(7.931,5.735)}
\gppoint{gp mark 0}{(7.931,5.789)}
\gppoint{gp mark 0}{(7.931,5.871)}
\gppoint{gp mark 0}{(7.931,6.011)}
\gppoint{gp mark 0}{(7.931,5.856)}
\gppoint{gp mark 0}{(7.936,5.580)}
\gppoint{gp mark 0}{(7.936,6.038)}
\gppoint{gp mark 0}{(7.936,5.873)}
\gppoint{gp mark 0}{(7.940,5.532)}
\gppoint{gp mark 0}{(7.940,5.990)}
\gppoint{gp mark 0}{(7.944,6.032)}
\gppoint{gp mark 0}{(7.944,6.195)}
\gppoint{gp mark 0}{(7.944,5.987)}
\gppoint{gp mark 0}{(7.949,5.506)}
\gppoint{gp mark 0}{(7.949,5.439)}
\gppoint{gp mark 0}{(7.953,5.281)}
\gppoint{gp mark 0}{(7.953,5.926)}
\gppoint{gp mark 0}{(7.957,6.475)}
\gppoint{gp mark 0}{(7.957,5.679)}
\gppoint{gp mark 0}{(7.957,6.671)}
\gppoint{gp mark 0}{(7.957,5.574)}
\gppoint{gp mark 0}{(7.957,5.727)}
\gppoint{gp mark 0}{(7.957,5.869)}
\gppoint{gp mark 0}{(7.957,5.765)}
\gppoint{gp mark 0}{(7.957,6.210)}
\gppoint{gp mark 0}{(7.957,6.198)}
\gppoint{gp mark 0}{(7.957,5.899)}
\gppoint{gp mark 0}{(7.957,5.545)}
\gppoint{gp mark 0}{(7.962,5.389)}
\gppoint{gp mark 0}{(7.962,5.747)}
\gppoint{gp mark 0}{(7.962,5.519)}
\gppoint{gp mark 0}{(7.962,5.312)}
\gppoint{gp mark 0}{(7.962,6.268)}
\gppoint{gp mark 0}{(7.966,5.878)}
\gppoint{gp mark 0}{(7.966,5.966)}
\gppoint{gp mark 0}{(7.966,5.690)}
\gppoint{gp mark 0}{(7.966,5.777)}
\gppoint{gp mark 0}{(7.966,5.727)}
\gppoint{gp mark 0}{(7.970,6.179)}
\gppoint{gp mark 0}{(7.970,5.453)}
\gppoint{gp mark 0}{(7.970,5.397)}
\gppoint{gp mark 0}{(7.974,5.893)}
\gppoint{gp mark 0}{(7.974,5.535)}
\gppoint{gp mark 0}{(7.974,5.817)}
\gppoint{gp mark 0}{(7.979,5.607)}
\gppoint{gp mark 0}{(7.979,5.703)}
\gppoint{gp mark 0}{(7.979,5.895)}
\gppoint{gp mark 0}{(7.983,5.719)}
\gppoint{gp mark 0}{(7.983,6.438)}
\gppoint{gp mark 0}{(7.987,5.905)}
\gppoint{gp mark 0}{(7.987,5.930)}
\gppoint{gp mark 0}{(7.987,5.564)}
\gppoint{gp mark 0}{(7.987,5.843)}
\gppoint{gp mark 0}{(7.987,5.564)}
\gppoint{gp mark 0}{(7.987,6.493)}
\gppoint{gp mark 0}{(7.987,6.433)}
\gppoint{gp mark 0}{(7.987,5.794)}
\gppoint{gp mark 0}{(7.991,6.209)}
\gppoint{gp mark 0}{(7.991,6.315)}
\gppoint{gp mark 0}{(7.991,6.125)}
\gppoint{gp mark 0}{(7.991,6.395)}
\gppoint{gp mark 0}{(7.996,6.401)}
\gppoint{gp mark 0}{(7.996,5.757)}
\gppoint{gp mark 0}{(7.996,5.616)}
\gppoint{gp mark 0}{(7.996,6.050)}
\gppoint{gp mark 0}{(7.996,5.682)}
\gppoint{gp mark 0}{(7.996,5.610)}
\gppoint{gp mark 0}{(8.000,5.637)}
\gppoint{gp mark 0}{(8.000,5.616)}
\gppoint{gp mark 0}{(8.000,5.659)}
\gppoint{gp mark 0}{(8.000,5.682)}
\gppoint{gp mark 0}{(8.000,5.695)}
\gppoint{gp mark 0}{(8.000,5.519)}
\gppoint{gp mark 0}{(8.004,5.727)}
\gppoint{gp mark 0}{(8.004,5.934)}
\gppoint{gp mark 0}{(8.008,5.648)}
\gppoint{gp mark 0}{(8.008,5.552)}
\gppoint{gp mark 0}{(8.008,5.750)}
\gppoint{gp mark 0}{(8.008,6.117)}
\gppoint{gp mark 0}{(8.008,5.847)}
\gppoint{gp mark 0}{(8.008,5.631)}
\gppoint{gp mark 0}{(8.012,5.962)}
\gppoint{gp mark 0}{(8.012,6.455)}
\gppoint{gp mark 0}{(8.012,5.682)}
\gppoint{gp mark 0}{(8.012,5.729)}
\gppoint{gp mark 0}{(8.016,5.662)}
\gppoint{gp mark 0}{(8.016,5.706)}
\gppoint{gp mark 0}{(8.016,5.750)}
\gppoint{gp mark 0}{(8.016,5.637)}
\gppoint{gp mark 0}{(8.020,6.045)}
\gppoint{gp mark 0}{(8.020,6.050)}
\gppoint{gp mark 0}{(8.020,6.753)}
\gppoint{gp mark 0}{(8.020,5.847)}
\gppoint{gp mark 0}{(8.020,5.610)}
\gppoint{gp mark 0}{(8.020,5.794)}
\gppoint{gp mark 0}{(8.024,5.698)}
\gppoint{gp mark 0}{(8.024,5.645)}
\gppoint{gp mark 0}{(8.024,5.865)}
\gppoint{gp mark 0}{(8.024,6.783)}
\gppoint{gp mark 0}{(8.028,5.873)}
\gppoint{gp mark 0}{(8.028,5.845)}
\gppoint{gp mark 0}{(8.028,5.724)}
\gppoint{gp mark 0}{(8.028,5.782)}
\gppoint{gp mark 0}{(8.033,5.782)}
\gppoint{gp mark 0}{(8.033,5.727)}
\gppoint{gp mark 0}{(8.033,6.201)}
\gppoint{gp mark 0}{(8.037,5.893)}
\gppoint{gp mark 0}{(8.037,5.815)}
\gppoint{gp mark 0}{(8.037,6.088)}
\gppoint{gp mark 0}{(8.041,6.201)}
\gppoint{gp mark 0}{(8.041,5.492)}
\gppoint{gp mark 0}{(8.041,5.589)}
\gppoint{gp mark 0}{(8.041,5.502)}
\gppoint{gp mark 0}{(8.041,5.836)}
\gppoint{gp mark 0}{(8.041,5.992)}
\gppoint{gp mark 0}{(8.041,6.379)}
\gppoint{gp mark 0}{(8.041,6.213)}
\gppoint{gp mark 0}{(8.041,5.552)}
\gppoint{gp mark 0}{(8.041,6.020)}
\gppoint{gp mark 0}{(8.045,5.690)}
\gppoint{gp mark 0}{(8.045,5.794)}
\gppoint{gp mark 0}{(8.045,5.770)}
\gppoint{gp mark 0}{(8.045,6.633)}
\gppoint{gp mark 0}{(8.045,5.792)}
\gppoint{gp mark 0}{(8.045,5.909)}
\gppoint{gp mark 0}{(8.045,6.352)}
\gppoint{gp mark 0}{(8.049,6.244)}
\gppoint{gp mark 0}{(8.049,5.834)}
\gppoint{gp mark 0}{(8.049,5.657)}
\gppoint{gp mark 0}{(8.049,5.871)}
\gppoint{gp mark 0}{(8.049,6.458)}
\gppoint{gp mark 0}{(8.053,6.552)}
\gppoint{gp mark 0}{(8.053,5.634)}
\gppoint{gp mark 0}{(8.053,5.971)}
\gppoint{gp mark 0}{(8.053,6.552)}
\gppoint{gp mark 0}{(8.053,5.637)}
\gppoint{gp mark 0}{(8.053,5.737)}
\gppoint{gp mark 0}{(8.053,5.539)}
\gppoint{gp mark 0}{(8.053,5.928)}
\gppoint{gp mark 0}{(8.053,5.876)}
\gppoint{gp mark 0}{(8.057,6.216)}
\gppoint{gp mark 0}{(8.057,6.210)}
\gppoint{gp mark 0}{(8.057,6.069)}
\gppoint{gp mark 0}{(8.057,5.631)}
\gppoint{gp mark 0}{(8.057,6.009)}
\gppoint{gp mark 0}{(8.057,5.747)}
\gppoint{gp mark 0}{(8.061,6.424)}
\gppoint{gp mark 0}{(8.061,5.592)}
\gppoint{gp mark 0}{(8.061,5.796)}
\gppoint{gp mark 0}{(8.064,5.679)}
\gppoint{gp mark 0}{(8.064,5.801)}
\gppoint{gp mark 0}{(8.064,5.854)}
\gppoint{gp mark 0}{(8.064,6.195)}
\gppoint{gp mark 0}{(8.064,5.719)}
\gppoint{gp mark 0}{(8.064,5.727)}
\gppoint{gp mark 0}{(8.068,5.831)}
\gppoint{gp mark 0}{(8.068,5.701)}
\gppoint{gp mark 0}{(8.068,5.831)}
\gppoint{gp mark 0}{(8.068,5.519)}
\gppoint{gp mark 0}{(8.072,5.834)}
\gppoint{gp mark 0}{(8.076,5.836)}
\gppoint{gp mark 0}{(8.076,5.571)}
\gppoint{gp mark 0}{(8.076,5.732)}
\gppoint{gp mark 0}{(8.076,6.258)}
\gppoint{gp mark 0}{(8.076,5.860)}
\gppoint{gp mark 0}{(8.076,5.815)}
\gppoint{gp mark 0}{(8.076,5.665)}
\gppoint{gp mark 0}{(8.080,5.532)}
\gppoint{gp mark 0}{(8.080,5.824)}
\gppoint{gp mark 0}{(8.080,6.219)}
\gppoint{gp mark 0}{(8.080,5.607)}
\gppoint{gp mark 0}{(8.080,5.607)}
\gppoint{gp mark 0}{(8.084,5.580)}
\gppoint{gp mark 0}{(8.084,6.456)}
\gppoint{gp mark 0}{(8.084,5.601)}
\gppoint{gp mark 0}{(8.084,5.863)}
\gppoint{gp mark 0}{(8.088,5.992)}
\gppoint{gp mark 0}{(8.088,5.716)}
\gppoint{gp mark 0}{(8.088,5.880)}
\gppoint{gp mark 0}{(8.092,5.695)}
\gppoint{gp mark 0}{(8.092,5.964)}
\gppoint{gp mark 0}{(8.092,6.062)}
\gppoint{gp mark 0}{(8.092,5.714)}
\gppoint{gp mark 0}{(8.092,5.849)}
\gppoint{gp mark 0}{(8.092,6.267)}
\gppoint{gp mark 0}{(8.092,5.922)}
\gppoint{gp mark 0}{(8.092,5.727)}
\gppoint{gp mark 0}{(8.092,5.695)}
\gppoint{gp mark 0}{(8.096,6.014)}
\gppoint{gp mark 0}{(8.096,5.849)}
\gppoint{gp mark 0}{(8.099,5.854)}
\gppoint{gp mark 0}{(8.099,5.240)}
\gppoint{gp mark 0}{(8.099,5.716)}
\gppoint{gp mark 0}{(8.103,6.018)}
\gppoint{gp mark 0}{(8.103,5.719)}
\gppoint{gp mark 0}{(8.107,5.740)}
\gppoint{gp mark 0}{(8.107,5.610)}
\gppoint{gp mark 0}{(8.107,5.954)}
\gppoint{gp mark 0}{(8.107,5.849)}
\gppoint{gp mark 0}{(8.107,5.698)}
\gppoint{gp mark 0}{(8.111,5.424)}
\gppoint{gp mark 0}{(8.111,5.555)}
\gppoint{gp mark 0}{(8.115,5.813)}
\gppoint{gp mark 0}{(8.118,5.840)}
\gppoint{gp mark 0}{(8.118,6.320)}
\gppoint{gp mark 0}{(8.118,6.346)}
\gppoint{gp mark 0}{(8.118,6.214)}
\gppoint{gp mark 0}{(8.122,6.610)}
\gppoint{gp mark 0}{(8.122,5.665)}
\gppoint{gp mark 0}{(8.126,6.016)}
\gppoint{gp mark 0}{(8.126,5.467)}
\gppoint{gp mark 0}{(8.130,5.772)}
\gppoint{gp mark 0}{(8.133,5.642)}
\gppoint{gp mark 0}{(8.133,5.676)}
\gppoint{gp mark 0}{(8.133,5.665)}
\gppoint{gp mark 0}{(8.133,6.306)}
\gppoint{gp mark 0}{(8.137,5.750)}
\gppoint{gp mark 0}{(8.137,5.695)}
\gppoint{gp mark 0}{(8.141,5.561)}
\gppoint{gp mark 0}{(8.148,6.420)}
\gppoint{gp mark 0}{(8.148,5.956)}
\gppoint{gp mark 0}{(8.148,5.789)}
\gppoint{gp mark 0}{(8.148,5.724)}
\gppoint{gp mark 0}{(8.148,6.280)}
\gppoint{gp mark 0}{(8.159,5.628)}
\gppoint{gp mark 0}{(8.159,5.886)}
\gppoint{gp mark 0}{(8.163,5.668)}
\gppoint{gp mark 0}{(8.166,6.535)}
\gppoint{gp mark 0}{(8.166,5.803)}
\gppoint{gp mark 0}{(8.170,5.891)}
\gppoint{gp mark 0}{(8.174,6.018)}
\gppoint{gp mark 0}{(8.174,5.907)}
\gppoint{gp mark 0}{(8.181,5.869)}
\gppoint{gp mark 0}{(8.184,5.634)}
\gppoint{gp mark 0}{(8.184,6.397)}
\gppoint{gp mark 0}{(8.184,6.478)}
\gppoint{gp mark 0}{(8.188,5.745)}
\gppoint{gp mark 0}{(8.188,5.775)}
\gppoint{gp mark 0}{(8.191,6.328)}
\gppoint{gp mark 0}{(8.195,5.760)}
\gppoint{gp mark 0}{(8.195,5.888)}
\gppoint{gp mark 0}{(8.195,5.719)}
\gppoint{gp mark 0}{(8.199,5.722)}
\gppoint{gp mark 0}{(8.202,5.735)}
\gppoint{gp mark 0}{(8.206,6.130)}
\gppoint{gp mark 0}{(8.209,6.413)}
\gppoint{gp mark 0}{(8.209,6.310)}
\gppoint{gp mark 0}{(8.212,6.775)}
\gppoint{gp mark 0}{(8.216,6.435)}
\gppoint{gp mark 0}{(8.219,5.716)}
\gppoint{gp mark 0}{(8.219,5.729)}
\gppoint{gp mark 0}{(8.223,5.806)}
\gppoint{gp mark 0}{(8.223,6.233)}
\gppoint{gp mark 0}{(8.250,6.371)}
\gppoint{gp mark 0}{(8.254,5.607)}
\gppoint{gp mark 0}{(8.257,6.480)}
\gppoint{gp mark 0}{(8.257,6.495)}
\gppoint{gp mark 0}{(8.257,5.886)}
\gppoint{gp mark 0}{(8.260,5.918)}
\gppoint{gp mark 0}{(8.264,5.829)}
\gppoint{gp mark 0}{(8.267,6.296)}
\gppoint{gp mark 0}{(8.267,6.117)}
\gppoint{gp mark 0}{(8.270,6.363)}
\gppoint{gp mark 0}{(8.273,5.762)}
\gppoint{gp mark 0}{(8.277,6.062)}
\gppoint{gp mark 0}{(8.277,5.485)}
\gppoint{gp mark 0}{(8.283,6.124)}
\gppoint{gp mark 0}{(8.287,5.914)}
\gppoint{gp mark 0}{(8.290,5.727)}
\gppoint{gp mark 0}{(8.293,5.843)}
\gppoint{gp mark 0}{(8.296,5.671)}
\gppoint{gp mark 0}{(8.300,5.901)}
\gppoint{gp mark 0}{(8.303,5.711)}
\gppoint{gp mark 0}{(8.309,5.852)}
\gppoint{gp mark 0}{(8.312,6.332)}
\gppoint{gp mark 0}{(8.316,6.173)}
\gppoint{gp mark 0}{(8.319,5.994)}
\gppoint{gp mark 0}{(8.322,6.293)}
\gppoint{gp mark 0}{(8.322,5.481)}
\gppoint{gp mark 0}{(8.325,6.111)}
\gppoint{gp mark 0}{(8.328,5.745)}
\gppoint{gp mark 0}{(8.332,6.728)}
\gppoint{gp mark 0}{(8.338,6.040)}
\gppoint{gp mark 0}{(8.347,5.973)}
\gppoint{gp mark 0}{(8.353,6.343)}
\gppoint{gp mark 0}{(8.366,6.874)}
\gppoint{gp mark 0}{(8.375,5.950)}
\gppoint{gp mark 0}{(8.375,6.328)}
\gppoint{gp mark 0}{(8.375,5.735)}
\gppoint{gp mark 0}{(8.375,6.545)}
\gppoint{gp mark 0}{(8.378,5.858)}
\gppoint{gp mark 0}{(8.381,5.719)}
\gppoint{gp mark 0}{(8.390,5.777)}
\gppoint{gp mark 0}{(8.416,5.990)}
\gppoint{gp mark 0}{(8.425,6.380)}
\gppoint{gp mark 0}{(8.428,7.160)}
\gppoint{gp mark 0}{(8.437,6.370)}
\gppoint{gp mark 0}{(8.440,6.389)}
\gppoint{gp mark 0}{(8.442,6.588)}
\gppoint{gp mark 0}{(8.454,5.801)}
\gppoint{gp mark 0}{(8.454,5.607)}
\gppoint{gp mark 0}{(8.462,5.891)}
\gppoint{gp mark 0}{(8.474,5.558)}
\gppoint{gp mark 0}{(8.487,6.160)}
\gppoint{gp mark 0}{(8.493,6.309)}
\gppoint{gp mark 0}{(8.496,6.450)}
\gppoint{gp mark 0}{(8.498,5.719)}
\gppoint{gp mark 0}{(8.515,5.353)}
\gppoint{gp mark 0}{(8.517,6.119)}
\gppoint{gp mark 0}{(8.520,6.661)}
\gppoint{gp mark 0}{(8.531,5.979)}
\gppoint{gp mark 0}{(8.533,5.787)}
\gppoint{gp mark 0}{(8.554,6.790)}
\gppoint{gp mark 0}{(8.565,6.117)}
\gppoint{gp mark 0}{(8.570,5.888)}
\gppoint{gp mark 0}{(8.572,7.777)}
\gppoint{gp mark 0}{(8.575,6.127)}
\gppoint{gp mark 0}{(8.580,5.464)}
\gppoint{gp mark 0}{(8.603,6.136)}
\gppoint{gp mark 0}{(8.618,6.054)}
\gppoint{gp mark 0}{(8.637,4.775)}
\gppoint{gp mark 0}{(8.664,5.924)}
\gppoint{gp mark 0}{(8.676,5.967)}
\gppoint{gp mark 0}{(8.685,6.093)}
\gppoint{gp mark 0}{(8.749,5.762)}
\gppoint{gp mark 0}{(8.755,6.157)}
\gppoint{gp mark 0}{(8.766,5.401)}
\gppoint{gp mark 0}{(8.811,6.216)}
\gppoint{gp mark 0}{(8.834,7.151)}
\gppoint{gp mark 0}{(8.858,6.404)}
\gppoint{gp mark 0}{(8.925,6.308)}
\gppoint{gp mark 0}{(8.960,6.630)}
\gppoint{gp mark 0}{(8.981,6.313)}
\gppoint{gp mark 0}{(9.041,8.349)}
\gppoint{gp mark 0}{(9.076,6.204)}
\gppoint{gp mark 0}{(9.090,6.410)}
\gppoint{gp mark 0}{(9.142,6.279)}
\gppoint{gp mark 0}{(9.203,6.568)}
\gppoint{gp mark 0}{(9.258,6.592)}
\gppoint{gp mark 0}{(9.417,8.008)}
\gppoint{gp mark 0}{(9.502,6.190)}
\gppoint{gp mark 0}{(9.642,6.601)}
\gppoint{gp mark 0}{(9.842,7.780)}
\gppoint{gp mark 0}{(9.935,7.134)}
\node[gp node left] at (9.927,1.935) {$N$ };
\gpcolor{color=gp lt color 2}
\gpsetlinetype{gp lt plot 0}
\draw[gp path] (10.663,1.935)--(11.579,1.935);
\draw[gp path] (1.320,1.201)--(1.427,1.276)--(1.535,1.350)--(1.642,1.425)--(1.749,1.500)%
  --(1.857,1.574)--(1.964,1.649)--(2.071,1.724)--(2.179,1.799)--(2.286,1.873)--(2.393,1.948)%
  --(2.501,2.023)--(2.608,2.097)--(2.715,2.172)--(2.823,2.247)--(2.930,2.322)--(3.037,2.396)%
  --(3.145,2.471)--(3.252,2.546)--(3.360,2.620)--(3.467,2.695)--(3.574,2.770)--(3.682,2.844)%
  --(3.789,2.919)--(3.896,2.994)--(4.004,3.069)--(4.111,3.143)--(4.218,3.218)--(4.326,3.293)%
  --(4.433,3.367)--(4.540,3.442)--(4.648,3.517)--(4.755,3.592)--(4.862,3.666)--(4.970,3.741)%
  --(5.077,3.816)--(5.184,3.890)--(5.292,3.965)--(5.399,4.040)--(5.506,4.114)--(5.614,4.189)%
  --(5.721,4.264)--(5.828,4.339)--(5.936,4.413)--(6.043,4.488)--(6.150,4.563)--(6.258,4.637)%
  --(6.365,4.712)--(6.472,4.787)--(6.580,4.862)--(6.687,4.936)--(6.795,5.011)--(6.902,5.086)%
  --(7.009,5.160)--(7.117,5.235)--(7.224,5.310)--(7.331,5.385)--(7.439,5.459)--(7.546,5.534)%
  --(7.653,5.609)--(7.761,5.683)--(7.868,5.758)--(7.975,5.833)--(8.083,5.907)--(8.190,5.982)%
  --(8.297,6.057)--(8.405,6.132)--(8.512,6.206)--(8.619,6.281)--(8.727,6.356)--(8.834,6.430)%
  --(8.941,6.505)--(9.049,6.580)--(9.156,6.655)--(9.263,6.729)--(9.371,6.804)--(9.478,6.879)%
  --(9.585,6.953)--(9.693,7.028)--(9.800,7.103)--(9.907,7.177)--(10.015,7.252)--(10.122,7.327)%
  --(10.230,7.402)--(10.337,7.476)--(10.444,7.551)--(10.552,7.626)--(10.659,7.700)--(10.766,7.775)%
  --(10.874,7.850)--(10.981,7.925)--(11.088,7.999)--(11.196,8.074)--(11.303,8.149)--(11.410,8.223)%
  --(11.518,8.298)--(11.625,8.373)--(11.637,8.381);
\gpcolor{color=gp lt color border}
\node[gp node left] at (9.927,1.627) {$N^2$ };
\gpcolor{color=gp lt color 3}
\draw[gp path] (10.663,1.627)--(11.579,1.627);
\draw[gp path] (2.574,0.985)--(2.608,1.033)--(2.715,1.183)--(2.823,1.332)--(2.930,1.481)%
  --(3.037,1.631)--(3.145,1.780)--(3.252,1.930)--(3.360,2.079)--(3.467,2.228)--(3.574,2.378)%
  --(3.682,2.527)--(3.789,2.677)--(3.896,2.826)--(4.004,2.975)--(4.111,3.125)--(4.218,3.274)%
  --(4.326,3.424)--(4.433,3.573)--(4.540,3.723)--(4.648,3.872)--(4.755,4.021)--(4.862,4.171)%
  --(4.970,4.320)--(5.077,4.470)--(5.184,4.619)--(5.292,4.768)--(5.399,4.918)--(5.506,5.067)%
  --(5.614,5.217)--(5.721,5.366)--(5.828,5.516)--(5.936,5.665)--(6.043,5.814)--(6.150,5.964)%
  --(6.258,6.113)--(6.365,6.263)--(6.472,6.412)--(6.580,6.561)--(6.687,6.711)--(6.795,6.860)%
  --(6.902,7.010)--(7.009,7.159)--(7.117,7.308)--(7.224,7.458)--(7.331,7.607)--(7.439,7.757)%
  --(7.546,7.906)--(7.653,8.056)--(7.761,8.205)--(7.868,8.354)--(7.887,8.381);
\gpcolor{color=gp lt color border}
\node[gp node left] at (9.927,1.319) {$N^3$ };
\gpcolor{color=gp lt color 6}
\draw[gp path] (10.663,1.319)--(11.579,1.319);
\draw[gp path] (3.063,0.985)--(3.145,1.155)--(3.252,1.379)--(3.360,1.604)--(3.467,1.828)%
  --(3.574,2.052)--(3.682,2.276)--(3.789,2.500)--(3.896,2.724)--(4.004,2.948)--(4.111,3.172)%
  --(4.218,3.396)--(4.326,3.621)--(4.433,3.845)--(4.540,4.069)--(4.648,4.293)--(4.755,4.517)%
  --(4.862,4.741)--(4.970,4.965)--(5.077,5.189)--(5.184,5.414)--(5.292,5.638)--(5.399,5.862)%
  --(5.506,6.086)--(5.614,6.310)--(5.721,6.534)--(5.828,6.758)--(5.936,6.982)--(6.043,7.207)%
  --(6.150,7.431)--(6.258,7.655)--(6.365,7.879)--(6.472,8.103)--(6.580,8.327)--(6.606,8.381);
\gpcolor{color=gp lt color border}
\gpsetlinetype{gp lt border}
\gpsetlinewidth{1.00}
\draw[gp path] (1.320,8.381)--(1.320,0.985)--(11.947,0.985)--(11.947,8.381)--cycle;
%% coordinates of the plot area
\gpdefrectangularnode{gp plot 1}{\pgfpoint{1.320cm}{0.985cm}}{\pgfpoint{11.947cm}{8.381cm}}
\end{tikzpicture}
%% gnuplot variables

      \end{myplot}

      \begin{myplot}%
        {Зависимость времени, потраченного на объединение множеств, от размера
        метода}%
        {plot:simple_ops}
        \begin{tikzpicture}[gnuplot]
%% generated with GNUPLOT 4.5p0 (Lua 5.1; terminal rev. 99, script rev. 98)
%% 27.05.2011 12:41:08
\path (0.000,0.000) rectangle (12.500,8.750);
\gpcolor{color=gp lt color border}
\gpsetlinetype{gp lt border}
\gpsetlinewidth{1.00}
\draw[gp path] (1.320,0.985)--(1.500,0.985);
\draw[gp path] (11.947,0.985)--(11.767,0.985);
\node[gp node right] at (1.136,0.985) {$10^{0}$};
\draw[gp path] (1.320,1.439)--(1.410,1.439);
\draw[gp path] (11.947,1.439)--(11.857,1.439);
\draw[gp path] (1.320,2.039)--(1.410,2.039);
\draw[gp path] (11.947,2.039)--(11.857,2.039);
\draw[gp path] (1.320,2.347)--(1.410,2.347);
\draw[gp path] (11.947,2.347)--(11.857,2.347);
\draw[gp path] (1.320,2.493)--(1.500,2.493);
\draw[gp path] (11.947,2.493)--(11.767,2.493);
\node[gp node right] at (1.136,2.493) {$10^{1}$};
\draw[gp path] (1.320,2.948)--(1.410,2.948);
\draw[gp path] (11.947,2.948)--(11.857,2.948);
\draw[gp path] (1.320,3.548)--(1.410,3.548);
\draw[gp path] (11.947,3.548)--(11.857,3.548);
\draw[gp path] (1.320,3.856)--(1.410,3.856);
\draw[gp path] (11.947,3.856)--(11.857,3.856);
\draw[gp path] (1.320,4.002)--(1.500,4.002);
\draw[gp path] (11.947,4.002)--(11.767,4.002);
\node[gp node right] at (1.136,4.002) {$10^{2}$};
\draw[gp path] (1.320,4.456)--(1.410,4.456);
\draw[gp path] (11.947,4.456)--(11.857,4.456);
\draw[gp path] (1.320,5.056)--(1.410,5.056);
\draw[gp path] (11.947,5.056)--(11.857,5.056);
\draw[gp path] (1.320,5.364)--(1.410,5.364);
\draw[gp path] (11.947,5.364)--(11.857,5.364);
\draw[gp path] (1.320,5.510)--(1.500,5.510);
\draw[gp path] (11.947,5.510)--(11.767,5.510);
\node[gp node right] at (1.136,5.510) {$10^{3}$};
\draw[gp path] (1.320,5.964)--(1.410,5.964);
\draw[gp path] (11.947,5.964)--(11.857,5.964);
\draw[gp path] (1.320,6.565)--(1.410,6.565);
\draw[gp path] (11.947,6.565)--(11.857,6.565);
\draw[gp path] (1.320,6.873)--(1.410,6.873);
\draw[gp path] (11.947,6.873)--(11.857,6.873);
\draw[gp path] (1.320,7.019)--(1.500,7.019);
\draw[gp path] (11.947,7.019)--(11.767,7.019);
\node[gp node right] at (1.136,7.019) {$10^{4}$};
\draw[gp path] (1.320,7.473)--(1.410,7.473);
\draw[gp path] (11.947,7.473)--(11.857,7.473);
\draw[gp path] (1.320,8.073)--(1.410,8.073);
\draw[gp path] (11.947,8.073)--(11.857,8.073);
\draw[gp path] (1.320,8.381)--(1.410,8.381);
\draw[gp path] (11.947,8.381)--(11.857,8.381);
\draw[gp path] (1.320,0.985)--(1.320,1.165);
\draw[gp path] (1.320,8.381)--(1.320,8.201);
\node[gp node center] at (1.320,0.677) {$10^{0}$};
\draw[gp path] (1.972,0.985)--(1.972,1.075);
\draw[gp path] (1.972,8.381)--(1.972,8.291);
\draw[gp path] (2.835,0.985)--(2.835,1.075);
\draw[gp path] (2.835,8.381)--(2.835,8.291);
\draw[gp path] (3.277,0.985)--(3.277,1.075);
\draw[gp path] (3.277,8.381)--(3.277,8.291);
\draw[gp path] (3.487,0.985)--(3.487,1.165);
\draw[gp path] (3.487,8.381)--(3.487,8.201);
\node[gp node center] at (3.487,0.677) {$10^{1}$};
\draw[gp path] (4.140,0.985)--(4.140,1.075);
\draw[gp path] (4.140,8.381)--(4.140,8.291);
\draw[gp path] (5.002,0.985)--(5.002,1.075);
\draw[gp path] (5.002,8.381)--(5.002,8.291);
\draw[gp path] (5.445,0.985)--(5.445,1.075);
\draw[gp path] (5.445,8.381)--(5.445,8.291);
\draw[gp path] (5.655,0.985)--(5.655,1.165);
\draw[gp path] (5.655,8.381)--(5.655,8.201);
\node[gp node center] at (5.655,0.677) {$10^{2}$};
\draw[gp path] (6.307,0.985)--(6.307,1.075);
\draw[gp path] (6.307,8.381)--(6.307,8.291);
\draw[gp path] (7.170,0.985)--(7.170,1.075);
\draw[gp path] (7.170,8.381)--(7.170,8.291);
\draw[gp path] (7.612,0.985)--(7.612,1.075);
\draw[gp path] (7.612,8.381)--(7.612,8.291);
\draw[gp path] (7.822,0.985)--(7.822,1.165);
\draw[gp path] (7.822,8.381)--(7.822,8.201);
\node[gp node center] at (7.822,0.677) {$10^{3}$};
\draw[gp path] (8.475,0.985)--(8.475,1.075);
\draw[gp path] (8.475,8.381)--(8.475,8.291);
\draw[gp path] (9.337,0.985)--(9.337,1.075);
\draw[gp path] (9.337,8.381)--(9.337,8.291);
\draw[gp path] (9.780,0.985)--(9.780,1.075);
\draw[gp path] (9.780,8.381)--(9.780,8.291);
\draw[gp path] (9.990,0.985)--(9.990,1.165);
\draw[gp path] (9.990,8.381)--(9.990,8.201);
\node[gp node center] at (9.990,0.677) {$10^{4}$};
\draw[gp path] (10.642,0.985)--(10.642,1.075);
\draw[gp path] (10.642,8.381)--(10.642,8.291);
\draw[gp path] (11.505,0.985)--(11.505,1.075);
\draw[gp path] (11.505,8.381)--(11.505,8.291);
\draw[gp path] (11.947,0.985)--(11.947,1.075);
\draw[gp path] (11.947,8.381)--(11.947,8.291);
\draw[gp path] (1.320,8.381)--(1.320,0.985)--(11.947,0.985)--(11.947,8.381)--cycle;
\node[gp node center,rotate=-270] at (0.246,4.683) {Время, в условных единицах};
\node[gp node center] at (6.633,0.215) {Количество присваиваний};
\gpsetlinewidth{2.00}
\gpsetpointsize{4.00}
\gppoint{gp mark 0}{(2.835,2.878)}
\gppoint{gp mark 0}{(2.835,2.493)}
\gppoint{gp mark 0}{(2.835,2.493)}
\gppoint{gp mark 0}{(2.835,2.878)}
\gppoint{gp mark 0}{(2.835,2.665)}
\gppoint{gp mark 0}{(2.835,2.493)}
\gppoint{gp mark 0}{(2.835,2.801)}
\gppoint{gp mark 0}{(2.835,2.493)}
\gppoint{gp mark 0}{(2.835,2.801)}
\gppoint{gp mark 0}{(2.835,2.665)}
\gppoint{gp mark 0}{(2.835,2.665)}
\gppoint{gp mark 0}{(2.835,2.801)}
\gppoint{gp mark 0}{(2.835,2.801)}
\gppoint{gp mark 0}{(2.835,2.759)}
\gppoint{gp mark 0}{(2.835,2.759)}
\gppoint{gp mark 0}{(2.835,2.878)}
\gppoint{gp mark 0}{(2.835,2.878)}
\gppoint{gp mark 0}{(2.835,2.878)}
\gppoint{gp mark 0}{(2.835,2.493)}
\gppoint{gp mark 0}{(2.835,2.801)}
\gppoint{gp mark 0}{(2.835,2.493)}
\gppoint{gp mark 0}{(2.835,2.801)}
\gppoint{gp mark 0}{(2.835,2.493)}
\gppoint{gp mark 0}{(2.835,2.493)}
\gppoint{gp mark 0}{(2.835,2.878)}
\gppoint{gp mark 0}{(2.835,2.493)}
\gppoint{gp mark 0}{(2.835,2.493)}
\gppoint{gp mark 0}{(3.007,2.979)}
\gppoint{gp mark 0}{(3.007,2.979)}
\gppoint{gp mark 0}{(3.007,2.759)}
\gppoint{gp mark 0}{(3.007,3.144)}
\gppoint{gp mark 0}{(3.007,2.979)}
\gppoint{gp mark 0}{(3.007,2.979)}
\gppoint{gp mark 0}{(3.007,2.493)}
\gppoint{gp mark 0}{(3.007,2.665)}
\gppoint{gp mark 0}{(3.007,2.801)}
\gppoint{gp mark 0}{(3.007,2.979)}
\gppoint{gp mark 0}{(3.007,2.759)}
\gppoint{gp mark 0}{(3.007,2.979)}
\gppoint{gp mark 0}{(3.007,2.801)}
\gppoint{gp mark 0}{(3.007,2.759)}
\gppoint{gp mark 0}{(3.007,2.801)}
\gppoint{gp mark 0}{(3.007,2.665)}
\gppoint{gp mark 0}{(3.007,2.759)}
\gppoint{gp mark 0}{(3.007,2.665)}
\gppoint{gp mark 0}{(3.007,2.979)}
\gppoint{gp mark 0}{(3.007,2.759)}
\gppoint{gp mark 0}{(3.007,2.801)}
\gppoint{gp mark 0}{(3.007,2.665)}
\gppoint{gp mark 0}{(3.007,2.801)}
\gppoint{gp mark 0}{(3.007,2.979)}
\gppoint{gp mark 0}{(3.007,2.665)}
\gppoint{gp mark 0}{(3.007,2.801)}
\gppoint{gp mark 0}{(3.007,2.665)}
\gppoint{gp mark 0}{(3.007,2.801)}
\gppoint{gp mark 0}{(3.007,2.424)}
\gppoint{gp mark 0}{(3.152,2.801)}
\gppoint{gp mark 0}{(3.152,2.665)}
\gppoint{gp mark 0}{(3.152,3.067)}
\gppoint{gp mark 0}{(3.152,2.801)}
\gppoint{gp mark 0}{(3.152,2.801)}
\gppoint{gp mark 0}{(3.152,2.665)}
\gppoint{gp mark 0}{(3.152,2.878)}
\gppoint{gp mark 0}{(3.152,2.665)}
\gppoint{gp mark 0}{(3.152,3.067)}
\gppoint{gp mark 0}{(3.152,3.119)}
\gppoint{gp mark 0}{(3.152,2.914)}
\gppoint{gp mark 0}{(3.152,2.665)}
\gppoint{gp mark 0}{(3.152,2.665)}
\gppoint{gp mark 0}{(3.152,2.801)}
\gppoint{gp mark 0}{(3.152,2.801)}
\gppoint{gp mark 0}{(3.152,2.665)}
\gppoint{gp mark 0}{(3.152,2.801)}
\gppoint{gp mark 0}{(3.152,2.801)}
\gppoint{gp mark 0}{(3.152,2.665)}
\gppoint{gp mark 0}{(3.152,3.067)}
\gppoint{gp mark 0}{(3.152,3.010)}
\gppoint{gp mark 0}{(3.152,2.914)}
\gppoint{gp mark 0}{(3.152,3.010)}
\gppoint{gp mark 0}{(3.152,2.914)}
\gppoint{gp mark 0}{(3.152,2.914)}
\gppoint{gp mark 0}{(3.152,2.665)}
\gppoint{gp mark 0}{(3.152,2.665)}
\gppoint{gp mark 0}{(3.152,3.351)}
\gppoint{gp mark 0}{(3.152,2.665)}
\gppoint{gp mark 0}{(3.152,2.801)}
\gppoint{gp mark 0}{(3.152,2.665)}
\gppoint{gp mark 0}{(3.152,3.276)}
\gppoint{gp mark 0}{(3.152,2.665)}
\gppoint{gp mark 0}{(3.152,2.665)}
\gppoint{gp mark 0}{(3.152,3.010)}
\gppoint{gp mark 0}{(3.152,3.276)}
\gppoint{gp mark 0}{(3.152,2.665)}
\gppoint{gp mark 0}{(3.152,2.665)}
\gppoint{gp mark 0}{(3.152,2.801)}
\gppoint{gp mark 0}{(3.152,2.665)}
\gppoint{gp mark 0}{(3.152,3.119)}
\gppoint{gp mark 0}{(3.152,3.213)}
\gppoint{gp mark 0}{(3.152,3.276)}
\gppoint{gp mark 0}{(3.152,2.665)}
\gppoint{gp mark 0}{(3.152,3.255)}
\gppoint{gp mark 0}{(3.152,3.402)}
\gppoint{gp mark 0}{(3.152,2.665)}
\gppoint{gp mark 0}{(3.152,2.665)}
\gppoint{gp mark 0}{(3.152,3.213)}
\gppoint{gp mark 0}{(3.152,3.010)}
\gppoint{gp mark 0}{(3.152,3.255)}
\gppoint{gp mark 0}{(3.152,3.235)}
\gppoint{gp mark 0}{(3.152,2.665)}
\gppoint{gp mark 0}{(3.152,2.665)}
\gppoint{gp mark 0}{(3.152,2.665)}
\gppoint{gp mark 0}{(3.152,2.801)}
\gppoint{gp mark 0}{(3.152,2.665)}
\gppoint{gp mark 0}{(3.152,3.276)}
\gppoint{gp mark 0}{(3.152,3.255)}
\gppoint{gp mark 0}{(3.152,2.665)}
\gppoint{gp mark 0}{(3.152,3.479)}
\gppoint{gp mark 0}{(3.152,3.255)}
\gppoint{gp mark 0}{(3.152,2.665)}
\gppoint{gp mark 0}{(3.152,2.665)}
\gppoint{gp mark 0}{(3.152,2.665)}
\gppoint{gp mark 0}{(3.152,2.914)}
\gppoint{gp mark 0}{(3.152,2.493)}
\gppoint{gp mark 0}{(3.152,3.010)}
\gppoint{gp mark 0}{(3.152,3.276)}
\gppoint{gp mark 0}{(3.152,2.665)}
\gppoint{gp mark 0}{(3.152,2.665)}
\gppoint{gp mark 0}{(3.152,2.801)}
\gppoint{gp mark 0}{(3.152,2.801)}
\gppoint{gp mark 0}{(3.152,2.665)}
\gppoint{gp mark 0}{(3.152,2.914)}
\gppoint{gp mark 0}{(3.152,2.665)}
\gppoint{gp mark 0}{(3.152,2.914)}
\gppoint{gp mark 0}{(3.152,2.665)}
\gppoint{gp mark 0}{(3.152,3.067)}
\gppoint{gp mark 0}{(3.152,2.665)}
\gppoint{gp mark 0}{(3.152,2.914)}
\gppoint{gp mark 0}{(3.152,2.665)}
\gppoint{gp mark 0}{(3.152,2.613)}
\gppoint{gp mark 0}{(3.152,2.801)}
\gppoint{gp mark 0}{(3.152,3.119)}
\gppoint{gp mark 0}{(3.152,3.295)}
\gppoint{gp mark 0}{(3.152,2.665)}
\gppoint{gp mark 0}{(3.152,3.213)}
\gppoint{gp mark 0}{(3.152,2.801)}
\gppoint{gp mark 0}{(3.152,2.665)}
\gppoint{gp mark 0}{(3.152,2.665)}
\gppoint{gp mark 0}{(3.152,2.841)}
\gppoint{gp mark 0}{(3.152,3.119)}
\gppoint{gp mark 0}{(3.152,3.119)}
\gppoint{gp mark 0}{(3.152,2.914)}
\gppoint{gp mark 0}{(3.152,3.119)}
\gppoint{gp mark 0}{(3.152,3.479)}
\gppoint{gp mark 0}{(3.152,3.235)}
\gppoint{gp mark 0}{(3.152,2.914)}
\gppoint{gp mark 0}{(3.152,3.235)}
\gppoint{gp mark 0}{(3.152,3.255)}
\gppoint{gp mark 0}{(3.152,2.979)}
\gppoint{gp mark 0}{(3.152,2.914)}
\gppoint{gp mark 0}{(3.152,2.665)}
\gppoint{gp mark 0}{(3.152,2.878)}
\gppoint{gp mark 0}{(3.152,3.235)}
\gppoint{gp mark 0}{(3.152,2.665)}
\gppoint{gp mark 0}{(3.152,3.235)}
\gppoint{gp mark 0}{(3.152,3.255)}
\gppoint{gp mark 0}{(3.152,2.665)}
\gppoint{gp mark 0}{(3.152,3.010)}
\gppoint{gp mark 0}{(3.152,3.119)}
\gppoint{gp mark 0}{(3.152,2.665)}
\gppoint{gp mark 0}{(3.152,3.213)}
\gppoint{gp mark 0}{(3.152,2.665)}
\gppoint{gp mark 0}{(3.152,2.665)}
\gppoint{gp mark 0}{(3.152,3.119)}
\gppoint{gp mark 0}{(3.152,3.119)}
\gppoint{gp mark 0}{(3.152,3.295)}
\gppoint{gp mark 0}{(3.152,3.010)}
\gppoint{gp mark 0}{(3.152,2.878)}
\gppoint{gp mark 0}{(3.152,2.665)}
\gppoint{gp mark 0}{(3.152,2.665)}
\gppoint{gp mark 0}{(3.152,2.801)}
\gppoint{gp mark 0}{(3.152,3.119)}
\gppoint{gp mark 0}{(3.152,3.119)}
\gppoint{gp mark 0}{(3.152,2.665)}
\gppoint{gp mark 0}{(3.152,3.010)}
\gppoint{gp mark 0}{(3.152,3.119)}
\gppoint{gp mark 0}{(3.152,3.119)}
\gppoint{gp mark 0}{(3.152,3.119)}
\gppoint{gp mark 0}{(3.152,2.841)}
\gppoint{gp mark 0}{(3.152,3.144)}
\gppoint{gp mark 0}{(3.152,2.665)}
\gppoint{gp mark 0}{(3.152,3.144)}
\gppoint{gp mark 0}{(3.152,3.119)}
\gppoint{gp mark 0}{(3.152,2.914)}
\gppoint{gp mark 0}{(3.152,2.801)}
\gppoint{gp mark 0}{(3.152,2.665)}
\gppoint{gp mark 0}{(3.152,2.665)}
\gppoint{gp mark 0}{(3.152,2.801)}
\gppoint{gp mark 0}{(3.152,3.010)}
\gppoint{gp mark 0}{(3.152,3.144)}
\gppoint{gp mark 0}{(3.152,3.144)}
\gppoint{gp mark 0}{(3.152,3.119)}
\gppoint{gp mark 0}{(3.152,2.801)}
\gppoint{gp mark 0}{(3.152,3.119)}
\gppoint{gp mark 0}{(3.152,3.168)}
\gppoint{gp mark 0}{(3.152,2.801)}
\gppoint{gp mark 0}{(3.152,2.801)}
\gppoint{gp mark 0}{(3.152,3.119)}
\gppoint{gp mark 0}{(3.152,3.119)}
\gppoint{gp mark 0}{(3.152,3.119)}
\gppoint{gp mark 0}{(3.152,3.119)}
\gppoint{gp mark 0}{(3.152,3.119)}
\gppoint{gp mark 0}{(3.152,3.119)}
\gppoint{gp mark 0}{(3.152,3.119)}
\gppoint{gp mark 0}{(3.152,3.255)}
\gppoint{gp mark 0}{(3.152,3.119)}
\gppoint{gp mark 0}{(3.152,3.119)}
\gppoint{gp mark 0}{(3.152,3.119)}
\gppoint{gp mark 0}{(3.152,3.119)}
\gppoint{gp mark 0}{(3.152,3.119)}
\gppoint{gp mark 0}{(3.152,3.119)}
\gppoint{gp mark 0}{(3.152,3.119)}
\gppoint{gp mark 0}{(3.152,3.119)}
\gppoint{gp mark 0}{(3.152,3.119)}
\gppoint{gp mark 0}{(3.152,3.119)}
\gppoint{gp mark 0}{(3.152,3.119)}
\gppoint{gp mark 0}{(3.152,3.119)}
\gppoint{gp mark 0}{(3.152,3.119)}
\gppoint{gp mark 0}{(3.152,3.119)}
\gppoint{gp mark 0}{(3.152,3.119)}
\gppoint{gp mark 0}{(3.152,3.119)}
\gppoint{gp mark 0}{(3.152,3.119)}
\gppoint{gp mark 0}{(3.152,3.119)}
\gppoint{gp mark 0}{(3.152,3.119)}
\gppoint{gp mark 0}{(3.152,3.119)}
\gppoint{gp mark 0}{(3.152,3.119)}
\gppoint{gp mark 0}{(3.152,3.144)}
\gppoint{gp mark 0}{(3.152,3.119)}
\gppoint{gp mark 0}{(3.152,3.119)}
\gppoint{gp mark 0}{(3.152,3.119)}
\gppoint{gp mark 0}{(3.152,2.665)}
\gppoint{gp mark 0}{(3.152,3.119)}
\gppoint{gp mark 0}{(3.152,3.119)}
\gppoint{gp mark 0}{(3.152,2.801)}
\gppoint{gp mark 0}{(3.152,3.119)}
\gppoint{gp mark 0}{(3.152,3.119)}
\gppoint{gp mark 0}{(3.152,3.119)}
\gppoint{gp mark 0}{(3.152,3.119)}
\gppoint{gp mark 0}{(3.152,3.119)}
\gppoint{gp mark 0}{(3.152,3.119)}
\gppoint{gp mark 0}{(3.152,3.119)}
\gppoint{gp mark 0}{(3.152,3.119)}
\gppoint{gp mark 0}{(3.152,3.119)}
\gppoint{gp mark 0}{(3.152,3.119)}
\gppoint{gp mark 0}{(3.152,3.119)}
\gppoint{gp mark 0}{(3.152,3.119)}
\gppoint{gp mark 0}{(3.152,2.801)}
\gppoint{gp mark 0}{(3.152,2.801)}
\gppoint{gp mark 0}{(3.152,3.119)}
\gppoint{gp mark 0}{(3.152,2.665)}
\gppoint{gp mark 0}{(3.152,3.119)}
\gppoint{gp mark 0}{(3.152,3.119)}
\gppoint{gp mark 0}{(3.152,3.119)}
\gppoint{gp mark 0}{(3.152,2.801)}
\gppoint{gp mark 0}{(3.152,2.979)}
\gppoint{gp mark 0}{(3.152,3.119)}
\gppoint{gp mark 0}{(3.152,3.119)}
\gppoint{gp mark 0}{(3.152,3.119)}
\gppoint{gp mark 0}{(3.152,2.801)}
\gppoint{gp mark 0}{(3.152,3.119)}
\gppoint{gp mark 0}{(3.152,2.665)}
\gppoint{gp mark 0}{(3.152,3.119)}
\gppoint{gp mark 0}{(3.152,3.119)}
\gppoint{gp mark 0}{(3.152,3.119)}
\gppoint{gp mark 0}{(3.152,3.119)}
\gppoint{gp mark 0}{(3.152,3.119)}
\gppoint{gp mark 0}{(3.152,3.119)}
\gppoint{gp mark 0}{(3.152,2.665)}
\gppoint{gp mark 0}{(3.152,2.665)}
\gppoint{gp mark 0}{(3.152,3.119)}
\gppoint{gp mark 0}{(3.152,3.314)}
\gppoint{gp mark 0}{(3.152,3.119)}
\gppoint{gp mark 0}{(3.152,2.665)}
\gppoint{gp mark 0}{(3.152,2.665)}
\gppoint{gp mark 0}{(3.152,3.119)}
\gppoint{gp mark 0}{(3.152,3.119)}
\gppoint{gp mark 0}{(3.152,3.119)}
\gppoint{gp mark 0}{(3.152,3.119)}
\gppoint{gp mark 0}{(3.152,3.119)}
\gppoint{gp mark 0}{(3.152,3.119)}
\gppoint{gp mark 0}{(3.152,3.119)}
\gppoint{gp mark 0}{(3.152,3.119)}
\gppoint{gp mark 0}{(3.152,3.119)}
\gppoint{gp mark 0}{(3.152,2.665)}
\gppoint{gp mark 0}{(3.152,3.119)}
\gppoint{gp mark 0}{(3.152,3.119)}
\gppoint{gp mark 0}{(3.152,2.665)}
\gppoint{gp mark 0}{(3.152,3.119)}
\gppoint{gp mark 0}{(3.152,3.119)}
\gppoint{gp mark 0}{(3.152,3.119)}
\gppoint{gp mark 0}{(3.152,3.119)}
\gppoint{gp mark 0}{(3.152,2.801)}
\gppoint{gp mark 0}{(3.152,3.119)}
\gppoint{gp mark 0}{(3.152,3.119)}
\gppoint{gp mark 0}{(3.152,3.119)}
\gppoint{gp mark 0}{(3.152,3.119)}
\gppoint{gp mark 0}{(3.152,3.119)}
\gppoint{gp mark 0}{(3.152,2.914)}
\gppoint{gp mark 0}{(3.152,3.119)}
\gppoint{gp mark 0}{(3.152,3.119)}
\gppoint{gp mark 0}{(3.152,3.119)}
\gppoint{gp mark 0}{(3.152,3.119)}
\gppoint{gp mark 0}{(3.152,2.801)}
\gppoint{gp mark 0}{(3.152,3.119)}
\gppoint{gp mark 0}{(3.152,3.119)}
\gppoint{gp mark 0}{(3.152,3.119)}
\gppoint{gp mark 0}{(3.152,3.119)}
\gppoint{gp mark 0}{(3.152,3.119)}
\gppoint{gp mark 0}{(3.152,3.119)}
\gppoint{gp mark 0}{(3.152,3.119)}
\gppoint{gp mark 0}{(3.152,3.119)}
\gppoint{gp mark 0}{(3.152,3.119)}
\gppoint{gp mark 0}{(3.152,3.119)}
\gppoint{gp mark 0}{(3.152,3.119)}
\gppoint{gp mark 0}{(3.152,3.119)}
\gppoint{gp mark 0}{(3.152,3.119)}
\gppoint{gp mark 0}{(3.152,2.665)}
\gppoint{gp mark 0}{(3.277,3.010)}
\gppoint{gp mark 0}{(3.277,3.213)}
\gppoint{gp mark 0}{(3.277,3.434)}
\gppoint{gp mark 0}{(3.277,2.914)}
\gppoint{gp mark 0}{(3.277,3.479)}
\gppoint{gp mark 0}{(3.277,3.434)}
\gppoint{gp mark 0}{(3.277,3.402)}
\gppoint{gp mark 0}{(3.277,2.801)}
\gppoint{gp mark 0}{(3.277,3.434)}
\gppoint{gp mark 0}{(3.277,2.801)}
\gppoint{gp mark 0}{(3.277,2.914)}
\gppoint{gp mark 0}{(3.277,2.878)}
\gppoint{gp mark 0}{(3.277,2.914)}
\gppoint{gp mark 0}{(3.277,3.213)}
\gppoint{gp mark 0}{(3.277,2.801)}
\gppoint{gp mark 0}{(3.277,3.144)}
\gppoint{gp mark 0}{(3.277,3.521)}
\gppoint{gp mark 0}{(3.277,2.801)}
\gppoint{gp mark 0}{(3.277,3.010)}
\gppoint{gp mark 0}{(3.277,2.914)}
\gppoint{gp mark 0}{(3.277,2.914)}
\gppoint{gp mark 0}{(3.277,2.801)}
\gppoint{gp mark 0}{(3.277,3.010)}
\gppoint{gp mark 0}{(3.277,2.801)}
\gppoint{gp mark 0}{(3.277,3.333)}
\gppoint{gp mark 0}{(3.277,3.255)}
\gppoint{gp mark 0}{(3.277,3.094)}
\gppoint{gp mark 0}{(3.277,2.914)}
\gppoint{gp mark 0}{(3.277,2.878)}
\gppoint{gp mark 0}{(3.277,3.314)}
\gppoint{gp mark 0}{(3.277,3.493)}
\gppoint{gp mark 0}{(3.277,2.979)}
\gppoint{gp mark 0}{(3.277,3.314)}
\gppoint{gp mark 0}{(3.277,3.368)}
\gppoint{gp mark 0}{(3.277,2.801)}
\gppoint{gp mark 0}{(3.277,3.213)}
\gppoint{gp mark 0}{(3.277,3.521)}
\gppoint{gp mark 0}{(3.277,3.213)}
\gppoint{gp mark 0}{(3.277,3.418)}
\gppoint{gp mark 0}{(3.277,3.067)}
\gppoint{gp mark 0}{(3.277,3.067)}
\gppoint{gp mark 0}{(3.277,3.144)}
\gppoint{gp mark 0}{(3.277,2.801)}
\gppoint{gp mark 0}{(3.277,3.010)}
\gppoint{gp mark 0}{(3.277,2.801)}
\gppoint{gp mark 0}{(3.277,3.010)}
\gppoint{gp mark 0}{(3.277,2.914)}
\gppoint{gp mark 0}{(3.277,3.039)}
\gppoint{gp mark 0}{(3.277,2.979)}
\gppoint{gp mark 0}{(3.277,2.801)}
\gppoint{gp mark 0}{(3.277,3.402)}
\gppoint{gp mark 0}{(3.277,3.434)}
\gppoint{gp mark 0}{(3.277,3.213)}
\gppoint{gp mark 0}{(3.277,3.213)}
\gppoint{gp mark 0}{(3.277,3.144)}
\gppoint{gp mark 0}{(3.277,2.914)}
\gppoint{gp mark 0}{(3.277,2.914)}
\gppoint{gp mark 0}{(3.277,2.801)}
\gppoint{gp mark 0}{(3.277,3.039)}
\gppoint{gp mark 0}{(3.277,2.801)}
\gppoint{gp mark 0}{(3.277,2.801)}
\gppoint{gp mark 0}{(3.277,3.434)}
\gppoint{gp mark 0}{(3.277,3.434)}
\gppoint{gp mark 0}{(3.277,3.434)}
\gppoint{gp mark 0}{(3.277,3.010)}
\gppoint{gp mark 0}{(3.277,3.434)}
\gppoint{gp mark 0}{(3.277,3.434)}
\gppoint{gp mark 0}{(3.277,2.801)}
\gppoint{gp mark 0}{(3.277,2.801)}
\gppoint{gp mark 0}{(3.277,3.449)}
\gppoint{gp mark 0}{(3.277,3.333)}
\gppoint{gp mark 0}{(3.277,2.914)}
\gppoint{gp mark 0}{(3.277,3.144)}
\gppoint{gp mark 0}{(3.277,2.665)}
\gppoint{gp mark 0}{(3.277,2.979)}
\gppoint{gp mark 0}{(3.277,2.979)}
\gppoint{gp mark 0}{(3.277,2.665)}
\gppoint{gp mark 0}{(3.277,3.333)}
\gppoint{gp mark 0}{(3.277,3.213)}
\gppoint{gp mark 0}{(3.277,2.665)}
\gppoint{gp mark 0}{(3.277,2.801)}
\gppoint{gp mark 0}{(3.277,3.067)}
\gppoint{gp mark 0}{(3.277,3.434)}
\gppoint{gp mark 0}{(3.277,3.010)}
\gppoint{gp mark 0}{(3.277,3.010)}
\gppoint{gp mark 0}{(3.277,2.878)}
\gppoint{gp mark 0}{(3.277,2.914)}
\gppoint{gp mark 0}{(3.277,3.067)}
\gppoint{gp mark 0}{(3.277,2.914)}
\gppoint{gp mark 0}{(3.277,3.434)}
\gppoint{gp mark 0}{(3.277,3.295)}
\gppoint{gp mark 0}{(3.277,3.213)}
\gppoint{gp mark 0}{(3.277,3.402)}
\gppoint{gp mark 0}{(3.277,3.295)}
\gppoint{gp mark 0}{(3.277,2.914)}
\gppoint{gp mark 0}{(3.277,2.948)}
\gppoint{gp mark 0}{(3.277,3.402)}
\gppoint{gp mark 0}{(3.277,2.801)}
\gppoint{gp mark 0}{(3.277,2.914)}
\gppoint{gp mark 0}{(3.277,2.665)}
\gppoint{gp mark 0}{(3.277,3.295)}
\gppoint{gp mark 0}{(3.277,3.449)}
\gppoint{gp mark 0}{(3.277,3.276)}
\gppoint{gp mark 0}{(3.277,3.067)}
\gppoint{gp mark 0}{(3.277,2.801)}
\gppoint{gp mark 0}{(3.277,3.213)}
\gppoint{gp mark 0}{(3.277,3.535)}
\gppoint{gp mark 0}{(3.277,2.914)}
\gppoint{gp mark 0}{(3.277,3.119)}
\gppoint{gp mark 0}{(3.277,2.914)}
\gppoint{gp mark 0}{(3.277,3.168)}
\gppoint{gp mark 0}{(3.277,2.979)}
\gppoint{gp mark 0}{(3.277,3.402)}
\gppoint{gp mark 0}{(3.277,2.914)}
\gppoint{gp mark 0}{(3.277,3.402)}
\gppoint{gp mark 0}{(3.277,2.801)}
\gppoint{gp mark 0}{(3.277,3.168)}
\gppoint{gp mark 0}{(3.277,3.333)}
\gppoint{gp mark 0}{(3.277,3.067)}
\gppoint{gp mark 0}{(3.277,2.801)}
\gppoint{gp mark 0}{(3.277,2.914)}
\gppoint{gp mark 0}{(3.277,3.333)}
\gppoint{gp mark 0}{(3.277,2.878)}
\gppoint{gp mark 0}{(3.277,3.191)}
\gppoint{gp mark 0}{(3.277,3.010)}
\gppoint{gp mark 0}{(3.277,3.368)}
\gppoint{gp mark 0}{(3.277,3.039)}
\gppoint{gp mark 0}{(3.277,3.119)}
\gppoint{gp mark 0}{(3.277,3.119)}
\gppoint{gp mark 0}{(3.277,3.119)}
\gppoint{gp mark 0}{(3.277,2.979)}
\gppoint{gp mark 0}{(3.277,3.039)}
\gppoint{gp mark 0}{(3.277,2.801)}
\gppoint{gp mark 0}{(3.277,3.385)}
\gppoint{gp mark 0}{(3.277,2.759)}
\gppoint{gp mark 0}{(3.277,2.914)}
\gppoint{gp mark 0}{(3.277,2.801)}
\gppoint{gp mark 0}{(3.277,2.801)}
\gppoint{gp mark 0}{(3.277,2.801)}
\gppoint{gp mark 0}{(3.277,3.144)}
\gppoint{gp mark 0}{(3.277,2.914)}
\gppoint{gp mark 0}{(3.277,2.914)}
\gppoint{gp mark 0}{(3.277,3.385)}
\gppoint{gp mark 0}{(3.277,3.168)}
\gppoint{gp mark 0}{(3.277,2.979)}
\gppoint{gp mark 0}{(3.277,3.276)}
\gppoint{gp mark 0}{(3.277,3.039)}
\gppoint{gp mark 0}{(3.277,2.914)}
\gppoint{gp mark 0}{(3.277,2.979)}
\gppoint{gp mark 0}{(3.277,2.801)}
\gppoint{gp mark 0}{(3.277,2.914)}
\gppoint{gp mark 0}{(3.277,2.914)}
\gppoint{gp mark 0}{(3.277,2.914)}
\gppoint{gp mark 0}{(3.277,3.276)}
\gppoint{gp mark 0}{(3.277,3.010)}
\gppoint{gp mark 0}{(3.277,2.914)}
\gppoint{gp mark 0}{(3.277,3.314)}
\gppoint{gp mark 0}{(3.277,3.276)}
\gppoint{gp mark 0}{(3.277,2.801)}
\gppoint{gp mark 0}{(3.277,2.914)}
\gppoint{gp mark 0}{(3.277,2.914)}
\gppoint{gp mark 0}{(3.277,2.948)}
\gppoint{gp mark 0}{(3.277,2.914)}
\gppoint{gp mark 0}{(3.277,2.914)}
\gppoint{gp mark 0}{(3.277,3.144)}
\gppoint{gp mark 0}{(3.277,3.144)}
\gppoint{gp mark 0}{(3.277,3.314)}
\gppoint{gp mark 0}{(3.277,3.314)}
\gppoint{gp mark 0}{(3.277,2.914)}
\gppoint{gp mark 0}{(3.277,3.067)}
\gppoint{gp mark 0}{(3.277,2.914)}
\gppoint{gp mark 0}{(3.277,3.213)}
\gppoint{gp mark 0}{(3.277,2.801)}
\gppoint{gp mark 0}{(3.277,3.067)}
\gppoint{gp mark 0}{(3.277,2.914)}
\gppoint{gp mark 0}{(3.277,3.351)}
\gppoint{gp mark 0}{(3.277,3.255)}
\gppoint{gp mark 0}{(3.277,3.255)}
\gppoint{gp mark 0}{(3.277,3.255)}
\gppoint{gp mark 0}{(3.277,3.039)}
\gppoint{gp mark 0}{(3.277,3.295)}
\gppoint{gp mark 0}{(3.277,2.801)}
\gppoint{gp mark 0}{(3.277,3.010)}
\gppoint{gp mark 0}{(3.277,2.914)}
\gppoint{gp mark 0}{(3.277,3.067)}
\gppoint{gp mark 0}{(3.277,2.801)}
\gppoint{gp mark 0}{(3.277,3.144)}
\gppoint{gp mark 0}{(3.277,2.914)}
\gppoint{gp mark 0}{(3.277,3.144)}
\gppoint{gp mark 0}{(3.277,3.351)}
\gppoint{gp mark 0}{(3.277,3.434)}
\gppoint{gp mark 0}{(3.277,3.144)}
\gppoint{gp mark 0}{(3.277,2.979)}
\gppoint{gp mark 0}{(3.277,3.333)}
\gppoint{gp mark 0}{(3.277,3.255)}
\gppoint{gp mark 0}{(3.277,3.368)}
\gppoint{gp mark 0}{(3.277,3.119)}
\gppoint{gp mark 0}{(3.277,3.144)}
\gppoint{gp mark 0}{(3.388,2.914)}
\gppoint{gp mark 0}{(3.388,3.010)}
\gppoint{gp mark 0}{(3.388,2.979)}
\gppoint{gp mark 0}{(3.388,3.010)}
\gppoint{gp mark 0}{(3.388,2.878)}
\gppoint{gp mark 0}{(3.388,3.213)}
\gppoint{gp mark 0}{(3.388,3.010)}
\gppoint{gp mark 0}{(3.388,3.402)}
\gppoint{gp mark 0}{(3.388,2.801)}
\gppoint{gp mark 0}{(3.388,3.402)}
\gppoint{gp mark 0}{(3.388,3.010)}
\gppoint{gp mark 0}{(3.388,3.010)}
\gppoint{gp mark 0}{(3.388,2.801)}
\gppoint{gp mark 0}{(3.388,2.801)}
\gppoint{gp mark 0}{(3.388,3.213)}
\gppoint{gp mark 0}{(3.388,3.010)}
\gppoint{gp mark 0}{(3.388,2.914)}
\gppoint{gp mark 0}{(3.388,3.067)}
\gppoint{gp mark 0}{(3.388,3.144)}
\gppoint{gp mark 0}{(3.388,3.168)}
\gppoint{gp mark 0}{(3.388,3.255)}
\gppoint{gp mark 0}{(3.388,2.801)}
\gppoint{gp mark 0}{(3.388,3.144)}
\gppoint{gp mark 0}{(3.388,3.255)}
\gppoint{gp mark 0}{(3.388,3.010)}
\gppoint{gp mark 0}{(3.388,3.010)}
\gppoint{gp mark 0}{(3.388,2.914)}
\gppoint{gp mark 0}{(3.388,3.314)}
\gppoint{gp mark 0}{(3.388,3.610)}
\gppoint{gp mark 0}{(3.388,2.914)}
\gppoint{gp mark 0}{(3.388,3.168)}
\gppoint{gp mark 0}{(3.388,2.914)}
\gppoint{gp mark 0}{(3.388,2.914)}
\gppoint{gp mark 0}{(3.388,3.010)}
\gppoint{gp mark 0}{(3.388,3.434)}
\gppoint{gp mark 0}{(3.388,2.979)}
\gppoint{gp mark 0}{(3.388,3.010)}
\gppoint{gp mark 0}{(3.388,2.801)}
\gppoint{gp mark 0}{(3.388,3.314)}
\gppoint{gp mark 0}{(3.388,3.010)}
\gppoint{gp mark 0}{(3.388,3.333)}
\gppoint{gp mark 0}{(3.388,3.067)}
\gppoint{gp mark 0}{(3.388,3.255)}
\gppoint{gp mark 0}{(3.388,2.801)}
\gppoint{gp mark 0}{(3.388,2.801)}
\gppoint{gp mark 0}{(3.388,3.333)}
\gppoint{gp mark 0}{(3.388,3.144)}
\gppoint{gp mark 0}{(3.388,3.213)}
\gppoint{gp mark 0}{(3.388,3.144)}
\gppoint{gp mark 0}{(3.388,3.010)}
\gppoint{gp mark 0}{(3.388,3.295)}
\gppoint{gp mark 0}{(3.388,3.119)}
\gppoint{gp mark 0}{(3.388,3.010)}
\gppoint{gp mark 0}{(3.388,3.144)}
\gppoint{gp mark 0}{(3.388,3.010)}
\gppoint{gp mark 0}{(3.388,3.010)}
\gppoint{gp mark 0}{(3.388,3.010)}
\gppoint{gp mark 0}{(3.388,3.144)}
\gppoint{gp mark 0}{(3.388,3.295)}
\gppoint{gp mark 0}{(3.388,3.094)}
\gppoint{gp mark 0}{(3.388,3.067)}
\gppoint{gp mark 0}{(3.388,3.213)}
\gppoint{gp mark 0}{(3.388,3.144)}
\gppoint{gp mark 0}{(3.388,3.144)}
\gppoint{gp mark 0}{(3.388,3.213)}
\gppoint{gp mark 0}{(3.388,2.914)}
\gppoint{gp mark 0}{(3.388,2.914)}
\gppoint{gp mark 0}{(3.388,3.094)}
\gppoint{gp mark 0}{(3.388,3.010)}
\gppoint{gp mark 0}{(3.388,2.665)}
\gppoint{gp mark 0}{(3.388,2.801)}
\gppoint{gp mark 0}{(3.388,3.464)}
\gppoint{gp mark 0}{(3.388,2.914)}
\gppoint{gp mark 0}{(3.388,3.144)}
\gppoint{gp mark 0}{(3.388,3.168)}
\gppoint{gp mark 0}{(3.388,2.914)}
\gppoint{gp mark 0}{(3.388,3.235)}
\gppoint{gp mark 0}{(3.388,2.914)}
\gppoint{gp mark 0}{(3.388,3.010)}
\gppoint{gp mark 0}{(3.388,2.801)}
\gppoint{gp mark 0}{(3.388,3.144)}
\gppoint{gp mark 0}{(3.388,3.402)}
\gppoint{gp mark 0}{(3.388,3.010)}
\gppoint{gp mark 0}{(3.388,3.010)}
\gppoint{gp mark 0}{(3.388,3.094)}
\gppoint{gp mark 0}{(3.388,3.521)}
\gppoint{gp mark 0}{(3.388,3.010)}
\gppoint{gp mark 0}{(3.388,3.493)}
\gppoint{gp mark 0}{(3.388,2.979)}
\gppoint{gp mark 0}{(3.388,3.094)}
\gppoint{gp mark 0}{(3.388,3.067)}
\gppoint{gp mark 0}{(3.388,3.067)}
\gppoint{gp mark 0}{(3.388,3.010)}
\gppoint{gp mark 0}{(3.388,2.801)}
\gppoint{gp mark 0}{(3.388,3.067)}
\gppoint{gp mark 0}{(3.388,2.801)}
\gppoint{gp mark 0}{(3.388,3.067)}
\gppoint{gp mark 0}{(3.388,3.144)}
\gppoint{gp mark 0}{(3.388,2.801)}
\gppoint{gp mark 0}{(3.388,3.144)}
\gppoint{gp mark 0}{(3.388,3.010)}
\gppoint{gp mark 0}{(3.388,3.010)}
\gppoint{gp mark 0}{(3.388,3.434)}
\gppoint{gp mark 0}{(3.388,3.094)}
\gppoint{gp mark 0}{(3.388,3.144)}
\gppoint{gp mark 0}{(3.388,3.094)}
\gppoint{gp mark 0}{(3.388,3.385)}
\gppoint{gp mark 0}{(3.388,3.839)}
\gppoint{gp mark 0}{(3.388,3.067)}
\gppoint{gp mark 0}{(3.388,3.434)}
\gppoint{gp mark 0}{(3.388,2.801)}
\gppoint{gp mark 0}{(3.388,2.801)}
\gppoint{gp mark 0}{(3.388,2.801)}
\gppoint{gp mark 0}{(3.388,2.801)}
\gppoint{gp mark 0}{(3.388,3.010)}
\gppoint{gp mark 0}{(3.388,3.333)}
\gppoint{gp mark 0}{(3.388,2.665)}
\gppoint{gp mark 0}{(3.388,2.665)}
\gppoint{gp mark 0}{(3.388,3.333)}
\gppoint{gp mark 0}{(3.388,3.333)}
\gppoint{gp mark 0}{(3.388,2.801)}
\gppoint{gp mark 0}{(3.388,3.094)}
\gppoint{gp mark 0}{(3.388,3.067)}
\gppoint{gp mark 0}{(3.388,2.801)}
\gppoint{gp mark 0}{(3.388,2.801)}
\gppoint{gp mark 0}{(3.388,2.801)}
\gppoint{gp mark 0}{(3.388,2.801)}
\gppoint{gp mark 0}{(3.388,2.914)}
\gppoint{gp mark 0}{(3.388,3.010)}
\gppoint{gp mark 0}{(3.388,2.801)}
\gppoint{gp mark 0}{(3.388,3.067)}
\gppoint{gp mark 0}{(3.388,3.402)}
\gppoint{gp mark 0}{(3.388,3.010)}
\gppoint{gp mark 0}{(3.388,3.010)}
\gppoint{gp mark 0}{(3.388,3.235)}
\gppoint{gp mark 0}{(3.388,2.801)}
\gppoint{gp mark 0}{(3.388,2.801)}
\gppoint{gp mark 0}{(3.388,3.010)}
\gppoint{gp mark 0}{(3.388,3.418)}
\gppoint{gp mark 0}{(3.388,2.801)}
\gppoint{gp mark 0}{(3.388,3.094)}
\gppoint{gp mark 0}{(3.388,2.801)}
\gppoint{gp mark 0}{(3.388,2.801)}
\gppoint{gp mark 0}{(3.388,3.168)}
\gppoint{gp mark 0}{(3.388,2.801)}
\gppoint{gp mark 0}{(3.388,3.507)}
\gppoint{gp mark 0}{(3.388,3.094)}
\gppoint{gp mark 0}{(3.388,2.801)}
\gppoint{gp mark 0}{(3.388,2.801)}
\gppoint{gp mark 0}{(3.388,2.801)}
\gppoint{gp mark 0}{(3.388,2.801)}
\gppoint{gp mark 0}{(3.388,2.801)}
\gppoint{gp mark 0}{(3.388,2.665)}
\gppoint{gp mark 0}{(3.388,2.801)}
\gppoint{gp mark 0}{(3.388,2.979)}
\gppoint{gp mark 0}{(3.388,2.801)}
\gppoint{gp mark 0}{(3.388,2.801)}
\gppoint{gp mark 0}{(3.388,2.801)}
\gppoint{gp mark 0}{(3.388,2.801)}
\gppoint{gp mark 0}{(3.388,3.255)}
\gppoint{gp mark 0}{(3.388,3.255)}
\gppoint{gp mark 0}{(3.388,2.801)}
\gppoint{gp mark 0}{(3.388,2.979)}
\gppoint{gp mark 0}{(3.388,3.255)}
\gppoint{gp mark 0}{(3.388,2.801)}
\gppoint{gp mark 0}{(3.388,3.144)}
\gppoint{gp mark 0}{(3.388,2.979)}
\gppoint{gp mark 0}{(3.388,2.801)}
\gppoint{gp mark 0}{(3.388,3.010)}
\gppoint{gp mark 0}{(3.388,3.314)}
\gppoint{gp mark 0}{(3.388,3.144)}
\gppoint{gp mark 0}{(3.388,2.914)}
\gppoint{gp mark 0}{(3.388,3.255)}
\gppoint{gp mark 0}{(3.388,2.914)}
\gppoint{gp mark 0}{(3.388,2.979)}
\gppoint{gp mark 0}{(3.388,3.094)}
\gppoint{gp mark 0}{(3.388,3.548)}
\gppoint{gp mark 0}{(3.388,3.479)}
\gppoint{gp mark 0}{(3.388,3.094)}
\gppoint{gp mark 0}{(3.388,3.168)}
\gppoint{gp mark 0}{(3.388,2.914)}
\gppoint{gp mark 0}{(3.388,3.010)}
\gppoint{gp mark 0}{(3.388,2.914)}
\gppoint{gp mark 0}{(3.388,3.213)}
\gppoint{gp mark 0}{(3.388,3.333)}
\gppoint{gp mark 0}{(3.388,3.295)}
\gppoint{gp mark 0}{(3.388,3.295)}
\gppoint{gp mark 0}{(3.388,3.094)}
\gppoint{gp mark 0}{(3.388,3.067)}
\gppoint{gp mark 0}{(3.388,2.801)}
\gppoint{gp mark 0}{(3.388,2.801)}
\gppoint{gp mark 0}{(3.388,3.010)}
\gppoint{gp mark 0}{(3.388,2.801)}
\gppoint{gp mark 0}{(3.388,3.805)}
\gppoint{gp mark 0}{(3.388,3.010)}
\gppoint{gp mark 0}{(3.388,3.010)}
\gppoint{gp mark 0}{(3.388,2.914)}
\gppoint{gp mark 0}{(3.388,3.255)}
\gppoint{gp mark 0}{(3.388,3.010)}
\gppoint{gp mark 0}{(3.388,3.010)}
\gppoint{gp mark 0}{(3.388,3.276)}
\gppoint{gp mark 0}{(3.388,2.914)}
\gppoint{gp mark 0}{(3.388,3.213)}
\gppoint{gp mark 0}{(3.388,3.010)}
\gppoint{gp mark 0}{(3.388,3.213)}
\gppoint{gp mark 0}{(3.388,3.213)}
\gppoint{gp mark 0}{(3.388,2.914)}
\gppoint{gp mark 0}{(3.388,3.010)}
\gppoint{gp mark 0}{(3.388,3.010)}
\gppoint{gp mark 0}{(3.388,2.914)}
\gppoint{gp mark 0}{(3.388,3.144)}
\gppoint{gp mark 0}{(3.388,3.010)}
\gppoint{gp mark 0}{(3.388,3.191)}
\gppoint{gp mark 0}{(3.388,3.191)}
\gppoint{gp mark 0}{(3.388,3.039)}
\gppoint{gp mark 0}{(3.388,2.914)}
\gppoint{gp mark 0}{(3.388,3.255)}
\gppoint{gp mark 0}{(3.388,3.144)}
\gppoint{gp mark 0}{(3.388,3.010)}
\gppoint{gp mark 0}{(3.388,3.314)}
\gppoint{gp mark 0}{(3.388,2.979)}
\gppoint{gp mark 0}{(3.388,3.144)}
\gppoint{gp mark 0}{(3.388,3.010)}
\gppoint{gp mark 0}{(3.388,3.333)}
\gppoint{gp mark 0}{(3.388,3.314)}
\gppoint{gp mark 0}{(3.388,2.801)}
\gppoint{gp mark 0}{(3.388,3.521)}
\gppoint{gp mark 0}{(3.388,3.255)}
\gppoint{gp mark 0}{(3.388,2.801)}
\gppoint{gp mark 0}{(3.388,3.010)}
\gppoint{gp mark 0}{(3.388,3.368)}
\gppoint{gp mark 0}{(3.388,2.801)}
\gppoint{gp mark 0}{(3.388,2.914)}
\gppoint{gp mark 0}{(3.388,2.914)}
\gppoint{gp mark 0}{(3.388,2.801)}
\gppoint{gp mark 0}{(3.388,3.010)}
\gppoint{gp mark 0}{(3.388,3.333)}
\gppoint{gp mark 0}{(3.388,3.276)}
\gppoint{gp mark 0}{(3.388,3.010)}
\gppoint{gp mark 0}{(3.388,3.144)}
\gppoint{gp mark 0}{(3.388,3.333)}
\gppoint{gp mark 0}{(3.388,3.333)}
\gppoint{gp mark 0}{(3.388,3.333)}
\gppoint{gp mark 0}{(3.388,3.168)}
\gppoint{gp mark 0}{(3.388,3.333)}
\gppoint{gp mark 0}{(3.388,3.333)}
\gppoint{gp mark 0}{(3.388,3.168)}
\gppoint{gp mark 0}{(3.388,2.914)}
\gppoint{gp mark 0}{(3.388,3.385)}
\gppoint{gp mark 0}{(3.388,2.914)}
\gppoint{gp mark 0}{(3.388,2.979)}
\gppoint{gp mark 0}{(3.388,3.255)}
\gppoint{gp mark 0}{(3.388,2.979)}
\gppoint{gp mark 0}{(3.388,3.255)}
\gppoint{gp mark 0}{(3.388,3.295)}
\gppoint{gp mark 0}{(3.388,3.213)}
\gppoint{gp mark 0}{(3.388,3.010)}
\gppoint{gp mark 0}{(3.388,2.801)}
\gppoint{gp mark 0}{(3.388,2.801)}
\gppoint{gp mark 0}{(3.388,3.094)}
\gppoint{gp mark 0}{(3.388,3.144)}
\gppoint{gp mark 0}{(3.388,3.067)}
\gppoint{gp mark 0}{(3.388,3.295)}
\gppoint{gp mark 0}{(3.388,3.351)}
\gppoint{gp mark 0}{(3.388,2.801)}
\gppoint{gp mark 0}{(3.388,2.801)}
\gppoint{gp mark 0}{(3.388,2.914)}
\gppoint{gp mark 0}{(3.388,2.801)}
\gppoint{gp mark 0}{(3.388,3.067)}
\gppoint{gp mark 0}{(3.388,3.067)}
\gppoint{gp mark 0}{(3.388,3.067)}
\gppoint{gp mark 0}{(3.388,3.010)}
\gppoint{gp mark 0}{(3.388,3.067)}
\gppoint{gp mark 0}{(3.388,3.010)}
\gppoint{gp mark 0}{(3.388,2.801)}
\gppoint{gp mark 0}{(3.388,3.213)}
\gppoint{gp mark 0}{(3.388,3.067)}
\gppoint{gp mark 0}{(3.388,3.144)}
\gppoint{gp mark 0}{(3.388,3.010)}
\gppoint{gp mark 0}{(3.388,3.144)}
\gppoint{gp mark 0}{(3.388,2.801)}
\gppoint{gp mark 0}{(3.388,3.333)}
\gppoint{gp mark 0}{(3.388,2.801)}
\gppoint{gp mark 0}{(3.388,3.144)}
\gppoint{gp mark 0}{(3.388,3.255)}
\gppoint{gp mark 0}{(3.388,3.144)}
\gppoint{gp mark 0}{(3.388,3.213)}
\gppoint{gp mark 0}{(3.388,2.801)}
\gppoint{gp mark 0}{(3.388,3.295)}
\gppoint{gp mark 0}{(3.388,3.213)}
\gppoint{gp mark 0}{(3.388,3.368)}
\gppoint{gp mark 0}{(3.388,3.295)}
\gppoint{gp mark 0}{(3.388,3.368)}
\gppoint{gp mark 0}{(3.388,3.168)}
\gppoint{gp mark 0}{(3.388,3.094)}
\gppoint{gp mark 0}{(3.388,3.168)}
\gppoint{gp mark 0}{(3.388,3.010)}
\gppoint{gp mark 0}{(3.388,3.094)}
\gppoint{gp mark 0}{(3.388,3.010)}
\gppoint{gp mark 0}{(3.388,3.010)}
\gppoint{gp mark 0}{(3.388,3.039)}
\gppoint{gp mark 0}{(3.388,3.276)}
\gppoint{gp mark 0}{(3.388,3.010)}
\gppoint{gp mark 0}{(3.388,3.067)}
\gppoint{gp mark 0}{(3.388,2.801)}
\gppoint{gp mark 0}{(3.388,2.801)}
\gppoint{gp mark 0}{(3.388,3.276)}
\gppoint{gp mark 0}{(3.388,3.094)}
\gppoint{gp mark 0}{(3.388,3.010)}
\gppoint{gp mark 0}{(3.388,2.801)}
\gppoint{gp mark 0}{(3.388,2.801)}
\gppoint{gp mark 0}{(3.388,2.801)}
\gppoint{gp mark 0}{(3.388,2.801)}
\gppoint{gp mark 0}{(3.388,2.801)}
\gppoint{gp mark 0}{(3.388,3.010)}
\gppoint{gp mark 0}{(3.388,3.010)}
\gppoint{gp mark 0}{(3.388,3.067)}
\gppoint{gp mark 0}{(3.388,2.801)}
\gppoint{gp mark 0}{(3.388,2.801)}
\gppoint{gp mark 0}{(3.388,3.067)}
\gppoint{gp mark 0}{(3.388,2.914)}
\gppoint{gp mark 0}{(3.388,2.914)}
\gppoint{gp mark 0}{(3.388,3.010)}
\gppoint{gp mark 0}{(3.388,3.295)}
\gppoint{gp mark 0}{(3.388,2.914)}
\gppoint{gp mark 0}{(3.388,3.010)}
\gppoint{gp mark 0}{(3.388,3.010)}
\gppoint{gp mark 0}{(3.388,2.914)}
\gppoint{gp mark 0}{(3.388,3.010)}
\gppoint{gp mark 0}{(3.388,3.010)}
\gppoint{gp mark 0}{(3.388,2.979)}
\gppoint{gp mark 0}{(3.388,2.801)}
\gppoint{gp mark 0}{(3.388,3.010)}
\gppoint{gp mark 0}{(3.388,3.333)}
\gppoint{gp mark 0}{(3.388,2.914)}
\gppoint{gp mark 0}{(3.388,3.010)}
\gppoint{gp mark 0}{(3.388,2.914)}
\gppoint{gp mark 0}{(3.388,3.191)}
\gppoint{gp mark 0}{(3.388,2.914)}
\gppoint{gp mark 0}{(3.388,3.144)}
\gppoint{gp mark 0}{(3.388,2.914)}
\gppoint{gp mark 0}{(3.388,2.914)}
\gppoint{gp mark 0}{(3.388,3.418)}
\gppoint{gp mark 0}{(3.388,2.801)}
\gppoint{gp mark 0}{(3.388,3.168)}
\gppoint{gp mark 0}{(3.388,3.144)}
\gppoint{gp mark 0}{(3.388,3.449)}
\gppoint{gp mark 0}{(3.388,3.010)}
\gppoint{gp mark 0}{(3.388,3.067)}
\gppoint{gp mark 0}{(3.388,3.479)}
\gppoint{gp mark 0}{(3.388,2.914)}
\gppoint{gp mark 0}{(3.388,3.449)}
\gppoint{gp mark 0}{(3.388,3.521)}
\gppoint{gp mark 0}{(3.388,3.144)}
\gppoint{gp mark 0}{(3.388,2.801)}
\gppoint{gp mark 0}{(3.388,3.010)}
\gppoint{gp mark 0}{(3.388,3.094)}
\gppoint{gp mark 0}{(3.388,3.010)}
\gppoint{gp mark 0}{(3.388,3.010)}
\gppoint{gp mark 0}{(3.388,3.010)}
\gppoint{gp mark 0}{(3.388,3.067)}
\gppoint{gp mark 0}{(3.388,3.094)}
\gppoint{gp mark 0}{(3.388,3.144)}
\gppoint{gp mark 0}{(3.388,3.067)}
\gppoint{gp mark 0}{(3.388,3.010)}
\gppoint{gp mark 0}{(3.388,2.801)}
\gppoint{gp mark 0}{(3.388,2.914)}
\gppoint{gp mark 0}{(3.388,3.010)}
\gppoint{gp mark 0}{(3.388,3.255)}
\gppoint{gp mark 0}{(3.388,3.067)}
\gppoint{gp mark 0}{(3.388,2.801)}
\gppoint{gp mark 0}{(3.388,3.144)}
\gppoint{gp mark 0}{(3.388,3.010)}
\gppoint{gp mark 0}{(3.388,3.067)}
\gppoint{gp mark 0}{(3.388,3.010)}
\gppoint{gp mark 0}{(3.388,3.010)}
\gppoint{gp mark 0}{(3.388,3.094)}
\gppoint{gp mark 0}{(3.388,2.801)}
\gppoint{gp mark 0}{(3.388,2.914)}
\gppoint{gp mark 0}{(3.388,3.144)}
\gppoint{gp mark 0}{(3.388,3.368)}
\gppoint{gp mark 0}{(3.388,3.276)}
\gppoint{gp mark 0}{(3.388,2.914)}
\gppoint{gp mark 0}{(3.388,3.010)}
\gppoint{gp mark 0}{(3.388,3.368)}
\gppoint{gp mark 0}{(3.388,3.144)}
\gppoint{gp mark 0}{(3.388,3.144)}
\gppoint{gp mark 0}{(3.388,2.914)}
\gppoint{gp mark 0}{(3.388,3.255)}
\gppoint{gp mark 0}{(3.388,2.914)}
\gppoint{gp mark 0}{(3.388,3.255)}
\gppoint{gp mark 0}{(3.388,3.507)}
\gppoint{gp mark 0}{(3.388,2.914)}
\gppoint{gp mark 0}{(3.388,3.094)}
\gppoint{gp mark 0}{(3.388,3.213)}
\gppoint{gp mark 0}{(3.388,3.010)}
\gppoint{gp mark 0}{(3.388,3.144)}
\gppoint{gp mark 0}{(3.388,3.010)}
\gppoint{gp mark 0}{(3.388,2.914)}
\gppoint{gp mark 0}{(3.388,3.235)}
\gppoint{gp mark 0}{(3.388,3.295)}
\gppoint{gp mark 0}{(3.388,2.914)}
\gppoint{gp mark 0}{(3.388,2.914)}
\gppoint{gp mark 0}{(3.388,3.235)}
\gppoint{gp mark 0}{(3.388,2.801)}
\gppoint{gp mark 0}{(3.388,3.010)}
\gppoint{gp mark 0}{(3.388,3.314)}
\gppoint{gp mark 0}{(3.388,3.418)}
\gppoint{gp mark 0}{(3.388,3.255)}
\gppoint{gp mark 0}{(3.388,2.801)}
\gppoint{gp mark 0}{(3.388,3.010)}
\gppoint{gp mark 0}{(3.388,3.010)}
\gppoint{gp mark 0}{(3.388,2.914)}
\gppoint{gp mark 0}{(3.388,3.144)}
\gppoint{gp mark 0}{(3.388,3.368)}
\gppoint{gp mark 0}{(3.388,3.094)}
\gppoint{gp mark 0}{(3.388,3.067)}
\gppoint{gp mark 0}{(3.487,3.213)}
\gppoint{gp mark 0}{(3.487,3.010)}
\gppoint{gp mark 0}{(3.487,3.010)}
\gppoint{gp mark 0}{(3.487,3.144)}
\gppoint{gp mark 0}{(3.487,3.010)}
\gppoint{gp mark 0}{(3.487,3.010)}
\gppoint{gp mark 0}{(3.487,3.094)}
\gppoint{gp mark 0}{(3.487,3.010)}
\gppoint{gp mark 0}{(3.487,3.610)}
\gppoint{gp mark 0}{(3.487,3.255)}
\gppoint{gp mark 0}{(3.487,3.213)}
\gppoint{gp mark 0}{(3.487,3.010)}
\gppoint{gp mark 0}{(3.487,3.295)}
\gppoint{gp mark 0}{(3.487,3.144)}
\gppoint{gp mark 0}{(3.487,3.010)}
\gppoint{gp mark 0}{(3.487,3.610)}
\gppoint{gp mark 0}{(3.487,3.010)}
\gppoint{gp mark 0}{(3.487,3.010)}
\gppoint{gp mark 0}{(3.487,3.295)}
\gppoint{gp mark 0}{(3.487,3.333)}
\gppoint{gp mark 0}{(3.487,3.235)}
\gppoint{gp mark 0}{(3.487,3.295)}
\gppoint{gp mark 0}{(3.487,3.598)}
\gppoint{gp mark 0}{(3.487,3.119)}
\gppoint{gp mark 0}{(3.487,3.010)}
\gppoint{gp mark 0}{(3.487,3.610)}
\gppoint{gp mark 0}{(3.487,3.010)}
\gppoint{gp mark 0}{(3.487,3.351)}
\gppoint{gp mark 0}{(3.487,2.979)}
\gppoint{gp mark 0}{(3.487,3.119)}
\gppoint{gp mark 0}{(3.487,3.094)}
\gppoint{gp mark 0}{(3.487,3.535)}
\gppoint{gp mark 0}{(3.487,3.610)}
\gppoint{gp mark 0}{(3.487,3.144)}
\gppoint{gp mark 0}{(3.487,2.801)}
\gppoint{gp mark 0}{(3.487,3.610)}
\gppoint{gp mark 0}{(3.487,3.119)}
\gppoint{gp mark 0}{(3.487,2.801)}
\gppoint{gp mark 0}{(3.487,3.276)}
\gppoint{gp mark 0}{(3.487,3.610)}
\gppoint{gp mark 0}{(3.487,3.255)}
\gppoint{gp mark 0}{(3.487,3.094)}
\gppoint{gp mark 0}{(3.487,3.010)}
\gppoint{gp mark 0}{(3.487,3.295)}
\gppoint{gp mark 0}{(3.487,3.119)}
\gppoint{gp mark 0}{(3.487,3.368)}
\gppoint{gp mark 0}{(3.487,3.276)}
\gppoint{gp mark 0}{(3.487,3.010)}
\gppoint{gp mark 0}{(3.487,3.010)}
\gppoint{gp mark 0}{(3.487,3.235)}
\gppoint{gp mark 0}{(3.487,3.010)}
\gppoint{gp mark 0}{(3.487,3.295)}
\gppoint{gp mark 0}{(3.487,3.276)}
\gppoint{gp mark 0}{(3.487,3.010)}
\gppoint{gp mark 0}{(3.487,3.333)}
\gppoint{gp mark 0}{(3.487,3.521)}
\gppoint{gp mark 0}{(3.487,3.010)}
\gppoint{gp mark 0}{(3.487,3.368)}
\gppoint{gp mark 0}{(3.487,3.276)}
\gppoint{gp mark 0}{(3.487,3.276)}
\gppoint{gp mark 0}{(3.487,3.010)}
\gppoint{gp mark 0}{(3.487,3.235)}
\gppoint{gp mark 0}{(3.487,3.493)}
\gppoint{gp mark 0}{(3.487,3.094)}
\gppoint{gp mark 0}{(3.487,3.010)}
\gppoint{gp mark 0}{(3.487,3.094)}
\gppoint{gp mark 0}{(3.487,3.094)}
\gppoint{gp mark 0}{(3.487,3.368)}
\gppoint{gp mark 0}{(3.487,3.094)}
\gppoint{gp mark 0}{(3.487,3.255)}
\gppoint{gp mark 0}{(3.487,3.493)}
\gppoint{gp mark 0}{(3.487,3.144)}
\gppoint{gp mark 0}{(3.487,3.213)}
\gppoint{gp mark 0}{(3.487,3.822)}
\gppoint{gp mark 0}{(3.487,3.493)}
\gppoint{gp mark 0}{(3.487,3.010)}
\gppoint{gp mark 0}{(3.487,3.010)}
\gppoint{gp mark 0}{(3.487,3.822)}
\gppoint{gp mark 0}{(3.487,3.610)}
\gppoint{gp mark 0}{(3.487,3.094)}
\gppoint{gp mark 0}{(3.487,3.368)}
\gppoint{gp mark 0}{(3.487,3.368)}
\gppoint{gp mark 0}{(3.487,3.368)}
\gppoint{gp mark 0}{(3.487,3.610)}
\gppoint{gp mark 0}{(3.487,3.368)}
\gppoint{gp mark 0}{(3.487,2.914)}
\gppoint{gp mark 0}{(3.487,3.610)}
\gppoint{gp mark 0}{(3.487,3.622)}
\gppoint{gp mark 0}{(3.487,3.368)}
\gppoint{gp mark 0}{(3.487,3.535)}
\gppoint{gp mark 0}{(3.487,3.191)}
\gppoint{gp mark 0}{(3.487,3.368)}
\gppoint{gp mark 0}{(3.487,3.235)}
\gppoint{gp mark 0}{(3.487,3.368)}
\gppoint{gp mark 0}{(3.487,3.368)}
\gppoint{gp mark 0}{(3.487,2.801)}
\gppoint{gp mark 0}{(3.487,3.872)}
\gppoint{gp mark 0}{(3.487,3.610)}
\gppoint{gp mark 0}{(3.487,3.822)}
\gppoint{gp mark 0}{(3.487,3.610)}
\gppoint{gp mark 0}{(3.487,3.094)}
\gppoint{gp mark 0}{(3.487,3.276)}
\gppoint{gp mark 0}{(3.487,3.610)}
\gppoint{gp mark 0}{(3.487,2.948)}
\gppoint{gp mark 0}{(3.487,3.213)}
\gppoint{gp mark 0}{(3.487,3.039)}
\gppoint{gp mark 0}{(3.487,3.094)}
\gppoint{gp mark 0}{(3.487,3.144)}
\gppoint{gp mark 0}{(3.487,3.010)}
\gppoint{gp mark 0}{(3.487,3.368)}
\gppoint{gp mark 0}{(3.487,3.010)}
\gppoint{gp mark 0}{(3.487,3.144)}
\gppoint{gp mark 0}{(3.487,3.067)}
\gppoint{gp mark 0}{(3.487,3.191)}
\gppoint{gp mark 0}{(3.487,3.276)}
\gppoint{gp mark 0}{(3.487,3.493)}
\gppoint{gp mark 0}{(3.487,3.610)}
\gppoint{gp mark 0}{(3.487,3.010)}
\gppoint{gp mark 0}{(3.487,3.067)}
\gppoint{gp mark 0}{(3.487,2.914)}
\gppoint{gp mark 0}{(3.487,3.010)}
\gppoint{gp mark 0}{(3.487,2.914)}
\gppoint{gp mark 0}{(3.487,2.914)}
\gppoint{gp mark 0}{(3.487,3.521)}
\gppoint{gp mark 0}{(3.487,3.010)}
\gppoint{gp mark 0}{(3.487,3.119)}
\gppoint{gp mark 0}{(3.487,3.622)}
\gppoint{gp mark 0}{(3.487,3.235)}
\gppoint{gp mark 0}{(3.487,3.314)}
\gppoint{gp mark 0}{(3.487,2.801)}
\gppoint{gp mark 0}{(3.487,3.144)}
\gppoint{gp mark 0}{(3.487,3.010)}
\gppoint{gp mark 0}{(3.487,2.914)}
\gppoint{gp mark 0}{(3.487,3.235)}
\gppoint{gp mark 0}{(3.487,3.535)}
\gppoint{gp mark 0}{(3.487,3.144)}
\gppoint{gp mark 0}{(3.487,3.276)}
\gppoint{gp mark 0}{(3.487,3.434)}
\gppoint{gp mark 0}{(3.487,3.351)}
\gppoint{gp mark 0}{(3.487,3.759)}
\gppoint{gp mark 0}{(3.487,3.521)}
\gppoint{gp mark 0}{(3.487,3.235)}
\gppoint{gp mark 0}{(3.487,3.010)}
\gppoint{gp mark 0}{(3.487,3.010)}
\gppoint{gp mark 0}{(3.487,3.402)}
\gppoint{gp mark 0}{(3.487,3.213)}
\gppoint{gp mark 0}{(3.487,2.979)}
\gppoint{gp mark 0}{(3.487,3.235)}
\gppoint{gp mark 0}{(3.487,3.493)}
\gppoint{gp mark 0}{(3.487,3.699)}
\gppoint{gp mark 0}{(3.487,2.878)}
\gppoint{gp mark 0}{(3.487,3.094)}
\gppoint{gp mark 0}{(3.487,3.295)}
\gppoint{gp mark 0}{(3.487,2.914)}
\gppoint{gp mark 0}{(3.487,3.144)}
\gppoint{gp mark 0}{(3.487,3.119)}
\gppoint{gp mark 0}{(3.487,3.144)}
\gppoint{gp mark 0}{(3.487,3.010)}
\gppoint{gp mark 0}{(3.487,3.010)}
\gppoint{gp mark 0}{(3.487,3.295)}
\gppoint{gp mark 0}{(3.487,3.144)}
\gppoint{gp mark 0}{(3.487,3.010)}
\gppoint{gp mark 0}{(3.487,3.119)}
\gppoint{gp mark 0}{(3.487,3.276)}
\gppoint{gp mark 0}{(3.487,3.314)}
\gppoint{gp mark 0}{(3.487,2.914)}
\gppoint{gp mark 0}{(3.487,3.656)}
\gppoint{gp mark 0}{(3.487,3.385)}
\gppoint{gp mark 0}{(3.487,3.573)}
\gppoint{gp mark 0}{(3.487,3.067)}
\gppoint{gp mark 0}{(3.487,3.255)}
\gppoint{gp mark 0}{(3.487,3.255)}
\gppoint{gp mark 0}{(3.487,3.235)}
\gppoint{gp mark 0}{(3.487,3.067)}
\gppoint{gp mark 0}{(3.487,3.368)}
\gppoint{gp mark 0}{(3.487,3.168)}
\gppoint{gp mark 0}{(3.487,2.914)}
\gppoint{gp mark 0}{(3.487,2.801)}
\gppoint{gp mark 0}{(3.487,3.010)}
\gppoint{gp mark 0}{(3.487,3.610)}
\gppoint{gp mark 0}{(3.487,3.191)}
\gppoint{gp mark 0}{(3.487,3.010)}
\gppoint{gp mark 0}{(3.487,3.010)}
\gppoint{gp mark 0}{(3.487,3.295)}
\gppoint{gp mark 0}{(3.487,3.351)}
\gppoint{gp mark 0}{(3.487,3.094)}
\gppoint{gp mark 0}{(3.487,3.213)}
\gppoint{gp mark 0}{(3.487,3.213)}
\gppoint{gp mark 0}{(3.487,2.914)}
\gppoint{gp mark 0}{(3.487,3.010)}
\gppoint{gp mark 0}{(3.487,3.235)}
\gppoint{gp mark 0}{(3.487,3.094)}
\gppoint{gp mark 0}{(3.487,3.010)}
\gppoint{gp mark 0}{(3.487,3.168)}
\gppoint{gp mark 0}{(3.487,2.914)}
\gppoint{gp mark 0}{(3.487,2.914)}
\gppoint{gp mark 0}{(3.487,3.094)}
\gppoint{gp mark 0}{(3.487,2.914)}
\gppoint{gp mark 0}{(3.487,3.094)}
\gppoint{gp mark 0}{(3.487,3.276)}
\gppoint{gp mark 0}{(3.487,3.213)}
\gppoint{gp mark 0}{(3.487,3.295)}
\gppoint{gp mark 0}{(3.487,3.295)}
\gppoint{gp mark 0}{(3.487,3.144)}
\gppoint{gp mark 0}{(3.487,3.213)}
\gppoint{gp mark 0}{(3.487,3.094)}
\gppoint{gp mark 0}{(3.487,2.801)}
\gppoint{gp mark 0}{(3.487,3.295)}
\gppoint{gp mark 0}{(3.487,3.010)}
\gppoint{gp mark 0}{(3.487,3.010)}
\gppoint{gp mark 0}{(3.487,3.010)}
\gppoint{gp mark 0}{(3.487,3.010)}
\gppoint{gp mark 0}{(3.487,3.822)}
\gppoint{gp mark 0}{(3.487,3.213)}
\gppoint{gp mark 0}{(3.487,3.402)}
\gppoint{gp mark 0}{(3.487,2.979)}
\gppoint{gp mark 0}{(3.487,3.191)}
\gppoint{gp mark 0}{(3.487,2.914)}
\gppoint{gp mark 0}{(3.487,2.914)}
\gppoint{gp mark 0}{(3.487,3.295)}
\gppoint{gp mark 0}{(3.487,3.144)}
\gppoint{gp mark 0}{(3.487,3.333)}
\gppoint{gp mark 0}{(3.487,3.333)}
\gppoint{gp mark 0}{(3.487,3.434)}
\gppoint{gp mark 0}{(3.487,2.801)}
\gppoint{gp mark 0}{(3.487,3.368)}
\gppoint{gp mark 0}{(3.487,3.573)}
\gppoint{gp mark 0}{(3.487,3.295)}
\gppoint{gp mark 0}{(3.487,3.385)}
\gppoint{gp mark 0}{(3.487,3.507)}
\gppoint{gp mark 0}{(3.487,3.067)}
\gppoint{gp mark 0}{(3.487,3.094)}
\gppoint{gp mark 0}{(3.487,3.295)}
\gppoint{gp mark 0}{(3.487,3.094)}
\gppoint{gp mark 0}{(3.487,3.094)}
\gppoint{gp mark 0}{(3.487,3.094)}
\gppoint{gp mark 0}{(3.487,2.801)}
\gppoint{gp mark 0}{(3.487,3.351)}
\gppoint{gp mark 0}{(3.487,3.276)}
\gppoint{gp mark 0}{(3.487,3.213)}
\gppoint{gp mark 0}{(3.487,3.168)}
\gppoint{gp mark 0}{(3.487,3.645)}
\gppoint{gp mark 0}{(3.487,3.168)}
\gppoint{gp mark 0}{(3.487,2.914)}
\gppoint{gp mark 0}{(3.487,3.351)}
\gppoint{gp mark 0}{(3.487,3.168)}
\gppoint{gp mark 0}{(3.487,3.010)}
\gppoint{gp mark 0}{(3.487,3.094)}
\gppoint{gp mark 0}{(3.487,3.010)}
\gppoint{gp mark 0}{(3.487,3.067)}
\gppoint{gp mark 0}{(3.487,3.010)}
\gppoint{gp mark 0}{(3.487,3.333)}
\gppoint{gp mark 0}{(3.487,3.493)}
\gppoint{gp mark 0}{(3.487,3.094)}
\gppoint{gp mark 0}{(3.487,3.351)}
\gppoint{gp mark 0}{(3.487,3.010)}
\gppoint{gp mark 0}{(3.487,3.067)}
\gppoint{gp mark 0}{(3.487,3.314)}
\gppoint{gp mark 0}{(3.487,3.586)}
\gppoint{gp mark 0}{(3.487,3.010)}
\gppoint{gp mark 0}{(3.487,3.434)}
\gppoint{gp mark 0}{(3.487,3.010)}
\gppoint{gp mark 0}{(3.487,3.479)}
\gppoint{gp mark 0}{(3.487,3.010)}
\gppoint{gp mark 0}{(3.487,3.010)}
\gppoint{gp mark 0}{(3.487,3.119)}
\gppoint{gp mark 0}{(3.487,3.010)}
\gppoint{gp mark 0}{(3.487,3.094)}
\gppoint{gp mark 0}{(3.487,3.333)}
\gppoint{gp mark 0}{(3.487,3.213)}
\gppoint{gp mark 0}{(3.487,3.010)}
\gppoint{gp mark 0}{(3.487,3.010)}
\gppoint{gp mark 0}{(3.487,3.010)}
\gppoint{gp mark 0}{(3.487,3.010)}
\gppoint{gp mark 0}{(3.487,3.010)}
\gppoint{gp mark 0}{(3.487,3.168)}
\gppoint{gp mark 0}{(3.487,3.351)}
\gppoint{gp mark 0}{(3.487,3.314)}
\gppoint{gp mark 0}{(3.487,2.948)}
\gppoint{gp mark 0}{(3.487,2.914)}
\gppoint{gp mark 0}{(3.487,3.402)}
\gppoint{gp mark 0}{(3.487,3.710)}
\gppoint{gp mark 0}{(3.487,3.213)}
\gppoint{gp mark 0}{(3.487,3.144)}
\gppoint{gp mark 0}{(3.487,3.295)}
\gppoint{gp mark 0}{(3.487,3.295)}
\gppoint{gp mark 0}{(3.487,3.276)}
\gppoint{gp mark 0}{(3.487,2.914)}
\gppoint{gp mark 0}{(3.487,3.449)}
\gppoint{gp mark 0}{(3.487,2.665)}
\gppoint{gp mark 0}{(3.487,3.351)}
\gppoint{gp mark 0}{(3.487,3.094)}
\gppoint{gp mark 0}{(3.487,3.168)}
\gppoint{gp mark 0}{(3.487,3.314)}
\gppoint{gp mark 0}{(3.487,3.368)}
\gppoint{gp mark 0}{(3.487,3.314)}
\gppoint{gp mark 0}{(3.487,3.213)}
\gppoint{gp mark 0}{(3.487,3.314)}
\gppoint{gp mark 0}{(3.487,3.067)}
\gppoint{gp mark 0}{(3.487,3.368)}
\gppoint{gp mark 0}{(3.487,3.314)}
\gppoint{gp mark 0}{(3.487,3.314)}
\gppoint{gp mark 0}{(3.487,3.010)}
\gppoint{gp mark 0}{(3.487,3.276)}
\gppoint{gp mark 0}{(3.487,3.333)}
\gppoint{gp mark 0}{(3.487,3.144)}
\gppoint{gp mark 0}{(3.487,3.351)}
\gppoint{gp mark 0}{(3.487,3.314)}
\gppoint{gp mark 0}{(3.487,3.010)}
\gppoint{gp mark 0}{(3.487,3.385)}
\gppoint{gp mark 0}{(3.487,3.144)}
\gppoint{gp mark 0}{(3.487,3.010)}
\gppoint{gp mark 0}{(3.487,3.067)}
\gppoint{gp mark 0}{(3.487,3.796)}
\gppoint{gp mark 0}{(3.487,3.144)}
\gppoint{gp mark 0}{(3.487,3.796)}
\gppoint{gp mark 0}{(3.487,3.276)}
\gppoint{gp mark 0}{(3.487,3.796)}
\gppoint{gp mark 0}{(3.487,2.914)}
\gppoint{gp mark 0}{(3.487,3.067)}
\gppoint{gp mark 0}{(3.487,3.094)}
\gppoint{gp mark 0}{(3.487,3.333)}
\gppoint{gp mark 0}{(3.487,3.213)}
\gppoint{gp mark 0}{(3.487,3.561)}
\gppoint{gp mark 0}{(3.487,3.385)}
\gppoint{gp mark 0}{(3.487,3.235)}
\gppoint{gp mark 0}{(3.487,3.535)}
\gppoint{gp mark 0}{(3.487,3.402)}
\gppoint{gp mark 0}{(3.487,3.333)}
\gppoint{gp mark 0}{(3.487,3.333)}
\gppoint{gp mark 0}{(3.487,3.333)}
\gppoint{gp mark 0}{(3.487,3.333)}
\gppoint{gp mark 0}{(3.487,3.168)}
\gppoint{gp mark 0}{(3.487,3.213)}
\gppoint{gp mark 0}{(3.487,3.295)}
\gppoint{gp mark 0}{(3.487,3.333)}
\gppoint{gp mark 0}{(3.487,3.295)}
\gppoint{gp mark 0}{(3.487,3.168)}
\gppoint{gp mark 0}{(3.487,3.295)}
\gppoint{gp mark 0}{(3.487,3.010)}
\gppoint{gp mark 0}{(3.487,3.010)}
\gppoint{gp mark 0}{(3.487,3.144)}
\gppoint{gp mark 0}{(3.487,4.058)}
\gppoint{gp mark 0}{(3.487,3.010)}
\gppoint{gp mark 0}{(3.487,3.295)}
\gppoint{gp mark 0}{(3.487,3.067)}
\gppoint{gp mark 0}{(3.487,3.276)}
\gppoint{gp mark 0}{(3.487,3.168)}
\gppoint{gp mark 0}{(3.487,3.168)}
\gppoint{gp mark 0}{(3.487,2.914)}
\gppoint{gp mark 0}{(3.487,3.094)}
\gppoint{gp mark 0}{(3.487,3.168)}
\gppoint{gp mark 0}{(3.487,3.094)}
\gppoint{gp mark 0}{(3.487,3.010)}
\gppoint{gp mark 0}{(3.487,3.168)}
\gppoint{gp mark 0}{(3.487,2.914)}
\gppoint{gp mark 0}{(3.487,3.168)}
\gppoint{gp mark 0}{(3.487,2.979)}
\gppoint{gp mark 0}{(3.487,3.479)}
\gppoint{gp mark 0}{(3.487,3.168)}
\gppoint{gp mark 0}{(3.487,3.535)}
\gppoint{gp mark 0}{(3.487,3.295)}
\gppoint{gp mark 0}{(3.487,3.168)}
\gppoint{gp mark 0}{(3.487,3.168)}
\gppoint{gp mark 0}{(3.487,2.914)}
\gppoint{gp mark 0}{(3.487,3.402)}
\gppoint{gp mark 0}{(3.487,3.168)}
\gppoint{gp mark 0}{(3.487,3.434)}
\gppoint{gp mark 0}{(3.487,3.235)}
\gppoint{gp mark 0}{(3.487,3.067)}
\gppoint{gp mark 0}{(3.487,3.213)}
\gppoint{gp mark 0}{(3.487,3.235)}
\gppoint{gp mark 0}{(3.487,3.067)}
\gppoint{gp mark 0}{(3.487,3.213)}
\gppoint{gp mark 0}{(3.487,3.067)}
\gppoint{gp mark 0}{(3.487,3.333)}
\gppoint{gp mark 0}{(3.487,3.333)}
\gppoint{gp mark 0}{(3.487,3.255)}
\gppoint{gp mark 0}{(3.487,3.119)}
\gppoint{gp mark 0}{(3.487,3.333)}
\gppoint{gp mark 0}{(3.487,3.333)}
\gppoint{gp mark 0}{(3.487,3.479)}
\gppoint{gp mark 0}{(3.487,3.168)}
\gppoint{gp mark 0}{(3.487,3.010)}
\gppoint{gp mark 0}{(3.487,3.295)}
\gppoint{gp mark 0}{(3.487,3.235)}
\gppoint{gp mark 0}{(3.487,2.914)}
\gppoint{gp mark 0}{(3.487,3.402)}
\gppoint{gp mark 0}{(3.487,2.914)}
\gppoint{gp mark 0}{(3.487,3.094)}
\gppoint{gp mark 0}{(3.487,3.276)}
\gppoint{gp mark 0}{(3.487,3.213)}
\gppoint{gp mark 0}{(3.487,3.094)}
\gppoint{gp mark 0}{(3.487,3.295)}
\gppoint{gp mark 0}{(3.487,3.010)}
\gppoint{gp mark 0}{(3.487,3.295)}
\gppoint{gp mark 0}{(3.487,3.295)}
\gppoint{gp mark 0}{(3.487,2.914)}
\gppoint{gp mark 0}{(3.487,3.295)}
\gppoint{gp mark 0}{(3.487,3.010)}
\gppoint{gp mark 0}{(3.487,3.010)}
\gppoint{gp mark 0}{(3.487,3.144)}
\gppoint{gp mark 0}{(3.487,3.168)}
\gppoint{gp mark 0}{(3.487,3.144)}
\gppoint{gp mark 0}{(3.487,3.385)}
\gppoint{gp mark 0}{(3.487,3.010)}
\gppoint{gp mark 0}{(3.487,3.235)}
\gppoint{gp mark 0}{(3.487,3.010)}
\gppoint{gp mark 0}{(3.487,2.914)}
\gppoint{gp mark 0}{(3.487,3.535)}
\gppoint{gp mark 0}{(3.487,3.295)}
\gppoint{gp mark 0}{(3.487,3.213)}
\gppoint{gp mark 0}{(3.487,3.168)}
\gppoint{gp mark 0}{(3.487,3.276)}
\gppoint{gp mark 0}{(3.487,3.276)}
\gppoint{gp mark 0}{(3.487,3.276)}
\gppoint{gp mark 0}{(3.487,2.914)}
\gppoint{gp mark 0}{(3.487,3.010)}
\gppoint{gp mark 0}{(3.487,3.010)}
\gppoint{gp mark 0}{(3.487,3.010)}
\gppoint{gp mark 0}{(3.487,3.010)}
\gppoint{gp mark 0}{(3.487,3.144)}
\gppoint{gp mark 0}{(3.487,3.213)}
\gppoint{gp mark 0}{(3.487,3.067)}
\gppoint{gp mark 0}{(3.487,3.351)}
\gppoint{gp mark 0}{(3.487,2.914)}
\gppoint{gp mark 0}{(3.487,3.144)}
\gppoint{gp mark 0}{(3.487,2.914)}
\gppoint{gp mark 0}{(3.487,2.914)}
\gppoint{gp mark 0}{(3.487,3.235)}
\gppoint{gp mark 0}{(3.577,3.168)}
\gppoint{gp mark 0}{(3.577,3.295)}
\gppoint{gp mark 0}{(3.577,3.168)}
\gppoint{gp mark 0}{(3.577,3.213)}
\gppoint{gp mark 0}{(3.577,3.235)}
\gppoint{gp mark 0}{(3.577,3.418)}
\gppoint{gp mark 0}{(3.577,3.634)}
\gppoint{gp mark 0}{(3.577,3.213)}
\gppoint{gp mark 0}{(3.577,3.067)}
\gppoint{gp mark 0}{(3.577,3.213)}
\gppoint{gp mark 0}{(3.577,3.094)}
\gppoint{gp mark 0}{(3.577,3.067)}
\gppoint{gp mark 0}{(3.577,3.067)}
\gppoint{gp mark 0}{(3.577,3.548)}
\gppoint{gp mark 0}{(3.577,3.067)}
\gppoint{gp mark 0}{(3.577,3.067)}
\gppoint{gp mark 0}{(3.577,3.333)}
\gppoint{gp mark 0}{(3.577,3.276)}
\gppoint{gp mark 0}{(3.577,3.351)}
\gppoint{gp mark 0}{(3.577,3.010)}
\gppoint{gp mark 0}{(3.577,3.094)}
\gppoint{gp mark 0}{(3.577,3.067)}
\gppoint{gp mark 0}{(3.577,3.368)}
\gppoint{gp mark 0}{(3.577,3.276)}
\gppoint{gp mark 0}{(3.577,3.295)}
\gppoint{gp mark 0}{(3.577,3.235)}
\gppoint{gp mark 0}{(3.577,3.094)}
\gppoint{gp mark 0}{(3.577,3.548)}
\gppoint{gp mark 0}{(3.577,3.010)}
\gppoint{gp mark 0}{(3.577,3.314)}
\gppoint{gp mark 0}{(3.577,2.914)}
\gppoint{gp mark 0}{(3.577,3.119)}
\gppoint{gp mark 0}{(3.577,3.168)}
\gppoint{gp mark 0}{(3.577,3.094)}
\gppoint{gp mark 0}{(3.577,3.168)}
\gppoint{gp mark 0}{(3.577,3.385)}
\gppoint{gp mark 0}{(3.577,3.168)}
\gppoint{gp mark 0}{(3.577,3.333)}
\gppoint{gp mark 0}{(3.577,3.333)}
\gppoint{gp mark 0}{(3.577,2.948)}
\gppoint{gp mark 0}{(3.577,2.914)}
\gppoint{gp mark 0}{(3.577,3.119)}
\gppoint{gp mark 0}{(3.577,3.067)}
\gppoint{gp mark 0}{(3.577,3.276)}
\gppoint{gp mark 0}{(3.577,3.493)}
\gppoint{gp mark 0}{(3.577,3.010)}
\gppoint{gp mark 0}{(3.577,3.094)}
\gppoint{gp mark 0}{(3.577,3.168)}
\gppoint{gp mark 0}{(3.577,2.914)}
\gppoint{gp mark 0}{(3.577,3.168)}
\gppoint{gp mark 0}{(3.577,3.191)}
\gppoint{gp mark 0}{(3.577,3.010)}
\gppoint{gp mark 0}{(3.577,3.213)}
\gppoint{gp mark 0}{(3.577,3.010)}
\gppoint{gp mark 0}{(3.577,3.507)}
\gppoint{gp mark 0}{(3.577,3.094)}
\gppoint{gp mark 0}{(3.577,3.119)}
\gppoint{gp mark 0}{(3.577,2.914)}
\gppoint{gp mark 0}{(3.577,3.402)}
\gppoint{gp mark 0}{(3.577,3.010)}
\gppoint{gp mark 0}{(3.577,2.914)}
\gppoint{gp mark 0}{(3.577,3.314)}
\gppoint{gp mark 0}{(3.577,3.464)}
\gppoint{gp mark 0}{(3.577,3.449)}
\gppoint{gp mark 0}{(3.577,3.010)}
\gppoint{gp mark 0}{(3.577,2.801)}
\gppoint{gp mark 0}{(3.577,3.276)}
\gppoint{gp mark 0}{(3.577,3.276)}
\gppoint{gp mark 0}{(3.577,3.144)}
\gppoint{gp mark 0}{(3.577,3.213)}
\gppoint{gp mark 0}{(3.577,3.010)}
\gppoint{gp mark 0}{(3.577,3.010)}
\gppoint{gp mark 0}{(3.577,3.276)}
\gppoint{gp mark 0}{(3.577,3.402)}
\gppoint{gp mark 0}{(3.577,3.144)}
\gppoint{gp mark 0}{(3.577,3.235)}
\gppoint{gp mark 0}{(3.577,3.010)}
\gppoint{gp mark 0}{(3.577,2.801)}
\gppoint{gp mark 0}{(3.577,3.168)}
\gppoint{gp mark 0}{(3.577,3.144)}
\gppoint{gp mark 0}{(3.577,3.010)}
\gppoint{gp mark 0}{(3.577,3.213)}
\gppoint{gp mark 0}{(3.577,3.351)}
\gppoint{gp mark 0}{(3.577,3.464)}
\gppoint{gp mark 0}{(3.577,3.235)}
\gppoint{gp mark 0}{(3.577,2.914)}
\gppoint{gp mark 0}{(3.577,3.235)}
\gppoint{gp mark 0}{(3.577,3.493)}
\gppoint{gp mark 0}{(3.577,3.067)}
\gppoint{gp mark 0}{(3.577,3.067)}
\gppoint{gp mark 0}{(3.577,3.067)}
\gppoint{gp mark 0}{(3.577,3.235)}
\gppoint{gp mark 0}{(3.577,3.255)}
\gppoint{gp mark 0}{(3.577,3.235)}
\gppoint{gp mark 0}{(3.577,3.449)}
\gppoint{gp mark 0}{(3.577,3.235)}
\gppoint{gp mark 0}{(3.577,3.351)}
\gppoint{gp mark 0}{(3.577,3.493)}
\gppoint{gp mark 0}{(3.577,3.678)}
\gppoint{gp mark 0}{(3.577,3.235)}
\gppoint{gp mark 0}{(3.577,3.144)}
\gppoint{gp mark 0}{(3.577,3.464)}
\gppoint{gp mark 0}{(3.577,3.235)}
\gppoint{gp mark 0}{(3.577,3.507)}
\gppoint{gp mark 0}{(3.577,3.276)}
\gppoint{gp mark 0}{(3.577,3.678)}
\gppoint{gp mark 0}{(3.577,2.914)}
\gppoint{gp mark 0}{(3.577,3.295)}
\gppoint{gp mark 0}{(3.577,2.914)}
\gppoint{gp mark 0}{(3.577,2.979)}
\gppoint{gp mark 0}{(3.577,3.168)}
\gppoint{gp mark 0}{(3.577,3.464)}
\gppoint{gp mark 0}{(3.577,3.333)}
\gppoint{gp mark 0}{(3.577,3.010)}
\gppoint{gp mark 0}{(3.577,3.010)}
\gppoint{gp mark 0}{(3.577,3.368)}
\gppoint{gp mark 0}{(3.577,3.191)}
\gppoint{gp mark 0}{(3.577,3.622)}
\gppoint{gp mark 0}{(3.577,3.010)}
\gppoint{gp mark 0}{(3.577,3.314)}
\gppoint{gp mark 0}{(3.577,3.314)}
\gppoint{gp mark 0}{(3.577,3.548)}
\gppoint{gp mark 0}{(3.577,3.235)}
\gppoint{gp mark 0}{(3.577,3.168)}
\gppoint{gp mark 0}{(3.577,3.168)}
\gppoint{gp mark 0}{(3.577,3.168)}
\gppoint{gp mark 0}{(3.577,3.740)}
\gppoint{gp mark 0}{(3.577,3.067)}
\gppoint{gp mark 0}{(3.577,3.740)}
\gppoint{gp mark 0}{(3.577,3.856)}
\gppoint{gp mark 0}{(3.577,3.418)}
\gppoint{gp mark 0}{(3.577,2.914)}
\gppoint{gp mark 0}{(3.577,3.276)}
\gppoint{gp mark 0}{(3.577,3.067)}
\gppoint{gp mark 0}{(3.577,3.213)}
\gppoint{gp mark 0}{(3.577,3.351)}
\gppoint{gp mark 0}{(3.577,3.634)}
\gppoint{gp mark 0}{(3.577,3.144)}
\gppoint{gp mark 0}{(3.577,3.144)}
\gppoint{gp mark 0}{(3.577,3.548)}
\gppoint{gp mark 0}{(3.577,3.144)}
\gppoint{gp mark 0}{(3.577,3.295)}
\gppoint{gp mark 0}{(3.577,3.295)}
\gppoint{gp mark 0}{(3.577,3.351)}
\gppoint{gp mark 0}{(3.577,3.010)}
\gppoint{gp mark 0}{(3.577,3.351)}
\gppoint{gp mark 0}{(3.577,3.402)}
\gppoint{gp mark 0}{(3.577,3.010)}
\gppoint{gp mark 0}{(3.577,3.368)}
\gppoint{gp mark 0}{(3.577,3.434)}
\gppoint{gp mark 0}{(3.577,3.295)}
\gppoint{gp mark 0}{(3.577,3.119)}
\gppoint{gp mark 0}{(3.577,3.276)}
\gppoint{gp mark 0}{(3.577,3.213)}
\gppoint{gp mark 0}{(3.577,3.535)}
\gppoint{gp mark 0}{(3.577,3.276)}
\gppoint{gp mark 0}{(3.577,3.094)}
\gppoint{gp mark 0}{(3.577,3.418)}
\gppoint{gp mark 0}{(3.577,3.778)}
\gppoint{gp mark 0}{(3.577,2.979)}
\gppoint{gp mark 0}{(3.577,3.535)}
\gppoint{gp mark 0}{(3.577,3.573)}
\gppoint{gp mark 0}{(3.577,3.094)}
\gppoint{gp mark 0}{(3.577,3.813)}
\gppoint{gp mark 0}{(3.577,3.010)}
\gppoint{gp mark 0}{(3.577,3.787)}
\gppoint{gp mark 0}{(3.577,3.094)}
\gppoint{gp mark 0}{(3.577,3.094)}
\gppoint{gp mark 0}{(3.577,3.787)}
\gppoint{gp mark 0}{(3.577,3.094)}
\gppoint{gp mark 0}{(3.577,3.094)}
\gppoint{gp mark 0}{(3.577,3.067)}
\gppoint{gp mark 0}{(3.577,3.255)}
\gppoint{gp mark 0}{(3.577,2.914)}
\gppoint{gp mark 0}{(3.577,2.801)}
\gppoint{gp mark 0}{(3.577,3.010)}
\gppoint{gp mark 0}{(3.577,4.015)}
\gppoint{gp mark 0}{(3.577,2.914)}
\gppoint{gp mark 0}{(3.577,2.914)}
\gppoint{gp mark 0}{(3.577,3.255)}
\gppoint{gp mark 0}{(3.577,2.914)}
\gppoint{gp mark 0}{(3.577,3.276)}
\gppoint{gp mark 0}{(3.577,3.333)}
\gppoint{gp mark 0}{(3.577,3.295)}
\gppoint{gp mark 0}{(3.577,3.295)}
\gppoint{gp mark 0}{(3.577,3.295)}
\gppoint{gp mark 0}{(3.577,3.010)}
\gppoint{gp mark 0}{(3.577,3.368)}
\gppoint{gp mark 0}{(3.577,3.314)}
\gppoint{gp mark 0}{(3.577,3.368)}
\gppoint{gp mark 0}{(3.577,2.801)}
\gppoint{gp mark 0}{(3.577,3.434)}
\gppoint{gp mark 0}{(3.577,3.067)}
\gppoint{gp mark 0}{(3.577,3.168)}
\gppoint{gp mark 0}{(3.577,3.255)}
\gppoint{gp mark 0}{(3.577,3.295)}
\gppoint{gp mark 0}{(3.577,3.548)}
\gppoint{gp mark 0}{(3.577,3.276)}
\gppoint{gp mark 0}{(3.577,3.255)}
\gppoint{gp mark 0}{(3.577,3.213)}
\gppoint{gp mark 0}{(3.577,3.213)}
\gppoint{gp mark 0}{(3.577,3.796)}
\gppoint{gp mark 0}{(3.577,3.449)}
\gppoint{gp mark 0}{(3.577,3.191)}
\gppoint{gp mark 0}{(3.577,3.295)}
\gppoint{gp mark 0}{(3.577,2.914)}
\gppoint{gp mark 0}{(3.577,3.010)}
\gppoint{gp mark 0}{(3.577,3.010)}
\gppoint{gp mark 0}{(3.577,3.778)}
\gppoint{gp mark 0}{(3.577,3.191)}
\gppoint{gp mark 0}{(3.577,3.787)}
\gppoint{gp mark 0}{(3.577,3.276)}
\gppoint{gp mark 0}{(3.577,3.010)}
\gppoint{gp mark 0}{(3.577,3.010)}
\gppoint{gp mark 0}{(3.577,2.914)}
\gppoint{gp mark 0}{(3.577,3.548)}
\gppoint{gp mark 0}{(3.577,3.464)}
\gppoint{gp mark 0}{(3.577,3.235)}
\gppoint{gp mark 0}{(3.577,3.010)}
\gppoint{gp mark 0}{(3.577,3.255)}
\gppoint{gp mark 0}{(3.577,3.144)}
\gppoint{gp mark 0}{(3.577,3.805)}
\gppoint{gp mark 0}{(3.577,3.010)}
\gppoint{gp mark 0}{(3.577,3.464)}
\gppoint{gp mark 0}{(3.577,3.168)}
\gppoint{gp mark 0}{(3.577,3.144)}
\gppoint{gp mark 0}{(3.577,3.235)}
\gppoint{gp mark 0}{(3.577,3.010)}
\gppoint{gp mark 0}{(3.577,3.010)}
\gppoint{gp mark 0}{(3.577,3.656)}
\gppoint{gp mark 0}{(3.577,2.979)}
\gppoint{gp mark 0}{(3.577,3.168)}
\gppoint{gp mark 0}{(3.577,3.144)}
\gppoint{gp mark 0}{(3.577,3.235)}
\gppoint{gp mark 0}{(3.577,3.213)}
\gppoint{gp mark 0}{(3.577,3.144)}
\gppoint{gp mark 0}{(3.577,3.645)}
\gppoint{gp mark 0}{(3.577,3.295)}
\gppoint{gp mark 0}{(3.577,3.822)}
\gppoint{gp mark 0}{(3.577,3.368)}
\gppoint{gp mark 0}{(3.577,3.010)}
\gppoint{gp mark 0}{(3.577,3.255)}
\gppoint{gp mark 0}{(3.577,3.333)}
\gppoint{gp mark 0}{(3.577,3.235)}
\gppoint{gp mark 0}{(3.577,3.010)}
\gppoint{gp mark 0}{(3.577,3.295)}
\gppoint{gp mark 0}{(3.577,3.787)}
\gppoint{gp mark 0}{(3.577,3.235)}
\gppoint{gp mark 0}{(3.577,3.479)}
\gppoint{gp mark 0}{(3.577,3.368)}
\gppoint{gp mark 0}{(3.577,3.940)}
\gppoint{gp mark 0}{(3.577,3.368)}
\gppoint{gp mark 0}{(3.577,2.914)}
\gppoint{gp mark 0}{(3.577,2.914)}
\gppoint{gp mark 0}{(3.577,3.168)}
\gppoint{gp mark 0}{(3.577,3.168)}
\gppoint{gp mark 0}{(3.577,3.168)}
\gppoint{gp mark 0}{(3.577,3.385)}
\gppoint{gp mark 0}{(3.577,3.434)}
\gppoint{gp mark 0}{(3.577,3.168)}
\gppoint{gp mark 0}{(3.577,2.914)}
\gppoint{gp mark 0}{(3.577,3.368)}
\gppoint{gp mark 0}{(3.577,3.010)}
\gppoint{gp mark 0}{(3.577,3.385)}
\gppoint{gp mark 0}{(3.577,3.255)}
\gppoint{gp mark 0}{(3.577,3.434)}
\gppoint{gp mark 0}{(3.577,3.314)}
\gppoint{gp mark 0}{(3.577,3.813)}
\gppoint{gp mark 0}{(3.577,3.351)}
\gppoint{gp mark 0}{(3.577,3.213)}
\gppoint{gp mark 0}{(3.577,3.464)}
\gppoint{gp mark 0}{(3.577,3.368)}
\gppoint{gp mark 0}{(3.577,3.010)}
\gppoint{gp mark 0}{(3.577,3.168)}
\gppoint{gp mark 0}{(3.577,3.295)}
\gppoint{gp mark 0}{(3.577,3.368)}
\gppoint{gp mark 0}{(3.577,3.213)}
\gppoint{gp mark 0}{(3.577,3.368)}
\gppoint{gp mark 0}{(3.577,3.561)}
\gppoint{gp mark 0}{(3.577,3.010)}
\gppoint{gp mark 0}{(3.577,3.213)}
\gppoint{gp mark 0}{(3.577,3.255)}
\gppoint{gp mark 0}{(3.577,3.168)}
\gppoint{gp mark 0}{(3.577,3.255)}
\gppoint{gp mark 0}{(3.577,3.368)}
\gppoint{gp mark 0}{(3.577,3.094)}
\gppoint{gp mark 0}{(3.577,2.914)}
\gppoint{gp mark 0}{(3.577,3.368)}
\gppoint{gp mark 0}{(3.577,3.010)}
\gppoint{gp mark 0}{(3.577,3.402)}
\gppoint{gp mark 0}{(3.577,3.010)}
\gppoint{gp mark 0}{(3.577,3.235)}
\gppoint{gp mark 0}{(3.577,3.911)}
\gppoint{gp mark 0}{(3.577,3.368)}
\gppoint{gp mark 0}{(3.577,3.010)}
\gppoint{gp mark 0}{(3.577,3.368)}
\gppoint{gp mark 0}{(3.577,3.010)}
\gppoint{gp mark 0}{(3.577,3.368)}
\gppoint{gp mark 0}{(3.577,3.368)}
\gppoint{gp mark 0}{(3.577,3.010)}
\gppoint{gp mark 0}{(3.577,3.385)}
\gppoint{gp mark 0}{(3.577,3.295)}
\gppoint{gp mark 0}{(3.577,3.368)}
\gppoint{gp mark 0}{(3.577,3.333)}
\gppoint{gp mark 0}{(3.577,3.888)}
\gppoint{gp mark 0}{(3.577,3.368)}
\gppoint{gp mark 0}{(3.577,3.535)}
\gppoint{gp mark 0}{(3.577,3.276)}
\gppoint{gp mark 0}{(3.577,3.813)}
\gppoint{gp mark 0}{(3.577,3.010)}
\gppoint{gp mark 0}{(3.577,3.333)}
\gppoint{gp mark 0}{(3.577,3.561)}
\gppoint{gp mark 0}{(3.577,3.561)}
\gppoint{gp mark 0}{(3.577,3.067)}
\gppoint{gp mark 0}{(3.577,2.914)}
\gppoint{gp mark 0}{(3.577,3.094)}
\gppoint{gp mark 0}{(3.577,3.119)}
\gppoint{gp mark 0}{(3.577,3.235)}
\gppoint{gp mark 0}{(3.577,3.010)}
\gppoint{gp mark 0}{(3.577,3.295)}
\gppoint{gp mark 0}{(3.577,3.368)}
\gppoint{gp mark 0}{(3.577,3.144)}
\gppoint{gp mark 0}{(3.577,3.094)}
\gppoint{gp mark 0}{(3.577,3.144)}
\gppoint{gp mark 0}{(3.577,3.434)}
\gppoint{gp mark 0}{(3.577,3.548)}
\gppoint{gp mark 0}{(3.577,3.449)}
\gppoint{gp mark 0}{(3.577,3.010)}
\gppoint{gp mark 0}{(3.577,3.493)}
\gppoint{gp mark 0}{(3.577,3.010)}
\gppoint{gp mark 0}{(3.577,3.368)}
\gppoint{gp mark 0}{(3.577,2.914)}
\gppoint{gp mark 0}{(3.577,3.368)}
\gppoint{gp mark 0}{(3.577,3.119)}
\gppoint{gp mark 0}{(3.577,3.144)}
\gppoint{gp mark 0}{(3.577,2.914)}
\gppoint{gp mark 0}{(3.577,3.385)}
\gppoint{gp mark 0}{(3.577,3.464)}
\gppoint{gp mark 0}{(3.577,3.067)}
\gppoint{gp mark 0}{(3.577,3.235)}
\gppoint{gp mark 0}{(3.577,3.295)}
\gppoint{gp mark 0}{(3.577,3.368)}
\gppoint{gp mark 0}{(3.577,3.144)}
\gppoint{gp mark 0}{(3.577,3.434)}
\gppoint{gp mark 0}{(3.577,3.295)}
\gppoint{gp mark 0}{(3.577,3.434)}
\gppoint{gp mark 0}{(3.577,3.368)}
\gppoint{gp mark 0}{(3.577,3.144)}
\gppoint{gp mark 0}{(3.577,3.368)}
\gppoint{gp mark 0}{(3.577,3.548)}
\gppoint{gp mark 0}{(3.577,3.573)}
\gppoint{gp mark 0}{(3.577,3.144)}
\gppoint{gp mark 0}{(3.577,3.368)}
\gppoint{gp mark 0}{(3.577,3.368)}
\gppoint{gp mark 0}{(3.577,3.010)}
\gppoint{gp mark 0}{(3.577,3.010)}
\gppoint{gp mark 0}{(3.577,3.368)}
\gppoint{gp mark 0}{(3.577,3.368)}
\gppoint{gp mark 0}{(3.577,3.144)}
\gppoint{gp mark 0}{(3.577,3.191)}
\gppoint{gp mark 0}{(3.577,2.914)}
\gppoint{gp mark 0}{(3.577,3.010)}
\gppoint{gp mark 0}{(3.577,3.010)}
\gppoint{gp mark 0}{(3.577,3.368)}
\gppoint{gp mark 0}{(3.577,3.276)}
\gppoint{gp mark 0}{(3.577,3.067)}
\gppoint{gp mark 0}{(3.577,3.010)}
\gppoint{gp mark 0}{(3.577,3.449)}
\gppoint{gp mark 0}{(3.577,3.368)}
\gppoint{gp mark 0}{(3.577,3.094)}
\gppoint{gp mark 0}{(3.577,3.144)}
\gppoint{gp mark 0}{(3.577,3.067)}
\gppoint{gp mark 0}{(3.577,3.094)}
\gppoint{gp mark 0}{(3.577,3.368)}
\gppoint{gp mark 0}{(3.577,3.010)}
\gppoint{gp mark 0}{(3.577,3.295)}
\gppoint{gp mark 0}{(3.577,3.351)}
\gppoint{gp mark 0}{(3.577,3.144)}
\gppoint{gp mark 0}{(3.577,3.255)}
\gppoint{gp mark 0}{(3.577,3.368)}
\gppoint{gp mark 0}{(3.577,3.010)}
\gppoint{gp mark 0}{(3.577,3.168)}
\gppoint{gp mark 0}{(3.577,3.094)}
\gppoint{gp mark 0}{(3.577,3.493)}
\gppoint{gp mark 0}{(3.577,3.479)}
\gppoint{gp mark 0}{(3.577,3.493)}
\gppoint{gp mark 0}{(3.577,3.351)}
\gppoint{gp mark 0}{(3.577,3.213)}
\gppoint{gp mark 0}{(3.577,3.144)}
\gppoint{gp mark 0}{(3.577,3.586)}
\gppoint{gp mark 0}{(3.577,3.667)}
\gppoint{gp mark 0}{(3.577,3.368)}
\gppoint{gp mark 0}{(3.577,3.067)}
\gppoint{gp mark 0}{(3.577,3.255)}
\gppoint{gp mark 0}{(3.577,3.010)}
\gppoint{gp mark 0}{(3.577,3.493)}
\gppoint{gp mark 0}{(3.577,3.493)}
\gppoint{gp mark 0}{(3.577,3.010)}
\gppoint{gp mark 0}{(3.577,3.094)}
\gppoint{gp mark 0}{(3.577,3.213)}
\gppoint{gp mark 0}{(3.577,3.314)}
\gppoint{gp mark 0}{(3.577,3.255)}
\gppoint{gp mark 0}{(3.577,3.010)}
\gppoint{gp mark 0}{(3.577,2.914)}
\gppoint{gp mark 0}{(3.577,3.434)}
\gppoint{gp mark 0}{(3.577,2.914)}
\gppoint{gp mark 0}{(3.577,2.801)}
\gppoint{gp mark 0}{(3.577,3.434)}
\gppoint{gp mark 0}{(3.577,3.368)}
\gppoint{gp mark 0}{(3.577,3.434)}
\gppoint{gp mark 0}{(3.577,3.010)}
\gppoint{gp mark 0}{(3.577,3.449)}
\gppoint{gp mark 0}{(3.577,3.493)}
\gppoint{gp mark 0}{(3.577,3.368)}
\gppoint{gp mark 0}{(3.577,3.010)}
\gppoint{gp mark 0}{(3.577,2.914)}
\gppoint{gp mark 0}{(3.577,3.493)}
\gppoint{gp mark 0}{(3.577,3.368)}
\gppoint{gp mark 0}{(3.577,3.493)}
\gppoint{gp mark 0}{(3.577,3.368)}
\gppoint{gp mark 0}{(3.577,3.434)}
\gppoint{gp mark 0}{(3.577,3.235)}
\gppoint{gp mark 0}{(3.577,3.493)}
\gppoint{gp mark 0}{(3.577,3.368)}
\gppoint{gp mark 0}{(3.577,3.493)}
\gppoint{gp mark 0}{(3.577,3.493)}
\gppoint{gp mark 0}{(3.577,3.368)}
\gppoint{gp mark 0}{(3.577,3.368)}
\gppoint{gp mark 0}{(3.577,3.010)}
\gppoint{gp mark 0}{(3.577,3.144)}
\gppoint{gp mark 0}{(3.577,3.094)}
\gppoint{gp mark 0}{(3.577,3.067)}
\gppoint{gp mark 0}{(3.577,3.434)}
\gppoint{gp mark 0}{(3.577,3.010)}
\gppoint{gp mark 0}{(3.577,3.493)}
\gppoint{gp mark 0}{(3.577,3.213)}
\gppoint{gp mark 0}{(3.577,3.493)}
\gppoint{gp mark 0}{(3.577,3.368)}
\gppoint{gp mark 0}{(3.577,3.493)}
\gppoint{gp mark 0}{(3.577,3.094)}
\gppoint{gp mark 0}{(3.577,3.094)}
\gppoint{gp mark 0}{(3.577,3.493)}
\gppoint{gp mark 0}{(3.577,3.493)}
\gppoint{gp mark 0}{(3.577,3.493)}
\gppoint{gp mark 0}{(3.577,3.434)}
\gppoint{gp mark 0}{(3.577,3.493)}
\gppoint{gp mark 0}{(3.577,3.235)}
\gppoint{gp mark 0}{(3.577,3.493)}
\gppoint{gp mark 0}{(3.577,3.368)}
\gppoint{gp mark 0}{(3.577,3.493)}
\gppoint{gp mark 0}{(3.577,3.634)}
\gppoint{gp mark 0}{(3.577,3.434)}
\gppoint{gp mark 0}{(3.577,3.235)}
\gppoint{gp mark 0}{(3.577,3.493)}
\gppoint{gp mark 0}{(3.577,3.368)}
\gppoint{gp mark 0}{(3.577,3.235)}
\gppoint{gp mark 0}{(3.577,3.535)}
\gppoint{gp mark 0}{(3.577,2.914)}
\gppoint{gp mark 0}{(3.577,3.856)}
\gppoint{gp mark 0}{(3.577,3.235)}
\gppoint{gp mark 0}{(3.577,3.235)}
\gppoint{gp mark 0}{(3.577,3.255)}
\gppoint{gp mark 0}{(3.577,3.168)}
\gppoint{gp mark 0}{(3.577,3.094)}
\gppoint{gp mark 0}{(3.577,3.094)}
\gppoint{gp mark 0}{(3.577,3.276)}
\gppoint{gp mark 0}{(3.577,3.548)}
\gppoint{gp mark 0}{(3.577,3.634)}
\gppoint{gp mark 0}{(3.577,3.191)}
\gppoint{gp mark 0}{(3.577,3.255)}
\gppoint{gp mark 0}{(3.577,3.778)}
\gppoint{gp mark 0}{(3.577,3.191)}
\gppoint{gp mark 0}{(3.577,3.235)}
\gppoint{gp mark 0}{(3.577,2.914)}
\gppoint{gp mark 0}{(3.577,3.368)}
\gppoint{gp mark 0}{(3.577,3.333)}
\gppoint{gp mark 0}{(3.577,3.094)}
\gppoint{gp mark 0}{(3.577,3.368)}
\gppoint{gp mark 0}{(3.577,3.213)}
\gppoint{gp mark 0}{(3.577,3.094)}
\gppoint{gp mark 0}{(3.577,3.402)}
\gppoint{gp mark 0}{(3.577,2.914)}
\gppoint{gp mark 0}{(3.577,3.368)}
\gppoint{gp mark 0}{(3.577,3.434)}
\gppoint{gp mark 0}{(3.577,3.368)}
\gppoint{gp mark 0}{(3.577,3.276)}
\gppoint{gp mark 0}{(3.577,3.464)}
\gppoint{gp mark 0}{(3.577,3.479)}
\gppoint{gp mark 0}{(3.577,3.368)}
\gppoint{gp mark 0}{(3.577,2.801)}
\gppoint{gp mark 0}{(3.577,3.368)}
\gppoint{gp mark 0}{(3.577,3.368)}
\gppoint{gp mark 0}{(3.577,3.368)}
\gppoint{gp mark 0}{(3.577,3.333)}
\gppoint{gp mark 0}{(3.577,3.094)}
\gppoint{gp mark 0}{(3.577,3.368)}
\gppoint{gp mark 0}{(3.577,3.191)}
\gppoint{gp mark 0}{(3.577,3.213)}
\gppoint{gp mark 0}{(3.577,2.801)}
\gppoint{gp mark 0}{(3.577,3.168)}
\gppoint{gp mark 0}{(3.577,2.914)}
\gppoint{gp mark 0}{(3.577,3.368)}
\gppoint{gp mark 0}{(3.577,3.010)}
\gppoint{gp mark 0}{(3.577,3.368)}
\gppoint{gp mark 0}{(3.577,3.368)}
\gppoint{gp mark 0}{(3.577,3.368)}
\gppoint{gp mark 0}{(3.577,3.144)}
\gppoint{gp mark 0}{(3.577,3.464)}
\gppoint{gp mark 0}{(3.577,3.314)}
\gppoint{gp mark 0}{(3.577,3.144)}
\gppoint{gp mark 0}{(3.577,3.144)}
\gppoint{gp mark 0}{(3.577,3.191)}
\gppoint{gp mark 0}{(3.577,3.255)}
\gppoint{gp mark 0}{(3.577,3.094)}
\gppoint{gp mark 0}{(3.577,3.434)}
\gppoint{gp mark 0}{(3.577,3.295)}
\gppoint{gp mark 0}{(3.577,3.402)}
\gppoint{gp mark 0}{(3.577,3.191)}
\gppoint{gp mark 0}{(3.577,3.094)}
\gppoint{gp mark 0}{(3.577,2.914)}
\gppoint{gp mark 0}{(3.577,3.168)}
\gppoint{gp mark 0}{(3.577,2.878)}
\gppoint{gp mark 0}{(3.577,3.144)}
\gppoint{gp mark 0}{(3.577,3.010)}
\gppoint{gp mark 0}{(3.659,3.094)}
\gppoint{gp mark 0}{(3.659,3.144)}
\gppoint{gp mark 0}{(3.659,3.314)}
\gppoint{gp mark 0}{(3.659,3.586)}
\gppoint{gp mark 0}{(3.659,3.903)}
\gppoint{gp mark 0}{(3.659,3.255)}
\gppoint{gp mark 0}{(3.659,3.213)}
\gppoint{gp mark 0}{(3.659,2.914)}
\gppoint{gp mark 0}{(3.659,3.094)}
\gppoint{gp mark 0}{(3.659,3.314)}
\gppoint{gp mark 0}{(3.659,3.094)}
\gppoint{gp mark 0}{(3.659,3.314)}
\gppoint{gp mark 0}{(3.659,3.333)}
\gppoint{gp mark 0}{(3.659,3.333)}
\gppoint{gp mark 0}{(3.659,3.314)}
\gppoint{gp mark 0}{(3.659,3.888)}
\gppoint{gp mark 0}{(3.659,3.507)}
\gppoint{gp mark 0}{(3.659,3.213)}
\gppoint{gp mark 0}{(3.659,3.213)}
\gppoint{gp mark 0}{(3.659,3.561)}
\gppoint{gp mark 0}{(3.659,3.561)}
\gppoint{gp mark 0}{(3.659,3.191)}
\gppoint{gp mark 0}{(3.659,3.368)}
\gppoint{gp mark 0}{(3.659,3.368)}
\gppoint{gp mark 0}{(3.659,3.010)}
\gppoint{gp mark 0}{(3.659,3.333)}
\gppoint{gp mark 0}{(3.659,3.434)}
\gppoint{gp mark 0}{(3.659,3.888)}
\gppoint{gp mark 0}{(3.659,3.168)}
\gppoint{gp mark 0}{(3.659,3.235)}
\gppoint{gp mark 0}{(3.659,3.295)}
\gppoint{gp mark 0}{(3.659,3.039)}
\gppoint{gp mark 0}{(3.659,3.010)}
\gppoint{gp mark 0}{(3.659,3.094)}
\gppoint{gp mark 0}{(3.659,2.914)}
\gppoint{gp mark 0}{(3.659,3.010)}
\gppoint{gp mark 0}{(3.659,3.656)}
\gppoint{gp mark 0}{(3.659,3.213)}
\gppoint{gp mark 0}{(3.659,3.778)}
\gppoint{gp mark 0}{(3.659,3.610)}
\gppoint{gp mark 0}{(3.659,3.213)}
\gppoint{gp mark 0}{(3.659,3.235)}
\gppoint{gp mark 0}{(3.659,3.235)}
\gppoint{gp mark 0}{(3.659,3.418)}
\gppoint{gp mark 0}{(3.659,3.418)}
\gppoint{gp mark 0}{(3.659,3.235)}
\gppoint{gp mark 0}{(3.659,3.235)}
\gppoint{gp mark 0}{(3.659,3.168)}
\gppoint{gp mark 0}{(3.659,2.914)}
\gppoint{gp mark 0}{(3.659,2.665)}
\gppoint{gp mark 0}{(3.659,3.839)}
\gppoint{gp mark 0}{(3.659,3.888)}
\gppoint{gp mark 0}{(3.659,3.255)}
\gppoint{gp mark 0}{(3.659,3.521)}
\gppoint{gp mark 0}{(3.659,3.255)}
\gppoint{gp mark 0}{(3.659,2.914)}
\gppoint{gp mark 0}{(3.659,3.586)}
\gppoint{gp mark 0}{(3.659,2.914)}
\gppoint{gp mark 0}{(3.659,3.235)}
\gppoint{gp mark 0}{(3.659,3.144)}
\gppoint{gp mark 0}{(3.659,3.368)}
\gppoint{gp mark 0}{(3.659,3.168)}
\gppoint{gp mark 0}{(3.659,3.168)}
\gppoint{gp mark 0}{(3.659,3.645)}
\gppoint{gp mark 0}{(3.659,3.168)}
\gppoint{gp mark 0}{(3.659,3.351)}
\gppoint{gp mark 0}{(3.659,3.213)}
\gppoint{gp mark 0}{(3.659,3.333)}
\gppoint{gp mark 0}{(3.659,3.586)}
\gppoint{gp mark 0}{(3.659,3.168)}
\gppoint{gp mark 0}{(3.659,3.295)}
\gppoint{gp mark 0}{(3.659,3.295)}
\gppoint{gp mark 0}{(3.659,3.168)}
\gppoint{gp mark 0}{(3.659,3.213)}
\gppoint{gp mark 0}{(3.659,3.314)}
\gppoint{gp mark 0}{(3.659,3.521)}
\gppoint{gp mark 0}{(3.659,3.213)}
\gppoint{gp mark 0}{(3.659,3.213)}
\gppoint{gp mark 0}{(3.659,3.333)}
\gppoint{gp mark 0}{(3.659,3.235)}
\gppoint{gp mark 0}{(3.659,3.507)}
\gppoint{gp mark 0}{(3.659,2.914)}
\gppoint{gp mark 0}{(3.659,3.295)}
\gppoint{gp mark 0}{(3.659,3.434)}
\gppoint{gp mark 0}{(3.659,3.168)}
\gppoint{gp mark 0}{(3.659,2.979)}
\gppoint{gp mark 0}{(3.659,3.586)}
\gppoint{gp mark 0}{(3.659,3.276)}
\gppoint{gp mark 0}{(3.659,3.276)}
\gppoint{gp mark 0}{(3.659,3.368)}
\gppoint{gp mark 0}{(3.659,2.979)}
\gppoint{gp mark 0}{(3.659,3.561)}
\gppoint{gp mark 0}{(3.659,3.434)}
\gppoint{gp mark 0}{(3.659,2.979)}
\gppoint{gp mark 0}{(3.659,3.276)}
\gppoint{gp mark 0}{(3.659,3.351)}
\gppoint{gp mark 0}{(3.659,3.295)}
\gppoint{gp mark 0}{(3.659,3.385)}
\gppoint{gp mark 0}{(3.659,3.144)}
\gppoint{gp mark 0}{(3.659,3.586)}
\gppoint{gp mark 0}{(3.659,3.213)}
\gppoint{gp mark 0}{(3.659,3.434)}
\gppoint{gp mark 0}{(3.659,2.914)}
\gppoint{gp mark 0}{(3.659,3.749)}
\gppoint{gp mark 0}{(3.659,3.010)}
\gppoint{gp mark 0}{(3.659,3.094)}
\gppoint{gp mark 0}{(3.659,3.418)}
\gppoint{gp mark 0}{(3.659,3.094)}
\gppoint{gp mark 0}{(3.659,3.295)}
\gppoint{gp mark 0}{(3.659,3.010)}
\gppoint{gp mark 0}{(3.659,3.094)}
\gppoint{gp mark 0}{(3.659,3.434)}
\gppoint{gp mark 0}{(3.659,2.801)}
\gppoint{gp mark 0}{(3.659,3.598)}
\gppoint{gp mark 0}{(3.659,3.586)}
\gppoint{gp mark 0}{(3.659,3.255)}
\gppoint{gp mark 0}{(3.659,3.645)}
\gppoint{gp mark 0}{(3.659,3.479)}
\gppoint{gp mark 0}{(3.659,2.914)}
\gppoint{gp mark 0}{(3.659,3.586)}
\gppoint{gp mark 0}{(3.659,3.847)}
\gppoint{gp mark 0}{(3.659,3.586)}
\gppoint{gp mark 0}{(3.659,3.235)}
\gppoint{gp mark 0}{(3.659,3.235)}
\gppoint{gp mark 0}{(3.659,3.276)}
\gppoint{gp mark 0}{(3.659,3.314)}
\gppoint{gp mark 0}{(3.659,3.645)}
\gppoint{gp mark 0}{(3.659,3.094)}
\gppoint{gp mark 0}{(3.659,3.094)}
\gppoint{gp mark 0}{(3.659,3.314)}
\gppoint{gp mark 0}{(3.659,3.255)}
\gppoint{gp mark 0}{(3.659,3.645)}
\gppoint{gp mark 0}{(3.659,3.094)}
\gppoint{gp mark 0}{(3.659,3.434)}
\gppoint{gp mark 0}{(3.659,2.914)}
\gppoint{gp mark 0}{(3.659,3.010)}
\gppoint{gp mark 0}{(3.659,3.094)}
\gppoint{gp mark 0}{(3.659,3.333)}
\gppoint{gp mark 0}{(3.659,3.094)}
\gppoint{gp mark 0}{(3.659,3.586)}
\gppoint{gp mark 0}{(3.659,2.914)}
\gppoint{gp mark 0}{(3.659,3.213)}
\gppoint{gp mark 0}{(3.659,3.634)}
\gppoint{gp mark 0}{(3.659,3.213)}
\gppoint{gp mark 0}{(3.659,3.720)}
\gppoint{gp mark 0}{(3.659,3.010)}
\gppoint{gp mark 0}{(3.659,3.368)}
\gppoint{gp mark 0}{(3.659,3.586)}
\gppoint{gp mark 0}{(3.659,3.434)}
\gppoint{gp mark 0}{(3.659,3.213)}
\gppoint{gp mark 0}{(3.659,4.034)}
\gppoint{gp mark 0}{(3.659,3.368)}
\gppoint{gp mark 0}{(3.659,3.314)}
\gppoint{gp mark 0}{(3.659,3.213)}
\gppoint{gp mark 0}{(3.659,3.213)}
\gppoint{gp mark 0}{(3.659,3.255)}
\gppoint{gp mark 0}{(3.659,3.368)}
\gppoint{gp mark 0}{(3.659,3.255)}
\gppoint{gp mark 0}{(3.659,3.235)}
\gppoint{gp mark 0}{(3.659,3.521)}
\gppoint{gp mark 0}{(3.659,3.094)}
\gppoint{gp mark 0}{(3.659,3.213)}
\gppoint{gp mark 0}{(3.659,3.314)}
\gppoint{gp mark 0}{(3.659,3.010)}
\gppoint{gp mark 0}{(3.659,3.191)}
\gppoint{gp mark 0}{(3.659,3.255)}
\gppoint{gp mark 0}{(3.659,3.586)}
\gppoint{gp mark 0}{(3.659,3.434)}
\gppoint{gp mark 0}{(3.659,3.749)}
\gppoint{gp mark 0}{(3.659,3.255)}
\gppoint{gp mark 0}{(3.659,3.235)}
\gppoint{gp mark 0}{(3.659,3.255)}
\gppoint{gp mark 0}{(3.659,3.119)}
\gppoint{gp mark 0}{(3.659,3.213)}
\gppoint{gp mark 0}{(3.659,3.213)}
\gppoint{gp mark 0}{(3.659,3.213)}
\gppoint{gp mark 0}{(3.659,3.796)}
\gppoint{gp mark 0}{(3.659,3.276)}
\gppoint{gp mark 0}{(3.659,3.255)}
\gppoint{gp mark 0}{(3.659,3.333)}
\gppoint{gp mark 0}{(3.659,3.314)}
\gppoint{gp mark 0}{(3.659,3.255)}
\gppoint{gp mark 0}{(3.659,3.191)}
\gppoint{gp mark 0}{(3.659,3.010)}
\gppoint{gp mark 0}{(3.659,3.119)}
\gppoint{gp mark 0}{(3.659,3.368)}
\gppoint{gp mark 0}{(3.659,3.213)}
\gppoint{gp mark 0}{(3.659,3.720)}
\gppoint{gp mark 0}{(3.659,3.295)}
\gppoint{gp mark 0}{(3.659,3.464)}
\gppoint{gp mark 0}{(3.659,3.094)}
\gppoint{gp mark 0}{(3.659,3.010)}
\gppoint{gp mark 0}{(3.659,3.295)}
\gppoint{gp mark 0}{(3.659,3.586)}
\gppoint{gp mark 0}{(3.659,3.586)}
\gppoint{gp mark 0}{(3.659,3.847)}
\gppoint{gp mark 0}{(3.659,3.333)}
\gppoint{gp mark 0}{(3.659,2.914)}
\gppoint{gp mark 0}{(3.659,3.847)}
\gppoint{gp mark 0}{(3.659,3.010)}
\gppoint{gp mark 0}{(3.659,3.010)}
\gppoint{gp mark 0}{(3.659,3.535)}
\gppoint{gp mark 0}{(3.659,3.094)}
\gppoint{gp mark 0}{(3.659,3.094)}
\gppoint{gp mark 0}{(3.659,3.144)}
\gppoint{gp mark 0}{(3.659,3.968)}
\gppoint{gp mark 0}{(3.659,3.586)}
\gppoint{gp mark 0}{(3.659,3.586)}
\gppoint{gp mark 0}{(3.659,3.144)}
\gppoint{gp mark 0}{(3.659,3.094)}
\gppoint{gp mark 0}{(3.659,3.235)}
\gppoint{gp mark 0}{(3.659,3.645)}
\gppoint{gp mark 0}{(3.659,3.586)}
\gppoint{gp mark 0}{(3.659,3.094)}
\gppoint{gp mark 0}{(3.659,3.295)}
\gppoint{gp mark 0}{(3.659,3.333)}
\gppoint{gp mark 0}{(3.659,3.094)}
\gppoint{gp mark 0}{(3.659,3.010)}
\gppoint{gp mark 0}{(3.659,3.067)}
\gppoint{gp mark 0}{(3.659,3.094)}
\gppoint{gp mark 0}{(3.659,3.144)}
\gppoint{gp mark 0}{(3.659,3.010)}
\gppoint{gp mark 0}{(3.659,3.586)}
\gppoint{gp mark 0}{(3.659,3.586)}
\gppoint{gp mark 0}{(3.659,2.914)}
\gppoint{gp mark 0}{(3.659,2.914)}
\gppoint{gp mark 0}{(3.659,2.979)}
\gppoint{gp mark 0}{(3.659,3.678)}
\gppoint{gp mark 0}{(3.659,3.678)}
\gppoint{gp mark 0}{(3.659,3.678)}
\gppoint{gp mark 0}{(3.659,3.586)}
\gppoint{gp mark 0}{(3.659,3.094)}
\gppoint{gp mark 0}{(3.659,3.678)}
\gppoint{gp mark 0}{(3.659,3.678)}
\gppoint{gp mark 0}{(3.659,3.678)}
\gppoint{gp mark 0}{(3.659,3.144)}
\gppoint{gp mark 0}{(3.659,3.586)}
\gppoint{gp mark 0}{(3.659,3.678)}
\gppoint{gp mark 0}{(3.659,3.521)}
\gppoint{gp mark 0}{(3.659,2.914)}
\gppoint{gp mark 0}{(3.659,3.678)}
\gppoint{gp mark 0}{(3.659,3.168)}
\gppoint{gp mark 0}{(3.659,3.039)}
\gppoint{gp mark 0}{(3.659,3.678)}
\gppoint{gp mark 0}{(3.659,3.678)}
\gppoint{gp mark 0}{(3.659,3.168)}
\gppoint{gp mark 0}{(3.659,3.645)}
\gppoint{gp mark 0}{(3.659,3.586)}
\gppoint{gp mark 0}{(3.659,3.678)}
\gppoint{gp mark 0}{(3.659,3.094)}
\gppoint{gp mark 0}{(3.659,2.914)}
\gppoint{gp mark 0}{(3.659,3.678)}
\gppoint{gp mark 0}{(3.659,3.678)}
\gppoint{gp mark 0}{(3.659,3.586)}
\gppoint{gp mark 0}{(3.659,3.295)}
\gppoint{gp mark 0}{(3.659,3.276)}
\gppoint{gp mark 0}{(3.659,2.914)}
\gppoint{gp mark 0}{(3.659,2.914)}
\gppoint{gp mark 0}{(3.659,3.586)}
\gppoint{gp mark 0}{(3.659,3.586)}
\gppoint{gp mark 0}{(3.659,3.610)}
\gppoint{gp mark 0}{(3.659,3.010)}
\gppoint{gp mark 0}{(3.659,3.235)}
\gppoint{gp mark 0}{(3.659,3.610)}
\gppoint{gp mark 0}{(3.659,3.255)}
\gppoint{gp mark 0}{(3.659,3.610)}
\gppoint{gp mark 0}{(3.659,3.586)}
\gppoint{gp mark 0}{(3.659,3.144)}
\gppoint{gp mark 0}{(3.659,3.010)}
\gppoint{gp mark 0}{(3.659,3.787)}
\gppoint{gp mark 0}{(3.659,3.645)}
\gppoint{gp mark 0}{(3.659,3.787)}
\gppoint{gp mark 0}{(3.659,3.144)}
\gppoint{gp mark 0}{(3.659,3.787)}
\gppoint{gp mark 0}{(3.659,3.787)}
\gppoint{gp mark 0}{(3.659,3.586)}
\gppoint{gp mark 0}{(3.659,3.235)}
\gppoint{gp mark 0}{(3.659,3.010)}
\gppoint{gp mark 0}{(3.659,3.787)}
\gppoint{gp mark 0}{(3.659,3.787)}
\gppoint{gp mark 0}{(3.659,3.787)}
\gppoint{gp mark 0}{(3.659,3.787)}
\gppoint{gp mark 0}{(3.659,3.586)}
\gppoint{gp mark 0}{(3.659,3.144)}
\gppoint{gp mark 0}{(3.659,3.010)}
\gppoint{gp mark 0}{(3.659,3.144)}
\gppoint{gp mark 0}{(3.659,3.191)}
\gppoint{gp mark 0}{(3.659,3.507)}
\gppoint{gp mark 0}{(3.659,3.586)}
\gppoint{gp mark 0}{(3.659,3.276)}
\gppoint{gp mark 0}{(3.659,3.094)}
\gppoint{gp mark 0}{(3.659,3.276)}
\gppoint{gp mark 0}{(3.659,3.368)}
\gppoint{gp mark 0}{(3.659,3.094)}
\gppoint{gp mark 0}{(3.659,3.573)}
\gppoint{gp mark 0}{(3.659,3.535)}
\gppoint{gp mark 0}{(3.659,3.191)}
\gppoint{gp mark 0}{(3.659,3.144)}
\gppoint{gp mark 0}{(3.659,3.144)}
\gppoint{gp mark 0}{(3.659,3.535)}
\gppoint{gp mark 0}{(3.659,3.144)}
\gppoint{gp mark 0}{(3.659,3.191)}
\gppoint{gp mark 0}{(3.659,3.010)}
\gppoint{gp mark 0}{(3.659,3.255)}
\gppoint{gp mark 0}{(3.659,3.213)}
\gppoint{gp mark 0}{(3.659,3.191)}
\gppoint{gp mark 0}{(3.659,3.586)}
\gppoint{gp mark 0}{(3.659,3.645)}
\gppoint{gp mark 0}{(3.659,3.168)}
\gppoint{gp mark 0}{(3.659,3.213)}
\gppoint{gp mark 0}{(3.659,3.586)}
\gppoint{gp mark 0}{(3.659,3.586)}
\gppoint{gp mark 0}{(3.659,3.010)}
\gppoint{gp mark 0}{(3.659,3.255)}
\gppoint{gp mark 0}{(3.659,3.479)}
\gppoint{gp mark 0}{(3.659,3.586)}
\gppoint{gp mark 0}{(3.659,3.586)}
\gppoint{gp mark 0}{(3.659,3.168)}
\gppoint{gp mark 0}{(3.659,3.548)}
\gppoint{gp mark 0}{(3.659,3.573)}
\gppoint{gp mark 0}{(3.659,3.094)}
\gppoint{gp mark 0}{(3.659,3.586)}
\gppoint{gp mark 0}{(3.659,3.678)}
\gppoint{gp mark 0}{(3.659,3.094)}
\gppoint{gp mark 0}{(3.659,3.213)}
\gppoint{gp mark 0}{(3.659,3.168)}
\gppoint{gp mark 0}{(3.659,3.276)}
\gppoint{gp mark 0}{(3.659,3.094)}
\gppoint{gp mark 0}{(3.659,3.213)}
\gppoint{gp mark 0}{(3.659,3.144)}
\gppoint{gp mark 0}{(3.659,3.434)}
\gppoint{gp mark 0}{(3.659,3.276)}
\gppoint{gp mark 0}{(3.659,3.586)}
\gppoint{gp mark 0}{(3.659,3.213)}
\gppoint{gp mark 0}{(3.659,2.914)}
\gppoint{gp mark 0}{(3.659,3.295)}
\gppoint{gp mark 0}{(3.659,3.295)}
\gppoint{gp mark 0}{(3.659,3.678)}
\gppoint{gp mark 0}{(3.659,3.333)}
\gppoint{gp mark 0}{(3.659,3.645)}
\gppoint{gp mark 0}{(3.659,3.759)}
\gppoint{gp mark 0}{(3.659,3.295)}
\gppoint{gp mark 0}{(3.659,3.586)}
\gppoint{gp mark 0}{(3.659,3.168)}
\gppoint{gp mark 0}{(3.659,3.144)}
\gppoint{gp mark 0}{(3.659,3.667)}
\gppoint{gp mark 0}{(3.659,3.144)}
\gppoint{gp mark 0}{(3.659,3.385)}
\gppoint{gp mark 0}{(3.659,3.276)}
\gppoint{gp mark 0}{(3.659,3.213)}
\gppoint{gp mark 0}{(3.659,3.213)}
\gppoint{gp mark 0}{(3.659,3.418)}
\gppoint{gp mark 0}{(3.659,3.213)}
\gppoint{gp mark 0}{(3.659,3.535)}
\gppoint{gp mark 0}{(3.659,3.622)}
\gppoint{gp mark 0}{(3.659,3.213)}
\gppoint{gp mark 0}{(3.659,3.094)}
\gppoint{gp mark 0}{(3.659,3.385)}
\gppoint{gp mark 0}{(3.659,3.010)}
\gppoint{gp mark 0}{(3.659,3.645)}
\gppoint{gp mark 0}{(3.659,3.535)}
\gppoint{gp mark 0}{(3.659,3.535)}
\gppoint{gp mark 0}{(3.659,3.622)}
\gppoint{gp mark 0}{(3.659,3.295)}
\gppoint{gp mark 0}{(3.659,3.333)}
\gppoint{gp mark 0}{(3.659,3.333)}
\gppoint{gp mark 0}{(3.659,3.586)}
\gppoint{gp mark 0}{(3.659,3.213)}
\gppoint{gp mark 0}{(3.659,3.586)}
\gppoint{gp mark 0}{(3.659,3.213)}
\gppoint{gp mark 0}{(3.659,3.213)}
\gppoint{gp mark 0}{(3.659,3.213)}
\gppoint{gp mark 0}{(3.659,3.213)}
\gppoint{gp mark 0}{(3.659,3.507)}
\gppoint{gp mark 0}{(3.659,3.622)}
\gppoint{gp mark 0}{(3.659,3.586)}
\gppoint{gp mark 0}{(3.659,3.622)}
\gppoint{gp mark 0}{(3.659,3.586)}
\gppoint{gp mark 0}{(3.659,3.351)}
\gppoint{gp mark 0}{(3.659,3.213)}
\gppoint{gp mark 0}{(3.659,3.276)}
\gppoint{gp mark 0}{(3.659,3.479)}
\gppoint{gp mark 0}{(3.659,3.295)}
\gppoint{gp mark 0}{(3.659,3.351)}
\gppoint{gp mark 0}{(3.659,3.586)}
\gppoint{gp mark 0}{(3.659,3.351)}
\gppoint{gp mark 0}{(3.659,3.351)}
\gppoint{gp mark 0}{(3.659,3.314)}
\gppoint{gp mark 0}{(3.659,3.464)}
\gppoint{gp mark 0}{(3.659,3.351)}
\gppoint{gp mark 0}{(3.659,3.144)}
\gppoint{gp mark 0}{(3.659,2.948)}
\gppoint{gp mark 0}{(3.659,3.464)}
\gppoint{gp mark 0}{(3.659,3.144)}
\gppoint{gp mark 0}{(3.659,3.144)}
\gppoint{gp mark 0}{(3.659,3.168)}
\gppoint{gp mark 0}{(3.659,3.276)}
\gppoint{gp mark 0}{(3.659,3.333)}
\gppoint{gp mark 0}{(3.659,3.213)}
\gppoint{gp mark 0}{(3.659,3.213)}
\gppoint{gp mark 0}{(3.659,3.094)}
\gppoint{gp mark 0}{(3.659,3.213)}
\gppoint{gp mark 0}{(3.659,3.235)}
\gppoint{gp mark 0}{(3.659,3.586)}
\gppoint{gp mark 0}{(3.659,3.235)}
\gppoint{gp mark 0}{(3.659,3.418)}
\gppoint{gp mark 0}{(3.659,3.235)}
\gppoint{gp mark 0}{(3.659,3.586)}
\gppoint{gp mark 0}{(3.659,3.144)}
\gppoint{gp mark 0}{(3.659,3.235)}
\gppoint{gp mark 0}{(3.659,3.010)}
\gppoint{gp mark 0}{(3.659,3.645)}
\gppoint{gp mark 0}{(3.659,3.094)}
\gppoint{gp mark 0}{(3.659,2.914)}
\gppoint{gp mark 0}{(3.659,3.295)}
\gppoint{gp mark 0}{(3.659,3.385)}
\gppoint{gp mark 0}{(3.659,3.010)}
\gppoint{gp mark 0}{(3.659,3.010)}
\gppoint{gp mark 0}{(3.659,3.464)}
\gppoint{gp mark 0}{(3.659,3.586)}
\gppoint{gp mark 0}{(3.659,3.295)}
\gppoint{gp mark 0}{(3.659,3.295)}
\gppoint{gp mark 0}{(3.659,3.235)}
\gppoint{gp mark 0}{(3.659,3.168)}
\gppoint{gp mark 0}{(3.659,3.213)}
\gppoint{gp mark 0}{(3.659,3.168)}
\gppoint{gp mark 0}{(3.659,3.368)}
\gppoint{gp mark 0}{(3.659,3.645)}
\gppoint{gp mark 0}{(3.659,3.368)}
\gppoint{gp mark 0}{(3.659,2.801)}
\gppoint{gp mark 0}{(3.659,3.168)}
\gppoint{gp mark 0}{(3.659,3.645)}
\gppoint{gp mark 0}{(3.659,3.586)}
\gppoint{gp mark 0}{(3.659,3.010)}
\gppoint{gp mark 0}{(3.659,3.314)}
\gppoint{gp mark 0}{(3.659,3.094)}
\gppoint{gp mark 0}{(3.659,3.094)}
\gppoint{gp mark 0}{(3.659,3.276)}
\gppoint{gp mark 0}{(3.659,2.914)}
\gppoint{gp mark 0}{(3.659,3.586)}
\gppoint{gp mark 0}{(3.659,3.418)}
\gppoint{gp mark 0}{(3.659,3.586)}
\gppoint{gp mark 0}{(3.659,3.276)}
\gppoint{gp mark 0}{(3.659,3.434)}
\gppoint{gp mark 0}{(3.659,3.351)}
\gppoint{gp mark 0}{(3.659,3.434)}
\gppoint{gp mark 0}{(3.659,3.094)}
\gppoint{gp mark 0}{(3.659,3.351)}
\gppoint{gp mark 0}{(3.659,3.094)}
\gppoint{gp mark 0}{(3.659,3.010)}
\gppoint{gp mark 0}{(3.659,3.586)}
\gppoint{gp mark 0}{(3.659,3.402)}
\gppoint{gp mark 0}{(3.659,3.548)}
\gppoint{gp mark 0}{(3.659,3.535)}
\gppoint{gp mark 0}{(3.659,3.418)}
\gppoint{gp mark 0}{(3.659,3.094)}
\gppoint{gp mark 0}{(3.659,3.678)}
\gppoint{gp mark 0}{(3.659,3.094)}
\gppoint{gp mark 0}{(3.659,3.094)}
\gppoint{gp mark 0}{(3.659,3.094)}
\gppoint{gp mark 0}{(3.659,3.314)}
\gppoint{gp mark 0}{(3.659,3.586)}
\gppoint{gp mark 0}{(3.659,3.314)}
\gppoint{gp mark 0}{(3.659,3.645)}
\gppoint{gp mark 0}{(3.659,3.067)}
\gppoint{gp mark 0}{(3.659,3.276)}
\gppoint{gp mark 0}{(3.659,3.276)}
\gppoint{gp mark 0}{(3.659,3.213)}
\gppoint{gp mark 0}{(3.659,3.295)}
\gppoint{gp mark 0}{(3.659,3.610)}
\gppoint{gp mark 0}{(3.659,2.914)}
\gppoint{gp mark 0}{(3.659,3.561)}
\gppoint{gp mark 0}{(3.659,3.168)}
\gppoint{gp mark 0}{(3.659,3.144)}
\gppoint{gp mark 0}{(3.659,3.622)}
\gppoint{gp mark 0}{(3.659,3.255)}
\gppoint{gp mark 0}{(3.659,3.276)}
\gppoint{gp mark 0}{(3.659,3.689)}
\gppoint{gp mark 0}{(3.659,3.479)}
\gppoint{gp mark 0}{(3.659,3.168)}
\gppoint{gp mark 0}{(3.659,3.314)}
\gppoint{gp mark 0}{(3.659,3.276)}
\gppoint{gp mark 0}{(3.659,3.586)}
\gppoint{gp mark 0}{(3.659,3.276)}
\gppoint{gp mark 0}{(3.659,3.656)}
\gppoint{gp mark 0}{(3.659,3.656)}
\gppoint{gp mark 0}{(3.659,3.535)}
\gppoint{gp mark 0}{(3.659,3.561)}
\gppoint{gp mark 0}{(3.659,3.213)}
\gppoint{gp mark 0}{(3.659,3.168)}
\gppoint{gp mark 0}{(3.659,3.213)}
\gppoint{gp mark 0}{(3.659,3.010)}
\gppoint{gp mark 0}{(3.659,3.094)}
\gppoint{gp mark 0}{(3.659,3.094)}
\gppoint{gp mark 0}{(3.659,3.561)}
\gppoint{gp mark 0}{(3.659,3.094)}
\gppoint{gp mark 0}{(3.659,3.094)}
\gppoint{gp mark 0}{(3.659,3.351)}
\gppoint{gp mark 0}{(3.659,3.645)}
\gppoint{gp mark 0}{(3.659,3.351)}
\gppoint{gp mark 0}{(3.659,3.351)}
\gppoint{gp mark 0}{(3.659,3.351)}
\gppoint{gp mark 0}{(3.659,3.351)}
\gppoint{gp mark 0}{(3.659,3.351)}
\gppoint{gp mark 0}{(3.659,2.914)}
\gppoint{gp mark 0}{(3.659,3.314)}
\gppoint{gp mark 0}{(3.659,3.213)}
\gppoint{gp mark 0}{(3.659,3.586)}
\gppoint{gp mark 0}{(3.659,3.586)}
\gppoint{gp mark 0}{(3.659,3.521)}
\gppoint{gp mark 0}{(3.659,3.521)}
\gppoint{gp mark 0}{(3.659,3.168)}
\gppoint{gp mark 0}{(3.659,3.213)}
\gppoint{gp mark 0}{(3.659,3.418)}
\gppoint{gp mark 0}{(3.659,3.168)}
\gppoint{gp mark 0}{(3.659,3.586)}
\gppoint{gp mark 0}{(3.659,3.144)}
\gppoint{gp mark 0}{(3.659,3.586)}
\gppoint{gp mark 0}{(3.659,3.255)}
\gppoint{gp mark 0}{(3.659,3.213)}
\gppoint{gp mark 0}{(3.659,3.235)}
\gppoint{gp mark 0}{(3.659,3.213)}
\gppoint{gp mark 0}{(3.659,3.610)}
\gppoint{gp mark 0}{(3.734,3.351)}
\gppoint{gp mark 0}{(3.734,3.213)}
\gppoint{gp mark 0}{(3.734,2.914)}
\gppoint{gp mark 0}{(3.734,3.464)}
\gppoint{gp mark 0}{(3.734,3.368)}
\gppoint{gp mark 0}{(3.734,3.926)}
\gppoint{gp mark 0}{(3.734,3.730)}
\gppoint{gp mark 0}{(3.734,3.368)}
\gppoint{gp mark 0}{(3.734,3.622)}
\gppoint{gp mark 0}{(3.734,3.368)}
\gppoint{gp mark 0}{(3.734,3.295)}
\gppoint{gp mark 0}{(3.734,3.368)}
\gppoint{gp mark 0}{(3.734,3.368)}
\gppoint{gp mark 0}{(3.734,3.295)}
\gppoint{gp mark 0}{(3.734,3.295)}
\gppoint{gp mark 0}{(3.734,3.094)}
\gppoint{gp mark 0}{(3.734,3.094)}
\gppoint{gp mark 0}{(3.734,3.094)}
\gppoint{gp mark 0}{(3.734,3.094)}
\gppoint{gp mark 0}{(3.734,3.094)}
\gppoint{gp mark 0}{(3.734,3.094)}
\gppoint{gp mark 0}{(3.734,3.493)}
\gppoint{gp mark 0}{(3.734,3.385)}
\gppoint{gp mark 0}{(3.734,3.548)}
\gppoint{gp mark 0}{(3.734,3.493)}
\gppoint{gp mark 0}{(3.734,3.295)}
\gppoint{gp mark 0}{(3.734,3.548)}
\gppoint{gp mark 0}{(3.734,3.548)}
\gppoint{gp mark 0}{(3.734,3.235)}
\gppoint{gp mark 0}{(3.734,4.040)}
\gppoint{gp mark 0}{(3.734,3.351)}
\gppoint{gp mark 0}{(3.734,3.464)}
\gppoint{gp mark 0}{(3.734,3.295)}
\gppoint{gp mark 0}{(3.734,3.094)}
\gppoint{gp mark 0}{(3.734,3.368)}
\gppoint{gp mark 0}{(3.734,3.213)}
\gppoint{gp mark 0}{(3.734,3.368)}
\gppoint{gp mark 0}{(3.734,3.333)}
\gppoint{gp mark 0}{(3.734,3.094)}
\gppoint{gp mark 0}{(3.734,3.168)}
\gppoint{gp mark 0}{(3.734,3.168)}
\gppoint{gp mark 0}{(3.734,3.464)}
\gppoint{gp mark 0}{(3.734,3.464)}
\gppoint{gp mark 0}{(3.734,3.094)}
\gppoint{gp mark 0}{(3.734,3.598)}
\gppoint{gp mark 0}{(3.734,3.507)}
\gppoint{gp mark 0}{(3.734,3.573)}
\gppoint{gp mark 0}{(3.734,3.449)}
\gppoint{gp mark 0}{(3.734,3.645)}
\gppoint{gp mark 0}{(3.734,3.333)}
\gppoint{gp mark 0}{(3.734,3.213)}
\gppoint{gp mark 0}{(3.734,3.813)}
\gppoint{gp mark 0}{(3.734,3.622)}
\gppoint{gp mark 0}{(3.734,3.634)}
\gppoint{gp mark 0}{(3.734,3.235)}
\gppoint{gp mark 0}{(3.734,3.235)}
\gppoint{gp mark 0}{(3.734,3.295)}
\gppoint{gp mark 0}{(3.734,3.235)}
\gppoint{gp mark 0}{(3.734,3.831)}
\gppoint{gp mark 0}{(3.734,3.213)}
\gppoint{gp mark 0}{(3.734,3.213)}
\gppoint{gp mark 0}{(3.734,3.168)}
\gppoint{gp mark 0}{(3.734,3.561)}
\gppoint{gp mark 0}{(3.734,3.213)}
\gppoint{gp mark 0}{(3.734,3.213)}
\gppoint{gp mark 0}{(3.734,3.276)}
\gppoint{gp mark 0}{(3.734,3.449)}
\gppoint{gp mark 0}{(3.734,3.351)}
\gppoint{gp mark 0}{(3.734,3.351)}
\gppoint{gp mark 0}{(3.734,3.678)}
\gppoint{gp mark 0}{(3.734,3.094)}
\gppoint{gp mark 0}{(3.734,3.213)}
\gppoint{gp mark 0}{(3.734,3.368)}
\gppoint{gp mark 0}{(3.734,3.295)}
\gppoint{gp mark 0}{(3.734,3.094)}
\gppoint{gp mark 0}{(3.734,2.914)}
\gppoint{gp mark 0}{(3.734,3.434)}
\gppoint{gp mark 0}{(3.734,3.010)}
\gppoint{gp mark 0}{(3.734,3.656)}
\gppoint{gp mark 0}{(3.734,3.213)}
\gppoint{gp mark 0}{(3.734,3.213)}
\gppoint{gp mark 0}{(3.734,3.521)}
\gppoint{gp mark 0}{(3.734,3.144)}
\gppoint{gp mark 0}{(3.734,3.521)}
\gppoint{gp mark 0}{(3.734,3.235)}
\gppoint{gp mark 0}{(3.734,3.276)}
\gppoint{gp mark 0}{(3.734,3.333)}
\gppoint{gp mark 0}{(3.734,3.548)}
\gppoint{gp mark 0}{(3.734,3.385)}
\gppoint{gp mark 0}{(3.734,3.295)}
\gppoint{gp mark 0}{(3.734,3.213)}
\gppoint{gp mark 0}{(3.734,3.368)}
\gppoint{gp mark 0}{(3.734,3.295)}
\gppoint{gp mark 0}{(3.734,3.368)}
\gppoint{gp mark 0}{(3.734,3.039)}
\gppoint{gp mark 0}{(3.734,3.385)}
\gppoint{gp mark 0}{(3.734,3.094)}
\gppoint{gp mark 0}{(3.734,3.235)}
\gppoint{gp mark 0}{(3.734,3.385)}
\gppoint{gp mark 0}{(3.734,3.010)}
\gppoint{gp mark 0}{(3.734,3.010)}
\gppoint{gp mark 0}{(3.734,3.368)}
\gppoint{gp mark 0}{(3.734,3.434)}
\gppoint{gp mark 0}{(3.734,3.235)}
\gppoint{gp mark 0}{(3.734,3.168)}
\gppoint{gp mark 0}{(3.734,3.094)}
\gppoint{gp mark 0}{(3.734,3.434)}
\gppoint{gp mark 0}{(3.734,3.796)}
\gppoint{gp mark 0}{(3.734,3.385)}
\gppoint{gp mark 0}{(3.734,3.235)}
\gppoint{gp mark 0}{(3.734,3.434)}
\gppoint{gp mark 0}{(3.734,3.796)}
\gppoint{gp mark 0}{(3.734,3.010)}
\gppoint{gp mark 0}{(3.734,3.094)}
\gppoint{gp mark 0}{(3.734,3.368)}
\gppoint{gp mark 0}{(3.734,3.548)}
\gppoint{gp mark 0}{(3.734,3.168)}
\gppoint{gp mark 0}{(3.734,3.094)}
\gppoint{gp mark 0}{(3.734,3.235)}
\gppoint{gp mark 0}{(3.734,3.276)}
\gppoint{gp mark 0}{(3.734,3.368)}
\gppoint{gp mark 0}{(3.734,3.276)}
\gppoint{gp mark 0}{(3.734,3.402)}
\gppoint{gp mark 0}{(3.734,3.067)}
\gppoint{gp mark 0}{(3.734,3.235)}
\gppoint{gp mark 0}{(3.734,2.914)}
\gppoint{gp mark 0}{(3.734,3.418)}
\gppoint{gp mark 0}{(3.734,3.168)}
\gppoint{gp mark 0}{(3.734,3.598)}
\gppoint{gp mark 0}{(3.734,3.094)}
\gppoint{gp mark 0}{(3.734,3.067)}
\gppoint{gp mark 0}{(3.734,3.276)}
\gppoint{gp mark 0}{(3.734,3.667)}
\gppoint{gp mark 0}{(3.734,3.598)}
\gppoint{gp mark 0}{(3.734,4.021)}
\gppoint{gp mark 0}{(3.734,3.295)}
\gppoint{gp mark 0}{(3.734,3.598)}
\gppoint{gp mark 0}{(3.734,3.740)}
\gppoint{gp mark 0}{(3.734,3.168)}
\gppoint{gp mark 0}{(3.734,3.235)}
\gppoint{gp mark 0}{(3.734,3.768)}
\gppoint{gp mark 0}{(3.734,3.548)}
\gppoint{gp mark 0}{(3.734,3.740)}
\gppoint{gp mark 0}{(3.734,3.720)}
\gppoint{gp mark 0}{(3.734,3.094)}
\gppoint{gp mark 0}{(3.734,3.168)}
\gppoint{gp mark 0}{(3.734,3.094)}
\gppoint{gp mark 0}{(3.734,3.010)}
\gppoint{gp mark 0}{(3.734,3.235)}
\gppoint{gp mark 0}{(3.734,3.295)}
\gppoint{gp mark 0}{(3.734,3.010)}
\gppoint{gp mark 0}{(3.734,3.598)}
\gppoint{gp mark 0}{(3.734,3.010)}
\gppoint{gp mark 0}{(3.734,3.295)}
\gppoint{gp mark 0}{(3.734,3.402)}
\gppoint{gp mark 0}{(3.734,3.434)}
\gppoint{gp mark 0}{(3.734,3.622)}
\gppoint{gp mark 0}{(3.734,3.094)}
\gppoint{gp mark 0}{(3.734,3.548)}
\gppoint{gp mark 0}{(3.734,3.213)}
\gppoint{gp mark 0}{(3.734,3.168)}
\gppoint{gp mark 0}{(3.734,3.010)}
\gppoint{gp mark 0}{(3.734,3.276)}
\gppoint{gp mark 0}{(3.734,3.678)}
\gppoint{gp mark 0}{(3.734,3.295)}
\gppoint{gp mark 0}{(3.734,3.213)}
\gppoint{gp mark 0}{(3.734,3.276)}
\gppoint{gp mark 0}{(3.734,3.351)}
\gppoint{gp mark 0}{(3.734,3.276)}
\gppoint{gp mark 0}{(3.734,3.094)}
\gppoint{gp mark 0}{(3.734,3.368)}
\gppoint{gp mark 0}{(3.734,4.272)}
\gppoint{gp mark 0}{(3.734,3.598)}
\gppoint{gp mark 0}{(3.734,3.295)}
\gppoint{gp mark 0}{(3.734,3.434)}
\gppoint{gp mark 0}{(3.734,3.168)}
\gppoint{gp mark 0}{(3.734,3.434)}
\gppoint{gp mark 0}{(3.734,3.010)}
\gppoint{gp mark 0}{(3.734,3.010)}
\gppoint{gp mark 0}{(3.734,3.094)}
\gppoint{gp mark 0}{(3.734,3.094)}
\gppoint{gp mark 0}{(3.734,3.191)}
\gppoint{gp mark 0}{(3.734,3.656)}
\gppoint{gp mark 0}{(3.734,3.276)}
\gppoint{gp mark 0}{(3.734,2.979)}
\gppoint{gp mark 0}{(3.734,3.333)}
\gppoint{gp mark 0}{(3.734,3.940)}
\gppoint{gp mark 0}{(3.734,3.295)}
\gppoint{gp mark 0}{(3.734,3.295)}
\gppoint{gp mark 0}{(3.734,3.067)}
\gppoint{gp mark 0}{(3.734,3.168)}
\gppoint{gp mark 0}{(3.734,3.010)}
\gppoint{gp mark 0}{(3.734,3.434)}
\gppoint{gp mark 0}{(3.734,3.295)}
\gppoint{gp mark 0}{(3.734,3.094)}
\gppoint{gp mark 0}{(3.734,3.276)}
\gppoint{gp mark 0}{(3.734,3.295)}
\gppoint{gp mark 0}{(3.734,3.493)}
\gppoint{gp mark 0}{(3.734,3.295)}
\gppoint{gp mark 0}{(3.734,3.213)}
\gppoint{gp mark 0}{(3.734,3.464)}
\gppoint{gp mark 0}{(3.734,3.385)}
\gppoint{gp mark 0}{(3.734,3.548)}
\gppoint{gp mark 0}{(3.734,3.276)}
\gppoint{gp mark 0}{(3.734,3.995)}
\gppoint{gp mark 0}{(3.734,3.276)}
\gppoint{gp mark 0}{(3.734,3.010)}
\gppoint{gp mark 0}{(3.734,3.507)}
\gppoint{gp mark 0}{(3.734,3.464)}
\gppoint{gp mark 0}{(3.734,2.914)}
\gppoint{gp mark 0}{(3.734,3.740)}
\gppoint{gp mark 0}{(3.734,3.507)}
\gppoint{gp mark 0}{(3.734,3.449)}
\gppoint{gp mark 0}{(3.734,3.168)}
\gppoint{gp mark 0}{(3.734,3.010)}
\gppoint{gp mark 0}{(3.734,3.144)}
\gppoint{gp mark 0}{(3.734,3.333)}
\gppoint{gp mark 0}{(3.734,3.010)}
\gppoint{gp mark 0}{(3.734,3.010)}
\gppoint{gp mark 0}{(3.734,3.995)}
\gppoint{gp mark 0}{(3.734,3.295)}
\gppoint{gp mark 0}{(3.734,3.295)}
\gppoint{gp mark 0}{(3.734,3.295)}
\gppoint{gp mark 0}{(3.734,3.168)}
\gppoint{gp mark 0}{(3.734,3.586)}
\gppoint{gp mark 0}{(3.734,3.787)}
\gppoint{gp mark 0}{(3.734,3.276)}
\gppoint{gp mark 0}{(3.734,3.493)}
\gppoint{gp mark 0}{(3.734,3.479)}
\gppoint{gp mark 0}{(3.734,3.168)}
\gppoint{gp mark 0}{(3.734,3.903)}
\gppoint{gp mark 0}{(3.734,3.010)}
\gppoint{gp mark 0}{(3.734,3.094)}
\gppoint{gp mark 0}{(3.734,3.235)}
\gppoint{gp mark 0}{(3.734,3.464)}
\gppoint{gp mark 0}{(3.734,3.418)}
\gppoint{gp mark 0}{(3.734,3.434)}
\gppoint{gp mark 0}{(3.734,3.255)}
\gppoint{gp mark 0}{(3.734,3.295)}
\gppoint{gp mark 0}{(3.734,3.295)}
\gppoint{gp mark 0}{(3.734,3.144)}
\gppoint{gp mark 0}{(3.734,3.213)}
\gppoint{gp mark 0}{(3.734,3.010)}
\gppoint{gp mark 0}{(3.734,3.235)}
\gppoint{gp mark 0}{(3.734,3.010)}
\gppoint{gp mark 0}{(3.734,3.010)}
\gppoint{gp mark 0}{(3.734,3.094)}
\gppoint{gp mark 0}{(3.734,3.548)}
\gppoint{gp mark 0}{(3.734,3.699)}
\gppoint{gp mark 0}{(3.734,3.094)}
\gppoint{gp mark 0}{(3.734,3.548)}
\gppoint{gp mark 0}{(3.734,3.235)}
\gppoint{gp mark 0}{(3.734,3.094)}
\gppoint{gp mark 0}{(3.734,3.418)}
\gppoint{gp mark 0}{(3.734,3.295)}
\gppoint{gp mark 0}{(3.734,3.507)}
\gppoint{gp mark 0}{(3.734,3.010)}
\gppoint{gp mark 0}{(3.734,3.094)}
\gppoint{gp mark 0}{(3.734,3.276)}
\gppoint{gp mark 0}{(3.734,3.295)}
\gppoint{gp mark 0}{(3.734,3.888)}
\gppoint{gp mark 0}{(3.734,3.535)}
\gppoint{gp mark 0}{(3.734,3.094)}
\gppoint{gp mark 0}{(3.734,3.507)}
\gppoint{gp mark 0}{(3.734,3.548)}
\gppoint{gp mark 0}{(3.734,3.479)}
\gppoint{gp mark 0}{(3.734,3.295)}
\gppoint{gp mark 0}{(3.734,3.548)}
\gppoint{gp mark 0}{(3.734,3.548)}
\gppoint{gp mark 0}{(3.734,3.385)}
\gppoint{gp mark 0}{(3.734,3.385)}
\gppoint{gp mark 0}{(3.734,3.699)}
\gppoint{gp mark 0}{(3.734,3.385)}
\gppoint{gp mark 0}{(3.734,3.368)}
\gppoint{gp mark 0}{(3.734,3.464)}
\gppoint{gp mark 0}{(3.734,4.058)}
\gppoint{gp mark 0}{(3.734,3.493)}
\gppoint{gp mark 0}{(3.734,3.276)}
\gppoint{gp mark 0}{(3.734,3.385)}
\gppoint{gp mark 0}{(3.734,3.094)}
\gppoint{gp mark 0}{(3.734,3.521)}
\gppoint{gp mark 0}{(3.734,3.573)}
\gppoint{gp mark 0}{(3.734,3.880)}
\gppoint{gp mark 0}{(3.734,3.385)}
\gppoint{gp mark 0}{(3.734,3.351)}
\gppoint{gp mark 0}{(3.734,3.368)}
\gppoint{gp mark 0}{(3.734,3.213)}
\gppoint{gp mark 0}{(3.734,3.449)}
\gppoint{gp mark 0}{(3.734,3.573)}
\gppoint{gp mark 0}{(3.734,3.144)}
\gppoint{gp mark 0}{(3.734,3.235)}
\gppoint{gp mark 0}{(3.734,3.368)}
\gppoint{gp mark 0}{(3.734,3.368)}
\gppoint{gp mark 0}{(3.734,3.368)}
\gppoint{gp mark 0}{(3.734,3.235)}
\gppoint{gp mark 0}{(3.734,3.634)}
\gppoint{gp mark 0}{(3.734,3.622)}
\gppoint{gp mark 0}{(3.734,3.720)}
\gppoint{gp mark 0}{(3.734,3.622)}
\gppoint{gp mark 0}{(3.734,3.368)}
\gppoint{gp mark 0}{(3.734,3.368)}
\gppoint{gp mark 0}{(3.734,3.368)}
\gppoint{gp mark 0}{(3.734,3.351)}
\gppoint{gp mark 0}{(3.734,3.314)}
\gppoint{gp mark 0}{(3.734,3.434)}
\gppoint{gp mark 0}{(3.734,3.402)}
\gppoint{gp mark 0}{(3.734,3.213)}
\gppoint{gp mark 0}{(3.734,3.314)}
\gppoint{gp mark 0}{(3.734,3.622)}
\gppoint{gp mark 0}{(3.734,3.561)}
\gppoint{gp mark 0}{(3.734,3.385)}
\gppoint{gp mark 0}{(3.734,3.276)}
\gppoint{gp mark 0}{(3.734,3.368)}
\gppoint{gp mark 0}{(3.734,3.634)}
\gppoint{gp mark 0}{(3.734,3.276)}
\gppoint{gp mark 0}{(3.734,3.314)}
\gppoint{gp mark 0}{(3.734,4.110)}
\gppoint{gp mark 0}{(3.734,3.314)}
\gppoint{gp mark 0}{(3.734,3.235)}
\gppoint{gp mark 0}{(3.734,3.333)}
\gppoint{gp mark 0}{(3.734,3.010)}
\gppoint{gp mark 0}{(3.734,3.276)}
\gppoint{gp mark 0}{(3.734,3.464)}
\gppoint{gp mark 0}{(3.734,3.213)}
\gppoint{gp mark 0}{(3.734,3.368)}
\gppoint{gp mark 0}{(3.734,3.235)}
\gppoint{gp mark 0}{(3.734,3.368)}
\gppoint{gp mark 0}{(3.734,3.368)}
\gppoint{gp mark 0}{(3.734,3.255)}
\gppoint{gp mark 0}{(3.734,3.368)}
\gppoint{gp mark 0}{(3.734,3.010)}
\gppoint{gp mark 0}{(3.734,3.067)}
\gppoint{gp mark 0}{(3.734,3.385)}
\gppoint{gp mark 0}{(3.734,3.295)}
\gppoint{gp mark 0}{(3.734,3.656)}
\gppoint{gp mark 0}{(3.734,3.010)}
\gppoint{gp mark 0}{(3.734,3.449)}
\gppoint{gp mark 0}{(3.734,3.168)}
\gppoint{gp mark 0}{(3.734,4.099)}
\gppoint{gp mark 0}{(3.734,3.634)}
\gppoint{gp mark 0}{(3.734,3.010)}
\gppoint{gp mark 0}{(3.734,3.368)}
\gppoint{gp mark 0}{(3.734,3.010)}
\gppoint{gp mark 0}{(3.734,3.010)}
\gppoint{gp mark 0}{(3.734,3.368)}
\gppoint{gp mark 0}{(3.734,3.368)}
\gppoint{gp mark 0}{(3.734,3.010)}
\gppoint{gp mark 0}{(3.734,3.255)}
\gppoint{gp mark 0}{(3.734,3.255)}
\gppoint{gp mark 0}{(3.734,3.434)}
\gppoint{gp mark 0}{(3.734,3.010)}
\gppoint{gp mark 0}{(3.734,3.276)}
\gppoint{gp mark 0}{(3.734,3.418)}
\gppoint{gp mark 0}{(3.734,3.213)}
\gppoint{gp mark 0}{(3.734,4.008)}
\gppoint{gp mark 0}{(3.734,3.276)}
\gppoint{gp mark 0}{(3.734,3.314)}
\gppoint{gp mark 0}{(3.734,3.213)}
\gppoint{gp mark 0}{(3.734,3.235)}
\gppoint{gp mark 0}{(3.734,3.010)}
\gppoint{gp mark 0}{(3.734,3.699)}
\gppoint{gp mark 0}{(3.734,3.385)}
\gppoint{gp mark 0}{(3.734,3.276)}
\gppoint{gp mark 0}{(3.734,3.235)}
\gppoint{gp mark 0}{(3.734,3.235)}
\gppoint{gp mark 0}{(3.734,3.213)}
\gppoint{gp mark 0}{(3.734,4.008)}
\gppoint{gp mark 0}{(3.734,4.064)}
\gppoint{gp mark 0}{(3.734,3.144)}
\gppoint{gp mark 0}{(3.734,3.561)}
\gppoint{gp mark 0}{(3.734,3.402)}
\gppoint{gp mark 0}{(3.734,3.235)}
\gppoint{gp mark 0}{(3.734,3.351)}
\gppoint{gp mark 0}{(3.734,3.010)}
\gppoint{gp mark 0}{(3.734,3.213)}
\gppoint{gp mark 0}{(3.734,3.351)}
\gppoint{gp mark 0}{(3.734,3.235)}
\gppoint{gp mark 0}{(3.734,3.730)}
\gppoint{gp mark 0}{(3.734,3.699)}
\gppoint{gp mark 0}{(3.734,3.010)}
\gppoint{gp mark 0}{(3.734,3.351)}
\gppoint{gp mark 0}{(3.734,3.010)}
\gppoint{gp mark 0}{(3.734,3.276)}
\gppoint{gp mark 0}{(3.734,3.493)}
\gppoint{gp mark 0}{(3.734,3.493)}
\gppoint{gp mark 0}{(3.734,3.094)}
\gppoint{gp mark 0}{(3.734,3.561)}
\gppoint{gp mark 0}{(3.734,3.235)}
\gppoint{gp mark 0}{(3.734,3.622)}
\gppoint{gp mark 0}{(3.734,3.535)}
\gppoint{gp mark 0}{(3.734,3.010)}
\gppoint{gp mark 0}{(3.734,3.368)}
\gppoint{gp mark 0}{(3.734,3.213)}
\gppoint{gp mark 0}{(3.734,3.535)}
\gppoint{gp mark 0}{(3.734,3.535)}
\gppoint{gp mark 0}{(3.734,3.535)}
\gppoint{gp mark 0}{(3.734,3.213)}
\gppoint{gp mark 0}{(3.734,3.168)}
\gppoint{gp mark 0}{(3.734,3.168)}
\gppoint{gp mark 0}{(3.734,3.598)}
\gppoint{gp mark 0}{(3.734,3.535)}
\gppoint{gp mark 0}{(3.734,3.535)}
\gppoint{gp mark 0}{(3.734,3.622)}
\gppoint{gp mark 0}{(3.734,3.255)}
\gppoint{gp mark 0}{(3.734,3.535)}
\gppoint{gp mark 0}{(3.734,3.535)}
\gppoint{gp mark 0}{(3.734,3.333)}
\gppoint{gp mark 0}{(3.734,3.213)}
\gppoint{gp mark 0}{(3.734,3.535)}
\gppoint{gp mark 0}{(3.734,3.535)}
\gppoint{gp mark 0}{(3.734,3.535)}
\gppoint{gp mark 0}{(3.734,3.598)}
\gppoint{gp mark 0}{(3.734,3.535)}
\gppoint{gp mark 0}{(3.734,3.094)}
\gppoint{gp mark 0}{(3.734,3.535)}
\gppoint{gp mark 0}{(3.734,3.598)}
\gppoint{gp mark 0}{(3.734,3.535)}
\gppoint{gp mark 0}{(3.734,3.759)}
\gppoint{gp mark 0}{(3.734,3.535)}
\gppoint{gp mark 0}{(3.734,3.598)}
\gppoint{gp mark 0}{(3.734,3.598)}
\gppoint{gp mark 0}{(3.734,3.464)}
\gppoint{gp mark 0}{(3.734,3.598)}
\gppoint{gp mark 0}{(3.734,3.094)}
\gppoint{gp mark 0}{(3.734,3.010)}
\gppoint{gp mark 0}{(3.734,3.351)}
\gppoint{gp mark 0}{(3.734,3.276)}
\gppoint{gp mark 0}{(3.734,3.010)}
\gppoint{gp mark 0}{(3.734,3.351)}
\gppoint{gp mark 0}{(3.734,3.434)}
\gppoint{gp mark 0}{(3.734,3.010)}
\gppoint{gp mark 0}{(3.734,3.276)}
\gppoint{gp mark 0}{(3.734,3.168)}
\gppoint{gp mark 0}{(3.734,3.168)}
\gppoint{gp mark 0}{(3.734,3.010)}
\gppoint{gp mark 0}{(3.734,3.235)}
\gppoint{gp mark 0}{(3.734,3.598)}
\gppoint{gp mark 0}{(3.734,3.295)}
\gppoint{gp mark 0}{(3.734,3.094)}
\gppoint{gp mark 0}{(3.734,3.598)}
\gppoint{gp mark 0}{(3.734,3.276)}
\gppoint{gp mark 0}{(3.734,2.914)}
\gppoint{gp mark 0}{(3.734,2.914)}
\gppoint{gp mark 0}{(3.734,3.479)}
\gppoint{gp mark 0}{(3.734,3.213)}
\gppoint{gp mark 0}{(3.734,3.418)}
\gppoint{gp mark 0}{(3.734,3.255)}
\gppoint{gp mark 0}{(3.734,3.168)}
\gppoint{gp mark 0}{(3.734,3.094)}
\gppoint{gp mark 0}{(3.734,3.368)}
\gppoint{gp mark 0}{(3.734,3.276)}
\gppoint{gp mark 0}{(3.734,3.010)}
\gppoint{gp mark 0}{(3.734,3.295)}
\gppoint{gp mark 0}{(3.734,3.010)}
\gppoint{gp mark 0}{(3.734,3.295)}
\gppoint{gp mark 0}{(3.734,3.094)}
\gppoint{gp mark 0}{(3.734,3.656)}
\gppoint{gp mark 0}{(3.734,3.276)}
\gppoint{gp mark 0}{(3.734,3.295)}
\gppoint{gp mark 0}{(3.734,3.213)}
\gppoint{gp mark 0}{(3.734,3.610)}
\gppoint{gp mark 0}{(3.734,3.094)}
\gppoint{gp mark 0}{(3.734,3.314)}
\gppoint{gp mark 0}{(3.734,3.561)}
\gppoint{gp mark 0}{(3.734,3.561)}
\gppoint{gp mark 0}{(3.734,3.235)}
\gppoint{gp mark 0}{(3.734,3.493)}
\gppoint{gp mark 0}{(3.734,3.333)}
\gppoint{gp mark 0}{(3.734,3.740)}
\gppoint{gp mark 0}{(3.734,3.740)}
\gppoint{gp mark 0}{(3.734,3.276)}
\gppoint{gp mark 0}{(3.734,3.548)}
\gppoint{gp mark 0}{(3.734,3.213)}
\gppoint{gp mark 0}{(3.734,3.213)}
\gppoint{gp mark 0}{(3.734,3.610)}
\gppoint{gp mark 0}{(3.734,3.010)}
\gppoint{gp mark 0}{(3.734,3.235)}
\gppoint{gp mark 0}{(3.734,3.213)}
\gppoint{gp mark 0}{(3.734,3.535)}
\gppoint{gp mark 0}{(3.734,3.094)}
\gppoint{gp mark 0}{(3.734,3.449)}
\gppoint{gp mark 0}{(3.734,3.235)}
\gppoint{gp mark 0}{(3.734,3.464)}
\gppoint{gp mark 0}{(3.734,3.464)}
\gppoint{gp mark 0}{(3.734,4.105)}
\gppoint{gp mark 0}{(3.734,3.314)}
\gppoint{gp mark 0}{(3.734,3.464)}
\gppoint{gp mark 0}{(3.734,3.213)}
\gppoint{gp mark 0}{(3.734,3.464)}
\gppoint{gp mark 0}{(3.734,3.276)}
\gppoint{gp mark 0}{(3.734,3.561)}
\gppoint{gp mark 0}{(3.734,3.314)}
\gppoint{gp mark 0}{(3.734,3.094)}
\gppoint{gp mark 0}{(3.734,3.385)}
\gppoint{gp mark 0}{(3.734,3.094)}
\gppoint{gp mark 0}{(3.734,3.561)}
\gppoint{gp mark 0}{(3.734,3.535)}
\gppoint{gp mark 0}{(3.734,3.656)}
\gppoint{gp mark 0}{(3.734,3.656)}
\gppoint{gp mark 0}{(3.734,3.168)}
\gppoint{gp mark 0}{(3.734,3.168)}
\gppoint{gp mark 0}{(3.734,3.678)}
\gppoint{gp mark 0}{(3.734,3.479)}
\gppoint{gp mark 0}{(3.734,3.168)}
\gppoint{gp mark 0}{(3.734,3.295)}
\gppoint{gp mark 0}{(3.734,3.368)}
\gppoint{gp mark 0}{(3.734,3.418)}
\gppoint{gp mark 0}{(3.734,3.464)}
\gppoint{gp mark 0}{(3.734,3.094)}
\gppoint{gp mark 0}{(3.734,3.213)}
\gppoint{gp mark 0}{(3.734,3.368)}
\gppoint{gp mark 0}{(3.734,3.333)}
\gppoint{gp mark 0}{(3.734,3.402)}
\gppoint{gp mark 0}{(3.734,3.535)}
\gppoint{gp mark 0}{(3.734,3.385)}
\gppoint{gp mark 0}{(3.734,3.213)}
\gppoint{gp mark 0}{(3.734,3.295)}
\gppoint{gp mark 0}{(3.734,3.449)}
\gppoint{gp mark 0}{(3.734,3.449)}
\gppoint{gp mark 0}{(3.734,3.213)}
\gppoint{gp mark 0}{(3.734,3.235)}
\gppoint{gp mark 0}{(3.734,3.213)}
\gppoint{gp mark 0}{(3.734,3.168)}
\gppoint{gp mark 0}{(3.734,3.295)}
\gppoint{gp mark 0}{(3.734,3.314)}
\gppoint{gp mark 0}{(3.734,3.168)}
\gppoint{gp mark 0}{(3.734,3.213)}
\gppoint{gp mark 0}{(3.734,3.094)}
\gppoint{gp mark 0}{(3.734,3.449)}
\gppoint{gp mark 0}{(3.734,3.368)}
\gppoint{gp mark 0}{(3.734,3.213)}
\gppoint{gp mark 0}{(3.734,3.191)}
\gppoint{gp mark 0}{(3.734,3.295)}
\gppoint{gp mark 0}{(3.734,3.940)}
\gppoint{gp mark 0}{(3.734,3.213)}
\gppoint{gp mark 0}{(3.804,3.548)}
\gppoint{gp mark 0}{(3.804,3.622)}
\gppoint{gp mark 0}{(3.804,3.507)}
\gppoint{gp mark 0}{(3.804,3.385)}
\gppoint{gp mark 0}{(3.804,3.464)}
\gppoint{gp mark 0}{(3.804,3.464)}
\gppoint{gp mark 0}{(3.804,3.449)}
\gppoint{gp mark 0}{(3.804,3.385)}
\gppoint{gp mark 0}{(3.804,3.295)}
\gppoint{gp mark 0}{(3.804,3.449)}
\gppoint{gp mark 0}{(3.804,3.813)}
\gppoint{gp mark 0}{(3.804,3.813)}
\gppoint{gp mark 0}{(3.804,3.586)}
\gppoint{gp mark 0}{(3.804,3.634)}
\gppoint{gp mark 0}{(3.804,3.314)}
\gppoint{gp mark 0}{(3.804,3.610)}
\gppoint{gp mark 0}{(3.804,3.418)}
\gppoint{gp mark 0}{(3.804,3.010)}
\gppoint{gp mark 0}{(3.804,3.402)}
\gppoint{gp mark 0}{(3.804,3.094)}
\gppoint{gp mark 0}{(3.804,4.198)}
\gppoint{gp mark 0}{(3.804,3.479)}
\gppoint{gp mark 0}{(3.804,2.979)}
\gppoint{gp mark 0}{(3.804,3.385)}
\gppoint{gp mark 0}{(3.804,3.464)}
\gppoint{gp mark 0}{(3.804,3.464)}
\gppoint{gp mark 0}{(3.804,3.586)}
\gppoint{gp mark 0}{(3.804,3.586)}
\gppoint{gp mark 0}{(3.804,3.094)}
\gppoint{gp mark 0}{(3.804,3.385)}
\gppoint{gp mark 0}{(3.804,3.168)}
\gppoint{gp mark 0}{(3.804,3.464)}
\gppoint{gp mark 0}{(3.804,3.678)}
\gppoint{gp mark 0}{(3.804,3.235)}
\gppoint{gp mark 0}{(3.804,3.010)}
\gppoint{gp mark 0}{(3.804,3.144)}
\gppoint{gp mark 0}{(3.804,3.689)}
\gppoint{gp mark 0}{(3.804,3.094)}
\gppoint{gp mark 0}{(3.804,3.368)}
\gppoint{gp mark 0}{(3.804,3.479)}
\gppoint{gp mark 0}{(3.804,3.598)}
\gppoint{gp mark 0}{(3.804,3.385)}
\gppoint{gp mark 0}{(3.804,3.094)}
\gppoint{gp mark 0}{(3.804,3.351)}
\gppoint{gp mark 0}{(3.804,3.351)}
\gppoint{gp mark 0}{(3.804,3.094)}
\gppoint{gp mark 0}{(3.804,3.351)}
\gppoint{gp mark 0}{(3.804,3.434)}
\gppoint{gp mark 0}{(3.804,3.094)}
\gppoint{gp mark 0}{(3.804,3.573)}
\gppoint{gp mark 0}{(3.804,4.198)}
\gppoint{gp mark 0}{(3.804,3.213)}
\gppoint{gp mark 0}{(3.804,3.699)}
\gppoint{gp mark 0}{(3.804,3.094)}
\gppoint{gp mark 0}{(3.804,3.699)}
\gppoint{gp mark 0}{(3.804,3.094)}
\gppoint{gp mark 0}{(3.804,3.295)}
\gppoint{gp mark 0}{(3.804,3.213)}
\gppoint{gp mark 0}{(3.804,3.351)}
\gppoint{gp mark 0}{(3.804,3.168)}
\gppoint{gp mark 0}{(3.804,3.094)}
\gppoint{gp mark 0}{(3.804,3.610)}
\gppoint{gp mark 0}{(3.804,3.740)}
\gppoint{gp mark 0}{(3.804,3.740)}
\gppoint{gp mark 0}{(3.804,3.740)}
\gppoint{gp mark 0}{(3.804,3.094)}
\gppoint{gp mark 0}{(3.804,3.740)}
\gppoint{gp mark 0}{(3.804,3.740)}
\gppoint{gp mark 0}{(3.804,3.094)}
\gppoint{gp mark 0}{(3.804,3.740)}
\gppoint{gp mark 0}{(3.804,3.740)}
\gppoint{gp mark 0}{(3.804,3.740)}
\gppoint{gp mark 0}{(3.804,3.276)}
\gppoint{gp mark 0}{(3.804,3.213)}
\gppoint{gp mark 0}{(3.804,3.434)}
\gppoint{gp mark 0}{(3.804,3.740)}
\gppoint{gp mark 0}{(3.804,3.094)}
\gppoint{gp mark 0}{(3.804,3.521)}
\gppoint{gp mark 0}{(3.804,3.813)}
\gppoint{gp mark 0}{(3.804,3.813)}
\gppoint{gp mark 0}{(3.804,3.213)}
\gppoint{gp mark 0}{(3.804,3.678)}
\gppoint{gp mark 0}{(3.804,3.740)}
\gppoint{gp mark 0}{(3.804,3.368)}
\gppoint{gp mark 0}{(3.804,3.351)}
\gppoint{gp mark 0}{(3.804,3.740)}
\gppoint{gp mark 0}{(3.804,3.740)}
\gppoint{gp mark 0}{(3.804,3.926)}
\gppoint{gp mark 0}{(3.804,3.926)}
\gppoint{gp mark 0}{(3.804,3.730)}
\gppoint{gp mark 0}{(3.804,3.314)}
\gppoint{gp mark 0}{(3.804,3.740)}
\gppoint{gp mark 0}{(3.804,3.094)}
\gppoint{gp mark 0}{(3.804,3.548)}
\gppoint{gp mark 0}{(3.804,3.276)}
\gppoint{gp mark 0}{(3.804,3.295)}
\gppoint{gp mark 0}{(3.804,3.678)}
\gppoint{gp mark 0}{(3.804,3.235)}
\gppoint{gp mark 0}{(3.804,3.295)}
\gppoint{gp mark 0}{(3.804,3.895)}
\gppoint{gp mark 0}{(3.804,3.168)}
\gppoint{gp mark 0}{(3.804,3.926)}
\gppoint{gp mark 0}{(3.804,3.926)}
\gppoint{gp mark 0}{(3.804,3.598)}
\gppoint{gp mark 0}{(3.804,3.634)}
\gppoint{gp mark 0}{(3.804,3.926)}
\gppoint{gp mark 0}{(3.804,3.351)}
\gppoint{gp mark 0}{(3.804,3.168)}
\gppoint{gp mark 0}{(3.804,3.235)}
\gppoint{gp mark 0}{(3.804,3.634)}
\gppoint{gp mark 0}{(3.804,3.094)}
\gppoint{gp mark 0}{(3.804,3.586)}
\gppoint{gp mark 0}{(3.804,3.295)}
\gppoint{gp mark 0}{(3.804,3.434)}
\gppoint{gp mark 0}{(3.804,3.235)}
\gppoint{gp mark 0}{(3.804,3.094)}
\gppoint{gp mark 0}{(3.804,3.295)}
\gppoint{gp mark 0}{(3.804,3.548)}
\gppoint{gp mark 0}{(3.804,3.385)}
\gppoint{gp mark 0}{(3.804,3.010)}
\gppoint{gp mark 0}{(3.804,3.903)}
\gppoint{gp mark 0}{(3.804,3.813)}
\gppoint{gp mark 0}{(3.804,3.094)}
\gppoint{gp mark 0}{(3.804,3.464)}
\gppoint{gp mark 0}{(3.804,3.548)}
\gppoint{gp mark 0}{(3.804,3.168)}
\gppoint{gp mark 0}{(3.804,3.926)}
\gppoint{gp mark 0}{(3.804,3.235)}
\gppoint{gp mark 0}{(3.804,3.368)}
\gppoint{gp mark 0}{(3.804,3.333)}
\gppoint{gp mark 0}{(3.804,3.493)}
\gppoint{gp mark 0}{(3.804,3.926)}
\gppoint{gp mark 0}{(3.804,3.314)}
\gppoint{gp mark 0}{(3.804,3.493)}
\gppoint{gp mark 0}{(3.804,3.864)}
\gppoint{gp mark 0}{(3.804,3.213)}
\gppoint{gp mark 0}{(3.804,3.213)}
\gppoint{gp mark 0}{(3.804,3.168)}
\gppoint{gp mark 0}{(3.804,3.067)}
\gppoint{gp mark 0}{(3.804,4.064)}
\gppoint{gp mark 0}{(3.804,3.295)}
\gppoint{gp mark 0}{(3.804,3.493)}
\gppoint{gp mark 0}{(3.804,3.573)}
\gppoint{gp mark 0}{(3.804,3.010)}
\gppoint{gp mark 0}{(3.804,3.235)}
\gppoint{gp mark 0}{(3.804,3.276)}
\gppoint{gp mark 0}{(3.804,3.235)}
\gppoint{gp mark 0}{(3.804,3.010)}
\gppoint{gp mark 0}{(3.804,3.168)}
\gppoint{gp mark 0}{(3.804,3.168)}
\gppoint{gp mark 0}{(3.804,3.872)}
\gppoint{gp mark 0}{(3.804,3.493)}
\gppoint{gp mark 0}{(3.804,3.464)}
\gppoint{gp mark 0}{(3.804,3.255)}
\gppoint{gp mark 0}{(3.804,3.094)}
\gppoint{gp mark 0}{(3.804,3.667)}
\gppoint{gp mark 0}{(3.804,3.464)}
\gppoint{gp mark 0}{(3.804,4.015)}
\gppoint{gp mark 0}{(3.804,3.699)}
\gppoint{gp mark 0}{(3.804,3.573)}
\gppoint{gp mark 0}{(3.804,3.622)}
\gppoint{gp mark 0}{(3.804,3.434)}
\gppoint{gp mark 0}{(3.804,3.010)}
\gppoint{gp mark 0}{(3.804,3.548)}
\gppoint{gp mark 0}{(3.804,3.402)}
\gppoint{gp mark 0}{(3.804,4.052)}
\gppoint{gp mark 0}{(3.804,4.021)}
\gppoint{gp mark 0}{(3.804,3.813)}
\gppoint{gp mark 0}{(3.804,3.385)}
\gppoint{gp mark 0}{(3.804,3.888)}
\gppoint{gp mark 0}{(3.804,3.385)}
\gppoint{gp mark 0}{(3.804,3.235)}
\gppoint{gp mark 0}{(3.804,3.385)}
\gppoint{gp mark 0}{(3.804,3.168)}
\gppoint{gp mark 0}{(3.804,3.276)}
\gppoint{gp mark 0}{(3.804,3.295)}
\gppoint{gp mark 0}{(3.804,3.010)}
\gppoint{gp mark 0}{(3.804,3.586)}
\gppoint{gp mark 0}{(3.804,3.507)}
\gppoint{gp mark 0}{(3.804,3.911)}
\gppoint{gp mark 0}{(3.804,3.168)}
\gppoint{gp mark 0}{(3.804,3.010)}
\gppoint{gp mark 0}{(3.804,3.010)}
\gppoint{gp mark 0}{(3.804,3.678)}
\gppoint{gp mark 0}{(3.804,3.010)}
\gppoint{gp mark 0}{(3.804,3.667)}
\gppoint{gp mark 0}{(3.804,3.010)}
\gppoint{gp mark 0}{(3.804,3.255)}
\gppoint{gp mark 0}{(3.804,3.235)}
\gppoint{gp mark 0}{(3.804,3.010)}
\gppoint{gp mark 0}{(3.804,3.667)}
\gppoint{gp mark 0}{(3.804,3.368)}
\gppoint{gp mark 0}{(3.804,3.667)}
\gppoint{gp mark 0}{(3.804,3.368)}
\gppoint{gp mark 0}{(3.804,3.276)}
\gppoint{gp mark 0}{(3.804,4.028)}
\gppoint{gp mark 0}{(3.804,3.667)}
\gppoint{gp mark 0}{(3.804,3.168)}
\gppoint{gp mark 0}{(3.804,3.864)}
\gppoint{gp mark 0}{(3.804,3.213)}
\gppoint{gp mark 0}{(3.804,3.276)}
\gppoint{gp mark 0}{(3.804,3.235)}
\gppoint{gp mark 0}{(3.804,4.179)}
\gppoint{gp mark 0}{(3.804,3.276)}
\gppoint{gp mark 0}{(3.804,3.368)}
\gppoint{gp mark 0}{(3.804,3.493)}
\gppoint{gp mark 0}{(3.804,3.434)}
\gppoint{gp mark 0}{(3.804,3.168)}
\gppoint{gp mark 0}{(3.804,3.385)}
\gppoint{gp mark 0}{(3.804,4.628)}
\gppoint{gp mark 0}{(3.804,3.333)}
\gppoint{gp mark 0}{(3.804,3.295)}
\gppoint{gp mark 0}{(3.804,3.805)}
\gppoint{gp mark 0}{(3.804,3.449)}
\gppoint{gp mark 0}{(3.804,3.235)}
\gppoint{gp mark 0}{(3.804,3.094)}
\gppoint{gp mark 0}{(3.804,3.464)}
\gppoint{gp mark 0}{(3.804,3.730)}
\gppoint{gp mark 0}{(3.804,3.213)}
\gppoint{gp mark 0}{(3.804,3.954)}
\gppoint{gp mark 0}{(3.804,3.235)}
\gppoint{gp mark 0}{(3.804,3.434)}
\gppoint{gp mark 0}{(3.804,3.295)}
\gppoint{gp mark 0}{(3.804,4.002)}
\gppoint{gp mark 0}{(3.804,3.911)}
\gppoint{gp mark 0}{(3.804,3.094)}
\gppoint{gp mark 0}{(3.804,3.535)}
\gppoint{gp mark 0}{(3.804,3.418)}
\gppoint{gp mark 0}{(3.804,3.255)}
\gppoint{gp mark 0}{(3.804,3.720)}
\gppoint{gp mark 0}{(3.804,3.295)}
\gppoint{gp mark 0}{(3.804,3.507)}
\gppoint{gp mark 0}{(3.804,3.402)}
\gppoint{gp mark 0}{(3.804,3.010)}
\gppoint{gp mark 0}{(3.804,3.975)}
\gppoint{gp mark 0}{(3.804,3.094)}
\gppoint{gp mark 0}{(3.804,3.368)}
\gppoint{gp mark 0}{(3.804,3.010)}
\gppoint{gp mark 0}{(3.804,3.295)}
\gppoint{gp mark 0}{(3.804,3.235)}
\gppoint{gp mark 0}{(3.804,3.168)}
\gppoint{gp mark 0}{(3.804,3.094)}
\gppoint{gp mark 0}{(3.804,3.586)}
\gppoint{gp mark 0}{(3.804,3.402)}
\gppoint{gp mark 0}{(3.804,3.094)}
\gppoint{gp mark 0}{(3.804,3.449)}
\gppoint{gp mark 0}{(3.804,3.586)}
\gppoint{gp mark 0}{(3.804,3.368)}
\gppoint{gp mark 0}{(3.804,3.434)}
\gppoint{gp mark 0}{(3.804,3.464)}
\gppoint{gp mark 0}{(3.804,3.295)}
\gppoint{gp mark 0}{(3.804,3.295)}
\gppoint{gp mark 0}{(3.804,3.634)}
\gppoint{gp mark 0}{(3.804,3.507)}
\gppoint{gp mark 0}{(3.804,3.368)}
\gppoint{gp mark 0}{(3.804,3.333)}
\gppoint{gp mark 0}{(3.804,3.295)}
\gppoint{gp mark 0}{(3.804,3.295)}
\gppoint{gp mark 0}{(3.804,4.046)}
\gppoint{gp mark 0}{(3.804,3.768)}
\gppoint{gp mark 0}{(3.804,3.295)}
\gppoint{gp mark 0}{(3.804,3.507)}
\gppoint{gp mark 0}{(3.804,3.213)}
\gppoint{gp mark 0}{(3.804,3.295)}
\gppoint{gp mark 0}{(3.804,3.689)}
\gppoint{gp mark 0}{(3.804,3.295)}
\gppoint{gp mark 0}{(3.804,3.094)}
\gppoint{gp mark 0}{(3.804,3.010)}
\gppoint{gp mark 0}{(3.804,3.385)}
\gppoint{gp mark 0}{(3.804,3.314)}
\gppoint{gp mark 0}{(3.804,3.351)}
\gppoint{gp mark 0}{(3.804,3.168)}
\gppoint{gp mark 0}{(3.804,3.276)}
\gppoint{gp mark 0}{(3.804,3.295)}
\gppoint{gp mark 0}{(3.804,3.418)}
\gppoint{gp mark 0}{(3.804,3.710)}
\gppoint{gp mark 0}{(3.804,3.507)}
\gppoint{gp mark 0}{(3.804,3.740)}
\gppoint{gp mark 0}{(3.804,3.314)}
\gppoint{gp mark 0}{(3.804,3.314)}
\gppoint{gp mark 0}{(3.804,3.710)}
\gppoint{gp mark 0}{(3.804,3.449)}
\gppoint{gp mark 0}{(3.804,3.314)}
\gppoint{gp mark 0}{(3.804,3.493)}
\gppoint{gp mark 0}{(3.804,3.521)}
\gppoint{gp mark 0}{(3.804,3.493)}
\gppoint{gp mark 0}{(3.804,3.710)}
\gppoint{gp mark 0}{(3.804,3.449)}
\gppoint{gp mark 0}{(3.804,3.333)}
\gppoint{gp mark 0}{(3.804,3.094)}
\gppoint{gp mark 0}{(3.804,3.548)}
\gppoint{gp mark 0}{(3.804,3.295)}
\gppoint{gp mark 0}{(3.804,3.464)}
\gppoint{gp mark 0}{(3.804,3.144)}
\gppoint{gp mark 0}{(3.804,3.368)}
\gppoint{gp mark 0}{(3.804,3.434)}
\gppoint{gp mark 0}{(3.804,3.434)}
\gppoint{gp mark 0}{(3.804,3.235)}
\gppoint{gp mark 0}{(3.804,3.010)}
\gppoint{gp mark 0}{(3.804,3.586)}
\gppoint{gp mark 0}{(3.804,3.385)}
\gppoint{gp mark 0}{(3.804,3.067)}
\gppoint{gp mark 0}{(3.804,3.507)}
\gppoint{gp mark 0}{(3.804,3.235)}
\gppoint{gp mark 0}{(3.804,3.493)}
\gppoint{gp mark 0}{(3.804,3.805)}
\gppoint{gp mark 0}{(3.804,3.507)}
\gppoint{gp mark 0}{(3.804,3.213)}
\gppoint{gp mark 0}{(3.804,3.385)}
\gppoint{gp mark 0}{(3.804,3.276)}
\gppoint{gp mark 0}{(3.804,3.235)}
\gppoint{gp mark 0}{(3.804,3.507)}
\gppoint{gp mark 0}{(3.804,3.094)}
\gppoint{gp mark 0}{(3.804,3.402)}
\gppoint{gp mark 0}{(3.804,3.507)}
\gppoint{gp mark 0}{(3.804,3.645)}
\gppoint{gp mark 0}{(3.804,3.314)}
\gppoint{gp mark 0}{(3.804,3.144)}
\gppoint{gp mark 0}{(3.804,3.464)}
\gppoint{gp mark 0}{(3.804,3.402)}
\gppoint{gp mark 0}{(3.804,3.235)}
\gppoint{gp mark 0}{(3.804,3.535)}
\gppoint{gp mark 0}{(3.804,3.295)}
\gppoint{gp mark 0}{(3.804,2.914)}
\gppoint{gp mark 0}{(3.804,3.402)}
\gppoint{gp mark 0}{(3.804,3.402)}
\gppoint{gp mark 0}{(3.804,3.094)}
\gppoint{gp mark 0}{(3.804,3.351)}
\gppoint{gp mark 0}{(3.804,3.168)}
\gppoint{gp mark 0}{(3.804,3.667)}
\gppoint{gp mark 0}{(3.804,3.667)}
\gppoint{gp mark 0}{(3.804,3.168)}
\gppoint{gp mark 0}{(3.804,3.191)}
\gppoint{gp mark 0}{(3.804,3.333)}
\gppoint{gp mark 0}{(3.804,3.351)}
\gppoint{gp mark 0}{(3.804,3.749)}
\gppoint{gp mark 0}{(3.804,3.351)}
\gppoint{gp mark 0}{(3.804,3.168)}
\gppoint{gp mark 0}{(3.804,3.720)}
\gppoint{gp mark 0}{(3.804,3.856)}
\gppoint{gp mark 0}{(3.804,4.015)}
\gppoint{gp mark 0}{(3.804,3.094)}
\gppoint{gp mark 0}{(3.804,3.094)}
\gppoint{gp mark 0}{(3.804,3.449)}
\gppoint{gp mark 0}{(3.804,3.235)}
\gppoint{gp mark 0}{(3.804,3.402)}
\gppoint{gp mark 0}{(3.804,3.295)}
\gppoint{gp mark 0}{(3.804,3.402)}
\gppoint{gp mark 0}{(3.804,3.507)}
\gppoint{gp mark 0}{(3.804,3.561)}
\gppoint{gp mark 0}{(3.804,3.561)}
\gppoint{gp mark 0}{(3.804,3.168)}
\gppoint{gp mark 0}{(3.804,2.979)}
\gppoint{gp mark 0}{(3.804,4.052)}
\gppoint{gp mark 0}{(3.804,3.333)}
\gppoint{gp mark 0}{(3.804,3.402)}
\gppoint{gp mark 0}{(3.804,3.235)}
\gppoint{gp mark 0}{(3.804,3.402)}
\gppoint{gp mark 0}{(3.804,3.168)}
\gppoint{gp mark 0}{(3.804,3.464)}
\gppoint{gp mark 0}{(3.804,3.730)}
\gppoint{gp mark 0}{(3.804,3.385)}
\gppoint{gp mark 0}{(3.804,3.351)}
\gppoint{gp mark 0}{(3.804,3.235)}
\gppoint{gp mark 0}{(3.804,3.235)}
\gppoint{gp mark 0}{(3.804,3.276)}
\gppoint{gp mark 0}{(3.804,3.493)}
\gppoint{gp mark 0}{(3.804,3.333)}
\gppoint{gp mark 0}{(3.804,3.822)}
\gppoint{gp mark 0}{(3.804,3.507)}
\gppoint{gp mark 0}{(3.804,3.213)}
\gppoint{gp mark 0}{(3.804,3.144)}
\gppoint{gp mark 0}{(3.804,3.385)}
\gppoint{gp mark 0}{(3.804,4.110)}
\gppoint{gp mark 0}{(3.804,3.276)}
\gppoint{gp mark 0}{(3.804,3.168)}
\gppoint{gp mark 0}{(3.804,3.276)}
\gppoint{gp mark 0}{(3.804,4.236)}
\gppoint{gp mark 0}{(3.804,3.213)}
\gppoint{gp mark 0}{(3.804,3.276)}
\gppoint{gp mark 0}{(3.804,3.168)}
\gppoint{gp mark 0}{(3.804,3.749)}
\gppoint{gp mark 0}{(3.804,3.586)}
\gppoint{gp mark 0}{(3.804,4.015)}
\gppoint{gp mark 0}{(3.804,3.213)}
\gppoint{gp mark 0}{(3.804,3.168)}
\gppoint{gp mark 0}{(3.804,3.094)}
\gppoint{gp mark 0}{(3.804,3.235)}
\gppoint{gp mark 0}{(3.804,3.094)}
\gppoint{gp mark 0}{(3.804,3.561)}
\gppoint{gp mark 0}{(3.804,3.479)}
\gppoint{gp mark 0}{(3.804,3.255)}
\gppoint{gp mark 0}{(3.804,3.295)}
\gppoint{gp mark 0}{(3.804,3.295)}
\gppoint{gp mark 0}{(3.804,3.678)}
\gppoint{gp mark 0}{(3.804,3.168)}
\gppoint{gp mark 0}{(3.804,3.351)}
\gppoint{gp mark 0}{(3.804,3.449)}
\gppoint{gp mark 0}{(3.804,3.586)}
\gppoint{gp mark 0}{(3.804,3.094)}
\gppoint{gp mark 0}{(3.804,3.678)}
\gppoint{gp mark 0}{(3.804,3.385)}
\gppoint{gp mark 0}{(3.804,3.385)}
\gppoint{gp mark 0}{(3.804,3.333)}
\gppoint{gp mark 0}{(3.804,3.385)}
\gppoint{gp mark 0}{(3.804,3.402)}
\gppoint{gp mark 0}{(3.804,3.191)}
\gppoint{gp mark 0}{(3.804,3.813)}
\gppoint{gp mark 0}{(3.804,3.295)}
\gppoint{gp mark 0}{(3.804,3.368)}
\gppoint{gp mark 0}{(3.804,3.418)}
\gppoint{gp mark 0}{(3.804,3.768)}
\gppoint{gp mark 0}{(3.804,3.010)}
\gppoint{gp mark 0}{(3.804,3.479)}
\gppoint{gp mark 0}{(3.804,3.333)}
\gppoint{gp mark 0}{(3.804,3.479)}
\gppoint{gp mark 0}{(3.804,3.493)}
\gppoint{gp mark 0}{(3.804,3.168)}
\gppoint{gp mark 0}{(3.804,3.168)}
\gppoint{gp mark 0}{(3.804,3.351)}
\gppoint{gp mark 0}{(3.804,3.010)}
\gppoint{gp mark 0}{(3.804,3.213)}
\gppoint{gp mark 0}{(3.804,3.449)}
\gppoint{gp mark 0}{(3.804,3.168)}
\gppoint{gp mark 0}{(3.804,3.213)}
\gppoint{gp mark 0}{(3.804,3.168)}
\gppoint{gp mark 0}{(3.804,3.507)}
\gppoint{gp mark 0}{(3.804,3.168)}
\gppoint{gp mark 0}{(3.804,3.507)}
\gppoint{gp mark 0}{(3.804,3.067)}
\gppoint{gp mark 0}{(3.804,3.368)}
\gppoint{gp mark 0}{(3.804,3.144)}
\gppoint{gp mark 0}{(3.804,3.010)}
\gppoint{gp mark 0}{(3.804,4.127)}
\gppoint{gp mark 0}{(3.804,3.368)}
\gppoint{gp mark 0}{(3.804,3.094)}
\gppoint{gp mark 0}{(3.804,3.094)}
\gppoint{gp mark 0}{(3.804,3.168)}
\gppoint{gp mark 0}{(3.804,3.168)}
\gppoint{gp mark 0}{(3.804,3.507)}
\gppoint{gp mark 0}{(3.804,3.573)}
\gppoint{gp mark 0}{(3.804,3.598)}
\gppoint{gp mark 0}{(3.804,3.548)}
\gppoint{gp mark 0}{(3.804,3.507)}
\gppoint{gp mark 0}{(3.804,3.094)}
\gppoint{gp mark 0}{(3.804,3.094)}
\gppoint{gp mark 0}{(3.804,3.168)}
\gppoint{gp mark 0}{(3.804,3.805)}
\gppoint{gp mark 0}{(3.804,3.434)}
\gppoint{gp mark 0}{(3.804,3.010)}
\gppoint{gp mark 0}{(3.804,3.434)}
\gppoint{gp mark 0}{(3.804,3.598)}
\gppoint{gp mark 0}{(3.804,3.434)}
\gppoint{gp mark 0}{(3.804,3.213)}
\gppoint{gp mark 0}{(3.804,3.235)}
\gppoint{gp mark 0}{(3.804,3.573)}
\gppoint{gp mark 0}{(3.804,3.094)}
\gppoint{gp mark 0}{(3.804,3.094)}
\gppoint{gp mark 0}{(3.804,3.168)}
\gppoint{gp mark 0}{(3.804,3.521)}
\gppoint{gp mark 0}{(3.804,3.168)}
\gppoint{gp mark 0}{(3.804,3.598)}
\gppoint{gp mark 0}{(3.804,3.010)}
\gppoint{gp mark 0}{(3.804,3.968)}
\gppoint{gp mark 0}{(3.804,3.402)}
\gppoint{gp mark 0}{(3.804,3.010)}
\gppoint{gp mark 0}{(3.804,3.385)}
\gppoint{gp mark 0}{(3.804,3.598)}
\gppoint{gp mark 0}{(3.804,3.730)}
\gppoint{gp mark 0}{(3.804,3.295)}
\gppoint{gp mark 0}{(3.804,3.968)}
\gppoint{gp mark 0}{(3.804,3.276)}
\gppoint{gp mark 0}{(3.804,3.235)}
\gppoint{gp mark 0}{(3.804,3.598)}
\gppoint{gp mark 0}{(3.804,3.634)}
\gppoint{gp mark 0}{(3.804,3.385)}
\gppoint{gp mark 0}{(3.804,3.235)}
\gppoint{gp mark 0}{(3.804,3.385)}
\gppoint{gp mark 0}{(3.804,3.548)}
\gppoint{gp mark 0}{(3.804,3.385)}
\gppoint{gp mark 0}{(3.804,3.385)}
\gppoint{gp mark 0}{(3.804,3.010)}
\gppoint{gp mark 0}{(3.804,3.548)}
\gppoint{gp mark 0}{(3.804,3.094)}
\gppoint{gp mark 0}{(3.804,3.385)}
\gppoint{gp mark 0}{(3.804,3.385)}
\gppoint{gp mark 0}{(3.804,3.749)}
\gppoint{gp mark 0}{(3.804,3.385)}
\gppoint{gp mark 0}{(3.804,3.295)}
\gppoint{gp mark 0}{(3.869,3.368)}
\gppoint{gp mark 0}{(3.869,3.314)}
\gppoint{gp mark 0}{(3.869,3.856)}
\gppoint{gp mark 0}{(3.869,3.656)}
\gppoint{gp mark 0}{(3.869,3.634)}
\gppoint{gp mark 0}{(3.869,3.656)}
\gppoint{gp mark 0}{(3.869,3.191)}
\gppoint{gp mark 0}{(3.869,4.179)}
\gppoint{gp mark 0}{(3.869,3.656)}
\gppoint{gp mark 0}{(3.869,3.213)}
\gppoint{gp mark 0}{(3.869,3.449)}
\gppoint{gp mark 0}{(3.869,3.449)}
\gppoint{gp mark 0}{(3.869,3.449)}
\gppoint{gp mark 0}{(3.869,3.385)}
\gppoint{gp mark 0}{(3.869,3.888)}
\gppoint{gp mark 0}{(3.869,3.418)}
\gppoint{gp mark 0}{(3.869,3.213)}
\gppoint{gp mark 0}{(3.869,3.634)}
\gppoint{gp mark 0}{(3.869,3.168)}
\gppoint{gp mark 0}{(3.869,3.493)}
\gppoint{gp mark 0}{(3.869,3.213)}
\gppoint{gp mark 0}{(3.869,3.333)}
\gppoint{gp mark 0}{(3.869,3.449)}
\gppoint{gp mark 0}{(3.869,3.749)}
\gppoint{gp mark 0}{(3.869,3.168)}
\gppoint{gp mark 0}{(3.869,3.548)}
\gppoint{gp mark 0}{(3.869,3.622)}
\gppoint{gp mark 0}{(3.869,3.276)}
\gppoint{gp mark 0}{(3.869,3.094)}
\gppoint{gp mark 0}{(3.869,4.143)}
\gppoint{gp mark 0}{(3.869,3.561)}
\gppoint{gp mark 0}{(3.869,3.479)}
\gppoint{gp mark 0}{(3.869,3.168)}
\gppoint{gp mark 0}{(3.869,3.368)}
\gppoint{gp mark 0}{(3.869,3.276)}
\gppoint{gp mark 0}{(3.869,3.507)}
\gppoint{gp mark 0}{(3.869,3.213)}
\gppoint{gp mark 0}{(3.869,3.778)}
\gppoint{gp mark 0}{(3.869,3.235)}
\gppoint{gp mark 0}{(3.869,3.989)}
\gppoint{gp mark 0}{(3.869,3.778)}
\gppoint{gp mark 0}{(3.869,3.449)}
\gppoint{gp mark 0}{(3.869,3.168)}
\gppoint{gp mark 0}{(3.869,3.168)}
\gppoint{gp mark 0}{(3.869,3.094)}
\gppoint{gp mark 0}{(3.869,3.213)}
\gppoint{gp mark 0}{(3.869,3.168)}
\gppoint{gp mark 0}{(3.869,4.052)}
\gppoint{gp mark 0}{(3.869,3.368)}
\gppoint{gp mark 0}{(3.869,3.276)}
\gppoint{gp mark 0}{(3.869,3.634)}
\gppoint{gp mark 0}{(3.869,3.276)}
\gppoint{gp mark 0}{(3.869,3.895)}
\gppoint{gp mark 0}{(3.869,3.634)}
\gppoint{gp mark 0}{(3.869,3.759)}
\gppoint{gp mark 0}{(3.869,3.094)}
\gppoint{gp mark 0}{(3.869,3.094)}
\gppoint{gp mark 0}{(3.869,3.333)}
\gppoint{gp mark 0}{(3.869,3.634)}
\gppoint{gp mark 0}{(3.869,3.333)}
\gppoint{gp mark 0}{(3.869,3.573)}
\gppoint{gp mark 0}{(3.869,3.213)}
\gppoint{gp mark 0}{(3.869,3.094)}
\gppoint{gp mark 0}{(3.869,3.168)}
\gppoint{gp mark 0}{(3.869,3.368)}
\gppoint{gp mark 0}{(3.869,3.464)}
\gppoint{gp mark 0}{(3.869,3.402)}
\gppoint{gp mark 0}{(3.869,3.622)}
\gppoint{gp mark 0}{(3.869,3.168)}
\gppoint{gp mark 0}{(3.869,3.689)}
\gppoint{gp mark 0}{(3.869,3.418)}
\gppoint{gp mark 0}{(3.869,3.622)}
\gppoint{gp mark 0}{(3.869,3.213)}
\gppoint{gp mark 0}{(3.869,3.333)}
\gppoint{gp mark 0}{(3.869,3.521)}
\gppoint{gp mark 0}{(3.869,3.094)}
\gppoint{gp mark 0}{(3.869,3.507)}
\gppoint{gp mark 0}{(3.869,3.368)}
\gppoint{gp mark 0}{(3.869,3.168)}
\gppoint{gp mark 0}{(3.869,3.333)}
\gppoint{gp mark 0}{(3.869,3.645)}
\gppoint{gp mark 0}{(3.869,3.418)}
\gppoint{gp mark 0}{(3.869,3.168)}
\gppoint{gp mark 0}{(3.869,3.168)}
\gppoint{gp mark 0}{(3.869,3.368)}
\gppoint{gp mark 0}{(3.869,3.368)}
\gppoint{gp mark 0}{(3.869,4.280)}
\gppoint{gp mark 0}{(3.869,3.720)}
\gppoint{gp mark 0}{(3.869,3.535)}
\gppoint{gp mark 0}{(3.869,3.418)}
\gppoint{gp mark 0}{(3.869,4.070)}
\gppoint{gp mark 0}{(3.869,3.314)}
\gppoint{gp mark 0}{(3.869,3.213)}
\gppoint{gp mark 0}{(3.869,3.168)}
\gppoint{gp mark 0}{(3.869,3.402)}
\gppoint{gp mark 0}{(3.869,3.856)}
\gppoint{gp mark 0}{(3.869,3.434)}
\gppoint{gp mark 0}{(3.869,3.368)}
\gppoint{gp mark 0}{(3.869,3.656)}
\gppoint{gp mark 0}{(3.869,3.418)}
\gppoint{gp mark 0}{(3.869,3.295)}
\gppoint{gp mark 0}{(3.869,3.796)}
\gppoint{gp mark 0}{(3.869,3.314)}
\gppoint{gp mark 0}{(3.869,3.507)}
\gppoint{gp mark 0}{(3.869,3.168)}
\gppoint{gp mark 0}{(3.869,3.548)}
\gppoint{gp mark 0}{(3.869,2.914)}
\gppoint{gp mark 0}{(3.869,3.276)}
\gppoint{gp mark 0}{(3.869,3.759)}
\gppoint{gp mark 0}{(3.869,3.368)}
\gppoint{gp mark 0}{(3.869,3.276)}
\gppoint{gp mark 0}{(3.869,3.418)}
\gppoint{gp mark 0}{(3.869,3.888)}
\gppoint{gp mark 0}{(3.869,3.479)}
\gppoint{gp mark 0}{(3.869,2.979)}
\gppoint{gp mark 0}{(3.869,2.979)}
\gppoint{gp mark 0}{(3.869,3.368)}
\gppoint{gp mark 0}{(3.869,3.168)}
\gppoint{gp mark 0}{(3.869,3.434)}
\gppoint{gp mark 0}{(3.869,3.634)}
\gppoint{gp mark 0}{(3.869,3.730)}
\gppoint{gp mark 0}{(3.869,3.368)}
\gppoint{gp mark 0}{(3.869,3.276)}
\gppoint{gp mark 0}{(3.869,3.689)}
\gppoint{gp mark 0}{(3.869,3.479)}
\gppoint{gp mark 0}{(3.869,3.479)}
\gppoint{gp mark 0}{(3.869,3.730)}
\gppoint{gp mark 0}{(3.869,3.314)}
\gppoint{gp mark 0}{(3.869,3.333)}
\gppoint{gp mark 0}{(3.869,3.235)}
\gppoint{gp mark 0}{(3.869,3.010)}
\gppoint{gp mark 0}{(3.869,3.295)}
\gppoint{gp mark 0}{(3.869,3.235)}
\gppoint{gp mark 0}{(3.869,3.295)}
\gppoint{gp mark 0}{(3.869,3.586)}
\gppoint{gp mark 0}{(3.869,3.235)}
\gppoint{gp mark 0}{(3.869,3.295)}
\gppoint{gp mark 0}{(3.869,3.449)}
\gppoint{gp mark 0}{(3.869,3.276)}
\gppoint{gp mark 0}{(3.869,3.385)}
\gppoint{gp mark 0}{(3.869,3.168)}
\gppoint{gp mark 0}{(3.869,3.434)}
\gppoint{gp mark 0}{(3.869,3.667)}
\gppoint{gp mark 0}{(3.869,3.856)}
\gppoint{gp mark 0}{(3.869,3.385)}
\gppoint{gp mark 0}{(3.869,3.368)}
\gppoint{gp mark 0}{(3.869,3.895)}
\gppoint{gp mark 0}{(3.869,3.548)}
\gppoint{gp mark 0}{(3.869,4.116)}
\gppoint{gp mark 0}{(3.869,3.368)}
\gppoint{gp mark 0}{(3.869,3.730)}
\gppoint{gp mark 0}{(3.869,3.351)}
\gppoint{gp mark 0}{(3.869,3.385)}
\gppoint{gp mark 0}{(3.869,3.295)}
\gppoint{gp mark 0}{(3.869,3.368)}
\gppoint{gp mark 0}{(3.869,3.667)}
\gppoint{gp mark 0}{(3.869,3.730)}
\gppoint{gp mark 0}{(3.869,3.295)}
\gppoint{gp mark 0}{(3.869,3.168)}
\gppoint{gp mark 0}{(3.869,3.168)}
\gppoint{gp mark 0}{(3.869,3.314)}
\gppoint{gp mark 0}{(3.869,3.295)}
\gppoint{gp mark 0}{(3.869,3.333)}
\gppoint{gp mark 0}{(3.869,3.418)}
\gppoint{gp mark 0}{(3.869,3.168)}
\gppoint{gp mark 0}{(3.869,3.276)}
\gppoint{gp mark 0}{(3.869,3.351)}
\gppoint{gp mark 0}{(3.869,3.954)}
\gppoint{gp mark 0}{(3.869,3.191)}
\gppoint{gp mark 0}{(3.869,3.039)}
\gppoint{gp mark 0}{(3.869,3.295)}
\gppoint{gp mark 0}{(3.869,3.656)}
\gppoint{gp mark 0}{(3.869,3.493)}
\gppoint{gp mark 0}{(3.869,3.548)}
\gppoint{gp mark 0}{(3.869,3.168)}
\gppoint{gp mark 0}{(3.869,3.610)}
\gppoint{gp mark 0}{(3.869,3.573)}
\gppoint{gp mark 0}{(3.869,3.094)}
\gppoint{gp mark 0}{(3.869,3.368)}
\gppoint{gp mark 0}{(3.869,3.796)}
\gppoint{gp mark 0}{(3.869,3.418)}
\gppoint{gp mark 0}{(3.869,3.094)}
\gppoint{gp mark 0}{(3.869,3.507)}
\gppoint{gp mark 0}{(3.869,3.880)}
\gppoint{gp mark 0}{(3.869,3.235)}
\gppoint{gp mark 0}{(3.869,3.434)}
\gppoint{gp mark 0}{(3.869,3.119)}
\gppoint{gp mark 0}{(3.869,3.689)}
\gppoint{gp mark 0}{(3.869,3.235)}
\gppoint{gp mark 0}{(3.869,4.105)}
\gppoint{gp mark 0}{(3.869,3.787)}
\gppoint{gp mark 0}{(3.869,3.759)}
\gppoint{gp mark 0}{(3.869,3.872)}
\gppoint{gp mark 0}{(3.869,4.015)}
\gppoint{gp mark 0}{(3.869,3.535)}
\gppoint{gp mark 0}{(3.869,3.168)}
\gppoint{gp mark 0}{(3.869,3.235)}
\gppoint{gp mark 0}{(3.869,3.796)}
\gppoint{gp mark 0}{(3.869,3.521)}
\gppoint{gp mark 0}{(3.869,3.168)}
\gppoint{gp mark 0}{(3.869,3.418)}
\gppoint{gp mark 0}{(3.869,3.385)}
\gppoint{gp mark 0}{(3.869,3.295)}
\gppoint{gp mark 0}{(3.869,3.548)}
\gppoint{gp mark 0}{(3.869,3.333)}
\gppoint{gp mark 0}{(3.869,3.235)}
\gppoint{gp mark 0}{(3.869,3.276)}
\gppoint{gp mark 0}{(3.869,3.368)}
\gppoint{gp mark 0}{(3.869,3.235)}
\gppoint{gp mark 0}{(3.869,3.368)}
\gppoint{gp mark 0}{(3.869,3.586)}
\gppoint{gp mark 0}{(3.869,3.689)}
\gppoint{gp mark 0}{(3.869,3.295)}
\gppoint{gp mark 0}{(3.869,3.402)}
\gppoint{gp mark 0}{(3.869,3.759)}
\gppoint{gp mark 0}{(3.869,3.351)}
\gppoint{gp mark 0}{(3.869,3.645)}
\gppoint{gp mark 0}{(3.869,3.351)}
\gppoint{gp mark 0}{(3.869,3.548)}
\gppoint{gp mark 0}{(3.869,3.276)}
\gppoint{gp mark 0}{(3.869,3.759)}
\gppoint{gp mark 0}{(3.869,3.740)}
\gppoint{gp mark 0}{(3.869,3.368)}
\gppoint{gp mark 0}{(3.869,3.368)}
\gppoint{gp mark 0}{(3.869,3.449)}
\gppoint{gp mark 0}{(3.869,3.856)}
\gppoint{gp mark 0}{(3.869,3.856)}
\gppoint{gp mark 0}{(3.869,3.094)}
\gppoint{gp mark 0}{(3.869,3.276)}
\gppoint{gp mark 0}{(3.869,3.645)}
\gppoint{gp mark 0}{(3.869,3.402)}
\gppoint{gp mark 0}{(3.869,3.168)}
\gppoint{gp mark 0}{(3.869,3.295)}
\gppoint{gp mark 0}{(3.869,3.598)}
\gppoint{gp mark 0}{(3.869,3.493)}
\gppoint{gp mark 0}{(3.869,3.645)}
\gppoint{gp mark 0}{(3.869,3.010)}
\gppoint{gp mark 0}{(3.869,3.720)}
\gppoint{gp mark 0}{(3.869,3.479)}
\gppoint{gp mark 0}{(3.869,3.710)}
\gppoint{gp mark 0}{(3.869,3.067)}
\gppoint{gp mark 0}{(3.869,4.164)}
\gppoint{gp mark 0}{(3.869,3.610)}
\gppoint{gp mark 0}{(3.869,3.168)}
\gppoint{gp mark 0}{(3.869,3.561)}
\gppoint{gp mark 0}{(3.869,3.168)}
\gppoint{gp mark 0}{(3.869,3.645)}
\gppoint{gp mark 0}{(3.869,3.295)}
\gppoint{gp mark 0}{(3.869,3.796)}
\gppoint{gp mark 0}{(3.869,3.720)}
\gppoint{gp mark 0}{(3.869,3.168)}
\gppoint{gp mark 0}{(3.869,3.168)}
\gppoint{gp mark 0}{(3.869,4.121)}
\gppoint{gp mark 0}{(3.869,3.449)}
\gppoint{gp mark 0}{(3.869,3.418)}
\gppoint{gp mark 0}{(3.869,3.880)}
\gppoint{gp mark 0}{(3.869,4.093)}
\gppoint{gp mark 0}{(3.869,3.645)}
\gppoint{gp mark 0}{(3.869,3.418)}
\gppoint{gp mark 0}{(3.869,3.479)}
\gppoint{gp mark 0}{(3.869,3.434)}
\gppoint{gp mark 0}{(3.869,3.645)}
\gppoint{gp mark 0}{(3.869,3.168)}
\gppoint{gp mark 0}{(3.869,3.645)}
\gppoint{gp mark 0}{(3.869,3.314)}
\gppoint{gp mark 0}{(3.869,3.333)}
\gppoint{gp mark 0}{(3.869,3.213)}
\gppoint{gp mark 0}{(3.869,3.699)}
\gppoint{gp mark 0}{(3.869,4.093)}
\gppoint{gp mark 0}{(3.869,3.094)}
\gppoint{gp mark 0}{(3.869,3.645)}
\gppoint{gp mark 0}{(3.869,3.479)}
\gppoint{gp mark 0}{(3.869,3.645)}
\gppoint{gp mark 0}{(3.869,3.521)}
\gppoint{gp mark 0}{(3.869,3.464)}
\gppoint{gp mark 0}{(3.869,3.464)}
\gppoint{gp mark 0}{(3.869,3.507)}
\gppoint{gp mark 0}{(3.869,3.314)}
\gppoint{gp mark 0}{(3.869,4.093)}
\gppoint{gp mark 0}{(3.869,4.326)}
\gppoint{gp mark 0}{(3.869,3.622)}
\gppoint{gp mark 0}{(3.869,3.295)}
\gppoint{gp mark 0}{(3.869,3.295)}
\gppoint{gp mark 0}{(3.869,4.143)}
\gppoint{gp mark 0}{(3.869,3.235)}
\gppoint{gp mark 0}{(3.869,3.385)}
\gppoint{gp mark 0}{(3.869,3.168)}
\gppoint{gp mark 0}{(3.869,3.449)}
\gppoint{gp mark 0}{(3.869,3.645)}
\gppoint{gp mark 0}{(3.869,3.710)}
\gppoint{gp mark 0}{(3.869,3.168)}
\gppoint{gp mark 0}{(3.869,3.656)}
\gppoint{gp mark 0}{(3.869,3.235)}
\gppoint{gp mark 0}{(3.869,3.933)}
\gppoint{gp mark 0}{(3.869,3.586)}
\gppoint{gp mark 0}{(3.869,3.493)}
\gppoint{gp mark 0}{(3.869,3.333)}
\gppoint{gp mark 0}{(3.869,3.464)}
\gppoint{gp mark 0}{(3.869,3.903)}
\gppoint{gp mark 0}{(3.869,3.903)}
\gppoint{gp mark 0}{(3.869,3.464)}
\gppoint{gp mark 0}{(3.869,3.368)}
\gppoint{gp mark 0}{(3.869,3.314)}
\gppoint{gp mark 0}{(3.869,3.940)}
\gppoint{gp mark 0}{(3.869,3.314)}
\gppoint{gp mark 0}{(3.869,3.740)}
\gppoint{gp mark 0}{(3.869,3.622)}
\gppoint{gp mark 0}{(3.869,3.368)}
\gppoint{gp mark 0}{(3.869,4.127)}
\gppoint{gp mark 0}{(3.869,3.418)}
\gppoint{gp mark 0}{(3.869,3.314)}
\gppoint{gp mark 0}{(3.869,3.464)}
\gppoint{gp mark 0}{(3.869,3.168)}
\gppoint{gp mark 0}{(3.869,3.295)}
\gppoint{gp mark 0}{(3.869,3.333)}
\gppoint{gp mark 0}{(3.869,3.333)}
\gppoint{gp mark 0}{(3.869,3.168)}
\gppoint{gp mark 0}{(3.869,3.168)}
\gppoint{gp mark 0}{(3.869,3.010)}
\gppoint{gp mark 0}{(3.869,3.295)}
\gppoint{gp mark 0}{(3.869,3.548)}
\gppoint{gp mark 0}{(3.869,3.094)}
\gppoint{gp mark 0}{(3.869,3.168)}
\gppoint{gp mark 0}{(3.869,3.276)}
\gppoint{gp mark 0}{(3.869,3.010)}
\gppoint{gp mark 0}{(3.869,3.434)}
\gppoint{gp mark 0}{(3.869,3.314)}
\gppoint{gp mark 0}{(3.869,3.255)}
\gppoint{gp mark 0}{(3.869,3.010)}
\gppoint{gp mark 0}{(3.869,3.314)}
\gppoint{gp mark 0}{(3.869,3.749)}
\gppoint{gp mark 0}{(3.869,3.610)}
\gppoint{gp mark 0}{(3.869,3.368)}
\gppoint{gp mark 0}{(3.869,3.168)}
\gppoint{gp mark 0}{(3.869,3.255)}
\gppoint{gp mark 0}{(3.869,3.521)}
\gppoint{gp mark 0}{(3.869,3.094)}
\gppoint{gp mark 0}{(3.869,3.622)}
\gppoint{gp mark 0}{(3.869,3.295)}
\gppoint{gp mark 0}{(3.869,3.464)}
\gppoint{gp mark 0}{(3.869,3.368)}
\gppoint{gp mark 0}{(3.869,3.689)}
\gppoint{gp mark 0}{(3.869,3.656)}
\gppoint{gp mark 0}{(3.869,3.740)}
\gppoint{gp mark 0}{(3.869,3.740)}
\gppoint{gp mark 0}{(3.869,3.094)}
\gppoint{gp mark 0}{(3.869,3.351)}
\gppoint{gp mark 0}{(3.869,3.314)}
\gppoint{gp mark 0}{(3.869,3.418)}
\gppoint{gp mark 0}{(3.869,3.740)}
\gppoint{gp mark 0}{(3.869,3.010)}
\gppoint{gp mark 0}{(3.869,3.010)}
\gppoint{gp mark 0}{(3.869,3.168)}
\gppoint{gp mark 0}{(3.869,3.493)}
\gppoint{gp mark 0}{(3.869,3.573)}
\gppoint{gp mark 0}{(3.869,3.351)}
\gppoint{gp mark 0}{(3.869,3.402)}
\gppoint{gp mark 0}{(3.869,3.710)}
\gppoint{gp mark 0}{(3.869,3.094)}
\gppoint{gp mark 0}{(3.869,3.678)}
\gppoint{gp mark 0}{(3.869,3.689)}
\gppoint{gp mark 0}{(3.869,3.507)}
\gppoint{gp mark 0}{(3.869,3.778)}
\gppoint{gp mark 0}{(3.869,3.730)}
\gppoint{gp mark 0}{(3.869,3.699)}
\gppoint{gp mark 0}{(3.869,3.418)}
\gppoint{gp mark 0}{(3.869,3.678)}
\gppoint{gp mark 0}{(3.869,3.276)}
\gppoint{gp mark 0}{(3.869,3.295)}
\gppoint{gp mark 0}{(3.869,3.276)}
\gppoint{gp mark 0}{(3.869,3.740)}
\gppoint{gp mark 0}{(3.869,3.434)}
\gppoint{gp mark 0}{(3.869,3.561)}
\gppoint{gp mark 0}{(3.869,3.144)}
\gppoint{gp mark 0}{(3.869,3.295)}
\gppoint{gp mark 0}{(3.869,3.535)}
\gppoint{gp mark 0}{(3.869,3.168)}
\gppoint{gp mark 0}{(3.869,3.402)}
\gppoint{gp mark 0}{(3.869,3.235)}
\gppoint{gp mark 0}{(3.869,3.434)}
\gppoint{gp mark 0}{(3.869,3.368)}
\gppoint{gp mark 0}{(3.869,3.434)}
\gppoint{gp mark 0}{(3.869,3.235)}
\gppoint{gp mark 0}{(3.869,3.235)}
\gppoint{gp mark 0}{(3.869,3.255)}
\gppoint{gp mark 0}{(3.869,3.418)}
\gppoint{gp mark 0}{(3.869,3.561)}
\gppoint{gp mark 0}{(3.869,3.094)}
\gppoint{gp mark 0}{(3.869,3.213)}
\gppoint{gp mark 0}{(3.869,3.961)}
\gppoint{gp mark 0}{(3.869,3.385)}
\gppoint{gp mark 0}{(3.869,3.759)}
\gppoint{gp mark 0}{(3.869,3.759)}
\gppoint{gp mark 0}{(3.869,3.276)}
\gppoint{gp mark 0}{(3.869,3.586)}
\gppoint{gp mark 0}{(3.869,3.634)}
\gppoint{gp mark 0}{(3.869,3.610)}
\gppoint{gp mark 0}{(3.869,3.351)}
\gppoint{gp mark 0}{(3.869,3.634)}
\gppoint{gp mark 0}{(3.869,3.434)}
\gppoint{gp mark 0}{(3.869,3.561)}
\gppoint{gp mark 0}{(3.869,3.535)}
\gppoint{gp mark 0}{(3.869,3.402)}
\gppoint{gp mark 0}{(3.869,3.276)}
\gppoint{gp mark 0}{(3.869,3.507)}
\gppoint{gp mark 0}{(3.869,3.586)}
\gppoint{gp mark 0}{(3.869,3.479)}
\gppoint{gp mark 0}{(3.869,3.667)}
\gppoint{gp mark 0}{(3.869,3.548)}
\gppoint{gp mark 0}{(3.869,3.656)}
\gppoint{gp mark 0}{(3.869,3.402)}
\gppoint{gp mark 0}{(3.869,3.895)}
\gppoint{gp mark 0}{(3.869,4.052)}
\gppoint{gp mark 0}{(3.869,3.678)}
\gppoint{gp mark 0}{(3.869,3.276)}
\gppoint{gp mark 0}{(3.869,3.493)}
\gppoint{gp mark 0}{(3.869,3.295)}
\gppoint{gp mark 0}{(3.869,3.295)}
\gppoint{gp mark 0}{(3.869,3.067)}
\gppoint{gp mark 0}{(3.869,3.295)}
\gppoint{gp mark 0}{(3.869,3.548)}
\gppoint{gp mark 0}{(3.869,3.368)}
\gppoint{gp mark 0}{(3.869,4.250)}
\gppoint{gp mark 0}{(3.869,3.119)}
\gppoint{gp mark 0}{(3.869,3.235)}
\gppoint{gp mark 0}{(3.869,3.333)}
\gppoint{gp mark 0}{(3.869,3.314)}
\gppoint{gp mark 0}{(3.869,3.255)}
\gppoint{gp mark 0}{(3.869,3.926)}
\gppoint{gp mark 0}{(3.869,3.434)}
\gppoint{gp mark 0}{(3.869,3.535)}
\gppoint{gp mark 0}{(3.869,3.168)}
\gppoint{gp mark 0}{(3.869,3.276)}
\gppoint{gp mark 0}{(3.869,3.276)}
\gppoint{gp mark 0}{(3.869,3.276)}
\gppoint{gp mark 0}{(3.869,3.872)}
\gppoint{gp mark 0}{(3.869,3.276)}
\gppoint{gp mark 0}{(3.869,3.548)}
\gppoint{gp mark 0}{(3.869,3.351)}
\gppoint{gp mark 0}{(3.869,3.521)}
\gppoint{gp mark 0}{(3.869,3.295)}
\gppoint{gp mark 0}{(3.869,3.449)}
\gppoint{gp mark 0}{(3.869,3.535)}
\gppoint{gp mark 0}{(3.869,3.235)}
\gppoint{gp mark 0}{(3.869,3.168)}
\gppoint{gp mark 0}{(3.869,3.368)}
\gppoint{gp mark 0}{(3.869,3.168)}
\gppoint{gp mark 0}{(3.869,4.040)}
\gppoint{gp mark 0}{(3.869,3.094)}
\gppoint{gp mark 0}{(3.869,3.434)}
\gppoint{gp mark 0}{(3.869,3.168)}
\gppoint{gp mark 0}{(3.869,3.598)}
\gppoint{gp mark 0}{(3.869,3.710)}
\gppoint{gp mark 0}{(3.869,3.094)}
\gppoint{gp mark 0}{(3.869,4.645)}
\gppoint{gp mark 0}{(3.869,3.839)}
\gppoint{gp mark 0}{(3.869,3.351)}
\gppoint{gp mark 0}{(3.869,3.276)}
\gppoint{gp mark 0}{(3.869,3.586)}
\gppoint{gp mark 0}{(3.869,3.449)}
\gppoint{gp mark 0}{(3.869,3.402)}
\gppoint{gp mark 0}{(3.869,3.144)}
\gppoint{gp mark 0}{(3.869,3.368)}
\gppoint{gp mark 0}{(3.930,3.535)}
\gppoint{gp mark 0}{(3.930,3.385)}
\gppoint{gp mark 0}{(3.930,3.940)}
\gppoint{gp mark 0}{(3.930,3.333)}
\gppoint{gp mark 0}{(3.930,3.295)}
\gppoint{gp mark 0}{(3.930,3.385)}
\gppoint{gp mark 0}{(3.930,3.402)}
\gppoint{gp mark 0}{(3.930,3.333)}
\gppoint{gp mark 0}{(3.930,3.449)}
\gppoint{gp mark 0}{(3.930,3.449)}
\gppoint{gp mark 0}{(3.930,3.191)}
\gppoint{gp mark 0}{(3.930,2.914)}
\gppoint{gp mark 0}{(3.930,3.507)}
\gppoint{gp mark 0}{(3.930,3.333)}
\gppoint{gp mark 0}{(3.930,3.276)}
\gppoint{gp mark 0}{(3.930,3.333)}
\gppoint{gp mark 0}{(3.930,3.385)}
\gppoint{gp mark 0}{(3.930,3.385)}
\gppoint{gp mark 0}{(3.930,3.333)}
\gppoint{gp mark 0}{(3.930,3.295)}
\gppoint{gp mark 0}{(3.930,3.333)}
\gppoint{gp mark 0}{(3.930,3.449)}
\gppoint{gp mark 0}{(3.930,3.276)}
\gppoint{gp mark 0}{(3.930,3.333)}
\gppoint{gp mark 0}{(3.930,3.418)}
\gppoint{gp mark 0}{(3.930,3.561)}
\gppoint{gp mark 0}{(3.930,3.333)}
\gppoint{gp mark 0}{(3.930,4.076)}
\gppoint{gp mark 0}{(3.930,3.730)}
\gppoint{gp mark 0}{(3.930,3.749)}
\gppoint{gp mark 0}{(3.930,3.333)}
\gppoint{gp mark 0}{(3.930,3.449)}
\gppoint{gp mark 0}{(3.930,3.333)}
\gppoint{gp mark 0}{(3.930,3.333)}
\gppoint{gp mark 0}{(3.930,3.418)}
\gppoint{gp mark 0}{(3.930,3.418)}
\gppoint{gp mark 0}{(3.930,3.864)}
\gppoint{gp mark 0}{(3.930,3.168)}
\gppoint{gp mark 0}{(3.930,3.351)}
\gppoint{gp mark 0}{(3.930,3.235)}
\gppoint{gp mark 0}{(3.930,3.351)}
\gppoint{gp mark 0}{(3.930,3.434)}
\gppoint{gp mark 0}{(3.930,3.385)}
\gppoint{gp mark 0}{(3.930,3.235)}
\gppoint{gp mark 0}{(3.930,3.479)}
\gppoint{gp mark 0}{(3.930,3.333)}
\gppoint{gp mark 0}{(3.930,3.333)}
\gppoint{gp mark 0}{(3.930,3.710)}
\gppoint{gp mark 0}{(3.930,3.333)}
\gppoint{gp mark 0}{(3.930,3.333)}
\gppoint{gp mark 0}{(3.930,3.351)}
\gppoint{gp mark 0}{(3.930,3.656)}
\gppoint{gp mark 0}{(3.930,3.449)}
\gppoint{gp mark 0}{(3.930,3.787)}
\gppoint{gp mark 0}{(3.930,3.656)}
\gppoint{gp mark 0}{(3.930,3.656)}
\gppoint{gp mark 0}{(3.930,4.208)}
\gppoint{gp mark 0}{(3.930,4.088)}
\gppoint{gp mark 0}{(3.930,3.740)}
\gppoint{gp mark 0}{(3.930,3.689)}
\gppoint{gp mark 0}{(3.930,3.535)}
\gppoint{gp mark 0}{(3.930,3.622)}
\gppoint{gp mark 0}{(3.930,3.535)}
\gppoint{gp mark 0}{(3.930,3.610)}
\gppoint{gp mark 0}{(3.930,3.535)}
\gppoint{gp mark 0}{(3.930,3.479)}
\gppoint{gp mark 0}{(3.930,3.535)}
\gppoint{gp mark 0}{(3.930,3.622)}
\gppoint{gp mark 0}{(3.930,3.710)}
\gppoint{gp mark 0}{(3.930,3.295)}
\gppoint{gp mark 0}{(3.930,3.961)}
\gppoint{gp mark 0}{(3.930,3.479)}
\gppoint{gp mark 0}{(3.930,3.610)}
\gppoint{gp mark 0}{(3.930,3.464)}
\gppoint{gp mark 0}{(3.930,3.295)}
\gppoint{gp mark 0}{(3.930,3.586)}
\gppoint{gp mark 0}{(3.930,3.749)}
\gppoint{gp mark 0}{(3.930,3.385)}
\gppoint{gp mark 0}{(3.930,3.235)}
\gppoint{gp mark 0}{(3.930,4.245)}
\gppoint{gp mark 0}{(3.930,3.351)}
\gppoint{gp mark 0}{(3.930,3.333)}
\gppoint{gp mark 0}{(3.930,3.864)}
\gppoint{gp mark 0}{(3.930,3.947)}
\gppoint{gp mark 0}{(3.930,3.385)}
\gppoint{gp mark 0}{(3.930,3.521)}
\gppoint{gp mark 0}{(3.930,3.235)}
\gppoint{gp mark 0}{(3.930,4.008)}
\gppoint{gp mark 0}{(3.930,3.759)}
\gppoint{gp mark 0}{(3.930,3.586)}
\gppoint{gp mark 0}{(3.930,3.168)}
\gppoint{gp mark 0}{(3.930,3.493)}
\gppoint{gp mark 0}{(3.930,3.449)}
\gppoint{gp mark 0}{(3.930,3.464)}
\gppoint{gp mark 0}{(3.930,3.548)}
\gppoint{gp mark 0}{(3.930,3.276)}
\gppoint{gp mark 0}{(3.930,3.144)}
\gppoint{gp mark 0}{(3.930,3.276)}
\gppoint{gp mark 0}{(3.930,3.449)}
\gppoint{gp mark 0}{(3.930,3.880)}
\gppoint{gp mark 0}{(3.930,3.479)}
\gppoint{gp mark 0}{(3.930,3.295)}
\gppoint{gp mark 0}{(3.930,3.418)}
\gppoint{gp mark 0}{(3.930,3.610)}
\gppoint{gp mark 0}{(3.930,3.610)}
\gppoint{gp mark 0}{(3.930,3.678)}
\gppoint{gp mark 0}{(3.930,3.535)}
\gppoint{gp mark 0}{(3.930,3.295)}
\gppoint{gp mark 0}{(3.930,3.333)}
\gppoint{gp mark 0}{(3.930,3.864)}
\gppoint{gp mark 0}{(3.930,3.778)}
\gppoint{gp mark 0}{(3.930,3.822)}
\gppoint{gp mark 0}{(3.930,3.903)}
\gppoint{gp mark 0}{(3.930,3.368)}
\gppoint{gp mark 0}{(3.930,3.759)}
\gppoint{gp mark 0}{(3.930,3.479)}
\gppoint{gp mark 0}{(3.930,3.507)}
\gppoint{gp mark 0}{(3.930,3.759)}
\gppoint{gp mark 0}{(3.930,3.168)}
\gppoint{gp mark 0}{(3.930,3.598)}
\gppoint{gp mark 0}{(3.930,3.573)}
\gppoint{gp mark 0}{(3.930,3.333)}
\gppoint{gp mark 0}{(3.930,3.598)}
\gppoint{gp mark 0}{(3.930,3.645)}
\gppoint{gp mark 0}{(3.930,3.385)}
\gppoint{gp mark 0}{(3.930,3.880)}
\gppoint{gp mark 0}{(3.930,3.168)}
\gppoint{gp mark 0}{(3.930,3.368)}
\gppoint{gp mark 0}{(3.930,3.368)}
\gppoint{gp mark 0}{(3.930,3.449)}
\gppoint{gp mark 0}{(3.930,3.418)}
\gppoint{gp mark 0}{(3.930,3.645)}
\gppoint{gp mark 0}{(3.930,3.333)}
\gppoint{gp mark 0}{(3.930,3.610)}
\gppoint{gp mark 0}{(3.930,3.449)}
\gppoint{gp mark 0}{(3.930,3.213)}
\gppoint{gp mark 0}{(3.930,3.507)}
\gppoint{gp mark 0}{(3.930,3.434)}
\gppoint{gp mark 0}{(3.930,3.235)}
\gppoint{gp mark 0}{(3.930,3.333)}
\gppoint{gp mark 0}{(3.930,3.521)}
\gppoint{gp mark 0}{(3.930,3.333)}
\gppoint{gp mark 0}{(3.930,4.127)}
\gppoint{gp mark 0}{(3.930,3.333)}
\gppoint{gp mark 0}{(3.930,3.368)}
\gppoint{gp mark 0}{(3.930,3.168)}
\gppoint{gp mark 0}{(3.930,3.995)}
\gppoint{gp mark 0}{(3.930,3.255)}
\gppoint{gp mark 0}{(3.930,3.710)}
\gppoint{gp mark 0}{(3.930,3.333)}
\gppoint{gp mark 0}{(3.930,3.333)}
\gppoint{gp mark 0}{(3.930,3.479)}
\gppoint{gp mark 0}{(3.930,3.586)}
\gppoint{gp mark 0}{(3.930,3.493)}
\gppoint{gp mark 0}{(3.930,3.449)}
\gppoint{gp mark 0}{(3.930,3.235)}
\gppoint{gp mark 0}{(3.930,3.521)}
\gppoint{gp mark 0}{(3.930,3.333)}
\gppoint{gp mark 0}{(3.930,3.333)}
\gppoint{gp mark 0}{(3.930,3.961)}
\gppoint{gp mark 0}{(3.930,3.548)}
\gppoint{gp mark 0}{(3.930,3.418)}
\gppoint{gp mark 0}{(3.930,3.768)}
\gppoint{gp mark 0}{(3.930,3.548)}
\gppoint{gp mark 0}{(3.930,3.449)}
\gppoint{gp mark 0}{(3.930,3.787)}
\gppoint{gp mark 0}{(3.930,3.787)}
\gppoint{gp mark 0}{(3.930,3.573)}
\gppoint{gp mark 0}{(3.930,3.368)}
\gppoint{gp mark 0}{(3.930,3.418)}
\gppoint{gp mark 0}{(3.930,3.778)}
\gppoint{gp mark 0}{(3.930,3.864)}
\gppoint{gp mark 0}{(3.930,4.208)}
\gppoint{gp mark 0}{(3.930,3.385)}
\gppoint{gp mark 0}{(3.930,3.622)}
\gppoint{gp mark 0}{(3.930,3.880)}
\gppoint{gp mark 0}{(3.930,3.449)}
\gppoint{gp mark 0}{(3.930,3.940)}
\gppoint{gp mark 0}{(3.930,3.947)}
\gppoint{gp mark 0}{(3.930,3.385)}
\gppoint{gp mark 0}{(3.930,3.507)}
\gppoint{gp mark 0}{(3.930,3.911)}
\gppoint{gp mark 0}{(3.930,3.333)}
\gppoint{gp mark 0}{(3.930,3.493)}
\gppoint{gp mark 0}{(3.930,3.667)}
\gppoint{gp mark 0}{(3.930,3.598)}
\gppoint{gp mark 0}{(3.930,3.094)}
\gppoint{gp mark 0}{(3.930,3.940)}
\gppoint{gp mark 0}{(3.930,3.434)}
\gppoint{gp mark 0}{(3.930,3.548)}
\gppoint{gp mark 0}{(3.930,3.368)}
\gppoint{gp mark 0}{(3.930,3.689)}
\gppoint{gp mark 0}{(3.930,3.903)}
\gppoint{gp mark 0}{(3.930,3.418)}
\gppoint{gp mark 0}{(3.930,3.368)}
\gppoint{gp mark 0}{(3.930,3.493)}
\gppoint{gp mark 0}{(3.930,3.778)}
\gppoint{gp mark 0}{(3.930,3.548)}
\gppoint{gp mark 0}{(3.930,3.402)}
\gppoint{gp mark 0}{(3.930,3.561)}
\gppoint{gp mark 0}{(3.930,3.255)}
\gppoint{gp mark 0}{(3.930,3.598)}
\gppoint{gp mark 0}{(3.930,3.010)}
\gppoint{gp mark 0}{(3.930,3.168)}
\gppoint{gp mark 0}{(3.930,3.847)}
\gppoint{gp mark 0}{(3.930,3.295)}
\gppoint{gp mark 0}{(3.930,3.561)}
\gppoint{gp mark 0}{(3.930,3.168)}
\gppoint{gp mark 0}{(3.930,3.507)}
\gppoint{gp mark 0}{(3.930,3.548)}
\gppoint{gp mark 0}{(3.930,3.235)}
\gppoint{gp mark 0}{(3.930,4.334)}
\gppoint{gp mark 0}{(3.930,3.449)}
\gppoint{gp mark 0}{(3.930,3.168)}
\gppoint{gp mark 0}{(3.930,3.535)}
\gppoint{gp mark 0}{(3.930,2.979)}
\gppoint{gp mark 0}{(3.930,3.535)}
\gppoint{gp mark 0}{(3.930,3.464)}
\gppoint{gp mark 0}{(3.930,3.333)}
\gppoint{gp mark 0}{(3.930,3.535)}
\gppoint{gp mark 0}{(3.930,3.449)}
\gppoint{gp mark 0}{(3.930,3.168)}
\gppoint{gp mark 0}{(3.930,3.535)}
\gppoint{gp mark 0}{(3.930,3.402)}
\gppoint{gp mark 0}{(3.930,3.276)}
\gppoint{gp mark 0}{(3.930,3.449)}
\gppoint{gp mark 0}{(3.930,3.449)}
\gppoint{gp mark 0}{(3.930,3.385)}
\gppoint{gp mark 0}{(3.930,3.479)}
\gppoint{gp mark 0}{(3.930,3.479)}
\gppoint{gp mark 0}{(3.930,3.464)}
\gppoint{gp mark 0}{(3.930,3.535)}
\gppoint{gp mark 0}{(3.930,3.434)}
\gppoint{gp mark 0}{(3.930,3.493)}
\gppoint{gp mark 0}{(3.930,3.276)}
\gppoint{gp mark 0}{(3.930,3.710)}
\gppoint{gp mark 0}{(3.930,3.385)}
\gppoint{gp mark 0}{(3.930,3.094)}
\gppoint{gp mark 0}{(3.930,3.235)}
\gppoint{gp mark 0}{(3.930,3.168)}
\gppoint{gp mark 0}{(3.930,3.796)}
\gppoint{gp mark 0}{(3.930,3.548)}
\gppoint{gp mark 0}{(3.930,3.314)}
\gppoint{gp mark 0}{(3.930,3.213)}
\gppoint{gp mark 0}{(3.930,3.235)}
\gppoint{gp mark 0}{(3.930,3.699)}
\gppoint{gp mark 0}{(3.930,3.493)}
\gppoint{gp mark 0}{(3.930,3.449)}
\gppoint{gp mark 0}{(3.930,3.168)}
\gppoint{gp mark 0}{(3.930,3.276)}
\gppoint{gp mark 0}{(3.930,3.778)}
\gppoint{gp mark 0}{(3.930,4.194)}
\gppoint{gp mark 0}{(3.930,4.132)}
\gppoint{gp mark 0}{(3.930,3.094)}
\gppoint{gp mark 0}{(3.930,3.255)}
\gppoint{gp mark 0}{(3.930,3.094)}
\gppoint{gp mark 0}{(3.930,3.634)}
\gppoint{gp mark 0}{(3.930,3.333)}
\gppoint{gp mark 0}{(3.930,3.689)}
\gppoint{gp mark 0}{(3.930,3.213)}
\gppoint{gp mark 0}{(3.930,3.796)}
\gppoint{gp mark 0}{(3.930,3.434)}
\gppoint{gp mark 0}{(3.930,3.434)}
\gppoint{gp mark 0}{(3.930,3.507)}
\gppoint{gp mark 0}{(3.930,3.479)}
\gppoint{gp mark 0}{(3.930,3.168)}
\gppoint{gp mark 0}{(3.930,3.235)}
\gppoint{gp mark 0}{(3.930,3.822)}
\gppoint{gp mark 0}{(3.930,3.094)}
\gppoint{gp mark 0}{(3.930,3.903)}
\gppoint{gp mark 0}{(3.930,3.168)}
\gppoint{gp mark 0}{(3.930,3.622)}
\gppoint{gp mark 0}{(3.930,3.880)}
\gppoint{gp mark 0}{(3.930,3.975)}
\gppoint{gp mark 0}{(3.930,3.521)}
\gppoint{gp mark 0}{(3.930,3.235)}
\gppoint{gp mark 0}{(3.930,3.351)}
\gppoint{gp mark 0}{(3.930,3.235)}
\gppoint{gp mark 0}{(3.930,3.586)}
\gppoint{gp mark 0}{(3.930,3.507)}
\gppoint{gp mark 0}{(3.930,3.235)}
\gppoint{gp mark 0}{(3.930,3.434)}
\gppoint{gp mark 0}{(3.930,3.351)}
\gppoint{gp mark 0}{(3.930,3.235)}
\gppoint{gp mark 0}{(3.930,3.418)}
\gppoint{gp mark 0}{(3.930,3.561)}
\gppoint{gp mark 0}{(3.930,3.418)}
\gppoint{gp mark 0}{(3.930,3.418)}
\gppoint{gp mark 0}{(3.930,3.645)}
\gppoint{gp mark 0}{(3.930,3.586)}
\gppoint{gp mark 0}{(3.930,3.235)}
\gppoint{gp mark 0}{(3.930,3.168)}
\gppoint{gp mark 0}{(3.930,3.548)}
\gppoint{gp mark 0}{(3.930,3.586)}
\gppoint{gp mark 0}{(3.930,3.385)}
\gppoint{gp mark 0}{(3.930,3.235)}
\gppoint{gp mark 0}{(3.930,3.351)}
\gppoint{gp mark 0}{(3.930,3.880)}
\gppoint{gp mark 0}{(3.930,3.507)}
\gppoint{gp mark 0}{(3.930,3.449)}
\gppoint{gp mark 0}{(3.930,3.235)}
\gppoint{gp mark 0}{(3.930,4.058)}
\gppoint{gp mark 0}{(3.930,3.295)}
\gppoint{gp mark 0}{(3.930,4.408)}
\gppoint{gp mark 0}{(3.930,3.418)}
\gppoint{gp mark 0}{(3.930,3.464)}
\gppoint{gp mark 0}{(3.930,3.507)}
\gppoint{gp mark 0}{(3.930,3.778)}
\gppoint{gp mark 0}{(3.930,3.813)}
\gppoint{gp mark 0}{(3.930,3.645)}
\gppoint{gp mark 0}{(3.930,3.535)}
\gppoint{gp mark 0}{(3.930,3.678)}
\gppoint{gp mark 0}{(3.930,3.464)}
\gppoint{gp mark 0}{(3.930,3.507)}
\gppoint{gp mark 0}{(3.930,3.333)}
\gppoint{gp mark 0}{(3.930,3.295)}
\gppoint{gp mark 0}{(3.930,3.168)}
\gppoint{gp mark 0}{(3.930,3.385)}
\gppoint{gp mark 0}{(3.930,3.434)}
\gppoint{gp mark 0}{(3.930,3.402)}
\gppoint{gp mark 0}{(3.930,3.333)}
\gppoint{gp mark 0}{(3.930,3.333)}
\gppoint{gp mark 0}{(3.930,3.295)}
\gppoint{gp mark 0}{(3.930,3.535)}
\gppoint{gp mark 0}{(3.930,3.010)}
\gppoint{gp mark 0}{(3.930,3.385)}
\gppoint{gp mark 0}{(3.930,3.493)}
\gppoint{gp mark 0}{(3.930,3.598)}
\gppoint{gp mark 0}{(3.930,3.255)}
\gppoint{gp mark 0}{(3.930,3.213)}
\gppoint{gp mark 0}{(3.930,3.295)}
\gppoint{gp mark 0}{(3.930,3.168)}
\gppoint{gp mark 0}{(3.930,3.493)}
\gppoint{gp mark 0}{(3.930,3.598)}
\gppoint{gp mark 0}{(3.930,3.351)}
\gppoint{gp mark 0}{(3.930,3.385)}
\gppoint{gp mark 0}{(3.930,3.918)}
\gppoint{gp mark 0}{(3.930,3.598)}
\gppoint{gp mark 0}{(3.930,3.940)}
\gppoint{gp mark 0}{(3.930,3.168)}
\gppoint{gp mark 0}{(3.930,3.573)}
\gppoint{gp mark 0}{(3.930,3.434)}
\gppoint{gp mark 0}{(3.930,3.598)}
\gppoint{gp mark 0}{(3.930,4.198)}
\gppoint{gp mark 0}{(3.930,3.645)}
\gppoint{gp mark 0}{(3.930,3.586)}
\gppoint{gp mark 0}{(3.930,3.561)}
\gppoint{gp mark 0}{(3.930,3.521)}
\gppoint{gp mark 0}{(3.930,3.634)}
\gppoint{gp mark 0}{(3.930,3.418)}
\gppoint{gp mark 0}{(3.930,3.622)}
\gppoint{gp mark 0}{(3.930,3.168)}
\gppoint{gp mark 0}{(3.930,3.787)}
\gppoint{gp mark 0}{(3.930,3.235)}
\gppoint{gp mark 0}{(3.930,3.926)}
\gppoint{gp mark 0}{(3.930,3.464)}
\gppoint{gp mark 0}{(3.930,3.521)}
\gppoint{gp mark 0}{(3.930,3.385)}
\gppoint{gp mark 0}{(3.930,3.351)}
\gppoint{gp mark 0}{(3.930,3.759)}
\gppoint{gp mark 0}{(3.930,3.822)}
\gppoint{gp mark 0}{(3.930,3.314)}
\gppoint{gp mark 0}{(3.930,3.995)}
\gppoint{gp mark 0}{(3.930,3.954)}
\gppoint{gp mark 0}{(3.930,3.961)}
\gppoint{gp mark 0}{(3.930,3.168)}
\gppoint{gp mark 0}{(3.930,3.276)}
\gppoint{gp mark 0}{(3.930,3.333)}
\gppoint{gp mark 0}{(3.930,3.507)}
\gppoint{gp mark 0}{(3.930,4.164)}
\gppoint{gp mark 0}{(3.930,4.164)}
\gppoint{gp mark 0}{(3.930,3.954)}
\gppoint{gp mark 0}{(3.930,3.493)}
\gppoint{gp mark 0}{(3.930,3.385)}
\gppoint{gp mark 0}{(3.930,3.493)}
\gppoint{gp mark 0}{(3.930,3.434)}
\gppoint{gp mark 0}{(3.930,3.235)}
\gppoint{gp mark 0}{(3.930,3.839)}
\gppoint{gp mark 0}{(3.930,3.740)}
\gppoint{gp mark 0}{(3.930,3.094)}
\gppoint{gp mark 0}{(3.930,3.235)}
\gppoint{gp mark 0}{(3.930,3.333)}
\gppoint{gp mark 0}{(3.930,3.168)}
\gppoint{gp mark 0}{(3.930,3.847)}
\gppoint{gp mark 0}{(3.930,3.598)}
\gppoint{gp mark 0}{(3.930,3.314)}
\gppoint{gp mark 0}{(3.930,3.918)}
\gppoint{gp mark 0}{(3.930,3.434)}
\gppoint{gp mark 0}{(3.930,3.895)}
\gppoint{gp mark 0}{(3.930,3.314)}
\gppoint{gp mark 0}{(3.930,3.464)}
\gppoint{gp mark 0}{(3.930,3.493)}
\gppoint{gp mark 0}{(3.930,3.710)}
\gppoint{gp mark 0}{(3.930,3.903)}
\gppoint{gp mark 0}{(3.930,3.521)}
\gppoint{gp mark 0}{(3.930,3.521)}
\gppoint{gp mark 0}{(3.930,3.213)}
\gppoint{gp mark 0}{(3.930,3.314)}
\gppoint{gp mark 0}{(3.930,3.434)}
\gppoint{gp mark 0}{(3.930,3.276)}
\gppoint{gp mark 0}{(3.930,3.418)}
\gppoint{gp mark 0}{(3.930,3.418)}
\gppoint{gp mark 0}{(3.930,3.847)}
\gppoint{gp mark 0}{(3.930,3.610)}
\gppoint{gp mark 0}{(3.930,3.954)}
\gppoint{gp mark 0}{(3.930,3.926)}
\gppoint{gp mark 0}{(3.930,3.094)}
\gppoint{gp mark 0}{(3.930,3.535)}
\gppoint{gp mark 0}{(3.930,3.094)}
\gppoint{gp mark 0}{(3.930,3.954)}
\gppoint{gp mark 0}{(3.930,3.464)}
\gppoint{gp mark 0}{(3.930,3.314)}
\gppoint{gp mark 0}{(3.930,3.778)}
\gppoint{gp mark 0}{(3.930,3.479)}
\gppoint{gp mark 0}{(3.930,3.314)}
\gppoint{gp mark 0}{(3.930,3.235)}
\gppoint{gp mark 0}{(3.930,3.710)}
\gppoint{gp mark 0}{(3.930,3.168)}
\gppoint{gp mark 0}{(3.930,3.759)}
\gppoint{gp mark 0}{(3.930,3.903)}
\gppoint{gp mark 0}{(3.930,3.351)}
\gppoint{gp mark 0}{(3.930,3.710)}
\gppoint{gp mark 0}{(3.930,3.479)}
\gppoint{gp mark 0}{(3.930,3.954)}
\gppoint{gp mark 0}{(3.930,3.351)}
\gppoint{gp mark 0}{(3.987,3.479)}
\gppoint{gp mark 0}{(3.987,3.656)}
\gppoint{gp mark 0}{(3.987,3.667)}
\gppoint{gp mark 0}{(3.987,3.010)}
\gppoint{gp mark 0}{(3.987,3.385)}
\gppoint{gp mark 0}{(3.987,3.385)}
\gppoint{gp mark 0}{(3.987,3.385)}
\gppoint{gp mark 0}{(3.987,3.385)}
\gppoint{gp mark 0}{(3.987,4.250)}
\gppoint{gp mark 0}{(3.987,3.213)}
\gppoint{gp mark 0}{(3.987,3.787)}
\gppoint{gp mark 0}{(3.987,3.094)}
\gppoint{gp mark 0}{(3.987,4.121)}
\gppoint{gp mark 0}{(3.987,3.385)}
\gppoint{gp mark 0}{(3.987,3.507)}
\gppoint{gp mark 0}{(3.987,3.449)}
\gppoint{gp mark 0}{(3.987,3.094)}
\gppoint{gp mark 0}{(3.987,3.276)}
\gppoint{gp mark 0}{(3.987,3.235)}
\gppoint{gp mark 0}{(3.987,3.940)}
\gppoint{gp mark 0}{(3.987,3.168)}
\gppoint{gp mark 0}{(3.987,3.911)}
\gppoint{gp mark 0}{(3.987,3.333)}
\gppoint{gp mark 0}{(3.987,3.213)}
\gppoint{gp mark 0}{(3.987,3.656)}
\gppoint{gp mark 0}{(3.987,3.759)}
\gppoint{gp mark 0}{(3.987,3.449)}
\gppoint{gp mark 0}{(3.987,3.598)}
\gppoint{gp mark 0}{(3.987,3.385)}
\gppoint{gp mark 0}{(3.987,4.040)}
\gppoint{gp mark 0}{(3.987,3.434)}
\gppoint{gp mark 0}{(3.987,3.235)}
\gppoint{gp mark 0}{(3.987,3.094)}
\gppoint{gp mark 0}{(3.987,3.982)}
\gppoint{gp mark 0}{(3.987,3.678)}
\gppoint{gp mark 0}{(3.987,3.351)}
\gppoint{gp mark 0}{(3.987,3.434)}
\gppoint{gp mark 0}{(3.987,3.598)}
\gppoint{gp mark 0}{(3.987,3.634)}
\gppoint{gp mark 0}{(3.987,3.385)}
\gppoint{gp mark 0}{(3.987,3.796)}
\gppoint{gp mark 0}{(3.987,3.418)}
\gppoint{gp mark 0}{(3.987,3.548)}
\gppoint{gp mark 0}{(3.987,3.385)}
\gppoint{gp mark 0}{(3.987,3.449)}
\gppoint{gp mark 0}{(3.987,3.385)}
\gppoint{gp mark 0}{(3.987,3.351)}
\gppoint{gp mark 0}{(3.987,3.740)}
\gppoint{gp mark 0}{(3.987,3.235)}
\gppoint{gp mark 0}{(3.987,3.295)}
\gppoint{gp mark 0}{(3.987,3.295)}
\gppoint{gp mark 0}{(3.987,3.464)}
\gppoint{gp mark 0}{(3.987,3.610)}
\gppoint{gp mark 0}{(3.987,3.586)}
\gppoint{gp mark 0}{(3.987,4.040)}
\gppoint{gp mark 0}{(3.987,3.295)}
\gppoint{gp mark 0}{(3.987,3.749)}
\gppoint{gp mark 0}{(3.987,3.385)}
\gppoint{gp mark 0}{(3.987,3.449)}
\gppoint{gp mark 0}{(3.987,3.010)}
\gppoint{gp mark 0}{(3.987,3.402)}
\gppoint{gp mark 0}{(3.987,3.434)}
\gppoint{gp mark 0}{(3.987,3.699)}
\gppoint{gp mark 0}{(3.987,3.235)}
\gppoint{gp mark 0}{(3.987,3.610)}
\gppoint{gp mark 0}{(3.987,3.872)}
\gppoint{gp mark 0}{(3.987,3.235)}
\gppoint{gp mark 0}{(3.987,3.434)}
\gppoint{gp mark 0}{(3.987,3.295)}
\gppoint{gp mark 0}{(3.987,3.235)}
\gppoint{gp mark 0}{(3.987,3.418)}
\gppoint{gp mark 0}{(3.987,3.418)}
\gppoint{gp mark 0}{(3.987,3.947)}
\gppoint{gp mark 0}{(3.987,3.168)}
\gppoint{gp mark 0}{(3.987,3.434)}
\gppoint{gp mark 0}{(3.987,3.968)}
\gppoint{gp mark 0}{(3.987,3.434)}
\gppoint{gp mark 0}{(3.987,3.434)}
\gppoint{gp mark 0}{(3.987,3.213)}
\gppoint{gp mark 0}{(3.987,3.464)}
\gppoint{gp mark 0}{(3.987,3.168)}
\gppoint{gp mark 0}{(3.987,3.235)}
\gppoint{gp mark 0}{(3.987,3.235)}
\gppoint{gp mark 0}{(3.987,3.418)}
\gppoint{gp mark 0}{(3.987,3.954)}
\gppoint{gp mark 0}{(3.987,3.493)}
\gppoint{gp mark 0}{(3.987,3.479)}
\gppoint{gp mark 0}{(3.987,3.235)}
\gppoint{gp mark 0}{(3.987,3.235)}
\gppoint{gp mark 0}{(3.987,3.235)}
\gppoint{gp mark 0}{(3.987,3.385)}
\gppoint{gp mark 0}{(3.987,3.385)}
\gppoint{gp mark 0}{(3.987,3.449)}
\gppoint{gp mark 0}{(3.987,3.235)}
\gppoint{gp mark 0}{(3.987,3.385)}
\gppoint{gp mark 0}{(3.987,3.385)}
\gppoint{gp mark 0}{(3.987,3.831)}
\gppoint{gp mark 0}{(3.987,3.385)}
\gppoint{gp mark 0}{(3.987,3.822)}
\gppoint{gp mark 0}{(3.987,3.351)}
\gppoint{gp mark 0}{(3.987,3.656)}
\gppoint{gp mark 0}{(3.987,3.276)}
\gppoint{gp mark 0}{(3.987,3.479)}
\gppoint{gp mark 0}{(3.987,3.385)}
\gppoint{gp mark 0}{(3.987,3.235)}
\gppoint{gp mark 0}{(3.987,3.479)}
\gppoint{gp mark 0}{(3.987,3.235)}
\gppoint{gp mark 0}{(3.987,3.385)}
\gppoint{gp mark 0}{(3.987,3.479)}
\gppoint{gp mark 0}{(3.987,3.314)}
\gppoint{gp mark 0}{(3.987,3.235)}
\gppoint{gp mark 0}{(3.987,3.689)}
\gppoint{gp mark 0}{(3.987,3.449)}
\gppoint{gp mark 0}{(3.987,3.645)}
\gppoint{gp mark 0}{(3.987,3.968)}
\gppoint{gp mark 0}{(3.987,3.548)}
\gppoint{gp mark 0}{(3.987,3.720)}
\gppoint{gp mark 0}{(3.987,3.645)}
\gppoint{gp mark 0}{(3.987,4.021)}
\gppoint{gp mark 0}{(3.987,3.385)}
\gppoint{gp mark 0}{(3.987,3.561)}
\gppoint{gp mark 0}{(3.987,4.021)}
\gppoint{gp mark 0}{(3.987,3.385)}
\gppoint{gp mark 0}{(3.987,4.121)}
\gppoint{gp mark 0}{(3.987,3.385)}
\gppoint{gp mark 0}{(3.987,3.385)}
\gppoint{gp mark 0}{(3.987,3.778)}
\gppoint{gp mark 0}{(3.987,3.385)}
\gppoint{gp mark 0}{(3.987,3.385)}
\gppoint{gp mark 0}{(3.987,3.535)}
\gppoint{gp mark 0}{(3.987,3.479)}
\gppoint{gp mark 0}{(3.987,3.094)}
\gppoint{gp mark 0}{(3.987,3.385)}
\gppoint{gp mark 0}{(3.987,3.561)}
\gppoint{gp mark 0}{(3.987,3.295)}
\gppoint{gp mark 0}{(3.987,3.535)}
\gppoint{gp mark 0}{(3.987,3.740)}
\gppoint{gp mark 0}{(3.987,3.548)}
\gppoint{gp mark 0}{(3.987,3.610)}
\gppoint{gp mark 0}{(3.987,3.561)}
\gppoint{gp mark 0}{(3.987,3.839)}
\gppoint{gp mark 0}{(3.987,3.351)}
\gppoint{gp mark 0}{(3.987,3.235)}
\gppoint{gp mark 0}{(3.987,3.010)}
\gppoint{gp mark 0}{(3.987,3.385)}
\gppoint{gp mark 0}{(3.987,3.418)}
\gppoint{gp mark 0}{(3.987,3.235)}
\gppoint{gp mark 0}{(3.987,3.940)}
\gppoint{gp mark 0}{(3.987,3.759)}
\gppoint{gp mark 0}{(3.987,3.385)}
\gppoint{gp mark 0}{(3.987,3.402)}
\gppoint{gp mark 0}{(3.987,3.333)}
\gppoint{gp mark 0}{(3.987,3.805)}
\gppoint{gp mark 0}{(3.987,3.385)}
\gppoint{gp mark 0}{(3.987,3.895)}
\gppoint{gp mark 0}{(3.987,3.847)}
\gppoint{gp mark 0}{(3.987,3.351)}
\gppoint{gp mark 0}{(3.987,3.521)}
\gppoint{gp mark 0}{(3.987,3.656)}
\gppoint{gp mark 0}{(3.987,3.168)}
\gppoint{gp mark 0}{(3.987,3.418)}
\gppoint{gp mark 0}{(3.987,3.678)}
\gppoint{gp mark 0}{(3.987,3.144)}
\gppoint{gp mark 0}{(3.987,3.634)}
\gppoint{gp mark 0}{(3.987,3.730)}
\gppoint{gp mark 0}{(3.987,3.385)}
\gppoint{gp mark 0}{(3.987,3.385)}
\gppoint{gp mark 0}{(3.987,4.143)}
\gppoint{gp mark 0}{(3.987,3.610)}
\gppoint{gp mark 0}{(3.987,3.385)}
\gppoint{gp mark 0}{(3.987,3.678)}
\gppoint{gp mark 0}{(3.987,3.276)}
\gppoint{gp mark 0}{(3.987,3.385)}
\gppoint{gp mark 0}{(3.987,3.333)}
\gppoint{gp mark 0}{(3.987,3.295)}
\gppoint{gp mark 0}{(3.987,3.535)}
\gppoint{gp mark 0}{(3.987,3.385)}
\gppoint{gp mark 0}{(3.987,3.333)}
\gppoint{gp mark 0}{(3.987,3.561)}
\gppoint{gp mark 0}{(3.987,3.333)}
\gppoint{gp mark 0}{(3.987,3.385)}
\gppoint{gp mark 0}{(3.987,3.548)}
\gppoint{gp mark 0}{(3.987,3.689)}
\gppoint{gp mark 0}{(3.987,3.768)}
\gppoint{gp mark 0}{(3.987,3.449)}
\gppoint{gp mark 0}{(3.987,3.368)}
\gppoint{gp mark 0}{(3.987,3.678)}
\gppoint{gp mark 0}{(3.987,3.333)}
\gppoint{gp mark 0}{(3.987,3.449)}
\gppoint{gp mark 0}{(3.987,3.385)}
\gppoint{gp mark 0}{(3.987,3.888)}
\gppoint{gp mark 0}{(3.987,3.368)}
\gppoint{gp mark 0}{(3.987,3.434)}
\gppoint{gp mark 0}{(3.987,3.385)}
\gppoint{gp mark 0}{(3.987,3.720)}
\gppoint{gp mark 0}{(3.987,3.385)}
\gppoint{gp mark 0}{(3.987,3.449)}
\gppoint{gp mark 0}{(3.987,3.235)}
\gppoint{gp mark 0}{(3.987,3.295)}
\gppoint{gp mark 0}{(3.987,3.385)}
\gppoint{gp mark 0}{(3.987,3.768)}
\gppoint{gp mark 0}{(3.987,3.521)}
\gppoint{gp mark 0}{(3.987,3.368)}
\gppoint{gp mark 0}{(3.987,3.903)}
\gppoint{gp mark 0}{(3.987,3.622)}
\gppoint{gp mark 0}{(3.987,3.911)}
\gppoint{gp mark 0}{(3.987,3.796)}
\gppoint{gp mark 0}{(3.987,3.561)}
\gppoint{gp mark 0}{(3.987,3.493)}
\gppoint{gp mark 0}{(3.987,3.493)}
\gppoint{gp mark 0}{(3.987,3.094)}
\gppoint{gp mark 0}{(3.987,3.586)}
\gppoint{gp mark 0}{(3.987,3.645)}
\gppoint{gp mark 0}{(3.987,3.385)}
\gppoint{gp mark 0}{(3.987,3.333)}
\gppoint{gp mark 0}{(3.987,3.434)}
\gppoint{gp mark 0}{(3.987,3.434)}
\gppoint{gp mark 0}{(3.987,3.493)}
\gppoint{gp mark 0}{(3.987,3.434)}
\gppoint{gp mark 0}{(3.987,3.434)}
\gppoint{gp mark 0}{(3.987,3.622)}
\gppoint{gp mark 0}{(3.987,3.434)}
\gppoint{gp mark 0}{(3.987,3.598)}
\gppoint{gp mark 0}{(3.987,3.434)}
\gppoint{gp mark 0}{(3.987,3.235)}
\gppoint{gp mark 0}{(3.987,3.521)}
\gppoint{gp mark 0}{(3.987,3.168)}
\gppoint{gp mark 0}{(3.987,3.535)}
\gppoint{gp mark 0}{(3.987,4.222)}
\gppoint{gp mark 0}{(3.987,3.385)}
\gppoint{gp mark 0}{(3.987,3.385)}
\gppoint{gp mark 0}{(3.987,3.295)}
\gppoint{gp mark 0}{(3.987,3.434)}
\gppoint{gp mark 0}{(3.987,3.402)}
\gppoint{gp mark 0}{(3.987,3.586)}
\gppoint{gp mark 0}{(3.987,3.699)}
\gppoint{gp mark 0}{(3.987,3.535)}
\gppoint{gp mark 0}{(3.987,3.730)}
\gppoint{gp mark 0}{(3.987,3.768)}
\gppoint{gp mark 0}{(3.987,3.493)}
\gppoint{gp mark 0}{(3.987,3.548)}
\gppoint{gp mark 0}{(3.987,3.385)}
\gppoint{gp mark 0}{(3.987,3.314)}
\gppoint{gp mark 0}{(3.987,3.402)}
\gppoint{gp mark 0}{(3.987,3.213)}
\gppoint{gp mark 0}{(3.987,3.235)}
\gppoint{gp mark 0}{(3.987,3.402)}
\gppoint{gp mark 0}{(3.987,3.573)}
\gppoint{gp mark 0}{(3.987,3.368)}
\gppoint{gp mark 0}{(3.987,3.235)}
\gppoint{gp mark 0}{(3.987,3.351)}
\gppoint{gp mark 0}{(3.987,3.351)}
\gppoint{gp mark 0}{(3.987,3.573)}
\gppoint{gp mark 0}{(3.987,3.402)}
\gppoint{gp mark 0}{(3.987,3.768)}
\gppoint{gp mark 0}{(3.987,3.720)}
\gppoint{gp mark 0}{(3.987,3.573)}
\gppoint{gp mark 0}{(3.987,3.689)}
\gppoint{gp mark 0}{(3.987,3.402)}
\gppoint{gp mark 0}{(3.987,3.385)}
\gppoint{gp mark 0}{(3.987,3.385)}
\gppoint{gp mark 0}{(3.987,3.385)}
\gppoint{gp mark 0}{(3.987,3.235)}
\gppoint{gp mark 0}{(3.987,3.535)}
\gppoint{gp mark 0}{(3.987,3.434)}
\gppoint{gp mark 0}{(3.987,3.235)}
\gppoint{gp mark 0}{(3.987,3.402)}
\gppoint{gp mark 0}{(3.987,3.918)}
\gppoint{gp mark 0}{(3.987,3.385)}
\gppoint{gp mark 0}{(3.987,3.385)}
\gppoint{gp mark 0}{(3.987,4.099)}
\gppoint{gp mark 0}{(3.987,3.385)}
\gppoint{gp mark 0}{(3.987,3.333)}
\gppoint{gp mark 0}{(3.987,4.093)}
\gppoint{gp mark 0}{(3.987,3.634)}
\gppoint{gp mark 0}{(3.987,3.235)}
\gppoint{gp mark 0}{(3.987,3.235)}
\gppoint{gp mark 0}{(3.987,3.535)}
\gppoint{gp mark 0}{(3.987,3.235)}
\gppoint{gp mark 0}{(3.987,3.402)}
\gppoint{gp mark 0}{(3.987,3.385)}
\gppoint{gp mark 0}{(3.987,3.402)}
\gppoint{gp mark 0}{(3.987,3.434)}
\gppoint{gp mark 0}{(3.987,4.064)}
\gppoint{gp mark 0}{(3.987,3.094)}
\gppoint{gp mark 0}{(3.987,3.351)}
\gppoint{gp mark 0}{(3.987,3.434)}
\gppoint{gp mark 0}{(3.987,3.493)}
\gppoint{gp mark 0}{(3.987,3.434)}
\gppoint{gp mark 0}{(3.987,3.434)}
\gppoint{gp mark 0}{(3.987,3.235)}
\gppoint{gp mark 0}{(3.987,3.235)}
\gppoint{gp mark 0}{(3.987,3.449)}
\gppoint{gp mark 0}{(3.987,3.333)}
\gppoint{gp mark 0}{(3.987,3.699)}
\gppoint{gp mark 0}{(3.987,3.235)}
\gppoint{gp mark 0}{(3.987,3.235)}
\gppoint{gp mark 0}{(3.987,3.402)}
\gppoint{gp mark 0}{(3.987,3.479)}
\gppoint{gp mark 0}{(3.987,3.479)}
\gppoint{gp mark 0}{(3.987,3.598)}
\gppoint{gp mark 0}{(3.987,3.699)}
\gppoint{gp mark 0}{(3.987,3.235)}
\gppoint{gp mark 0}{(3.987,3.434)}
\gppoint{gp mark 0}{(3.987,3.402)}
\gppoint{gp mark 0}{(3.987,3.656)}
\gppoint{gp mark 0}{(3.987,3.235)}
\gppoint{gp mark 0}{(3.987,3.295)}
\gppoint{gp mark 0}{(3.987,3.314)}
\gppoint{gp mark 0}{(3.987,3.911)}
\gppoint{gp mark 0}{(3.987,3.586)}
\gppoint{gp mark 0}{(3.987,3.740)}
\gppoint{gp mark 0}{(3.987,3.368)}
\gppoint{gp mark 0}{(3.987,3.975)}
\gppoint{gp mark 0}{(3.987,3.449)}
\gppoint{gp mark 0}{(3.987,3.586)}
\gppoint{gp mark 0}{(3.987,3.831)}
\gppoint{gp mark 0}{(3.987,3.235)}
\gppoint{gp mark 0}{(3.987,3.385)}
\gppoint{gp mark 0}{(3.987,3.507)}
\gppoint{gp mark 0}{(3.987,3.168)}
\gppoint{gp mark 0}{(3.987,3.385)}
\gppoint{gp mark 0}{(3.987,3.521)}
\gppoint{gp mark 0}{(3.987,3.521)}
\gppoint{gp mark 0}{(3.987,3.521)}
\gppoint{gp mark 0}{(3.987,4.401)}
\gppoint{gp mark 0}{(3.987,3.521)}
\gppoint{gp mark 0}{(3.987,3.521)}
\gppoint{gp mark 0}{(3.987,3.521)}
\gppoint{gp mark 0}{(3.987,3.235)}
\gppoint{gp mark 0}{(3.987,4.372)}
\gppoint{gp mark 0}{(3.987,3.235)}
\gppoint{gp mark 0}{(3.987,4.040)}
\gppoint{gp mark 0}{(3.987,3.235)}
\gppoint{gp mark 0}{(3.987,3.720)}
\gppoint{gp mark 0}{(3.987,4.127)}
\gppoint{gp mark 0}{(3.987,3.634)}
\gppoint{gp mark 0}{(3.987,3.521)}
\gppoint{gp mark 0}{(3.987,4.137)}
\gppoint{gp mark 0}{(3.987,3.622)}
\gppoint{gp mark 0}{(3.987,3.235)}
\gppoint{gp mark 0}{(3.987,3.094)}
\gppoint{gp mark 0}{(3.987,3.368)}
\gppoint{gp mark 0}{(3.987,3.847)}
\gppoint{gp mark 0}{(3.987,3.847)}
\gppoint{gp mark 0}{(3.987,3.521)}
\gppoint{gp mark 0}{(3.987,3.368)}
\gppoint{gp mark 0}{(3.987,3.368)}
\gppoint{gp mark 0}{(3.987,3.368)}
\gppoint{gp mark 0}{(3.987,3.295)}
\gppoint{gp mark 0}{(3.987,3.368)}
\gppoint{gp mark 0}{(3.987,3.276)}
\gppoint{gp mark 0}{(3.987,3.368)}
\gppoint{gp mark 0}{(3.987,3.507)}
\gppoint{gp mark 0}{(3.987,3.479)}
\gppoint{gp mark 0}{(3.987,3.479)}
\gppoint{gp mark 0}{(3.987,3.368)}
\gppoint{gp mark 0}{(3.987,3.368)}
\gppoint{gp mark 0}{(3.987,4.143)}
\gppoint{gp mark 0}{(3.987,3.010)}
\gppoint{gp mark 0}{(3.987,3.667)}
\gppoint{gp mark 0}{(3.987,3.368)}
\gppoint{gp mark 0}{(3.987,3.368)}
\gppoint{gp mark 0}{(3.987,3.235)}
\gppoint{gp mark 0}{(3.987,3.351)}
\gppoint{gp mark 0}{(3.987,3.368)}
\gppoint{gp mark 0}{(3.987,3.368)}
\gppoint{gp mark 0}{(3.987,3.368)}
\gppoint{gp mark 0}{(3.987,3.368)}
\gppoint{gp mark 0}{(3.987,3.385)}
\gppoint{gp mark 0}{(3.987,3.368)}
\gppoint{gp mark 0}{(3.987,3.368)}
\gppoint{gp mark 0}{(3.987,3.368)}
\gppoint{gp mark 0}{(3.987,3.368)}
\gppoint{gp mark 0}{(3.987,3.368)}
\gppoint{gp mark 0}{(3.987,3.276)}
\gppoint{gp mark 0}{(3.987,3.434)}
\gppoint{gp mark 0}{(3.987,3.434)}
\gppoint{gp mark 0}{(3.987,4.429)}
\gppoint{gp mark 0}{(3.987,3.548)}
\gppoint{gp mark 0}{(3.987,3.119)}
\gppoint{gp mark 0}{(3.987,3.368)}
\gppoint{gp mark 0}{(3.987,4.267)}
\gppoint{gp mark 0}{(3.987,4.046)}
\gppoint{gp mark 0}{(3.987,3.333)}
\gppoint{gp mark 0}{(3.987,4.259)}
\gppoint{gp mark 0}{(3.987,3.434)}
\gppoint{gp mark 0}{(3.987,3.598)}
\gppoint{gp mark 0}{(3.987,3.385)}
\gppoint{gp mark 0}{(3.987,3.598)}
\gppoint{gp mark 0}{(3.987,3.368)}
\gppoint{gp mark 0}{(3.987,3.368)}
\gppoint{gp mark 0}{(3.987,3.418)}
\gppoint{gp mark 0}{(3.987,3.434)}
\gppoint{gp mark 0}{(3.987,3.418)}
\gppoint{gp mark 0}{(3.987,3.385)}
\gppoint{gp mark 0}{(3.987,3.351)}
\gppoint{gp mark 0}{(3.987,3.507)}
\gppoint{gp mark 0}{(3.987,3.351)}
\gppoint{gp mark 0}{(3.987,3.586)}
\gppoint{gp mark 0}{(3.987,3.418)}
\gppoint{gp mark 0}{(3.987,4.002)}
\gppoint{gp mark 0}{(3.987,3.295)}
\gppoint{gp mark 0}{(3.987,4.002)}
\gppoint{gp mark 0}{(3.987,3.689)}
\gppoint{gp mark 0}{(3.987,4.179)}
\gppoint{gp mark 0}{(3.987,3.235)}
\gppoint{gp mark 0}{(3.987,3.434)}
\gppoint{gp mark 0}{(3.987,3.368)}
\gppoint{gp mark 0}{(3.987,3.368)}
\gppoint{gp mark 0}{(3.987,3.351)}
\gppoint{gp mark 0}{(3.987,3.434)}
\gppoint{gp mark 0}{(3.987,3.434)}
\gppoint{gp mark 0}{(3.987,3.805)}
\gppoint{gp mark 0}{(3.987,3.368)}
\gppoint{gp mark 0}{(3.987,3.805)}
\gppoint{gp mark 0}{(3.987,3.168)}
\gppoint{gp mark 0}{(3.987,3.895)}
\gppoint{gp mark 0}{(3.987,3.989)}
\gppoint{gp mark 0}{(3.987,3.368)}
\gppoint{gp mark 0}{(3.987,3.598)}
\gppoint{gp mark 0}{(3.987,3.368)}
\gppoint{gp mark 0}{(3.987,3.168)}
\gppoint{gp mark 0}{(3.987,3.418)}
\gppoint{gp mark 0}{(3.987,3.368)}
\gppoint{gp mark 0}{(3.987,3.368)}
\gppoint{gp mark 0}{(3.987,3.368)}
\gppoint{gp mark 0}{(3.987,3.368)}
\gppoint{gp mark 0}{(3.987,3.418)}
\gppoint{gp mark 0}{(3.987,4.148)}
\gppoint{gp mark 0}{(3.987,3.418)}
\gppoint{gp mark 0}{(3.987,3.989)}
\gppoint{gp mark 0}{(3.987,3.368)}
\gppoint{gp mark 0}{(3.987,3.678)}
\gppoint{gp mark 0}{(3.987,3.235)}
\gppoint{gp mark 0}{(3.987,3.720)}
\gppoint{gp mark 0}{(3.987,3.368)}
\gppoint{gp mark 0}{(3.987,3.368)}
\gppoint{gp mark 0}{(3.987,3.548)}
\gppoint{gp mark 0}{(3.987,3.402)}
\gppoint{gp mark 0}{(3.987,3.918)}
\gppoint{gp mark 0}{(3.987,3.749)}
\gppoint{gp mark 0}{(3.987,3.434)}
\gppoint{gp mark 0}{(3.987,3.368)}
\gppoint{gp mark 0}{(3.987,3.493)}
\gppoint{gp mark 0}{(3.987,3.351)}
\gppoint{gp mark 0}{(3.987,3.168)}
\gppoint{gp mark 0}{(3.987,3.720)}
\gppoint{gp mark 0}{(3.987,3.235)}
\gppoint{gp mark 0}{(3.987,3.235)}
\gppoint{gp mark 0}{(3.987,4.121)}
\gppoint{gp mark 0}{(3.987,3.351)}
\gppoint{gp mark 0}{(3.987,3.667)}
\gppoint{gp mark 0}{(3.987,3.548)}
\gppoint{gp mark 0}{(3.987,3.276)}
\gppoint{gp mark 0}{(3.987,2.914)}
\gppoint{gp mark 0}{(3.987,3.586)}
\gppoint{gp mark 0}{(3.987,4.289)}
\gppoint{gp mark 0}{(3.987,3.548)}
\gppoint{gp mark 0}{(3.987,3.535)}
\gppoint{gp mark 0}{(4.041,3.598)}
\gppoint{gp mark 0}{(4.041,3.493)}
\gppoint{gp mark 0}{(4.041,3.961)}
\gppoint{gp mark 0}{(4.041,3.720)}
\gppoint{gp mark 0}{(4.041,3.598)}
\gppoint{gp mark 0}{(4.041,3.295)}
\gppoint{gp mark 0}{(4.041,3.645)}
\gppoint{gp mark 0}{(4.041,3.402)}
\gppoint{gp mark 0}{(4.041,3.402)}
\gppoint{gp mark 0}{(4.041,3.351)}
\gppoint{gp mark 0}{(4.041,4.034)}
\gppoint{gp mark 0}{(4.041,3.418)}
\gppoint{gp mark 0}{(4.041,3.493)}
\gppoint{gp mark 0}{(4.041,3.720)}
\gppoint{gp mark 0}{(4.041,3.368)}
\gppoint{gp mark 0}{(4.041,3.235)}
\gppoint{gp mark 0}{(4.041,3.276)}
\gppoint{gp mark 0}{(4.041,3.493)}
\gppoint{gp mark 0}{(4.041,3.872)}
\gppoint{gp mark 0}{(4.041,3.493)}
\gppoint{gp mark 0}{(4.041,3.418)}
\gppoint{gp mark 0}{(4.041,3.678)}
\gppoint{gp mark 0}{(4.041,3.213)}
\gppoint{gp mark 0}{(4.041,3.678)}
\gppoint{gp mark 0}{(4.041,3.699)}
\gppoint{gp mark 0}{(4.041,3.656)}
\gppoint{gp mark 0}{(4.041,3.521)}
\gppoint{gp mark 0}{(4.041,3.778)}
\gppoint{gp mark 0}{(4.041,3.351)}
\gppoint{gp mark 0}{(4.041,3.521)}
\gppoint{gp mark 0}{(4.041,3.449)}
\gppoint{gp mark 0}{(4.041,3.710)}
\gppoint{gp mark 0}{(4.041,3.479)}
\gppoint{gp mark 0}{(4.041,3.740)}
\gppoint{gp mark 0}{(4.041,3.778)}
\gppoint{gp mark 0}{(4.041,3.831)}
\gppoint{gp mark 0}{(4.041,3.464)}
\gppoint{gp mark 0}{(4.041,3.645)}
\gppoint{gp mark 0}{(4.041,3.586)}
\gppoint{gp mark 0}{(4.041,3.479)}
\gppoint{gp mark 0}{(4.041,3.235)}
\gppoint{gp mark 0}{(4.041,3.434)}
\gppoint{gp mark 0}{(4.041,3.255)}
\gppoint{gp mark 0}{(4.041,3.610)}
\gppoint{gp mark 0}{(4.041,3.678)}
\gppoint{gp mark 0}{(4.041,3.434)}
\gppoint{gp mark 0}{(4.041,3.385)}
\gppoint{gp mark 0}{(4.041,4.088)}
\gppoint{gp mark 0}{(4.041,3.678)}
\gppoint{gp mark 0}{(4.041,3.235)}
\gppoint{gp mark 0}{(4.041,3.094)}
\gppoint{gp mark 0}{(4.041,4.158)}
\gppoint{gp mark 0}{(4.041,3.479)}
\gppoint{gp mark 0}{(4.041,3.749)}
\gppoint{gp mark 0}{(4.041,3.168)}
\gppoint{gp mark 0}{(4.041,3.573)}
\gppoint{gp mark 0}{(4.041,3.645)}
\gppoint{gp mark 0}{(4.041,3.573)}
\gppoint{gp mark 0}{(4.041,3.295)}
\gppoint{gp mark 0}{(4.041,4.254)}
\gppoint{gp mark 0}{(4.041,3.667)}
\gppoint{gp mark 0}{(4.041,3.418)}
\gppoint{gp mark 0}{(4.041,3.418)}
\gppoint{gp mark 0}{(4.041,3.895)}
\gppoint{gp mark 0}{(4.041,3.449)}
\gppoint{gp mark 0}{(4.041,3.368)}
\gppoint{gp mark 0}{(4.041,3.295)}
\gppoint{gp mark 0}{(4.041,3.573)}
\gppoint{gp mark 0}{(4.041,3.333)}
\gppoint{gp mark 0}{(4.041,3.561)}
\gppoint{gp mark 0}{(4.041,3.548)}
\gppoint{gp mark 0}{(4.041,3.759)}
\gppoint{gp mark 0}{(4.041,3.678)}
\gppoint{gp mark 0}{(4.041,3.402)}
\gppoint{gp mark 0}{(4.041,3.598)}
\gppoint{gp mark 0}{(4.041,3.586)}
\gppoint{gp mark 0}{(4.041,3.351)}
\gppoint{gp mark 0}{(4.041,3.402)}
\gppoint{gp mark 0}{(4.041,3.872)}
\gppoint{gp mark 0}{(4.041,4.148)}
\gppoint{gp mark 0}{(4.041,4.259)}
\gppoint{gp mark 0}{(4.041,3.168)}
\gppoint{gp mark 0}{(4.041,3.235)}
\gppoint{gp mark 0}{(4.041,3.402)}
\gppoint{gp mark 0}{(4.041,3.168)}
\gppoint{gp mark 0}{(4.041,3.094)}
\gppoint{gp mark 0}{(4.041,3.449)}
\gppoint{gp mark 0}{(4.041,3.667)}
\gppoint{gp mark 0}{(4.041,3.144)}
\gppoint{gp mark 0}{(4.041,3.720)}
\gppoint{gp mark 0}{(4.041,3.402)}
\gppoint{gp mark 0}{(4.041,3.759)}
\gppoint{gp mark 0}{(4.041,3.333)}
\gppoint{gp mark 0}{(4.041,3.235)}
\gppoint{gp mark 0}{(4.041,3.645)}
\gppoint{gp mark 0}{(4.041,3.351)}
\gppoint{gp mark 0}{(4.041,3.351)}
\gppoint{gp mark 0}{(4.041,3.880)}
\gppoint{gp mark 0}{(4.041,3.295)}
\gppoint{gp mark 0}{(4.041,3.351)}
\gppoint{gp mark 0}{(4.041,3.720)}
\gppoint{gp mark 0}{(4.041,3.598)}
\gppoint{gp mark 0}{(4.041,3.720)}
\gppoint{gp mark 0}{(4.041,4.028)}
\gppoint{gp mark 0}{(4.041,4.276)}
\gppoint{gp mark 0}{(4.041,3.295)}
\gppoint{gp mark 0}{(4.041,3.333)}
\gppoint{gp mark 0}{(4.041,3.656)}
\gppoint{gp mark 0}{(4.041,3.787)}
\gppoint{gp mark 0}{(4.041,3.740)}
\gppoint{gp mark 0}{(4.041,3.434)}
\gppoint{gp mark 0}{(4.041,3.295)}
\gppoint{gp mark 0}{(4.041,3.434)}
\gppoint{gp mark 0}{(4.041,3.385)}
\gppoint{gp mark 0}{(4.041,3.449)}
\gppoint{gp mark 0}{(4.041,3.434)}
\gppoint{gp mark 0}{(4.041,3.667)}
\gppoint{gp mark 0}{(4.041,3.667)}
\gppoint{gp mark 0}{(4.041,3.235)}
\gppoint{gp mark 0}{(4.041,3.667)}
\gppoint{gp mark 0}{(4.041,3.168)}
\gppoint{gp mark 0}{(4.041,3.634)}
\gppoint{gp mark 0}{(4.041,3.144)}
\gppoint{gp mark 0}{(4.041,3.813)}
\gppoint{gp mark 0}{(4.041,3.295)}
\gppoint{gp mark 0}{(4.041,3.548)}
\gppoint{gp mark 0}{(4.041,3.813)}
\gppoint{gp mark 0}{(4.041,3.947)}
\gppoint{gp mark 0}{(4.041,3.351)}
\gppoint{gp mark 0}{(4.041,4.058)}
\gppoint{gp mark 0}{(4.041,3.740)}
\gppoint{gp mark 0}{(4.041,3.778)}
\gppoint{gp mark 0}{(4.041,3.351)}
\gppoint{gp mark 0}{(4.041,3.954)}
\gppoint{gp mark 0}{(4.041,3.295)}
\gppoint{gp mark 0}{(4.041,3.678)}
\gppoint{gp mark 0}{(4.041,3.295)}
\gppoint{gp mark 0}{(4.041,3.464)}
\gppoint{gp mark 0}{(4.041,3.778)}
\gppoint{gp mark 0}{(4.041,3.168)}
\gppoint{gp mark 0}{(4.041,3.295)}
\gppoint{gp mark 0}{(4.041,3.235)}
\gppoint{gp mark 0}{(4.041,3.610)}
\gppoint{gp mark 0}{(4.041,3.385)}
\gppoint{gp mark 0}{(4.041,3.699)}
\gppoint{gp mark 0}{(4.041,3.295)}
\gppoint{gp mark 0}{(4.041,3.213)}
\gppoint{gp mark 0}{(4.041,3.926)}
\gppoint{gp mark 0}{(4.041,3.385)}
\gppoint{gp mark 0}{(4.041,3.573)}
\gppoint{gp mark 0}{(4.041,3.449)}
\gppoint{gp mark 0}{(4.041,3.667)}
\gppoint{gp mark 0}{(4.041,3.493)}
\gppoint{gp mark 0}{(4.041,3.464)}
\gppoint{gp mark 0}{(4.041,3.235)}
\gppoint{gp mark 0}{(4.041,3.464)}
\gppoint{gp mark 0}{(4.041,3.699)}
\gppoint{gp mark 0}{(4.041,3.385)}
\gppoint{gp mark 0}{(4.041,3.385)}
\gppoint{gp mark 0}{(4.041,3.699)}
\gppoint{gp mark 0}{(4.041,3.351)}
\gppoint{gp mark 0}{(4.041,3.911)}
\gppoint{gp mark 0}{(4.041,3.667)}
\gppoint{gp mark 0}{(4.041,3.813)}
\gppoint{gp mark 0}{(4.041,3.667)}
\gppoint{gp mark 0}{(4.041,3.561)}
\gppoint{gp mark 0}{(4.041,3.645)}
\gppoint{gp mark 0}{(4.041,3.535)}
\gppoint{gp mark 0}{(4.041,3.295)}
\gppoint{gp mark 0}{(4.041,3.464)}
\gppoint{gp mark 0}{(4.041,3.678)}
\gppoint{gp mark 0}{(4.041,4.093)}
\gppoint{gp mark 0}{(4.041,3.418)}
\gppoint{gp mark 0}{(4.041,3.535)}
\gppoint{gp mark 0}{(4.041,4.189)}
\gppoint{gp mark 0}{(4.041,3.678)}
\gppoint{gp mark 0}{(4.041,3.561)}
\gppoint{gp mark 0}{(4.041,3.333)}
\gppoint{gp mark 0}{(4.041,3.333)}
\gppoint{gp mark 0}{(4.041,3.333)}
\gppoint{gp mark 0}{(4.041,3.333)}
\gppoint{gp mark 0}{(4.041,4.040)}
\gppoint{gp mark 0}{(4.041,3.333)}
\gppoint{gp mark 0}{(4.041,4.203)}
\gppoint{gp mark 0}{(4.041,3.464)}
\gppoint{gp mark 0}{(4.041,3.464)}
\gppoint{gp mark 0}{(4.041,3.464)}
\gppoint{gp mark 0}{(4.041,3.464)}
\gppoint{gp mark 0}{(4.041,3.464)}
\gppoint{gp mark 0}{(4.041,3.464)}
\gppoint{gp mark 0}{(4.041,3.464)}
\gppoint{gp mark 0}{(4.041,3.039)}
\gppoint{gp mark 0}{(4.041,3.418)}
\gppoint{gp mark 0}{(4.041,3.561)}
\gppoint{gp mark 0}{(4.041,3.720)}
\gppoint{gp mark 0}{(4.041,3.678)}
\gppoint{gp mark 0}{(4.041,3.710)}
\gppoint{gp mark 0}{(4.041,3.720)}
\gppoint{gp mark 0}{(4.041,3.645)}
\gppoint{gp mark 0}{(4.041,3.586)}
\gppoint{gp mark 0}{(4.041,3.561)}
\gppoint{gp mark 0}{(4.041,3.975)}
\gppoint{gp mark 0}{(4.041,3.689)}
\gppoint{gp mark 0}{(4.041,3.385)}
\gppoint{gp mark 0}{(4.041,3.235)}
\gppoint{gp mark 0}{(4.041,3.645)}
\gppoint{gp mark 0}{(4.041,3.796)}
\gppoint{gp mark 0}{(4.041,3.418)}
\gppoint{gp mark 0}{(4.041,3.822)}
\gppoint{gp mark 0}{(4.041,3.464)}
\gppoint{gp mark 0}{(4.041,3.561)}
\gppoint{gp mark 0}{(4.041,4.082)}
\gppoint{gp mark 0}{(4.041,3.493)}
\gppoint{gp mark 0}{(4.041,3.333)}
\gppoint{gp mark 0}{(4.041,3.598)}
\gppoint{gp mark 0}{(4.041,3.689)}
\gppoint{gp mark 0}{(4.041,3.699)}
\gppoint{gp mark 0}{(4.041,3.351)}
\gppoint{gp mark 0}{(4.041,3.561)}
\gppoint{gp mark 0}{(4.041,3.699)}
\gppoint{gp mark 0}{(4.041,3.418)}
\gppoint{gp mark 0}{(4.041,3.418)}
\gppoint{gp mark 0}{(4.041,3.402)}
\gppoint{gp mark 0}{(4.041,3.699)}
\gppoint{gp mark 0}{(4.041,4.076)}
\gppoint{gp mark 0}{(4.041,3.778)}
\gppoint{gp mark 0}{(4.041,3.730)}
\gppoint{gp mark 0}{(4.041,3.213)}
\gppoint{gp mark 0}{(4.041,3.402)}
\gppoint{gp mark 0}{(4.041,3.586)}
\gppoint{gp mark 0}{(4.041,3.521)}
\gppoint{gp mark 0}{(4.041,3.730)}
\gppoint{gp mark 0}{(4.041,3.535)}
\gppoint{gp mark 0}{(4.041,3.610)}
\gppoint{gp mark 0}{(4.041,3.961)}
\gppoint{gp mark 0}{(4.041,4.093)}
\gppoint{gp mark 0}{(4.041,3.521)}
\gppoint{gp mark 0}{(4.041,3.276)}
\gppoint{gp mark 0}{(4.041,3.947)}
\gppoint{gp mark 0}{(4.041,3.295)}
\gppoint{gp mark 0}{(4.041,3.656)}
\gppoint{gp mark 0}{(4.041,3.295)}
\gppoint{gp mark 0}{(4.041,3.720)}
\gppoint{gp mark 0}{(4.041,3.464)}
\gppoint{gp mark 0}{(4.041,3.493)}
\gppoint{gp mark 0}{(4.041,3.434)}
\gppoint{gp mark 0}{(4.041,3.295)}
\gppoint{gp mark 0}{(4.041,3.968)}
\gppoint{gp mark 0}{(4.041,3.449)}
\gppoint{gp mark 0}{(4.041,4.110)}
\gppoint{gp mark 0}{(4.041,3.276)}
\gppoint{gp mark 0}{(4.041,3.968)}
\gppoint{gp mark 0}{(4.041,3.667)}
\gppoint{gp mark 0}{(4.041,3.768)}
\gppoint{gp mark 0}{(4.041,3.768)}
\gppoint{gp mark 0}{(4.041,4.028)}
\gppoint{gp mark 0}{(4.041,4.028)}
\gppoint{gp mark 0}{(4.041,4.263)}
\gppoint{gp mark 0}{(4.041,3.610)}
\gppoint{gp mark 0}{(4.041,3.521)}
\gppoint{gp mark 0}{(4.041,3.778)}
\gppoint{gp mark 0}{(4.041,3.656)}
\gppoint{gp mark 0}{(4.041,3.634)}
\gppoint{gp mark 0}{(4.041,3.255)}
\gppoint{gp mark 0}{(4.041,4.121)}
\gppoint{gp mark 0}{(4.041,3.954)}
\gppoint{gp mark 0}{(4.041,3.368)}
\gppoint{gp mark 0}{(4.041,3.464)}
\gppoint{gp mark 0}{(4.041,3.385)}
\gppoint{gp mark 0}{(4.041,3.699)}
\gppoint{gp mark 0}{(4.041,3.449)}
\gppoint{gp mark 0}{(4.041,3.634)}
\gppoint{gp mark 0}{(4.041,3.805)}
\gppoint{gp mark 0}{(4.041,4.254)}
\gppoint{gp mark 0}{(4.041,3.610)}
\gppoint{gp mark 0}{(4.041,3.645)}
\gppoint{gp mark 0}{(4.041,3.610)}
\gppoint{gp mark 0}{(4.041,3.464)}
\gppoint{gp mark 0}{(4.041,3.872)}
\gppoint{gp mark 0}{(4.041,3.872)}
\gppoint{gp mark 0}{(4.041,3.720)}
\gppoint{gp mark 0}{(4.041,3.975)}
\gppoint{gp mark 0}{(4.041,3.295)}
\gppoint{gp mark 0}{(4.041,3.864)}
\gppoint{gp mark 0}{(4.041,3.548)}
\gppoint{gp mark 0}{(4.041,3.689)}
\gppoint{gp mark 0}{(4.041,3.507)}
\gppoint{gp mark 0}{(4.041,3.521)}
\gppoint{gp mark 0}{(4.041,3.235)}
\gppoint{gp mark 0}{(4.041,3.521)}
\gppoint{gp mark 0}{(4.041,3.295)}
\gppoint{gp mark 0}{(4.041,3.918)}
\gppoint{gp mark 0}{(4.041,3.822)}
\gppoint{gp mark 0}{(4.041,3.385)}
\gppoint{gp mark 0}{(4.041,3.168)}
\gppoint{gp mark 0}{(4.041,3.586)}
\gppoint{gp mark 0}{(4.041,3.787)}
\gppoint{gp mark 0}{(4.041,3.645)}
\gppoint{gp mark 0}{(4.041,3.995)}
\gppoint{gp mark 0}{(4.041,3.872)}
\gppoint{gp mark 0}{(4.041,3.699)}
\gppoint{gp mark 0}{(4.041,3.656)}
\gppoint{gp mark 0}{(4.041,3.295)}
\gppoint{gp mark 0}{(4.041,3.872)}
\gppoint{gp mark 0}{(4.041,3.622)}
\gppoint{gp mark 0}{(4.041,3.787)}
\gppoint{gp mark 0}{(4.041,3.235)}
\gppoint{gp mark 0}{(4.041,3.295)}
\gppoint{gp mark 0}{(4.041,4.137)}
\gppoint{gp mark 0}{(4.041,3.730)}
\gppoint{gp mark 0}{(4.041,3.926)}
\gppoint{gp mark 0}{(4.041,3.645)}
\gppoint{gp mark 0}{(4.041,3.295)}
\gppoint{gp mark 0}{(4.041,3.493)}
\gppoint{gp mark 0}{(4.041,3.699)}
\gppoint{gp mark 0}{(4.041,3.402)}
\gppoint{gp mark 0}{(4.041,3.295)}
\gppoint{gp mark 0}{(4.041,3.333)}
\gppoint{gp mark 0}{(4.041,3.689)}
\gppoint{gp mark 0}{(4.041,3.493)}
\gppoint{gp mark 0}{(4.041,3.295)}
\gppoint{gp mark 0}{(4.041,3.622)}
\gppoint{gp mark 0}{(4.041,3.768)}
\gppoint{gp mark 0}{(4.041,3.464)}
\gppoint{gp mark 0}{(4.041,3.689)}
\gppoint{gp mark 0}{(4.041,4.002)}
\gppoint{gp mark 0}{(4.041,3.351)}
\gppoint{gp mark 0}{(4.041,3.968)}
\gppoint{gp mark 0}{(4.041,3.548)}
\gppoint{gp mark 0}{(4.041,3.479)}
\gppoint{gp mark 0}{(4.041,3.768)}
\gppoint{gp mark 0}{(4.041,4.263)}
\gppoint{gp mark 0}{(4.041,3.645)}
\gppoint{gp mark 0}{(4.041,3.295)}
\gppoint{gp mark 0}{(4.041,4.070)}
\gppoint{gp mark 0}{(4.041,3.295)}
\gppoint{gp mark 0}{(4.041,3.720)}
\gppoint{gp mark 0}{(4.041,3.314)}
\gppoint{gp mark 0}{(4.041,3.720)}
\gppoint{gp mark 0}{(4.041,3.839)}
\gppoint{gp mark 0}{(4.041,3.535)}
\gppoint{gp mark 0}{(4.041,3.535)}
\gppoint{gp mark 0}{(4.041,4.088)}
\gppoint{gp mark 0}{(4.041,3.434)}
\gppoint{gp mark 0}{(4.041,3.535)}
\gppoint{gp mark 0}{(4.041,3.493)}
\gppoint{gp mark 0}{(4.041,4.093)}
\gppoint{gp mark 0}{(4.041,3.667)}
\gppoint{gp mark 0}{(4.041,4.533)}
\gppoint{gp mark 0}{(4.041,3.351)}
\gppoint{gp mark 0}{(4.041,3.968)}
\gppoint{gp mark 0}{(4.041,3.351)}
\gppoint{gp mark 0}{(4.041,3.645)}
\gppoint{gp mark 0}{(4.041,3.418)}
\gppoint{gp mark 0}{(4.041,3.656)}
\gppoint{gp mark 0}{(4.041,4.263)}
\gppoint{gp mark 0}{(4.041,3.295)}
\gppoint{gp mark 0}{(4.041,3.634)}
\gppoint{gp mark 0}{(4.041,4.297)}
\gppoint{gp mark 0}{(4.041,4.076)}
\gppoint{gp mark 0}{(4.041,3.094)}
\gppoint{gp mark 0}{(4.041,3.449)}
\gppoint{gp mark 0}{(4.041,3.864)}
\gppoint{gp mark 0}{(4.041,4.052)}
\gppoint{gp mark 0}{(4.041,3.479)}
\gppoint{gp mark 0}{(4.041,3.434)}
\gppoint{gp mark 0}{(4.041,3.710)}
\gppoint{gp mark 0}{(4.041,3.598)}
\gppoint{gp mark 0}{(4.041,3.872)}
\gppoint{gp mark 0}{(4.041,4.105)}
\gppoint{gp mark 0}{(4.041,3.418)}
\gppoint{gp mark 0}{(4.041,3.645)}
\gppoint{gp mark 0}{(4.041,3.464)}
\gppoint{gp mark 0}{(4.041,3.493)}
\gppoint{gp mark 0}{(4.041,3.926)}
\gppoint{gp mark 0}{(4.041,3.940)}
\gppoint{gp mark 0}{(4.041,3.548)}
\gppoint{gp mark 0}{(4.041,3.768)}
\gppoint{gp mark 0}{(4.041,3.168)}
\gppoint{gp mark 0}{(4.041,3.561)}
\gppoint{gp mark 0}{(4.041,3.822)}
\gppoint{gp mark 0}{(4.041,3.402)}
\gppoint{gp mark 0}{(4.092,4.512)}
\gppoint{gp mark 0}{(4.092,3.561)}
\gppoint{gp mark 0}{(4.092,4.466)}
\gppoint{gp mark 0}{(4.092,4.241)}
\gppoint{gp mark 0}{(4.092,3.720)}
\gppoint{gp mark 0}{(4.092,3.479)}
\gppoint{gp mark 0}{(4.092,3.434)}
\gppoint{gp mark 0}{(4.092,3.333)}
\gppoint{gp mark 0}{(4.092,3.759)}
\gppoint{gp mark 0}{(4.092,3.434)}
\gppoint{gp mark 0}{(4.092,3.385)}
\gppoint{gp mark 0}{(4.092,3.678)}
\gppoint{gp mark 0}{(4.092,3.561)}
\gppoint{gp mark 0}{(4.092,3.493)}
\gppoint{gp mark 0}{(4.092,4.110)}
\gppoint{gp mark 0}{(4.092,3.561)}
\gppoint{gp mark 0}{(4.092,3.656)}
\gppoint{gp mark 0}{(4.092,4.469)}
\gppoint{gp mark 0}{(4.092,3.740)}
\gppoint{gp mark 0}{(4.092,4.110)}
\gppoint{gp mark 0}{(4.092,4.184)}
\gppoint{gp mark 0}{(4.092,3.730)}
\gppoint{gp mark 0}{(4.092,3.880)}
\gppoint{gp mark 0}{(4.092,3.656)}
\gppoint{gp mark 0}{(4.092,3.787)}
\gppoint{gp mark 0}{(4.092,3.521)}
\gppoint{gp mark 0}{(4.092,3.385)}
\gppoint{gp mark 0}{(4.092,3.656)}
\gppoint{gp mark 0}{(4.092,4.368)}
\gppoint{gp mark 0}{(4.092,3.561)}
\gppoint{gp mark 0}{(4.092,3.667)}
\gppoint{gp mark 0}{(4.092,3.759)}
\gppoint{gp mark 0}{(4.092,3.805)}
\gppoint{gp mark 0}{(4.092,3.847)}
\gppoint{gp mark 0}{(4.092,3.573)}
\gppoint{gp mark 0}{(4.092,4.556)}
\gppoint{gp mark 0}{(4.092,3.235)}
\gppoint{gp mark 0}{(4.092,4.008)}
\gppoint{gp mark 0}{(4.092,3.418)}
\gppoint{gp mark 0}{(4.092,3.634)}
\gppoint{gp mark 0}{(4.092,3.667)}
\gppoint{gp mark 0}{(4.092,3.667)}
\gppoint{gp mark 0}{(4.092,3.598)}
\gppoint{gp mark 0}{(4.092,3.622)}
\gppoint{gp mark 0}{(4.092,3.634)}
\gppoint{gp mark 0}{(4.092,3.333)}
\gppoint{gp mark 0}{(4.092,3.622)}
\gppoint{gp mark 0}{(4.092,3.667)}
\gppoint{gp mark 0}{(4.092,3.449)}
\gppoint{gp mark 0}{(4.092,4.093)}
\gppoint{gp mark 0}{(4.092,3.689)}
\gppoint{gp mark 0}{(4.092,3.561)}
\gppoint{gp mark 0}{(4.092,3.548)}
\gppoint{gp mark 0}{(4.092,3.720)}
\gppoint{gp mark 0}{(4.092,3.813)}
\gppoint{gp mark 0}{(4.092,3.479)}
\gppoint{gp mark 0}{(4.092,3.645)}
\gppoint{gp mark 0}{(4.092,3.521)}
\gppoint{gp mark 0}{(4.092,3.689)}
\gppoint{gp mark 0}{(4.092,3.449)}
\gppoint{gp mark 0}{(4.092,3.656)}
\gppoint{gp mark 0}{(4.092,3.385)}
\gppoint{gp mark 0}{(4.092,3.385)}
\gppoint{gp mark 0}{(4.092,3.634)}
\gppoint{gp mark 0}{(4.092,3.710)}
\gppoint{gp mark 0}{(4.092,3.235)}
\gppoint{gp mark 0}{(4.092,3.610)}
\gppoint{gp mark 0}{(4.092,4.132)}
\gppoint{gp mark 0}{(4.092,3.351)}
\gppoint{gp mark 0}{(4.092,3.740)}
\gppoint{gp mark 0}{(4.092,3.235)}
\gppoint{gp mark 0}{(4.092,3.678)}
\gppoint{gp mark 0}{(4.092,3.464)}
\gppoint{gp mark 0}{(4.092,4.021)}
\gppoint{gp mark 0}{(4.092,3.610)}
\gppoint{gp mark 0}{(4.092,3.351)}
\gppoint{gp mark 0}{(4.092,3.276)}
\gppoint{gp mark 0}{(4.092,3.787)}
\gppoint{gp mark 0}{(4.092,3.598)}
\gppoint{gp mark 0}{(4.092,3.464)}
\gppoint{gp mark 0}{(4.092,3.548)}
\gppoint{gp mark 0}{(4.092,3.933)}
\gppoint{gp mark 0}{(4.092,3.656)}
\gppoint{gp mark 0}{(4.092,3.295)}
\gppoint{gp mark 0}{(4.092,3.598)}
\gppoint{gp mark 0}{(4.092,3.535)}
\gppoint{gp mark 0}{(4.092,3.656)}
\gppoint{gp mark 0}{(4.092,3.895)}
\gppoint{gp mark 0}{(4.092,3.351)}
\gppoint{gp mark 0}{(4.092,3.464)}
\gppoint{gp mark 0}{(4.092,3.351)}
\gppoint{gp mark 0}{(4.092,4.099)}
\gppoint{gp mark 0}{(4.092,3.864)}
\gppoint{gp mark 0}{(4.092,4.058)}
\gppoint{gp mark 0}{(4.092,4.058)}
\gppoint{gp mark 0}{(4.092,3.839)}
\gppoint{gp mark 0}{(4.092,3.507)}
\gppoint{gp mark 0}{(4.092,3.598)}
\gppoint{gp mark 0}{(4.092,3.740)}
\gppoint{gp mark 0}{(4.092,3.839)}
\gppoint{gp mark 0}{(4.092,3.678)}
\gppoint{gp mark 0}{(4.092,3.839)}
\gppoint{gp mark 0}{(4.092,3.598)}
\gppoint{gp mark 0}{(4.092,3.656)}
\gppoint{gp mark 0}{(4.092,3.822)}
\gppoint{gp mark 0}{(4.092,3.768)}
\gppoint{gp mark 0}{(4.092,3.656)}
\gppoint{gp mark 0}{(4.092,3.888)}
\gppoint{gp mark 0}{(4.092,3.730)}
\gppoint{gp mark 0}{(4.092,3.888)}
\gppoint{gp mark 0}{(4.092,3.678)}
\gppoint{gp mark 0}{(4.092,3.351)}
\gppoint{gp mark 0}{(4.092,3.479)}
\gppoint{gp mark 0}{(4.092,3.548)}
\gppoint{gp mark 0}{(4.092,3.402)}
\gppoint{gp mark 0}{(4.092,3.839)}
\gppoint{gp mark 0}{(4.092,3.333)}
\gppoint{gp mark 0}{(4.092,3.402)}
\gppoint{gp mark 0}{(4.092,3.720)}
\gppoint{gp mark 0}{(4.092,3.586)}
\gppoint{gp mark 0}{(4.092,4.082)}
\gppoint{gp mark 0}{(4.092,3.961)}
\gppoint{gp mark 0}{(4.092,3.418)}
\gppoint{gp mark 0}{(4.092,3.710)}
\gppoint{gp mark 0}{(4.092,3.678)}
\gppoint{gp mark 0}{(4.092,3.535)}
\gppoint{gp mark 0}{(4.092,3.880)}
\gppoint{gp mark 0}{(4.092,3.548)}
\gppoint{gp mark 0}{(4.092,3.839)}
\gppoint{gp mark 0}{(4.092,3.888)}
\gppoint{gp mark 0}{(4.092,3.880)}
\gppoint{gp mark 0}{(4.092,3.351)}
\gppoint{gp mark 0}{(4.092,3.634)}
\gppoint{gp mark 0}{(4.092,3.656)}
\gppoint{gp mark 0}{(4.092,3.598)}
\gppoint{gp mark 0}{(4.092,3.778)}
\gppoint{gp mark 0}{(4.092,3.586)}
\gppoint{gp mark 0}{(4.092,3.434)}
\gppoint{gp mark 0}{(4.092,3.839)}
\gppoint{gp mark 0}{(4.092,3.667)}
\gppoint{gp mark 0}{(4.092,3.839)}
\gppoint{gp mark 0}{(4.092,3.295)}
\gppoint{gp mark 0}{(4.092,3.667)}
\gppoint{gp mark 0}{(4.092,3.622)}
\gppoint{gp mark 0}{(4.092,3.561)}
\gppoint{gp mark 0}{(4.092,3.831)}
\gppoint{gp mark 0}{(4.092,3.235)}
\gppoint{gp mark 0}{(4.092,3.010)}
\gppoint{gp mark 0}{(4.092,3.995)}
\gppoint{gp mark 0}{(4.092,3.839)}
\gppoint{gp mark 0}{(4.092,3.839)}
\gppoint{gp mark 0}{(4.092,3.839)}
\gppoint{gp mark 0}{(4.092,3.839)}
\gppoint{gp mark 0}{(4.092,3.888)}
\gppoint{gp mark 0}{(4.092,3.598)}
\gppoint{gp mark 0}{(4.092,3.598)}
\gppoint{gp mark 0}{(4.092,3.276)}
\gppoint{gp mark 0}{(4.092,3.598)}
\gppoint{gp mark 0}{(4.092,3.351)}
\gppoint{gp mark 0}{(4.092,3.667)}
\gppoint{gp mark 0}{(4.092,3.586)}
\gppoint{gp mark 0}{(4.092,3.573)}
\gppoint{gp mark 0}{(4.092,3.634)}
\gppoint{gp mark 0}{(4.092,3.598)}
\gppoint{gp mark 0}{(4.092,3.573)}
\gppoint{gp mark 0}{(4.092,3.351)}
\gppoint{gp mark 0}{(4.092,3.933)}
\gppoint{gp mark 0}{(4.092,3.402)}
\gppoint{gp mark 0}{(4.092,3.351)}
\gppoint{gp mark 0}{(4.092,3.351)}
\gppoint{gp mark 0}{(4.092,4.314)}
\gppoint{gp mark 0}{(4.092,4.174)}
\gppoint{gp mark 0}{(4.092,3.598)}
\gppoint{gp mark 0}{(4.092,3.295)}
\gppoint{gp mark 0}{(4.092,4.302)}
\gppoint{gp mark 0}{(4.092,3.548)}
\gppoint{gp mark 0}{(4.092,3.710)}
\gppoint{gp mark 0}{(4.092,3.880)}
\gppoint{gp mark 0}{(4.092,3.573)}
\gppoint{gp mark 0}{(4.092,3.667)}
\gppoint{gp mark 0}{(4.092,3.634)}
\gppoint{gp mark 0}{(4.092,3.535)}
\gppoint{gp mark 0}{(4.092,3.385)}
\gppoint{gp mark 0}{(4.092,3.276)}
\gppoint{gp mark 0}{(4.092,3.678)}
\gppoint{gp mark 0}{(4.092,3.749)}
\gppoint{gp mark 0}{(4.092,3.759)}
\gppoint{gp mark 0}{(4.092,3.720)}
\gppoint{gp mark 0}{(4.092,3.479)}
\gppoint{gp mark 0}{(4.092,3.888)}
\gppoint{gp mark 0}{(4.092,3.872)}
\gppoint{gp mark 0}{(4.092,3.521)}
\gppoint{gp mark 0}{(4.092,3.351)}
\gppoint{gp mark 0}{(4.092,3.094)}
\gppoint{gp mark 0}{(4.092,3.831)}
\gppoint{gp mark 0}{(4.092,3.507)}
\gppoint{gp mark 0}{(4.092,3.911)}
\gppoint{gp mark 0}{(4.092,3.295)}
\gppoint{gp mark 0}{(4.092,3.479)}
\gppoint{gp mark 0}{(4.092,3.521)}
\gppoint{gp mark 0}{(4.092,3.449)}
\gppoint{gp mark 0}{(4.092,3.479)}
\gppoint{gp mark 0}{(4.092,3.434)}
\gppoint{gp mark 0}{(4.092,3.418)}
\gppoint{gp mark 0}{(4.092,3.521)}
\gppoint{gp mark 0}{(4.092,4.241)}
\gppoint{gp mark 0}{(4.092,3.521)}
\gppoint{gp mark 0}{(4.092,4.008)}
\gppoint{gp mark 0}{(4.092,4.034)}
\gppoint{gp mark 0}{(4.092,3.368)}
\gppoint{gp mark 0}{(4.092,4.314)}
\gppoint{gp mark 0}{(4.092,3.561)}
\gppoint{gp mark 0}{(4.092,3.548)}
\gppoint{gp mark 0}{(4.092,3.168)}
\gppoint{gp mark 0}{(4.092,4.064)}
\gppoint{gp mark 0}{(4.092,4.034)}
\gppoint{gp mark 0}{(4.092,3.521)}
\gppoint{gp mark 0}{(4.092,3.333)}
\gppoint{gp mark 0}{(4.092,3.094)}
\gppoint{gp mark 0}{(4.092,3.314)}
\gppoint{gp mark 0}{(4.092,3.667)}
\gppoint{gp mark 0}{(4.092,3.667)}
\gppoint{gp mark 0}{(4.092,3.933)}
\gppoint{gp mark 0}{(4.092,4.232)}
\gppoint{gp mark 0}{(4.092,3.888)}
\gppoint{gp mark 0}{(4.092,4.267)}
\gppoint{gp mark 0}{(4.092,4.280)}
\gppoint{gp mark 0}{(4.092,3.464)}
\gppoint{gp mark 0}{(4.092,3.778)}
\gppoint{gp mark 0}{(4.092,3.586)}
\gppoint{gp mark 0}{(4.092,3.235)}
\gppoint{gp mark 0}{(4.092,3.634)}
\gppoint{gp mark 0}{(4.092,3.667)}
\gppoint{gp mark 0}{(4.092,3.333)}
\gppoint{gp mark 0}{(4.092,3.975)}
\gppoint{gp mark 0}{(4.092,3.699)}
\gppoint{gp mark 0}{(4.092,4.232)}
\gppoint{gp mark 0}{(4.092,3.667)}
\gppoint{gp mark 0}{(4.092,4.110)}
\gppoint{gp mark 0}{(4.092,3.699)}
\gppoint{gp mark 0}{(4.092,3.351)}
\gppoint{gp mark 0}{(4.092,3.464)}
\gppoint{gp mark 0}{(4.092,3.678)}
\gppoint{gp mark 0}{(4.092,3.276)}
\gppoint{gp mark 0}{(4.092,3.521)}
\gppoint{gp mark 0}{(4.092,3.276)}
\gppoint{gp mark 0}{(4.092,3.678)}
\gppoint{gp mark 0}{(4.092,3.813)}
\gppoint{gp mark 0}{(4.092,4.064)}
\gppoint{gp mark 0}{(4.092,3.351)}
\gppoint{gp mark 0}{(4.092,3.548)}
\gppoint{gp mark 0}{(4.092,3.561)}
\gppoint{gp mark 0}{(4.092,3.464)}
\gppoint{gp mark 0}{(4.092,4.064)}
\gppoint{gp mark 0}{(4.092,3.295)}
\gppoint{gp mark 0}{(4.092,3.982)}
\gppoint{gp mark 0}{(4.092,3.493)}
\gppoint{gp mark 0}{(4.092,4.064)}
\gppoint{gp mark 0}{(4.092,4.405)}
\gppoint{gp mark 0}{(4.092,3.295)}
\gppoint{gp mark 0}{(4.092,3.918)}
\gppoint{gp mark 0}{(4.092,3.622)}
\gppoint{gp mark 0}{(4.092,3.699)}
\gppoint{gp mark 0}{(4.092,3.385)}
\gppoint{gp mark 0}{(4.092,3.678)}
\gppoint{gp mark 0}{(4.092,3.402)}
\gppoint{gp mark 0}{(4.092,4.222)}
\gppoint{gp mark 0}{(4.092,3.634)}
\gppoint{gp mark 0}{(4.092,3.385)}
\gppoint{gp mark 0}{(4.092,3.561)}
\gppoint{gp mark 0}{(4.092,3.813)}
\gppoint{gp mark 0}{(4.092,3.535)}
\gppoint{gp mark 0}{(4.092,3.235)}
\gppoint{gp mark 0}{(4.092,3.493)}
\gppoint{gp mark 0}{(4.092,3.351)}
\gppoint{gp mark 0}{(4.092,4.127)}
\gppoint{gp mark 0}{(4.092,4.203)}
\gppoint{gp mark 0}{(4.092,3.656)}
\gppoint{gp mark 0}{(4.092,3.351)}
\gppoint{gp mark 0}{(4.092,3.598)}
\gppoint{gp mark 0}{(4.092,3.295)}
\gppoint{gp mark 0}{(4.092,3.888)}
\gppoint{gp mark 0}{(4.092,3.479)}
\gppoint{gp mark 0}{(4.092,3.710)}
\gppoint{gp mark 0}{(4.092,4.276)}
\gppoint{gp mark 0}{(4.092,3.521)}
\gppoint{gp mark 0}{(4.092,3.493)}
\gppoint{gp mark 0}{(4.092,3.351)}
\gppoint{gp mark 0}{(4.092,3.888)}
\gppoint{gp mark 0}{(4.092,4.052)}
\gppoint{gp mark 0}{(4.092,3.634)}
\gppoint{gp mark 0}{(4.092,3.255)}
\gppoint{gp mark 0}{(4.092,3.507)}
\gppoint{gp mark 0}{(4.092,3.507)}
\gppoint{gp mark 0}{(4.092,3.333)}
\gppoint{gp mark 0}{(4.092,3.434)}
\gppoint{gp mark 0}{(4.092,3.982)}
\gppoint{gp mark 0}{(4.092,3.402)}
\gppoint{gp mark 0}{(4.092,3.295)}
\gppoint{gp mark 0}{(4.092,4.070)}
\gppoint{gp mark 0}{(4.092,3.434)}
\gppoint{gp mark 0}{(4.092,3.434)}
\gppoint{gp mark 0}{(4.092,3.351)}
\gppoint{gp mark 0}{(4.092,3.434)}
\gppoint{gp mark 0}{(4.092,3.573)}
\gppoint{gp mark 0}{(4.092,3.778)}
\gppoint{gp mark 0}{(4.092,3.168)}
\gppoint{gp mark 0}{(4.092,3.168)}
\gppoint{gp mark 0}{(4.140,3.656)}
\gppoint{gp mark 0}{(4.140,3.351)}
\gppoint{gp mark 0}{(4.140,3.610)}
\gppoint{gp mark 0}{(4.140,3.586)}
\gppoint{gp mark 0}{(4.140,3.689)}
\gppoint{gp mark 0}{(4.140,3.622)}
\gppoint{gp mark 0}{(4.140,4.153)}
\gppoint{gp mark 0}{(4.140,4.121)}
\gppoint{gp mark 0}{(4.140,3.295)}
\gppoint{gp mark 0}{(4.140,3.255)}
\gppoint{gp mark 0}{(4.140,3.295)}
\gppoint{gp mark 0}{(4.140,4.082)}
\gppoint{gp mark 0}{(4.140,3.464)}
\gppoint{gp mark 0}{(4.140,3.720)}
\gppoint{gp mark 0}{(4.140,3.968)}
\gppoint{gp mark 0}{(4.140,3.710)}
\gppoint{gp mark 0}{(4.140,3.449)}
\gppoint{gp mark 0}{(4.140,3.521)}
\gppoint{gp mark 0}{(4.140,3.678)}
\gppoint{gp mark 0}{(4.140,3.995)}
\gppoint{gp mark 0}{(4.140,3.449)}
\gppoint{gp mark 0}{(4.140,3.449)}
\gppoint{gp mark 0}{(4.140,3.740)}
\gppoint{gp mark 0}{(4.140,4.110)}
\gppoint{gp mark 0}{(4.140,3.295)}
\gppoint{gp mark 0}{(4.140,3.449)}
\gppoint{gp mark 0}{(4.140,3.699)}
\gppoint{gp mark 0}{(4.140,3.295)}
\gppoint{gp mark 0}{(4.140,3.699)}
\gppoint{gp mark 0}{(4.140,4.349)}
\gppoint{gp mark 0}{(4.140,3.255)}
\gppoint{gp mark 0}{(4.140,3.333)}
\gppoint{gp mark 0}{(4.140,3.507)}
\gppoint{gp mark 0}{(4.140,3.911)}
\gppoint{gp mark 0}{(4.140,3.573)}
\gppoint{gp mark 0}{(4.140,3.464)}
\gppoint{gp mark 0}{(4.140,3.368)}
\gppoint{gp mark 0}{(4.140,3.235)}
\gppoint{gp mark 0}{(4.140,3.598)}
\gppoint{gp mark 0}{(4.140,4.179)}
\gppoint{gp mark 0}{(4.140,4.015)}
\gppoint{gp mark 0}{(4.140,3.561)}
\gppoint{gp mark 0}{(4.140,3.622)}
\gppoint{gp mark 0}{(4.140,4.002)}
\gppoint{gp mark 0}{(4.140,3.689)}
\gppoint{gp mark 0}{(4.140,3.730)}
\gppoint{gp mark 0}{(4.140,3.402)}
\gppoint{gp mark 0}{(4.140,3.740)}
\gppoint{gp mark 0}{(4.140,3.235)}
\gppoint{gp mark 0}{(4.140,3.402)}
\gppoint{gp mark 0}{(4.140,3.831)}
\gppoint{gp mark 0}{(4.140,4.116)}
\gppoint{gp mark 0}{(4.140,3.730)}
\gppoint{gp mark 0}{(4.140,3.749)}
\gppoint{gp mark 0}{(4.140,3.656)}
\gppoint{gp mark 0}{(4.140,3.656)}
\gppoint{gp mark 0}{(4.140,3.449)}
\gppoint{gp mark 0}{(4.140,4.272)}
\gppoint{gp mark 0}{(4.140,3.586)}
\gppoint{gp mark 0}{(4.140,3.689)}
\gppoint{gp mark 0}{(4.140,3.622)}
\gppoint{gp mark 0}{(4.140,3.689)}
\gppoint{gp mark 0}{(4.140,3.610)}
\gppoint{gp mark 0}{(4.140,3.402)}
\gppoint{gp mark 0}{(4.140,3.351)}
\gppoint{gp mark 0}{(4.140,3.479)}
\gppoint{gp mark 0}{(4.140,3.759)}
\gppoint{gp mark 0}{(4.140,3.831)}
\gppoint{gp mark 0}{(4.140,3.449)}
\gppoint{gp mark 0}{(4.140,3.573)}
\gppoint{gp mark 0}{(4.140,3.561)}
\gppoint{gp mark 0}{(4.140,4.376)}
\gppoint{gp mark 0}{(4.140,3.535)}
\gppoint{gp mark 0}{(4.140,3.872)}
\gppoint{gp mark 0}{(4.140,4.412)}
\gppoint{gp mark 0}{(4.140,3.778)}
\gppoint{gp mark 0}{(4.140,4.412)}
\gppoint{gp mark 0}{(4.140,3.418)}
\gppoint{gp mark 0}{(4.140,4.121)}
\gppoint{gp mark 0}{(4.140,3.730)}
\gppoint{gp mark 0}{(4.140,3.493)}
\gppoint{gp mark 0}{(4.140,3.678)}
\gppoint{gp mark 0}{(4.140,3.872)}
\gppoint{gp mark 0}{(4.140,3.507)}
\gppoint{gp mark 0}{(4.140,3.847)}
\gppoint{gp mark 0}{(4.140,4.408)}
\gppoint{gp mark 0}{(4.140,3.561)}
\gppoint{gp mark 0}{(4.140,4.034)}
\gppoint{gp mark 0}{(4.140,3.521)}
\gppoint{gp mark 0}{(4.140,4.322)}
\gppoint{gp mark 0}{(4.140,3.255)}
\gppoint{gp mark 0}{(4.140,3.656)}
\gppoint{gp mark 0}{(4.140,3.759)}
\gppoint{gp mark 0}{(4.140,4.236)}
\gppoint{gp mark 0}{(4.140,4.153)}
\gppoint{gp mark 0}{(4.140,3.918)}
\gppoint{gp mark 0}{(4.140,3.561)}
\gppoint{gp mark 0}{(4.140,3.561)}
\gppoint{gp mark 0}{(4.140,3.888)}
\gppoint{gp mark 0}{(4.140,4.137)}
\gppoint{gp mark 0}{(4.140,3.710)}
\gppoint{gp mark 0}{(4.140,3.598)}
\gppoint{gp mark 0}{(4.140,3.918)}
\gppoint{gp mark 0}{(4.140,3.521)}
\gppoint{gp mark 0}{(4.140,3.561)}
\gppoint{gp mark 0}{(4.140,4.153)}
\gppoint{gp mark 0}{(4.140,3.903)}
\gppoint{gp mark 0}{(4.140,3.768)}
\gppoint{gp mark 0}{(4.140,3.720)}
\gppoint{gp mark 0}{(4.140,3.995)}
\gppoint{gp mark 0}{(4.140,3.678)}
\gppoint{gp mark 0}{(4.140,3.561)}
\gppoint{gp mark 0}{(4.140,3.235)}
\gppoint{gp mark 0}{(4.140,3.535)}
\gppoint{gp mark 0}{(4.140,3.699)}
\gppoint{gp mark 0}{(4.140,3.493)}
\gppoint{gp mark 0}{(4.140,3.402)}
\gppoint{gp mark 0}{(4.140,3.787)}
\gppoint{gp mark 0}{(4.140,4.015)}
\gppoint{gp mark 0}{(4.140,3.720)}
\gppoint{gp mark 0}{(4.140,3.699)}
\gppoint{gp mark 0}{(4.140,3.847)}
\gppoint{gp mark 0}{(4.140,3.645)}
\gppoint{gp mark 0}{(4.140,4.015)}
\gppoint{gp mark 0}{(4.140,4.046)}
\gppoint{gp mark 0}{(4.140,3.434)}
\gppoint{gp mark 0}{(4.140,4.314)}
\gppoint{gp mark 0}{(4.140,3.493)}
\gppoint{gp mark 0}{(4.140,3.449)}
\gppoint{gp mark 0}{(4.140,3.645)}
\gppoint{gp mark 0}{(4.140,3.464)}
\gppoint{gp mark 0}{(4.140,3.434)}
\gppoint{gp mark 0}{(4.140,3.872)}
\gppoint{gp mark 0}{(4.140,3.385)}
\gppoint{gp mark 0}{(4.140,3.295)}
\gppoint{gp mark 0}{(4.140,3.903)}
\gppoint{gp mark 0}{(4.140,3.678)}
\gppoint{gp mark 0}{(4.140,3.493)}
\gppoint{gp mark 0}{(4.140,4.357)}
\gppoint{gp mark 0}{(4.140,3.678)}
\gppoint{gp mark 0}{(4.140,3.586)}
\gppoint{gp mark 0}{(4.140,3.813)}
\gppoint{gp mark 0}{(4.140,3.678)}
\gppoint{gp mark 0}{(4.140,3.493)}
\gppoint{gp mark 0}{(4.140,3.678)}
\gppoint{gp mark 0}{(4.140,3.493)}
\gppoint{gp mark 0}{(4.140,3.586)}
\gppoint{gp mark 0}{(4.140,3.796)}
\gppoint{gp mark 0}{(4.140,3.351)}
\gppoint{gp mark 0}{(4.140,3.749)}
\gppoint{gp mark 0}{(4.140,4.070)}
\gppoint{gp mark 0}{(4.140,3.235)}
\gppoint{gp mark 0}{(4.140,3.464)}
\gppoint{gp mark 0}{(4.140,3.678)}
\gppoint{gp mark 0}{(4.140,4.143)}
\gppoint{gp mark 0}{(4.140,4.436)}
\gppoint{gp mark 0}{(4.140,3.678)}
\gppoint{gp mark 0}{(4.140,3.678)}
\gppoint{gp mark 0}{(4.140,4.289)}
\gppoint{gp mark 0}{(4.140,3.434)}
\gppoint{gp mark 0}{(4.140,4.365)}
\gppoint{gp mark 0}{(4.140,3.573)}
\gppoint{gp mark 0}{(4.140,3.493)}
\gppoint{gp mark 0}{(4.140,3.645)}
\gppoint{gp mark 0}{(4.140,4.419)}
\gppoint{gp mark 0}{(4.140,4.015)}
\gppoint{gp mark 0}{(4.140,3.720)}
\gppoint{gp mark 0}{(4.140,4.322)}
\gppoint{gp mark 0}{(4.140,3.535)}
\gppoint{gp mark 0}{(4.140,3.918)}
\gppoint{gp mark 0}{(4.140,3.888)}
\gppoint{gp mark 0}{(4.140,3.656)}
\gppoint{gp mark 0}{(4.140,3.778)}
\gppoint{gp mark 0}{(4.140,3.778)}
\gppoint{gp mark 0}{(4.140,4.218)}
\gppoint{gp mark 0}{(4.140,3.699)}
\gppoint{gp mark 0}{(4.140,3.813)}
\gppoint{gp mark 0}{(4.140,3.678)}
\gppoint{gp mark 0}{(4.140,3.678)}
\gppoint{gp mark 0}{(4.140,3.778)}
\gppoint{gp mark 0}{(4.140,4.419)}
\gppoint{gp mark 0}{(4.140,3.276)}
\gppoint{gp mark 0}{(4.140,3.295)}
\gppoint{gp mark 0}{(4.140,3.888)}
\gppoint{gp mark 0}{(4.140,3.813)}
\gppoint{gp mark 0}{(4.140,3.689)}
\gppoint{gp mark 0}{(4.140,3.787)}
\gppoint{gp mark 0}{(4.140,3.678)}
\gppoint{gp mark 0}{(4.140,3.521)}
\gppoint{gp mark 0}{(4.140,4.280)}
\gppoint{gp mark 0}{(4.140,4.387)}
\gppoint{gp mark 0}{(4.140,3.768)}
\gppoint{gp mark 0}{(4.140,3.872)}
\gppoint{gp mark 0}{(4.140,3.535)}
\gppoint{gp mark 0}{(4.140,3.895)}
\gppoint{gp mark 0}{(4.140,4.208)}
\gppoint{gp mark 0}{(4.140,3.839)}
\gppoint{gp mark 0}{(4.140,4.419)}
\gppoint{gp mark 0}{(4.140,3.768)}
\gppoint{gp mark 0}{(4.140,3.839)}
\gppoint{gp mark 0}{(4.140,3.385)}
\gppoint{gp mark 0}{(4.140,4.330)}
\gppoint{gp mark 0}{(4.140,3.493)}
\gppoint{gp mark 0}{(4.140,3.872)}
\gppoint{gp mark 0}{(4.140,3.678)}
\gppoint{gp mark 0}{(4.140,3.548)}
\gppoint{gp mark 0}{(4.140,3.548)}
\gppoint{gp mark 0}{(4.140,3.622)}
\gppoint{gp mark 0}{(4.140,3.351)}
\gppoint{gp mark 0}{(4.140,3.864)}
\gppoint{gp mark 0}{(4.140,3.493)}
\gppoint{gp mark 0}{(4.140,3.385)}
\gppoint{gp mark 0}{(4.140,3.888)}
\gppoint{gp mark 0}{(4.140,4.070)}
\gppoint{gp mark 0}{(4.140,3.385)}
\gppoint{gp mark 0}{(4.140,3.805)}
\gppoint{gp mark 0}{(4.140,3.255)}
\gppoint{gp mark 0}{(4.140,3.295)}
\gppoint{gp mark 0}{(4.140,3.295)}
\gppoint{gp mark 0}{(4.140,4.419)}
\gppoint{gp mark 0}{(4.140,3.796)}
\gppoint{gp mark 0}{(4.140,3.295)}
\gppoint{gp mark 0}{(4.140,3.689)}
\gppoint{gp mark 0}{(4.140,4.419)}
\gppoint{gp mark 0}{(4.140,3.295)}
\gppoint{gp mark 0}{(4.140,3.656)}
\gppoint{gp mark 0}{(4.140,3.888)}
\gppoint{gp mark 0}{(4.140,3.493)}
\gppoint{gp mark 0}{(4.140,3.822)}
\gppoint{gp mark 0}{(4.140,3.493)}
\gppoint{gp mark 0}{(4.140,3.933)}
\gppoint{gp mark 0}{(4.140,4.419)}
\gppoint{gp mark 0}{(4.140,4.419)}
\gppoint{gp mark 0}{(4.140,4.419)}
\gppoint{gp mark 0}{(4.140,3.385)}
\gppoint{gp mark 0}{(4.140,3.351)}
\gppoint{gp mark 0}{(4.140,3.678)}
\gppoint{gp mark 0}{(4.140,4.419)}
\gppoint{gp mark 0}{(4.140,4.556)}
\gppoint{gp mark 0}{(4.140,3.940)}
\gppoint{gp mark 0}{(4.140,4.318)}
\gppoint{gp mark 0}{(4.140,4.419)}
\gppoint{gp mark 0}{(4.140,4.259)}
\gppoint{gp mark 0}{(4.140,3.385)}
\gppoint{gp mark 0}{(4.140,3.678)}
\gppoint{gp mark 0}{(4.140,3.778)}
\gppoint{gp mark 0}{(4.140,3.689)}
\gppoint{gp mark 0}{(4.140,3.656)}
\gppoint{gp mark 0}{(4.140,4.015)}
\gppoint{gp mark 0}{(4.140,3.710)}
\gppoint{gp mark 0}{(4.140,3.418)}
\gppoint{gp mark 0}{(4.140,3.872)}
\gppoint{gp mark 0}{(4.140,3.333)}
\gppoint{gp mark 0}{(4.140,3.678)}
\gppoint{gp mark 0}{(4.140,3.586)}
\gppoint{gp mark 0}{(4.140,3.586)}
\gppoint{gp mark 0}{(4.140,3.918)}
\gppoint{gp mark 0}{(4.140,3.954)}
\gppoint{gp mark 0}{(4.140,3.678)}
\gppoint{gp mark 0}{(4.140,3.796)}
\gppoint{gp mark 0}{(4.140,3.678)}
\gppoint{gp mark 0}{(4.140,3.864)}
\gppoint{gp mark 0}{(4.140,3.864)}
\gppoint{gp mark 0}{(4.140,4.070)}
\gppoint{gp mark 0}{(4.140,3.561)}
\gppoint{gp mark 0}{(4.140,3.561)}
\gppoint{gp mark 0}{(4.140,4.250)}
\gppoint{gp mark 0}{(4.140,3.864)}
\gppoint{gp mark 0}{(4.140,3.864)}
\gppoint{gp mark 0}{(4.140,3.678)}
\gppoint{gp mark 0}{(4.140,3.351)}
\gppoint{gp mark 0}{(4.140,3.864)}
\gppoint{gp mark 0}{(4.140,3.864)}
\gppoint{gp mark 0}{(4.140,3.689)}
\gppoint{gp mark 0}{(4.140,3.678)}
\gppoint{gp mark 0}{(4.140,3.918)}
\gppoint{gp mark 0}{(4.140,4.169)}
\gppoint{gp mark 0}{(4.140,3.678)}
\gppoint{gp mark 0}{(4.140,4.302)}
\gppoint{gp mark 0}{(4.140,4.058)}
\gppoint{gp mark 0}{(4.140,3.561)}
\gppoint{gp mark 0}{(4.140,3.822)}
\gppoint{gp mark 0}{(4.140,3.235)}
\gppoint{gp mark 0}{(4.140,4.293)}
\gppoint{gp mark 0}{(4.140,3.982)}
\gppoint{gp mark 0}{(4.140,3.678)}
\gppoint{gp mark 0}{(4.140,3.678)}
\gppoint{gp mark 0}{(4.140,3.933)}
\gppoint{gp mark 0}{(4.140,3.634)}
\gppoint{gp mark 0}{(4.140,3.295)}
\gppoint{gp mark 0}{(4.140,4.189)}
\gppoint{gp mark 0}{(4.140,3.749)}
\gppoint{gp mark 0}{(4.140,3.822)}
\gppoint{gp mark 0}{(4.140,3.434)}
\gppoint{gp mark 0}{(4.140,3.778)}
\gppoint{gp mark 0}{(4.140,4.318)}
\gppoint{gp mark 0}{(4.140,3.678)}
\gppoint{gp mark 0}{(4.140,4.272)}
\gppoint{gp mark 0}{(4.140,4.272)}
\gppoint{gp mark 0}{(4.140,3.622)}
\gppoint{gp mark 0}{(4.140,4.021)}
\gppoint{gp mark 0}{(4.140,3.561)}
\gppoint{gp mark 0}{(4.140,3.678)}
\gppoint{gp mark 0}{(4.140,3.678)}
\gppoint{gp mark 0}{(4.140,4.002)}
\gppoint{gp mark 0}{(4.140,3.235)}
\gppoint{gp mark 0}{(4.140,3.464)}
\gppoint{gp mark 0}{(4.140,4.021)}
\gppoint{gp mark 0}{(4.140,3.598)}
\gppoint{gp mark 0}{(4.140,3.507)}
\gppoint{gp mark 0}{(4.140,3.678)}
\gppoint{gp mark 0}{(4.140,3.561)}
\gppoint{gp mark 0}{(4.140,4.232)}
\gppoint{gp mark 0}{(4.140,3.449)}
\gppoint{gp mark 0}{(4.140,3.678)}
\gppoint{gp mark 0}{(4.140,3.434)}
\gppoint{gp mark 0}{(4.140,3.678)}
\gppoint{gp mark 0}{(4.140,3.351)}
\gppoint{gp mark 0}{(4.140,3.730)}
\gppoint{gp mark 0}{(4.186,3.507)}
\gppoint{gp mark 0}{(4.186,3.493)}
\gppoint{gp mark 0}{(4.186,3.982)}
\gppoint{gp mark 0}{(4.186,3.796)}
\gppoint{gp mark 0}{(4.186,3.720)}
\gppoint{gp mark 0}{(4.186,3.768)}
\gppoint{gp mark 0}{(4.186,3.778)}
\gppoint{gp mark 0}{(4.186,4.194)}
\gppoint{gp mark 0}{(4.186,4.326)}
\gppoint{gp mark 0}{(4.186,3.911)}
\gppoint{gp mark 0}{(4.186,3.385)}
\gppoint{gp mark 0}{(4.186,3.351)}
\gppoint{gp mark 0}{(4.186,3.168)}
\gppoint{gp mark 0}{(4.186,3.667)}
\gppoint{gp mark 0}{(4.186,3.678)}
\gppoint{gp mark 0}{(4.186,4.326)}
\gppoint{gp mark 0}{(4.186,3.449)}
\gppoint{gp mark 0}{(4.186,3.989)}
\gppoint{gp mark 0}{(4.186,3.418)}
\gppoint{gp mark 0}{(4.186,3.535)}
\gppoint{gp mark 0}{(4.186,4.530)}
\gppoint{gp mark 0}{(4.186,3.598)}
\gppoint{gp mark 0}{(4.186,3.295)}
\gppoint{gp mark 0}{(4.186,3.634)}
\gppoint{gp mark 0}{(4.186,3.689)}
\gppoint{gp mark 0}{(4.186,3.995)}
\gppoint{gp mark 0}{(4.186,3.586)}
\gppoint{gp mark 0}{(4.186,3.434)}
\gppoint{gp mark 0}{(4.186,3.402)}
\gppoint{gp mark 0}{(4.186,3.634)}
\gppoint{gp mark 0}{(4.186,4.573)}
\gppoint{gp mark 0}{(4.186,3.598)}
\gppoint{gp mark 0}{(4.186,3.402)}
\gppoint{gp mark 0}{(4.186,3.402)}
\gppoint{gp mark 0}{(4.186,3.521)}
\gppoint{gp mark 0}{(4.186,3.548)}
\gppoint{gp mark 0}{(4.186,3.903)}
\gppoint{gp mark 0}{(4.186,3.678)}
\gppoint{gp mark 0}{(4.186,3.759)}
\gppoint{gp mark 0}{(4.186,3.856)}
\gppoint{gp mark 0}{(4.186,3.622)}
\gppoint{gp mark 0}{(4.186,3.548)}
\gppoint{gp mark 0}{(4.186,3.418)}
\gppoint{gp mark 0}{(4.186,3.213)}
\gppoint{gp mark 0}{(4.186,3.975)}
\gppoint{gp mark 0}{(4.186,3.434)}
\gppoint{gp mark 0}{(4.186,4.330)}
\gppoint{gp mark 0}{(4.186,4.153)}
\gppoint{gp mark 0}{(4.186,3.385)}
\gppoint{gp mark 0}{(4.186,3.479)}
\gppoint{gp mark 0}{(4.186,3.548)}
\gppoint{gp mark 0}{(4.186,3.710)}
\gppoint{gp mark 0}{(4.186,3.759)}
\gppoint{gp mark 0}{(4.186,3.548)}
\gppoint{gp mark 0}{(4.186,4.491)}
\gppoint{gp mark 0}{(4.186,3.805)}
\gppoint{gp mark 0}{(4.186,3.385)}
\gppoint{gp mark 0}{(4.186,3.434)}
\gppoint{gp mark 0}{(4.186,3.645)}
\gppoint{gp mark 0}{(4.186,3.521)}
\gppoint{gp mark 0}{(4.186,3.740)}
\gppoint{gp mark 0}{(4.186,3.507)}
\gppoint{gp mark 0}{(4.186,3.573)}
\gppoint{gp mark 0}{(4.186,3.235)}
\gppoint{gp mark 0}{(4.186,4.267)}
\gppoint{gp mark 0}{(4.186,3.535)}
\gppoint{gp mark 0}{(4.186,3.368)}
\gppoint{gp mark 0}{(4.186,3.940)}
\gppoint{gp mark 0}{(4.186,3.610)}
\gppoint{gp mark 0}{(4.186,3.479)}
\gppoint{gp mark 0}{(4.186,3.730)}
\gppoint{gp mark 0}{(4.186,3.610)}
\gppoint{gp mark 0}{(4.186,3.586)}
\gppoint{gp mark 0}{(4.186,3.975)}
\gppoint{gp mark 0}{(4.186,3.333)}
\gppoint{gp mark 0}{(4.186,3.656)}
\gppoint{gp mark 0}{(4.186,3.402)}
\gppoint{gp mark 0}{(4.186,3.778)}
\gppoint{gp mark 0}{(4.186,3.822)}
\gppoint{gp mark 0}{(4.186,3.598)}
\gppoint{gp mark 0}{(4.186,3.730)}
\gppoint{gp mark 0}{(4.186,3.610)}
\gppoint{gp mark 0}{(4.186,3.402)}
\gppoint{gp mark 0}{(4.186,3.822)}
\gppoint{gp mark 0}{(4.186,4.338)}
\gppoint{gp mark 0}{(4.186,3.947)}
\gppoint{gp mark 0}{(4.186,3.507)}
\gppoint{gp mark 0}{(4.186,3.385)}
\gppoint{gp mark 0}{(4.186,3.740)}
\gppoint{gp mark 0}{(4.186,3.586)}
\gppoint{gp mark 0}{(4.186,3.989)}
\gppoint{gp mark 0}{(4.186,3.947)}
\gppoint{gp mark 0}{(4.186,3.622)}
\gppoint{gp mark 0}{(4.186,3.507)}
\gppoint{gp mark 0}{(4.186,3.730)}
\gppoint{gp mark 0}{(4.186,3.507)}
\gppoint{gp mark 0}{(4.186,4.028)}
\gppoint{gp mark 0}{(4.186,3.493)}
\gppoint{gp mark 0}{(4.186,3.610)}
\gppoint{gp mark 0}{(4.186,3.656)}
\gppoint{gp mark 0}{(4.186,3.645)}
\gppoint{gp mark 0}{(4.186,3.989)}
\gppoint{gp mark 0}{(4.186,3.507)}
\gppoint{gp mark 0}{(4.186,3.548)}
\gppoint{gp mark 0}{(4.186,4.099)}
\gppoint{gp mark 0}{(4.186,3.507)}
\gppoint{gp mark 0}{(4.186,3.493)}
\gppoint{gp mark 0}{(4.186,3.479)}
\gppoint{gp mark 0}{(4.186,3.449)}
\gppoint{gp mark 0}{(4.186,3.548)}
\gppoint{gp mark 0}{(4.186,3.678)}
\gppoint{gp mark 0}{(4.186,3.813)}
\gppoint{gp mark 0}{(4.186,3.464)}
\gppoint{gp mark 0}{(4.186,3.548)}
\gppoint{gp mark 0}{(4.186,3.947)}
\gppoint{gp mark 0}{(4.186,3.699)}
\gppoint{gp mark 0}{(4.186,3.720)}
\gppoint{gp mark 0}{(4.186,3.333)}
\gppoint{gp mark 0}{(4.186,3.864)}
\gppoint{gp mark 0}{(4.186,3.995)}
\gppoint{gp mark 0}{(4.186,3.730)}
\gppoint{gp mark 0}{(4.186,3.333)}
\gppoint{gp mark 0}{(4.186,3.402)}
\gppoint{gp mark 0}{(4.186,4.267)}
\gppoint{gp mark 0}{(4.186,3.402)}
\gppoint{gp mark 0}{(4.186,3.778)}
\gppoint{gp mark 0}{(4.186,3.813)}
\gppoint{gp mark 0}{(4.186,4.046)}
\gppoint{gp mark 0}{(4.186,3.368)}
\gppoint{gp mark 0}{(4.186,3.479)}
\gppoint{gp mark 0}{(4.186,3.521)}
\gppoint{gp mark 0}{(4.186,3.573)}
\gppoint{gp mark 0}{(4.186,3.947)}
\gppoint{gp mark 0}{(4.186,3.768)}
\gppoint{gp mark 0}{(4.186,3.521)}
\gppoint{gp mark 0}{(4.186,3.610)}
\gppoint{gp mark 0}{(4.186,4.232)}
\gppoint{gp mark 0}{(4.186,3.634)}
\gppoint{gp mark 0}{(4.186,3.710)}
\gppoint{gp mark 0}{(4.186,3.678)}
\gppoint{gp mark 0}{(4.186,3.989)}
\gppoint{gp mark 0}{(4.186,4.028)}
\gppoint{gp mark 0}{(4.186,3.880)}
\gppoint{gp mark 0}{(4.186,3.813)}
\gppoint{gp mark 0}{(4.186,3.634)}
\gppoint{gp mark 0}{(4.186,3.678)}
\gppoint{gp mark 0}{(4.186,3.796)}
\gppoint{gp mark 0}{(4.186,3.479)}
\gppoint{gp mark 0}{(4.186,3.368)}
\gppoint{gp mark 0}{(4.186,3.548)}
\gppoint{gp mark 0}{(4.186,3.645)}
\gppoint{gp mark 0}{(4.186,4.015)}
\gppoint{gp mark 0}{(4.186,3.622)}
\gppoint{gp mark 0}{(4.186,3.479)}
\gppoint{gp mark 0}{(4.186,3.351)}
\gppoint{gp mark 0}{(4.186,4.398)}
\gppoint{gp mark 0}{(4.186,3.622)}
\gppoint{gp mark 0}{(4.186,4.002)}
\gppoint{gp mark 0}{(4.186,3.548)}
\gppoint{gp mark 0}{(4.186,3.351)}
\gppoint{gp mark 0}{(4.186,3.982)}
\gppoint{gp mark 0}{(4.186,3.689)}
\gppoint{gp mark 0}{(4.186,4.218)}
\gppoint{gp mark 0}{(4.186,4.610)}
\gppoint{gp mark 0}{(4.186,4.137)}
\gppoint{gp mark 0}{(4.186,3.940)}
\gppoint{gp mark 0}{(4.186,3.813)}
\gppoint{gp mark 0}{(4.186,3.276)}
\gppoint{gp mark 0}{(4.186,3.813)}
\gppoint{gp mark 0}{(4.186,4.158)}
\gppoint{gp mark 0}{(4.186,3.464)}
\gppoint{gp mark 0}{(4.186,4.158)}
\gppoint{gp mark 0}{(4.186,3.864)}
\gppoint{gp mark 0}{(4.186,3.995)}
\gppoint{gp mark 0}{(4.186,3.656)}
\gppoint{gp mark 0}{(4.186,3.805)}
\gppoint{gp mark 0}{(4.186,3.813)}
\gppoint{gp mark 0}{(4.186,3.548)}
\gppoint{gp mark 0}{(4.186,4.034)}
\gppoint{gp mark 0}{(4.186,3.778)}
\gppoint{gp mark 0}{(4.186,4.259)}
\gppoint{gp mark 0}{(4.186,3.805)}
\gppoint{gp mark 0}{(4.186,3.740)}
\gppoint{gp mark 0}{(4.186,3.385)}
\gppoint{gp mark 0}{(4.186,3.622)}
\gppoint{gp mark 0}{(4.186,3.856)}
\gppoint{gp mark 0}{(4.186,3.805)}
\gppoint{gp mark 0}{(4.186,3.493)}
\gppoint{gp mark 0}{(4.186,3.656)}
\gppoint{gp mark 0}{(4.186,3.295)}
\gppoint{gp mark 0}{(4.186,4.401)}
\gppoint{gp mark 0}{(4.186,3.880)}
\gppoint{gp mark 0}{(4.186,3.813)}
\gppoint{gp mark 0}{(4.186,3.598)}
\gppoint{gp mark 0}{(4.186,3.822)}
\gppoint{gp mark 0}{(4.186,4.099)}
\gppoint{gp mark 0}{(4.186,3.710)}
\gppoint{gp mark 0}{(4.186,3.864)}
\gppoint{gp mark 0}{(4.186,3.645)}
\gppoint{gp mark 0}{(4.186,3.730)}
\gppoint{gp mark 0}{(4.186,3.351)}
\gppoint{gp mark 0}{(4.186,3.847)}
\gppoint{gp mark 0}{(4.186,3.493)}
\gppoint{gp mark 0}{(4.186,3.759)}
\gppoint{gp mark 0}{(4.186,3.678)}
\gppoint{gp mark 0}{(4.186,3.730)}
\gppoint{gp mark 0}{(4.186,4.015)}
\gppoint{gp mark 0}{(4.186,4.132)}
\gppoint{gp mark 0}{(4.186,3.521)}
\gppoint{gp mark 0}{(4.186,3.521)}
\gppoint{gp mark 0}{(4.186,3.449)}
\gppoint{gp mark 0}{(4.186,3.805)}
\gppoint{gp mark 0}{(4.186,4.099)}
\gppoint{gp mark 0}{(4.186,3.493)}
\gppoint{gp mark 0}{(4.186,3.493)}
\gppoint{gp mark 0}{(4.186,3.548)}
\gppoint{gp mark 0}{(4.186,4.164)}
\gppoint{gp mark 0}{(4.186,3.864)}
\gppoint{gp mark 0}{(4.186,3.968)}
\gppoint{gp mark 0}{(4.186,3.710)}
\gppoint{gp mark 0}{(4.186,3.730)}
\gppoint{gp mark 0}{(4.186,3.730)}
\gppoint{gp mark 0}{(4.186,3.402)}
\gppoint{gp mark 0}{(4.186,3.434)}
\gppoint{gp mark 0}{(4.186,3.548)}
\gppoint{gp mark 0}{(4.186,3.295)}
\gppoint{gp mark 0}{(4.186,3.710)}
\gppoint{gp mark 0}{(4.186,3.598)}
\gppoint{gp mark 0}{(4.186,3.333)}
\gppoint{gp mark 0}{(4.186,3.493)}
\gppoint{gp mark 0}{(4.186,4.015)}
\gppoint{gp mark 0}{(4.186,3.667)}
\gppoint{gp mark 0}{(4.186,4.028)}
\gppoint{gp mark 0}{(4.186,4.412)}
\gppoint{gp mark 0}{(4.186,4.158)}
\gppoint{gp mark 0}{(4.186,3.720)}
\gppoint{gp mark 0}{(4.186,3.730)}
\gppoint{gp mark 0}{(4.186,3.479)}
\gppoint{gp mark 0}{(4.186,3.699)}
\gppoint{gp mark 0}{(4.186,3.740)}
\gppoint{gp mark 0}{(4.186,3.295)}
\gppoint{gp mark 0}{(4.186,3.880)}
\gppoint{gp mark 0}{(4.186,3.796)}
\gppoint{gp mark 0}{(4.186,4.515)}
\gppoint{gp mark 0}{(4.186,3.699)}
\gppoint{gp mark 0}{(4.186,4.398)}
\gppoint{gp mark 0}{(4.186,3.610)}
\gppoint{gp mark 0}{(4.186,3.645)}
\gppoint{gp mark 0}{(4.186,3.759)}
\gppoint{gp mark 0}{(4.186,4.412)}
\gppoint{gp mark 0}{(4.186,4.028)}
\gppoint{gp mark 0}{(4.186,3.573)}
\gppoint{gp mark 0}{(4.186,3.740)}
\gppoint{gp mark 0}{(4.186,3.351)}
\gppoint{gp mark 0}{(4.186,3.813)}
\gppoint{gp mark 0}{(4.230,3.610)}
\gppoint{gp mark 0}{(4.230,3.888)}
\gppoint{gp mark 0}{(4.230,3.768)}
\gppoint{gp mark 0}{(4.230,4.310)}
\gppoint{gp mark 0}{(4.230,3.586)}
\gppoint{gp mark 0}{(4.230,3.699)}
\gppoint{gp mark 0}{(4.230,4.518)}
\gppoint{gp mark 0}{(4.230,4.164)}
\gppoint{gp mark 0}{(4.230,4.357)}
\gppoint{gp mark 0}{(4.230,4.419)}
\gppoint{gp mark 0}{(4.230,3.699)}
\gppoint{gp mark 0}{(4.230,3.521)}
\gppoint{gp mark 0}{(4.230,3.010)}
\gppoint{gp mark 0}{(4.230,4.285)}
\gppoint{gp mark 0}{(4.230,3.507)}
\gppoint{gp mark 0}{(4.230,3.872)}
\gppoint{gp mark 0}{(4.230,3.710)}
\gppoint{gp mark 0}{(4.230,3.434)}
\gppoint{gp mark 0}{(4.230,4.592)}
\gppoint{gp mark 0}{(4.230,3.796)}
\gppoint{gp mark 0}{(4.230,4.082)}
\gppoint{gp mark 0}{(4.230,3.895)}
\gppoint{gp mark 0}{(4.230,3.805)}
\gppoint{gp mark 0}{(4.230,3.351)}
\gppoint{gp mark 0}{(4.230,4.419)}
\gppoint{gp mark 0}{(4.230,3.710)}
\gppoint{gp mark 0}{(4.230,3.954)}
\gppoint{gp mark 0}{(4.230,3.730)}
\gppoint{gp mark 0}{(4.230,3.982)}
\gppoint{gp mark 0}{(4.230,3.864)}
\gppoint{gp mark 0}{(4.230,3.805)}
\gppoint{gp mark 0}{(4.230,3.740)}
\gppoint{gp mark 0}{(4.230,3.982)}
\gppoint{gp mark 0}{(4.230,3.667)}
\gppoint{gp mark 0}{(4.230,3.982)}
\gppoint{gp mark 0}{(4.230,3.521)}
\gppoint{gp mark 0}{(4.230,3.548)}
\gppoint{gp mark 0}{(4.230,3.880)}
\gppoint{gp mark 0}{(4.230,3.548)}
\gppoint{gp mark 0}{(4.230,4.222)}
\gppoint{gp mark 0}{(4.230,3.710)}
\gppoint{gp mark 0}{(4.230,3.778)}
\gppoint{gp mark 0}{(4.230,3.805)}
\gppoint{gp mark 0}{(4.230,4.419)}
\gppoint{gp mark 0}{(4.230,3.351)}
\gppoint{gp mark 0}{(4.230,3.689)}
\gppoint{gp mark 0}{(4.230,4.148)}
\gppoint{gp mark 0}{(4.230,3.918)}
\gppoint{gp mark 0}{(4.230,4.401)}
\gppoint{gp mark 0}{(4.230,3.968)}
\gppoint{gp mark 0}{(4.230,4.342)}
\gppoint{gp mark 0}{(4.230,3.940)}
\gppoint{gp mark 0}{(4.230,3.634)}
\gppoint{gp mark 0}{(4.230,3.740)}
\gppoint{gp mark 0}{(4.230,3.888)}
\gppoint{gp mark 0}{(4.230,3.573)}
\gppoint{gp mark 0}{(4.230,4.419)}
\gppoint{gp mark 0}{(4.230,3.720)}
\gppoint{gp mark 0}{(4.230,4.222)}
\gppoint{gp mark 0}{(4.230,3.610)}
\gppoint{gp mark 0}{(4.230,3.989)}
\gppoint{gp mark 0}{(4.230,4.179)}
\gppoint{gp mark 0}{(4.230,3.645)}
\gppoint{gp mark 0}{(4.230,3.710)}
\gppoint{gp mark 0}{(4.230,3.730)}
\gppoint{gp mark 0}{(4.230,3.351)}
\gppoint{gp mark 0}{(4.230,4.143)}
\gppoint{gp mark 0}{(4.230,3.598)}
\gppoint{gp mark 0}{(4.230,3.710)}
\gppoint{gp mark 0}{(4.230,3.778)}
\gppoint{gp mark 0}{(4.230,3.911)}
\gppoint{gp mark 0}{(4.230,4.143)}
\gppoint{gp mark 0}{(4.230,3.010)}
\gppoint{gp mark 0}{(4.230,3.634)}
\gppoint{gp mark 0}{(4.230,3.493)}
\gppoint{gp mark 0}{(4.230,3.561)}
\gppoint{gp mark 0}{(4.230,3.968)}
\gppoint{gp mark 0}{(4.230,4.076)}
\gppoint{gp mark 0}{(4.230,4.280)}
\gppoint{gp mark 0}{(4.230,3.586)}
\gppoint{gp mark 0}{(4.230,3.778)}
\gppoint{gp mark 0}{(4.230,4.422)}
\gppoint{gp mark 0}{(4.230,4.412)}
\gppoint{gp mark 0}{(4.230,3.351)}
\gppoint{gp mark 0}{(4.230,3.805)}
\gppoint{gp mark 0}{(4.230,3.449)}
\gppoint{gp mark 0}{(4.230,3.947)}
\gppoint{gp mark 0}{(4.230,3.940)}
\gppoint{gp mark 0}{(4.230,3.449)}
\gppoint{gp mark 0}{(4.230,4.105)}
\gppoint{gp mark 0}{(4.230,3.586)}
\gppoint{gp mark 0}{(4.230,3.598)}
\gppoint{gp mark 0}{(4.230,3.940)}
\gppoint{gp mark 0}{(4.230,3.968)}
\gppoint{gp mark 0}{(4.230,3.634)}
\gppoint{gp mark 0}{(4.230,3.645)}
\gppoint{gp mark 0}{(4.230,3.119)}
\gppoint{gp mark 0}{(4.230,3.634)}
\gppoint{gp mark 0}{(4.230,3.586)}
\gppoint{gp mark 0}{(4.230,3.535)}
\gppoint{gp mark 0}{(4.230,3.561)}
\gppoint{gp mark 0}{(4.230,3.778)}
\gppoint{gp mark 0}{(4.230,3.805)}
\gppoint{gp mark 0}{(4.230,3.561)}
\gppoint{gp mark 0}{(4.230,3.368)}
\gppoint{gp mark 0}{(4.230,3.933)}
\gppoint{gp mark 0}{(4.230,3.710)}
\gppoint{gp mark 0}{(4.230,3.314)}
\gppoint{gp mark 0}{(4.230,4.330)}
\gppoint{gp mark 0}{(4.230,3.449)}
\gppoint{gp mark 0}{(4.230,3.645)}
\gppoint{gp mark 0}{(4.230,3.749)}
\gppoint{gp mark 0}{(4.230,4.586)}
\gppoint{gp mark 0}{(4.230,3.634)}
\gppoint{gp mark 0}{(4.230,3.759)}
\gppoint{gp mark 0}{(4.230,3.561)}
\gppoint{gp mark 0}{(4.230,3.235)}
\gppoint{gp mark 0}{(4.230,4.116)}
\gppoint{gp mark 0}{(4.230,3.839)}
\gppoint{gp mark 0}{(4.230,3.947)}
\gppoint{gp mark 0}{(4.230,3.402)}
\gppoint{gp mark 0}{(4.230,4.241)}
\gppoint{gp mark 0}{(4.230,3.678)}
\gppoint{gp mark 0}{(4.230,3.598)}
\gppoint{gp mark 0}{(4.230,4.179)}
\gppoint{gp mark 0}{(4.230,3.918)}
\gppoint{gp mark 0}{(4.230,3.918)}
\gppoint{gp mark 0}{(4.230,3.699)}
\gppoint{gp mark 0}{(4.230,3.710)}
\gppoint{gp mark 0}{(4.230,3.351)}
\gppoint{gp mark 0}{(4.230,3.314)}
\gppoint{gp mark 0}{(4.230,4.342)}
\gppoint{gp mark 0}{(4.230,3.521)}
\gppoint{gp mark 0}{(4.230,3.521)}
\gppoint{gp mark 0}{(4.230,3.535)}
\gppoint{gp mark 0}{(4.230,3.351)}
\gppoint{gp mark 0}{(4.230,4.046)}
\gppoint{gp mark 0}{(4.230,4.433)}
\gppoint{gp mark 0}{(4.230,3.561)}
\gppoint{gp mark 0}{(4.230,3.402)}
\gppoint{gp mark 0}{(4.230,3.561)}
\gppoint{gp mark 0}{(4.230,4.194)}
\gppoint{gp mark 0}{(4.230,3.678)}
\gppoint{gp mark 0}{(4.230,3.276)}
\gppoint{gp mark 0}{(4.230,3.535)}
\gppoint{gp mark 0}{(4.230,3.351)}
\gppoint{gp mark 0}{(4.230,3.521)}
\gppoint{gp mark 0}{(4.230,3.699)}
\gppoint{gp mark 0}{(4.230,4.592)}
\gppoint{gp mark 0}{(4.230,3.507)}
\gppoint{gp mark 0}{(4.230,3.402)}
\gppoint{gp mark 0}{(4.230,3.464)}
\gppoint{gp mark 0}{(4.230,3.720)}
\gppoint{gp mark 0}{(4.230,3.507)}
\gppoint{gp mark 0}{(4.230,4.070)}
\gppoint{gp mark 0}{(4.230,3.385)}
\gppoint{gp mark 0}{(4.230,3.895)}
\gppoint{gp mark 0}{(4.230,3.235)}
\gppoint{gp mark 0}{(4.230,3.610)}
\gppoint{gp mark 0}{(4.230,3.622)}
\gppoint{gp mark 0}{(4.230,3.385)}
\gppoint{gp mark 0}{(4.230,3.507)}
\gppoint{gp mark 0}{(4.230,3.947)}
\gppoint{gp mark 0}{(4.230,3.479)}
\gppoint{gp mark 0}{(4.230,3.968)}
\gppoint{gp mark 0}{(4.230,3.295)}
\gppoint{gp mark 0}{(4.230,3.689)}
\gppoint{gp mark 0}{(4.230,4.218)}
\gppoint{gp mark 0}{(4.230,4.203)}
\gppoint{gp mark 0}{(4.230,3.610)}
\gppoint{gp mark 0}{(4.230,3.839)}
\gppoint{gp mark 0}{(4.230,3.645)}
\gppoint{gp mark 0}{(4.230,3.548)}
\gppoint{gp mark 0}{(4.230,3.888)}
\gppoint{gp mark 0}{(4.230,3.314)}
\gppoint{gp mark 0}{(4.230,3.678)}
\gppoint{gp mark 0}{(4.230,3.759)}
\gppoint{gp mark 0}{(4.230,4.028)}
\gppoint{gp mark 0}{(4.230,3.796)}
\gppoint{gp mark 0}{(4.230,3.982)}
\gppoint{gp mark 0}{(4.230,3.895)}
\gppoint{gp mark 0}{(4.230,3.895)}
\gppoint{gp mark 0}{(4.230,3.586)}
\gppoint{gp mark 0}{(4.230,4.028)}
\gppoint{gp mark 0}{(4.230,4.116)}
\gppoint{gp mark 0}{(4.230,3.521)}
\gppoint{gp mark 0}{(4.230,3.521)}
\gppoint{gp mark 0}{(4.230,4.194)}
\gppoint{gp mark 0}{(4.230,3.667)}
\gppoint{gp mark 0}{(4.230,3.402)}
\gppoint{gp mark 0}{(4.230,4.116)}
\gppoint{gp mark 0}{(4.230,3.895)}
\gppoint{gp mark 0}{(4.230,4.368)}
\gppoint{gp mark 0}{(4.230,3.493)}
\gppoint{gp mark 0}{(4.230,3.699)}
\gppoint{gp mark 0}{(4.230,4.314)}
\gppoint{gp mark 0}{(4.230,3.968)}
\gppoint{gp mark 0}{(4.230,3.521)}
\gppoint{gp mark 0}{(4.230,3.645)}
\gppoint{gp mark 0}{(4.230,3.479)}
\gppoint{gp mark 0}{(4.230,3.598)}
\gppoint{gp mark 0}{(4.230,3.402)}
\gppoint{gp mark 0}{(4.230,3.864)}
\gppoint{gp mark 0}{(4.230,4.408)}
\gppoint{gp mark 0}{(4.230,3.507)}
\gppoint{gp mark 0}{(4.230,3.759)}
\gppoint{gp mark 0}{(4.230,3.622)}
\gppoint{gp mark 0}{(4.230,3.880)}
\gppoint{gp mark 0}{(4.230,3.947)}
\gppoint{gp mark 0}{(4.230,3.975)}
\gppoint{gp mark 0}{(4.230,4.322)}
\gppoint{gp mark 0}{(4.230,3.778)}
\gppoint{gp mark 0}{(4.230,4.093)}
\gppoint{gp mark 0}{(4.230,3.521)}
\gppoint{gp mark 0}{(4.230,4.143)}
\gppoint{gp mark 0}{(4.230,3.805)}
\gppoint{gp mark 0}{(4.230,4.643)}
\gppoint{gp mark 0}{(4.230,4.218)}
\gppoint{gp mark 0}{(4.230,3.521)}
\gppoint{gp mark 0}{(4.230,4.334)}
\gppoint{gp mark 0}{(4.230,3.880)}
\gppoint{gp mark 0}{(4.230,4.623)}
\gppoint{gp mark 0}{(4.230,3.586)}
\gppoint{gp mark 0}{(4.230,3.586)}
\gppoint{gp mark 0}{(4.230,4.213)}
\gppoint{gp mark 0}{(4.230,3.710)}
\gppoint{gp mark 0}{(4.230,4.046)}
\gppoint{gp mark 0}{(4.230,3.573)}
\gppoint{gp mark 0}{(4.230,3.864)}
\gppoint{gp mark 0}{(4.230,3.689)}
\gppoint{gp mark 0}{(4.230,3.710)}
\gppoint{gp mark 0}{(4.230,4.046)}
\gppoint{gp mark 0}{(4.230,3.368)}
\gppoint{gp mark 0}{(4.230,3.521)}
\gppoint{gp mark 0}{(4.230,4.383)}
\gppoint{gp mark 0}{(4.230,3.351)}
\gppoint{gp mark 0}{(4.230,3.610)}
\gppoint{gp mark 0}{(4.230,4.099)}
\gppoint{gp mark 0}{(4.230,3.449)}
\gppoint{gp mark 0}{(4.230,4.241)}
\gppoint{gp mark 0}{(4.230,3.749)}
\gppoint{gp mark 0}{(4.230,4.383)}
\gppoint{gp mark 0}{(4.230,3.548)}
\gppoint{gp mark 0}{(4.230,3.586)}
\gppoint{gp mark 0}{(4.230,3.933)}
\gppoint{gp mark 0}{(4.230,3.314)}
\gppoint{gp mark 0}{(4.230,3.911)}
\gppoint{gp mark 0}{(4.230,3.351)}
\gppoint{gp mark 0}{(4.230,3.689)}
\gppoint{gp mark 0}{(4.230,4.387)}
\gppoint{gp mark 0}{(4.230,3.598)}
\gppoint{gp mark 0}{(4.230,4.002)}
\gppoint{gp mark 0}{(4.230,3.610)}
\gppoint{gp mark 0}{(4.230,3.678)}
\gppoint{gp mark 0}{(4.230,3.710)}
\gppoint{gp mark 0}{(4.230,3.787)}
\gppoint{gp mark 0}{(4.230,4.169)}
\gppoint{gp mark 0}{(4.230,4.040)}
\gppoint{gp mark 0}{(4.230,3.507)}
\gppoint{gp mark 0}{(4.230,3.598)}
\gppoint{gp mark 0}{(4.230,3.880)}
\gppoint{gp mark 0}{(4.230,3.434)}
\gppoint{gp mark 0}{(4.230,3.449)}
\gppoint{gp mark 0}{(4.230,4.028)}
\gppoint{gp mark 0}{(4.230,3.295)}
\gppoint{gp mark 0}{(4.230,3.954)}
\gppoint{gp mark 0}{(4.230,3.351)}
\gppoint{gp mark 0}{(4.230,3.235)}
\gppoint{gp mark 0}{(4.230,3.333)}
\gppoint{gp mark 0}{(4.230,4.412)}
\gppoint{gp mark 0}{(4.230,3.235)}
\gppoint{gp mark 0}{(4.230,3.295)}
\gppoint{gp mark 0}{(4.230,3.667)}
\gppoint{gp mark 0}{(4.230,3.678)}
\gppoint{gp mark 0}{(4.230,4.174)}
\gppoint{gp mark 0}{(4.230,3.813)}
\gppoint{gp mark 0}{(4.230,3.276)}
\gppoint{gp mark 0}{(4.230,3.678)}
\gppoint{gp mark 0}{(4.230,3.598)}
\gppoint{gp mark 0}{(4.230,3.689)}
\gppoint{gp mark 0}{(4.230,3.678)}
\gppoint{gp mark 0}{(4.230,3.699)}
\gppoint{gp mark 0}{(4.230,3.402)}
\gppoint{gp mark 0}{(4.271,3.856)}
\gppoint{gp mark 0}{(4.271,4.110)}
\gppoint{gp mark 0}{(4.271,4.164)}
\gppoint{gp mark 0}{(4.271,4.121)}
\gppoint{gp mark 0}{(4.271,3.911)}
\gppoint{gp mark 0}{(4.271,3.610)}
\gppoint{gp mark 0}{(4.271,3.645)}
\gppoint{gp mark 0}{(4.271,4.586)}
\gppoint{gp mark 0}{(4.271,4.387)}
\gppoint{gp mark 0}{(4.271,3.796)}
\gppoint{gp mark 0}{(4.271,3.903)}
\gppoint{gp mark 0}{(4.271,3.759)}
\gppoint{gp mark 0}{(4.271,3.535)}
\gppoint{gp mark 0}{(4.271,3.548)}
\gppoint{gp mark 0}{(4.271,3.778)}
\gppoint{gp mark 0}{(4.271,3.995)}
\gppoint{gp mark 0}{(4.271,3.730)}
\gppoint{gp mark 0}{(4.271,3.895)}
\gppoint{gp mark 0}{(4.271,3.895)}
\gppoint{gp mark 0}{(4.271,4.506)}
\gppoint{gp mark 0}{(4.271,4.338)}
\gppoint{gp mark 0}{(4.271,4.338)}
\gppoint{gp mark 0}{(4.271,3.598)}
\gppoint{gp mark 0}{(4.271,4.105)}
\gppoint{gp mark 0}{(4.271,4.245)}
\gppoint{gp mark 0}{(4.271,4.245)}
\gppoint{gp mark 0}{(4.271,3.847)}
\gppoint{gp mark 0}{(4.271,3.493)}
\gppoint{gp mark 0}{(4.271,3.947)}
\gppoint{gp mark 0}{(4.271,3.864)}
\gppoint{gp mark 0}{(4.271,3.710)}
\gppoint{gp mark 0}{(4.271,3.402)}
\gppoint{gp mark 0}{(4.271,3.634)}
\gppoint{gp mark 0}{(4.271,3.699)}
\gppoint{gp mark 0}{(4.271,3.634)}
\gppoint{gp mark 0}{(4.271,3.634)}
\gppoint{gp mark 0}{(4.271,3.888)}
\gppoint{gp mark 0}{(4.271,4.697)}
\gppoint{gp mark 0}{(4.271,3.656)}
\gppoint{gp mark 0}{(4.271,3.813)}
\gppoint{gp mark 0}{(4.271,3.678)}
\gppoint{gp mark 0}{(4.271,3.710)}
\gppoint{gp mark 0}{(4.271,3.351)}
\gppoint{gp mark 0}{(4.271,3.989)}
\gppoint{gp mark 0}{(4.271,4.306)}
\gppoint{gp mark 0}{(4.271,4.334)}
\gppoint{gp mark 0}{(4.271,4.334)}
\gppoint{gp mark 0}{(4.271,4.179)}
\gppoint{gp mark 0}{(4.271,4.070)}
\gppoint{gp mark 0}{(4.271,3.940)}
\gppoint{gp mark 0}{(4.271,3.434)}
\gppoint{gp mark 0}{(4.271,3.535)}
\gppoint{gp mark 0}{(4.271,3.548)}
\gppoint{gp mark 0}{(4.271,3.895)}
\gppoint{gp mark 0}{(4.271,3.822)}
\gppoint{gp mark 0}{(4.271,4.653)}
\gppoint{gp mark 0}{(4.271,3.535)}
\gppoint{gp mark 0}{(4.271,4.310)}
\gppoint{gp mark 0}{(4.271,3.521)}
\gppoint{gp mark 0}{(4.271,3.622)}
\gppoint{gp mark 0}{(4.271,3.333)}
\gppoint{gp mark 0}{(4.271,4.267)}
\gppoint{gp mark 0}{(4.271,4.368)}
\gppoint{gp mark 0}{(4.271,3.610)}
\gppoint{gp mark 0}{(4.271,4.334)}
\gppoint{gp mark 0}{(4.271,3.822)}
\gppoint{gp mark 0}{(4.271,4.127)}
\gppoint{gp mark 0}{(4.271,3.759)}
\gppoint{gp mark 0}{(4.271,3.333)}
\gppoint{gp mark 0}{(4.271,3.822)}
\gppoint{gp mark 0}{(4.271,4.132)}
\gppoint{gp mark 0}{(4.271,3.610)}
\gppoint{gp mark 0}{(4.271,3.730)}
\gppoint{gp mark 0}{(4.271,3.778)}
\gppoint{gp mark 0}{(4.271,3.940)}
\gppoint{gp mark 0}{(4.271,3.622)}
\gppoint{gp mark 0}{(4.271,4.408)}
\gppoint{gp mark 0}{(4.271,3.778)}
\gppoint{gp mark 0}{(4.271,3.710)}
\gppoint{gp mark 0}{(4.271,3.822)}
\gppoint{gp mark 0}{(4.271,3.295)}
\gppoint{gp mark 0}{(4.271,3.493)}
\gppoint{gp mark 0}{(4.271,3.548)}
\gppoint{gp mark 0}{(4.271,3.402)}
\gppoint{gp mark 0}{(4.271,3.699)}
\gppoint{gp mark 0}{(4.271,3.402)}
\gppoint{gp mark 0}{(4.271,3.645)}
\gppoint{gp mark 0}{(4.271,3.295)}
\gppoint{gp mark 0}{(4.271,3.778)}
\gppoint{gp mark 0}{(4.271,3.493)}
\gppoint{gp mark 0}{(4.271,3.493)}
\gppoint{gp mark 0}{(4.271,3.787)}
\gppoint{gp mark 0}{(4.271,3.521)}
\gppoint{gp mark 0}{(4.271,3.895)}
\gppoint{gp mark 0}{(4.271,3.813)}
\gppoint{gp mark 0}{(4.271,3.856)}
\gppoint{gp mark 0}{(4.271,3.895)}
\gppoint{gp mark 0}{(4.271,3.434)}
\gppoint{gp mark 0}{(4.271,4.127)}
\gppoint{gp mark 0}{(4.271,3.872)}
\gppoint{gp mark 0}{(4.271,4.408)}
\gppoint{gp mark 0}{(4.271,3.856)}
\gppoint{gp mark 0}{(4.271,3.822)}
\gppoint{gp mark 0}{(4.271,3.535)}
\gppoint{gp mark 0}{(4.271,3.402)}
\gppoint{gp mark 0}{(4.271,3.895)}
\gppoint{gp mark 0}{(4.271,3.535)}
\gppoint{gp mark 0}{(4.271,3.493)}
\gppoint{gp mark 0}{(4.271,3.598)}
\gppoint{gp mark 0}{(4.271,3.831)}
\gppoint{gp mark 0}{(4.271,3.598)}
\gppoint{gp mark 0}{(4.271,3.699)}
\gppoint{gp mark 0}{(4.271,3.645)}
\gppoint{gp mark 0}{(4.271,3.598)}
\gppoint{gp mark 0}{(4.271,3.768)}
\gppoint{gp mark 0}{(4.271,3.493)}
\gppoint{gp mark 0}{(4.271,3.903)}
\gppoint{gp mark 0}{(4.271,3.926)}
\gppoint{gp mark 0}{(4.271,4.121)}
\gppoint{gp mark 0}{(4.271,3.493)}
\gppoint{gp mark 0}{(4.271,3.839)}
\gppoint{gp mark 0}{(4.271,3.610)}
\gppoint{gp mark 0}{(4.271,3.778)}
\gppoint{gp mark 0}{(4.271,3.493)}
\gppoint{gp mark 0}{(4.271,3.645)}
\gppoint{gp mark 0}{(4.271,3.507)}
\gppoint{gp mark 0}{(4.271,3.385)}
\gppoint{gp mark 0}{(4.271,3.968)}
\gppoint{gp mark 0}{(4.271,3.831)}
\gppoint{gp mark 0}{(4.271,3.813)}
\gppoint{gp mark 0}{(4.271,3.645)}
\gppoint{gp mark 0}{(4.271,3.768)}
\gppoint{gp mark 0}{(4.271,3.689)}
\gppoint{gp mark 0}{(4.271,4.357)}
\gppoint{gp mark 0}{(4.271,3.822)}
\gppoint{gp mark 0}{(4.271,4.116)}
\gppoint{gp mark 0}{(4.271,4.488)}
\gppoint{gp mark 0}{(4.271,3.561)}
\gppoint{gp mark 0}{(4.271,4.028)}
\gppoint{gp mark 0}{(4.271,4.105)}
\gppoint{gp mark 0}{(4.271,3.521)}
\gppoint{gp mark 0}{(4.271,3.813)}
\gppoint{gp mark 0}{(4.271,3.610)}
\gppoint{gp mark 0}{(4.271,4.034)}
\gppoint{gp mark 0}{(4.271,3.831)}
\gppoint{gp mark 0}{(4.271,3.768)}
\gppoint{gp mark 0}{(4.271,3.535)}
\gppoint{gp mark 0}{(4.271,4.203)}
\gppoint{gp mark 0}{(4.271,3.610)}
\gppoint{gp mark 0}{(4.271,3.333)}
\gppoint{gp mark 0}{(4.271,4.245)}
\gppoint{gp mark 0}{(4.271,4.034)}
\gppoint{gp mark 0}{(4.271,4.245)}
\gppoint{gp mark 0}{(4.271,3.656)}
\gppoint{gp mark 0}{(4.271,3.689)}
\gppoint{gp mark 0}{(4.271,3.507)}
\gppoint{gp mark 0}{(4.271,4.482)}
\gppoint{gp mark 0}{(4.271,4.021)}
\gppoint{gp mark 0}{(4.271,3.351)}
\gppoint{gp mark 0}{(4.271,4.028)}
\gppoint{gp mark 0}{(4.271,4.127)}
\gppoint{gp mark 0}{(4.271,3.813)}
\gppoint{gp mark 0}{(4.271,3.813)}
\gppoint{gp mark 0}{(4.271,4.127)}
\gppoint{gp mark 0}{(4.271,3.479)}
\gppoint{gp mark 0}{(4.271,3.645)}
\gppoint{gp mark 0}{(4.271,3.759)}
\gppoint{gp mark 0}{(4.271,3.805)}
\gppoint{gp mark 0}{(4.271,3.749)}
\gppoint{gp mark 0}{(4.271,3.954)}
\gppoint{gp mark 0}{(4.271,3.333)}
\gppoint{gp mark 0}{(4.271,3.402)}
\gppoint{gp mark 0}{(4.271,3.961)}
\gppoint{gp mark 0}{(4.271,3.831)}
\gppoint{gp mark 0}{(4.271,4.076)}
\gppoint{gp mark 0}{(4.271,3.521)}
\gppoint{gp mark 0}{(4.271,3.493)}
\gppoint{gp mark 0}{(4.271,3.645)}
\gppoint{gp mark 0}{(4.271,3.479)}
\gppoint{gp mark 0}{(4.271,3.656)}
\gppoint{gp mark 0}{(4.271,3.449)}
\gppoint{gp mark 0}{(4.271,3.561)}
\gppoint{gp mark 0}{(4.271,3.507)}
\gppoint{gp mark 0}{(4.271,3.535)}
\gppoint{gp mark 0}{(4.271,3.787)}
\gppoint{gp mark 0}{(4.271,3.689)}
\gppoint{gp mark 0}{(4.271,4.482)}
\gppoint{gp mark 0}{(4.271,3.822)}
\gppoint{gp mark 0}{(4.271,3.507)}
\gppoint{gp mark 0}{(4.271,4.334)}
\gppoint{gp mark 0}{(4.271,3.831)}
\gppoint{gp mark 0}{(4.271,3.911)}
\gppoint{gp mark 0}{(4.271,3.940)}
\gppoint{gp mark 0}{(4.271,3.493)}
\gppoint{gp mark 0}{(4.271,4.127)}
\gppoint{gp mark 0}{(4.271,3.805)}
\gppoint{gp mark 0}{(4.271,3.479)}
\gppoint{gp mark 0}{(4.271,4.143)}
\gppoint{gp mark 0}{(4.271,3.759)}
\gppoint{gp mark 0}{(4.271,3.561)}
\gppoint{gp mark 0}{(4.271,3.402)}
\gppoint{gp mark 0}{(4.271,3.493)}
\gppoint{gp mark 0}{(4.271,3.295)}
\gppoint{gp mark 0}{(4.271,4.028)}
\gppoint{gp mark 0}{(4.271,4.058)}
\gppoint{gp mark 0}{(4.271,3.926)}
\gppoint{gp mark 0}{(4.271,3.895)}
\gppoint{gp mark 0}{(4.271,3.689)}
\gppoint{gp mark 0}{(4.271,3.831)}
\gppoint{gp mark 0}{(4.271,3.911)}
\gppoint{gp mark 0}{(4.271,3.895)}
\gppoint{gp mark 0}{(4.271,3.168)}
\gppoint{gp mark 0}{(4.271,3.768)}
\gppoint{gp mark 0}{(4.271,3.010)}
\gppoint{gp mark 0}{(4.271,3.740)}
\gppoint{gp mark 0}{(4.271,3.933)}
\gppoint{gp mark 0}{(4.271,3.831)}
\gppoint{gp mark 0}{(4.271,3.740)}
\gppoint{gp mark 0}{(4.271,4.076)}
\gppoint{gp mark 0}{(4.271,4.203)}
\gppoint{gp mark 0}{(4.271,3.995)}
\gppoint{gp mark 0}{(4.271,3.856)}
\gppoint{gp mark 0}{(4.271,3.710)}
\gppoint{gp mark 0}{(4.271,3.831)}
\gppoint{gp mark 0}{(4.271,4.076)}
\gppoint{gp mark 0}{(4.271,4.289)}
\gppoint{gp mark 0}{(4.271,3.989)}
\gppoint{gp mark 0}{(4.271,3.989)}
\gppoint{gp mark 0}{(4.271,3.933)}
\gppoint{gp mark 0}{(4.271,3.856)}
\gppoint{gp mark 0}{(4.271,3.864)}
\gppoint{gp mark 0}{(4.271,3.449)}
\gppoint{gp mark 0}{(4.271,4.497)}
\gppoint{gp mark 0}{(4.271,3.856)}
\gppoint{gp mark 0}{(4.271,3.933)}
\gppoint{gp mark 0}{(4.271,3.730)}
\gppoint{gp mark 0}{(4.271,3.479)}
\gppoint{gp mark 0}{(4.271,3.768)}
\gppoint{gp mark 0}{(4.271,3.831)}
\gppoint{gp mark 0}{(4.271,3.645)}
\gppoint{gp mark 0}{(4.271,3.989)}
\gppoint{gp mark 0}{(4.271,3.385)}
\gppoint{gp mark 0}{(4.271,3.864)}
\gppoint{gp mark 0}{(4.271,3.740)}
\gppoint{gp mark 0}{(4.271,3.831)}
\gppoint{gp mark 0}{(4.271,4.429)}
\gppoint{gp mark 0}{(4.271,4.148)}
\gppoint{gp mark 0}{(4.271,4.391)}
\gppoint{gp mark 0}{(4.271,3.295)}
\gppoint{gp mark 0}{(4.271,3.699)}
\gppoint{gp mark 0}{(4.271,3.989)}
\gppoint{gp mark 0}{(4.271,3.535)}
\gppoint{gp mark 0}{(4.271,4.158)}
\gppoint{gp mark 0}{(4.271,4.153)}
\gppoint{gp mark 0}{(4.271,3.561)}
\gppoint{gp mark 0}{(4.271,4.153)}
\gppoint{gp mark 0}{(4.271,3.479)}
\gppoint{gp mark 0}{(4.271,4.153)}
\gppoint{gp mark 0}{(4.311,3.295)}
\gppoint{gp mark 0}{(4.311,3.940)}
\gppoint{gp mark 0}{(4.311,3.813)}
\gppoint{gp mark 0}{(4.311,3.796)}
\gppoint{gp mark 0}{(4.311,4.272)}
\gppoint{gp mark 0}{(4.311,3.667)}
\gppoint{gp mark 0}{(4.311,4.318)}
\gppoint{gp mark 0}{(4.311,4.267)}
\gppoint{gp mark 0}{(4.311,3.479)}
\gppoint{gp mark 0}{(4.311,3.699)}
\gppoint{gp mark 0}{(4.311,3.586)}
\gppoint{gp mark 0}{(4.311,3.961)}
\gppoint{gp mark 0}{(4.311,4.227)}
\gppoint{gp mark 0}{(4.311,4.015)}
\gppoint{gp mark 0}{(4.311,3.535)}
\gppoint{gp mark 0}{(4.311,3.822)}
\gppoint{gp mark 0}{(4.311,3.622)}
\gppoint{gp mark 0}{(4.311,4.015)}
\gppoint{gp mark 0}{(4.311,3.535)}
\gppoint{gp mark 0}{(4.311,3.730)}
\gppoint{gp mark 0}{(4.311,3.561)}
\gppoint{gp mark 0}{(4.311,3.778)}
\gppoint{gp mark 0}{(4.311,3.610)}
\gppoint{gp mark 0}{(4.311,4.028)}
\gppoint{gp mark 0}{(4.311,3.730)}
\gppoint{gp mark 0}{(4.311,3.831)}
\gppoint{gp mark 0}{(4.311,4.076)}
\gppoint{gp mark 0}{(4.311,3.493)}
\gppoint{gp mark 0}{(4.311,3.730)}
\gppoint{gp mark 0}{(4.311,3.479)}
\gppoint{gp mark 0}{(4.311,3.586)}
\gppoint{gp mark 0}{(4.311,3.839)}
\gppoint{gp mark 0}{(4.311,4.699)}
\gppoint{gp mark 0}{(4.311,3.610)}
\gppoint{gp mark 0}{(4.311,3.720)}
\gppoint{gp mark 0}{(4.311,3.535)}
\gppoint{gp mark 0}{(4.311,3.831)}
\gppoint{gp mark 0}{(4.311,3.759)}
\gppoint{gp mark 0}{(4.311,4.338)}
\gppoint{gp mark 0}{(4.311,4.433)}
\gppoint{gp mark 0}{(4.311,4.040)}
\gppoint{gp mark 0}{(4.311,4.082)}
\gppoint{gp mark 0}{(4.311,3.961)}
\gppoint{gp mark 0}{(4.311,3.561)}
\gppoint{gp mark 0}{(4.311,3.521)}
\gppoint{gp mark 0}{(4.311,3.434)}
\gppoint{gp mark 0}{(4.311,3.255)}
\gppoint{gp mark 0}{(4.311,4.132)}
\gppoint{gp mark 0}{(4.311,4.028)}
\gppoint{gp mark 0}{(4.311,4.443)}
\gppoint{gp mark 0}{(4.311,4.132)}
\gppoint{gp mark 0}{(4.311,4.297)}
\gppoint{gp mark 0}{(4.311,3.622)}
\gppoint{gp mark 0}{(4.311,4.322)}
\gppoint{gp mark 0}{(4.311,3.535)}
\gppoint{gp mark 0}{(4.311,3.521)}
\gppoint{gp mark 0}{(4.311,4.419)}
\gppoint{gp mark 0}{(4.311,3.778)}
\gppoint{gp mark 0}{(4.311,3.521)}
\gppoint{gp mark 0}{(4.311,4.008)}
\gppoint{gp mark 0}{(4.311,3.521)}
\gppoint{gp mark 0}{(4.311,4.132)}
\gppoint{gp mark 0}{(4.311,3.385)}
\gppoint{gp mark 0}{(4.311,4.076)}
\gppoint{gp mark 0}{(4.311,3.678)}
\gppoint{gp mark 0}{(4.311,3.911)}
\gppoint{gp mark 0}{(4.311,4.132)}
\gppoint{gp mark 0}{(4.311,4.076)}
\gppoint{gp mark 0}{(4.311,4.076)}
\gppoint{gp mark 0}{(4.311,4.132)}
\gppoint{gp mark 0}{(4.311,3.573)}
\gppoint{gp mark 0}{(4.311,3.507)}
\gppoint{gp mark 0}{(4.311,4.148)}
\gppoint{gp mark 0}{(4.311,3.968)}
\gppoint{gp mark 0}{(4.311,3.434)}
\gppoint{gp mark 0}{(4.311,3.535)}
\gppoint{gp mark 0}{(4.311,3.968)}
\gppoint{gp mark 0}{(4.311,3.645)}
\gppoint{gp mark 0}{(4.311,3.678)}
\gppoint{gp mark 0}{(4.311,4.132)}
\gppoint{gp mark 0}{(4.311,3.740)}
\gppoint{gp mark 0}{(4.311,3.813)}
\gppoint{gp mark 0}{(4.311,3.975)}
\gppoint{gp mark 0}{(4.311,4.076)}
\gppoint{gp mark 0}{(4.311,4.338)}
\gppoint{gp mark 0}{(4.311,4.189)}
\gppoint{gp mark 0}{(4.311,3.610)}
\gppoint{gp mark 0}{(4.311,4.153)}
\gppoint{gp mark 0}{(4.311,3.645)}
\gppoint{gp mark 0}{(4.311,4.443)}
\gppoint{gp mark 0}{(4.311,4.365)}
\gppoint{gp mark 0}{(4.311,3.918)}
\gppoint{gp mark 0}{(4.311,3.831)}
\gppoint{gp mark 0}{(4.311,3.895)}
\gppoint{gp mark 0}{(4.311,3.535)}
\gppoint{gp mark 0}{(4.311,3.720)}
\gppoint{gp mark 0}{(4.311,4.446)}
\gppoint{gp mark 0}{(4.311,3.699)}
\gppoint{gp mark 0}{(4.311,3.822)}
\gppoint{gp mark 0}{(4.311,4.330)}
\gppoint{gp mark 0}{(4.311,3.479)}
\gppoint{gp mark 0}{(4.311,3.740)}
\gppoint{gp mark 0}{(4.311,3.656)}
\gppoint{gp mark 0}{(4.311,3.730)}
\gppoint{gp mark 0}{(4.311,3.479)}
\gppoint{gp mark 0}{(4.311,4.285)}
\gppoint{gp mark 0}{(4.311,3.880)}
\gppoint{gp mark 0}{(4.311,3.759)}
\gppoint{gp mark 0}{(4.311,3.911)}
\gppoint{gp mark 0}{(4.311,3.535)}
\gppoint{gp mark 0}{(4.311,4.121)}
\gppoint{gp mark 0}{(4.311,4.174)}
\gppoint{gp mark 0}{(4.311,3.831)}
\gppoint{gp mark 0}{(4.311,3.730)}
\gppoint{gp mark 0}{(4.311,3.586)}
\gppoint{gp mark 0}{(4.311,3.535)}
\gppoint{gp mark 0}{(4.311,4.594)}
\gppoint{gp mark 0}{(4.311,4.174)}
\gppoint{gp mark 0}{(4.311,4.174)}
\gppoint{gp mark 0}{(4.311,4.174)}
\gppoint{gp mark 0}{(4.311,3.598)}
\gppoint{gp mark 0}{(4.311,3.678)}
\gppoint{gp mark 0}{(4.311,4.232)}
\gppoint{gp mark 0}{(4.311,3.847)}
\gppoint{gp mark 0}{(4.311,3.796)}
\gppoint{gp mark 0}{(4.311,3.847)}
\gppoint{gp mark 0}{(4.311,4.164)}
\gppoint{gp mark 0}{(4.311,4.164)}
\gppoint{gp mark 0}{(4.311,3.778)}
\gppoint{gp mark 0}{(4.311,3.689)}
\gppoint{gp mark 0}{(4.311,3.699)}
\gppoint{gp mark 0}{(4.311,3.839)}
\gppoint{gp mark 0}{(4.311,3.720)}
\gppoint{gp mark 0}{(4.311,3.995)}
\gppoint{gp mark 0}{(4.311,3.831)}
\gppoint{gp mark 0}{(4.311,4.082)}
\gppoint{gp mark 0}{(4.311,3.449)}
\gppoint{gp mark 0}{(4.311,4.099)}
\gppoint{gp mark 0}{(4.311,3.656)}
\gppoint{gp mark 0}{(4.311,3.995)}
\gppoint{gp mark 0}{(4.311,4.116)}
\gppoint{gp mark 0}{(4.311,3.856)}
\gppoint{gp mark 0}{(4.311,3.831)}
\gppoint{gp mark 0}{(4.311,3.768)}
\gppoint{gp mark 0}{(4.311,3.678)}
\gppoint{gp mark 0}{(4.311,3.678)}
\gppoint{gp mark 0}{(4.311,3.548)}
\gppoint{gp mark 0}{(4.311,4.227)}
\gppoint{gp mark 0}{(4.311,3.796)}
\gppoint{gp mark 0}{(4.311,3.634)}
\gppoint{gp mark 0}{(4.311,3.598)}
\gppoint{gp mark 0}{(4.311,3.479)}
\gppoint{gp mark 0}{(4.311,3.995)}
\gppoint{gp mark 0}{(4.311,3.521)}
\gppoint{gp mark 0}{(4.311,3.493)}
\gppoint{gp mark 0}{(4.311,3.493)}
\gppoint{gp mark 0}{(4.311,4.034)}
\gppoint{gp mark 0}{(4.311,4.034)}
\gppoint{gp mark 0}{(4.311,3.634)}
\gppoint{gp mark 0}{(4.311,4.227)}
\gppoint{gp mark 0}{(4.311,4.008)}
\gppoint{gp mark 0}{(4.311,4.034)}
\gppoint{gp mark 0}{(4.311,3.749)}
\gppoint{gp mark 0}{(4.311,3.667)}
\gppoint{gp mark 0}{(4.311,3.796)}
\gppoint{gp mark 0}{(4.311,3.730)}
\gppoint{gp mark 0}{(4.311,4.082)}
\gppoint{gp mark 0}{(4.311,3.351)}
\gppoint{gp mark 0}{(4.311,4.052)}
\gppoint{gp mark 0}{(4.311,3.730)}
\gppoint{gp mark 0}{(4.311,3.911)}
\gppoint{gp mark 0}{(4.311,3.864)}
\gppoint{gp mark 0}{(4.311,3.839)}
\gppoint{gp mark 0}{(4.311,3.699)}
\gppoint{gp mark 0}{(4.311,4.533)}
\gppoint{gp mark 0}{(4.311,4.184)}
\gppoint{gp mark 0}{(4.311,4.082)}
\gppoint{gp mark 0}{(4.311,4.706)}
\gppoint{gp mark 0}{(4.311,3.295)}
\gppoint{gp mark 0}{(4.311,3.839)}
\gppoint{gp mark 0}{(4.311,3.493)}
\gppoint{gp mark 0}{(4.311,3.535)}
\gppoint{gp mark 0}{(4.311,3.796)}
\gppoint{gp mark 0}{(4.311,3.645)}
\gppoint{gp mark 0}{(4.311,3.333)}
\gppoint{gp mark 0}{(4.311,3.805)}
\gppoint{gp mark 0}{(4.311,4.672)}
\gppoint{gp mark 0}{(4.311,3.645)}
\gppoint{gp mark 0}{(4.311,3.645)}
\gppoint{gp mark 0}{(4.311,3.813)}
\gppoint{gp mark 0}{(4.311,3.402)}
\gppoint{gp mark 0}{(4.311,3.622)}
\gppoint{gp mark 0}{(4.311,3.493)}
\gppoint{gp mark 0}{(4.311,3.740)}
\gppoint{gp mark 0}{(4.311,4.515)}
\gppoint{gp mark 0}{(4.311,4.034)}
\gppoint{gp mark 0}{(4.311,4.530)}
\gppoint{gp mark 0}{(4.311,3.847)}
\gppoint{gp mark 0}{(4.311,3.598)}
\gppoint{gp mark 0}{(4.311,3.678)}
\gppoint{gp mark 0}{(4.311,3.598)}
\gppoint{gp mark 0}{(4.311,3.634)}
\gppoint{gp mark 0}{(4.311,3.720)}
\gppoint{gp mark 0}{(4.311,3.888)}
\gppoint{gp mark 0}{(4.311,4.515)}
\gppoint{gp mark 0}{(4.311,3.678)}
\gppoint{gp mark 0}{(4.311,3.856)}
\gppoint{gp mark 0}{(4.311,3.954)}
\gppoint{gp mark 0}{(4.311,3.720)}
\gppoint{gp mark 0}{(4.311,3.847)}
\gppoint{gp mark 0}{(4.311,4.227)}
\gppoint{gp mark 0}{(4.311,3.295)}
\gppoint{gp mark 0}{(4.311,3.730)}
\gppoint{gp mark 0}{(4.311,4.227)}
\gppoint{gp mark 0}{(4.311,3.699)}
\gppoint{gp mark 0}{(4.311,3.479)}
\gppoint{gp mark 0}{(4.311,3.634)}
\gppoint{gp mark 0}{(4.311,3.645)}
\gppoint{gp mark 0}{(4.311,3.645)}
\gppoint{gp mark 0}{(4.311,3.634)}
\gppoint{gp mark 0}{(4.311,3.479)}
\gppoint{gp mark 0}{(4.311,3.535)}
\gppoint{gp mark 0}{(4.311,3.255)}
\gppoint{gp mark 0}{(4.311,4.158)}
\gppoint{gp mark 0}{(4.311,4.550)}
\gppoint{gp mark 0}{(4.311,3.634)}
\gppoint{gp mark 0}{(4.311,4.021)}
\gppoint{gp mark 0}{(4.311,3.689)}
\gppoint{gp mark 0}{(4.311,3.689)}
\gppoint{gp mark 0}{(4.311,3.255)}
\gppoint{gp mark 0}{(4.311,3.622)}
\gppoint{gp mark 0}{(4.311,3.778)}
\gppoint{gp mark 0}{(4.311,3.449)}
\gppoint{gp mark 0}{(4.311,3.535)}
\gppoint{gp mark 0}{(4.311,4.276)}
\gppoint{gp mark 0}{(4.311,3.710)}
\gppoint{gp mark 0}{(4.311,4.008)}
\gppoint{gp mark 0}{(4.311,4.584)}
\gppoint{gp mark 0}{(4.311,4.338)}
\gppoint{gp mark 0}{(4.311,4.310)}
\gppoint{gp mark 0}{(4.311,4.527)}
\gppoint{gp mark 0}{(4.311,4.509)}
\gppoint{gp mark 0}{(4.311,4.681)}
\gppoint{gp mark 0}{(4.311,3.995)}
\gppoint{gp mark 0}{(4.311,3.645)}
\gppoint{gp mark 0}{(4.311,3.667)}
\gppoint{gp mark 0}{(4.311,3.634)}
\gppoint{gp mark 0}{(4.311,4.208)}
\gppoint{gp mark 0}{(4.311,3.535)}
\gppoint{gp mark 0}{(4.311,3.933)}
\gppoint{gp mark 0}{(4.311,3.610)}
\gppoint{gp mark 0}{(4.311,3.521)}
\gppoint{gp mark 0}{(4.311,3.561)}
\gppoint{gp mark 0}{(4.311,3.895)}
\gppoint{gp mark 0}{(4.311,4.121)}
\gppoint{gp mark 0}{(4.311,3.831)}
\gppoint{gp mark 0}{(4.311,3.351)}
\gppoint{gp mark 0}{(4.311,4.028)}
\gppoint{gp mark 0}{(4.311,4.218)}
\gppoint{gp mark 0}{(4.311,3.449)}
\gppoint{gp mark 0}{(4.311,3.535)}
\gppoint{gp mark 0}{(4.311,3.507)}
\gppoint{gp mark 0}{(4.311,4.318)}
\gppoint{gp mark 0}{(4.311,4.070)}
\gppoint{gp mark 0}{(4.311,3.926)}
\gppoint{gp mark 0}{(4.311,3.740)}
\gppoint{gp mark 0}{(4.350,3.610)}
\gppoint{gp mark 0}{(4.350,4.391)}
\gppoint{gp mark 0}{(4.350,3.730)}
\gppoint{gp mark 0}{(4.350,3.610)}
\gppoint{gp mark 0}{(4.350,4.088)}
\gppoint{gp mark 0}{(4.350,3.464)}
\gppoint{gp mark 0}{(4.350,3.839)}
\gppoint{gp mark 0}{(4.350,3.940)}
\gppoint{gp mark 0}{(4.350,3.333)}
\gppoint{gp mark 0}{(4.350,3.610)}
\gppoint{gp mark 0}{(4.350,4.046)}
\gppoint{gp mark 0}{(4.350,3.982)}
\gppoint{gp mark 0}{(4.350,3.535)}
\gppoint{gp mark 0}{(4.350,3.479)}
\gppoint{gp mark 0}{(4.350,4.132)}
\gppoint{gp mark 0}{(4.350,4.263)}
\gppoint{gp mark 0}{(4.350,3.449)}
\gppoint{gp mark 0}{(4.350,3.535)}
\gppoint{gp mark 0}{(4.350,4.660)}
\gppoint{gp mark 0}{(4.350,3.768)}
\gppoint{gp mark 0}{(4.350,3.822)}
\gppoint{gp mark 0}{(4.350,4.189)}
\gppoint{gp mark 0}{(4.350,3.888)}
\gppoint{gp mark 0}{(4.350,4.326)}
\gppoint{gp mark 0}{(4.350,3.333)}
\gppoint{gp mark 0}{(4.350,3.402)}
\gppoint{gp mark 0}{(4.350,3.610)}
\gppoint{gp mark 0}{(4.350,3.787)}
\gppoint{gp mark 0}{(4.350,3.796)}
\gppoint{gp mark 0}{(4.350,3.768)}
\gppoint{gp mark 0}{(4.350,4.179)}
\gppoint{gp mark 0}{(4.350,3.351)}
\gppoint{gp mark 0}{(4.350,4.082)}
\gppoint{gp mark 0}{(4.350,3.831)}
\gppoint{gp mark 0}{(4.350,3.402)}
\gppoint{gp mark 0}{(4.350,4.548)}
\gppoint{gp mark 0}{(4.350,3.479)}
\gppoint{gp mark 0}{(4.350,3.449)}
\gppoint{gp mark 0}{(4.350,3.918)}
\gppoint{gp mark 0}{(4.350,4.735)}
\gppoint{gp mark 0}{(4.350,3.493)}
\gppoint{gp mark 0}{(4.350,4.768)}
\gppoint{gp mark 0}{(4.350,3.586)}
\gppoint{gp mark 0}{(4.350,3.710)}
\gppoint{gp mark 0}{(4.350,3.368)}
\gppoint{gp mark 0}{(4.350,3.895)}
\gppoint{gp mark 0}{(4.350,3.856)}
\gppoint{gp mark 0}{(4.350,3.796)}
\gppoint{gp mark 0}{(4.350,4.208)}
\gppoint{gp mark 0}{(4.350,3.479)}
\gppoint{gp mark 0}{(4.350,4.208)}
\gppoint{gp mark 0}{(4.350,4.132)}
\gppoint{gp mark 0}{(4.350,3.598)}
\gppoint{gp mark 0}{(4.350,3.586)}
\gppoint{gp mark 0}{(4.350,4.143)}
\gppoint{gp mark 0}{(4.350,3.968)}
\gppoint{gp mark 0}{(4.350,3.610)}
\gppoint{gp mark 0}{(4.350,3.610)}
\gppoint{gp mark 0}{(4.350,3.493)}
\gppoint{gp mark 0}{(4.350,4.064)}
\gppoint{gp mark 0}{(4.350,4.475)}
\gppoint{gp mark 0}{(4.350,3.856)}
\gppoint{gp mark 0}{(4.350,3.710)}
\gppoint{gp mark 0}{(4.350,3.656)}
\gppoint{gp mark 0}{(4.350,3.699)}
\gppoint{gp mark 0}{(4.350,3.449)}
\gppoint{gp mark 0}{(4.350,3.535)}
\gppoint{gp mark 0}{(4.350,3.710)}
\gppoint{gp mark 0}{(4.350,3.678)}
\gppoint{gp mark 0}{(4.350,3.839)}
\gppoint{gp mark 0}{(4.350,3.699)}
\gppoint{gp mark 0}{(4.350,3.634)}
\gppoint{gp mark 0}{(4.350,4.002)}
\gppoint{gp mark 0}{(4.350,3.822)}
\gppoint{gp mark 0}{(4.350,3.535)}
\gppoint{gp mark 0}{(4.350,3.787)}
\gppoint{gp mark 0}{(4.350,4.527)}
\gppoint{gp mark 0}{(4.350,4.908)}
\gppoint{gp mark 0}{(4.350,3.561)}
\gppoint{gp mark 0}{(4.350,3.699)}
\gppoint{gp mark 0}{(4.350,4.121)}
\gppoint{gp mark 0}{(4.350,3.710)}
\gppoint{gp mark 0}{(4.350,3.787)}
\gppoint{gp mark 0}{(4.350,3.822)}
\gppoint{gp mark 0}{(4.350,3.903)}
\gppoint{gp mark 0}{(4.350,4.515)}
\gppoint{gp mark 0}{(4.350,4.208)}
\gppoint{gp mark 0}{(4.350,3.493)}
\gppoint{gp mark 0}{(4.350,4.208)}
\gppoint{gp mark 0}{(4.350,3.699)}
\gppoint{gp mark 0}{(4.350,3.561)}
\gppoint{gp mark 0}{(4.350,3.535)}
\gppoint{gp mark 0}{(4.350,4.208)}
\gppoint{gp mark 0}{(4.350,3.740)}
\gppoint{gp mark 0}{(4.350,4.208)}
\gppoint{gp mark 0}{(4.350,3.699)}
\gppoint{gp mark 0}{(4.350,4.208)}
\gppoint{gp mark 0}{(4.350,3.710)}
\gppoint{gp mark 0}{(4.350,3.839)}
\gppoint{gp mark 0}{(4.350,4.194)}
\gppoint{gp mark 0}{(4.350,4.198)}
\gppoint{gp mark 0}{(4.350,3.598)}
\gppoint{gp mark 0}{(4.350,4.254)}
\gppoint{gp mark 0}{(4.350,4.021)}
\gppoint{gp mark 0}{(4.350,3.351)}
\gppoint{gp mark 0}{(4.350,3.730)}
\gppoint{gp mark 0}{(4.350,3.787)}
\gppoint{gp mark 0}{(4.350,3.493)}
\gppoint{gp mark 0}{(4.350,4.314)}
\gppoint{gp mark 0}{(4.350,4.326)}
\gppoint{gp mark 0}{(4.350,3.856)}
\gppoint{gp mark 0}{(4.350,3.493)}
\gppoint{gp mark 0}{(4.350,3.888)}
\gppoint{gp mark 0}{(4.350,3.598)}
\gppoint{gp mark 0}{(4.350,3.796)}
\gppoint{gp mark 0}{(4.350,3.351)}
\gppoint{gp mark 0}{(4.350,3.634)}
\gppoint{gp mark 0}{(4.350,3.940)}
\gppoint{gp mark 0}{(4.350,4.908)}
\gppoint{gp mark 0}{(4.350,3.940)}
\gppoint{gp mark 0}{(4.350,3.535)}
\gppoint{gp mark 0}{(4.350,3.561)}
\gppoint{gp mark 0}{(4.350,3.787)}
\gppoint{gp mark 0}{(4.350,3.645)}
\gppoint{gp mark 0}{(4.350,3.521)}
\gppoint{gp mark 0}{(4.350,4.194)}
\gppoint{gp mark 0}{(4.350,3.667)}
\gppoint{gp mark 0}{(4.350,3.995)}
\gppoint{gp mark 0}{(4.350,4.715)}
\gppoint{gp mark 0}{(4.350,3.235)}
\gppoint{gp mark 0}{(4.350,3.699)}
\gppoint{gp mark 0}{(4.350,3.449)}
\gppoint{gp mark 0}{(4.350,3.645)}
\gppoint{gp mark 0}{(4.350,3.933)}
\gppoint{gp mark 0}{(4.350,4.314)}
\gppoint{gp mark 0}{(4.350,3.548)}
\gppoint{gp mark 0}{(4.350,3.561)}
\gppoint{gp mark 0}{(4.350,4.076)}
\gppoint{gp mark 0}{(4.350,3.768)}
\gppoint{gp mark 0}{(4.350,3.622)}
\gppoint{gp mark 0}{(4.350,3.749)}
\gppoint{gp mark 0}{(4.350,3.749)}
\gppoint{gp mark 0}{(4.350,4.132)}
\gppoint{gp mark 0}{(4.350,4.132)}
\gppoint{gp mark 0}{(4.350,4.040)}
\gppoint{gp mark 0}{(4.350,4.021)}
\gppoint{gp mark 0}{(4.350,3.561)}
\gppoint{gp mark 0}{(4.350,3.368)}
\gppoint{gp mark 0}{(4.350,3.610)}
\gppoint{gp mark 0}{(4.350,3.720)}
\gppoint{gp mark 0}{(4.350,4.459)}
\gppoint{gp mark 0}{(4.350,3.720)}
\gppoint{gp mark 0}{(4.350,4.076)}
\gppoint{gp mark 0}{(4.350,3.493)}
\gppoint{gp mark 0}{(4.350,3.839)}
\gppoint{gp mark 0}{(4.350,3.918)}
\gppoint{gp mark 0}{(4.350,3.864)}
\gppoint{gp mark 0}{(4.350,3.822)}
\gppoint{gp mark 0}{(4.350,4.110)}
\gppoint{gp mark 0}{(4.350,3.610)}
\gppoint{gp mark 0}{(4.350,4.034)}
\gppoint{gp mark 0}{(4.350,4.326)}
\gppoint{gp mark 0}{(4.350,3.787)}
\gppoint{gp mark 0}{(4.350,3.903)}
\gppoint{gp mark 0}{(4.350,3.598)}
\gppoint{gp mark 0}{(4.350,4.028)}
\gppoint{gp mark 0}{(4.350,4.028)}
\gppoint{gp mark 0}{(4.350,3.903)}
\gppoint{gp mark 0}{(4.350,3.507)}
\gppoint{gp mark 0}{(4.350,3.839)}
\gppoint{gp mark 0}{(4.350,4.082)}
\gppoint{gp mark 0}{(4.350,3.535)}
\gppoint{gp mark 0}{(4.350,3.535)}
\gppoint{gp mark 0}{(4.350,4.099)}
\gppoint{gp mark 0}{(4.350,4.357)}
\gppoint{gp mark 0}{(4.350,3.598)}
\gppoint{gp mark 0}{(4.350,3.933)}
\gppoint{gp mark 0}{(4.350,3.561)}
\gppoint{gp mark 0}{(4.350,3.418)}
\gppoint{gp mark 0}{(4.350,3.961)}
\gppoint{gp mark 0}{(4.350,3.839)}
\gppoint{gp mark 0}{(4.350,3.678)}
\gppoint{gp mark 0}{(4.350,3.787)}
\gppoint{gp mark 0}{(4.350,4.143)}
\gppoint{gp mark 0}{(4.350,3.813)}
\gppoint{gp mark 0}{(4.350,4.638)}
\gppoint{gp mark 0}{(4.350,3.645)}
\gppoint{gp mark 0}{(4.350,4.034)}
\gppoint{gp mark 0}{(4.350,3.839)}
\gppoint{gp mark 0}{(4.350,3.720)}
\gppoint{gp mark 0}{(4.350,3.805)}
\gppoint{gp mark 0}{(4.350,3.479)}
\gppoint{gp mark 0}{(4.350,4.021)}
\gppoint{gp mark 0}{(4.350,3.586)}
\gppoint{gp mark 0}{(4.350,3.598)}
\gppoint{gp mark 0}{(4.350,3.895)}
\gppoint{gp mark 0}{(4.350,4.349)}
\gppoint{gp mark 0}{(4.350,3.831)}
\gppoint{gp mark 0}{(4.350,3.864)}
\gppoint{gp mark 0}{(4.350,3.598)}
\gppoint{gp mark 0}{(4.350,3.598)}
\gppoint{gp mark 0}{(4.350,3.968)}
\gppoint{gp mark 0}{(4.350,3.954)}
\gppoint{gp mark 0}{(4.350,3.895)}
\gppoint{gp mark 0}{(4.350,4.346)}
\gppoint{gp mark 0}{(4.350,3.449)}
\gppoint{gp mark 0}{(4.350,3.598)}
\gppoint{gp mark 0}{(4.350,3.449)}
\gppoint{gp mark 0}{(4.350,4.082)}
\gppoint{gp mark 0}{(4.350,3.872)}
\gppoint{gp mark 0}{(4.350,3.573)}
\gppoint{gp mark 0}{(4.350,4.293)}
\gppoint{gp mark 0}{(4.350,4.148)}
\gppoint{gp mark 0}{(4.350,3.235)}
\gppoint{gp mark 0}{(4.350,4.475)}
\gppoint{gp mark 0}{(4.350,3.645)}
\gppoint{gp mark 0}{(4.350,3.982)}
\gppoint{gp mark 0}{(4.350,4.218)}
\gppoint{gp mark 0}{(4.350,3.610)}
\gppoint{gp mark 0}{(4.350,3.847)}
\gppoint{gp mark 0}{(4.350,3.822)}
\gppoint{gp mark 0}{(4.350,3.947)}
\gppoint{gp mark 0}{(4.350,3.822)}
\gppoint{gp mark 0}{(4.350,3.610)}
\gppoint{gp mark 0}{(4.350,3.622)}
\gppoint{gp mark 0}{(4.350,4.263)}
\gppoint{gp mark 0}{(4.350,3.493)}
\gppoint{gp mark 0}{(4.350,4.169)}
\gppoint{gp mark 0}{(4.350,4.046)}
\gppoint{gp mark 0}{(4.350,3.911)}
\gppoint{gp mark 0}{(4.350,3.768)}
\gppoint{gp mark 0}{(4.350,4.653)}
\gppoint{gp mark 0}{(4.350,3.479)}
\gppoint{gp mark 0}{(4.350,4.137)}
\gppoint{gp mark 0}{(4.350,3.449)}
\gppoint{gp mark 0}{(4.350,3.235)}
\gppoint{gp mark 0}{(4.350,4.227)}
\gppoint{gp mark 0}{(4.350,4.203)}
\gppoint{gp mark 0}{(4.350,4.280)}
\gppoint{gp mark 0}{(4.350,3.710)}
\gppoint{gp mark 0}{(4.350,3.710)}
\gppoint{gp mark 0}{(4.350,4.169)}
\gppoint{gp mark 0}{(4.350,3.926)}
\gppoint{gp mark 0}{(4.350,4.556)}
\gppoint{gp mark 0}{(4.350,4.137)}
\gppoint{gp mark 0}{(4.350,3.839)}
\gppoint{gp mark 0}{(4.350,3.699)}
\gppoint{gp mark 0}{(4.350,3.759)}
\gppoint{gp mark 0}{(4.350,3.449)}
\gppoint{gp mark 0}{(4.350,3.402)}
\gppoint{gp mark 0}{(4.350,3.749)}
\gppoint{gp mark 0}{(4.350,3.740)}
\gppoint{gp mark 0}{(4.350,4.021)}
\gppoint{gp mark 0}{(4.350,3.449)}
\gppoint{gp mark 0}{(4.350,3.839)}
\gppoint{gp mark 0}{(4.350,3.645)}
\gppoint{gp mark 0}{(4.350,4.276)}
\gppoint{gp mark 0}{(4.350,3.645)}
\gppoint{gp mark 0}{(4.387,3.903)}
\gppoint{gp mark 0}{(4.387,4.387)}
\gppoint{gp mark 0}{(4.387,3.926)}
\gppoint{gp mark 0}{(4.387,3.535)}
\gppoint{gp mark 0}{(4.387,3.880)}
\gppoint{gp mark 0}{(4.387,4.153)}
\gppoint{gp mark 0}{(4.387,4.174)}
\gppoint{gp mark 0}{(4.387,4.245)}
\gppoint{gp mark 0}{(4.387,4.093)}
\gppoint{gp mark 0}{(4.387,3.634)}
\gppoint{gp mark 0}{(4.387,4.218)}
\gppoint{gp mark 0}{(4.387,3.968)}
\gppoint{gp mark 0}{(4.387,3.926)}
\gppoint{gp mark 0}{(4.387,4.116)}
\gppoint{gp mark 0}{(4.387,3.749)}
\gppoint{gp mark 0}{(4.387,4.203)}
\gppoint{gp mark 0}{(4.387,3.933)}
\gppoint{gp mark 0}{(4.387,3.573)}
\gppoint{gp mark 0}{(4.387,3.610)}
\gppoint{gp mark 0}{(4.387,3.926)}
\gppoint{gp mark 0}{(4.387,3.926)}
\gppoint{gp mark 0}{(4.387,3.864)}
\gppoint{gp mark 0}{(4.387,3.975)}
\gppoint{gp mark 0}{(4.387,3.610)}
\gppoint{gp mark 0}{(4.387,3.586)}
\gppoint{gp mark 0}{(4.387,3.856)}
\gppoint{gp mark 0}{(4.387,3.839)}
\gppoint{gp mark 0}{(4.387,3.598)}
\gppoint{gp mark 0}{(4.387,3.710)}
\gppoint{gp mark 0}{(4.387,3.610)}
\gppoint{gp mark 0}{(4.387,4.052)}
\gppoint{gp mark 0}{(4.387,3.822)}
\gppoint{gp mark 0}{(4.387,3.610)}
\gppoint{gp mark 0}{(4.387,4.937)}
\gppoint{gp mark 0}{(4.387,3.561)}
\gppoint{gp mark 0}{(4.387,4.488)}
\gppoint{gp mark 0}{(4.387,3.805)}
\gppoint{gp mark 0}{(4.387,3.796)}
\gppoint{gp mark 0}{(4.387,3.975)}
\gppoint{gp mark 0}{(4.387,3.610)}
\gppoint{gp mark 0}{(4.387,3.449)}
\gppoint{gp mark 0}{(4.387,4.052)}
\gppoint{gp mark 0}{(4.387,3.730)}
\gppoint{gp mark 0}{(4.387,3.982)}
\gppoint{gp mark 0}{(4.387,4.099)}
\gppoint{gp mark 0}{(4.387,3.449)}
\gppoint{gp mark 0}{(4.387,4.398)}
\gppoint{gp mark 0}{(4.387,3.975)}
\gppoint{gp mark 0}{(4.387,3.168)}
\gppoint{gp mark 0}{(4.387,3.947)}
\gppoint{gp mark 0}{(4.387,3.975)}
\gppoint{gp mark 0}{(4.387,3.864)}
\gppoint{gp mark 0}{(4.387,4.536)}
\gppoint{gp mark 0}{(4.387,3.926)}
\gppoint{gp mark 0}{(4.387,3.768)}
\gppoint{gp mark 0}{(4.387,3.805)}
\gppoint{gp mark 0}{(4.387,3.493)}
\gppoint{gp mark 0}{(4.387,4.058)}
\gppoint{gp mark 0}{(4.387,4.318)}
\gppoint{gp mark 0}{(4.387,3.759)}
\gppoint{gp mark 0}{(4.387,4.008)}
\gppoint{gp mark 0}{(4.387,3.622)}
\gppoint{gp mark 0}{(4.387,3.796)}
\gppoint{gp mark 0}{(4.387,3.813)}
\gppoint{gp mark 0}{(4.387,3.888)}
\gppoint{gp mark 0}{(4.387,3.926)}
\gppoint{gp mark 0}{(4.387,3.573)}
\gppoint{gp mark 0}{(4.387,4.322)}
\gppoint{gp mark 0}{(4.387,4.326)}
\gppoint{gp mark 0}{(4.387,3.805)}
\gppoint{gp mark 0}{(4.387,4.717)}
\gppoint{gp mark 0}{(4.387,4.127)}
\gppoint{gp mark 0}{(4.387,3.656)}
\gppoint{gp mark 0}{(4.387,4.310)}
\gppoint{gp mark 0}{(4.387,3.479)}
\gppoint{gp mark 0}{(4.387,3.926)}
\gppoint{gp mark 0}{(4.387,4.245)}
\gppoint{gp mark 0}{(4.387,4.245)}
\gppoint{gp mark 0}{(4.387,3.926)}
\gppoint{gp mark 0}{(4.387,3.989)}
\gppoint{gp mark 0}{(4.387,4.158)}
\gppoint{gp mark 0}{(4.387,3.710)}
\gppoint{gp mark 0}{(4.387,3.813)}
\gppoint{gp mark 0}{(4.387,4.169)}
\gppoint{gp mark 0}{(4.387,3.864)}
\gppoint{gp mark 0}{(4.387,3.730)}
\gppoint{gp mark 0}{(4.387,3.926)}
\gppoint{gp mark 0}{(4.387,4.227)}
\gppoint{gp mark 0}{(4.387,4.227)}
\gppoint{gp mark 0}{(4.387,4.429)}
\gppoint{gp mark 0}{(4.387,4.232)}
\gppoint{gp mark 0}{(4.387,3.895)}
\gppoint{gp mark 0}{(4.387,3.872)}
\gppoint{gp mark 0}{(4.387,4.280)}
\gppoint{gp mark 0}{(4.387,4.110)}
\gppoint{gp mark 0}{(4.387,3.710)}
\gppoint{gp mark 0}{(4.387,3.768)}
\gppoint{gp mark 0}{(4.387,4.169)}
\gppoint{gp mark 0}{(4.387,3.768)}
\gppoint{gp mark 0}{(4.387,4.148)}
\gppoint{gp mark 0}{(4.387,3.678)}
\gppoint{gp mark 0}{(4.387,4.302)}
\gppoint{gp mark 0}{(4.387,3.926)}
\gppoint{gp mark 0}{(4.387,3.888)}
\gppoint{gp mark 0}{(4.387,3.982)}
\gppoint{gp mark 0}{(4.387,3.888)}
\gppoint{gp mark 0}{(4.387,3.610)}
\gppoint{gp mark 0}{(4.387,3.610)}
\gppoint{gp mark 0}{(4.387,3.645)}
\gppoint{gp mark 0}{(4.387,3.911)}
\gppoint{gp mark 0}{(4.387,3.895)}
\gppoint{gp mark 0}{(4.387,3.975)}
\gppoint{gp mark 0}{(4.387,3.888)}
\gppoint{gp mark 0}{(4.387,3.995)}
\gppoint{gp mark 0}{(4.387,3.918)}
\gppoint{gp mark 0}{(4.387,3.610)}
\gppoint{gp mark 0}{(4.387,3.933)}
\gppoint{gp mark 0}{(4.387,3.645)}
\gppoint{gp mark 0}{(4.387,3.710)}
\gppoint{gp mark 0}{(4.387,3.634)}
\gppoint{gp mark 0}{(4.387,3.839)}
\gppoint{gp mark 0}{(4.387,3.926)}
\gppoint{gp mark 0}{(4.387,3.598)}
\gppoint{gp mark 0}{(4.387,3.699)}
\gppoint{gp mark 0}{(4.387,3.864)}
\gppoint{gp mark 0}{(4.387,4.099)}
\gppoint{gp mark 0}{(4.387,4.365)}
\gppoint{gp mark 0}{(4.387,4.469)}
\gppoint{gp mark 0}{(4.387,3.667)}
\gppoint{gp mark 0}{(4.387,3.548)}
\gppoint{gp mark 0}{(4.387,3.730)}
\gppoint{gp mark 0}{(4.387,4.380)}
\gppoint{gp mark 0}{(4.387,4.550)}
\gppoint{gp mark 0}{(4.387,3.895)}
\gppoint{gp mark 0}{(4.387,4.550)}
\gppoint{gp mark 0}{(4.387,3.968)}
\gppoint{gp mark 0}{(4.387,4.394)}
\gppoint{gp mark 0}{(4.387,3.699)}
\gppoint{gp mark 0}{(4.387,4.236)}
\gppoint{gp mark 0}{(4.387,3.872)}
\gppoint{gp mark 0}{(4.387,3.787)}
\gppoint{gp mark 0}{(4.387,3.667)}
\gppoint{gp mark 0}{(4.387,3.656)}
\gppoint{gp mark 0}{(4.387,4.040)}
\gppoint{gp mark 0}{(4.387,3.778)}
\gppoint{gp mark 0}{(4.387,3.926)}
\gppoint{gp mark 0}{(4.387,4.040)}
\gppoint{gp mark 0}{(4.387,3.667)}
\gppoint{gp mark 0}{(4.387,4.143)}
\gppoint{gp mark 0}{(4.387,3.449)}
\gppoint{gp mark 0}{(4.387,4.034)}
\gppoint{gp mark 0}{(4.387,3.535)}
\gppoint{gp mark 0}{(4.387,3.805)}
\gppoint{gp mark 0}{(4.387,3.888)}
\gppoint{gp mark 0}{(4.387,3.720)}
\gppoint{gp mark 0}{(4.387,3.768)}
\gppoint{gp mark 0}{(4.387,4.058)}
\gppoint{gp mark 0}{(4.387,3.768)}
\gppoint{gp mark 0}{(4.387,3.667)}
\gppoint{gp mark 0}{(4.387,4.203)}
\gppoint{gp mark 0}{(4.387,4.203)}
\gppoint{gp mark 0}{(4.387,4.203)}
\gppoint{gp mark 0}{(4.387,4.169)}
\gppoint{gp mark 0}{(4.387,3.768)}
\gppoint{gp mark 0}{(4.387,4.021)}
\gppoint{gp mark 0}{(4.387,3.989)}
\gppoint{gp mark 0}{(4.387,3.947)}
\gppoint{gp mark 0}{(4.387,4.052)}
\gppoint{gp mark 0}{(4.387,4.594)}
\gppoint{gp mark 0}{(4.387,4.040)}
\gppoint{gp mark 0}{(4.387,4.040)}
\gppoint{gp mark 0}{(4.387,4.259)}
\gppoint{gp mark 0}{(4.387,3.667)}
\gppoint{gp mark 0}{(4.387,3.864)}
\gppoint{gp mark 0}{(4.387,3.740)}
\gppoint{gp mark 0}{(4.387,4.116)}
\gppoint{gp mark 0}{(4.387,3.805)}
\gppoint{gp mark 0}{(4.387,4.169)}
\gppoint{gp mark 0}{(4.387,3.831)}
\gppoint{gp mark 0}{(4.387,4.597)}
\gppoint{gp mark 0}{(4.387,3.385)}
\gppoint{gp mark 0}{(4.387,4.227)}
\gppoint{gp mark 0}{(4.387,3.586)}
\gppoint{gp mark 0}{(4.387,4.169)}
\gppoint{gp mark 0}{(4.387,3.610)}
\gppoint{gp mark 0}{(4.387,3.586)}
\gppoint{gp mark 0}{(4.387,4.227)}
\gppoint{gp mark 0}{(4.387,3.548)}
\gppoint{gp mark 0}{(4.387,3.548)}
\gppoint{gp mark 0}{(4.387,3.759)}
\gppoint{gp mark 0}{(4.387,4.028)}
\gppoint{gp mark 0}{(4.387,3.888)}
\gppoint{gp mark 0}{(4.387,3.598)}
\gppoint{gp mark 0}{(4.387,3.586)}
\gppoint{gp mark 0}{(4.387,3.535)}
\gppoint{gp mark 0}{(4.387,3.740)}
\gppoint{gp mark 0}{(4.387,3.449)}
\gppoint{gp mark 0}{(4.387,3.710)}
\gppoint{gp mark 0}{(4.387,3.805)}
\gppoint{gp mark 0}{(4.387,4.169)}
\gppoint{gp mark 0}{(4.387,3.749)}
\gppoint{gp mark 0}{(4.387,3.730)}
\gppoint{gp mark 0}{(4.387,3.968)}
\gppoint{gp mark 0}{(4.387,3.678)}
\gppoint{gp mark 0}{(4.387,3.561)}
\gppoint{gp mark 0}{(4.387,3.699)}
\gppoint{gp mark 0}{(4.387,3.720)}
\gppoint{gp mark 0}{(4.387,3.759)}
\gppoint{gp mark 0}{(4.387,3.385)}
\gppoint{gp mark 0}{(4.387,3.730)}
\gppoint{gp mark 0}{(4.387,3.982)}
\gppoint{gp mark 0}{(4.387,3.521)}
\gppoint{gp mark 0}{(4.387,3.768)}
\gppoint{gp mark 0}{(4.387,4.040)}
\gppoint{gp mark 0}{(4.387,4.040)}
\gppoint{gp mark 0}{(4.387,4.116)}
\gppoint{gp mark 0}{(4.387,4.105)}
\gppoint{gp mark 0}{(4.387,3.449)}
\gppoint{gp mark 0}{(4.387,3.710)}
\gppoint{gp mark 0}{(4.387,3.796)}
\gppoint{gp mark 0}{(4.387,3.699)}
\gppoint{gp mark 0}{(4.387,4.553)}
\gppoint{gp mark 0}{(4.387,3.787)}
\gppoint{gp mark 0}{(4.387,3.548)}
\gppoint{gp mark 0}{(4.387,3.975)}
\gppoint{gp mark 0}{(4.387,3.911)}
\gppoint{gp mark 0}{(4.387,4.148)}
\gppoint{gp mark 0}{(4.387,3.759)}
\gppoint{gp mark 0}{(4.387,3.768)}
\gppoint{gp mark 0}{(4.387,3.864)}
\gppoint{gp mark 0}{(4.387,4.383)}
\gppoint{gp mark 0}{(4.387,3.880)}
\gppoint{gp mark 0}{(4.387,3.831)}
\gppoint{gp mark 0}{(4.387,4.398)}
\gppoint{gp mark 0}{(4.422,3.385)}
\gppoint{gp mark 0}{(4.422,3.385)}
\gppoint{gp mark 0}{(4.422,3.831)}
\gppoint{gp mark 0}{(4.422,3.385)}
\gppoint{gp mark 0}{(4.422,3.385)}
\gppoint{gp mark 0}{(4.422,3.622)}
\gppoint{gp mark 0}{(4.422,3.656)}
\gppoint{gp mark 0}{(4.422,3.622)}
\gppoint{gp mark 0}{(4.422,3.434)}
\gppoint{gp mark 0}{(4.422,3.385)}
\gppoint{gp mark 0}{(4.422,3.385)}
\gppoint{gp mark 0}{(4.422,3.385)}
\gppoint{gp mark 0}{(4.422,3.507)}
\gppoint{gp mark 0}{(4.422,3.385)}
\gppoint{gp mark 0}{(4.422,3.434)}
\gppoint{gp mark 0}{(4.422,3.385)}
\gppoint{gp mark 0}{(4.422,4.021)}
\gppoint{gp mark 0}{(4.422,3.768)}
\gppoint{gp mark 0}{(4.422,3.548)}
\gppoint{gp mark 0}{(4.422,3.622)}
\gppoint{gp mark 0}{(4.422,3.402)}
\gppoint{gp mark 0}{(4.422,4.213)}
\gppoint{gp mark 0}{(4.422,3.710)}
\gppoint{gp mark 0}{(4.422,3.561)}
\gppoint{gp mark 0}{(4.422,4.314)}
\gppoint{gp mark 0}{(4.422,3.926)}
\gppoint{gp mark 0}{(4.422,4.070)}
\gppoint{gp mark 0}{(4.422,3.933)}
\gppoint{gp mark 0}{(4.422,3.933)}
\gppoint{gp mark 0}{(4.422,3.634)}
\gppoint{gp mark 0}{(4.422,3.975)}
\gppoint{gp mark 0}{(4.422,4.349)}
\gppoint{gp mark 0}{(4.422,4.426)}
\gppoint{gp mark 0}{(4.422,4.426)}
\gppoint{gp mark 0}{(4.422,3.586)}
\gppoint{gp mark 0}{(4.422,4.132)}
\gppoint{gp mark 0}{(4.422,4.021)}
\gppoint{gp mark 0}{(4.422,4.426)}
\gppoint{gp mark 0}{(4.422,3.822)}
\gppoint{gp mark 0}{(4.422,3.954)}
\gppoint{gp mark 0}{(4.422,3.573)}
\gppoint{gp mark 0}{(4.422,4.236)}
\gppoint{gp mark 0}{(4.422,3.831)}
\gppoint{gp mark 0}{(4.422,3.926)}
\gppoint{gp mark 0}{(4.422,3.720)}
\gppoint{gp mark 0}{(4.422,3.895)}
\gppoint{gp mark 0}{(4.422,3.989)}
\gppoint{gp mark 0}{(4.422,3.351)}
\gppoint{gp mark 0}{(4.422,3.872)}
\gppoint{gp mark 0}{(4.422,3.385)}
\gppoint{gp mark 0}{(4.422,3.805)}
\gppoint{gp mark 0}{(4.422,4.527)}
\gppoint{gp mark 0}{(4.422,3.995)}
\gppoint{gp mark 0}{(4.422,4.314)}
\gppoint{gp mark 0}{(4.422,4.302)}
\gppoint{gp mark 0}{(4.422,3.699)}
\gppoint{gp mark 0}{(4.422,3.872)}
\gppoint{gp mark 0}{(4.422,3.689)}
\gppoint{gp mark 0}{(4.422,3.656)}
\gppoint{gp mark 0}{(4.422,3.749)}
\gppoint{gp mark 0}{(4.422,3.856)}
\gppoint{gp mark 0}{(4.422,3.493)}
\gppoint{gp mark 0}{(4.422,3.645)}
\gppoint{gp mark 0}{(4.422,3.699)}
\gppoint{gp mark 0}{(4.422,3.493)}
\gppoint{gp mark 0}{(4.422,3.903)}
\gppoint{gp mark 0}{(4.422,3.689)}
\gppoint{gp mark 0}{(4.422,3.864)}
\gppoint{gp mark 0}{(4.422,3.678)}
\gppoint{gp mark 0}{(4.422,3.895)}
\gppoint{gp mark 0}{(4.422,3.573)}
\gppoint{gp mark 0}{(4.422,3.880)}
\gppoint{gp mark 0}{(4.422,3.699)}
\gppoint{gp mark 0}{(4.422,3.720)}
\gppoint{gp mark 0}{(4.422,4.082)}
\gppoint{gp mark 0}{(4.422,3.449)}
\gppoint{gp mark 0}{(4.422,3.699)}
\gppoint{gp mark 0}{(4.422,3.918)}
\gppoint{gp mark 0}{(4.422,4.127)}
\gppoint{gp mark 0}{(4.422,3.699)}
\gppoint{gp mark 0}{(4.422,3.645)}
\gppoint{gp mark 0}{(4.422,3.507)}
\gppoint{gp mark 0}{(4.422,4.222)}
\gppoint{gp mark 0}{(4.422,3.598)}
\gppoint{gp mark 0}{(4.422,3.759)}
\gppoint{gp mark 0}{(4.422,4.250)}
\gppoint{gp mark 0}{(4.422,3.699)}
\gppoint{gp mark 0}{(4.422,3.667)}
\gppoint{gp mark 0}{(4.422,3.622)}
\gppoint{gp mark 0}{(4.422,3.689)}
\gppoint{gp mark 0}{(4.422,3.699)}
\gppoint{gp mark 0}{(4.422,3.699)}
\gppoint{gp mark 0}{(4.422,3.645)}
\gppoint{gp mark 0}{(4.422,4.093)}
\gppoint{gp mark 0}{(4.422,4.280)}
\gppoint{gp mark 0}{(4.422,3.699)}
\gppoint{gp mark 0}{(4.422,3.586)}
\gppoint{gp mark 0}{(4.422,3.699)}
\gppoint{gp mark 0}{(4.422,3.749)}
\gppoint{gp mark 0}{(4.422,4.213)}
\gppoint{gp mark 0}{(4.422,3.645)}
\gppoint{gp mark 0}{(4.422,3.699)}
\gppoint{gp mark 0}{(4.422,3.961)}
\gppoint{gp mark 0}{(4.422,3.385)}
\gppoint{gp mark 0}{(4.422,3.385)}
\gppoint{gp mark 0}{(4.422,3.954)}
\gppoint{gp mark 0}{(4.422,3.622)}
\gppoint{gp mark 0}{(4.422,4.203)}
\gppoint{gp mark 0}{(4.422,3.645)}
\gppoint{gp mark 0}{(4.422,4.436)}
\gppoint{gp mark 0}{(4.422,4.456)}
\gppoint{gp mark 0}{(4.422,3.449)}
\gppoint{gp mark 0}{(4.422,3.730)}
\gppoint{gp mark 0}{(4.422,4.132)}
\gppoint{gp mark 0}{(4.422,3.872)}
\gppoint{gp mark 0}{(4.422,4.272)}
\gppoint{gp mark 0}{(4.422,3.847)}
\gppoint{gp mark 0}{(4.422,4.272)}
\gppoint{gp mark 0}{(4.422,3.864)}
\gppoint{gp mark 0}{(4.422,3.864)}
\gppoint{gp mark 0}{(4.422,3.730)}
\gppoint{gp mark 0}{(4.422,3.598)}
\gppoint{gp mark 0}{(4.422,4.497)}
\gppoint{gp mark 0}{(4.422,3.710)}
\gppoint{gp mark 0}{(4.422,3.598)}
\gppoint{gp mark 0}{(4.422,4.121)}
\gppoint{gp mark 0}{(4.422,4.263)}
\gppoint{gp mark 0}{(4.422,3.645)}
\gppoint{gp mark 0}{(4.422,3.689)}
\gppoint{gp mark 0}{(4.422,3.872)}
\gppoint{gp mark 0}{(4.422,3.710)}
\gppoint{gp mark 0}{(4.422,3.699)}
\gppoint{gp mark 0}{(4.422,4.110)}
\gppoint{gp mark 0}{(4.422,3.351)}
\gppoint{gp mark 0}{(4.422,3.645)}
\gppoint{gp mark 0}{(4.422,3.926)}
\gppoint{gp mark 0}{(4.422,3.982)}
\gppoint{gp mark 0}{(4.422,3.831)}
\gppoint{gp mark 0}{(4.422,3.831)}
\gppoint{gp mark 0}{(4.422,4.002)}
\gppoint{gp mark 0}{(4.422,3.449)}
\gppoint{gp mark 0}{(4.422,3.911)}
\gppoint{gp mark 0}{(4.422,4.028)}
\gppoint{gp mark 0}{(4.422,4.676)}
\gppoint{gp mark 0}{(4.422,3.954)}
\gppoint{gp mark 0}{(4.422,3.947)}
\gppoint{gp mark 0}{(4.422,3.385)}
\gppoint{gp mark 0}{(4.422,3.573)}
\gppoint{gp mark 0}{(4.422,3.561)}
\gppoint{gp mark 0}{(4.422,3.598)}
\gppoint{gp mark 0}{(4.422,4.263)}
\gppoint{gp mark 0}{(4.422,3.872)}
\gppoint{gp mark 0}{(4.422,4.556)}
\gppoint{gp mark 0}{(4.422,4.368)}
\gppoint{gp mark 0}{(4.422,3.434)}
\gppoint{gp mark 0}{(4.422,3.895)}
\gppoint{gp mark 0}{(4.422,4.093)}
\gppoint{gp mark 0}{(4.422,4.040)}
\gppoint{gp mark 0}{(4.422,3.548)}
\gppoint{gp mark 0}{(4.422,3.610)}
\gppoint{gp mark 0}{(4.422,3.730)}
\gppoint{gp mark 0}{(4.422,3.479)}
\gppoint{gp mark 0}{(4.422,3.822)}
\gppoint{gp mark 0}{(4.422,3.940)}
\gppoint{gp mark 0}{(4.422,3.720)}
\gppoint{gp mark 0}{(4.422,4.174)}
\gppoint{gp mark 0}{(4.422,3.720)}
\gppoint{gp mark 0}{(4.422,4.070)}
\gppoint{gp mark 0}{(4.422,3.911)}
\gppoint{gp mark 0}{(4.422,4.158)}
\gppoint{gp mark 0}{(4.422,4.158)}
\gppoint{gp mark 0}{(4.422,4.002)}
\gppoint{gp mark 0}{(4.422,3.645)}
\gppoint{gp mark 0}{(4.422,3.493)}
\gppoint{gp mark 0}{(4.422,3.656)}
\gppoint{gp mark 0}{(4.422,3.561)}
\gppoint{gp mark 0}{(4.422,3.634)}
\gppoint{gp mark 0}{(4.422,4.158)}
\gppoint{gp mark 0}{(4.422,4.686)}
\gppoint{gp mark 0}{(4.422,3.645)}
\gppoint{gp mark 0}{(4.422,3.645)}
\gppoint{gp mark 0}{(4.422,3.864)}
\gppoint{gp mark 0}{(4.422,3.667)}
\gppoint{gp mark 0}{(4.422,4.164)}
\gppoint{gp mark 0}{(4.422,3.982)}
\gppoint{gp mark 0}{(4.422,4.076)}
\gppoint{gp mark 0}{(4.422,4.357)}
\gppoint{gp mark 0}{(4.422,3.507)}
\gppoint{gp mark 0}{(4.422,4.008)}
\gppoint{gp mark 0}{(4.422,3.975)}
\gppoint{gp mark 0}{(4.422,3.805)}
\gppoint{gp mark 0}{(4.422,3.573)}
\gppoint{gp mark 0}{(4.422,3.351)}
\gppoint{gp mark 0}{(4.422,3.730)}
\gppoint{gp mark 0}{(4.422,3.778)}
\gppoint{gp mark 0}{(4.422,3.989)}
\gppoint{gp mark 0}{(4.422,4.164)}
\gppoint{gp mark 0}{(4.422,4.512)}
\gppoint{gp mark 0}{(4.422,3.402)}
\gppoint{gp mark 0}{(4.422,3.805)}
\gppoint{gp mark 0}{(4.422,3.872)}
\gppoint{gp mark 0}{(4.422,4.581)}
\gppoint{gp mark 0}{(4.422,3.895)}
\gppoint{gp mark 0}{(4.422,4.148)}
\gppoint{gp mark 0}{(4.422,3.778)}
\gppoint{gp mark 0}{(4.422,4.357)}
\gppoint{gp mark 0}{(4.422,3.947)}
\gppoint{gp mark 0}{(4.422,3.598)}
\gppoint{gp mark 0}{(4.422,3.940)}
\gppoint{gp mark 0}{(4.422,3.689)}
\gppoint{gp mark 0}{(4.422,3.778)}
\gppoint{gp mark 0}{(4.422,3.507)}
\gppoint{gp mark 0}{(4.422,4.730)}
\gppoint{gp mark 0}{(4.422,4.815)}
\gppoint{gp mark 0}{(4.422,4.245)}
\gppoint{gp mark 0}{(4.422,3.645)}
\gppoint{gp mark 0}{(4.422,3.689)}
\gppoint{gp mark 0}{(4.422,3.710)}
\gppoint{gp mark 0}{(4.422,3.507)}
\gppoint{gp mark 0}{(4.422,3.864)}
\gppoint{gp mark 0}{(4.422,3.710)}
\gppoint{gp mark 0}{(4.422,4.302)}
\gppoint{gp mark 0}{(4.422,3.903)}
\gppoint{gp mark 0}{(4.422,3.535)}
\gppoint{gp mark 0}{(4.422,3.903)}
\gppoint{gp mark 0}{(4.422,3.995)}
\gppoint{gp mark 0}{(4.422,3.598)}
\gppoint{gp mark 0}{(4.422,3.940)}
\gppoint{gp mark 0}{(4.422,3.813)}
\gppoint{gp mark 0}{(4.422,3.656)}
\gppoint{gp mark 0}{(4.422,3.507)}
\gppoint{gp mark 0}{(4.422,4.297)}
\gppoint{gp mark 0}{(4.422,4.232)}
\gppoint{gp mark 0}{(4.422,3.864)}
\gppoint{gp mark 0}{(4.422,3.449)}
\gppoint{gp mark 0}{(4.457,3.888)}
\gppoint{gp mark 0}{(4.457,3.535)}
\gppoint{gp mark 0}{(4.457,3.720)}
\gppoint{gp mark 0}{(4.457,3.895)}
\gppoint{gp mark 0}{(4.457,3.895)}
\gppoint{gp mark 0}{(4.457,3.535)}
\gppoint{gp mark 0}{(4.457,3.667)}
\gppoint{gp mark 0}{(4.457,3.888)}
\gppoint{gp mark 0}{(4.457,3.918)}
\gppoint{gp mark 0}{(4.457,4.391)}
\gppoint{gp mark 0}{(4.457,3.749)}
\gppoint{gp mark 0}{(4.457,4.267)}
\gppoint{gp mark 0}{(4.457,3.856)}
\gppoint{gp mark 0}{(4.457,3.895)}
\gppoint{gp mark 0}{(4.457,3.796)}
\gppoint{gp mark 0}{(4.457,3.710)}
\gppoint{gp mark 0}{(4.457,4.158)}
\gppoint{gp mark 0}{(4.457,3.656)}
\gppoint{gp mark 0}{(4.457,3.903)}
\gppoint{gp mark 0}{(4.457,3.493)}
\gppoint{gp mark 0}{(4.457,3.710)}
\gppoint{gp mark 0}{(4.457,3.749)}
\gppoint{gp mark 0}{(4.457,4.398)}
\gppoint{gp mark 0}{(4.457,3.667)}
\gppoint{gp mark 0}{(4.457,3.895)}
\gppoint{gp mark 0}{(4.457,4.732)}
\gppoint{gp mark 0}{(4.457,3.535)}
\gppoint{gp mark 0}{(4.457,4.158)}
\gppoint{gp mark 0}{(4.457,3.926)}
\gppoint{gp mark 0}{(4.457,3.926)}
\gppoint{gp mark 0}{(4.457,3.667)}
\gppoint{gp mark 0}{(4.457,4.132)}
\gppoint{gp mark 0}{(4.457,3.678)}
\gppoint{gp mark 0}{(4.457,3.449)}
\gppoint{gp mark 0}{(4.457,3.645)}
\gppoint{gp mark 0}{(4.457,3.449)}
\gppoint{gp mark 0}{(4.457,3.622)}
\gppoint{gp mark 0}{(4.457,3.975)}
\gppoint{gp mark 0}{(4.457,4.401)}
\gppoint{gp mark 0}{(4.457,4.760)}
\gppoint{gp mark 0}{(4.457,4.250)}
\gppoint{gp mark 0}{(4.457,3.839)}
\gppoint{gp mark 0}{(4.457,4.203)}
\gppoint{gp mark 0}{(4.457,3.888)}
\gppoint{gp mark 0}{(4.457,4.322)}
\gppoint{gp mark 0}{(4.457,3.573)}
\gppoint{gp mark 0}{(4.457,3.740)}
\gppoint{gp mark 0}{(4.457,3.740)}
\gppoint{gp mark 0}{(4.457,4.143)}
\gppoint{gp mark 0}{(4.457,4.143)}
\gppoint{gp mark 0}{(4.457,4.046)}
\gppoint{gp mark 0}{(4.457,4.394)}
\gppoint{gp mark 0}{(4.457,3.888)}
\gppoint{gp mark 0}{(4.457,3.667)}
\gppoint{gp mark 0}{(4.457,4.436)}
\gppoint{gp mark 0}{(4.457,3.831)}
\gppoint{gp mark 0}{(4.457,3.434)}
\gppoint{gp mark 0}{(4.457,4.015)}
\gppoint{gp mark 0}{(4.457,3.645)}
\gppoint{gp mark 0}{(4.457,3.493)}
\gppoint{gp mark 0}{(4.457,3.768)}
\gppoint{gp mark 0}{(4.457,3.656)}
\gppoint{gp mark 0}{(4.457,4.015)}
\gppoint{gp mark 0}{(4.457,4.453)}
\gppoint{gp mark 0}{(4.457,4.076)}
\gppoint{gp mark 0}{(4.457,4.076)}
\gppoint{gp mark 0}{(4.457,3.634)}
\gppoint{gp mark 0}{(4.457,4.082)}
\gppoint{gp mark 0}{(4.457,4.276)}
\gppoint{gp mark 0}{(4.457,3.634)}
\gppoint{gp mark 0}{(4.457,3.634)}
\gppoint{gp mark 0}{(4.457,3.720)}
\gppoint{gp mark 0}{(4.457,4.127)}
\gppoint{gp mark 0}{(4.457,4.028)}
\gppoint{gp mark 0}{(4.457,4.330)}
\gppoint{gp mark 0}{(4.457,3.933)}
\gppoint{gp mark 0}{(4.457,4.462)}
\gppoint{gp mark 0}{(4.457,3.759)}
\gppoint{gp mark 0}{(4.457,3.940)}
\gppoint{gp mark 0}{(4.457,3.598)}
\gppoint{gp mark 0}{(4.457,3.805)}
\gppoint{gp mark 0}{(4.457,3.856)}
\gppoint{gp mark 0}{(4.457,3.903)}
\gppoint{gp mark 0}{(4.457,4.179)}
\gppoint{gp mark 0}{(4.457,4.218)}
\gppoint{gp mark 0}{(4.457,3.903)}
\gppoint{gp mark 0}{(4.457,3.493)}
\gppoint{gp mark 0}{(4.457,4.326)}
\gppoint{gp mark 0}{(4.457,4.342)}
\gppoint{gp mark 0}{(4.457,3.768)}
\gppoint{gp mark 0}{(4.457,3.710)}
\gppoint{gp mark 0}{(4.457,3.947)}
\gppoint{gp mark 0}{(4.457,3.903)}
\gppoint{gp mark 0}{(4.457,3.645)}
\gppoint{gp mark 0}{(4.457,3.918)}
\gppoint{gp mark 0}{(4.457,3.918)}
\gppoint{gp mark 0}{(4.457,4.070)}
\gppoint{gp mark 0}{(4.457,3.598)}
\gppoint{gp mark 0}{(4.457,3.561)}
\gppoint{gp mark 0}{(4.457,3.656)}
\gppoint{gp mark 0}{(4.457,3.864)}
\gppoint{gp mark 0}{(4.457,4.302)}
\gppoint{gp mark 0}{(4.457,4.449)}
\gppoint{gp mark 0}{(4.457,4.314)}
\gppoint{gp mark 0}{(4.457,4.449)}
\gppoint{gp mark 0}{(4.457,4.449)}
\gppoint{gp mark 0}{(4.457,4.272)}
\gppoint{gp mark 0}{(4.457,3.699)}
\gppoint{gp mark 0}{(4.457,4.449)}
\gppoint{gp mark 0}{(4.457,4.449)}
\gppoint{gp mark 0}{(4.457,3.805)}
\gppoint{gp mark 0}{(4.457,3.961)}
\gppoint{gp mark 0}{(4.457,4.630)}
\gppoint{gp mark 0}{(4.457,4.449)}
\gppoint{gp mark 0}{(4.457,5.063)}
\gppoint{gp mark 0}{(4.457,3.710)}
\gppoint{gp mark 0}{(4.457,3.740)}
\gppoint{gp mark 0}{(4.457,4.169)}
\gppoint{gp mark 0}{(4.457,3.720)}
\gppoint{gp mark 0}{(4.457,4.756)}
\gppoint{gp mark 0}{(4.457,3.634)}
\gppoint{gp mark 0}{(4.457,4.422)}
\gppoint{gp mark 0}{(4.457,4.439)}
\gppoint{gp mark 0}{(4.457,3.730)}
\gppoint{gp mark 0}{(4.457,3.730)}
\gppoint{gp mark 0}{(4.457,4.082)}
\gppoint{gp mark 0}{(4.457,4.021)}
\gppoint{gp mark 0}{(4.457,3.813)}
\gppoint{gp mark 0}{(4.457,3.740)}
\gppoint{gp mark 0}{(4.457,4.293)}
\gppoint{gp mark 0}{(4.457,4.052)}
\gppoint{gp mark 0}{(4.457,3.961)}
\gppoint{gp mark 0}{(4.457,3.954)}
\gppoint{gp mark 0}{(4.457,3.989)}
\gppoint{gp mark 0}{(4.457,3.968)}
\gppoint{gp mark 0}{(4.457,3.740)}
\gppoint{gp mark 0}{(4.457,3.954)}
\gppoint{gp mark 0}{(4.457,4.143)}
\gppoint{gp mark 0}{(4.457,3.749)}
\gppoint{gp mark 0}{(4.457,4.088)}
\gppoint{gp mark 0}{(4.457,4.475)}
\gppoint{gp mark 0}{(4.457,4.189)}
\gppoint{gp mark 0}{(4.457,3.911)}
\gppoint{gp mark 0}{(4.457,4.346)}
\gppoint{gp mark 0}{(4.457,4.203)}
\gppoint{gp mark 0}{(4.457,3.759)}
\gppoint{gp mark 0}{(4.457,3.464)}
\gppoint{gp mark 0}{(4.457,3.535)}
\gppoint{gp mark 0}{(4.457,3.918)}
\gppoint{gp mark 0}{(4.457,3.699)}
\gppoint{gp mark 0}{(4.457,3.535)}
\gppoint{gp mark 0}{(4.457,3.699)}
\gppoint{gp mark 0}{(4.457,4.184)}
\gppoint{gp mark 0}{(4.457,3.895)}
\gppoint{gp mark 0}{(4.457,3.699)}
\gppoint{gp mark 0}{(4.457,4.679)}
\gppoint{gp mark 0}{(4.457,3.888)}
\gppoint{gp mark 0}{(4.457,3.730)}
\gppoint{gp mark 0}{(4.457,4.021)}
\gppoint{gp mark 0}{(4.457,4.318)}
\gppoint{gp mark 0}{(4.457,3.759)}
\gppoint{gp mark 0}{(4.457,3.402)}
\gppoint{gp mark 0}{(4.457,4.245)}
\gppoint{gp mark 0}{(4.457,4.318)}
\gppoint{gp mark 0}{(4.457,3.926)}
\gppoint{gp mark 0}{(4.457,3.699)}
\gppoint{gp mark 0}{(4.457,3.888)}
\gppoint{gp mark 0}{(4.457,4.466)}
\gppoint{gp mark 0}{(4.457,3.918)}
\gppoint{gp mark 0}{(4.457,3.535)}
\gppoint{gp mark 0}{(4.457,4.158)}
\gppoint{gp mark 0}{(4.457,3.610)}
\gppoint{gp mark 0}{(4.457,3.989)}
\gppoint{gp mark 0}{(4.457,4.272)}
\gppoint{gp mark 0}{(4.457,3.768)}
\gppoint{gp mark 0}{(4.457,3.805)}
\gppoint{gp mark 0}{(4.457,3.982)}
\gppoint{gp mark 0}{(4.457,3.926)}
\gppoint{gp mark 0}{(4.457,3.805)}
\gppoint{gp mark 0}{(4.457,3.768)}
\gppoint{gp mark 0}{(4.457,4.015)}
\gppoint{gp mark 0}{(4.457,3.947)}
\gppoint{gp mark 0}{(4.457,3.947)}
\gppoint{gp mark 0}{(4.457,3.968)}
\gppoint{gp mark 0}{(4.457,3.918)}
\gppoint{gp mark 0}{(4.457,4.368)}
\gppoint{gp mark 0}{(4.457,4.361)}
\gppoint{gp mark 0}{(4.457,3.796)}
\gppoint{gp mark 0}{(4.457,3.888)}
\gppoint{gp mark 0}{(4.457,3.535)}
\gppoint{gp mark 0}{(4.457,3.759)}
\gppoint{gp mark 0}{(4.457,3.872)}
\gppoint{gp mark 0}{(4.457,3.535)}
\gppoint{gp mark 0}{(4.457,3.822)}
\gppoint{gp mark 0}{(4.457,3.888)}
\gppoint{gp mark 0}{(4.457,3.995)}
\gppoint{gp mark 0}{(4.457,3.689)}
\gppoint{gp mark 0}{(4.457,3.720)}
\gppoint{gp mark 0}{(4.457,3.822)}
\gppoint{gp mark 0}{(4.457,4.518)}
\gppoint{gp mark 0}{(4.457,4.657)}
\gppoint{gp mark 0}{(4.457,3.796)}
\gppoint{gp mark 0}{(4.457,3.634)}
\gppoint{gp mark 0}{(4.457,4.203)}
\gppoint{gp mark 0}{(4.457,4.179)}
\gppoint{gp mark 0}{(4.457,3.805)}
\gppoint{gp mark 0}{(4.457,3.805)}
\gppoint{gp mark 0}{(4.457,3.610)}
\gppoint{gp mark 0}{(4.457,4.137)}
\gppoint{gp mark 0}{(4.457,4.137)}
\gppoint{gp mark 0}{(4.457,4.259)}
\gppoint{gp mark 0}{(4.457,3.872)}
\gppoint{gp mark 0}{(4.457,3.720)}
\gppoint{gp mark 0}{(4.490,4.422)}
\gppoint{gp mark 0}{(4.490,4.602)}
\gppoint{gp mark 0}{(4.490,3.918)}
\gppoint{gp mark 0}{(4.490,4.653)}
\gppoint{gp mark 0}{(4.490,3.535)}
\gppoint{gp mark 0}{(4.490,4.578)}
\gppoint{gp mark 0}{(4.490,4.148)}
\gppoint{gp mark 0}{(4.490,4.346)}
\gppoint{gp mark 0}{(4.490,3.768)}
\gppoint{gp mark 0}{(4.490,3.947)}
\gppoint{gp mark 0}{(4.490,3.535)}
\gppoint{gp mark 0}{(4.490,3.903)}
\gppoint{gp mark 0}{(4.490,4.951)}
\gppoint{gp mark 0}{(4.490,4.052)}
\gppoint{gp mark 0}{(4.490,4.346)}
\gppoint{gp mark 0}{(4.490,3.787)}
\gppoint{gp mark 0}{(4.490,4.052)}
\gppoint{gp mark 0}{(4.490,3.689)}
\gppoint{gp mark 0}{(4.490,4.064)}
\gppoint{gp mark 0}{(4.490,4.774)}
\gppoint{gp mark 0}{(4.490,3.479)}
\gppoint{gp mark 0}{(4.490,3.961)}
\gppoint{gp mark 0}{(4.490,3.947)}
\gppoint{gp mark 0}{(4.490,4.127)}
\gppoint{gp mark 0}{(4.490,3.975)}
\gppoint{gp mark 0}{(4.490,3.831)}
\gppoint{gp mark 0}{(4.490,3.787)}
\gppoint{gp mark 0}{(4.490,4.052)}
\gppoint{gp mark 0}{(4.490,3.656)}
\gppoint{gp mark 0}{(4.490,3.610)}
\gppoint{gp mark 0}{(4.490,3.678)}
\gppoint{gp mark 0}{(4.490,4.436)}
\gppoint{gp mark 0}{(4.490,4.184)}
\gppoint{gp mark 0}{(4.490,4.194)}
\gppoint{gp mark 0}{(4.490,4.052)}
\gppoint{gp mark 0}{(4.490,4.524)}
\gppoint{gp mark 0}{(4.490,3.479)}
\gppoint{gp mark 0}{(4.490,3.759)}
\gppoint{gp mark 0}{(4.490,3.634)}
\gppoint{gp mark 0}{(4.490,4.401)}
\gppoint{gp mark 0}{(4.490,4.105)}
\gppoint{gp mark 0}{(4.490,4.110)}
\gppoint{gp mark 0}{(4.490,3.796)}
\gppoint{gp mark 0}{(4.490,3.759)}
\gppoint{gp mark 0}{(4.490,4.116)}
\gppoint{gp mark 0}{(4.490,3.805)}
\gppoint{gp mark 0}{(4.490,3.730)}
\gppoint{gp mark 0}{(4.490,3.730)}
\gppoint{gp mark 0}{(4.490,3.903)}
\gppoint{gp mark 0}{(4.490,3.926)}
\gppoint{gp mark 0}{(4.490,4.361)}
\gppoint{gp mark 0}{(4.490,3.982)}
\gppoint{gp mark 0}{(4.490,3.926)}
\gppoint{gp mark 0}{(4.490,4.353)}
\gppoint{gp mark 0}{(4.490,4.002)}
\gppoint{gp mark 0}{(4.490,3.888)}
\gppoint{gp mark 0}{(4.490,3.918)}
\gppoint{gp mark 0}{(4.490,3.989)}
\gppoint{gp mark 0}{(4.490,4.179)}
\gppoint{gp mark 0}{(4.490,4.179)}
\gppoint{gp mark 0}{(4.490,4.052)}
\gppoint{gp mark 0}{(4.490,4.456)}
\gppoint{gp mark 0}{(4.490,4.052)}
\gppoint{gp mark 0}{(4.490,4.179)}
\gppoint{gp mark 0}{(4.490,4.148)}
\gppoint{gp mark 0}{(4.490,4.503)}
\gppoint{gp mark 0}{(4.490,4.179)}
\gppoint{gp mark 0}{(4.490,4.426)}
\gppoint{gp mark 0}{(4.490,4.052)}
\gppoint{gp mark 0}{(4.490,4.052)}
\gppoint{gp mark 0}{(4.490,3.903)}
\gppoint{gp mark 0}{(4.490,3.813)}
\gppoint{gp mark 0}{(4.490,3.940)}
\gppoint{gp mark 0}{(4.490,4.052)}
\gppoint{gp mark 0}{(4.490,3.548)}
\gppoint{gp mark 0}{(4.490,3.678)}
\gppoint{gp mark 0}{(4.490,3.759)}
\gppoint{gp mark 0}{(4.490,3.586)}
\gppoint{gp mark 0}{(4.490,3.689)}
\gppoint{gp mark 0}{(4.490,3.895)}
\gppoint{gp mark 0}{(4.490,4.184)}
\gppoint{gp mark 0}{(4.490,3.678)}
\gppoint{gp mark 0}{(4.490,4.008)}
\gppoint{gp mark 0}{(4.490,3.678)}
\gppoint{gp mark 0}{(4.490,3.926)}
\gppoint{gp mark 0}{(4.490,4.194)}
\gppoint{gp mark 0}{(4.490,4.184)}
\gppoint{gp mark 0}{(4.490,4.184)}
\gppoint{gp mark 0}{(4.490,4.472)}
\gppoint{gp mark 0}{(4.490,3.926)}
\gppoint{gp mark 0}{(4.490,4.040)}
\gppoint{gp mark 0}{(4.490,4.184)}
\gppoint{gp mark 0}{(4.490,3.975)}
\gppoint{gp mark 0}{(4.490,4.302)}
\gppoint{gp mark 0}{(4.490,4.259)}
\gppoint{gp mark 0}{(4.490,3.689)}
\gppoint{gp mark 0}{(4.490,4.259)}
\gppoint{gp mark 0}{(4.490,3.678)}
\gppoint{gp mark 0}{(4.490,3.778)}
\gppoint{gp mark 0}{(4.490,3.895)}
\gppoint{gp mark 0}{(4.490,3.926)}
\gppoint{gp mark 0}{(4.490,4.174)}
\gppoint{gp mark 0}{(4.490,3.839)}
\gppoint{gp mark 0}{(4.490,4.008)}
\gppoint{gp mark 0}{(4.490,4.137)}
\gppoint{gp mark 0}{(4.490,3.678)}
\gppoint{gp mark 0}{(4.490,4.189)}
\gppoint{gp mark 0}{(4.490,4.040)}
\gppoint{gp mark 0}{(4.490,3.749)}
\gppoint{gp mark 0}{(4.490,4.052)}
\gppoint{gp mark 0}{(4.490,3.634)}
\gppoint{gp mark 0}{(4.490,3.749)}
\gppoint{gp mark 0}{(4.490,3.975)}
\gppoint{gp mark 0}{(4.490,4.105)}
\gppoint{gp mark 0}{(4.490,4.533)}
\gppoint{gp mark 0}{(4.490,3.888)}
\gppoint{gp mark 0}{(4.490,4.052)}
\gppoint{gp mark 0}{(4.490,3.449)}
\gppoint{gp mark 0}{(4.490,3.926)}
\gppoint{gp mark 0}{(4.490,3.740)}
\gppoint{gp mark 0}{(4.490,3.787)}
\gppoint{gp mark 0}{(4.490,4.824)}
\gppoint{gp mark 0}{(4.490,3.918)}
\gppoint{gp mark 0}{(4.490,4.015)}
\gppoint{gp mark 0}{(4.490,3.982)}
\gppoint{gp mark 0}{(4.490,3.995)}
\gppoint{gp mark 0}{(4.490,3.903)}
\gppoint{gp mark 0}{(4.490,4.110)}
\gppoint{gp mark 0}{(4.490,3.961)}
\gppoint{gp mark 0}{(4.490,3.730)}
\gppoint{gp mark 0}{(4.490,4.099)}
\gppoint{gp mark 0}{(4.490,4.008)}
\gppoint{gp mark 0}{(4.490,3.759)}
\gppoint{gp mark 0}{(4.490,4.132)}
\gppoint{gp mark 0}{(4.490,3.895)}
\gppoint{gp mark 0}{(4.490,4.169)}
\gppoint{gp mark 0}{(4.490,3.678)}
\gppoint{gp mark 0}{(4.490,4.002)}
\gppoint{gp mark 0}{(4.490,3.418)}
\gppoint{gp mark 0}{(4.490,3.493)}
\gppoint{gp mark 0}{(4.490,4.148)}
\gppoint{gp mark 0}{(4.490,4.730)}
\gppoint{gp mark 0}{(4.490,4.148)}
\gppoint{gp mark 0}{(4.490,3.710)}
\gppoint{gp mark 0}{(4.490,4.028)}
\gppoint{gp mark 0}{(4.490,3.822)}
\gppoint{gp mark 0}{(4.490,3.768)}
\gppoint{gp mark 0}{(4.490,3.911)}
\gppoint{gp mark 0}{(4.490,3.880)}
\gppoint{gp mark 0}{(4.490,4.008)}
\gppoint{gp mark 0}{(4.490,4.070)}
\gppoint{gp mark 0}{(4.490,4.439)}
\gppoint{gp mark 0}{(4.490,4.695)}
\gppoint{gp mark 0}{(4.490,4.485)}
\gppoint{gp mark 0}{(4.490,4.194)}
\gppoint{gp mark 0}{(4.490,3.947)}
\gppoint{gp mark 0}{(4.490,4.824)}
\gppoint{gp mark 0}{(4.490,4.494)}
\gppoint{gp mark 0}{(4.490,4.550)}
\gppoint{gp mark 0}{(4.490,4.326)}
\gppoint{gp mark 0}{(4.490,4.883)}
\gppoint{gp mark 0}{(4.490,4.116)}
\gppoint{gp mark 0}{(4.490,3.903)}
\gppoint{gp mark 0}{(4.490,3.918)}
\gppoint{gp mark 0}{(4.490,3.903)}
\gppoint{gp mark 0}{(4.490,3.768)}
\gppoint{gp mark 0}{(4.490,4.028)}
\gppoint{gp mark 0}{(4.490,4.449)}
\gppoint{gp mark 0}{(4.490,3.699)}
\gppoint{gp mark 0}{(4.490,4.353)}
\gppoint{gp mark 0}{(4.490,4.959)}
\gppoint{gp mark 0}{(4.490,3.535)}
\gppoint{gp mark 0}{(4.490,3.778)}
\gppoint{gp mark 0}{(4.490,3.856)}
\gppoint{gp mark 0}{(4.490,3.895)}
\gppoint{gp mark 0}{(4.490,4.276)}
\gppoint{gp mark 0}{(4.490,4.730)}
\gppoint{gp mark 0}{(4.490,3.586)}
\gppoint{gp mark 0}{(4.490,3.831)}
\gppoint{gp mark 0}{(4.490,3.778)}
\gppoint{gp mark 0}{(4.490,4.189)}
\gppoint{gp mark 0}{(4.490,4.189)}
\gppoint{gp mark 0}{(4.490,4.189)}
\gppoint{gp mark 0}{(4.490,4.553)}
\gppoint{gp mark 0}{(4.490,4.357)}
\gppoint{gp mark 0}{(4.490,4.127)}
\gppoint{gp mark 0}{(4.490,4.497)}
\gppoint{gp mark 0}{(4.490,4.809)}
\gppoint{gp mark 0}{(4.490,3.918)}
\gppoint{gp mark 0}{(4.490,3.888)}
\gppoint{gp mark 0}{(4.490,3.864)}
\gppoint{gp mark 0}{(4.490,3.464)}
\gppoint{gp mark 0}{(4.490,4.028)}
\gppoint{gp mark 0}{(4.490,3.710)}
\gppoint{gp mark 0}{(4.522,3.689)}
\gppoint{gp mark 0}{(4.522,4.594)}
\gppoint{gp mark 0}{(4.522,4.293)}
\gppoint{gp mark 0}{(4.522,3.768)}
\gppoint{gp mark 0}{(4.522,3.975)}
\gppoint{gp mark 0}{(4.522,3.598)}
\gppoint{gp mark 0}{(4.522,3.933)}
\gppoint{gp mark 0}{(4.522,4.459)}
\gppoint{gp mark 0}{(4.522,4.241)}
\gppoint{gp mark 0}{(4.522,3.805)}
\gppoint{gp mark 0}{(4.522,4.285)}
\gppoint{gp mark 0}{(4.522,4.254)}
\gppoint{gp mark 0}{(4.522,4.436)}
\gppoint{gp mark 0}{(4.522,4.236)}
\gppoint{gp mark 0}{(4.522,3.586)}
\gppoint{gp mark 0}{(4.522,4.241)}
\gppoint{gp mark 0}{(4.522,3.787)}
\gppoint{gp mark 0}{(4.522,4.426)}
\gppoint{gp mark 0}{(4.522,4.330)}
\gppoint{gp mark 0}{(4.522,4.533)}
\gppoint{gp mark 0}{(4.522,4.459)}
\gppoint{gp mark 0}{(4.522,3.667)}
\gppoint{gp mark 0}{(4.522,4.459)}
\gppoint{gp mark 0}{(4.522,3.813)}
\gppoint{gp mark 0}{(4.522,4.250)}
\gppoint{gp mark 0}{(4.522,4.459)}
\gppoint{gp mark 0}{(4.522,3.968)}
\gppoint{gp mark 0}{(4.522,4.153)}
\gppoint{gp mark 0}{(4.522,4.169)}
\gppoint{gp mark 0}{(4.522,4.250)}
\gppoint{gp mark 0}{(4.522,4.028)}
\gppoint{gp mark 0}{(4.522,3.864)}
\gppoint{gp mark 0}{(4.522,4.380)}
\gppoint{gp mark 0}{(4.522,4.459)}
\gppoint{gp mark 0}{(4.522,3.813)}
\gppoint{gp mark 0}{(4.522,4.573)}
\gppoint{gp mark 0}{(4.522,3.295)}
\gppoint{gp mark 0}{(4.522,3.720)}
\gppoint{gp mark 0}{(4.522,3.678)}
\gppoint{gp mark 0}{(4.522,3.573)}
\gppoint{gp mark 0}{(4.522,4.391)}
\gppoint{gp mark 0}{(4.522,3.911)}
\gppoint{gp mark 0}{(4.522,3.911)}
\gppoint{gp mark 0}{(4.522,3.982)}
\gppoint{gp mark 0}{(4.522,4.800)}
\gppoint{gp mark 0}{(4.522,3.730)}
\gppoint{gp mark 0}{(4.522,3.975)}
\gppoint{gp mark 0}{(4.522,3.598)}
\gppoint{gp mark 0}{(4.522,4.245)}
\gppoint{gp mark 0}{(4.522,4.662)}
\gppoint{gp mark 0}{(4.522,3.598)}
\gppoint{gp mark 0}{(4.522,3.911)}
\gppoint{gp mark 0}{(4.522,3.947)}
\gppoint{gp mark 0}{(4.522,4.158)}
\gppoint{gp mark 0}{(4.522,3.911)}
\gppoint{gp mark 0}{(4.522,4.110)}
\gppoint{gp mark 0}{(4.522,3.940)}
\gppoint{gp mark 0}{(4.522,3.796)}
\gppoint{gp mark 0}{(4.522,4.046)}
\gppoint{gp mark 0}{(4.522,4.372)}
\gppoint{gp mark 0}{(4.522,4.058)}
\gppoint{gp mark 0}{(4.522,4.232)}
\gppoint{gp mark 0}{(4.522,4.365)}
\gppoint{gp mark 0}{(4.522,3.796)}
\gppoint{gp mark 0}{(4.522,4.232)}
\gppoint{gp mark 0}{(4.522,4.387)}
\gppoint{gp mark 0}{(4.522,4.213)}
\gppoint{gp mark 0}{(4.522,3.940)}
\gppoint{gp mark 0}{(4.522,3.947)}
\gppoint{gp mark 0}{(4.522,3.975)}
\gppoint{gp mark 0}{(4.522,3.622)}
\gppoint{gp mark 0}{(4.522,3.975)}
\gppoint{gp mark 0}{(4.522,3.493)}
\gppoint{gp mark 0}{(4.522,4.387)}
\gppoint{gp mark 0}{(4.522,3.864)}
\gppoint{gp mark 0}{(4.522,3.831)}
\gppoint{gp mark 0}{(4.522,3.822)}
\gppoint{gp mark 0}{(4.522,3.880)}
\gppoint{gp mark 0}{(4.522,4.164)}
\gppoint{gp mark 0}{(4.522,3.961)}
\gppoint{gp mark 0}{(4.522,4.250)}
\gppoint{gp mark 0}{(4.522,4.594)}
\gppoint{gp mark 0}{(4.522,3.634)}
\gppoint{gp mark 0}{(4.522,3.778)}
\gppoint{gp mark 0}{(4.522,3.634)}
\gppoint{gp mark 0}{(4.522,3.678)}
\gppoint{gp mark 0}{(4.522,3.995)}
\gppoint{gp mark 0}{(4.522,3.759)}
\gppoint{gp mark 0}{(4.522,4.189)}
\gppoint{gp mark 0}{(4.522,4.008)}
\gppoint{gp mark 0}{(4.522,3.759)}
\gppoint{gp mark 0}{(4.522,4.015)}
\gppoint{gp mark 0}{(4.522,4.058)}
\gppoint{gp mark 0}{(4.522,3.759)}
\gppoint{gp mark 0}{(4.522,3.622)}
\gppoint{gp mark 0}{(4.522,3.911)}
\gppoint{gp mark 0}{(4.522,3.561)}
\gppoint{gp mark 0}{(4.522,4.179)}
\gppoint{gp mark 0}{(4.522,3.822)}
\gppoint{gp mark 0}{(4.522,3.918)}
\gppoint{gp mark 0}{(4.522,3.918)}
\gppoint{gp mark 0}{(4.522,3.561)}
\gppoint{gp mark 0}{(4.522,3.787)}
\gppoint{gp mark 0}{(4.522,4.153)}
\gppoint{gp mark 0}{(4.522,4.584)}
\gppoint{gp mark 0}{(4.522,3.975)}
\gppoint{gp mark 0}{(4.522,3.975)}
\gppoint{gp mark 0}{(4.522,4.076)}
\gppoint{gp mark 0}{(4.522,4.415)}
\gppoint{gp mark 0}{(4.522,4.263)}
\gppoint{gp mark 0}{(4.522,4.625)}
\gppoint{gp mark 0}{(4.522,3.796)}
\gppoint{gp mark 0}{(4.522,3.989)}
\gppoint{gp mark 0}{(4.522,4.245)}
\gppoint{gp mark 0}{(4.522,4.064)}
\gppoint{gp mark 0}{(4.522,3.787)}
\gppoint{gp mark 0}{(4.522,4.314)}
\gppoint{gp mark 0}{(4.522,3.839)}
\gppoint{gp mark 0}{(4.522,3.598)}
\gppoint{gp mark 0}{(4.522,4.213)}
\gppoint{gp mark 0}{(4.522,3.759)}
\gppoint{gp mark 0}{(4.522,3.535)}
\gppoint{gp mark 0}{(4.522,3.493)}
\gppoint{gp mark 0}{(4.522,4.008)}
\gppoint{gp mark 0}{(4.522,3.911)}
\gppoint{gp mark 0}{(4.522,4.148)}
\gppoint{gp mark 0}{(4.522,3.622)}
\gppoint{gp mark 0}{(4.522,4.015)}
\gppoint{gp mark 0}{(4.522,4.040)}
\gppoint{gp mark 0}{(4.522,3.895)}
\gppoint{gp mark 0}{(4.522,3.968)}
\gppoint{gp mark 0}{(4.522,3.689)}
\gppoint{gp mark 0}{(4.522,3.667)}
\gppoint{gp mark 0}{(4.522,3.839)}
\gppoint{gp mark 0}{(4.522,4.164)}
\gppoint{gp mark 0}{(4.522,3.720)}
\gppoint{gp mark 0}{(4.522,4.241)}
\gppoint{gp mark 0}{(4.522,4.227)}
\gppoint{gp mark 0}{(4.522,3.895)}
\gppoint{gp mark 0}{(4.522,3.656)}
\gppoint{gp mark 0}{(4.522,4.015)}
\gppoint{gp mark 0}{(4.522,4.893)}
\gppoint{gp mark 0}{(4.522,4.302)}
\gppoint{gp mark 0}{(4.522,4.334)}
\gppoint{gp mark 0}{(4.522,3.805)}
\gppoint{gp mark 0}{(4.522,3.689)}
\gppoint{gp mark 0}{(4.522,3.689)}
\gppoint{gp mark 0}{(4.522,4.105)}
\gppoint{gp mark 0}{(4.522,3.831)}
\gppoint{gp mark 0}{(4.522,3.535)}
\gppoint{gp mark 0}{(4.522,3.961)}
\gppoint{gp mark 0}{(4.522,3.895)}
\gppoint{gp mark 0}{(4.522,3.926)}
\gppoint{gp mark 0}{(4.522,4.064)}
\gppoint{gp mark 0}{(4.522,4.334)}
\gppoint{gp mark 0}{(4.522,4.581)}
\gppoint{gp mark 0}{(4.522,3.839)}
\gppoint{gp mark 0}{(4.522,4.724)}
\gppoint{gp mark 0}{(4.522,3.493)}
\gppoint{gp mark 0}{(4.522,3.402)}
\gppoint{gp mark 0}{(4.522,3.926)}
\gppoint{gp mark 0}{(4.522,4.942)}
\gppoint{gp mark 0}{(4.522,3.954)}
\gppoint{gp mark 0}{(4.522,3.888)}
\gppoint{gp mark 0}{(4.522,4.581)}
\gppoint{gp mark 0}{(4.522,4.184)}
\gppoint{gp mark 0}{(4.522,3.295)}
\gppoint{gp mark 0}{(4.522,3.880)}
\gppoint{gp mark 0}{(4.522,3.730)}
\gppoint{gp mark 0}{(4.522,3.872)}
\gppoint{gp mark 0}{(4.522,3.710)}
\gppoint{gp mark 0}{(4.522,4.491)}
\gppoint{gp mark 0}{(4.522,4.064)}
\gppoint{gp mark 0}{(4.522,4.719)}
\gppoint{gp mark 0}{(4.522,3.645)}
\gppoint{gp mark 0}{(4.522,4.263)}
\gppoint{gp mark 0}{(4.522,3.982)}
\gppoint{gp mark 0}{(4.522,3.822)}
\gppoint{gp mark 0}{(4.522,4.456)}
\gppoint{gp mark 0}{(4.522,4.391)}
\gppoint{gp mark 0}{(4.522,4.198)}
\gppoint{gp mark 0}{(4.522,4.052)}
\gppoint{gp mark 0}{(4.522,4.383)}
\gppoint{gp mark 0}{(4.522,3.667)}
\gppoint{gp mark 0}{(4.522,4.034)}
\gppoint{gp mark 0}{(4.522,4.285)}
\gppoint{gp mark 0}{(4.522,3.926)}
\gppoint{gp mark 0}{(4.522,3.622)}
\gppoint{gp mark 0}{(4.522,4.488)}
\gppoint{gp mark 0}{(4.522,3.730)}
\gppoint{gp mark 0}{(4.522,3.856)}
\gppoint{gp mark 0}{(4.552,3.975)}
\gppoint{gp mark 0}{(4.552,3.535)}
\gppoint{gp mark 0}{(4.552,4.618)}
\gppoint{gp mark 0}{(4.552,4.545)}
\gppoint{gp mark 0}{(4.552,4.082)}
\gppoint{gp mark 0}{(4.552,4.034)}
\gppoint{gp mark 0}{(4.552,4.088)}
\gppoint{gp mark 0}{(4.552,4.121)}
\gppoint{gp mark 0}{(4.552,3.961)}
\gppoint{gp mark 0}{(4.552,3.903)}
\gppoint{gp mark 0}{(4.552,4.121)}
\gppoint{gp mark 0}{(4.552,3.730)}
\gppoint{gp mark 0}{(4.552,3.989)}
\gppoint{gp mark 0}{(4.552,4.357)}
\gppoint{gp mark 0}{(4.552,4.848)}
\gppoint{gp mark 0}{(4.552,3.656)}
\gppoint{gp mark 0}{(4.552,3.610)}
\gppoint{gp mark 0}{(4.552,3.610)}
\gppoint{gp mark 0}{(4.552,4.302)}
\gppoint{gp mark 0}{(4.552,3.813)}
\gppoint{gp mark 0}{(4.552,4.076)}
\gppoint{gp mark 0}{(4.552,4.232)}
\gppoint{gp mark 0}{(4.552,4.443)}
\gppoint{gp mark 0}{(4.552,3.847)}
\gppoint{gp mark 0}{(4.552,4.021)}
\gppoint{gp mark 0}{(4.552,4.503)}
\gppoint{gp mark 0}{(4.552,3.805)}
\gppoint{gp mark 0}{(4.552,3.903)}
\gppoint{gp mark 0}{(4.552,4.297)}
\gppoint{gp mark 0}{(4.552,3.864)}
\gppoint{gp mark 0}{(4.552,4.121)}
\gppoint{gp mark 0}{(4.552,4.302)}
\gppoint{gp mark 0}{(4.552,4.533)}
\gppoint{gp mark 0}{(4.552,3.989)}
\gppoint{gp mark 0}{(4.552,3.888)}
\gppoint{gp mark 0}{(4.552,3.903)}
\gppoint{gp mark 0}{(4.552,4.015)}
\gppoint{gp mark 0}{(4.552,4.046)}
\gppoint{gp mark 0}{(4.552,3.740)}
\gppoint{gp mark 0}{(4.552,3.656)}
\gppoint{gp mark 0}{(4.552,3.768)}
\gppoint{gp mark 0}{(4.552,4.326)}
\gppoint{gp mark 0}{(4.552,4.530)}
\gppoint{gp mark 0}{(4.552,4.293)}
\gppoint{gp mark 0}{(4.552,3.768)}
\gppoint{gp mark 0}{(4.552,4.021)}
\gppoint{gp mark 0}{(4.552,3.895)}
\gppoint{gp mark 0}{(4.552,4.169)}
\gppoint{gp mark 0}{(4.552,4.559)}
\gppoint{gp mark 0}{(4.552,4.263)}
\gppoint{gp mark 0}{(4.552,4.236)}
\gppoint{gp mark 0}{(4.552,4.638)}
\gppoint{gp mark 0}{(4.552,3.961)}
\gppoint{gp mark 0}{(4.552,3.805)}
\gppoint{gp mark 0}{(4.552,4.127)}
\gppoint{gp mark 0}{(4.552,3.864)}
\gppoint{gp mark 0}{(4.552,3.805)}
\gppoint{gp mark 0}{(4.552,4.093)}
\gppoint{gp mark 0}{(4.552,3.975)}
\gppoint{gp mark 0}{(4.552,3.831)}
\gppoint{gp mark 0}{(4.552,3.573)}
\gppoint{gp mark 0}{(4.552,3.989)}
\gppoint{gp mark 0}{(4.552,4.034)}
\gppoint{gp mark 0}{(4.552,3.989)}
\gppoint{gp mark 0}{(4.552,4.132)}
\gppoint{gp mark 0}{(4.552,3.954)}
\gppoint{gp mark 0}{(4.552,3.656)}
\gppoint{gp mark 0}{(4.552,4.453)}
\gppoint{gp mark 0}{(4.552,4.208)}
\gppoint{gp mark 0}{(4.552,3.634)}
\gppoint{gp mark 0}{(4.552,4.046)}
\gppoint{gp mark 0}{(4.552,3.010)}
\gppoint{gp mark 0}{(4.552,4.357)}
\gppoint{gp mark 0}{(4.552,3.831)}
\gppoint{gp mark 0}{(4.552,3.995)}
\gppoint{gp mark 0}{(4.552,3.888)}
\gppoint{gp mark 0}{(4.552,3.831)}
\gppoint{gp mark 0}{(4.552,3.903)}
\gppoint{gp mark 0}{(4.552,4.076)}
\gppoint{gp mark 0}{(4.552,4.121)}
\gppoint{gp mark 0}{(4.552,4.607)}
\gppoint{gp mark 0}{(4.552,3.548)}
\gppoint{gp mark 0}{(4.552,4.453)}
\gppoint{gp mark 0}{(4.552,3.710)}
\gppoint{gp mark 0}{(4.552,3.864)}
\gppoint{gp mark 0}{(4.552,3.911)}
\gppoint{gp mark 0}{(4.552,3.982)}
\gppoint{gp mark 0}{(4.552,4.040)}
\gppoint{gp mark 0}{(4.552,4.500)}
\gppoint{gp mark 0}{(4.552,4.443)}
\gppoint{gp mark 0}{(4.552,3.598)}
\gppoint{gp mark 0}{(4.552,4.189)}
\gppoint{gp mark 0}{(4.552,4.259)}
\gppoint{gp mark 0}{(4.552,4.267)}
\gppoint{gp mark 0}{(4.552,4.426)}
\gppoint{gp mark 0}{(4.552,4.099)}
\gppoint{gp mark 0}{(4.552,3.759)}
\gppoint{gp mark 0}{(4.552,3.880)}
\gppoint{gp mark 0}{(4.552,4.422)}
\gppoint{gp mark 0}{(4.552,4.276)}
\gppoint{gp mark 0}{(4.552,3.805)}
\gppoint{gp mark 0}{(4.552,3.656)}
\gppoint{gp mark 0}{(4.552,4.436)}
\gppoint{gp mark 0}{(4.552,3.911)}
\gppoint{gp mark 0}{(4.552,3.699)}
\gppoint{gp mark 0}{(4.552,4.500)}
\gppoint{gp mark 0}{(4.552,4.361)}
\gppoint{gp mark 0}{(4.552,3.796)}
\gppoint{gp mark 0}{(4.552,3.888)}
\gppoint{gp mark 0}{(4.552,4.607)}
\gppoint{gp mark 0}{(4.552,3.911)}
\gppoint{gp mark 0}{(4.552,3.847)}
\gppoint{gp mark 0}{(4.552,4.088)}
\gppoint{gp mark 0}{(4.552,3.740)}
\gppoint{gp mark 0}{(4.552,3.975)}
\gppoint{gp mark 0}{(4.552,4.070)}
\gppoint{gp mark 0}{(4.552,3.864)}
\gppoint{gp mark 0}{(4.552,3.634)}
\gppoint{gp mark 0}{(4.552,4.028)}
\gppoint{gp mark 0}{(4.552,4.121)}
\gppoint{gp mark 0}{(4.552,3.720)}
\gppoint{gp mark 0}{(4.552,3.667)}
\gppoint{gp mark 0}{(4.552,3.864)}
\gppoint{gp mark 0}{(4.552,3.368)}
\gppoint{gp mark 0}{(4.552,4.426)}
\gppoint{gp mark 0}{(4.552,3.645)}
\gppoint{gp mark 0}{(4.552,4.272)}
\gppoint{gp mark 0}{(4.552,3.768)}
\gppoint{gp mark 0}{(4.552,4.028)}
\gppoint{gp mark 0}{(4.552,3.768)}
\gppoint{gp mark 0}{(4.552,4.116)}
\gppoint{gp mark 0}{(4.552,4.002)}
\gppoint{gp mark 0}{(4.552,3.911)}
\gppoint{gp mark 0}{(4.552,4.625)}
\gppoint{gp mark 0}{(4.552,4.314)}
\gppoint{gp mark 0}{(4.552,3.768)}
\gppoint{gp mark 0}{(4.552,3.872)}
\gppoint{gp mark 0}{(4.552,3.895)}
\gppoint{gp mark 0}{(4.552,4.046)}
\gppoint{gp mark 0}{(4.552,4.040)}
\gppoint{gp mark 0}{(4.552,4.361)}
\gppoint{gp mark 0}{(4.552,4.453)}
\gppoint{gp mark 0}{(4.552,4.361)}
\gppoint{gp mark 0}{(4.552,3.787)}
\gppoint{gp mark 0}{(4.552,3.961)}
\gppoint{gp mark 0}{(4.552,4.276)}
\gppoint{gp mark 0}{(4.552,3.787)}
\gppoint{gp mark 0}{(4.552,3.982)}
\gppoint{gp mark 0}{(4.552,4.245)}
\gppoint{gp mark 0}{(4.552,4.436)}
\gppoint{gp mark 0}{(4.552,4.314)}
\gppoint{gp mark 0}{(4.552,3.895)}
\gppoint{gp mark 0}{(4.552,3.778)}
\gppoint{gp mark 0}{(4.552,4.330)}
\gppoint{gp mark 0}{(4.552,4.605)}
\gppoint{gp mark 0}{(4.552,3.954)}
\gppoint{gp mark 0}{(4.552,4.198)}
\gppoint{gp mark 0}{(4.552,3.872)}
\gppoint{gp mark 0}{(4.552,4.174)}
\gppoint{gp mark 0}{(4.552,3.796)}
\gppoint{gp mark 0}{(4.552,3.368)}
\gppoint{gp mark 0}{(4.552,4.330)}
\gppoint{gp mark 0}{(4.552,3.888)}
\gppoint{gp mark 0}{(4.552,4.121)}
\gppoint{gp mark 0}{(4.552,3.805)}
\gppoint{gp mark 0}{(4.582,3.710)}
\gppoint{gp mark 0}{(4.582,3.710)}
\gppoint{gp mark 0}{(4.582,4.046)}
\gppoint{gp mark 0}{(4.582,4.021)}
\gppoint{gp mark 0}{(4.582,4.567)}
\gppoint{gp mark 0}{(4.582,4.076)}
\gppoint{gp mark 0}{(4.582,4.548)}
\gppoint{gp mark 0}{(4.582,3.796)}
\gppoint{gp mark 0}{(4.582,4.137)}
\gppoint{gp mark 0}{(4.582,3.796)}
\gppoint{gp mark 0}{(4.582,3.796)}
\gppoint{gp mark 0}{(4.582,4.099)}
\gppoint{gp mark 0}{(4.582,3.813)}
\gppoint{gp mark 0}{(4.582,3.995)}
\gppoint{gp mark 0}{(4.582,3.847)}
\gppoint{gp mark 0}{(4.582,4.809)}
\gppoint{gp mark 0}{(4.582,4.383)}
\gppoint{gp mark 0}{(4.582,3.634)}
\gppoint{gp mark 0}{(4.582,3.730)}
\gppoint{gp mark 0}{(4.582,4.545)}
\gppoint{gp mark 0}{(4.582,3.535)}
\gppoint{gp mark 0}{(4.582,4.236)}
\gppoint{gp mark 0}{(4.582,3.933)}
\gppoint{gp mark 0}{(4.582,4.412)}
\gppoint{gp mark 0}{(4.582,4.780)}
\gppoint{gp mark 0}{(4.582,4.322)}
\gppoint{gp mark 0}{(4.582,4.412)}
\gppoint{gp mark 0}{(4.582,4.412)}
\gppoint{gp mark 0}{(4.582,4.116)}
\gppoint{gp mark 0}{(4.582,4.289)}
\gppoint{gp mark 0}{(4.582,3.493)}
\gppoint{gp mark 0}{(4.582,4.412)}
\gppoint{gp mark 0}{(4.582,4.602)}
\gppoint{gp mark 0}{(4.582,3.699)}
\gppoint{gp mark 0}{(4.582,4.562)}
\gppoint{gp mark 0}{(4.582,4.164)}
\gppoint{gp mark 0}{(4.582,4.559)}
\gppoint{gp mark 0}{(4.582,4.545)}
\gppoint{gp mark 0}{(4.582,5.166)}
\gppoint{gp mark 0}{(4.582,3.864)}
\gppoint{gp mark 0}{(4.582,4.046)}
\gppoint{gp mark 0}{(4.582,3.933)}
\gppoint{gp mark 0}{(4.582,3.975)}
\gppoint{gp mark 0}{(4.582,4.581)}
\gppoint{gp mark 0}{(4.582,4.046)}
\gppoint{gp mark 0}{(4.582,4.837)}
\gppoint{gp mark 0}{(4.582,4.711)}
\gppoint{gp mark 0}{(4.582,4.213)}
\gppoint{gp mark 0}{(4.582,4.774)}
\gppoint{gp mark 0}{(4.582,3.710)}
\gppoint{gp mark 0}{(4.582,3.911)}
\gppoint{gp mark 0}{(4.582,3.645)}
\gppoint{gp mark 0}{(4.582,4.564)}
\gppoint{gp mark 0}{(4.582,4.589)}
\gppoint{gp mark 0}{(4.582,4.052)}
\gppoint{gp mark 0}{(4.582,4.443)}
\gppoint{gp mark 0}{(4.582,3.813)}
\gppoint{gp mark 0}{(4.582,3.573)}
\gppoint{gp mark 0}{(4.582,4.174)}
\gppoint{gp mark 0}{(4.582,4.158)}
\gppoint{gp mark 0}{(4.582,3.645)}
\gppoint{gp mark 0}{(4.582,4.058)}
\gppoint{gp mark 0}{(4.582,3.926)}
\gppoint{gp mark 0}{(4.582,4.164)}
\gppoint{gp mark 0}{(4.582,4.817)}
\gppoint{gp mark 0}{(4.582,3.634)}
\gppoint{gp mark 0}{(4.582,4.158)}
\gppoint{gp mark 0}{(4.582,4.132)}
\gppoint{gp mark 0}{(4.582,4.322)}
\gppoint{gp mark 0}{(4.582,4.826)}
\gppoint{gp mark 0}{(4.582,4.015)}
\gppoint{gp mark 0}{(4.582,4.052)}
\gppoint{gp mark 0}{(4.582,3.954)}
\gppoint{gp mark 0}{(4.582,4.443)}
\gppoint{gp mark 0}{(4.582,3.995)}
\gppoint{gp mark 0}{(4.582,4.058)}
\gppoint{gp mark 0}{(4.582,4.533)}
\gppoint{gp mark 0}{(4.582,4.241)}
\gppoint{gp mark 0}{(4.582,3.796)}
\gppoint{gp mark 0}{(4.582,3.351)}
\gppoint{gp mark 0}{(4.582,4.533)}
\gppoint{gp mark 0}{(4.582,4.322)}
\gppoint{gp mark 0}{(4.582,3.351)}
\gppoint{gp mark 0}{(4.582,4.372)}
\gppoint{gp mark 0}{(4.582,4.764)}
\gppoint{gp mark 0}{(4.582,4.093)}
\gppoint{gp mark 0}{(4.582,4.469)}
\gppoint{gp mark 0}{(4.582,3.864)}
\gppoint{gp mark 0}{(4.582,4.105)}
\gppoint{gp mark 0}{(4.582,4.681)}
\gppoint{gp mark 0}{(4.582,3.710)}
\gppoint{gp mark 0}{(4.582,4.597)}
\gppoint{gp mark 0}{(4.582,4.469)}
\gppoint{gp mark 0}{(4.582,3.749)}
\gppoint{gp mark 0}{(4.582,4.408)}
\gppoint{gp mark 0}{(4.582,3.493)}
\gppoint{gp mark 0}{(4.582,3.493)}
\gppoint{gp mark 0}{(4.582,4.008)}
\gppoint{gp mark 0}{(4.582,4.076)}
\gppoint{gp mark 0}{(4.582,4.660)}
\gppoint{gp mark 0}{(4.582,4.533)}
\gppoint{gp mark 0}{(4.582,4.536)}
\gppoint{gp mark 0}{(4.582,3.402)}
\gppoint{gp mark 0}{(4.582,4.232)}
\gppoint{gp mark 0}{(4.582,4.506)}
\gppoint{gp mark 0}{(4.582,3.656)}
\gppoint{gp mark 0}{(4.582,4.259)}
\gppoint{gp mark 0}{(4.582,4.503)}
\gppoint{gp mark 0}{(4.582,4.387)}
\gppoint{gp mark 0}{(4.582,4.194)}
\gppoint{gp mark 0}{(4.582,4.218)}
\gppoint{gp mark 0}{(4.582,3.839)}
\gppoint{gp mark 0}{(4.582,3.689)}
\gppoint{gp mark 0}{(4.582,4.297)}
\gppoint{gp mark 0}{(4.582,3.740)}
\gppoint{gp mark 0}{(4.582,3.982)}
\gppoint{gp mark 0}{(4.582,3.667)}
\gppoint{gp mark 0}{(4.582,4.533)}
\gppoint{gp mark 0}{(4.582,4.241)}
\gppoint{gp mark 0}{(4.582,3.710)}
\gppoint{gp mark 0}{(4.582,3.730)}
\gppoint{gp mark 0}{(4.582,3.856)}
\gppoint{gp mark 0}{(4.582,3.730)}
\gppoint{gp mark 0}{(4.582,4.127)}
\gppoint{gp mark 0}{(4.582,4.021)}
\gppoint{gp mark 0}{(4.582,3.968)}
\gppoint{gp mark 0}{(4.582,4.110)}
\gppoint{gp mark 0}{(4.582,3.856)}
\gppoint{gp mark 0}{(4.582,4.064)}
\gppoint{gp mark 0}{(4.582,3.839)}
\gppoint{gp mark 0}{(4.582,3.954)}
\gppoint{gp mark 0}{(4.582,4.208)}
\gppoint{gp mark 0}{(4.582,3.888)}
\gppoint{gp mark 0}{(4.582,4.456)}
\gppoint{gp mark 0}{(4.582,3.573)}
\gppoint{gp mark 0}{(4.582,4.179)}
\gppoint{gp mark 0}{(4.582,4.488)}
\gppoint{gp mark 0}{(4.582,4.169)}
\gppoint{gp mark 0}{(4.582,3.856)}
\gppoint{gp mark 0}{(4.582,3.573)}
\gppoint{gp mark 0}{(4.582,4.429)}
\gppoint{gp mark 0}{(4.582,3.954)}
\gppoint{gp mark 0}{(4.582,3.968)}
\gppoint{gp mark 0}{(4.582,4.093)}
\gppoint{gp mark 0}{(4.611,4.179)}
\gppoint{gp mark 0}{(4.611,4.082)}
\gppoint{gp mark 0}{(4.611,4.326)}
\gppoint{gp mark 0}{(4.611,4.306)}
\gppoint{gp mark 0}{(4.611,3.787)}
\gppoint{gp mark 0}{(4.611,4.203)}
\gppoint{gp mark 0}{(4.611,4.158)}
\gppoint{gp mark 0}{(4.611,3.805)}
\gppoint{gp mark 0}{(4.611,4.203)}
\gppoint{gp mark 0}{(4.611,3.926)}
\gppoint{gp mark 0}{(4.611,3.720)}
\gppoint{gp mark 0}{(4.611,3.839)}
\gppoint{gp mark 0}{(4.611,4.203)}
\gppoint{gp mark 0}{(4.611,3.759)}
\gppoint{gp mark 0}{(4.611,4.203)}
\gppoint{gp mark 0}{(4.611,3.787)}
\gppoint{gp mark 0}{(4.611,4.607)}
\gppoint{gp mark 0}{(4.611,4.638)}
\gppoint{gp mark 0}{(4.611,4.607)}
\gppoint{gp mark 0}{(4.611,3.720)}
\gppoint{gp mark 0}{(4.611,4.387)}
\gppoint{gp mark 0}{(4.611,4.638)}
\gppoint{gp mark 0}{(4.611,3.911)}
\gppoint{gp mark 0}{(4.611,4.607)}
\gppoint{gp mark 0}{(4.611,3.989)}
\gppoint{gp mark 0}{(4.611,3.730)}
\gppoint{gp mark 0}{(4.611,4.607)}
\gppoint{gp mark 0}{(4.611,4.194)}
\gppoint{gp mark 0}{(4.611,4.008)}
\gppoint{gp mark 0}{(4.611,3.926)}
\gppoint{gp mark 0}{(4.611,4.213)}
\gppoint{gp mark 0}{(4.611,4.449)}
\gppoint{gp mark 0}{(4.611,3.787)}
\gppoint{gp mark 0}{(4.611,4.735)}
\gppoint{gp mark 0}{(4.611,3.768)}
\gppoint{gp mark 0}{(4.611,4.174)}
\gppoint{gp mark 0}{(4.611,4.843)}
\gppoint{gp mark 0}{(4.611,4.028)}
\gppoint{gp mark 0}{(4.611,4.462)}
\gppoint{gp mark 0}{(4.611,3.911)}
\gppoint{gp mark 0}{(4.611,4.515)}
\gppoint{gp mark 0}{(4.611,4.754)}
\gppoint{gp mark 0}{(4.611,3.678)}
\gppoint{gp mark 0}{(4.611,3.995)}
\gppoint{gp mark 0}{(4.611,4.121)}
\gppoint{gp mark 0}{(4.611,3.622)}
\gppoint{gp mark 0}{(4.611,4.208)}
\gppoint{gp mark 0}{(4.611,3.787)}
\gppoint{gp mark 0}{(4.611,3.888)}
\gppoint{gp mark 0}{(4.611,3.740)}
\gppoint{gp mark 0}{(4.611,4.218)}
\gppoint{gp mark 0}{(4.611,3.895)}
\gppoint{gp mark 0}{(4.611,4.667)}
\gppoint{gp mark 0}{(4.611,4.099)}
\gppoint{gp mark 0}{(4.611,4.280)}
\gppoint{gp mark 0}{(4.611,4.485)}
\gppoint{gp mark 0}{(4.611,4.491)}
\gppoint{gp mark 0}{(4.611,5.061)}
\gppoint{gp mark 0}{(4.611,3.740)}
\gppoint{gp mark 0}{(4.611,4.846)}
\gppoint{gp mark 0}{(4.611,3.856)}
\gppoint{gp mark 0}{(4.611,3.573)}
\gppoint{gp mark 0}{(4.611,4.236)}
\gppoint{gp mark 0}{(4.611,3.995)}
\gppoint{gp mark 0}{(4.611,3.759)}
\gppoint{gp mark 0}{(4.611,4.121)}
\gppoint{gp mark 0}{(4.611,4.342)}
\gppoint{gp mark 0}{(4.611,4.076)}
\gppoint{gp mark 0}{(4.611,4.798)}
\gppoint{gp mark 0}{(4.611,3.864)}
\gppoint{gp mark 0}{(4.611,3.699)}
\gppoint{gp mark 0}{(4.611,4.194)}
\gppoint{gp mark 0}{(4.611,3.699)}
\gppoint{gp mark 0}{(4.611,4.562)}
\gppoint{gp mark 0}{(4.611,3.918)}
\gppoint{gp mark 0}{(4.611,3.710)}
\gppoint{gp mark 0}{(4.611,4.132)}
\gppoint{gp mark 0}{(4.611,3.911)}
\gppoint{gp mark 0}{(4.611,4.169)}
\gppoint{gp mark 0}{(4.611,3.961)}
\gppoint{gp mark 0}{(4.611,3.872)}
\gppoint{gp mark 0}{(4.611,3.880)}
\gppoint{gp mark 0}{(4.611,4.198)}
\gppoint{gp mark 0}{(4.611,3.805)}
\gppoint{gp mark 0}{(4.611,4.426)}
\gppoint{gp mark 0}{(4.611,3.710)}
\gppoint{gp mark 0}{(4.611,4.349)}
\gppoint{gp mark 0}{(4.611,4.164)}
\gppoint{gp mark 0}{(4.611,4.076)}
\gppoint{gp mark 0}{(4.611,4.040)}
\gppoint{gp mark 0}{(4.611,3.995)}
\gppoint{gp mark 0}{(4.611,3.847)}
\gppoint{gp mark 0}{(4.611,4.387)}
\gppoint{gp mark 0}{(4.611,3.864)}
\gppoint{gp mark 0}{(4.611,4.983)}
\gppoint{gp mark 0}{(4.611,4.070)}
\gppoint{gp mark 0}{(4.611,3.872)}
\gppoint{gp mark 0}{(4.611,4.076)}
\gppoint{gp mark 0}{(4.611,4.638)}
\gppoint{gp mark 0}{(4.611,3.933)}
\gppoint{gp mark 0}{(4.611,4.365)}
\gppoint{gp mark 0}{(4.611,3.995)}
\gppoint{gp mark 0}{(4.611,4.387)}
\gppoint{gp mark 0}{(4.611,3.787)}
\gppoint{gp mark 0}{(4.611,3.954)}
\gppoint{gp mark 0}{(4.611,4.798)}
\gppoint{gp mark 0}{(4.611,3.740)}
\gppoint{gp mark 0}{(4.611,3.888)}
\gppoint{gp mark 0}{(4.611,3.864)}
\gppoint{gp mark 0}{(4.611,3.634)}
\gppoint{gp mark 0}{(4.611,3.995)}
\gppoint{gp mark 0}{(4.611,3.610)}
\gppoint{gp mark 0}{(4.611,3.888)}
\gppoint{gp mark 0}{(4.611,3.895)}
\gppoint{gp mark 0}{(4.611,4.137)}
\gppoint{gp mark 0}{(4.611,3.730)}
\gppoint{gp mark 0}{(4.611,3.720)}
\gppoint{gp mark 0}{(4.611,4.798)}
\gppoint{gp mark 0}{(4.611,3.749)}
\gppoint{gp mark 0}{(4.611,3.610)}
\gppoint{gp mark 0}{(4.611,4.021)}
\gppoint{gp mark 0}{(4.611,3.493)}
\gppoint{gp mark 0}{(4.611,3.333)}
\gppoint{gp mark 0}{(4.611,3.954)}
\gppoint{gp mark 0}{(4.611,3.759)}
\gppoint{gp mark 0}{(4.611,4.236)}
\gppoint{gp mark 0}{(4.611,4.594)}
\gppoint{gp mark 0}{(4.611,4.132)}
\gppoint{gp mark 0}{(4.611,4.285)}
\gppoint{gp mark 0}{(4.611,4.475)}
\gppoint{gp mark 0}{(4.611,3.954)}
\gppoint{gp mark 0}{(4.611,3.778)}
\gppoint{gp mark 0}{(4.611,4.334)}
\gppoint{gp mark 0}{(4.611,4.539)}
\gppoint{gp mark 0}{(4.611,4.046)}
\gppoint{gp mark 0}{(4.611,3.864)}
\gppoint{gp mark 0}{(4.611,4.293)}
\gppoint{gp mark 0}{(4.611,4.213)}
\gppoint{gp mark 0}{(4.611,4.322)}
\gppoint{gp mark 0}{(4.611,4.137)}
\gppoint{gp mark 0}{(4.611,4.391)}
\gppoint{gp mark 0}{(4.611,4.536)}
\gppoint{gp mark 0}{(4.611,4.346)}
\gppoint{gp mark 0}{(4.611,4.346)}
\gppoint{gp mark 0}{(4.611,4.121)}
\gppoint{gp mark 0}{(4.611,4.064)}
\gppoint{gp mark 0}{(4.611,3.940)}
\gppoint{gp mark 0}{(4.611,4.408)}
\gppoint{gp mark 0}{(4.611,4.536)}
\gppoint{gp mark 0}{(4.611,3.634)}
\gppoint{gp mark 0}{(4.611,3.968)}
\gppoint{gp mark 0}{(4.611,3.954)}
\gppoint{gp mark 0}{(4.611,3.831)}
\gppoint{gp mark 0}{(4.611,4.137)}
\gppoint{gp mark 0}{(4.611,3.847)}
\gppoint{gp mark 0}{(4.611,3.689)}
\gppoint{gp mark 0}{(4.611,4.021)}
\gppoint{gp mark 0}{(4.611,4.708)}
\gppoint{gp mark 0}{(4.611,4.198)}
\gppoint{gp mark 0}{(4.611,4.093)}
\gppoint{gp mark 0}{(4.611,4.254)}
\gppoint{gp mark 0}{(4.611,4.008)}
\gppoint{gp mark 0}{(4.611,4.148)}
\gppoint{gp mark 0}{(4.611,3.699)}
\gppoint{gp mark 0}{(4.611,4.127)}
\gppoint{gp mark 0}{(4.611,3.895)}
\gppoint{gp mark 0}{(4.611,4.798)}
\gppoint{gp mark 0}{(4.611,3.995)}
\gppoint{gp mark 0}{(4.611,3.276)}
\gppoint{gp mark 0}{(4.611,3.351)}
\gppoint{gp mark 0}{(4.611,4.153)}
\gppoint{gp mark 0}{(4.611,3.995)}
\gppoint{gp mark 0}{(4.611,4.638)}
\gppoint{gp mark 0}{(4.639,4.669)}
\gppoint{gp mark 0}{(4.639,4.127)}
\gppoint{gp mark 0}{(4.639,3.839)}
\gppoint{gp mark 0}{(4.639,4.105)}
\gppoint{gp mark 0}{(4.639,3.730)}
\gppoint{gp mark 0}{(4.639,4.179)}
\gppoint{gp mark 0}{(4.639,4.754)}
\gppoint{gp mark 0}{(4.639,4.925)}
\gppoint{gp mark 0}{(4.639,3.778)}
\gppoint{gp mark 0}{(4.639,4.383)}
\gppoint{gp mark 0}{(4.639,4.169)}
\gppoint{gp mark 0}{(4.639,4.446)}
\gppoint{gp mark 0}{(4.639,4.046)}
\gppoint{gp mark 0}{(4.639,3.645)}
\gppoint{gp mark 0}{(4.639,3.839)}
\gppoint{gp mark 0}{(4.639,4.064)}
\gppoint{gp mark 0}{(4.639,4.408)}
\gppoint{gp mark 0}{(4.639,3.982)}
\gppoint{gp mark 0}{(4.639,3.933)}
\gppoint{gp mark 0}{(4.639,3.610)}
\gppoint{gp mark 0}{(4.639,3.995)}
\gppoint{gp mark 0}{(4.639,4.550)}
\gppoint{gp mark 0}{(4.639,4.034)}
\gppoint{gp mark 0}{(4.639,4.469)}
\gppoint{gp mark 0}{(4.639,4.021)}
\gppoint{gp mark 0}{(4.639,4.198)}
\gppoint{gp mark 0}{(4.639,4.550)}
\gppoint{gp mark 0}{(4.639,3.656)}
\gppoint{gp mark 0}{(4.639,3.982)}
\gppoint{gp mark 0}{(4.639,4.263)}
\gppoint{gp mark 0}{(4.639,4.376)}
\gppoint{gp mark 0}{(4.639,4.132)}
\gppoint{gp mark 0}{(4.639,4.398)}
\gppoint{gp mark 0}{(4.639,3.710)}
\gppoint{gp mark 0}{(4.639,4.669)}
\gppoint{gp mark 0}{(4.639,3.710)}
\gppoint{gp mark 0}{(4.639,4.052)}
\gppoint{gp mark 0}{(4.639,3.787)}
\gppoint{gp mark 0}{(4.639,4.873)}
\gppoint{gp mark 0}{(4.639,4.562)}
\gppoint{gp mark 0}{(4.639,3.535)}
\gppoint{gp mark 0}{(4.639,3.610)}
\gppoint{gp mark 0}{(4.639,3.831)}
\gppoint{gp mark 0}{(4.639,4.028)}
\gppoint{gp mark 0}{(4.639,4.456)}
\gppoint{gp mark 0}{(4.639,3.839)}
\gppoint{gp mark 0}{(4.639,3.872)}
\gppoint{gp mark 0}{(4.639,4.539)}
\gppoint{gp mark 0}{(4.639,4.459)}
\gppoint{gp mark 0}{(4.639,4.093)}
\gppoint{gp mark 0}{(4.639,4.263)}
\gppoint{gp mark 0}{(4.639,4.263)}
\gppoint{gp mark 0}{(4.639,4.070)}
\gppoint{gp mark 0}{(4.639,4.584)}
\gppoint{gp mark 0}{(4.639,4.800)}
\gppoint{gp mark 0}{(4.639,3.645)}
\gppoint{gp mark 0}{(4.639,3.548)}
\gppoint{gp mark 0}{(4.639,3.961)}
\gppoint{gp mark 0}{(4.639,3.813)}
\gppoint{gp mark 0}{(4.639,4.318)}
\gppoint{gp mark 0}{(4.639,3.975)}
\gppoint{gp mark 0}{(4.639,4.770)}
\gppoint{gp mark 0}{(4.639,4.349)}
\gppoint{gp mark 0}{(4.639,4.875)}
\gppoint{gp mark 0}{(4.639,3.610)}
\gppoint{gp mark 0}{(4.639,3.805)}
\gppoint{gp mark 0}{(4.639,3.813)}
\gppoint{gp mark 0}{(4.639,4.706)}
\gppoint{gp mark 0}{(4.639,4.046)}
\gppoint{gp mark 0}{(4.639,4.310)}
\gppoint{gp mark 0}{(4.639,3.813)}
\gppoint{gp mark 0}{(4.639,3.839)}
\gppoint{gp mark 0}{(4.639,4.143)}
\gppoint{gp mark 0}{(4.639,3.740)}
\gppoint{gp mark 0}{(4.639,4.542)}
\gppoint{gp mark 0}{(4.639,3.667)}
\gppoint{gp mark 0}{(4.639,4.046)}
\gppoint{gp mark 0}{(4.639,4.562)}
\gppoint{gp mark 0}{(4.639,4.536)}
\gppoint{gp mark 0}{(4.639,4.760)}
\gppoint{gp mark 0}{(4.639,4.046)}
\gppoint{gp mark 0}{(4.639,4.439)}
\gppoint{gp mark 0}{(4.639,4.726)}
\gppoint{gp mark 0}{(4.639,4.179)}
\gppoint{gp mark 0}{(4.639,5.228)}
\gppoint{gp mark 0}{(4.639,4.099)}
\gppoint{gp mark 0}{(4.639,3.796)}
\gppoint{gp mark 0}{(4.639,4.093)}
\gppoint{gp mark 0}{(4.639,4.046)}
\gppoint{gp mark 0}{(4.639,3.895)}
\gppoint{gp mark 0}{(4.639,3.787)}
\gppoint{gp mark 0}{(4.639,4.241)}
\gppoint{gp mark 0}{(4.639,3.975)}
\gppoint{gp mark 0}{(4.639,3.749)}
\gppoint{gp mark 0}{(4.639,4.657)}
\gppoint{gp mark 0}{(4.639,4.099)}
\gppoint{gp mark 0}{(4.639,3.864)}
\gppoint{gp mark 0}{(4.639,3.968)}
\gppoint{gp mark 0}{(4.639,4.342)}
\gppoint{gp mark 0}{(4.639,4.289)}
\gppoint{gp mark 0}{(4.639,4.093)}
\gppoint{gp mark 0}{(4.639,4.662)}
\gppoint{gp mark 0}{(4.639,3.796)}
\gppoint{gp mark 0}{(4.639,4.665)}
\gppoint{gp mark 0}{(4.639,3.839)}
\gppoint{gp mark 0}{(4.639,3.740)}
\gppoint{gp mark 0}{(4.639,3.975)}
\gppoint{gp mark 0}{(4.639,4.153)}
\gppoint{gp mark 0}{(4.639,4.361)}
\gppoint{gp mark 0}{(4.639,3.778)}
\gppoint{gp mark 0}{(4.639,3.864)}
\gppoint{gp mark 0}{(4.639,4.387)}
\gppoint{gp mark 0}{(4.639,4.093)}
\gppoint{gp mark 0}{(4.639,4.864)}
\gppoint{gp mark 0}{(4.639,4.864)}
\gppoint{gp mark 0}{(4.639,3.573)}
\gppoint{gp mark 0}{(4.639,4.449)}
\gppoint{gp mark 0}{(4.639,3.720)}
\gppoint{gp mark 0}{(4.639,4.164)}
\gppoint{gp mark 0}{(4.639,4.241)}
\gppoint{gp mark 0}{(4.639,4.263)}
\gppoint{gp mark 0}{(4.639,4.194)}
\gppoint{gp mark 0}{(4.639,4.453)}
\gppoint{gp mark 0}{(4.639,4.198)}
\gppoint{gp mark 0}{(4.639,4.099)}
\gppoint{gp mark 0}{(4.639,4.280)}
\gppoint{gp mark 0}{(4.639,4.099)}
\gppoint{gp mark 0}{(4.639,3.903)}
\gppoint{gp mark 0}{(4.639,4.008)}
\gppoint{gp mark 0}{(4.639,4.524)}
\gppoint{gp mark 0}{(4.639,3.535)}
\gppoint{gp mark 0}{(4.639,4.368)}
\gppoint{gp mark 0}{(4.639,3.813)}
\gppoint{gp mark 0}{(4.667,3.847)}
\gppoint{gp mark 0}{(4.667,4.276)}
\gppoint{gp mark 0}{(4.667,3.847)}
\gppoint{gp mark 0}{(4.667,3.847)}
\gppoint{gp mark 0}{(4.667,3.847)}
\gppoint{gp mark 0}{(4.667,4.289)}
\gppoint{gp mark 0}{(4.667,3.759)}
\gppoint{gp mark 0}{(4.667,4.179)}
\gppoint{gp mark 0}{(4.667,4.401)}
\gppoint{gp mark 0}{(4.667,4.116)}
\gppoint{gp mark 0}{(4.667,4.116)}
\gppoint{gp mark 0}{(4.667,4.116)}
\gppoint{gp mark 0}{(4.667,4.478)}
\gppoint{gp mark 0}{(4.667,3.847)}
\gppoint{gp mark 0}{(4.667,4.245)}
\gppoint{gp mark 0}{(4.667,4.326)}
\gppoint{gp mark 0}{(4.667,4.628)}
\gppoint{gp mark 0}{(4.667,4.660)}
\gppoint{gp mark 0}{(4.667,4.034)}
\gppoint{gp mark 0}{(4.667,4.419)}
\gppoint{gp mark 0}{(4.667,4.575)}
\gppoint{gp mark 0}{(4.667,4.620)}
\gppoint{gp mark 0}{(4.667,4.203)}
\gppoint{gp mark 0}{(4.667,4.623)}
\gppoint{gp mark 0}{(4.667,3.903)}
\gppoint{gp mark 0}{(4.667,3.720)}
\gppoint{gp mark 0}{(4.667,3.954)}
\gppoint{gp mark 0}{(4.667,4.169)}
\gppoint{gp mark 0}{(4.667,4.774)}
\gppoint{gp mark 0}{(4.667,4.121)}
\gppoint{gp mark 0}{(4.667,3.856)}
\gppoint{gp mark 0}{(4.667,4.387)}
\gppoint{gp mark 0}{(4.667,4.002)}
\gppoint{gp mark 0}{(4.667,4.711)}
\gppoint{gp mark 0}{(4.667,3.847)}
\gppoint{gp mark 0}{(4.667,4.762)}
\gppoint{gp mark 0}{(4.667,4.567)}
\gppoint{gp mark 0}{(4.667,4.667)}
\gppoint{gp mark 0}{(4.667,3.911)}
\gppoint{gp mark 0}{(4.667,3.903)}
\gppoint{gp mark 0}{(4.667,4.326)}
\gppoint{gp mark 0}{(4.667,4.749)}
\gppoint{gp mark 0}{(4.667,3.010)}
\gppoint{gp mark 0}{(4.667,3.961)}
\gppoint{gp mark 0}{(4.667,3.822)}
\gppoint{gp mark 0}{(4.667,4.314)}
\gppoint{gp mark 0}{(4.667,4.158)}
\gppoint{gp mark 0}{(4.667,4.297)}
\gppoint{gp mark 0}{(4.667,3.895)}
\gppoint{gp mark 0}{(4.667,4.121)}
\gppoint{gp mark 0}{(4.667,3.918)}
\gppoint{gp mark 0}{(4.667,4.832)}
\gppoint{gp mark 0}{(4.667,4.314)}
\gppoint{gp mark 0}{(4.667,3.940)}
\gppoint{gp mark 0}{(4.667,3.903)}
\gppoint{gp mark 0}{(4.667,3.872)}
\gppoint{gp mark 0}{(4.667,4.314)}
\gppoint{gp mark 0}{(4.667,4.717)}
\gppoint{gp mark 0}{(4.667,3.911)}
\gppoint{gp mark 0}{(4.667,3.933)}
\gppoint{gp mark 0}{(4.667,4.342)}
\gppoint{gp mark 0}{(4.667,3.634)}
\gppoint{gp mark 0}{(4.667,3.759)}
\gppoint{gp mark 0}{(4.667,4.137)}
\gppoint{gp mark 0}{(4.667,4.697)}
\gppoint{gp mark 0}{(4.667,4.099)}
\gppoint{gp mark 0}{(4.667,5.385)}
\gppoint{gp mark 0}{(4.667,4.314)}
\gppoint{gp mark 0}{(4.667,4.628)}
\gppoint{gp mark 0}{(4.667,4.263)}
\gppoint{gp mark 0}{(4.667,4.584)}
\gppoint{gp mark 0}{(4.667,4.394)}
\gppoint{gp mark 0}{(4.667,3.940)}
\gppoint{gp mark 0}{(4.667,3.759)}
\gppoint{gp mark 0}{(4.667,4.628)}
\gppoint{gp mark 0}{(4.667,4.650)}
\gppoint{gp mark 0}{(4.667,4.164)}
\gppoint{gp mark 0}{(4.667,3.856)}
\gppoint{gp mark 0}{(4.667,4.880)}
\gppoint{gp mark 0}{(4.667,4.208)}
\gppoint{gp mark 0}{(4.667,3.872)}
\gppoint{gp mark 0}{(4.667,3.796)}
\gppoint{gp mark 0}{(4.667,4.263)}
\gppoint{gp mark 0}{(4.667,3.864)}
\gppoint{gp mark 0}{(4.667,4.383)}
\gppoint{gp mark 0}{(4.667,4.137)}
\gppoint{gp mark 0}{(4.667,4.334)}
\gppoint{gp mark 0}{(4.667,4.143)}
\gppoint{gp mark 0}{(4.667,3.911)}
\gppoint{gp mark 0}{(4.667,4.099)}
\gppoint{gp mark 0}{(4.667,4.383)}
\gppoint{gp mark 0}{(4.667,3.720)}
\gppoint{gp mark 0}{(4.667,4.562)}
\gppoint{gp mark 0}{(4.667,4.372)}
\gppoint{gp mark 0}{(4.667,3.847)}
\gppoint{gp mark 0}{(4.667,3.864)}
\gppoint{gp mark 0}{(4.667,4.635)}
\gppoint{gp mark 0}{(4.667,4.751)}
\gppoint{gp mark 0}{(4.667,4.310)}
\gppoint{gp mark 0}{(4.667,3.856)}
\gppoint{gp mark 0}{(4.667,3.880)}
\gppoint{gp mark 0}{(4.667,3.740)}
\gppoint{gp mark 0}{(4.667,4.419)}
\gppoint{gp mark 0}{(4.667,4.391)}
\gppoint{gp mark 0}{(4.667,4.506)}
\gppoint{gp mark 0}{(4.667,4.650)}
\gppoint{gp mark 0}{(4.667,3.610)}
\gppoint{gp mark 0}{(4.667,4.232)}
\gppoint{gp mark 0}{(4.667,4.002)}
\gppoint{gp mark 0}{(4.667,3.749)}
\gppoint{gp mark 0}{(4.667,4.064)}
\gppoint{gp mark 0}{(4.667,4.213)}
\gppoint{gp mark 0}{(4.667,3.903)}
\gppoint{gp mark 0}{(4.667,3.961)}
\gppoint{gp mark 0}{(4.667,3.402)}
\gppoint{gp mark 0}{(4.667,4.539)}
\gppoint{gp mark 0}{(4.667,3.880)}
\gppoint{gp mark 0}{(4.667,3.368)}
\gppoint{gp mark 0}{(4.667,3.968)}
\gppoint{gp mark 0}{(4.667,3.720)}
\gppoint{gp mark 0}{(4.667,4.064)}
\gppoint{gp mark 0}{(4.667,3.839)}
\gppoint{gp mark 0}{(4.667,3.947)}
\gppoint{gp mark 0}{(4.667,3.720)}
\gppoint{gp mark 0}{(4.667,3.720)}
\gppoint{gp mark 0}{(4.667,3.720)}
\gppoint{gp mark 0}{(4.667,4.064)}
\gppoint{gp mark 0}{(4.667,3.787)}
\gppoint{gp mark 0}{(4.667,3.926)}
\gppoint{gp mark 0}{(4.667,3.610)}
\gppoint{gp mark 0}{(4.667,4.509)}
\gppoint{gp mark 0}{(4.667,3.989)}
\gppoint{gp mark 0}{(4.667,4.052)}
\gppoint{gp mark 0}{(4.667,3.699)}
\gppoint{gp mark 0}{(4.667,4.189)}
\gppoint{gp mark 0}{(4.667,4.189)}
\gppoint{gp mark 0}{(4.667,3.918)}
\gppoint{gp mark 0}{(4.667,3.933)}
\gppoint{gp mark 0}{(4.667,4.394)}
\gppoint{gp mark 0}{(4.667,3.805)}
\gppoint{gp mark 0}{(4.693,4.070)}
\gppoint{gp mark 0}{(4.693,4.594)}
\gppoint{gp mark 0}{(4.693,4.866)}
\gppoint{gp mark 0}{(4.693,3.856)}
\gppoint{gp mark 0}{(4.693,3.888)}
\gppoint{gp mark 0}{(4.693,3.805)}
\gppoint{gp mark 0}{(4.693,4.747)}
\gppoint{gp mark 0}{(4.693,3.805)}
\gppoint{gp mark 0}{(4.693,4.099)}
\gppoint{gp mark 0}{(4.693,5.050)}
\gppoint{gp mark 0}{(4.693,4.774)}
\gppoint{gp mark 0}{(4.693,3.831)}
\gppoint{gp mark 0}{(4.693,4.174)}
\gppoint{gp mark 0}{(4.693,3.989)}
\gppoint{gp mark 0}{(4.693,4.472)}
\gppoint{gp mark 0}{(4.693,4.116)}
\gppoint{gp mark 0}{(4.693,4.774)}
\gppoint{gp mark 0}{(4.693,4.127)}
\gppoint{gp mark 0}{(4.693,4.208)}
\gppoint{gp mark 0}{(4.693,3.933)}
\gppoint{gp mark 0}{(4.693,4.715)}
\gppoint{gp mark 0}{(4.693,4.021)}
\gppoint{gp mark 0}{(4.693,4.015)}
\gppoint{gp mark 0}{(4.693,4.503)}
\gppoint{gp mark 0}{(4.693,4.722)}
\gppoint{gp mark 0}{(4.693,4.722)}
\gppoint{gp mark 0}{(4.693,3.982)}
\gppoint{gp mark 0}{(4.693,4.699)}
\gppoint{gp mark 0}{(4.693,3.768)}
\gppoint{gp mark 0}{(4.693,3.548)}
\gppoint{gp mark 0}{(4.693,4.267)}
\gppoint{gp mark 0}{(4.693,4.436)}
\gppoint{gp mark 0}{(4.693,3.831)}
\gppoint{gp mark 0}{(4.693,4.021)}
\gppoint{gp mark 0}{(4.693,3.787)}
\gppoint{gp mark 0}{(4.693,3.778)}
\gppoint{gp mark 0}{(4.693,4.064)}
\gppoint{gp mark 0}{(4.693,4.058)}
\gppoint{gp mark 0}{(4.693,3.822)}
\gppoint{gp mark 0}{(4.693,3.759)}
\gppoint{gp mark 0}{(4.693,5.089)}
\gppoint{gp mark 0}{(4.693,3.645)}
\gppoint{gp mark 0}{(4.693,4.443)}
\gppoint{gp mark 0}{(4.693,3.759)}
\gppoint{gp mark 0}{(4.693,3.759)}
\gppoint{gp mark 0}{(4.693,3.740)}
\gppoint{gp mark 0}{(4.693,3.864)}
\gppoint{gp mark 0}{(4.693,3.880)}
\gppoint{gp mark 0}{(4.693,3.968)}
\gppoint{gp mark 0}{(4.693,3.730)}
\gppoint{gp mark 0}{(4.693,3.749)}
\gppoint{gp mark 0}{(4.693,3.968)}
\gppoint{gp mark 0}{(4.693,4.944)}
\gppoint{gp mark 0}{(4.693,3.895)}
\gppoint{gp mark 0}{(4.693,3.856)}
\gppoint{gp mark 0}{(4.693,3.656)}
\gppoint{gp mark 0}{(4.693,4.285)}
\gppoint{gp mark 0}{(4.693,4.137)}
\gppoint{gp mark 0}{(4.693,4.194)}
\gppoint{gp mark 0}{(4.693,4.615)}
\gppoint{gp mark 0}{(4.693,3.805)}
\gppoint{gp mark 0}{(4.693,4.615)}
\gppoint{gp mark 0}{(4.693,4.153)}
\gppoint{gp mark 0}{(4.693,4.158)}
\gppoint{gp mark 0}{(4.693,3.864)}
\gppoint{gp mark 0}{(4.693,3.954)}
\gppoint{gp mark 0}{(4.693,3.778)}
\gppoint{gp mark 0}{(4.693,4.408)}
\gppoint{gp mark 0}{(4.693,3.864)}
\gppoint{gp mark 0}{(4.693,4.630)}
\gppoint{gp mark 0}{(4.693,4.015)}
\gppoint{gp mark 0}{(4.693,4.208)}
\gppoint{gp mark 0}{(4.693,4.002)}
\gppoint{gp mark 0}{(4.693,3.888)}
\gppoint{gp mark 0}{(4.693,4.905)}
\gppoint{gp mark 0}{(4.693,3.689)}
\gppoint{gp mark 0}{(4.693,4.127)}
\gppoint{gp mark 0}{(4.693,4.250)}
\gppoint{gp mark 0}{(4.693,4.907)}
\gppoint{gp mark 0}{(4.693,4.218)}
\gppoint{gp mark 0}{(4.693,4.302)}
\gppoint{gp mark 0}{(4.693,3.968)}
\gppoint{gp mark 0}{(4.693,3.720)}
\gppoint{gp mark 0}{(4.693,4.070)}
\gppoint{gp mark 0}{(4.693,4.419)}
\gppoint{gp mark 0}{(4.693,4.110)}
\gppoint{gp mark 0}{(4.693,4.250)}
\gppoint{gp mark 0}{(4.693,3.598)}
\gppoint{gp mark 0}{(4.693,4.330)}
\gppoint{gp mark 0}{(4.693,4.824)}
\gppoint{gp mark 0}{(4.693,4.429)}
\gppoint{gp mark 0}{(4.693,3.903)}
\gppoint{gp mark 0}{(4.693,4.110)}
\gppoint{gp mark 0}{(4.693,4.515)}
\gppoint{gp mark 0}{(4.693,3.982)}
\gppoint{gp mark 0}{(4.693,3.778)}
\gppoint{gp mark 0}{(4.693,3.926)}
\gppoint{gp mark 0}{(4.693,4.503)}
\gppoint{gp mark 0}{(4.693,4.728)}
\gppoint{gp mark 0}{(4.693,4.638)}
\gppoint{gp mark 0}{(4.693,3.678)}
\gppoint{gp mark 0}{(4.693,4.592)}
\gppoint{gp mark 0}{(4.693,3.759)}
\gppoint{gp mark 0}{(4.693,3.805)}
\gppoint{gp mark 0}{(4.693,4.405)}
\gppoint{gp mark 0}{(4.693,4.070)}
\gppoint{gp mark 0}{(4.693,4.127)}
\gppoint{gp mark 0}{(4.693,4.164)}
\gppoint{gp mark 0}{(4.693,3.880)}
\gppoint{gp mark 0}{(4.693,3.610)}
\gppoint{gp mark 0}{(4.693,4.581)}
\gppoint{gp mark 0}{(4.693,4.194)}
\gppoint{gp mark 0}{(4.693,3.645)}
\gppoint{gp mark 0}{(4.693,4.222)}
\gppoint{gp mark 0}{(4.693,4.330)}
\gppoint{gp mark 0}{(4.693,4.798)}
\gppoint{gp mark 0}{(4.693,4.121)}
\gppoint{gp mark 0}{(4.693,4.330)}
\gppoint{gp mark 0}{(4.693,4.218)}
\gppoint{gp mark 0}{(4.693,4.597)}
\gppoint{gp mark 0}{(4.693,3.720)}
\gppoint{gp mark 0}{(4.693,4.184)}
\gppoint{gp mark 0}{(4.693,4.349)}
\gppoint{gp mark 0}{(4.693,4.405)}
\gppoint{gp mark 0}{(4.693,4.726)}
\gppoint{gp mark 0}{(4.693,4.302)}
\gppoint{gp mark 0}{(4.693,4.726)}
\gppoint{gp mark 0}{(4.693,3.947)}
\gppoint{gp mark 0}{(4.693,3.989)}
\gppoint{gp mark 0}{(4.719,3.954)}
\gppoint{gp mark 0}{(4.719,3.831)}
\gppoint{gp mark 0}{(4.719,3.933)}
\gppoint{gp mark 0}{(4.719,3.831)}
\gppoint{gp mark 0}{(4.719,4.143)}
\gppoint{gp mark 0}{(4.719,3.831)}
\gppoint{gp mark 0}{(4.719,4.245)}
\gppoint{gp mark 0}{(4.719,3.911)}
\gppoint{gp mark 0}{(4.719,4.330)}
\gppoint{gp mark 0}{(4.719,4.008)}
\gppoint{gp mark 0}{(4.719,4.272)}
\gppoint{gp mark 0}{(4.719,3.888)}
\gppoint{gp mark 0}{(4.719,3.634)}
\gppoint{gp mark 0}{(4.719,4.272)}
\gppoint{gp mark 0}{(4.719,4.015)}
\gppoint{gp mark 0}{(4.719,3.911)}
\gppoint{gp mark 0}{(4.719,4.550)}
\gppoint{gp mark 0}{(4.719,4.536)}
\gppoint{gp mark 0}{(4.719,3.995)}
\gppoint{gp mark 0}{(4.719,4.330)}
\gppoint{gp mark 0}{(4.719,3.759)}
\gppoint{gp mark 0}{(4.719,4.015)}
\gppoint{gp mark 0}{(4.719,4.845)}
\gppoint{gp mark 0}{(4.719,4.845)}
\gppoint{gp mark 0}{(4.719,4.148)}
\gppoint{gp mark 0}{(4.719,4.213)}
\gppoint{gp mark 0}{(4.719,3.831)}
\gppoint{gp mark 0}{(4.719,3.610)}
\gppoint{gp mark 0}{(4.719,4.198)}
\gppoint{gp mark 0}{(4.719,4.466)}
\gppoint{gp mark 0}{(4.719,4.524)}
\gppoint{gp mark 0}{(4.719,4.272)}
\gppoint{gp mark 0}{(4.719,4.530)}
\gppoint{gp mark 0}{(4.719,4.786)}
\gppoint{gp mark 0}{(4.719,4.002)}
\gppoint{gp mark 0}{(4.719,3.872)}
\gppoint{gp mark 0}{(4.719,4.272)}
\gppoint{gp mark 0}{(4.719,3.813)}
\gppoint{gp mark 0}{(4.719,4.527)}
\gppoint{gp mark 0}{(4.719,4.002)}
\gppoint{gp mark 0}{(4.719,3.847)}
\gppoint{gp mark 0}{(4.719,4.433)}
\gppoint{gp mark 0}{(4.719,4.433)}
\gppoint{gp mark 0}{(4.719,3.961)}
\gppoint{gp mark 0}{(4.719,4.578)}
\gppoint{gp mark 0}{(4.719,4.330)}
\gppoint{gp mark 0}{(4.719,4.383)}
\gppoint{gp mark 0}{(4.719,3.787)}
\gppoint{gp mark 0}{(4.719,4.289)}
\gppoint{gp mark 0}{(4.719,4.749)}
\gppoint{gp mark 0}{(4.719,4.330)}
\gppoint{gp mark 0}{(4.719,4.724)}
\gppoint{gp mark 0}{(4.719,4.263)}
\gppoint{gp mark 0}{(4.719,3.903)}
\gppoint{gp mark 0}{(4.719,4.046)}
\gppoint{gp mark 0}{(4.719,3.895)}
\gppoint{gp mark 0}{(4.719,3.847)}
\gppoint{gp mark 0}{(4.719,4.515)}
\gppoint{gp mark 0}{(4.719,3.805)}
\gppoint{gp mark 0}{(4.719,4.105)}
\gppoint{gp mark 0}{(4.719,4.459)}
\gppoint{gp mark 0}{(4.719,4.008)}
\gppoint{gp mark 0}{(4.719,4.137)}
\gppoint{gp mark 0}{(4.719,4.436)}
\gppoint{gp mark 0}{(4.719,4.070)}
\gppoint{gp mark 0}{(4.719,4.653)}
\gppoint{gp mark 0}{(4.719,4.132)}
\gppoint{gp mark 0}{(4.719,5.069)}
\gppoint{gp mark 0}{(4.719,4.314)}
\gppoint{gp mark 0}{(4.719,3.839)}
\gppoint{gp mark 0}{(4.719,3.730)}
\gppoint{gp mark 0}{(4.719,4.326)}
\gppoint{gp mark 0}{(4.719,4.002)}
\gppoint{gp mark 0}{(4.719,3.961)}
\gppoint{gp mark 0}{(4.719,5.369)}
\gppoint{gp mark 0}{(4.719,4.153)}
\gppoint{gp mark 0}{(4.719,4.419)}
\gppoint{gp mark 0}{(4.719,4.436)}
\gppoint{gp mark 0}{(4.719,4.674)}
\gppoint{gp mark 0}{(4.719,4.285)}
\gppoint{gp mark 0}{(4.719,4.105)}
\gppoint{gp mark 0}{(4.719,3.872)}
\gppoint{gp mark 0}{(4.719,4.954)}
\gppoint{gp mark 0}{(4.719,3.778)}
\gppoint{gp mark 0}{(4.719,4.780)}
\gppoint{gp mark 0}{(4.719,4.821)}
\gppoint{gp mark 0}{(4.719,4.730)}
\gppoint{gp mark 0}{(4.719,5.069)}
\gppoint{gp mark 0}{(4.719,4.482)}
\gppoint{gp mark 0}{(4.719,4.426)}
\gppoint{gp mark 0}{(4.719,4.236)}
\gppoint{gp mark 0}{(4.719,4.334)}
\gppoint{gp mark 0}{(4.719,3.740)}
\gppoint{gp mark 0}{(4.719,4.804)}
\gppoint{gp mark 0}{(4.719,4.536)}
\gppoint{gp mark 0}{(4.719,4.110)}
\gppoint{gp mark 0}{(4.719,4.536)}
\gppoint{gp mark 0}{(4.719,4.236)}
\gppoint{gp mark 0}{(4.719,4.121)}
\gppoint{gp mark 0}{(4.719,4.272)}
\gppoint{gp mark 0}{(4.719,4.267)}
\gppoint{gp mark 0}{(4.719,4.213)}
\gppoint{gp mark 0}{(4.719,4.706)}
\gppoint{gp mark 0}{(4.719,4.545)}
\gppoint{gp mark 0}{(4.719,3.961)}
\gppoint{gp mark 0}{(4.719,4.082)}
\gppoint{gp mark 0}{(4.719,4.064)}
\gppoint{gp mark 0}{(4.719,4.314)}
\gppoint{gp mark 0}{(4.719,3.778)}
\gppoint{gp mark 0}{(4.719,3.839)}
\gppoint{gp mark 0}{(4.719,3.805)}
\gppoint{gp mark 0}{(4.719,4.380)}
\gppoint{gp mark 0}{(4.719,4.545)}
\gppoint{gp mark 0}{(4.719,4.845)}
\gppoint{gp mark 0}{(4.719,4.310)}
\gppoint{gp mark 0}{(4.719,4.121)}
\gppoint{gp mark 0}{(4.719,4.137)}
\gppoint{gp mark 0}{(4.719,4.545)}
\gppoint{gp mark 0}{(4.719,4.567)}
\gppoint{gp mark 0}{(4.719,4.046)}
\gppoint{gp mark 0}{(4.719,4.143)}
\gppoint{gp mark 0}{(4.719,4.401)}
\gppoint{gp mark 0}{(4.719,3.822)}
\gppoint{gp mark 0}{(4.719,4.545)}
\gppoint{gp mark 0}{(4.719,4.250)}
\gppoint{gp mark 0}{(4.719,4.002)}
\gppoint{gp mark 0}{(4.719,4.121)}
\gppoint{gp mark 0}{(4.719,3.989)}
\gppoint{gp mark 0}{(4.744,3.598)}
\gppoint{gp mark 0}{(4.744,4.539)}
\gppoint{gp mark 0}{(4.744,4.380)}
\gppoint{gp mark 0}{(4.744,4.052)}
\gppoint{gp mark 0}{(4.744,4.628)}
\gppoint{gp mark 0}{(4.744,3.947)}
\gppoint{gp mark 0}{(4.744,3.888)}
\gppoint{gp mark 0}{(4.744,3.740)}
\gppoint{gp mark 0}{(4.744,4.380)}
\gppoint{gp mark 0}{(4.744,4.863)}
\gppoint{gp mark 0}{(4.744,4.052)}
\gppoint{gp mark 0}{(4.744,4.756)}
\gppoint{gp mark 0}{(4.744,4.318)}
\gppoint{gp mark 0}{(4.744,4.893)}
\gppoint{gp mark 0}{(4.744,4.082)}
\gppoint{gp mark 0}{(4.744,4.008)}
\gppoint{gp mark 0}{(4.744,4.034)}
\gppoint{gp mark 0}{(4.744,4.245)}
\gppoint{gp mark 0}{(4.744,4.539)}
\gppoint{gp mark 0}{(4.744,4.015)}
\gppoint{gp mark 0}{(4.744,4.076)}
\gppoint{gp mark 0}{(4.744,4.515)}
\gppoint{gp mark 0}{(4.744,3.813)}
\gppoint{gp mark 0}{(4.744,4.052)}
\gppoint{gp mark 0}{(4.744,4.153)}
\gppoint{gp mark 0}{(4.744,3.831)}
\gppoint{gp mark 0}{(4.744,4.034)}
\gppoint{gp mark 0}{(4.744,4.380)}
\gppoint{gp mark 0}{(4.744,4.459)}
\gppoint{gp mark 0}{(4.744,4.401)}
\gppoint{gp mark 0}{(4.744,3.995)}
\gppoint{gp mark 0}{(4.744,4.008)}
\gppoint{gp mark 0}{(4.744,4.015)}
\gppoint{gp mark 0}{(4.744,4.436)}
\gppoint{gp mark 0}{(4.744,4.539)}
\gppoint{gp mark 0}{(4.744,4.472)}
\gppoint{gp mark 0}{(4.744,3.667)}
\gppoint{gp mark 0}{(4.744,4.174)}
\gppoint{gp mark 0}{(4.744,4.415)}
\gppoint{gp mark 0}{(4.744,3.911)}
\gppoint{gp mark 0}{(4.744,4.088)}
\gppoint{gp mark 0}{(4.744,3.839)}
\gppoint{gp mark 0}{(4.744,3.940)}
\gppoint{gp mark 0}{(4.744,3.982)}
\gppoint{gp mark 0}{(4.744,4.015)}
\gppoint{gp mark 0}{(4.744,3.880)}
\gppoint{gp mark 0}{(4.744,4.015)}
\gppoint{gp mark 0}{(4.744,4.921)}
\gppoint{gp mark 0}{(4.744,4.310)}
\gppoint{gp mark 0}{(4.744,3.880)}
\gppoint{gp mark 0}{(4.744,4.198)}
\gppoint{gp mark 0}{(4.744,4.105)}
\gppoint{gp mark 0}{(4.744,4.391)}
\gppoint{gp mark 0}{(4.744,4.127)}
\gppoint{gp mark 0}{(4.744,3.961)}
\gppoint{gp mark 0}{(4.744,3.918)}
\gppoint{gp mark 0}{(4.744,4.391)}
\gppoint{gp mark 0}{(4.744,4.293)}
\gppoint{gp mark 0}{(4.744,4.581)}
\gppoint{gp mark 0}{(4.744,4.293)}
\gppoint{gp mark 0}{(4.744,4.485)}
\gppoint{gp mark 0}{(4.744,3.933)}
\gppoint{gp mark 0}{(4.744,4.506)}
\gppoint{gp mark 0}{(4.744,4.564)}
\gppoint{gp mark 0}{(4.744,4.021)}
\gppoint{gp mark 0}{(4.744,4.293)}
\gppoint{gp mark 0}{(4.744,3.805)}
\gppoint{gp mark 0}{(4.744,4.570)}
\gppoint{gp mark 0}{(4.744,4.394)}
\gppoint{gp mark 0}{(4.744,5.081)}
\gppoint{gp mark 0}{(4.744,3.856)}
\gppoint{gp mark 0}{(4.744,3.864)}
\gppoint{gp mark 0}{(4.744,4.494)}
\gppoint{gp mark 0}{(4.744,4.615)}
\gppoint{gp mark 0}{(4.744,4.446)}
\gppoint{gp mark 0}{(4.744,4.179)}
\gppoint{gp mark 0}{(4.744,3.839)}
\gppoint{gp mark 0}{(4.744,3.805)}
\gppoint{gp mark 0}{(4.744,4.426)}
\gppoint{gp mark 0}{(4.744,4.289)}
\gppoint{gp mark 0}{(4.744,3.940)}
\gppoint{gp mark 0}{(4.744,5.120)}
\gppoint{gp mark 0}{(4.744,4.912)}
\gppoint{gp mark 0}{(4.744,4.218)}
\gppoint{gp mark 0}{(4.744,3.656)}
\gppoint{gp mark 0}{(4.744,4.198)}
\gppoint{gp mark 0}{(4.744,3.787)}
\gppoint{gp mark 0}{(4.744,4.208)}
\gppoint{gp mark 0}{(4.744,4.977)}
\gppoint{gp mark 0}{(4.744,4.792)}
\gppoint{gp mark 0}{(4.744,3.822)}
\gppoint{gp mark 0}{(4.744,4.208)}
\gppoint{gp mark 0}{(4.744,4.640)}
\gppoint{gp mark 0}{(4.744,4.293)}
\gppoint{gp mark 0}{(4.744,4.208)}
\gppoint{gp mark 0}{(4.744,4.376)}
\gppoint{gp mark 0}{(4.744,4.116)}
\gppoint{gp mark 0}{(4.744,3.796)}
\gppoint{gp mark 0}{(4.744,4.772)}
\gppoint{gp mark 0}{(4.744,4.008)}
\gppoint{gp mark 0}{(4.744,4.267)}
\gppoint{gp mark 0}{(4.744,4.907)}
\gppoint{gp mark 0}{(4.744,4.008)}
\gppoint{gp mark 0}{(4.744,4.602)}
\gppoint{gp mark 0}{(4.744,4.153)}
\gppoint{gp mark 0}{(4.744,4.488)}
\gppoint{gp mark 0}{(4.744,3.995)}
\gppoint{gp mark 0}{(4.744,3.888)}
\gppoint{gp mark 0}{(4.744,4.208)}
\gppoint{gp mark 0}{(4.744,4.391)}
\gppoint{gp mark 0}{(4.744,4.064)}
\gppoint{gp mark 0}{(4.744,4.093)}
\gppoint{gp mark 0}{(4.744,4.015)}
\gppoint{gp mark 0}{(4.744,4.570)}
\gppoint{gp mark 0}{(4.744,4.717)}
\gppoint{gp mark 0}{(4.744,4.008)}
\gppoint{gp mark 0}{(4.744,4.153)}
\gppoint{gp mark 0}{(4.744,5.255)}
\gppoint{gp mark 0}{(4.744,4.453)}
\gppoint{gp mark 0}{(4.744,4.756)}
\gppoint{gp mark 0}{(4.744,4.719)}
\gppoint{gp mark 0}{(4.744,3.888)}
\gppoint{gp mark 0}{(4.744,4.314)}
\gppoint{gp mark 0}{(4.744,4.184)}
\gppoint{gp mark 0}{(4.744,3.961)}
\gppoint{gp mark 0}{(4.744,3.493)}
\gppoint{gp mark 0}{(4.744,3.872)}
\gppoint{gp mark 0}{(4.744,4.376)}
\gppoint{gp mark 0}{(4.744,3.796)}
\gppoint{gp mark 0}{(4.744,4.236)}
\gppoint{gp mark 0}{(4.744,4.625)}
\gppoint{gp mark 0}{(4.744,4.459)}
\gppoint{gp mark 0}{(4.744,4.415)}
\gppoint{gp mark 0}{(4.744,4.737)}
\gppoint{gp mark 0}{(4.768,3.947)}
\gppoint{gp mark 0}{(4.768,4.478)}
\gppoint{gp mark 0}{(4.768,4.318)}
\gppoint{gp mark 0}{(4.768,4.770)}
\gppoint{gp mark 0}{(4.768,4.989)}
\gppoint{gp mark 0}{(4.768,4.353)}
\gppoint{gp mark 0}{(4.768,4.236)}
\gppoint{gp mark 0}{(4.768,3.895)}
\gppoint{gp mark 0}{(4.768,4.034)}
\gppoint{gp mark 0}{(4.768,3.895)}
\gppoint{gp mark 0}{(4.768,3.888)}
\gppoint{gp mark 0}{(4.768,3.895)}
\gppoint{gp mark 0}{(4.768,4.657)}
\gppoint{gp mark 0}{(4.768,3.796)}
\gppoint{gp mark 0}{(4.768,4.093)}
\gppoint{gp mark 0}{(4.768,4.280)}
\gppoint{gp mark 0}{(4.768,4.383)}
\gppoint{gp mark 0}{(4.768,4.310)}
\gppoint{gp mark 0}{(4.768,5.081)}
\gppoint{gp mark 0}{(4.768,3.888)}
\gppoint{gp mark 0}{(4.768,4.302)}
\gppoint{gp mark 0}{(4.768,4.052)}
\gppoint{gp mark 0}{(4.768,4.567)}
\gppoint{gp mark 0}{(4.768,3.872)}
\gppoint{gp mark 0}{(4.768,3.847)}
\gppoint{gp mark 0}{(4.768,4.586)}
\gppoint{gp mark 0}{(4.768,4.717)}
\gppoint{gp mark 0}{(4.768,5.188)}
\gppoint{gp mark 0}{(4.768,3.796)}
\gppoint{gp mark 0}{(4.768,4.232)}
\gppoint{gp mark 0}{(4.768,4.289)}
\gppoint{gp mark 0}{(4.768,3.796)}
\gppoint{gp mark 0}{(4.768,4.401)}
\gppoint{gp mark 0}{(4.768,4.368)}
\gppoint{gp mark 0}{(4.768,4.630)}
\gppoint{gp mark 0}{(4.768,4.433)}
\gppoint{gp mark 0}{(4.768,4.052)}
\gppoint{gp mark 0}{(4.768,3.710)}
\gppoint{gp mark 0}{(4.768,3.839)}
\gppoint{gp mark 0}{(4.768,4.662)}
\gppoint{gp mark 0}{(4.768,4.749)}
\gppoint{gp mark 0}{(4.768,4.208)}
\gppoint{gp mark 0}{(4.768,3.831)}
\gppoint{gp mark 0}{(4.768,3.710)}
\gppoint{gp mark 0}{(4.768,4.660)}
\gppoint{gp mark 0}{(4.768,4.592)}
\gppoint{gp mark 0}{(4.768,3.954)}
\gppoint{gp mark 0}{(4.768,3.982)}
\gppoint{gp mark 0}{(4.768,4.512)}
\gppoint{gp mark 0}{(4.768,4.245)}
\gppoint{gp mark 0}{(4.768,5.009)}
\gppoint{gp mark 0}{(4.768,4.548)}
\gppoint{gp mark 0}{(4.768,4.693)}
\gppoint{gp mark 0}{(4.768,4.338)}
\gppoint{gp mark 0}{(4.768,4.368)}
\gppoint{gp mark 0}{(4.768,5.069)}
\gppoint{gp mark 0}{(4.768,4.052)}
\gppoint{gp mark 0}{(4.768,3.926)}
\gppoint{gp mark 0}{(4.768,4.530)}
\gppoint{gp mark 0}{(4.768,3.895)}
\gppoint{gp mark 0}{(4.768,5.373)}
\gppoint{gp mark 0}{(4.768,5.105)}
\gppoint{gp mark 0}{(4.768,3.933)}
\gppoint{gp mark 0}{(4.768,4.518)}
\gppoint{gp mark 0}{(4.768,4.326)}
\gppoint{gp mark 0}{(4.768,4.116)}
\gppoint{gp mark 0}{(4.768,4.179)}
\gppoint{gp mark 0}{(4.768,5.248)}
\gppoint{gp mark 0}{(4.768,4.650)}
\gppoint{gp mark 0}{(4.768,4.665)}
\gppoint{gp mark 0}{(4.768,4.693)}
\gppoint{gp mark 0}{(4.768,4.711)}
\gppoint{gp mark 0}{(4.768,5.084)}
\gppoint{gp mark 0}{(4.768,4.613)}
\gppoint{gp mark 0}{(4.768,4.082)}
\gppoint{gp mark 0}{(4.768,5.160)}
\gppoint{gp mark 0}{(4.768,4.338)}
\gppoint{gp mark 0}{(4.768,5.084)}
\gppoint{gp mark 0}{(4.768,4.882)}
\gppoint{gp mark 0}{(4.768,4.819)}
\gppoint{gp mark 0}{(4.768,3.493)}
\gppoint{gp mark 0}{(4.768,5.084)}
\gppoint{gp mark 0}{(4.768,3.759)}
\gppoint{gp mark 0}{(4.768,4.116)}
\gppoint{gp mark 0}{(4.768,4.116)}
\gppoint{gp mark 0}{(4.768,3.720)}
\gppoint{gp mark 0}{(4.768,4.545)}
\gppoint{gp mark 0}{(4.768,4.093)}
\gppoint{gp mark 0}{(4.768,3.645)}
\gppoint{gp mark 0}{(4.768,4.798)}
\gppoint{gp mark 0}{(4.768,4.813)}
\gppoint{gp mark 0}{(4.768,4.046)}
\gppoint{gp mark 0}{(4.768,3.903)}
\gppoint{gp mark 0}{(4.768,5.021)}
\gppoint{gp mark 0}{(4.768,3.982)}
\gppoint{gp mark 0}{(4.768,4.280)}
\gppoint{gp mark 0}{(4.768,4.021)}
\gppoint{gp mark 0}{(4.768,4.314)}
\gppoint{gp mark 0}{(4.768,3.982)}
\gppoint{gp mark 0}{(4.768,3.645)}
\gppoint{gp mark 0}{(4.768,4.575)}
\gppoint{gp mark 0}{(4.768,3.947)}
\gppoint{gp mark 0}{(4.768,3.768)}
\gppoint{gp mark 0}{(4.768,4.153)}
\gppoint{gp mark 0}{(4.768,5.024)}
\gppoint{gp mark 0}{(4.768,5.373)}
\gppoint{gp mark 0}{(4.768,3.768)}
\gppoint{gp mark 0}{(4.768,3.740)}
\gppoint{gp mark 0}{(4.768,3.645)}
\gppoint{gp mark 0}{(4.768,4.730)}
\gppoint{gp mark 0}{(4.768,4.633)}
\gppoint{gp mark 0}{(4.768,3.975)}
\gppoint{gp mark 0}{(4.768,4.127)}
\gppoint{gp mark 0}{(4.768,3.933)}
\gppoint{gp mark 0}{(4.768,4.494)}
\gppoint{gp mark 0}{(4.768,4.143)}
\gppoint{gp mark 0}{(4.768,4.259)}
\gppoint{gp mark 0}{(4.768,4.245)}
\gppoint{gp mark 0}{(4.768,4.127)}
\gppoint{gp mark 0}{(4.768,4.564)}
\gppoint{gp mark 0}{(4.768,4.058)}
\gppoint{gp mark 0}{(4.768,4.082)}
\gppoint{gp mark 0}{(4.768,4.064)}
\gppoint{gp mark 0}{(4.768,4.542)}
\gppoint{gp mark 0}{(4.768,4.394)}
\gppoint{gp mark 0}{(4.768,4.099)}
\gppoint{gp mark 0}{(4.768,4.034)}
\gppoint{gp mark 0}{(4.768,5.177)}
\gppoint{gp mark 0}{(4.768,4.693)}
\gppoint{gp mark 0}{(4.768,4.093)}
\gppoint{gp mark 0}{(4.768,4.245)}
\gppoint{gp mark 0}{(4.768,3.895)}
\gppoint{gp mark 0}{(4.768,4.500)}
\gppoint{gp mark 0}{(4.768,3.968)}
\gppoint{gp mark 0}{(4.768,3.954)}
\gppoint{gp mark 0}{(4.768,4.198)}
\gppoint{gp mark 0}{(4.768,3.968)}
\gppoint{gp mark 0}{(4.768,4.693)}
\gppoint{gp mark 0}{(4.768,3.895)}
\gppoint{gp mark 0}{(4.768,4.950)}
\gppoint{gp mark 0}{(4.768,4.876)}
\gppoint{gp mark 0}{(4.768,3.947)}
\gppoint{gp mark 0}{(4.768,4.433)}
\gppoint{gp mark 0}{(4.768,4.314)}
\gppoint{gp mark 0}{(4.768,4.346)}
\gppoint{gp mark 0}{(4.768,3.926)}
\gppoint{gp mark 0}{(4.792,3.895)}
\gppoint{gp mark 0}{(4.792,3.872)}
\gppoint{gp mark 0}{(4.792,4.034)}
\gppoint{gp mark 0}{(4.792,3.895)}
\gppoint{gp mark 0}{(4.792,3.911)}
\gppoint{gp mark 0}{(4.792,4.898)}
\gppoint{gp mark 0}{(4.792,4.630)}
\gppoint{gp mark 0}{(4.792,3.989)}
\gppoint{gp mark 0}{(4.792,3.622)}
\gppoint{gp mark 0}{(4.792,4.780)}
\gppoint{gp mark 0}{(4.792,4.034)}
\gppoint{gp mark 0}{(4.792,4.353)}
\gppoint{gp mark 0}{(4.792,4.263)}
\gppoint{gp mark 0}{(4.792,4.408)}
\gppoint{gp mark 0}{(4.792,4.280)}
\gppoint{gp mark 0}{(4.792,3.961)}
\gppoint{gp mark 0}{(4.792,4.669)}
\gppoint{gp mark 0}{(4.792,4.148)}
\gppoint{gp mark 0}{(4.792,4.259)}
\gppoint{gp mark 0}{(4.792,3.872)}
\gppoint{gp mark 0}{(4.792,3.895)}
\gppoint{gp mark 0}{(4.792,5.192)}
\gppoint{gp mark 0}{(4.792,4.132)}
\gppoint{gp mark 0}{(4.792,4.314)}
\gppoint{gp mark 0}{(4.792,4.730)}
\gppoint{gp mark 0}{(4.792,3.968)}
\gppoint{gp mark 0}{(4.792,3.982)}
\gppoint{gp mark 0}{(4.792,4.236)}
\gppoint{gp mark 0}{(4.792,4.189)}
\gppoint{gp mark 0}{(4.792,4.148)}
\gppoint{gp mark 0}{(4.792,3.933)}
\gppoint{gp mark 0}{(4.792,4.728)}
\gppoint{gp mark 0}{(4.792,4.387)}
\gppoint{gp mark 0}{(4.792,4.841)}
\gppoint{gp mark 0}{(4.792,4.361)}
\gppoint{gp mark 0}{(4.792,4.070)}
\gppoint{gp mark 0}{(4.792,4.245)}
\gppoint{gp mark 0}{(4.792,4.342)}
\gppoint{gp mark 0}{(4.792,4.697)}
\gppoint{gp mark 0}{(4.792,4.453)}
\gppoint{gp mark 0}{(4.792,4.527)}
\gppoint{gp mark 0}{(4.792,3.730)}
\gppoint{gp mark 0}{(4.792,4.322)}
\gppoint{gp mark 0}{(4.792,4.099)}
\gppoint{gp mark 0}{(4.792,3.822)}
\gppoint{gp mark 0}{(4.792,4.034)}
\gppoint{gp mark 0}{(4.792,3.864)}
\gppoint{gp mark 0}{(4.792,3.933)}
\gppoint{gp mark 0}{(4.792,4.116)}
\gppoint{gp mark 0}{(4.792,4.880)}
\gppoint{gp mark 0}{(4.792,4.254)}
\gppoint{gp mark 0}{(4.792,4.194)}
\gppoint{gp mark 0}{(4.792,3.831)}
\gppoint{gp mark 0}{(4.792,4.338)}
\gppoint{gp mark 0}{(4.792,3.813)}
\gppoint{gp mark 0}{(4.792,3.888)}
\gppoint{gp mark 0}{(4.792,4.318)}
\gppoint{gp mark 0}{(4.792,4.040)}
\gppoint{gp mark 0}{(4.792,3.749)}
\gppoint{gp mark 0}{(4.792,4.469)}
\gppoint{gp mark 0}{(4.792,5.023)}
\gppoint{gp mark 0}{(4.792,4.695)}
\gppoint{gp mark 0}{(4.792,5.194)}
\gppoint{gp mark 0}{(4.792,4.500)}
\gppoint{gp mark 0}{(4.792,4.786)}
\gppoint{gp mark 0}{(4.792,4.613)}
\gppoint{gp mark 0}{(4.792,4.158)}
\gppoint{gp mark 0}{(4.792,4.575)}
\gppoint{gp mark 0}{(4.792,4.121)}
\gppoint{gp mark 0}{(4.792,3.689)}
\gppoint{gp mark 0}{(4.792,4.597)}
\gppoint{gp mark 0}{(4.792,4.786)}
\gppoint{gp mark 0}{(4.792,4.762)}
\gppoint{gp mark 0}{(4.792,4.002)}
\gppoint{gp mark 0}{(4.792,4.153)}
\gppoint{gp mark 0}{(4.792,5.206)}
\gppoint{gp mark 0}{(4.792,3.864)}
\gppoint{gp mark 0}{(4.792,3.995)}
\gppoint{gp mark 0}{(4.792,4.717)}
\gppoint{gp mark 0}{(4.792,4.491)}
\gppoint{gp mark 0}{(4.792,4.189)}
\gppoint{gp mark 0}{(4.792,3.813)}
\gppoint{gp mark 0}{(4.792,4.236)}
\gppoint{gp mark 0}{(4.792,4.306)}
\gppoint{gp mark 0}{(4.792,3.895)}
\gppoint{gp mark 0}{(4.792,4.805)}
\gppoint{gp mark 0}{(4.792,4.786)}
\gppoint{gp mark 0}{(4.792,4.046)}
\gppoint{gp mark 0}{(4.792,4.926)}
\gppoint{gp mark 0}{(4.792,4.488)}
\gppoint{gp mark 0}{(4.792,4.845)}
\gppoint{gp mark 0}{(4.792,4.524)}
\gppoint{gp mark 0}{(4.792,4.426)}
\gppoint{gp mark 0}{(4.792,4.749)}
\gppoint{gp mark 0}{(4.792,4.306)}
\gppoint{gp mark 0}{(4.792,4.076)}
\gppoint{gp mark 0}{(4.792,5.125)}
\gppoint{gp mark 0}{(4.792,4.676)}
\gppoint{gp mark 0}{(4.792,4.567)}
\gppoint{gp mark 0}{(4.792,3.975)}
\gppoint{gp mark 0}{(4.792,4.443)}
\gppoint{gp mark 0}{(4.792,3.975)}
\gppoint{gp mark 0}{(4.792,4.567)}
\gppoint{gp mark 0}{(4.792,4.760)}
\gppoint{gp mark 0}{(4.792,4.088)}
\gppoint{gp mark 0}{(4.792,4.592)}
\gppoint{gp mark 0}{(4.792,4.792)}
\gppoint{gp mark 0}{(4.792,4.575)}
\gppoint{gp mark 0}{(4.792,4.711)}
\gppoint{gp mark 0}{(4.792,4.756)}
\gppoint{gp mark 0}{(4.792,3.888)}
\gppoint{gp mark 0}{(4.792,4.346)}
\gppoint{gp mark 0}{(4.792,3.610)}
\gppoint{gp mark 0}{(4.792,5.114)}
\gppoint{gp mark 0}{(4.792,4.194)}
\gppoint{gp mark 0}{(4.792,4.297)}
\gppoint{gp mark 0}{(4.792,4.174)}
\gppoint{gp mark 0}{(4.792,4.028)}
\gppoint{gp mark 0}{(4.792,4.372)}
\gppoint{gp mark 0}{(4.792,4.021)}
\gppoint{gp mark 0}{(4.792,4.153)}
\gppoint{gp mark 0}{(4.792,4.488)}
\gppoint{gp mark 0}{(4.792,4.575)}
\gppoint{gp mark 0}{(4.792,3.968)}
\gppoint{gp mark 0}{(4.792,3.730)}
\gppoint{gp mark 0}{(4.816,4.426)}
\gppoint{gp mark 0}{(4.816,4.064)}
\gppoint{gp mark 0}{(4.816,4.232)}
\gppoint{gp mark 0}{(4.816,3.940)}
\gppoint{gp mark 0}{(4.816,4.189)}
\gppoint{gp mark 0}{(4.816,4.330)}
\gppoint{gp mark 0}{(4.816,3.947)}
\gppoint{gp mark 0}{(4.816,4.533)}
\gppoint{gp mark 0}{(4.816,3.872)}
\gppoint{gp mark 0}{(4.816,4.653)}
\gppoint{gp mark 0}{(4.816,4.164)}
\gppoint{gp mark 0}{(4.816,4.046)}
\gppoint{gp mark 0}{(4.816,3.872)}
\gppoint{gp mark 0}{(4.816,5.068)}
\gppoint{gp mark 0}{(4.816,4.285)}
\gppoint{gp mark 0}{(4.816,5.162)}
\gppoint{gp mark 0}{(4.816,3.787)}
\gppoint{gp mark 0}{(4.816,3.895)}
\gppoint{gp mark 0}{(4.816,4.250)}
\gppoint{gp mark 0}{(4.816,4.232)}
\gppoint{gp mark 0}{(4.816,4.208)}
\gppoint{gp mark 0}{(4.816,4.064)}
\gppoint{gp mark 0}{(4.816,3.822)}
\gppoint{gp mark 0}{(4.816,3.961)}
\gppoint{gp mark 0}{(4.816,4.179)}
\gppoint{gp mark 0}{(4.816,4.439)}
\gppoint{gp mark 0}{(4.816,4.741)}
\gppoint{gp mark 0}{(4.816,4.088)}
\gppoint{gp mark 0}{(4.816,4.158)}
\gppoint{gp mark 0}{(4.816,4.184)}
\gppoint{gp mark 0}{(4.816,4.174)}
\gppoint{gp mark 0}{(4.816,4.179)}
\gppoint{gp mark 0}{(4.816,4.110)}
\gppoint{gp mark 0}{(4.816,4.764)}
\gppoint{gp mark 0}{(4.816,4.788)}
\gppoint{gp mark 0}{(4.816,5.287)}
\gppoint{gp mark 0}{(4.816,4.342)}
\gppoint{gp mark 0}{(4.816,4.289)}
\gppoint{gp mark 0}{(4.816,3.749)}
\gppoint{gp mark 0}{(4.816,4.218)}
\gppoint{gp mark 0}{(4.816,4.439)}
\gppoint{gp mark 0}{(4.816,4.203)}
\gppoint{gp mark 0}{(4.816,3.778)}
\gppoint{gp mark 0}{(4.816,4.232)}
\gppoint{gp mark 0}{(4.816,4.046)}
\gppoint{gp mark 0}{(4.816,4.772)}
\gppoint{gp mark 0}{(4.816,4.931)}
\gppoint{gp mark 0}{(4.816,4.002)}
\gppoint{gp mark 0}{(4.816,4.365)}
\gppoint{gp mark 0}{(4.816,4.342)}
\gppoint{gp mark 0}{(4.816,4.756)}
\gppoint{gp mark 0}{(4.816,4.706)}
\gppoint{gp mark 0}{(4.816,4.449)}
\gppoint{gp mark 0}{(4.816,4.804)}
\gppoint{gp mark 0}{(4.816,5.377)}
\gppoint{gp mark 0}{(4.816,4.189)}
\gppoint{gp mark 0}{(4.816,4.581)}
\gppoint{gp mark 0}{(4.816,4.046)}
\gppoint{gp mark 0}{(4.816,4.726)}
\gppoint{gp mark 0}{(4.816,4.046)}
\gppoint{gp mark 0}{(4.816,4.297)}
\gppoint{gp mark 0}{(4.816,4.289)}
\gppoint{gp mark 0}{(4.816,4.653)}
\gppoint{gp mark 0}{(4.816,5.334)}
\gppoint{gp mark 0}{(4.816,4.105)}
\gppoint{gp mark 0}{(4.816,4.754)}
\gppoint{gp mark 0}{(4.816,4.153)}
\gppoint{gp mark 0}{(4.816,3.831)}
\gppoint{gp mark 0}{(4.816,4.745)}
\gppoint{gp mark 0}{(4.816,3.872)}
\gppoint{gp mark 0}{(4.816,3.678)}
\gppoint{gp mark 0}{(4.816,4.093)}
\gppoint{gp mark 0}{(4.816,4.236)}
\gppoint{gp mark 0}{(4.816,4.443)}
\gppoint{gp mark 0}{(4.816,4.164)}
\gppoint{gp mark 0}{(4.816,3.678)}
\gppoint{gp mark 0}{(4.816,3.740)}
\gppoint{gp mark 0}{(4.816,4.426)}
\gppoint{gp mark 0}{(4.816,4.686)}
\gppoint{gp mark 0}{(4.816,4.653)}
\gppoint{gp mark 0}{(4.816,3.880)}
\gppoint{gp mark 0}{(4.816,4.586)}
\gppoint{gp mark 0}{(4.816,3.880)}
\gppoint{gp mark 0}{(4.816,4.559)}
\gppoint{gp mark 0}{(4.816,4.643)}
\gppoint{gp mark 0}{(4.816,4.633)}
\gppoint{gp mark 0}{(4.816,4.338)}
\gppoint{gp mark 0}{(4.816,4.132)}
\gppoint{gp mark 0}{(4.816,4.289)}
\gppoint{gp mark 0}{(4.816,4.968)}
\gppoint{gp mark 0}{(4.816,4.920)}
\gppoint{gp mark 0}{(4.816,4.764)}
\gppoint{gp mark 0}{(4.816,4.902)}
\gppoint{gp mark 0}{(4.816,4.548)}
\gppoint{gp mark 0}{(4.816,5.134)}
\gppoint{gp mark 0}{(4.816,4.361)}
\gppoint{gp mark 0}{(4.816,4.545)}
\gppoint{gp mark 0}{(4.816,4.259)}
\gppoint{gp mark 0}{(4.816,4.408)}
\gppoint{gp mark 0}{(4.816,3.995)}
\gppoint{gp mark 0}{(4.816,4.110)}
\gppoint{gp mark 0}{(4.816,4.550)}
\gppoint{gp mark 0}{(4.816,4.015)}
\gppoint{gp mark 0}{(4.816,3.864)}
\gppoint{gp mark 0}{(4.816,3.656)}
\gppoint{gp mark 0}{(4.816,4.208)}
\gppoint{gp mark 0}{(4.816,4.208)}
\gppoint{gp mark 0}{(4.816,4.208)}
\gppoint{gp mark 0}{(4.816,5.215)}
\gppoint{gp mark 0}{(4.816,5.435)}
\gppoint{gp mark 0}{(4.816,4.751)}
\gppoint{gp mark 0}{(4.816,4.028)}
\gppoint{gp mark 0}{(4.816,4.368)}
\gppoint{gp mark 0}{(4.816,4.635)}
\gppoint{gp mark 0}{(4.816,4.174)}
\gppoint{gp mark 0}{(4.816,4.174)}
\gppoint{gp mark 0}{(4.816,3.710)}
\gppoint{gp mark 0}{(4.816,4.615)}
\gppoint{gp mark 0}{(4.816,4.306)}
\gppoint{gp mark 0}{(4.816,4.082)}
\gppoint{gp mark 0}{(4.816,4.164)}
\gppoint{gp mark 0}{(4.816,4.174)}
\gppoint{gp mark 0}{(4.816,3.740)}
\gppoint{gp mark 0}{(4.816,3.888)}
\gppoint{gp mark 0}{(4.816,4.330)}
\gppoint{gp mark 0}{(4.838,4.194)}
\gppoint{gp mark 0}{(4.838,3.720)}
\gppoint{gp mark 0}{(4.838,4.189)}
\gppoint{gp mark 0}{(4.838,3.933)}
\gppoint{gp mark 0}{(4.838,3.678)}
\gppoint{gp mark 0}{(4.838,4.503)}
\gppoint{gp mark 0}{(4.838,4.446)}
\gppoint{gp mark 0}{(4.838,4.203)}
\gppoint{gp mark 0}{(4.838,4.969)}
\gppoint{gp mark 0}{(4.838,4.832)}
\gppoint{gp mark 0}{(4.838,4.203)}
\gppoint{gp mark 0}{(4.838,4.189)}
\gppoint{gp mark 0}{(4.838,4.263)}
\gppoint{gp mark 0}{(4.838,4.194)}
\gppoint{gp mark 0}{(4.838,4.708)}
\gppoint{gp mark 0}{(4.838,4.697)}
\gppoint{gp mark 0}{(4.838,4.218)}
\gppoint{gp mark 0}{(4.838,4.564)}
\gppoint{gp mark 0}{(4.838,4.500)}
\gppoint{gp mark 0}{(4.838,4.500)}
\gppoint{gp mark 0}{(4.838,3.656)}
\gppoint{gp mark 0}{(4.838,4.989)}
\gppoint{gp mark 0}{(4.838,4.254)}
\gppoint{gp mark 0}{(4.838,4.553)}
\gppoint{gp mark 0}{(4.838,3.656)}
\gppoint{gp mark 0}{(4.838,3.656)}
\gppoint{gp mark 0}{(4.838,3.787)}
\gppoint{gp mark 0}{(4.838,4.550)}
\gppoint{gp mark 0}{(4.838,3.872)}
\gppoint{gp mark 0}{(4.838,5.098)}
\gppoint{gp mark 0}{(4.838,4.203)}
\gppoint{gp mark 0}{(4.838,3.933)}
\gppoint{gp mark 0}{(4.838,3.933)}
\gppoint{gp mark 0}{(4.838,4.429)}
\gppoint{gp mark 0}{(4.838,4.189)}
\gppoint{gp mark 0}{(4.838,4.826)}
\gppoint{gp mark 0}{(4.838,4.326)}
\gppoint{gp mark 0}{(4.838,4.745)}
\gppoint{gp mark 0}{(4.838,5.088)}
\gppoint{gp mark 0}{(4.838,4.153)}
\gppoint{gp mark 0}{(4.838,5.362)}
\gppoint{gp mark 0}{(4.838,4.439)}
\gppoint{gp mark 0}{(4.838,4.826)}
\gppoint{gp mark 0}{(4.838,4.310)}
\gppoint{gp mark 0}{(4.838,4.915)}
\gppoint{gp mark 0}{(4.838,4.052)}
\gppoint{gp mark 0}{(4.838,4.034)}
\gppoint{gp mark 0}{(4.838,3.895)}
\gppoint{gp mark 0}{(4.838,4.807)}
\gppoint{gp mark 0}{(4.838,4.439)}
\gppoint{gp mark 0}{(4.838,3.847)}
\gppoint{gp mark 0}{(4.838,3.895)}
\gppoint{gp mark 0}{(4.838,3.933)}
\gppoint{gp mark 0}{(4.838,4.058)}
\gppoint{gp mark 0}{(4.838,4.310)}
\gppoint{gp mark 0}{(4.838,4.826)}
\gppoint{gp mark 0}{(4.838,4.965)}
\gppoint{gp mark 0}{(4.838,4.741)}
\gppoint{gp mark 0}{(4.838,4.933)}
\gppoint{gp mark 0}{(4.838,4.121)}
\gppoint{gp mark 0}{(4.838,4.208)}
\gppoint{gp mark 0}{(4.838,4.254)}
\gppoint{gp mark 0}{(4.838,4.443)}
\gppoint{gp mark 0}{(4.838,4.310)}
\gppoint{gp mark 0}{(4.838,4.058)}
\gppoint{gp mark 0}{(4.838,4.503)}
\gppoint{gp mark 0}{(4.838,4.821)}
\gppoint{gp mark 0}{(4.838,4.376)}
\gppoint{gp mark 0}{(4.838,4.088)}
\gppoint{gp mark 0}{(4.838,4.088)}
\gppoint{gp mark 0}{(4.838,4.250)}
\gppoint{gp mark 0}{(4.838,4.116)}
\gppoint{gp mark 0}{(4.838,3.805)}
\gppoint{gp mark 0}{(4.838,4.376)}
\gppoint{gp mark 0}{(4.838,4.921)}
\gppoint{gp mark 0}{(4.838,3.911)}
\gppoint{gp mark 0}{(4.838,3.933)}
\gppoint{gp mark 0}{(4.838,4.082)}
\gppoint{gp mark 0}{(4.838,3.903)}
\gppoint{gp mark 0}{(4.838,5.035)}
\gppoint{gp mark 0}{(4.838,4.724)}
\gppoint{gp mark 0}{(4.838,4.717)}
\gppoint{gp mark 0}{(4.838,4.254)}
\gppoint{gp mark 0}{(4.838,3.989)}
\gppoint{gp mark 0}{(4.838,4.276)}
\gppoint{gp mark 0}{(4.838,4.638)}
\gppoint{gp mark 0}{(4.838,4.945)}
\gppoint{gp mark 0}{(4.838,4.076)}
\gppoint{gp mark 0}{(4.838,4.002)}
\gppoint{gp mark 0}{(4.838,4.040)}
\gppoint{gp mark 0}{(4.838,4.116)}
\gppoint{gp mark 0}{(4.838,4.527)}
\gppoint{gp mark 0}{(4.838,4.724)}
\gppoint{gp mark 0}{(4.838,4.459)}
\gppoint{gp mark 0}{(4.838,5.027)}
\gppoint{gp mark 0}{(4.838,4.250)}
\gppoint{gp mark 0}{(4.838,3.903)}
\gppoint{gp mark 0}{(4.838,4.121)}
\gppoint{gp mark 0}{(4.838,4.589)}
\gppoint{gp mark 0}{(4.838,3.813)}
\gppoint{gp mark 0}{(4.838,4.575)}
\gppoint{gp mark 0}{(4.838,4.028)}
\gppoint{gp mark 0}{(4.838,4.724)}
\gppoint{gp mark 0}{(4.838,5.329)}
\gppoint{gp mark 0}{(4.838,3.989)}
\gppoint{gp mark 0}{(4.838,4.121)}
\gppoint{gp mark 0}{(4.838,3.911)}
\gppoint{gp mark 0}{(4.838,4.798)}
\gppoint{gp mark 0}{(4.838,4.839)}
\gppoint{gp mark 0}{(4.838,4.088)}
\gppoint{gp mark 0}{(4.838,4.179)}
\gppoint{gp mark 0}{(4.838,4.174)}
\gppoint{gp mark 0}{(4.838,5.079)}
\gppoint{gp mark 0}{(4.838,4.607)}
\gppoint{gp mark 0}{(4.838,5.252)}
\gppoint{gp mark 0}{(4.838,3.933)}
\gppoint{gp mark 0}{(4.838,5.252)}
\gppoint{gp mark 0}{(4.838,4.236)}
\gppoint{gp mark 0}{(4.838,3.010)}
\gppoint{gp mark 0}{(4.838,4.021)}
\gppoint{gp mark 0}{(4.838,4.263)}
\gppoint{gp mark 0}{(4.838,4.527)}
\gppoint{gp mark 0}{(4.838,4.179)}
\gppoint{gp mark 0}{(4.838,4.306)}
\gppoint{gp mark 0}{(4.838,5.246)}
\gppoint{gp mark 0}{(4.860,4.697)}
\gppoint{gp mark 0}{(4.860,4.439)}
\gppoint{gp mark 0}{(4.860,5.092)}
\gppoint{gp mark 0}{(4.860,4.365)}
\gppoint{gp mark 0}{(4.860,5.069)}
\gppoint{gp mark 0}{(4.860,4.259)}
\gppoint{gp mark 0}{(4.860,4.179)}
\gppoint{gp mark 0}{(4.860,5.211)}
\gppoint{gp mark 0}{(4.860,4.615)}
\gppoint{gp mark 0}{(4.860,4.436)}
\gppoint{gp mark 0}{(4.860,4.040)}
\gppoint{gp mark 0}{(4.860,4.179)}
\gppoint{gp mark 0}{(4.860,4.028)}
\gppoint{gp mark 0}{(4.860,5.381)}
\gppoint{gp mark 0}{(4.860,4.488)}
\gppoint{gp mark 0}{(4.860,4.184)}
\gppoint{gp mark 0}{(4.860,4.310)}
\gppoint{gp mark 0}{(4.860,4.310)}
\gppoint{gp mark 0}{(4.860,4.672)}
\gppoint{gp mark 0}{(4.860,4.737)}
\gppoint{gp mark 0}{(4.860,4.728)}
\gppoint{gp mark 0}{(4.860,4.121)}
\gppoint{gp mark 0}{(4.860,4.788)}
\gppoint{gp mark 0}{(4.860,4.179)}
\gppoint{gp mark 0}{(4.860,4.882)}
\gppoint{gp mark 0}{(4.860,4.570)}
\gppoint{gp mark 0}{(4.860,4.110)}
\gppoint{gp mark 0}{(4.860,4.405)}
\gppoint{gp mark 0}{(4.860,3.933)}
\gppoint{gp mark 0}{(4.860,4.058)}
\gppoint{gp mark 0}{(4.860,3.975)}
\gppoint{gp mark 0}{(4.860,4.276)}
\gppoint{gp mark 0}{(4.860,4.174)}
\gppoint{gp mark 0}{(4.860,4.929)}
\gppoint{gp mark 0}{(4.860,4.184)}
\gppoint{gp mark 0}{(4.860,4.267)}
\gppoint{gp mark 0}{(4.860,5.086)}
\gppoint{gp mark 0}{(4.860,4.088)}
\gppoint{gp mark 0}{(4.860,3.933)}
\gppoint{gp mark 0}{(4.860,5.222)}
\gppoint{gp mark 0}{(4.860,3.911)}
\gppoint{gp mark 0}{(4.860,4.584)}
\gppoint{gp mark 0}{(4.860,4.028)}
\gppoint{gp mark 0}{(4.860,4.276)}
\gppoint{gp mark 0}{(4.860,4.276)}
\gppoint{gp mark 0}{(4.860,4.213)}
\gppoint{gp mark 0}{(4.860,4.494)}
\gppoint{gp mark 0}{(4.860,4.310)}
\gppoint{gp mark 0}{(4.860,4.871)}
\gppoint{gp mark 0}{(4.860,4.280)}
\gppoint{gp mark 0}{(4.860,4.132)}
\gppoint{gp mark 0}{(4.860,4.784)}
\gppoint{gp mark 0}{(4.860,3.954)}
\gppoint{gp mark 0}{(4.860,4.280)}
\gppoint{gp mark 0}{(4.860,3.880)}
\gppoint{gp mark 0}{(4.860,3.699)}
\gppoint{gp mark 0}{(4.860,3.872)}
\gppoint{gp mark 0}{(4.860,4.015)}
\gppoint{gp mark 0}{(4.860,3.933)}
\gppoint{gp mark 0}{(4.860,4.285)}
\gppoint{gp mark 0}{(4.860,4.466)}
\gppoint{gp mark 0}{(4.860,4.796)}
\gppoint{gp mark 0}{(4.860,4.034)}
\gppoint{gp mark 0}{(4.860,4.334)}
\gppoint{gp mark 0}{(4.860,4.426)}
\gppoint{gp mark 0}{(4.860,4.099)}
\gppoint{gp mark 0}{(4.860,4.046)}
\gppoint{gp mark 0}{(4.860,4.227)}
\gppoint{gp mark 0}{(4.860,4.334)}
\gppoint{gp mark 0}{(4.860,4.334)}
\gppoint{gp mark 0}{(4.860,4.387)}
\gppoint{gp mark 0}{(4.860,4.766)}
\gppoint{gp mark 0}{(4.860,3.796)}
\gppoint{gp mark 0}{(4.860,4.263)}
\gppoint{gp mark 0}{(4.860,4.405)}
\gppoint{gp mark 0}{(4.860,4.405)}
\gppoint{gp mark 0}{(4.860,4.338)}
\gppoint{gp mark 0}{(4.860,4.615)}
\gppoint{gp mark 0}{(4.860,4.040)}
\gppoint{gp mark 0}{(4.860,4.751)}
\gppoint{gp mark 0}{(4.860,3.831)}
\gppoint{gp mark 0}{(4.860,4.132)}
\gppoint{gp mark 0}{(4.860,4.545)}
\gppoint{gp mark 0}{(4.860,4.227)}
\gppoint{gp mark 0}{(4.860,4.408)}
\gppoint{gp mark 0}{(4.860,4.408)}
\gppoint{gp mark 0}{(4.860,3.368)}
\gppoint{gp mark 0}{(4.860,4.334)}
\gppoint{gp mark 0}{(4.860,5.440)}
\gppoint{gp mark 0}{(4.860,4.391)}
\gppoint{gp mark 0}{(4.860,5.323)}
\gppoint{gp mark 0}{(4.860,3.911)}
\gppoint{gp mark 0}{(4.860,4.607)}
\gppoint{gp mark 0}{(4.860,4.040)}
\gppoint{gp mark 0}{(4.860,3.989)}
\gppoint{gp mark 0}{(4.860,4.570)}
\gppoint{gp mark 0}{(4.860,3.839)}
\gppoint{gp mark 0}{(4.860,3.975)}
\gppoint{gp mark 0}{(4.860,3.796)}
\gppoint{gp mark 0}{(4.860,4.132)}
\gppoint{gp mark 0}{(4.860,4.361)}
\gppoint{gp mark 0}{(4.860,4.208)}
\gppoint{gp mark 0}{(4.860,4.306)}
\gppoint{gp mark 0}{(4.860,4.132)}
\gppoint{gp mark 0}{(4.860,4.361)}
\gppoint{gp mark 0}{(4.860,4.334)}
\gppoint{gp mark 0}{(4.860,4.127)}
\gppoint{gp mark 0}{(4.860,4.058)}
\gppoint{gp mark 0}{(4.860,3.903)}
\gppoint{gp mark 0}{(4.860,4.394)}
\gppoint{gp mark 0}{(4.860,4.088)}
\gppoint{gp mark 0}{(4.882,4.754)}
\gppoint{gp mark 0}{(4.882,4.137)}
\gppoint{gp mark 0}{(4.882,3.813)}
\gppoint{gp mark 0}{(4.882,4.419)}
\gppoint{gp mark 0}{(4.882,4.391)}
\gppoint{gp mark 0}{(4.882,4.828)}
\gppoint{gp mark 0}{(4.882,4.040)}
\gppoint{gp mark 0}{(4.882,4.326)}
\gppoint{gp mark 0}{(4.882,4.918)}
\gppoint{gp mark 0}{(4.882,3.864)}
\gppoint{gp mark 0}{(4.882,3.864)}
\gppoint{gp mark 0}{(4.882,4.222)}
\gppoint{gp mark 0}{(4.882,4.521)}
\gppoint{gp mark 0}{(4.882,4.236)}
\gppoint{gp mark 0}{(4.882,4.600)}
\gppoint{gp mark 0}{(4.882,4.706)}
\gppoint{gp mark 0}{(4.882,4.222)}
\gppoint{gp mark 0}{(4.882,4.530)}
\gppoint{gp mark 0}{(4.882,4.539)}
\gppoint{gp mark 0}{(4.882,4.469)}
\gppoint{gp mark 0}{(4.882,4.690)}
\gppoint{gp mark 0}{(4.882,4.241)}
\gppoint{gp mark 0}{(4.882,4.040)}
\gppoint{gp mark 0}{(4.882,4.429)}
\gppoint{gp mark 0}{(4.882,3.989)}
\gppoint{gp mark 0}{(4.882,4.722)}
\gppoint{gp mark 0}{(4.882,3.847)}
\gppoint{gp mark 0}{(4.882,5.418)}
\gppoint{gp mark 0}{(4.882,4.472)}
\gppoint{gp mark 0}{(4.882,3.933)}
\gppoint{gp mark 0}{(4.882,4.713)}
\gppoint{gp mark 0}{(4.882,4.110)}
\gppoint{gp mark 0}{(4.882,4.615)}
\gppoint{gp mark 0}{(4.882,4.338)}
\gppoint{gp mark 0}{(4.882,5.331)}
\gppoint{gp mark 0}{(4.882,4.693)}
\gppoint{gp mark 0}{(4.882,4.338)}
\gppoint{gp mark 0}{(4.882,3.989)}
\gppoint{gp mark 0}{(4.882,3.720)}
\gppoint{gp mark 0}{(4.882,4.488)}
\gppoint{gp mark 0}{(4.882,4.412)}
\gppoint{gp mark 0}{(4.882,3.805)}
\gppoint{gp mark 0}{(4.882,4.690)}
\gppoint{gp mark 0}{(4.882,5.101)}
\gppoint{gp mark 0}{(4.882,4.556)}
\gppoint{gp mark 0}{(4.882,4.380)}
\gppoint{gp mark 0}{(4.882,3.720)}
\gppoint{gp mark 0}{(4.882,4.553)}
\gppoint{gp mark 0}{(4.882,4.318)}
\gppoint{gp mark 0}{(4.882,3.933)}
\gppoint{gp mark 0}{(4.882,3.947)}
\gppoint{gp mark 0}{(4.882,3.831)}
\gppoint{gp mark 0}{(4.882,4.143)}
\gppoint{gp mark 0}{(4.882,4.567)}
\gppoint{gp mark 0}{(4.882,3.872)}
\gppoint{gp mark 0}{(4.882,4.314)}
\gppoint{gp mark 0}{(4.882,4.920)}
\gppoint{gp mark 0}{(4.882,4.236)}
\gppoint{gp mark 0}{(4.882,4.052)}
\gppoint{gp mark 0}{(4.882,4.936)}
\gppoint{gp mark 0}{(4.882,4.936)}
\gppoint{gp mark 0}{(4.882,3.872)}
\gppoint{gp mark 0}{(4.882,4.110)}
\gppoint{gp mark 0}{(4.882,5.248)}
\gppoint{gp mark 0}{(4.882,4.276)}
\gppoint{gp mark 0}{(4.882,3.947)}
\gppoint{gp mark 0}{(4.882,3.933)}
\gppoint{gp mark 0}{(4.882,5.127)}
\gppoint{gp mark 0}{(4.882,4.148)}
\gppoint{gp mark 0}{(4.882,4.597)}
\gppoint{gp mark 0}{(4.882,3.010)}
\gppoint{gp mark 0}{(4.882,4.046)}
\gppoint{gp mark 0}{(4.882,4.297)}
\gppoint{gp mark 0}{(4.882,4.189)}
\gppoint{gp mark 0}{(4.882,5.293)}
\gppoint{gp mark 0}{(4.882,3.933)}
\gppoint{gp mark 0}{(4.882,4.034)}
\gppoint{gp mark 0}{(4.882,3.010)}
\gppoint{gp mark 0}{(4.882,3.847)}
\gppoint{gp mark 0}{(4.882,4.213)}
\gppoint{gp mark 0}{(4.882,5.548)}
\gppoint{gp mark 0}{(4.882,4.921)}
\gppoint{gp mark 0}{(4.882,5.548)}
\gppoint{gp mark 0}{(4.882,4.817)}
\gppoint{gp mark 0}{(4.882,4.203)}
\gppoint{gp mark 0}{(4.882,4.236)}
\gppoint{gp mark 0}{(4.882,4.034)}
\gppoint{gp mark 0}{(4.882,4.034)}
\gppoint{gp mark 0}{(4.882,4.859)}
\gppoint{gp mark 0}{(4.882,5.208)}
\gppoint{gp mark 0}{(4.882,4.475)}
\gppoint{gp mark 0}{(4.882,4.475)}
\gppoint{gp mark 0}{(4.882,4.110)}
\gppoint{gp mark 0}{(4.882,4.597)}
\gppoint{gp mark 0}{(4.882,4.475)}
\gppoint{gp mark 0}{(4.882,5.548)}
\gppoint{gp mark 0}{(4.882,4.314)}
\gppoint{gp mark 0}{(4.882,4.912)}
\gppoint{gp mark 0}{(4.882,4.857)}
\gppoint{gp mark 0}{(4.882,4.615)}
\gppoint{gp mark 0}{(4.882,3.989)}
\gppoint{gp mark 0}{(4.882,4.326)}
\gppoint{gp mark 0}{(4.903,4.921)}
\gppoint{gp mark 0}{(4.903,5.300)}
\gppoint{gp mark 0}{(4.903,4.222)}
\gppoint{gp mark 0}{(4.903,4.232)}
\gppoint{gp mark 0}{(4.903,4.613)}
\gppoint{gp mark 0}{(4.903,3.749)}
\gppoint{gp mark 0}{(4.903,4.488)}
\gppoint{gp mark 0}{(4.903,4.660)}
\gppoint{gp mark 0}{(4.903,4.923)}
\gppoint{gp mark 0}{(4.903,4.449)}
\gppoint{gp mark 0}{(4.903,5.155)}
\gppoint{gp mark 0}{(4.903,4.459)}
\gppoint{gp mark 0}{(4.903,5.028)}
\gppoint{gp mark 0}{(4.903,4.015)}
\gppoint{gp mark 0}{(4.903,4.412)}
\gppoint{gp mark 0}{(4.903,3.926)}
\gppoint{gp mark 0}{(4.903,4.357)}
\gppoint{gp mark 0}{(4.903,4.143)}
\gppoint{gp mark 0}{(4.903,3.954)}
\gppoint{gp mark 0}{(4.903,3.954)}
\gppoint{gp mark 0}{(4.903,3.961)}
\gppoint{gp mark 0}{(4.903,4.394)}
\gppoint{gp mark 0}{(4.903,4.259)}
\gppoint{gp mark 0}{(4.903,4.836)}
\gppoint{gp mark 0}{(4.903,4.276)}
\gppoint{gp mark 0}{(4.903,4.127)}
\gppoint{gp mark 0}{(4.903,5.139)}
\gppoint{gp mark 0}{(4.903,4.683)}
\gppoint{gp mark 0}{(4.903,4.148)}
\gppoint{gp mark 0}{(4.903,4.198)}
\gppoint{gp mark 0}{(4.903,4.754)}
\gppoint{gp mark 0}{(4.903,4.148)}
\gppoint{gp mark 0}{(4.903,4.008)}
\gppoint{gp mark 0}{(4.903,3.895)}
\gppoint{gp mark 0}{(4.903,4.330)}
\gppoint{gp mark 0}{(4.903,5.103)}
\gppoint{gp mark 0}{(4.903,4.683)}
\gppoint{gp mark 0}{(4.903,4.306)}
\gppoint{gp mark 0}{(4.903,4.174)}
\gppoint{gp mark 0}{(4.903,4.372)}
\gppoint{gp mark 0}{(4.903,4.318)}
\gppoint{gp mark 0}{(4.903,3.995)}
\gppoint{gp mark 0}{(4.903,5.290)}
\gppoint{gp mark 0}{(4.903,4.280)}
\gppoint{gp mark 0}{(4.903,3.796)}
\gppoint{gp mark 0}{(4.903,4.907)}
\gppoint{gp mark 0}{(4.903,4.052)}
\gppoint{gp mark 0}{(4.903,4.578)}
\gppoint{gp mark 0}{(4.903,4.338)}
\gppoint{gp mark 0}{(4.903,4.542)}
\gppoint{gp mark 0}{(4.903,4.127)}
\gppoint{gp mark 0}{(4.903,4.542)}
\gppoint{gp mark 0}{(4.903,4.326)}
\gppoint{gp mark 0}{(4.903,4.394)}
\gppoint{gp mark 0}{(4.903,4.567)}
\gppoint{gp mark 0}{(4.903,3.768)}
\gppoint{gp mark 0}{(4.903,4.179)}
\gppoint{gp mark 0}{(4.903,4.843)}
\gppoint{gp mark 0}{(4.903,4.391)}
\gppoint{gp mark 0}{(4.903,5.355)}
\gppoint{gp mark 0}{(4.903,4.174)}
\gppoint{gp mark 0}{(4.903,3.856)}
\gppoint{gp mark 0}{(4.903,5.079)}
\gppoint{gp mark 0}{(4.903,4.285)}
\gppoint{gp mark 0}{(4.903,3.740)}
\gppoint{gp mark 0}{(4.903,4.368)}
\gppoint{gp mark 0}{(4.903,4.446)}
\gppoint{gp mark 0}{(4.903,4.127)}
\gppoint{gp mark 0}{(4.903,4.387)}
\gppoint{gp mark 0}{(4.903,4.213)}
\gppoint{gp mark 0}{(4.903,4.426)}
\gppoint{gp mark 0}{(4.903,4.613)}
\gppoint{gp mark 0}{(4.903,3.918)}
\gppoint{gp mark 0}{(4.903,4.947)}
\gppoint{gp mark 0}{(4.903,4.578)}
\gppoint{gp mark 0}{(4.903,4.213)}
\gppoint{gp mark 0}{(4.903,5.020)}
\gppoint{gp mark 0}{(4.903,5.009)}
\gppoint{gp mark 0}{(4.903,4.921)}
\gppoint{gp mark 0}{(4.903,4.758)}
\gppoint{gp mark 0}{(4.903,3.947)}
\gppoint{gp mark 0}{(4.903,5.380)}
\gppoint{gp mark 0}{(4.903,5.025)}
\gppoint{gp mark 0}{(4.903,4.758)}
\gppoint{gp mark 0}{(4.903,4.433)}
\gppoint{gp mark 0}{(4.903,4.758)}
\gppoint{gp mark 0}{(4.903,4.521)}
\gppoint{gp mark 0}{(4.903,4.570)}
\gppoint{gp mark 0}{(4.903,4.093)}
\gppoint{gp mark 0}{(4.903,5.380)}
\gppoint{gp mark 0}{(4.903,4.179)}
\gppoint{gp mark 0}{(4.924,4.536)}
\gppoint{gp mark 0}{(4.924,4.148)}
\gppoint{gp mark 0}{(4.924,4.836)}
\gppoint{gp mark 0}{(4.924,4.635)}
\gppoint{gp mark 0}{(4.924,4.110)}
\gppoint{gp mark 0}{(4.924,4.391)}
\gppoint{gp mark 0}{(4.924,4.405)}
\gppoint{gp mark 0}{(4.924,5.003)}
\gppoint{gp mark 0}{(4.924,4.110)}
\gppoint{gp mark 0}{(4.924,4.951)}
\gppoint{gp mark 0}{(4.924,4.088)}
\gppoint{gp mark 0}{(4.924,4.040)}
\gppoint{gp mark 0}{(4.924,5.004)}
\gppoint{gp mark 0}{(4.924,3.847)}
\gppoint{gp mark 0}{(4.924,3.961)}
\gppoint{gp mark 0}{(4.924,4.987)}
\gppoint{gp mark 0}{(4.924,3.856)}
\gppoint{gp mark 0}{(4.924,4.772)}
\gppoint{gp mark 0}{(4.924,4.937)}
\gppoint{gp mark 0}{(4.924,4.774)}
\gppoint{gp mark 0}{(4.924,3.813)}
\gppoint{gp mark 0}{(4.924,4.236)}
\gppoint{gp mark 0}{(4.924,4.735)}
\gppoint{gp mark 0}{(4.924,3.847)}
\gppoint{gp mark 0}{(4.924,4.969)}
\gppoint{gp mark 0}{(4.924,4.854)}
\gppoint{gp mark 0}{(4.924,3.947)}
\gppoint{gp mark 0}{(4.924,4.137)}
\gppoint{gp mark 0}{(4.924,5.375)}
\gppoint{gp mark 0}{(4.924,4.597)}
\gppoint{gp mark 0}{(4.924,4.602)}
\gppoint{gp mark 0}{(4.924,4.940)}
\gppoint{gp mark 0}{(4.924,4.002)}
\gppoint{gp mark 0}{(4.924,4.918)}
\gppoint{gp mark 0}{(4.924,3.888)}
\gppoint{gp mark 0}{(4.924,4.405)}
\gppoint{gp mark 0}{(4.924,4.227)}
\gppoint{gp mark 0}{(4.924,4.314)}
\gppoint{gp mark 0}{(4.924,3.888)}
\gppoint{gp mark 0}{(4.924,4.365)}
\gppoint{gp mark 0}{(4.924,4.870)}
\gppoint{gp mark 0}{(4.924,4.870)}
\gppoint{gp mark 0}{(4.924,4.826)}
\gppoint{gp mark 0}{(4.924,4.788)}
\gppoint{gp mark 0}{(4.924,4.506)}
\gppoint{gp mark 0}{(4.924,4.870)}
\gppoint{gp mark 0}{(4.924,4.405)}
\gppoint{gp mark 0}{(4.924,4.778)}
\gppoint{gp mark 0}{(4.924,4.259)}
\gppoint{gp mark 0}{(4.924,4.189)}
\gppoint{gp mark 0}{(4.924,4.826)}
\gppoint{gp mark 0}{(4.924,4.826)}
\gppoint{gp mark 0}{(4.924,4.241)}
\gppoint{gp mark 0}{(4.924,5.004)}
\gppoint{gp mark 0}{(4.924,4.826)}
\gppoint{gp mark 0}{(4.924,4.105)}
\gppoint{gp mark 0}{(4.924,4.500)}
\gppoint{gp mark 0}{(4.924,5.016)}
\gppoint{gp mark 0}{(4.924,4.870)}
\gppoint{gp mark 0}{(4.924,4.633)}
\gppoint{gp mark 0}{(4.924,4.870)}
\gppoint{gp mark 0}{(4.924,4.870)}
\gppoint{gp mark 0}{(4.924,4.306)}
\gppoint{gp mark 0}{(4.924,4.158)}
\gppoint{gp mark 0}{(4.924,4.408)}
\gppoint{gp mark 0}{(4.924,4.466)}
\gppoint{gp mark 0}{(4.924,4.310)}
\gppoint{gp mark 0}{(4.924,4.058)}
\gppoint{gp mark 0}{(4.924,4.143)}
\gppoint{gp mark 0}{(4.924,3.995)}
\gppoint{gp mark 0}{(4.924,5.083)}
\gppoint{gp mark 0}{(4.924,3.888)}
\gppoint{gp mark 0}{(4.924,4.870)}
\gppoint{gp mark 0}{(4.924,4.870)}
\gppoint{gp mark 0}{(4.924,4.870)}
\gppoint{gp mark 0}{(4.924,4.064)}
\gppoint{gp mark 0}{(4.924,4.870)}
\gppoint{gp mark 0}{(4.924,4.870)}
\gppoint{gp mark 0}{(4.924,4.870)}
\gppoint{gp mark 0}{(4.924,4.643)}
\gppoint{gp mark 0}{(4.924,5.443)}
\gppoint{gp mark 0}{(4.924,4.203)}
\gppoint{gp mark 0}{(4.924,4.562)}
\gppoint{gp mark 0}{(4.924,4.285)}
\gppoint{gp mark 0}{(4.924,5.092)}
\gppoint{gp mark 0}{(4.924,4.203)}
\gppoint{gp mark 0}{(4.924,4.203)}
\gppoint{gp mark 0}{(4.924,4.058)}
\gppoint{gp mark 0}{(4.924,4.826)}
\gppoint{gp mark 0}{(4.924,4.189)}
\gppoint{gp mark 0}{(4.924,4.790)}
\gppoint{gp mark 0}{(4.924,3.903)}
\gppoint{gp mark 0}{(4.924,4.232)}
\gppoint{gp mark 0}{(4.924,5.277)}
\gppoint{gp mark 0}{(4.924,4.203)}
\gppoint{gp mark 0}{(4.924,4.826)}
\gppoint{gp mark 0}{(4.924,4.711)}
\gppoint{gp mark 0}{(4.924,4.986)}
\gppoint{gp mark 0}{(4.924,4.826)}
\gppoint{gp mark 0}{(4.924,4.756)}
\gppoint{gp mark 0}{(4.924,4.330)}
\gppoint{gp mark 0}{(4.924,4.706)}
\gppoint{gp mark 0}{(4.924,4.660)}
\gppoint{gp mark 0}{(4.924,4.064)}
\gppoint{gp mark 0}{(4.924,4.826)}
\gppoint{gp mark 0}{(4.924,5.223)}
\gppoint{gp mark 0}{(4.924,5.586)}
\gppoint{gp mark 0}{(4.924,4.289)}
\gppoint{gp mark 0}{(4.924,4.643)}
\gppoint{gp mark 0}{(4.924,4.776)}
\gppoint{gp mark 0}{(4.924,3.995)}
\gppoint{gp mark 0}{(4.924,4.405)}
\gppoint{gp mark 0}{(4.924,4.184)}
\gppoint{gp mark 0}{(4.944,4.093)}
\gppoint{gp mark 0}{(4.944,4.895)}
\gppoint{gp mark 0}{(4.944,4.158)}
\gppoint{gp mark 0}{(4.944,4.986)}
\gppoint{gp mark 0}{(4.944,4.975)}
\gppoint{gp mark 0}{(4.944,5.046)}
\gppoint{gp mark 0}{(4.944,4.559)}
\gppoint{gp mark 0}{(4.944,5.441)}
\gppoint{gp mark 0}{(4.944,4.553)}
\gppoint{gp mark 0}{(4.944,3.813)}
\gppoint{gp mark 0}{(4.944,4.184)}
\gppoint{gp mark 0}{(4.944,4.137)}
\gppoint{gp mark 0}{(4.944,3.787)}
\gppoint{gp mark 0}{(4.944,4.937)}
\gppoint{gp mark 0}{(4.944,5.029)}
\gppoint{gp mark 0}{(4.944,4.412)}
\gppoint{gp mark 0}{(4.944,4.772)}
\gppoint{gp mark 0}{(4.944,4.878)}
\gppoint{gp mark 0}{(4.944,4.823)}
\gppoint{gp mark 0}{(4.944,4.868)}
\gppoint{gp mark 0}{(4.944,4.772)}
\gppoint{gp mark 0}{(4.944,5.084)}
\gppoint{gp mark 0}{(4.944,5.079)}
\gppoint{gp mark 0}{(4.944,4.717)}
\gppoint{gp mark 0}{(4.944,5.097)}
\gppoint{gp mark 0}{(4.944,4.521)}
\gppoint{gp mark 0}{(4.944,4.518)}
\gppoint{gp mark 0}{(4.944,4.491)}
\gppoint{gp mark 0}{(4.944,4.993)}
\gppoint{gp mark 0}{(4.944,4.349)}
\gppoint{gp mark 0}{(4.944,3.975)}
\gppoint{gp mark 0}{(4.944,4.058)}
\gppoint{gp mark 0}{(4.944,4.429)}
\gppoint{gp mark 0}{(4.944,5.237)}
\gppoint{gp mark 0}{(4.944,4.819)}
\gppoint{gp mark 0}{(4.944,4.686)}
\gppoint{gp mark 0}{(4.944,4.910)}
\gppoint{gp mark 0}{(4.944,4.475)}
\gppoint{gp mark 0}{(4.944,4.310)}
\gppoint{gp mark 0}{(4.944,4.600)}
\gppoint{gp mark 0}{(4.944,5.353)}
\gppoint{gp mark 0}{(4.944,4.184)}
\gppoint{gp mark 0}{(4.944,4.213)}
\gppoint{gp mark 0}{(4.944,3.573)}
\gppoint{gp mark 0}{(4.944,4.353)}
\gppoint{gp mark 0}{(4.944,4.218)}
\gppoint{gp mark 0}{(4.944,5.554)}
\gppoint{gp mark 0}{(4.944,5.282)}
\gppoint{gp mark 0}{(4.944,4.594)}
\gppoint{gp mark 0}{(4.944,4.843)}
\gppoint{gp mark 0}{(4.944,4.153)}
\gppoint{gp mark 0}{(4.944,4.711)}
\gppoint{gp mark 0}{(4.944,3.847)}
\gppoint{gp mark 0}{(4.944,4.398)}
\gppoint{gp mark 0}{(4.944,4.285)}
\gppoint{gp mark 0}{(4.944,4.326)}
\gppoint{gp mark 0}{(4.944,4.064)}
\gppoint{gp mark 0}{(4.944,3.975)}
\gppoint{gp mark 0}{(4.944,4.837)}
\gppoint{gp mark 0}{(4.944,4.880)}
\gppoint{gp mark 0}{(4.944,3.995)}
\gppoint{gp mark 0}{(4.944,3.933)}
\gppoint{gp mark 0}{(4.944,4.184)}
\gppoint{gp mark 0}{(4.944,4.194)}
\gppoint{gp mark 0}{(4.944,4.405)}
\gppoint{gp mark 0}{(4.944,3.813)}
\gppoint{gp mark 0}{(4.944,4.405)}
\gppoint{gp mark 0}{(4.944,3.880)}
\gppoint{gp mark 0}{(4.944,4.419)}
\gppoint{gp mark 0}{(4.944,4.015)}
\gppoint{gp mark 0}{(4.944,4.285)}
\gppoint{gp mark 0}{(4.944,4.605)}
\gppoint{gp mark 0}{(4.944,4.730)}
\gppoint{gp mark 0}{(4.944,4.635)}
\gppoint{gp mark 0}{(4.944,3.822)}
\gppoint{gp mark 0}{(4.944,4.394)}
\gppoint{gp mark 0}{(4.944,4.855)}
\gppoint{gp mark 0}{(4.944,4.971)}
\gppoint{gp mark 0}{(4.944,4.405)}
\gppoint{gp mark 0}{(4.944,4.310)}
\gppoint{gp mark 0}{(4.944,4.602)}
\gppoint{gp mark 0}{(4.944,4.648)}
\gppoint{gp mark 0}{(4.944,4.751)}
\gppoint{gp mark 0}{(4.944,4.690)}
\gppoint{gp mark 0}{(4.944,4.330)}
\gppoint{gp mark 0}{(4.964,4.208)}
\gppoint{gp mark 0}{(4.964,4.218)}
\gppoint{gp mark 0}{(4.964,3.678)}
\gppoint{gp mark 0}{(4.964,5.034)}
\gppoint{gp mark 0}{(4.964,4.953)}
\gppoint{gp mark 0}{(4.964,4.293)}
\gppoint{gp mark 0}{(4.964,4.655)}
\gppoint{gp mark 0}{(4.964,4.880)}
\gppoint{gp mark 0}{(4.964,3.872)}
\gppoint{gp mark 0}{(4.964,4.885)}
\gppoint{gp mark 0}{(4.964,4.921)}
\gppoint{gp mark 0}{(4.964,4.164)}
\gppoint{gp mark 0}{(4.964,4.137)}
\gppoint{gp mark 0}{(4.964,4.674)}
\gppoint{gp mark 0}{(4.964,3.839)}
\gppoint{gp mark 0}{(4.964,4.179)}
\gppoint{gp mark 0}{(4.964,4.357)}
\gppoint{gp mark 0}{(4.964,4.559)}
\gppoint{gp mark 0}{(4.964,4.218)}
\gppoint{gp mark 0}{(4.964,3.740)}
\gppoint{gp mark 0}{(4.964,4.741)}
\gppoint{gp mark 0}{(4.964,4.293)}
\gppoint{gp mark 0}{(4.964,5.079)}
\gppoint{gp mark 0}{(4.964,5.093)}
\gppoint{gp mark 0}{(4.964,4.272)}
\gppoint{gp mark 0}{(4.964,5.750)}
\gppoint{gp mark 0}{(4.964,4.198)}
\gppoint{gp mark 0}{(4.964,4.218)}
\gppoint{gp mark 0}{(4.964,4.391)}
\gppoint{gp mark 0}{(4.964,4.093)}
\gppoint{gp mark 0}{(4.964,4.021)}
\gppoint{gp mark 0}{(4.964,5.338)}
\gppoint{gp mark 0}{(4.964,4.218)}
\gppoint{gp mark 0}{(4.964,4.099)}
\gppoint{gp mark 0}{(4.964,4.509)}
\gppoint{gp mark 0}{(4.964,4.533)}
\gppoint{gp mark 0}{(4.964,5.056)}
\gppoint{gp mark 0}{(4.964,4.222)}
\gppoint{gp mark 0}{(4.964,4.940)}
\gppoint{gp mark 0}{(4.964,4.542)}
\gppoint{gp mark 0}{(4.964,4.222)}
\gppoint{gp mark 0}{(4.964,4.241)}
\gppoint{gp mark 0}{(4.964,4.241)}
\gppoint{gp mark 0}{(4.964,4.469)}
\gppoint{gp mark 0}{(4.964,4.232)}
\gppoint{gp mark 0}{(4.964,4.951)}
\gppoint{gp mark 0}{(4.964,5.334)}
\gppoint{gp mark 0}{(4.964,4.739)}
\gppoint{gp mark 0}{(4.964,3.954)}
\gppoint{gp mark 0}{(4.964,4.951)}
\gppoint{gp mark 0}{(4.964,4.536)}
\gppoint{gp mark 0}{(4.964,5.038)}
\gppoint{gp mark 0}{(4.964,4.372)}
\gppoint{gp mark 0}{(4.964,5.343)}
\gppoint{gp mark 0}{(4.964,4.342)}
\gppoint{gp mark 0}{(4.964,4.169)}
\gppoint{gp mark 0}{(4.964,3.933)}
\gppoint{gp mark 0}{(4.964,4.280)}
\gppoint{gp mark 0}{(4.964,4.708)}
\gppoint{gp mark 0}{(4.964,4.285)}
\gppoint{gp mark 0}{(4.964,4.349)}
\gppoint{gp mark 0}{(4.964,4.600)}
\gppoint{gp mark 0}{(4.964,4.963)}
\gppoint{gp mark 0}{(4.964,4.963)}
\gppoint{gp mark 0}{(4.964,5.439)}
\gppoint{gp mark 0}{(4.964,4.250)}
\gppoint{gp mark 0}{(4.964,4.203)}
\gppoint{gp mark 0}{(4.964,4.227)}
\gppoint{gp mark 0}{(4.964,4.330)}
\gppoint{gp mark 0}{(4.964,4.747)}
\gppoint{gp mark 0}{(4.964,4.302)}
\gppoint{gp mark 0}{(4.964,4.446)}
\gppoint{gp mark 0}{(4.964,4.259)}
\gppoint{gp mark 0}{(4.964,4.633)}
\gppoint{gp mark 0}{(4.964,5.027)}
\gppoint{gp mark 0}{(4.964,3.926)}
\gppoint{gp mark 0}{(4.964,4.485)}
\gppoint{gp mark 0}{(4.964,4.459)}
\gppoint{gp mark 0}{(4.964,3.947)}
\gppoint{gp mark 0}{(4.964,4.459)}
\gppoint{gp mark 0}{(4.964,4.276)}
\gppoint{gp mark 0}{(4.964,3.872)}
\gppoint{gp mark 0}{(4.964,4.236)}
\gppoint{gp mark 0}{(4.964,4.070)}
\gppoint{gp mark 0}{(4.964,4.326)}
\gppoint{gp mark 0}{(4.964,3.872)}
\gppoint{gp mark 0}{(4.964,3.805)}
\gppoint{gp mark 0}{(4.983,4.433)}
\gppoint{gp mark 0}{(4.983,4.349)}
\gppoint{gp mark 0}{(4.983,4.817)}
\gppoint{gp mark 0}{(4.983,4.855)}
\gppoint{gp mark 0}{(4.983,4.645)}
\gppoint{gp mark 0}{(4.983,4.376)}
\gppoint{gp mark 0}{(4.983,4.302)}
\gppoint{gp mark 0}{(4.983,4.817)}
\gppoint{gp mark 0}{(4.983,4.222)}
\gppoint{gp mark 0}{(4.983,4.263)}
\gppoint{gp mark 0}{(4.983,4.405)}
\gppoint{gp mark 0}{(4.983,4.189)}
\gppoint{gp mark 0}{(4.983,4.521)}
\gppoint{gp mark 0}{(4.983,4.015)}
\gppoint{gp mark 0}{(4.983,5.063)}
\gppoint{gp mark 0}{(4.983,4.882)}
\gppoint{gp mark 0}{(4.983,5.132)}
\gppoint{gp mark 0}{(4.983,4.997)}
\gppoint{gp mark 0}{(4.983,5.092)}
\gppoint{gp mark 0}{(4.983,4.227)}
\gppoint{gp mark 0}{(4.983,4.408)}
\gppoint{gp mark 0}{(4.983,4.821)}
\gppoint{gp mark 0}{(4.983,4.263)}
\gppoint{gp mark 0}{(4.983,4.977)}
\gppoint{gp mark 0}{(4.983,5.165)}
\gppoint{gp mark 0}{(4.983,4.960)}
\gppoint{gp mark 0}{(4.983,4.618)}
\gppoint{gp mark 0}{(4.983,4.655)}
\gppoint{gp mark 0}{(4.983,4.184)}
\gppoint{gp mark 0}{(4.983,4.293)}
\gppoint{gp mark 0}{(4.983,4.542)}
\gppoint{gp mark 0}{(4.983,3.872)}
\gppoint{gp mark 0}{(4.983,3.933)}
\gppoint{gp mark 0}{(4.983,4.594)}
\gppoint{gp mark 0}{(4.983,4.834)}
\gppoint{gp mark 0}{(4.983,4.443)}
\gppoint{gp mark 0}{(4.983,4.232)}
\gppoint{gp mark 0}{(4.983,3.954)}
\gppoint{gp mark 0}{(4.983,5.254)}
\gppoint{gp mark 0}{(4.983,5.076)}
\gppoint{gp mark 0}{(4.983,4.297)}
\gppoint{gp mark 0}{(4.983,4.708)}
\gppoint{gp mark 0}{(4.983,4.082)}
\gppoint{gp mark 0}{(4.983,4.401)}
\gppoint{gp mark 0}{(4.983,4.597)}
\gppoint{gp mark 0}{(4.983,4.263)}
\gppoint{gp mark 0}{(4.983,4.993)}
\gppoint{gp mark 0}{(4.983,4.184)}
\gppoint{gp mark 0}{(4.983,4.349)}
\gppoint{gp mark 0}{(4.983,3.872)}
\gppoint{gp mark 0}{(4.983,5.032)}
\gppoint{gp mark 0}{(4.983,4.143)}
\gppoint{gp mark 0}{(4.983,4.944)}
\gppoint{gp mark 0}{(4.983,4.893)}
\gppoint{gp mark 0}{(4.983,3.888)}
\gppoint{gp mark 0}{(4.983,4.462)}
\gppoint{gp mark 0}{(4.983,5.044)}
\gppoint{gp mark 0}{(4.983,4.232)}
\gppoint{gp mark 0}{(4.983,4.573)}
\gppoint{gp mark 0}{(4.983,3.918)}
\gppoint{gp mark 0}{(4.983,4.931)}
\gppoint{gp mark 0}{(4.983,3.822)}
\gppoint{gp mark 0}{(4.983,4.093)}
\gppoint{gp mark 0}{(4.983,4.674)}
\gppoint{gp mark 0}{(4.983,4.310)}
\gppoint{gp mark 0}{(4.983,4.110)}
\gppoint{gp mark 0}{(4.983,5.017)}
\gppoint{gp mark 0}{(4.983,5.091)}
\gppoint{gp mark 0}{(4.983,4.088)}
\gppoint{gp mark 0}{(4.983,4.628)}
\gppoint{gp mark 0}{(4.983,4.754)}
\gppoint{gp mark 0}{(4.983,4.724)}
\gppoint{gp mark 0}{(4.983,5.017)}
\gppoint{gp mark 0}{(5.002,4.272)}
\gppoint{gp mark 0}{(5.002,4.302)}
\gppoint{gp mark 0}{(5.002,5.523)}
\gppoint{gp mark 0}{(5.002,4.245)}
\gppoint{gp mark 0}{(5.002,4.439)}
\gppoint{gp mark 0}{(5.002,4.070)}
\gppoint{gp mark 0}{(5.002,4.681)}
\gppoint{gp mark 0}{(5.002,4.088)}
\gppoint{gp mark 0}{(5.002,4.208)}
\gppoint{gp mark 0}{(5.002,4.542)}
\gppoint{gp mark 0}{(5.002,4.088)}
\gppoint{gp mark 0}{(5.002,5.157)}
\gppoint{gp mark 0}{(5.002,4.093)}
\gppoint{gp mark 0}{(5.002,4.306)}
\gppoint{gp mark 0}{(5.002,4.472)}
\gppoint{gp mark 0}{(5.002,5.011)}
\gppoint{gp mark 0}{(5.002,4.662)}
\gppoint{gp mark 0}{(5.002,4.422)}
\gppoint{gp mark 0}{(5.002,4.732)}
\gppoint{gp mark 0}{(5.002,4.602)}
\gppoint{gp mark 0}{(5.002,4.208)}
\gppoint{gp mark 0}{(5.002,5.144)}
\gppoint{gp mark 0}{(5.002,4.310)}
\gppoint{gp mark 0}{(5.002,3.872)}
\gppoint{gp mark 0}{(5.002,4.412)}
\gppoint{gp mark 0}{(5.002,4.937)}
\gppoint{gp mark 0}{(5.002,4.306)}
\gppoint{gp mark 0}{(5.002,4.213)}
\gppoint{gp mark 0}{(5.002,5.052)}
\gppoint{gp mark 0}{(5.002,4.774)}
\gppoint{gp mark 0}{(5.002,5.046)}
\gppoint{gp mark 0}{(5.002,4.469)}
\gppoint{gp mark 0}{(5.002,4.338)}
\gppoint{gp mark 0}{(5.002,3.010)}
\gppoint{gp mark 0}{(5.002,4.158)}
\gppoint{gp mark 0}{(5.002,4.227)}
\gppoint{gp mark 0}{(5.002,4.887)}
\gppoint{gp mark 0}{(5.002,4.015)}
\gppoint{gp mark 0}{(5.002,4.855)}
\gppoint{gp mark 0}{(5.002,4.093)}
\gppoint{gp mark 0}{(5.002,4.008)}
\gppoint{gp mark 0}{(5.002,4.052)}
\gppoint{gp mark 0}{(5.002,4.093)}
\gppoint{gp mark 0}{(5.002,4.405)}
\gppoint{gp mark 0}{(5.002,5.101)}
\gppoint{gp mark 0}{(5.002,4.433)}
\gppoint{gp mark 0}{(5.002,4.623)}
\gppoint{gp mark 0}{(5.002,4.334)}
\gppoint{gp mark 0}{(5.002,4.804)}
\gppoint{gp mark 0}{(5.002,5.096)}
\gppoint{gp mark 0}{(5.002,4.052)}
\gppoint{gp mark 0}{(5.002,5.092)}
\gppoint{gp mark 0}{(5.002,3.768)}
\gppoint{gp mark 0}{(5.002,4.208)}
\gppoint{gp mark 0}{(5.002,4.121)}
\gppoint{gp mark 0}{(5.002,4.127)}
\gppoint{gp mark 0}{(5.002,4.380)}
\gppoint{gp mark 0}{(5.002,4.121)}
\gppoint{gp mark 0}{(5.002,4.885)}
\gppoint{gp mark 0}{(5.002,4.521)}
\gppoint{gp mark 0}{(5.002,4.446)}
\gppoint{gp mark 0}{(5.002,4.681)}
\gppoint{gp mark 0}{(5.002,4.824)}
\gppoint{gp mark 0}{(5.002,3.872)}
\gppoint{gp mark 0}{(5.002,3.954)}
\gppoint{gp mark 0}{(5.002,4.342)}
\gppoint{gp mark 0}{(5.002,5.132)}
\gppoint{gp mark 0}{(5.002,4.015)}
\gppoint{gp mark 0}{(5.002,4.088)}
\gppoint{gp mark 0}{(5.002,4.693)}
\gppoint{gp mark 0}{(5.002,4.536)}
\gppoint{gp mark 0}{(5.002,4.276)}
\gppoint{gp mark 0}{(5.002,4.630)}
\gppoint{gp mark 0}{(5.002,4.365)}
\gppoint{gp mark 0}{(5.002,5.016)}
\gppoint{gp mark 0}{(5.002,4.660)}
\gppoint{gp mark 0}{(5.002,4.426)}
\gppoint{gp mark 0}{(5.002,4.015)}
\gppoint{gp mark 0}{(5.002,3.903)}
\gppoint{gp mark 0}{(5.002,4.645)}
\gppoint{gp mark 0}{(5.002,4.105)}
\gppoint{gp mark 0}{(5.002,4.015)}
\gppoint{gp mark 0}{(5.002,4.819)}
\gppoint{gp mark 0}{(5.002,5.465)}
\gppoint{gp mark 0}{(5.002,4.758)}
\gppoint{gp mark 0}{(5.021,4.533)}
\gppoint{gp mark 0}{(5.021,4.859)}
\gppoint{gp mark 0}{(5.021,3.995)}
\gppoint{gp mark 0}{(5.021,4.681)}
\gppoint{gp mark 0}{(5.021,4.232)}
\gppoint{gp mark 0}{(5.021,4.890)}
\gppoint{gp mark 0}{(5.021,4.760)}
\gppoint{gp mark 0}{(5.021,4.368)}
\gppoint{gp mark 0}{(5.021,4.338)}
\gppoint{gp mark 0}{(5.021,4.947)}
\gppoint{gp mark 0}{(5.021,5.480)}
\gppoint{gp mark 0}{(5.021,4.387)}
\gppoint{gp mark 0}{(5.021,4.556)}
\gppoint{gp mark 0}{(5.021,4.318)}
\gppoint{gp mark 0}{(5.021,4.699)}
\gppoint{gp mark 0}{(5.021,4.623)}
\gppoint{gp mark 0}{(5.021,4.770)}
\gppoint{gp mark 0}{(5.021,4.512)}
\gppoint{gp mark 0}{(5.021,4.944)}
\gppoint{gp mark 0}{(5.021,4.719)}
\gppoint{gp mark 0}{(5.021,4.174)}
\gppoint{gp mark 0}{(5.021,5.437)}
\gppoint{gp mark 0}{(5.021,4.655)}
\gppoint{gp mark 0}{(5.021,4.794)}
\gppoint{gp mark 0}{(5.021,4.905)}
\gppoint{gp mark 0}{(5.021,4.322)}
\gppoint{gp mark 0}{(5.021,5.617)}
\gppoint{gp mark 0}{(5.021,4.491)}
\gppoint{gp mark 0}{(5.021,5.038)}
\gppoint{gp mark 0}{(5.021,4.503)}
\gppoint{gp mark 0}{(5.021,2.914)}
\gppoint{gp mark 0}{(5.021,3.864)}
\gppoint{gp mark 0}{(5.021,3.888)}
\gppoint{gp mark 0}{(5.021,4.250)}
\gppoint{gp mark 0}{(5.021,5.167)}
\gppoint{gp mark 0}{(5.021,4.021)}
\gppoint{gp mark 0}{(5.021,4.472)}
\gppoint{gp mark 0}{(5.021,4.503)}
\gppoint{gp mark 0}{(5.021,4.794)}
\gppoint{gp mark 0}{(5.021,3.982)}
\gppoint{gp mark 0}{(5.021,5.054)}
\gppoint{gp mark 0}{(5.021,5.222)}
\gppoint{gp mark 0}{(5.021,3.010)}
\gppoint{gp mark 0}{(5.021,3.954)}
\gppoint{gp mark 0}{(5.021,4.784)}
\gppoint{gp mark 0}{(5.021,3.768)}
\gppoint{gp mark 0}{(5.021,4.683)}
\gppoint{gp mark 0}{(5.021,5.129)}
\gppoint{gp mark 0}{(5.021,5.498)}
\gppoint{gp mark 0}{(5.021,4.064)}
\gppoint{gp mark 0}{(5.021,3.968)}
\gppoint{gp mark 0}{(5.021,4.189)}
\gppoint{gp mark 0}{(5.021,4.314)}
\gppoint{gp mark 0}{(5.021,5.065)}
\gppoint{gp mark 0}{(5.021,4.807)}
\gppoint{gp mark 0}{(5.021,4.623)}
\gppoint{gp mark 0}{(5.021,4.349)}
\gppoint{gp mark 0}{(5.021,4.635)}
\gppoint{gp mark 0}{(5.021,5.056)}
\gppoint{gp mark 0}{(5.021,4.760)}
\gppoint{gp mark 0}{(5.021,4.515)}
\gppoint{gp mark 0}{(5.021,4.194)}
\gppoint{gp mark 0}{(5.021,4.768)}
\gppoint{gp mark 0}{(5.021,4.353)}
\gppoint{gp mark 0}{(5.021,4.052)}
\gppoint{gp mark 0}{(5.021,4.839)}
\gppoint{gp mark 0}{(5.021,4.194)}
\gppoint{gp mark 0}{(5.021,4.782)}
\gppoint{gp mark 0}{(5.021,4.655)}
\gppoint{gp mark 0}{(5.021,4.741)}
\gppoint{gp mark 0}{(5.021,4.556)}
\gppoint{gp mark 0}{(5.021,4.110)}
\gppoint{gp mark 0}{(5.021,4.368)}
\gppoint{gp mark 0}{(5.021,4.518)}
\gppoint{gp mark 0}{(5.021,3.982)}
\gppoint{gp mark 0}{(5.021,4.923)}
\gppoint{gp mark 0}{(5.021,4.533)}
\gppoint{gp mark 0}{(5.021,4.314)}
\gppoint{gp mark 0}{(5.021,4.784)}
\gppoint{gp mark 0}{(5.021,3.856)}
\gppoint{gp mark 0}{(5.021,4.915)}
\gppoint{gp mark 0}{(5.021,4.008)}
\gppoint{gp mark 0}{(5.021,5.034)}
\gppoint{gp mark 0}{(5.039,5.431)}
\gppoint{gp mark 0}{(5.039,4.665)}
\gppoint{gp mark 0}{(5.039,4.515)}
\gppoint{gp mark 0}{(5.039,4.227)}
\gppoint{gp mark 0}{(5.039,3.940)}
\gppoint{gp mark 0}{(5.039,3.995)}
\gppoint{gp mark 0}{(5.039,4.760)}
\gppoint{gp mark 0}{(5.039,4.648)}
\gppoint{gp mark 0}{(5.039,4.241)}
\gppoint{gp mark 0}{(5.039,4.548)}
\gppoint{gp mark 0}{(5.039,4.446)}
\gppoint{gp mark 0}{(5.039,4.667)}
\gppoint{gp mark 0}{(5.039,4.213)}
\gppoint{gp mark 0}{(5.039,4.137)}
\gppoint{gp mark 0}{(5.039,4.276)}
\gppoint{gp mark 0}{(5.039,5.123)}
\gppoint{gp mark 0}{(5.039,4.368)}
\gppoint{gp mark 0}{(5.039,4.876)}
\gppoint{gp mark 0}{(5.039,4.613)}
\gppoint{gp mark 0}{(5.039,4.034)}
\gppoint{gp mark 0}{(5.039,4.845)}
\gppoint{gp mark 0}{(5.039,4.719)}
\gppoint{gp mark 0}{(5.039,4.376)}
\gppoint{gp mark 0}{(5.039,4.592)}
\gppoint{gp mark 0}{(5.039,4.088)}
\gppoint{gp mark 0}{(5.039,4.326)}
\gppoint{gp mark 0}{(5.039,4.326)}
\gppoint{gp mark 0}{(5.039,4.236)}
\gppoint{gp mark 0}{(5.039,4.660)}
\gppoint{gp mark 0}{(5.039,4.110)}
\gppoint{gp mark 0}{(5.039,4.070)}
\gppoint{gp mark 0}{(5.039,4.485)}
\gppoint{gp mark 0}{(5.039,4.368)}
\gppoint{gp mark 0}{(5.039,4.485)}
\gppoint{gp mark 0}{(5.039,4.683)}
\gppoint{gp mark 0}{(5.039,4.189)}
\gppoint{gp mark 0}{(5.039,4.293)}
\gppoint{gp mark 0}{(5.039,4.169)}
\gppoint{gp mark 0}{(5.039,4.908)}
\gppoint{gp mark 0}{(5.039,4.972)}
\gppoint{gp mark 0}{(5.039,4.717)}
\gppoint{gp mark 0}{(5.039,4.570)}
\gppoint{gp mark 0}{(5.039,4.302)}
\gppoint{gp mark 0}{(5.039,3.911)}
\gppoint{gp mark 0}{(5.039,4.910)}
\gppoint{gp mark 0}{(5.039,5.039)}
\gppoint{gp mark 0}{(5.039,4.559)}
\gppoint{gp mark 0}{(5.039,4.236)}
\gppoint{gp mark 0}{(5.039,3.768)}
\gppoint{gp mark 0}{(5.039,4.174)}
\gppoint{gp mark 0}{(5.039,4.232)}
\gppoint{gp mark 0}{(5.039,4.741)}
\gppoint{gp mark 0}{(5.039,5.063)}
\gppoint{gp mark 0}{(5.039,4.272)}
\gppoint{gp mark 0}{(5.039,4.280)}
\gppoint{gp mark 0}{(5.039,4.426)}
\gppoint{gp mark 0}{(5.039,4.021)}
\gppoint{gp mark 0}{(5.039,4.875)}
\gppoint{gp mark 0}{(5.039,5.044)}
\gppoint{gp mark 0}{(5.039,5.011)}
\gppoint{gp mark 0}{(5.039,4.272)}
\gppoint{gp mark 0}{(5.039,4.439)}
\gppoint{gp mark 0}{(5.039,4.415)}
\gppoint{gp mark 0}{(5.057,4.349)}
\gppoint{gp mark 0}{(5.057,4.706)}
\gppoint{gp mark 0}{(5.057,4.640)}
\gppoint{gp mark 0}{(5.057,3.888)}
\gppoint{gp mark 0}{(5.057,4.792)}
\gppoint{gp mark 0}{(5.057,4.792)}
\gppoint{gp mark 0}{(5.057,5.311)}
\gppoint{gp mark 0}{(5.057,4.706)}
\gppoint{gp mark 0}{(5.057,4.028)}
\gppoint{gp mark 0}{(5.057,4.002)}
\gppoint{gp mark 0}{(5.057,5.573)}
\gppoint{gp mark 0}{(5.057,5.339)}
\gppoint{gp mark 0}{(5.057,4.726)}
\gppoint{gp mark 0}{(5.057,4.093)}
\gppoint{gp mark 0}{(5.057,4.052)}
\gppoint{gp mark 0}{(5.057,4.774)}
\gppoint{gp mark 0}{(5.057,4.966)}
\gppoint{gp mark 0}{(5.057,4.798)}
\gppoint{gp mark 0}{(5.057,4.169)}
\gppoint{gp mark 0}{(5.057,5.315)}
\gppoint{gp mark 0}{(5.057,4.433)}
\gppoint{gp mark 0}{(5.057,5.150)}
\gppoint{gp mark 0}{(5.057,4.046)}
\gppoint{gp mark 0}{(5.057,4.640)}
\gppoint{gp mark 0}{(5.057,5.150)}
\gppoint{gp mark 0}{(5.057,4.648)}
\gppoint{gp mark 0}{(5.057,4.213)}
\gppoint{gp mark 0}{(5.057,3.895)}
\gppoint{gp mark 0}{(5.057,3.940)}
\gppoint{gp mark 0}{(5.057,4.405)}
\gppoint{gp mark 0}{(5.057,4.088)}
\gppoint{gp mark 0}{(5.057,4.942)}
\gppoint{gp mark 0}{(5.057,4.737)}
\gppoint{gp mark 0}{(5.057,4.203)}
\gppoint{gp mark 0}{(5.057,4.497)}
\gppoint{gp mark 0}{(5.057,5.009)}
\gppoint{gp mark 0}{(5.057,4.688)}
\gppoint{gp mark 0}{(5.057,5.461)}
\gppoint{gp mark 0}{(5.057,4.518)}
\gppoint{gp mark 0}{(5.057,5.373)}
\gppoint{gp mark 0}{(5.057,4.232)}
\gppoint{gp mark 0}{(5.057,4.950)}
\gppoint{gp mark 0}{(5.057,4.713)}
\gppoint{gp mark 0}{(5.057,4.408)}
\gppoint{gp mark 0}{(5.057,4.798)}
\gppoint{gp mark 0}{(5.057,5.373)}
\gppoint{gp mark 0}{(5.057,4.429)}
\gppoint{gp mark 0}{(5.057,4.245)}
\gppoint{gp mark 0}{(5.057,4.913)}
\gppoint{gp mark 0}{(5.057,4.158)}
\gppoint{gp mark 0}{(5.057,4.754)}
\gppoint{gp mark 0}{(5.057,4.693)}
\gppoint{gp mark 0}{(5.057,4.143)}
\gppoint{gp mark 0}{(5.057,4.365)}
\gppoint{gp mark 0}{(5.057,4.645)}
\gppoint{gp mark 0}{(5.057,4.466)}
\gppoint{gp mark 0}{(5.057,4.667)}
\gppoint{gp mark 0}{(5.057,4.164)}
\gppoint{gp mark 0}{(5.057,4.412)}
\gppoint{gp mark 0}{(5.057,5.190)}
\gppoint{gp mark 0}{(5.057,5.201)}
\gppoint{gp mark 0}{(5.057,4.917)}
\gppoint{gp mark 0}{(5.057,4.841)}
\gppoint{gp mark 0}{(5.057,4.826)}
\gppoint{gp mark 0}{(5.057,4.419)}
\gppoint{gp mark 0}{(5.057,4.338)}
\gppoint{gp mark 0}{(5.057,4.169)}
\gppoint{gp mark 0}{(5.057,4.564)}
\gppoint{gp mark 0}{(5.057,5.006)}
\gppoint{gp mark 0}{(5.057,4.841)}
\gppoint{gp mark 0}{(5.057,4.826)}
\gppoint{gp mark 0}{(5.057,4.088)}
\gppoint{gp mark 0}{(5.057,4.310)}
\gppoint{gp mark 0}{(5.057,4.745)}
\gppoint{gp mark 0}{(5.057,4.289)}
\gppoint{gp mark 0}{(5.057,4.028)}
\gppoint{gp mark 0}{(5.057,4.650)}
\gppoint{gp mark 0}{(5.057,5.258)}
\gppoint{gp mark 0}{(5.057,4.302)}
\gppoint{gp mark 0}{(5.075,4.285)}
\gppoint{gp mark 0}{(5.075,4.708)}
\gppoint{gp mark 0}{(5.075,4.357)}
\gppoint{gp mark 0}{(5.075,4.683)}
\gppoint{gp mark 0}{(5.075,4.272)}
\gppoint{gp mark 0}{(5.075,3.903)}
\gppoint{gp mark 0}{(5.075,4.272)}
\gppoint{gp mark 0}{(5.075,3.787)}
\gppoint{gp mark 0}{(5.075,5.081)}
\gppoint{gp mark 0}{(5.075,4.232)}
\gppoint{gp mark 0}{(5.075,4.436)}
\gppoint{gp mark 0}{(5.075,5.081)}
\gppoint{gp mark 0}{(5.075,4.453)}
\gppoint{gp mark 0}{(5.075,4.422)}
\gppoint{gp mark 0}{(5.075,4.951)}
\gppoint{gp mark 0}{(5.075,4.768)}
\gppoint{gp mark 0}{(5.075,4.105)}
\gppoint{gp mark 0}{(5.075,4.768)}
\gppoint{gp mark 0}{(5.075,4.660)}
\gppoint{gp mark 0}{(5.075,4.368)}
\gppoint{gp mark 0}{(5.075,5.403)}
\gppoint{gp mark 0}{(5.075,5.146)}
\gppoint{gp mark 0}{(5.075,4.426)}
\gppoint{gp mark 0}{(5.075,4.564)}
\gppoint{gp mark 0}{(5.075,4.688)}
\gppoint{gp mark 0}{(5.075,4.509)}
\gppoint{gp mark 0}{(5.075,4.453)}
\gppoint{gp mark 0}{(5.075,5.229)}
\gppoint{gp mark 0}{(5.075,5.445)}
\gppoint{gp mark 0}{(5.075,4.353)}
\gppoint{gp mark 0}{(5.075,4.184)}
\gppoint{gp mark 0}{(5.075,4.469)}
\gppoint{gp mark 0}{(5.075,4.620)}
\gppoint{gp mark 0}{(5.075,3.787)}
\gppoint{gp mark 0}{(5.075,4.679)}
\gppoint{gp mark 0}{(5.075,4.485)}
\gppoint{gp mark 0}{(5.075,4.285)}
\gppoint{gp mark 0}{(5.075,4.250)}
\gppoint{gp mark 0}{(5.075,4.064)}
\gppoint{gp mark 0}{(5.075,4.674)}
\gppoint{gp mark 0}{(5.075,5.252)}
\gppoint{gp mark 0}{(5.075,4.179)}
\gppoint{gp mark 0}{(5.075,4.488)}
\gppoint{gp mark 0}{(5.075,4.928)}
\gppoint{gp mark 0}{(5.075,4.768)}
\gppoint{gp mark 0}{(5.075,4.491)}
\gppoint{gp mark 0}{(5.075,4.093)}
\gppoint{gp mark 0}{(5.075,5.219)}
\gppoint{gp mark 0}{(5.075,4.542)}
\gppoint{gp mark 0}{(5.075,5.185)}
\gppoint{gp mark 0}{(5.075,4.855)}
\gppoint{gp mark 0}{(5.075,4.222)}
\gppoint{gp mark 0}{(5.075,4.638)}
\gppoint{gp mark 0}{(5.075,4.944)}
\gppoint{gp mark 0}{(5.075,4.127)}
\gppoint{gp mark 0}{(5.075,4.796)}
\gppoint{gp mark 0}{(5.075,4.475)}
\gppoint{gp mark 0}{(5.075,3.822)}
\gppoint{gp mark 0}{(5.075,5.108)}
\gppoint{gp mark 0}{(5.075,4.667)}
\gppoint{gp mark 0}{(5.075,4.837)}
\gppoint{gp mark 0}{(5.075,5.004)}
\gppoint{gp mark 0}{(5.075,5.067)}
\gppoint{gp mark 0}{(5.075,4.857)}
\gppoint{gp mark 0}{(5.075,4.459)}
\gppoint{gp mark 0}{(5.075,4.509)}
\gppoint{gp mark 0}{(5.075,4.034)}
\gppoint{gp mark 0}{(5.075,4.944)}
\gppoint{gp mark 0}{(5.075,4.788)}
\gppoint{gp mark 0}{(5.075,4.376)}
\gppoint{gp mark 0}{(5.092,5.123)}
\gppoint{gp mark 0}{(5.092,4.774)}
\gppoint{gp mark 0}{(5.092,3.787)}
\gppoint{gp mark 0}{(5.092,5.449)}
\gppoint{gp mark 0}{(5.092,4.453)}
\gppoint{gp mark 0}{(5.092,4.349)}
\gppoint{gp mark 0}{(5.092,4.143)}
\gppoint{gp mark 0}{(5.092,4.610)}
\gppoint{gp mark 0}{(5.092,4.184)}
\gppoint{gp mark 0}{(5.092,5.200)}
\gppoint{gp mark 0}{(5.092,4.436)}
\gppoint{gp mark 0}{(5.092,4.436)}
\gppoint{gp mark 0}{(5.092,4.836)}
\gppoint{gp mark 0}{(5.092,4.456)}
\gppoint{gp mark 0}{(5.092,4.925)}
\gppoint{gp mark 0}{(5.092,4.289)}
\gppoint{gp mark 0}{(5.092,4.497)}
\gppoint{gp mark 0}{(5.092,5.076)}
\gppoint{gp mark 0}{(5.092,4.908)}
\gppoint{gp mark 0}{(5.092,4.713)}
\gppoint{gp mark 0}{(5.092,4.227)}
\gppoint{gp mark 0}{(5.092,4.021)}
\gppoint{gp mark 0}{(5.092,4.581)}
\gppoint{gp mark 0}{(5.092,4.426)}
\gppoint{gp mark 0}{(5.092,3.759)}
\gppoint{gp mark 0}{(5.092,4.426)}
\gppoint{gp mark 0}{(5.092,4.607)}
\gppoint{gp mark 0}{(5.092,4.640)}
\gppoint{gp mark 0}{(5.092,4.578)}
\gppoint{gp mark 0}{(5.092,4.302)}
\gppoint{gp mark 0}{(5.092,4.660)}
\gppoint{gp mark 0}{(5.092,4.462)}
\gppoint{gp mark 0}{(5.092,4.488)}
\gppoint{gp mark 0}{(5.092,4.208)}
\gppoint{gp mark 0}{(5.092,4.893)}
\gppoint{gp mark 0}{(5.092,4.578)}
\gppoint{gp mark 0}{(5.092,5.404)}
\gppoint{gp mark 0}{(5.092,4.855)}
\gppoint{gp mark 0}{(5.092,4.562)}
\gppoint{gp mark 0}{(5.092,4.058)}
\gppoint{gp mark 0}{(5.092,4.272)}
\gppoint{gp mark 0}{(5.092,4.302)}
\gppoint{gp mark 0}{(5.092,5.079)}
\gppoint{gp mark 0}{(5.092,4.662)}
\gppoint{gp mark 0}{(5.092,3.918)}
\gppoint{gp mark 0}{(5.092,4.774)}
\gppoint{gp mark 0}{(5.092,4.618)}
\gppoint{gp mark 0}{(5.092,5.221)}
\gppoint{gp mark 0}{(5.092,4.944)}
\gppoint{gp mark 0}{(5.092,4.917)}
\gppoint{gp mark 0}{(5.092,4.459)}
\gppoint{gp mark 0}{(5.092,4.643)}
\gppoint{gp mark 0}{(5.092,4.028)}
\gppoint{gp mark 0}{(5.092,4.482)}
\gppoint{gp mark 0}{(5.092,4.509)}
\gppoint{gp mark 0}{(5.092,4.088)}
\gppoint{gp mark 0}{(5.092,4.548)}
\gppoint{gp mark 0}{(5.092,5.021)}
\gppoint{gp mark 0}{(5.092,5.175)}
\gppoint{gp mark 0}{(5.092,4.875)}
\gppoint{gp mark 0}{(5.092,4.903)}
\gppoint{gp mark 0}{(5.092,4.070)}
\gppoint{gp mark 0}{(5.092,4.280)}
\gppoint{gp mark 0}{(5.092,5.474)}
\gppoint{gp mark 0}{(5.092,4.669)}
\gppoint{gp mark 0}{(5.092,4.405)}
\gppoint{gp mark 0}{(5.092,4.697)}
\gppoint{gp mark 0}{(5.092,5.411)}
\gppoint{gp mark 0}{(5.092,4.672)}
\gppoint{gp mark 0}{(5.092,4.900)}
\gppoint{gp mark 0}{(5.092,5.262)}
\gppoint{gp mark 0}{(5.092,4.623)}
\gppoint{gp mark 0}{(5.092,5.000)}
\gppoint{gp mark 0}{(5.092,5.262)}
\gppoint{gp mark 0}{(5.092,4.623)}
\gppoint{gp mark 0}{(5.092,5.127)}
\gppoint{gp mark 0}{(5.092,4.908)}
\gppoint{gp mark 0}{(5.109,4.556)}
\gppoint{gp mark 0}{(5.109,5.023)}
\gppoint{gp mark 0}{(5.109,4.969)}
\gppoint{gp mark 0}{(5.109,4.845)}
\gppoint{gp mark 0}{(5.109,4.058)}
\gppoint{gp mark 0}{(5.109,5.396)}
\gppoint{gp mark 0}{(5.109,4.898)}
\gppoint{gp mark 0}{(5.109,4.695)}
\gppoint{gp mark 0}{(5.109,4.715)}
\gppoint{gp mark 0}{(5.109,4.302)}
\gppoint{gp mark 0}{(5.109,4.697)}
\gppoint{gp mark 0}{(5.109,4.625)}
\gppoint{gp mark 0}{(5.109,4.302)}
\gppoint{gp mark 0}{(5.109,4.920)}
\gppoint{gp mark 0}{(5.109,4.127)}
\gppoint{gp mark 0}{(5.109,5.039)}
\gppoint{gp mark 0}{(5.109,4.717)}
\gppoint{gp mark 0}{(5.109,4.562)}
\gppoint{gp mark 0}{(5.109,4.302)}
\gppoint{gp mark 0}{(5.109,5.096)}
\gppoint{gp mark 0}{(5.109,4.267)}
\gppoint{gp mark 0}{(5.109,4.845)}
\gppoint{gp mark 0}{(5.109,4.625)}
\gppoint{gp mark 0}{(5.109,4.415)}
\gppoint{gp mark 0}{(5.109,4.494)}
\gppoint{gp mark 0}{(5.109,4.929)}
\gppoint{gp mark 0}{(5.109,4.542)}
\gppoint{gp mark 0}{(5.109,4.732)}
\gppoint{gp mark 0}{(5.109,5.010)}
\gppoint{gp mark 0}{(5.109,4.433)}
\gppoint{gp mark 0}{(5.109,5.171)}
\gppoint{gp mark 0}{(5.109,5.171)}
\gppoint{gp mark 0}{(5.109,4.070)}
\gppoint{gp mark 0}{(5.109,4.469)}
\gppoint{gp mark 0}{(5.109,4.137)}
\gppoint{gp mark 0}{(5.109,4.285)}
\gppoint{gp mark 0}{(5.109,4.802)}
\gppoint{gp mark 0}{(5.109,4.064)}
\gppoint{gp mark 0}{(5.109,5.177)}
\gppoint{gp mark 0}{(5.109,4.380)}
\gppoint{gp mark 0}{(5.109,4.815)}
\gppoint{gp mark 0}{(5.109,4.623)}
\gppoint{gp mark 0}{(5.109,4.189)}
\gppoint{gp mark 0}{(5.109,4.443)}
\gppoint{gp mark 0}{(5.109,4.236)}
\gppoint{gp mark 0}{(5.109,4.376)}
\gppoint{gp mark 0}{(5.109,4.745)}
\gppoint{gp mark 0}{(5.109,4.034)}
\gppoint{gp mark 0}{(5.109,4.419)}
\gppoint{gp mark 0}{(5.109,5.170)}
\gppoint{gp mark 0}{(5.109,4.459)}
\gppoint{gp mark 0}{(5.109,4.524)}
\gppoint{gp mark 0}{(5.109,4.297)}
\gppoint{gp mark 0}{(5.109,4.093)}
\gppoint{gp mark 0}{(5.109,4.222)}
\gppoint{gp mark 0}{(5.109,4.070)}
\gppoint{gp mark 0}{(5.109,4.132)}
\gppoint{gp mark 0}{(5.109,4.189)}
\gppoint{gp mark 0}{(5.109,4.567)}
\gppoint{gp mark 0}{(5.109,5.063)}
\gppoint{gp mark 0}{(5.109,4.280)}
\gppoint{gp mark 0}{(5.109,4.947)}
\gppoint{gp mark 0}{(5.126,3.911)}
\gppoint{gp mark 0}{(5.126,4.821)}
\gppoint{gp mark 0}{(5.126,5.054)}
\gppoint{gp mark 0}{(5.126,4.550)}
\gppoint{gp mark 0}{(5.126,4.550)}
\gppoint{gp mark 0}{(5.126,4.570)}
\gppoint{gp mark 0}{(5.126,4.878)}
\gppoint{gp mark 0}{(5.126,5.286)}
\gppoint{gp mark 0}{(5.126,5.050)}
\gppoint{gp mark 0}{(5.126,5.775)}
\gppoint{gp mark 0}{(5.126,4.706)}
\gppoint{gp mark 0}{(5.126,4.208)}
\gppoint{gp mark 0}{(5.126,4.749)}
\gppoint{gp mark 0}{(5.126,4.088)}
\gppoint{gp mark 0}{(5.126,4.873)}
\gppoint{gp mark 0}{(5.126,4.581)}
\gppoint{gp mark 0}{(5.126,5.092)}
\gppoint{gp mark 0}{(5.126,4.194)}
\gppoint{gp mark 0}{(5.126,5.128)}
\gppoint{gp mark 0}{(5.126,5.293)}
\gppoint{gp mark 0}{(5.126,4.028)}
\gppoint{gp mark 0}{(5.126,5.464)}
\gppoint{gp mark 0}{(5.126,4.357)}
\gppoint{gp mark 0}{(5.126,4.550)}
\gppoint{gp mark 0}{(5.126,5.212)}
\gppoint{gp mark 0}{(5.126,4.855)}
\gppoint{gp mark 0}{(5.126,5.708)}
\gppoint{gp mark 0}{(5.126,4.635)}
\gppoint{gp mark 0}{(5.126,4.334)}
\gppoint{gp mark 0}{(5.126,4.542)}
\gppoint{gp mark 0}{(5.126,4.263)}
\gppoint{gp mark 0}{(5.126,5.301)}
\gppoint{gp mark 0}{(5.126,4.539)}
\gppoint{gp mark 0}{(5.126,5.212)}
\gppoint{gp mark 0}{(5.126,4.550)}
\gppoint{gp mark 0}{(5.126,4.189)}
\gppoint{gp mark 0}{(5.126,4.702)}
\gppoint{gp mark 0}{(5.126,4.376)}
\gppoint{gp mark 0}{(5.126,5.193)}
\gppoint{gp mark 0}{(5.126,4.488)}
\gppoint{gp mark 0}{(5.126,4.581)}
\gppoint{gp mark 0}{(5.126,5.383)}
\gppoint{gp mark 0}{(5.126,4.462)}
\gppoint{gp mark 0}{(5.126,5.039)}
\gppoint{gp mark 0}{(5.126,4.174)}
\gppoint{gp mark 0}{(5.126,3.933)}
\gppoint{gp mark 0}{(5.126,5.504)}
\gppoint{gp mark 0}{(5.126,4.330)}
\gppoint{gp mark 0}{(5.126,4.550)}
\gppoint{gp mark 0}{(5.126,4.741)}
\gppoint{gp mark 0}{(5.126,4.602)}
\gppoint{gp mark 0}{(5.126,3.010)}
\gppoint{gp mark 0}{(5.126,3.010)}
\gppoint{gp mark 0}{(5.126,4.559)}
\gppoint{gp mark 0}{(5.126,5.482)}
\gppoint{gp mark 0}{(5.126,4.302)}
\gppoint{gp mark 0}{(5.126,4.443)}
\gppoint{gp mark 0}{(5.126,4.724)}
\gppoint{gp mark 0}{(5.126,4.550)}
\gppoint{gp mark 0}{(5.126,4.387)}
\gppoint{gp mark 0}{(5.126,4.497)}
\gppoint{gp mark 0}{(5.126,4.559)}
\gppoint{gp mark 0}{(5.126,4.836)}
\gppoint{gp mark 0}{(5.126,4.405)}
\gppoint{gp mark 0}{(5.126,4.900)}
\gppoint{gp mark 0}{(5.126,4.567)}
\gppoint{gp mark 0}{(5.126,4.426)}
\gppoint{gp mark 0}{(5.142,5.031)}
\gppoint{gp mark 0}{(5.142,5.141)}
\gppoint{gp mark 0}{(5.142,4.174)}
\gppoint{gp mark 0}{(5.142,4.963)}
\gppoint{gp mark 0}{(5.142,5.498)}
\gppoint{gp mark 0}{(5.142,4.782)}
\gppoint{gp mark 0}{(5.142,4.809)}
\gppoint{gp mark 0}{(5.142,4.792)}
\gppoint{gp mark 0}{(5.142,4.902)}
\gppoint{gp mark 0}{(5.142,5.109)}
\gppoint{gp mark 0}{(5.142,5.944)}
\gppoint{gp mark 0}{(5.142,4.782)}
\gppoint{gp mark 0}{(5.142,4.559)}
\gppoint{gp mark 0}{(5.142,4.768)}
\gppoint{gp mark 0}{(5.142,3.975)}
\gppoint{gp mark 0}{(5.142,5.149)}
\gppoint{gp mark 0}{(5.142,5.084)}
\gppoint{gp mark 0}{(5.142,4.372)}
\gppoint{gp mark 0}{(5.142,4.931)}
\gppoint{gp mark 0}{(5.142,4.610)}
\gppoint{gp mark 0}{(5.142,4.394)}
\gppoint{gp mark 0}{(5.142,4.672)}
\gppoint{gp mark 0}{(5.142,3.995)}
\gppoint{gp mark 0}{(5.142,4.398)}
\gppoint{gp mark 0}{(5.142,5.003)}
\gppoint{gp mark 0}{(5.142,4.503)}
\gppoint{gp mark 0}{(5.142,5.519)}
\gppoint{gp mark 0}{(5.142,4.342)}
\gppoint{gp mark 0}{(5.142,5.136)}
\gppoint{gp mark 0}{(5.142,4.527)}
\gppoint{gp mark 0}{(5.142,4.472)}
\gppoint{gp mark 0}{(5.142,5.004)}
\gppoint{gp mark 0}{(5.142,5.031)}
\gppoint{gp mark 0}{(5.142,4.293)}
\gppoint{gp mark 0}{(5.142,4.768)}
\gppoint{gp mark 0}{(5.142,5.202)}
\gppoint{gp mark 0}{(5.142,4.737)}
\gppoint{gp mark 0}{(5.142,4.368)}
\gppoint{gp mark 0}{(5.142,4.365)}
\gppoint{gp mark 0}{(5.142,5.004)}
\gppoint{gp mark 0}{(5.142,5.028)}
\gppoint{gp mark 0}{(5.142,5.028)}
\gppoint{gp mark 0}{(5.142,5.028)}
\gppoint{gp mark 0}{(5.142,4.850)}
\gppoint{gp mark 0}{(5.142,5.507)}
\gppoint{gp mark 0}{(5.142,4.766)}
\gppoint{gp mark 0}{(5.142,4.506)}
\gppoint{gp mark 0}{(5.142,5.417)}
\gppoint{gp mark 0}{(5.142,5.194)}
\gppoint{gp mark 0}{(5.142,4.655)}
\gppoint{gp mark 0}{(5.142,4.947)}
\gppoint{gp mark 0}{(5.142,4.326)}
\gppoint{gp mark 0}{(5.142,4.459)}
\gppoint{gp mark 0}{(5.142,5.178)}
\gppoint{gp mark 0}{(5.142,5.261)}
\gppoint{gp mark 0}{(5.142,4.693)}
\gppoint{gp mark 0}{(5.142,4.008)}
\gppoint{gp mark 0}{(5.142,4.743)}
\gppoint{gp mark 0}{(5.142,5.139)}
\gppoint{gp mark 0}{(5.142,5.178)}
\gppoint{gp mark 0}{(5.142,4.179)}
\gppoint{gp mark 0}{(5.142,4.408)}
\gppoint{gp mark 0}{(5.142,4.322)}
\gppoint{gp mark 0}{(5.142,4.208)}
\gppoint{gp mark 0}{(5.142,4.153)}
\gppoint{gp mark 0}{(5.142,5.000)}
\gppoint{gp mark 0}{(5.142,4.947)}
\gppoint{gp mark 0}{(5.142,5.000)}
\gppoint{gp mark 0}{(5.158,4.823)}
\gppoint{gp mark 0}{(5.158,5.258)}
\gppoint{gp mark 0}{(5.158,4.880)}
\gppoint{gp mark 0}{(5.158,4.638)}
\gppoint{gp mark 0}{(5.158,4.915)}
\gppoint{gp mark 0}{(5.158,4.760)}
\gppoint{gp mark 0}{(5.158,4.306)}
\gppoint{gp mark 0}{(5.158,5.208)}
\gppoint{gp mark 0}{(5.158,4.772)}
\gppoint{gp mark 0}{(5.158,4.693)}
\gppoint{gp mark 0}{(5.158,5.115)}
\gppoint{gp mark 0}{(5.158,4.439)}
\gppoint{gp mark 0}{(5.158,5.051)}
\gppoint{gp mark 0}{(5.158,5.313)}
\gppoint{gp mark 0}{(5.158,4.660)}
\gppoint{gp mark 0}{(5.158,4.527)}
\gppoint{gp mark 0}{(5.158,4.158)}
\gppoint{gp mark 0}{(5.158,5.732)}
\gppoint{gp mark 0}{(5.158,4.615)}
\gppoint{gp mark 0}{(5.158,5.213)}
\gppoint{gp mark 0}{(5.158,4.581)}
\gppoint{gp mark 0}{(5.158,4.672)}
\gppoint{gp mark 0}{(5.158,4.672)}
\gppoint{gp mark 0}{(5.158,4.672)}
\gppoint{gp mark 0}{(5.158,4.730)}
\gppoint{gp mark 0}{(5.158,4.213)}
\gppoint{gp mark 0}{(5.158,5.367)}
\gppoint{gp mark 0}{(5.158,4.648)}
\gppoint{gp mark 0}{(5.158,4.778)}
\gppoint{gp mark 0}{(5.158,5.337)}
\gppoint{gp mark 0}{(5.158,5.619)}
\gppoint{gp mark 0}{(5.158,5.000)}
\gppoint{gp mark 0}{(5.158,4.957)}
\gppoint{gp mark 0}{(5.158,4.826)}
\gppoint{gp mark 0}{(5.158,4.046)}
\gppoint{gp mark 0}{(5.158,5.126)}
\gppoint{gp mark 0}{(5.158,4.524)}
\gppoint{gp mark 0}{(5.158,5.398)}
\gppoint{gp mark 0}{(5.158,5.023)}
\gppoint{gp mark 0}{(5.158,4.021)}
\gppoint{gp mark 0}{(5.158,5.091)}
\gppoint{gp mark 0}{(5.158,4.697)}
\gppoint{gp mark 0}{(5.158,4.302)}
\gppoint{gp mark 0}{(5.158,4.297)}
\gppoint{gp mark 0}{(5.158,4.640)}
\gppoint{gp mark 0}{(5.158,4.768)}
\gppoint{gp mark 0}{(5.158,5.208)}
\gppoint{gp mark 0}{(5.158,5.504)}
\gppoint{gp mark 0}{(5.158,4.745)}
\gppoint{gp mark 0}{(5.158,4.708)}
\gppoint{gp mark 0}{(5.158,4.137)}
\gppoint{gp mark 0}{(5.158,4.297)}
\gppoint{gp mark 0}{(5.158,4.683)}
\gppoint{gp mark 0}{(5.158,5.207)}
\gppoint{gp mark 0}{(5.158,4.310)}
\gppoint{gp mark 0}{(5.158,4.110)}
\gppoint{gp mark 0}{(5.158,4.297)}
\gppoint{gp mark 0}{(5.158,5.507)}
\gppoint{gp mark 0}{(5.158,5.159)}
\gppoint{gp mark 0}{(5.158,5.048)}
\gppoint{gp mark 0}{(5.158,4.143)}
\gppoint{gp mark 0}{(5.158,4.536)}
\gppoint{gp mark 0}{(5.158,4.802)}
\gppoint{gp mark 0}{(5.158,4.198)}
\gppoint{gp mark 0}{(5.158,4.679)}
\gppoint{gp mark 0}{(5.158,4.772)}
\gppoint{gp mark 0}{(5.158,4.088)}
\gppoint{gp mark 0}{(5.158,4.512)}
\gppoint{gp mark 0}{(5.158,4.530)}
\gppoint{gp mark 0}{(5.158,4.980)}
\gppoint{gp mark 0}{(5.158,4.436)}
\gppoint{gp mark 0}{(5.158,4.466)}
\gppoint{gp mark 0}{(5.158,4.980)}
\gppoint{gp mark 0}{(5.158,4.503)}
\gppoint{gp mark 0}{(5.158,4.780)}
\gppoint{gp mark 0}{(5.158,4.841)}
\gppoint{gp mark 0}{(5.158,5.747)}
\gppoint{gp mark 0}{(5.174,5.009)}
\gppoint{gp mark 0}{(5.174,4.453)}
\gppoint{gp mark 0}{(5.174,4.719)}
\gppoint{gp mark 0}{(5.174,4.975)}
\gppoint{gp mark 0}{(5.174,4.676)}
\gppoint{gp mark 0}{(5.174,4.836)}
\gppoint{gp mark 0}{(5.174,4.897)}
\gppoint{gp mark 0}{(5.174,4.222)}
\gppoint{gp mark 0}{(5.174,4.730)}
\gppoint{gp mark 0}{(5.174,4.592)}
\gppoint{gp mark 0}{(5.174,5.339)}
\gppoint{gp mark 0}{(5.174,4.826)}
\gppoint{gp mark 0}{(5.174,4.655)}
\gppoint{gp mark 0}{(5.174,4.500)}
\gppoint{gp mark 0}{(5.174,4.715)}
\gppoint{gp mark 0}{(5.174,4.383)}
\gppoint{gp mark 0}{(5.174,4.121)}
\gppoint{gp mark 0}{(5.174,4.573)}
\gppoint{gp mark 0}{(5.174,5.141)}
\gppoint{gp mark 0}{(5.174,4.338)}
\gppoint{gp mark 0}{(5.174,4.573)}
\gppoint{gp mark 0}{(5.174,4.650)}
\gppoint{gp mark 0}{(5.174,4.776)}
\gppoint{gp mark 0}{(5.174,4.650)}
\gppoint{gp mark 0}{(5.174,4.573)}
\gppoint{gp mark 0}{(5.174,5.003)}
\gppoint{gp mark 0}{(5.174,4.380)}
\gppoint{gp mark 0}{(5.174,5.175)}
\gppoint{gp mark 0}{(5.174,4.372)}
\gppoint{gp mark 0}{(5.174,4.449)}
\gppoint{gp mark 0}{(5.174,4.099)}
\gppoint{gp mark 0}{(5.174,4.153)}
\gppoint{gp mark 0}{(5.174,4.837)}
\gppoint{gp mark 0}{(5.174,4.837)}
\gppoint{gp mark 0}{(5.174,4.040)}
\gppoint{gp mark 0}{(5.174,5.175)}
\gppoint{gp mark 0}{(5.174,5.291)}
\gppoint{gp mark 0}{(5.174,4.819)}
\gppoint{gp mark 0}{(5.174,5.313)}
\gppoint{gp mark 0}{(5.174,4.809)}
\gppoint{gp mark 0}{(5.174,5.188)}
\gppoint{gp mark 0}{(5.174,4.963)}
\gppoint{gp mark 0}{(5.174,4.208)}
\gppoint{gp mark 0}{(5.174,4.592)}
\gppoint{gp mark 0}{(5.174,4.747)}
\gppoint{gp mark 0}{(5.174,5.039)}
\gppoint{gp mark 0}{(5.174,5.113)}
\gppoint{gp mark 0}{(5.174,5.313)}
\gppoint{gp mark 0}{(5.174,5.175)}
\gppoint{gp mark 0}{(5.174,4.227)}
\gppoint{gp mark 0}{(5.174,4.592)}
\gppoint{gp mark 0}{(5.174,5.602)}
\gppoint{gp mark 0}{(5.174,4.208)}
\gppoint{gp mark 0}{(5.174,4.415)}
\gppoint{gp mark 0}{(5.174,4.699)}
\gppoint{gp mark 0}{(5.174,5.510)}
\gppoint{gp mark 0}{(5.174,4.883)}
\gppoint{gp mark 0}{(5.174,4.415)}
\gppoint{gp mark 0}{(5.174,4.297)}
\gppoint{gp mark 0}{(5.174,4.875)}
\gppoint{gp mark 0}{(5.174,4.589)}
\gppoint{gp mark 0}{(5.174,4.726)}
\gppoint{gp mark 0}{(5.174,5.218)}
\gppoint{gp mark 0}{(5.174,4.983)}
\gppoint{gp mark 0}{(5.190,4.672)}
\gppoint{gp mark 0}{(5.190,3.402)}
\gppoint{gp mark 0}{(5.190,4.933)}
\gppoint{gp mark 0}{(5.190,4.127)}
\gppoint{gp mark 0}{(5.190,4.426)}
\gppoint{gp mark 0}{(5.190,4.494)}
\gppoint{gp mark 0}{(5.190,4.443)}
\gppoint{gp mark 0}{(5.190,4.564)}
\gppoint{gp mark 0}{(5.190,4.695)}
\gppoint{gp mark 0}{(5.190,5.104)}
\gppoint{gp mark 0}{(5.190,4.679)}
\gppoint{gp mark 0}{(5.190,4.882)}
\gppoint{gp mark 0}{(5.190,4.398)}
\gppoint{gp mark 0}{(5.190,5.418)}
\gppoint{gp mark 0}{(5.190,5.120)}
\gppoint{gp mark 0}{(5.190,4.774)}
\gppoint{gp mark 0}{(5.190,5.048)}
\gppoint{gp mark 0}{(5.190,5.151)}
\gppoint{gp mark 0}{(5.190,4.419)}
\gppoint{gp mark 0}{(5.190,4.882)}
\gppoint{gp mark 0}{(5.190,4.412)}
\gppoint{gp mark 0}{(5.190,4.837)}
\gppoint{gp mark 0}{(5.190,4.581)}
\gppoint{gp mark 0}{(5.190,4.882)}
\gppoint{gp mark 0}{(5.190,4.567)}
\gppoint{gp mark 0}{(5.190,5.552)}
\gppoint{gp mark 0}{(5.190,4.882)}
\gppoint{gp mark 0}{(5.190,4.615)}
\gppoint{gp mark 0}{(5.190,4.174)}
\gppoint{gp mark 0}{(5.190,4.882)}
\gppoint{gp mark 0}{(5.190,4.521)}
\gppoint{gp mark 0}{(5.190,3.796)}
\gppoint{gp mark 0}{(5.190,4.040)}
\gppoint{gp mark 0}{(5.190,5.302)}
\gppoint{gp mark 0}{(5.190,5.304)}
\gppoint{gp mark 0}{(5.190,5.251)}
\gppoint{gp mark 0}{(5.190,4.882)}
\gppoint{gp mark 0}{(5.190,5.353)}
\gppoint{gp mark 0}{(5.190,4.070)}
\gppoint{gp mark 0}{(5.190,5.236)}
\gppoint{gp mark 0}{(5.190,5.040)}
\gppoint{gp mark 0}{(5.190,4.895)}
\gppoint{gp mark 0}{(5.190,4.602)}
\gppoint{gp mark 0}{(5.190,5.418)}
\gppoint{gp mark 0}{(5.190,4.882)}
\gppoint{gp mark 0}{(5.190,4.093)}
\gppoint{gp mark 0}{(5.190,4.854)}
\gppoint{gp mark 0}{(5.190,4.987)}
\gppoint{gp mark 0}{(5.190,5.352)}
\gppoint{gp mark 0}{(5.190,4.980)}
\gppoint{gp mark 0}{(5.190,5.107)}
\gppoint{gp mark 0}{(5.190,5.107)}
\gppoint{gp mark 0}{(5.190,4.798)}
\gppoint{gp mark 0}{(5.190,5.031)}
\gppoint{gp mark 0}{(5.190,4.602)}
\gppoint{gp mark 0}{(5.190,4.446)}
\gppoint{gp mark 0}{(5.190,4.527)}
\gppoint{gp mark 0}{(5.190,4.203)}
\gppoint{gp mark 0}{(5.190,4.586)}
\gppoint{gp mark 0}{(5.190,5.120)}
\gppoint{gp mark 0}{(5.190,4.679)}
\gppoint{gp mark 0}{(5.190,4.446)}
\gppoint{gp mark 0}{(5.190,4.679)}
\gppoint{gp mark 0}{(5.190,4.618)}
\gppoint{gp mark 0}{(5.190,5.243)}
\gppoint{gp mark 0}{(5.190,4.674)}
\gppoint{gp mark 0}{(5.190,5.395)}
\gppoint{gp mark 0}{(5.190,4.082)}
\gppoint{gp mark 0}{(5.205,4.174)}
\gppoint{gp mark 0}{(5.205,5.547)}
\gppoint{gp mark 0}{(5.205,4.088)}
\gppoint{gp mark 0}{(5.205,4.338)}
\gppoint{gp mark 0}{(5.205,4.446)}
\gppoint{gp mark 0}{(5.205,4.878)}
\gppoint{gp mark 0}{(5.205,5.150)}
\gppoint{gp mark 0}{(5.205,5.197)}
\gppoint{gp mark 0}{(5.205,4.148)}
\gppoint{gp mark 0}{(5.205,4.794)}
\gppoint{gp mark 0}{(5.205,4.984)}
\gppoint{gp mark 0}{(5.205,4.643)}
\gppoint{gp mark 0}{(5.205,4.625)}
\gppoint{gp mark 0}{(5.205,4.625)}
\gppoint{gp mark 0}{(5.205,5.035)}
\gppoint{gp mark 0}{(5.205,4.338)}
\gppoint{gp mark 0}{(5.205,5.230)}
\gppoint{gp mark 0}{(5.205,4.338)}
\gppoint{gp mark 0}{(5.205,4.956)}
\gppoint{gp mark 0}{(5.205,4.267)}
\gppoint{gp mark 0}{(5.205,4.883)}
\gppoint{gp mark 0}{(5.205,5.010)}
\gppoint{gp mark 0}{(5.205,5.023)}
\gppoint{gp mark 0}{(5.205,4.824)}
\gppoint{gp mark 0}{(5.205,4.880)}
\gppoint{gp mark 0}{(5.205,5.021)}
\gppoint{gp mark 0}{(5.205,4.137)}
\gppoint{gp mark 0}{(5.205,5.279)}
\gppoint{gp mark 0}{(5.205,5.729)}
\gppoint{gp mark 0}{(5.205,4.394)}
\gppoint{gp mark 0}{(5.205,4.962)}
\gppoint{gp mark 0}{(5.205,5.011)}
\gppoint{gp mark 0}{(5.205,5.542)}
\gppoint{gp mark 0}{(5.205,4.322)}
\gppoint{gp mark 0}{(5.205,5.192)}
\gppoint{gp mark 0}{(5.205,4.848)}
\gppoint{gp mark 0}{(5.205,4.584)}
\gppoint{gp mark 0}{(5.205,5.262)}
\gppoint{gp mark 0}{(5.205,4.567)}
\gppoint{gp mark 0}{(5.205,5.223)}
\gppoint{gp mark 0}{(5.205,5.069)}
\gppoint{gp mark 0}{(5.205,4.494)}
\gppoint{gp mark 0}{(5.205,4.807)}
\gppoint{gp mark 0}{(5.205,4.456)}
\gppoint{gp mark 0}{(5.205,4.662)}
\gppoint{gp mark 0}{(5.205,5.117)}
\gppoint{gp mark 0}{(5.205,3.975)}
\gppoint{gp mark 0}{(5.205,4.306)}
\gppoint{gp mark 0}{(5.205,4.688)}
\gppoint{gp mark 0}{(5.205,4.845)}
\gppoint{gp mark 0}{(5.205,5.190)}
\gppoint{gp mark 0}{(5.205,5.190)}
\gppoint{gp mark 0}{(5.205,5.190)}
\gppoint{gp mark 0}{(5.205,4.732)}
\gppoint{gp mark 0}{(5.205,4.662)}
\gppoint{gp mark 0}{(5.205,3.010)}
\gppoint{gp mark 0}{(5.205,5.017)}
\gppoint{gp mark 0}{(5.205,4.236)}
\gppoint{gp mark 0}{(5.205,4.811)}
\gppoint{gp mark 0}{(5.205,5.571)}
\gppoint{gp mark 0}{(5.205,5.179)}
\gppoint{gp mark 0}{(5.205,5.519)}
\gppoint{gp mark 0}{(5.205,5.438)}
\gppoint{gp mark 0}{(5.205,4.855)}
\gppoint{gp mark 0}{(5.205,4.419)}
\gppoint{gp mark 0}{(5.205,5.010)}
\gppoint{gp mark 0}{(5.205,5.002)}
\gppoint{gp mark 0}{(5.205,4.289)}
\gppoint{gp mark 0}{(5.205,4.962)}
\gppoint{gp mark 0}{(5.205,4.962)}
\gppoint{gp mark 0}{(5.205,4.506)}
\gppoint{gp mark 0}{(5.220,4.398)}
\gppoint{gp mark 0}{(5.220,4.667)}
\gppoint{gp mark 0}{(5.220,4.070)}
\gppoint{gp mark 0}{(5.220,4.824)}
\gppoint{gp mark 0}{(5.220,5.508)}
\gppoint{gp mark 0}{(5.220,5.170)}
\gppoint{gp mark 0}{(5.220,4.600)}
\gppoint{gp mark 0}{(5.220,4.956)}
\gppoint{gp mark 0}{(5.220,4.515)}
\gppoint{gp mark 0}{(5.220,5.072)}
\gppoint{gp mark 0}{(5.220,4.683)}
\gppoint{gp mark 0}{(5.220,4.518)}
\gppoint{gp mark 0}{(5.220,4.338)}
\gppoint{gp mark 0}{(5.220,5.058)}
\gppoint{gp mark 0}{(5.220,4.164)}
\gppoint{gp mark 0}{(5.220,4.758)}
\gppoint{gp mark 0}{(5.220,4.807)}
\gppoint{gp mark 0}{(5.220,4.760)}
\gppoint{gp mark 0}{(5.220,5.508)}
\gppoint{gp mark 0}{(5.220,5.028)}
\gppoint{gp mark 0}{(5.220,4.521)}
\gppoint{gp mark 0}{(5.220,5.060)}
\gppoint{gp mark 0}{(5.220,5.088)}
\gppoint{gp mark 0}{(5.220,5.121)}
\gppoint{gp mark 0}{(5.220,4.790)}
\gppoint{gp mark 0}{(5.220,4.573)}
\gppoint{gp mark 0}{(5.220,5.465)}
\gppoint{gp mark 0}{(5.220,4.521)}
\gppoint{gp mark 0}{(5.220,5.344)}
\gppoint{gp mark 0}{(5.220,5.538)}
\gppoint{gp mark 0}{(5.220,4.503)}
\gppoint{gp mark 0}{(5.220,4.542)}
\gppoint{gp mark 0}{(5.220,4.790)}
\gppoint{gp mark 0}{(5.220,5.339)}
\gppoint{gp mark 0}{(5.220,5.067)}
\gppoint{gp mark 0}{(5.220,4.443)}
\gppoint{gp mark 0}{(5.220,4.836)}
\gppoint{gp mark 0}{(5.220,4.482)}
\gppoint{gp mark 0}{(5.220,5.067)}
\gppoint{gp mark 0}{(5.220,4.346)}
\gppoint{gp mark 0}{(5.220,5.109)}
\gppoint{gp mark 0}{(5.220,4.774)}
\gppoint{gp mark 0}{(5.220,4.398)}
\gppoint{gp mark 0}{(5.220,4.620)}
\gppoint{gp mark 0}{(5.220,5.344)}
\gppoint{gp mark 0}{(5.220,5.600)}
\gppoint{gp mark 0}{(5.220,5.067)}
\gppoint{gp mark 0}{(5.220,4.456)}
\gppoint{gp mark 0}{(5.220,4.297)}
\gppoint{gp mark 0}{(5.220,4.361)}
\gppoint{gp mark 0}{(5.220,4.917)}
\gppoint{gp mark 0}{(5.220,4.542)}
\gppoint{gp mark 0}{(5.220,4.790)}
\gppoint{gp mark 0}{(5.220,4.790)}
\gppoint{gp mark 0}{(5.220,4.263)}
\gppoint{gp mark 0}{(5.220,4.365)}
\gppoint{gp mark 0}{(5.220,5.055)}
\gppoint{gp mark 0}{(5.235,4.365)}
\gppoint{gp mark 0}{(5.235,5.023)}
\gppoint{gp mark 0}{(5.235,4.940)}
\gppoint{gp mark 0}{(5.235,4.015)}
\gppoint{gp mark 0}{(5.235,5.207)}
\gppoint{gp mark 0}{(5.235,4.222)}
\gppoint{gp mark 0}{(5.235,5.143)}
\gppoint{gp mark 0}{(5.235,4.419)}
\gppoint{gp mark 0}{(5.235,5.576)}
\gppoint{gp mark 0}{(5.235,4.855)}
\gppoint{gp mark 0}{(5.235,5.451)}
\gppoint{gp mark 0}{(5.235,4.864)}
\gppoint{gp mark 0}{(5.235,4.846)}
\gppoint{gp mark 0}{(5.235,4.824)}
\gppoint{gp mark 0}{(5.235,4.981)}
\gppoint{gp mark 0}{(5.235,4.981)}
\gppoint{gp mark 0}{(5.235,4.950)}
\gppoint{gp mark 0}{(5.235,4.972)}
\gppoint{gp mark 0}{(5.235,5.320)}
\gppoint{gp mark 0}{(5.235,5.070)}
\gppoint{gp mark 0}{(5.235,5.262)}
\gppoint{gp mark 0}{(5.235,4.735)}
\gppoint{gp mark 0}{(5.235,5.203)}
\gppoint{gp mark 0}{(5.235,4.876)}
\gppoint{gp mark 0}{(5.235,4.254)}
\gppoint{gp mark 0}{(5.235,4.550)}
\gppoint{gp mark 0}{(5.235,4.415)}
\gppoint{gp mark 0}{(5.235,5.161)}
\gppoint{gp mark 0}{(5.235,5.284)}
\gppoint{gp mark 0}{(5.235,4.429)}
\gppoint{gp mark 0}{(5.235,4.756)}
\gppoint{gp mark 0}{(5.235,4.870)}
\gppoint{gp mark 0}{(5.235,4.564)}
\gppoint{gp mark 0}{(5.235,4.821)}
\gppoint{gp mark 0}{(5.235,5.137)}
\gppoint{gp mark 0}{(5.235,4.429)}
\gppoint{gp mark 0}{(5.235,5.579)}
\gppoint{gp mark 0}{(5.235,5.065)}
\gppoint{gp mark 0}{(5.235,5.039)}
\gppoint{gp mark 0}{(5.235,4.782)}
\gppoint{gp mark 0}{(5.235,5.039)}
\gppoint{gp mark 0}{(5.235,4.679)}
\gppoint{gp mark 0}{(5.235,5.164)}
\gppoint{gp mark 0}{(5.235,4.213)}
\gppoint{gp mark 0}{(5.235,4.443)}
\gppoint{gp mark 0}{(5.235,5.048)}
\gppoint{gp mark 0}{(5.235,4.915)}
\gppoint{gp mark 0}{(5.235,4.326)}
\gppoint{gp mark 0}{(5.235,4.811)}
\gppoint{gp mark 0}{(5.235,4.137)}
\gppoint{gp mark 0}{(5.235,5.035)}
\gppoint{gp mark 0}{(5.235,5.392)}
\gppoint{gp mark 0}{(5.235,5.838)}
\gppoint{gp mark 0}{(5.249,4.401)}
\gppoint{gp mark 0}{(5.249,5.093)}
\gppoint{gp mark 0}{(5.249,4.132)}
\gppoint{gp mark 0}{(5.249,4.633)}
\gppoint{gp mark 0}{(5.249,4.132)}
\gppoint{gp mark 0}{(5.249,4.898)}
\gppoint{gp mark 0}{(5.249,5.429)}
\gppoint{gp mark 0}{(5.249,5.016)}
\gppoint{gp mark 0}{(5.249,4.318)}
\gppoint{gp mark 0}{(5.249,4.963)}
\gppoint{gp mark 0}{(5.249,4.453)}
\gppoint{gp mark 0}{(5.249,4.174)}
\gppoint{gp mark 0}{(5.249,5.025)}
\gppoint{gp mark 0}{(5.249,5.097)}
\gppoint{gp mark 0}{(5.249,4.164)}
\gppoint{gp mark 0}{(5.249,5.727)}
\gppoint{gp mark 0}{(5.249,5.238)}
\gppoint{gp mark 0}{(5.249,5.169)}
\gppoint{gp mark 0}{(5.249,4.267)}
\gppoint{gp mark 0}{(5.249,4.439)}
\gppoint{gp mark 0}{(5.249,4.398)}
\gppoint{gp mark 0}{(5.249,5.115)}
\gppoint{gp mark 0}{(5.249,4.368)}
\gppoint{gp mark 0}{(5.249,5.027)}
\gppoint{gp mark 0}{(5.249,5.105)}
\gppoint{gp mark 0}{(5.249,4.227)}
\gppoint{gp mark 0}{(5.249,4.638)}
\gppoint{gp mark 0}{(5.249,4.824)}
\gppoint{gp mark 0}{(5.249,4.453)}
\gppoint{gp mark 0}{(5.249,4.524)}
\gppoint{gp mark 0}{(5.249,4.314)}
\gppoint{gp mark 0}{(5.249,4.841)}
\gppoint{gp mark 0}{(5.249,4.638)}
\gppoint{gp mark 0}{(5.249,4.963)}
\gppoint{gp mark 0}{(5.249,4.774)}
\gppoint{gp mark 0}{(5.249,4.890)}
\gppoint{gp mark 0}{(5.249,4.762)}
\gppoint{gp mark 0}{(5.249,4.314)}
\gppoint{gp mark 0}{(5.249,4.776)}
\gppoint{gp mark 0}{(5.249,5.172)}
\gppoint{gp mark 0}{(5.249,5.281)}
\gppoint{gp mark 0}{(5.249,4.739)}
\gppoint{gp mark 0}{(5.249,4.515)}
\gppoint{gp mark 0}{(5.249,4.353)}
\gppoint{gp mark 0}{(5.249,4.870)}
\gppoint{gp mark 0}{(5.249,4.984)}
\gppoint{gp mark 0}{(5.249,5.335)}
\gppoint{gp mark 0}{(5.249,4.893)}
\gppoint{gp mark 0}{(5.249,4.245)}
\gppoint{gp mark 0}{(5.249,4.433)}
\gppoint{gp mark 0}{(5.249,4.518)}
\gppoint{gp mark 0}{(5.264,4.153)}
\gppoint{gp mark 0}{(5.264,4.478)}
\gppoint{gp mark 0}{(5.264,5.157)}
\gppoint{gp mark 0}{(5.264,4.357)}
\gppoint{gp mark 0}{(5.264,4.933)}
\gppoint{gp mark 0}{(5.264,5.426)}
\gppoint{gp mark 0}{(5.264,5.014)}
\gppoint{gp mark 0}{(5.264,4.365)}
\gppoint{gp mark 0}{(5.264,4.088)}
\gppoint{gp mark 0}{(5.264,4.259)}
\gppoint{gp mark 0}{(5.264,5.289)}
\gppoint{gp mark 0}{(5.264,4.500)}
\gppoint{gp mark 0}{(5.264,5.130)}
\gppoint{gp mark 0}{(5.264,5.432)}
\gppoint{gp mark 0}{(5.264,4.419)}
\gppoint{gp mark 0}{(5.264,4.478)}
\gppoint{gp mark 0}{(5.264,5.040)}
\gppoint{gp mark 0}{(5.264,4.236)}
\gppoint{gp mark 0}{(5.264,4.475)}
\gppoint{gp mark 0}{(5.264,4.527)}
\gppoint{gp mark 0}{(5.264,5.764)}
\gppoint{gp mark 0}{(5.264,5.081)}
\gppoint{gp mark 0}{(5.264,4.105)}
\gppoint{gp mark 0}{(5.264,5.415)}
\gppoint{gp mark 0}{(5.264,4.338)}
\gppoint{gp mark 0}{(5.264,4.208)}
\gppoint{gp mark 0}{(5.264,4.676)}
\gppoint{gp mark 0}{(5.264,4.931)}
\gppoint{gp mark 0}{(5.264,3.831)}
\gppoint{gp mark 0}{(5.264,4.365)}
\gppoint{gp mark 0}{(5.264,5.234)}
\gppoint{gp mark 0}{(5.264,5.716)}
\gppoint{gp mark 0}{(5.264,4.280)}
\gppoint{gp mark 0}{(5.264,5.234)}
\gppoint{gp mark 0}{(5.264,4.600)}
\gppoint{gp mark 0}{(5.264,5.400)}
\gppoint{gp mark 0}{(5.264,4.992)}
\gppoint{gp mark 0}{(5.264,5.184)}
\gppoint{gp mark 0}{(5.264,4.494)}
\gppoint{gp mark 0}{(5.264,4.334)}
\gppoint{gp mark 0}{(5.264,4.776)}
\gppoint{gp mark 0}{(5.264,4.841)}
\gppoint{gp mark 0}{(5.264,5.328)}
\gppoint{gp mark 0}{(5.264,4.380)}
\gppoint{gp mark 0}{(5.264,4.674)}
\gppoint{gp mark 0}{(5.264,4.433)}
\gppoint{gp mark 0}{(5.264,4.076)}
\gppoint{gp mark 0}{(5.264,5.184)}
\gppoint{gp mark 0}{(5.264,5.070)}
\gppoint{gp mark 0}{(5.264,4.929)}
\gppoint{gp mark 0}{(5.264,4.121)}
\gppoint{gp mark 0}{(5.278,5.017)}
\gppoint{gp mark 0}{(5.278,4.987)}
\gppoint{gp mark 0}{(5.278,5.157)}
\gppoint{gp mark 0}{(5.278,3.872)}
\gppoint{gp mark 0}{(5.278,4.232)}
\gppoint{gp mark 0}{(5.278,5.230)}
\gppoint{gp mark 0}{(5.278,4.116)}
\gppoint{gp mark 0}{(5.278,5.562)}
\gppoint{gp mark 0}{(5.278,5.311)}
\gppoint{gp mark 0}{(5.278,4.657)}
\gppoint{gp mark 0}{(5.278,4.276)}
\gppoint{gp mark 0}{(5.278,4.353)}
\gppoint{gp mark 0}{(5.278,5.144)}
\gppoint{gp mark 0}{(5.278,4.368)}
\gppoint{gp mark 0}{(5.278,4.164)}
\gppoint{gp mark 0}{(5.278,3.805)}
\gppoint{gp mark 0}{(5.278,5.096)}
\gppoint{gp mark 0}{(5.278,4.618)}
\gppoint{gp mark 0}{(5.278,5.097)}
\gppoint{gp mark 0}{(5.278,4.864)}
\gppoint{gp mark 0}{(5.278,4.453)}
\gppoint{gp mark 0}{(5.278,4.907)}
\gppoint{gp mark 0}{(5.278,4.429)}
\gppoint{gp mark 0}{(5.278,5.323)}
\gppoint{gp mark 0}{(5.278,5.050)}
\gppoint{gp mark 0}{(5.278,5.052)}
\gppoint{gp mark 0}{(5.278,4.179)}
\gppoint{gp mark 0}{(5.278,4.436)}
\gppoint{gp mark 0}{(5.278,5.140)}
\gppoint{gp mark 0}{(5.278,4.676)}
\gppoint{gp mark 0}{(5.278,3.805)}
\gppoint{gp mark 0}{(5.278,3.872)}
\gppoint{gp mark 0}{(5.278,5.132)}
\gppoint{gp mark 0}{(5.278,4.456)}
\gppoint{gp mark 0}{(5.278,4.550)}
\gppoint{gp mark 0}{(5.278,4.494)}
\gppoint{gp mark 0}{(5.278,5.641)}
\gppoint{gp mark 0}{(5.278,5.021)}
\gppoint{gp mark 0}{(5.278,4.792)}
\gppoint{gp mark 0}{(5.278,4.439)}
\gppoint{gp mark 0}{(5.278,4.218)}
\gppoint{gp mark 0}{(5.278,4.618)}
\gppoint{gp mark 0}{(5.278,4.846)}
\gppoint{gp mark 0}{(5.278,4.259)}
\gppoint{gp mark 0}{(5.278,4.702)}
\gppoint{gp mark 0}{(5.278,4.702)}
\gppoint{gp mark 0}{(5.278,5.227)}
\gppoint{gp mark 0}{(5.278,5.538)}
\gppoint{gp mark 0}{(5.278,4.387)}
\gppoint{gp mark 0}{(5.278,5.137)}
\gppoint{gp mark 0}{(5.278,4.488)}
\gppoint{gp mark 0}{(5.278,5.202)}
\gppoint{gp mark 0}{(5.278,4.372)}
\gppoint{gp mark 0}{(5.278,5.320)}
\gppoint{gp mark 0}{(5.278,4.276)}
\gppoint{gp mark 0}{(5.278,4.939)}
\gppoint{gp mark 0}{(5.278,4.895)}
\gppoint{gp mark 0}{(5.278,4.419)}
\gppoint{gp mark 0}{(5.278,4.575)}
\gppoint{gp mark 0}{(5.278,4.747)}
\gppoint{gp mark 0}{(5.292,5.082)}
\gppoint{gp mark 0}{(5.292,4.674)}
\gppoint{gp mark 0}{(5.292,5.354)}
\gppoint{gp mark 0}{(5.292,4.289)}
\gppoint{gp mark 0}{(5.292,5.664)}
\gppoint{gp mark 0}{(5.292,4.213)}
\gppoint{gp mark 0}{(5.292,4.679)}
\gppoint{gp mark 0}{(5.292,4.148)}
\gppoint{gp mark 0}{(5.292,4.796)}
\gppoint{gp mark 0}{(5.292,4.788)}
\gppoint{gp mark 0}{(5.292,5.086)}
\gppoint{gp mark 0}{(5.292,5.268)}
\gppoint{gp mark 0}{(5.292,4.559)}
\gppoint{gp mark 0}{(5.292,4.289)}
\gppoint{gp mark 0}{(5.292,5.501)}
\gppoint{gp mark 0}{(5.292,4.933)}
\gppoint{gp mark 0}{(5.292,4.472)}
\gppoint{gp mark 0}{(5.292,5.325)}
\gppoint{gp mark 0}{(5.292,4.618)}
\gppoint{gp mark 0}{(5.292,4.372)}
\gppoint{gp mark 0}{(5.292,4.821)}
\gppoint{gp mark 0}{(5.292,4.674)}
\gppoint{gp mark 0}{(5.292,4.584)}
\gppoint{gp mark 0}{(5.292,4.674)}
\gppoint{gp mark 0}{(5.292,4.811)}
\gppoint{gp mark 0}{(5.292,6.250)}
\gppoint{gp mark 0}{(5.292,5.736)}
\gppoint{gp mark 0}{(5.292,4.674)}
\gppoint{gp mark 0}{(5.292,5.403)}
\gppoint{gp mark 0}{(5.292,4.280)}
\gppoint{gp mark 0}{(5.292,4.743)}
\gppoint{gp mark 0}{(5.292,4.743)}
\gppoint{gp mark 0}{(5.292,4.969)}
\gppoint{gp mark 0}{(5.292,5.214)}
\gppoint{gp mark 0}{(5.292,4.667)}
\gppoint{gp mark 0}{(5.292,5.826)}
\gppoint{gp mark 0}{(5.292,4.586)}
\gppoint{gp mark 0}{(5.292,4.667)}
\gppoint{gp mark 0}{(5.292,5.064)}
\gppoint{gp mark 0}{(5.292,5.042)}
\gppoint{gp mark 0}{(5.292,5.064)}
\gppoint{gp mark 0}{(5.292,4.667)}
\gppoint{gp mark 0}{(5.292,4.947)}
\gppoint{gp mark 0}{(5.292,5.086)}
\gppoint{gp mark 0}{(5.292,4.330)}
\gppoint{gp mark 0}{(5.292,5.291)}
\gppoint{gp mark 0}{(5.292,5.430)}
\gppoint{gp mark 0}{(5.292,4.415)}
\gppoint{gp mark 0}{(5.292,5.000)}
\gppoint{gp mark 0}{(5.292,4.782)}
\gppoint{gp mark 0}{(5.292,4.845)}
\gppoint{gp mark 0}{(5.292,4.910)}
\gppoint{gp mark 0}{(5.292,4.804)}
\gppoint{gp mark 0}{(5.292,5.577)}
\gppoint{gp mark 0}{(5.292,5.159)}
\gppoint{gp mark 0}{(5.292,4.778)}
\gppoint{gp mark 0}{(5.306,4.846)}
\gppoint{gp mark 0}{(5.306,4.203)}
\gppoint{gp mark 0}{(5.306,4.272)}
\gppoint{gp mark 0}{(5.306,4.667)}
\gppoint{gp mark 0}{(5.306,4.272)}
\gppoint{gp mark 0}{(5.306,4.405)}
\gppoint{gp mark 0}{(5.306,5.217)}
\gppoint{gp mark 0}{(5.306,4.446)}
\gppoint{gp mark 0}{(5.306,4.743)}
\gppoint{gp mark 0}{(5.306,5.350)}
\gppoint{gp mark 0}{(5.306,4.977)}
\gppoint{gp mark 0}{(5.306,5.499)}
\gppoint{gp mark 0}{(5.306,5.121)}
\gppoint{gp mark 0}{(5.306,4.883)}
\gppoint{gp mark 0}{(5.306,4.954)}
\gppoint{gp mark 0}{(5.306,5.339)}
\gppoint{gp mark 0}{(5.306,4.845)}
\gppoint{gp mark 0}{(5.306,5.020)}
\gppoint{gp mark 0}{(5.306,5.058)}
\gppoint{gp mark 0}{(5.306,4.667)}
\gppoint{gp mark 0}{(5.306,5.473)}
\gppoint{gp mark 0}{(5.306,5.039)}
\gppoint{gp mark 0}{(5.306,4.556)}
\gppoint{gp mark 0}{(5.306,5.698)}
\gppoint{gp mark 0}{(5.306,4.478)}
\gppoint{gp mark 0}{(5.306,4.947)}
\gppoint{gp mark 0}{(5.306,4.953)}
\gppoint{gp mark 0}{(5.306,4.222)}
\gppoint{gp mark 0}{(5.306,4.310)}
\gppoint{gp mark 0}{(5.306,5.302)}
\gppoint{gp mark 0}{(5.306,5.316)}
\gppoint{gp mark 0}{(5.306,5.231)}
\gppoint{gp mark 0}{(5.306,4.947)}
\gppoint{gp mark 0}{(5.306,4.846)}
\gppoint{gp mark 0}{(5.306,4.365)}
\gppoint{gp mark 0}{(5.306,4.947)}
\gppoint{gp mark 0}{(5.306,4.194)}
\gppoint{gp mark 0}{(5.306,5.417)}
\gppoint{gp mark 0}{(5.306,4.854)}
\gppoint{gp mark 0}{(5.306,5.371)}
\gppoint{gp mark 0}{(5.306,5.217)}
\gppoint{gp mark 0}{(5.306,4.346)}
\gppoint{gp mark 0}{(5.306,4.456)}
\gppoint{gp mark 0}{(5.306,5.792)}
\gppoint{gp mark 0}{(5.306,5.311)}
\gppoint{gp mark 0}{(5.306,5.099)}
\gppoint{gp mark 0}{(5.306,5.322)}
\gppoint{gp mark 0}{(5.306,4.994)}
\gppoint{gp mark 0}{(5.306,4.880)}
\gppoint{gp mark 0}{(5.306,4.944)}
\gppoint{gp mark 0}{(5.306,4.635)}
\gppoint{gp mark 0}{(5.306,5.067)}
\gppoint{gp mark 0}{(5.306,4.272)}
\gppoint{gp mark 0}{(5.319,4.863)}
\gppoint{gp mark 0}{(5.319,4.655)}
\gppoint{gp mark 0}{(5.319,4.391)}
\gppoint{gp mark 0}{(5.319,4.121)}
\gppoint{gp mark 0}{(5.319,4.713)}
\gppoint{gp mark 0}{(5.319,5.416)}
\gppoint{gp mark 0}{(5.319,4.713)}
\gppoint{gp mark 0}{(5.319,4.739)}
\gppoint{gp mark 0}{(5.319,4.497)}
\gppoint{gp mark 0}{(5.319,4.960)}
\gppoint{gp mark 0}{(5.319,4.983)}
\gppoint{gp mark 0}{(5.319,4.306)}
\gppoint{gp mark 0}{(5.319,4.436)}
\gppoint{gp mark 0}{(5.319,4.581)}
\gppoint{gp mark 0}{(5.319,4.776)}
\gppoint{gp mark 0}{(5.319,4.923)}
\gppoint{gp mark 0}{(5.319,4.259)}
\gppoint{gp mark 0}{(5.319,5.097)}
\gppoint{gp mark 0}{(5.319,5.047)}
\gppoint{gp mark 0}{(5.319,5.097)}
\gppoint{gp mark 0}{(5.319,4.834)}
\gppoint{gp mark 0}{(5.319,5.238)}
\gppoint{gp mark 0}{(5.319,5.162)}
\gppoint{gp mark 0}{(5.319,5.511)}
\gppoint{gp mark 0}{(5.319,5.054)}
\gppoint{gp mark 0}{(5.319,5.437)}
\gppoint{gp mark 0}{(5.319,5.054)}
\gppoint{gp mark 0}{(5.319,4.469)}
\gppoint{gp mark 0}{(5.319,5.595)}
\gppoint{gp mark 0}{(5.319,4.802)}
\gppoint{gp mark 0}{(5.319,4.760)}
\gppoint{gp mark 0}{(5.319,5.056)}
\gppoint{gp mark 0}{(5.319,4.469)}
\gppoint{gp mark 0}{(5.319,5.078)}
\gppoint{gp mark 0}{(5.319,5.162)}
\gppoint{gp mark 0}{(5.319,5.293)}
\gppoint{gp mark 0}{(5.319,4.686)}
\gppoint{gp mark 0}{(5.319,5.097)}
\gppoint{gp mark 0}{(5.319,4.688)}
\gppoint{gp mark 0}{(5.319,4.834)}
\gppoint{gp mark 0}{(5.319,5.511)}
\gppoint{gp mark 0}{(5.319,5.435)}
\gppoint{gp mark 0}{(5.319,4.575)}
\gppoint{gp mark 0}{(5.319,5.540)}
\gppoint{gp mark 0}{(5.319,5.336)}
\gppoint{gp mark 0}{(5.319,5.044)}
\gppoint{gp mark 0}{(5.319,5.238)}
\gppoint{gp mark 0}{(5.319,5.197)}
\gppoint{gp mark 0}{(5.319,5.484)}
\gppoint{gp mark 0}{(5.319,4.713)}
\gppoint{gp mark 0}{(5.319,4.227)}
\gppoint{gp mark 0}{(5.319,5.238)}
\gppoint{gp mark 0}{(5.319,5.238)}
\gppoint{gp mark 0}{(5.319,4.969)}
\gppoint{gp mark 0}{(5.319,5.418)}
\gppoint{gp mark 0}{(5.319,4.713)}
\gppoint{gp mark 0}{(5.319,4.717)}
\gppoint{gp mark 0}{(5.319,4.189)}
\gppoint{gp mark 0}{(5.319,5.063)}
\gppoint{gp mark 0}{(5.319,4.859)}
\gppoint{gp mark 0}{(5.319,5.102)}
\gppoint{gp mark 0}{(5.332,5.717)}
\gppoint{gp mark 0}{(5.332,4.405)}
\gppoint{gp mark 0}{(5.332,4.850)}
\gppoint{gp mark 0}{(5.332,4.573)}
\gppoint{gp mark 0}{(5.332,5.656)}
\gppoint{gp mark 0}{(5.332,4.786)}
\gppoint{gp mark 0}{(5.332,4.635)}
\gppoint{gp mark 0}{(5.332,4.702)}
\gppoint{gp mark 0}{(5.332,5.242)}
\gppoint{gp mark 0}{(5.332,4.512)}
\gppoint{gp mark 0}{(5.332,5.512)}
\gppoint{gp mark 0}{(5.332,4.774)}
\gppoint{gp mark 0}{(5.332,4.848)}
\gppoint{gp mark 0}{(5.332,4.153)}
\gppoint{gp mark 0}{(5.332,5.184)}
\gppoint{gp mark 0}{(5.332,4.686)}
\gppoint{gp mark 0}{(5.332,4.297)}
\gppoint{gp mark 0}{(5.332,5.681)}
\gppoint{gp mark 0}{(5.332,5.128)}
\gppoint{gp mark 0}{(5.332,4.153)}
\gppoint{gp mark 0}{(5.332,4.620)}
\gppoint{gp mark 0}{(5.332,5.272)}
\gppoint{gp mark 0}{(5.332,4.920)}
\gppoint{gp mark 0}{(5.332,5.184)}
\gppoint{gp mark 0}{(5.332,5.203)}
\gppoint{gp mark 0}{(5.332,4.772)}
\gppoint{gp mark 0}{(5.332,5.454)}
\gppoint{gp mark 0}{(5.332,4.686)}
\gppoint{gp mark 0}{(5.332,4.153)}
\gppoint{gp mark 0}{(5.332,4.794)}
\gppoint{gp mark 0}{(5.332,5.236)}
\gppoint{gp mark 0}{(5.332,4.143)}
\gppoint{gp mark 0}{(5.332,5.046)}
\gppoint{gp mark 0}{(5.332,5.342)}
\gppoint{gp mark 0}{(5.332,5.025)}
\gppoint{gp mark 0}{(5.332,4.241)}
\gppoint{gp mark 0}{(5.332,5.889)}
\gppoint{gp mark 0}{(5.332,4.153)}
\gppoint{gp mark 0}{(5.332,5.060)}
\gppoint{gp mark 0}{(5.332,4.708)}
\gppoint{gp mark 0}{(5.332,5.369)}
\gppoint{gp mark 0}{(5.332,4.807)}
\gppoint{gp mark 0}{(5.332,4.567)}
\gppoint{gp mark 0}{(5.332,4.846)}
\gppoint{gp mark 0}{(5.332,5.291)}
\gppoint{gp mark 0}{(5.332,4.459)}
\gppoint{gp mark 0}{(5.346,5.205)}
\gppoint{gp mark 0}{(5.346,5.065)}
\gppoint{gp mark 0}{(5.346,4.194)}
\gppoint{gp mark 0}{(5.346,5.079)}
\gppoint{gp mark 0}{(5.346,5.262)}
\gppoint{gp mark 0}{(5.346,4.953)}
\gppoint{gp mark 0}{(5.346,4.491)}
\gppoint{gp mark 0}{(5.346,4.394)}
\gppoint{gp mark 0}{(5.346,4.132)}
\gppoint{gp mark 0}{(5.346,4.241)}
\gppoint{gp mark 0}{(5.346,4.562)}
\gppoint{gp mark 0}{(5.346,5.371)}
\gppoint{gp mark 0}{(5.346,4.875)}
\gppoint{gp mark 0}{(5.346,4.676)}
\gppoint{gp mark 0}{(5.346,5.434)}
\gppoint{gp mark 0}{(5.346,4.999)}
\gppoint{gp mark 0}{(5.346,5.044)}
\gppoint{gp mark 0}{(5.346,5.443)}
\gppoint{gp mark 0}{(5.346,5.142)}
\gppoint{gp mark 0}{(5.346,5.491)}
\gppoint{gp mark 0}{(5.346,5.137)}
\gppoint{gp mark 0}{(5.346,5.371)}
\gppoint{gp mark 0}{(5.346,4.542)}
\gppoint{gp mark 0}{(5.346,4.728)}
\gppoint{gp mark 0}{(5.346,4.203)}
\gppoint{gp mark 0}{(5.346,5.169)}
\gppoint{gp mark 0}{(5.346,4.953)}
\gppoint{gp mark 0}{(5.346,4.562)}
\gppoint{gp mark 0}{(5.346,4.953)}
\gppoint{gp mark 0}{(5.346,4.953)}
\gppoint{gp mark 0}{(5.346,5.248)}
\gppoint{gp mark 0}{(5.346,4.953)}
\gppoint{gp mark 0}{(5.346,4.863)}
\gppoint{gp mark 0}{(5.346,4.845)}
\gppoint{gp mark 0}{(5.346,4.782)}
\gppoint{gp mark 0}{(5.346,5.331)}
\gppoint{gp mark 0}{(5.346,5.739)}
\gppoint{gp mark 0}{(5.346,5.449)}
\gppoint{gp mark 0}{(5.346,4.953)}
\gppoint{gp mark 0}{(5.346,4.485)}
\gppoint{gp mark 0}{(5.346,5.034)}
\gppoint{gp mark 0}{(5.346,4.254)}
\gppoint{gp mark 0}{(5.346,4.953)}
\gppoint{gp mark 0}{(5.346,5.222)}
\gppoint{gp mark 0}{(5.346,4.391)}
\gppoint{gp mark 0}{(5.346,4.758)}
\gppoint{gp mark 0}{(5.346,4.959)}
\gppoint{gp mark 0}{(5.346,4.650)}
\gppoint{gp mark 0}{(5.346,5.331)}
\gppoint{gp mark 0}{(5.346,4.116)}
\gppoint{gp mark 0}{(5.346,5.303)}
\gppoint{gp mark 0}{(5.346,5.690)}
\gppoint{gp mark 0}{(5.346,5.162)}
\gppoint{gp mark 0}{(5.346,5.067)}
\gppoint{gp mark 0}{(5.346,4.222)}
\gppoint{gp mark 0}{(5.346,5.485)}
\gppoint{gp mark 0}{(5.346,5.151)}
\gppoint{gp mark 0}{(5.346,5.476)}
\gppoint{gp mark 0}{(5.346,5.248)}
\gppoint{gp mark 0}{(5.346,5.637)}
\gppoint{gp mark 0}{(5.346,5.368)}
\gppoint{gp mark 0}{(5.346,4.788)}
\gppoint{gp mark 0}{(5.346,5.009)}
\gppoint{gp mark 0}{(5.359,4.980)}
\gppoint{gp mark 0}{(5.359,4.948)}
\gppoint{gp mark 0}{(5.359,4.843)}
\gppoint{gp mark 0}{(5.359,4.469)}
\gppoint{gp mark 0}{(5.359,4.469)}
\gppoint{gp mark 0}{(5.359,3.961)}
\gppoint{gp mark 0}{(5.359,4.326)}
\gppoint{gp mark 0}{(5.359,5.197)}
\gppoint{gp mark 0}{(5.359,5.060)}
\gppoint{gp mark 0}{(5.359,4.143)}
\gppoint{gp mark 0}{(5.359,4.143)}
\gppoint{gp mark 0}{(5.359,4.944)}
\gppoint{gp mark 0}{(5.359,4.944)}
\gppoint{gp mark 0}{(5.359,4.306)}
\gppoint{gp mark 0}{(5.359,4.921)}
\gppoint{gp mark 0}{(5.359,4.542)}
\gppoint{gp mark 0}{(5.359,5.016)}
\gppoint{gp mark 0}{(5.359,4.667)}
\gppoint{gp mark 0}{(5.359,4.845)}
\gppoint{gp mark 0}{(5.359,4.950)}
\gppoint{gp mark 0}{(5.359,4.530)}
\gppoint{gp mark 0}{(5.359,5.096)}
\gppoint{gp mark 0}{(5.359,5.892)}
\gppoint{gp mark 0}{(5.359,4.368)}
\gppoint{gp mark 0}{(5.359,5.180)}
\gppoint{gp mark 0}{(5.359,5.364)}
\gppoint{gp mark 0}{(5.359,5.061)}
\gppoint{gp mark 0}{(5.359,4.506)}
\gppoint{gp mark 0}{(5.359,5.086)}
\gppoint{gp mark 0}{(5.359,5.046)}
\gppoint{gp mark 0}{(5.359,4.784)}
\gppoint{gp mark 0}{(5.359,5.611)}
\gppoint{gp mark 0}{(5.359,5.024)}
\gppoint{gp mark 0}{(5.359,5.488)}
\gppoint{gp mark 0}{(5.359,4.365)}
\gppoint{gp mark 0}{(5.359,4.855)}
\gppoint{gp mark 0}{(5.359,4.693)}
\gppoint{gp mark 0}{(5.359,5.281)}
\gppoint{gp mark 0}{(5.359,4.143)}
\gppoint{gp mark 0}{(5.359,5.585)}
\gppoint{gp mark 0}{(5.359,4.735)}
\gppoint{gp mark 0}{(5.359,5.140)}
\gppoint{gp mark 0}{(5.359,4.802)}
\gppoint{gp mark 0}{(5.359,4.625)}
\gppoint{gp mark 0}{(5.359,5.245)}
\gppoint{gp mark 0}{(5.359,5.210)}
\gppoint{gp mark 0}{(5.359,4.285)}
\gppoint{gp mark 0}{(5.371,5.094)}
\gppoint{gp mark 0}{(5.371,5.556)}
\gppoint{gp mark 0}{(5.371,5.094)}
\gppoint{gp mark 0}{(5.371,3.911)}
\gppoint{gp mark 0}{(5.371,5.152)}
\gppoint{gp mark 0}{(5.371,4.928)}
\gppoint{gp mark 0}{(5.371,4.839)}
\gppoint{gp mark 0}{(5.371,4.203)}
\gppoint{gp mark 0}{(5.371,5.449)}
\gppoint{gp mark 0}{(5.371,5.521)}
\gppoint{gp mark 0}{(5.371,6.220)}
\gppoint{gp mark 0}{(5.371,4.158)}
\gppoint{gp mark 0}{(5.371,5.353)}
\gppoint{gp mark 0}{(5.371,5.449)}
\gppoint{gp mark 0}{(5.371,4.613)}
\gppoint{gp mark 0}{(5.371,4.625)}
\gppoint{gp mark 0}{(5.371,4.902)}
\gppoint{gp mark 0}{(5.371,3.911)}
\gppoint{gp mark 0}{(5.371,4.875)}
\gppoint{gp mark 0}{(5.371,4.052)}
\gppoint{gp mark 0}{(5.371,5.668)}
\gppoint{gp mark 0}{(5.371,4.401)}
\gppoint{gp mark 0}{(5.371,5.364)}
\gppoint{gp mark 0}{(5.371,5.955)}
\gppoint{gp mark 0}{(5.371,5.461)}
\gppoint{gp mark 0}{(5.371,5.629)}
\gppoint{gp mark 0}{(5.371,5.426)}
\gppoint{gp mark 0}{(5.371,5.116)}
\gppoint{gp mark 0}{(5.371,4.913)}
\gppoint{gp mark 0}{(5.371,4.478)}
\gppoint{gp mark 0}{(5.371,5.383)}
\gppoint{gp mark 0}{(5.371,4.110)}
\gppoint{gp mark 0}{(5.371,3.759)}
\gppoint{gp mark 0}{(5.371,4.449)}
\gppoint{gp mark 0}{(5.371,4.898)}
\gppoint{gp mark 0}{(5.371,4.875)}
\gppoint{gp mark 0}{(5.371,4.650)}
\gppoint{gp mark 0}{(5.371,5.168)}
\gppoint{gp mark 0}{(5.371,4.739)}
\gppoint{gp mark 0}{(5.371,5.523)}
\gppoint{gp mark 0}{(5.371,4.592)}
\gppoint{gp mark 0}{(5.371,4.850)}
\gppoint{gp mark 0}{(5.371,5.331)}
\gppoint{gp mark 0}{(5.371,5.084)}
\gppoint{gp mark 0}{(5.371,4.940)}
\gppoint{gp mark 0}{(5.371,4.937)}
\gppoint{gp mark 0}{(5.371,4.213)}
\gppoint{gp mark 0}{(5.371,5.345)}
\gppoint{gp mark 0}{(5.371,4.702)}
\gppoint{gp mark 0}{(5.371,5.893)}
\gppoint{gp mark 0}{(5.371,5.109)}
\gppoint{gp mark 0}{(5.371,4.876)}
\gppoint{gp mark 0}{(5.384,4.391)}
\gppoint{gp mark 0}{(5.384,4.737)}
\gppoint{gp mark 0}{(5.384,5.262)}
\gppoint{gp mark 0}{(5.384,4.153)}
\gppoint{gp mark 0}{(5.384,5.381)}
\gppoint{gp mark 0}{(5.384,4.082)}
\gppoint{gp mark 0}{(5.384,4.419)}
\gppoint{gp mark 0}{(5.384,4.843)}
\gppoint{gp mark 0}{(5.384,4.469)}
\gppoint{gp mark 0}{(5.384,4.672)}
\gppoint{gp mark 0}{(5.384,4.241)}
\gppoint{gp mark 0}{(5.384,5.341)}
\gppoint{gp mark 0}{(5.384,4.174)}
\gppoint{gp mark 0}{(5.384,4.730)}
\gppoint{gp mark 0}{(5.384,4.807)}
\gppoint{gp mark 0}{(5.384,4.082)}
\gppoint{gp mark 0}{(5.384,5.201)}
\gppoint{gp mark 0}{(5.384,5.009)}
\gppoint{gp mark 0}{(5.384,5.461)}
\gppoint{gp mark 0}{(5.384,5.831)}
\gppoint{gp mark 0}{(5.384,5.394)}
\gppoint{gp mark 0}{(5.384,4.679)}
\gppoint{gp mark 0}{(5.384,5.223)}
\gppoint{gp mark 0}{(5.384,4.330)}
\gppoint{gp mark 0}{(5.384,4.241)}
\gppoint{gp mark 0}{(5.384,4.401)}
\gppoint{gp mark 0}{(5.384,5.310)}
\gppoint{gp mark 0}{(5.384,5.760)}
\gppoint{gp mark 0}{(5.384,5.249)}
\gppoint{gp mark 0}{(5.384,5.635)}
\gppoint{gp mark 0}{(5.384,4.236)}
\gppoint{gp mark 0}{(5.384,4.194)}
\gppoint{gp mark 0}{(5.384,5.635)}
\gppoint{gp mark 0}{(5.384,4.208)}
\gppoint{gp mark 0}{(5.384,5.086)}
\gppoint{gp mark 0}{(5.384,4.792)}
\gppoint{gp mark 0}{(5.384,5.451)}
\gppoint{gp mark 0}{(5.384,5.223)}
\gppoint{gp mark 0}{(5.384,4.882)}
\gppoint{gp mark 0}{(5.396,4.052)}
\gppoint{gp mark 0}{(5.396,4.635)}
\gppoint{gp mark 0}{(5.396,4.762)}
\gppoint{gp mark 0}{(5.396,4.762)}
\gppoint{gp mark 0}{(5.396,4.786)}
\gppoint{gp mark 0}{(5.396,5.197)}
\gppoint{gp mark 0}{(5.396,4.897)}
\gppoint{gp mark 0}{(5.396,4.419)}
\gppoint{gp mark 0}{(5.396,5.068)}
\gppoint{gp mark 0}{(5.396,5.586)}
\gppoint{gp mark 0}{(5.396,5.139)}
\gppoint{gp mark 0}{(5.396,4.289)}
\gppoint{gp mark 0}{(5.396,5.120)}
\gppoint{gp mark 0}{(5.396,4.665)}
\gppoint{gp mark 0}{(5.396,4.915)}
\gppoint{gp mark 0}{(5.396,4.052)}
\gppoint{gp mark 0}{(5.396,5.120)}
\gppoint{gp mark 0}{(5.396,5.120)}
\gppoint{gp mark 0}{(5.396,5.358)}
\gppoint{gp mark 0}{(5.396,4.052)}
\gppoint{gp mark 0}{(5.396,4.506)}
\gppoint{gp mark 0}{(5.396,5.426)}
\gppoint{gp mark 0}{(5.396,5.285)}
\gppoint{gp mark 0}{(5.396,5.123)}
\gppoint{gp mark 0}{(5.396,5.063)}
\gppoint{gp mark 0}{(5.396,4.052)}
\gppoint{gp mark 0}{(5.396,5.270)}
\gppoint{gp mark 0}{(5.396,4.433)}
\gppoint{gp mark 0}{(5.396,5.061)}
\gppoint{gp mark 0}{(5.396,4.052)}
\gppoint{gp mark 0}{(5.396,4.488)}
\gppoint{gp mark 0}{(5.396,5.121)}
\gppoint{gp mark 0}{(5.396,5.063)}
\gppoint{gp mark 0}{(5.396,5.501)}
\gppoint{gp mark 0}{(5.396,4.999)}
\gppoint{gp mark 0}{(5.396,4.052)}
\gppoint{gp mark 0}{(5.396,4.774)}
\gppoint{gp mark 0}{(5.396,4.052)}
\gppoint{gp mark 0}{(5.396,4.052)}
\gppoint{gp mark 0}{(5.396,4.052)}
\gppoint{gp mark 0}{(5.396,5.366)}
\gppoint{gp mark 0}{(5.396,4.052)}
\gppoint{gp mark 0}{(5.396,4.975)}
\gppoint{gp mark 0}{(5.396,5.477)}
\gppoint{gp mark 0}{(5.396,4.353)}
\gppoint{gp mark 0}{(5.396,5.393)}
\gppoint{gp mark 0}{(5.396,4.289)}
\gppoint{gp mark 0}{(5.396,5.631)}
\gppoint{gp mark 0}{(5.396,5.036)}
\gppoint{gp mark 0}{(5.409,5.479)}
\gppoint{gp mark 0}{(5.409,5.610)}
\gppoint{gp mark 0}{(5.409,5.239)}
\gppoint{gp mark 0}{(5.409,4.841)}
\gppoint{gp mark 0}{(5.409,5.237)}
\gppoint{gp mark 0}{(5.409,4.957)}
\gppoint{gp mark 0}{(5.409,5.331)}
\gppoint{gp mark 0}{(5.409,4.592)}
\gppoint{gp mark 0}{(5.409,5.510)}
\gppoint{gp mark 0}{(5.409,4.052)}
\gppoint{gp mark 0}{(5.409,5.159)}
\gppoint{gp mark 0}{(5.409,4.052)}
\gppoint{gp mark 0}{(5.409,4.515)}
\gppoint{gp mark 0}{(5.409,5.740)}
\gppoint{gp mark 0}{(5.409,5.232)}
\gppoint{gp mark 0}{(5.409,4.623)}
\gppoint{gp mark 0}{(5.409,5.543)}
\gppoint{gp mark 0}{(5.409,5.029)}
\gppoint{gp mark 0}{(5.409,5.273)}
\gppoint{gp mark 0}{(5.409,4.121)}
\gppoint{gp mark 0}{(5.409,4.485)}
\gppoint{gp mark 0}{(5.409,5.076)}
\gppoint{gp mark 0}{(5.409,4.674)}
\gppoint{gp mark 0}{(5.409,5.453)}
\gppoint{gp mark 0}{(5.409,5.640)}
\gppoint{gp mark 0}{(5.409,5.167)}
\gppoint{gp mark 0}{(5.409,4.052)}
\gppoint{gp mark 0}{(5.409,4.099)}
\gppoint{gp mark 0}{(5.409,5.070)}
\gppoint{gp mark 0}{(5.409,4.796)}
\gppoint{gp mark 0}{(5.409,5.440)}
\gppoint{gp mark 0}{(5.409,4.052)}
\gppoint{gp mark 0}{(5.409,4.213)}
\gppoint{gp mark 0}{(5.409,5.097)}
\gppoint{gp mark 0}{(5.409,4.052)}
\gppoint{gp mark 0}{(5.409,4.747)}
\gppoint{gp mark 0}{(5.409,5.614)}
\gppoint{gp mark 0}{(5.409,5.153)}
\gppoint{gp mark 0}{(5.409,5.077)}
\gppoint{gp mark 0}{(5.409,4.052)}
\gppoint{gp mark 0}{(5.409,4.456)}
\gppoint{gp mark 0}{(5.409,5.060)}
\gppoint{gp mark 0}{(5.409,5.249)}
\gppoint{gp mark 0}{(5.409,4.912)}
\gppoint{gp mark 0}{(5.421,5.693)}
\gppoint{gp mark 0}{(5.421,5.180)}
\gppoint{gp mark 0}{(5.421,5.468)}
\gppoint{gp mark 0}{(5.421,4.969)}
\gppoint{gp mark 0}{(5.421,5.334)}
\gppoint{gp mark 0}{(5.421,4.548)}
\gppoint{gp mark 0}{(5.421,4.052)}
\gppoint{gp mark 0}{(5.421,5.507)}
\gppoint{gp mark 0}{(5.421,5.537)}
\gppoint{gp mark 0}{(5.421,4.719)}
\gppoint{gp mark 0}{(5.421,5.148)}
\gppoint{gp mark 0}{(5.421,4.715)}
\gppoint{gp mark 0}{(5.421,5.104)}
\gppoint{gp mark 0}{(5.421,5.509)}
\gppoint{gp mark 0}{(5.421,4.365)}
\gppoint{gp mark 0}{(5.421,4.784)}
\gppoint{gp mark 0}{(5.421,4.052)}
\gppoint{gp mark 0}{(5.421,4.724)}
\gppoint{gp mark 0}{(5.421,4.052)}
\gppoint{gp mark 0}{(5.421,5.029)}
\gppoint{gp mark 0}{(5.421,5.704)}
\gppoint{gp mark 0}{(5.421,4.974)}
\gppoint{gp mark 0}{(5.421,4.052)}
\gppoint{gp mark 0}{(5.421,5.130)}
\gppoint{gp mark 0}{(5.421,5.684)}
\gppoint{gp mark 0}{(5.421,5.507)}
\gppoint{gp mark 0}{(5.421,5.059)}
\gppoint{gp mark 0}{(5.421,5.067)}
\gppoint{gp mark 0}{(5.421,5.048)}
\gppoint{gp mark 0}{(5.421,4.478)}
\gppoint{gp mark 0}{(5.421,5.114)}
\gppoint{gp mark 0}{(5.421,5.067)}
\gppoint{gp mark 0}{(5.421,4.986)}
\gppoint{gp mark 0}{(5.433,5.241)}
\gppoint{gp mark 0}{(5.433,5.123)}
\gppoint{gp mark 0}{(5.433,5.177)}
\gppoint{gp mark 0}{(5.433,4.686)}
\gppoint{gp mark 0}{(5.433,5.029)}
\gppoint{gp mark 0}{(5.433,5.196)}
\gppoint{gp mark 0}{(5.433,5.108)}
\gppoint{gp mark 0}{(5.433,5.862)}
\gppoint{gp mark 0}{(5.433,5.220)}
\gppoint{gp mark 0}{(5.433,5.140)}
\gppoint{gp mark 0}{(5.433,5.209)}
\gppoint{gp mark 0}{(5.433,5.140)}
\gppoint{gp mark 0}{(5.433,5.550)}
\gppoint{gp mark 0}{(5.433,5.233)}
\gppoint{gp mark 0}{(5.433,4.937)}
\gppoint{gp mark 0}{(5.433,5.140)}
\gppoint{gp mark 0}{(5.433,4.310)}
\gppoint{gp mark 0}{(5.433,4.456)}
\gppoint{gp mark 0}{(5.433,4.660)}
\gppoint{gp mark 0}{(5.433,4.760)}
\gppoint{gp mark 0}{(5.433,5.067)}
\gppoint{gp mark 0}{(5.433,5.146)}
\gppoint{gp mark 0}{(5.433,5.317)}
\gppoint{gp mark 0}{(5.433,5.092)}
\gppoint{gp mark 0}{(5.433,4.436)}
\gppoint{gp mark 0}{(5.433,5.078)}
\gppoint{gp mark 0}{(5.433,5.096)}
\gppoint{gp mark 0}{(5.433,5.241)}
\gppoint{gp mark 0}{(5.433,5.134)}
\gppoint{gp mark 0}{(5.433,5.028)}
\gppoint{gp mark 0}{(5.433,5.065)}
\gppoint{gp mark 0}{(5.433,5.580)}
\gppoint{gp mark 0}{(5.433,4.521)}
\gppoint{gp mark 0}{(5.433,5.044)}
\gppoint{gp mark 0}{(5.445,5.782)}
\gppoint{gp mark 0}{(5.445,4.586)}
\gppoint{gp mark 0}{(5.445,4.475)}
\gppoint{gp mark 0}{(5.445,4.986)}
\gppoint{gp mark 0}{(5.445,4.610)}
\gppoint{gp mark 0}{(5.445,4.322)}
\gppoint{gp mark 0}{(5.445,5.148)}
\gppoint{gp mark 0}{(5.445,5.690)}
\gppoint{gp mark 0}{(5.445,5.681)}
\gppoint{gp mark 0}{(5.445,5.690)}
\gppoint{gp mark 0}{(5.445,4.786)}
\gppoint{gp mark 0}{(5.445,4.713)}
\gppoint{gp mark 0}{(5.445,5.354)}
\gppoint{gp mark 0}{(5.445,4.713)}
\gppoint{gp mark 0}{(5.445,4.380)}
\gppoint{gp mark 0}{(5.445,5.168)}
\gppoint{gp mark 0}{(5.445,4.866)}
\gppoint{gp mark 0}{(5.445,4.975)}
\gppoint{gp mark 0}{(5.445,4.058)}
\gppoint{gp mark 0}{(5.445,4.607)}
\gppoint{gp mark 0}{(5.445,4.888)}
\gppoint{gp mark 0}{(5.445,4.807)}
\gppoint{gp mark 0}{(5.445,5.483)}
\gppoint{gp mark 0}{(5.445,4.584)}
\gppoint{gp mark 0}{(5.445,5.989)}
\gppoint{gp mark 0}{(5.445,4.971)}
\gppoint{gp mark 0}{(5.445,4.674)}
\gppoint{gp mark 0}{(5.445,5.044)}
\gppoint{gp mark 0}{(5.445,4.169)}
\gppoint{gp mark 0}{(5.445,4.674)}
\gppoint{gp mark 0}{(5.445,5.417)}
\gppoint{gp mark 0}{(5.445,5.067)}
\gppoint{gp mark 0}{(5.445,5.117)}
\gppoint{gp mark 0}{(5.445,4.897)}
\gppoint{gp mark 0}{(5.445,4.570)}
\gppoint{gp mark 0}{(5.445,5.025)}
\gppoint{gp mark 0}{(5.445,5.395)}
\gppoint{gp mark 0}{(5.445,5.395)}
\gppoint{gp mark 0}{(5.445,4.539)}
\gppoint{gp mark 0}{(5.445,5.205)}
\gppoint{gp mark 0}{(5.445,5.566)}
\gppoint{gp mark 0}{(5.445,4.704)}
\gppoint{gp mark 0}{(5.445,4.650)}
\gppoint{gp mark 0}{(5.445,4.478)}
\gppoint{gp mark 0}{(5.445,5.011)}
\gppoint{gp mark 0}{(5.445,5.205)}
\gppoint{gp mark 0}{(5.445,4.607)}
\gppoint{gp mark 0}{(5.445,4.702)}
\gppoint{gp mark 0}{(5.445,5.439)}
\gppoint{gp mark 0}{(5.445,5.566)}
\gppoint{gp mark 0}{(5.445,4.415)}
\gppoint{gp mark 0}{(5.445,5.312)}
\gppoint{gp mark 0}{(5.445,5.018)}
\gppoint{gp mark 0}{(5.456,5.237)}
\gppoint{gp mark 0}{(5.456,5.155)}
\gppoint{gp mark 0}{(5.456,4.697)}
\gppoint{gp mark 0}{(5.456,4.539)}
\gppoint{gp mark 0}{(5.456,4.780)}
\gppoint{gp mark 0}{(5.456,4.254)}
\gppoint{gp mark 0}{(5.456,5.518)}
\gppoint{gp mark 0}{(5.456,4.357)}
\gppoint{gp mark 0}{(5.456,4.676)}
\gppoint{gp mark 0}{(5.456,4.676)}
\gppoint{gp mark 0}{(5.456,4.570)}
\gppoint{gp mark 0}{(5.456,4.488)}
\gppoint{gp mark 0}{(5.456,5.040)}
\gppoint{gp mark 0}{(5.456,5.696)}
\gppoint{gp mark 0}{(5.456,5.193)}
\gppoint{gp mark 0}{(5.456,5.518)}
\gppoint{gp mark 0}{(5.456,4.589)}
\gppoint{gp mark 0}{(5.456,4.415)}
\gppoint{gp mark 0}{(5.456,4.353)}
\gppoint{gp mark 0}{(5.456,5.212)}
\gppoint{gp mark 0}{(5.456,5.330)}
\gppoint{gp mark 0}{(5.456,5.523)}
\gppoint{gp mark 0}{(5.456,5.495)}
\gppoint{gp mark 0}{(5.456,4.693)}
\gppoint{gp mark 0}{(5.456,5.262)}
\gppoint{gp mark 0}{(5.456,5.143)}
\gppoint{gp mark 0}{(5.456,5.412)}
\gppoint{gp mark 0}{(5.456,5.014)}
\gppoint{gp mark 0}{(5.456,5.211)}
\gppoint{gp mark 0}{(5.456,5.566)}
\gppoint{gp mark 0}{(5.456,5.081)}
\gppoint{gp mark 0}{(5.456,5.054)}
\gppoint{gp mark 0}{(5.456,4.963)}
\gppoint{gp mark 0}{(5.456,4.245)}
\gppoint{gp mark 0}{(5.456,5.514)}
\gppoint{gp mark 0}{(5.456,4.564)}
\gppoint{gp mark 0}{(5.456,4.805)}
\gppoint{gp mark 0}{(5.456,5.542)}
\gppoint{gp mark 0}{(5.456,4.800)}
\gppoint{gp mark 0}{(5.456,4.790)}
\gppoint{gp mark 0}{(5.456,5.165)}
\gppoint{gp mark 0}{(5.456,4.953)}
\gppoint{gp mark 0}{(5.456,5.234)}
\gppoint{gp mark 0}{(5.456,5.572)}
\gppoint{gp mark 0}{(5.456,4.533)}
\gppoint{gp mark 0}{(5.456,4.365)}
\gppoint{gp mark 0}{(5.456,4.553)}
\gppoint{gp mark 0}{(5.468,5.395)}
\gppoint{gp mark 0}{(5.468,4.334)}
\gppoint{gp mark 0}{(5.468,5.587)}
\gppoint{gp mark 0}{(5.468,4.391)}
\gppoint{gp mark 0}{(5.468,5.216)}
\gppoint{gp mark 0}{(5.468,4.310)}
\gppoint{gp mark 0}{(5.468,5.739)}
\gppoint{gp mark 0}{(5.468,5.098)}
\gppoint{gp mark 0}{(5.468,5.039)}
\gppoint{gp mark 0}{(5.468,4.939)}
\gppoint{gp mark 0}{(5.468,5.683)}
\gppoint{gp mark 0}{(5.468,5.358)}
\gppoint{gp mark 0}{(5.468,6.038)}
\gppoint{gp mark 0}{(5.468,5.611)}
\gppoint{gp mark 0}{(5.468,4.850)}
\gppoint{gp mark 0}{(5.468,4.625)}
\gppoint{gp mark 0}{(5.468,4.419)}
\gppoint{gp mark 0}{(5.468,5.327)}
\gppoint{gp mark 0}{(5.468,5.391)}
\gppoint{gp mark 0}{(5.468,4.880)}
\gppoint{gp mark 0}{(5.468,4.766)}
\gppoint{gp mark 0}{(5.468,6.025)}
\gppoint{gp mark 0}{(5.468,4.936)}
\gppoint{gp mark 0}{(5.468,4.158)}
\gppoint{gp mark 0}{(5.468,4.466)}
\gppoint{gp mark 0}{(5.468,5.358)}
\gppoint{gp mark 0}{(5.468,5.250)}
\gppoint{gp mark 0}{(5.468,5.673)}
\gppoint{gp mark 0}{(5.468,5.342)}
\gppoint{gp mark 0}{(5.468,4.968)}
\gppoint{gp mark 0}{(5.468,5.673)}
\gppoint{gp mark 0}{(5.468,5.645)}
\gppoint{gp mark 0}{(5.468,4.824)}
\gppoint{gp mark 0}{(5.468,5.241)}
\gppoint{gp mark 0}{(5.468,5.214)}
\gppoint{gp mark 0}{(5.468,5.282)}
\gppoint{gp mark 0}{(5.468,5.648)}
\gppoint{gp mark 0}{(5.468,5.386)}
\gppoint{gp mark 0}{(5.468,4.774)}
\gppoint{gp mark 0}{(5.468,5.160)}
\gppoint{gp mark 0}{(5.479,5.010)}
\gppoint{gp mark 0}{(5.479,5.243)}
\gppoint{gp mark 0}{(5.479,5.181)}
\gppoint{gp mark 0}{(5.479,5.874)}
\gppoint{gp mark 0}{(5.479,5.434)}
\gppoint{gp mark 0}{(5.479,5.433)}
\gppoint{gp mark 0}{(5.479,4.365)}
\gppoint{gp mark 0}{(5.479,4.357)}
\gppoint{gp mark 0}{(5.479,4.821)}
\gppoint{gp mark 0}{(5.479,5.520)}
\gppoint{gp mark 0}{(5.479,5.254)}
\gppoint{gp mark 0}{(5.479,4.443)}
\gppoint{gp mark 0}{(5.479,6.126)}
\gppoint{gp mark 0}{(5.479,5.790)}
\gppoint{gp mark 0}{(5.479,5.533)}
\gppoint{gp mark 0}{(5.479,4.807)}
\gppoint{gp mark 0}{(5.479,4.882)}
\gppoint{gp mark 0}{(5.479,5.004)}
\gppoint{gp mark 0}{(5.479,5.584)}
\gppoint{gp mark 0}{(5.479,5.498)}
\gppoint{gp mark 0}{(5.479,5.865)}
\gppoint{gp mark 0}{(5.479,4.539)}
\gppoint{gp mark 0}{(5.479,4.697)}
\gppoint{gp mark 0}{(5.479,5.477)}
\gppoint{gp mark 0}{(5.479,5.754)}
\gppoint{gp mark 0}{(5.479,4.597)}
\gppoint{gp mark 0}{(5.479,5.643)}
\gppoint{gp mark 0}{(5.479,4.665)}
\gppoint{gp mark 0}{(5.479,4.846)}
\gppoint{gp mark 0}{(5.479,4.969)}
\gppoint{gp mark 0}{(5.479,5.943)}
\gppoint{gp mark 0}{(5.479,4.786)}
\gppoint{gp mark 0}{(5.479,5.577)}
\gppoint{gp mark 0}{(5.479,4.527)}
\gppoint{gp mark 0}{(5.479,4.570)}
\gppoint{gp mark 0}{(5.479,4.717)}
\gppoint{gp mark 0}{(5.479,4.917)}
\gppoint{gp mark 0}{(5.491,4.688)}
\gppoint{gp mark 0}{(5.491,4.672)}
\gppoint{gp mark 0}{(5.491,5.320)}
\gppoint{gp mark 0}{(5.491,4.747)}
\gppoint{gp mark 0}{(5.491,5.842)}
\gppoint{gp mark 0}{(5.491,4.158)}
\gppoint{gp mark 0}{(5.491,5.514)}
\gppoint{gp mark 0}{(5.491,4.986)}
\gppoint{gp mark 0}{(5.491,5.587)}
\gppoint{gp mark 0}{(5.491,4.462)}
\gppoint{gp mark 0}{(5.491,5.587)}
\gppoint{gp mark 0}{(5.491,5.067)}
\gppoint{gp mark 0}{(5.491,4.485)}
\gppoint{gp mark 0}{(5.491,5.014)}
\gppoint{gp mark 0}{(5.491,4.584)}
\gppoint{gp mark 0}{(5.491,5.377)}
\gppoint{gp mark 0}{(5.491,4.898)}
\gppoint{gp mark 0}{(5.491,4.245)}
\gppoint{gp mark 0}{(5.491,5.537)}
\gppoint{gp mark 0}{(5.491,4.676)}
\gppoint{gp mark 0}{(5.491,4.338)}
\gppoint{gp mark 0}{(5.491,5.226)}
\gppoint{gp mark 0}{(5.491,5.114)}
\gppoint{gp mark 0}{(5.491,5.737)}
\gppoint{gp mark 0}{(5.491,5.108)}
\gppoint{gp mark 0}{(5.491,5.130)}
\gppoint{gp mark 0}{(5.491,5.539)}
\gppoint{gp mark 0}{(5.491,5.331)}
\gppoint{gp mark 0}{(5.491,4.665)}
\gppoint{gp mark 0}{(5.491,5.128)}
\gppoint{gp mark 0}{(5.491,5.144)}
\gppoint{gp mark 0}{(5.491,4.832)}
\gppoint{gp mark 0}{(5.491,5.272)}
\gppoint{gp mark 0}{(5.491,4.915)}
\gppoint{gp mark 0}{(5.491,4.688)}
\gppoint{gp mark 0}{(5.491,4.690)}
\gppoint{gp mark 0}{(5.502,5.271)}
\gppoint{gp mark 0}{(5.502,5.208)}
\gppoint{gp mark 0}{(5.502,4.715)}
\gppoint{gp mark 0}{(5.502,4.076)}
\gppoint{gp mark 0}{(5.502,4.974)}
\gppoint{gp mark 0}{(5.502,5.332)}
\gppoint{gp mark 0}{(5.502,4.494)}
\gppoint{gp mark 0}{(5.502,4.931)}
\gppoint{gp mark 0}{(5.502,5.456)}
\gppoint{gp mark 0}{(5.502,4.357)}
\gppoint{gp mark 0}{(5.502,5.563)}
\gppoint{gp mark 0}{(5.502,5.972)}
\gppoint{gp mark 0}{(5.502,4.121)}
\gppoint{gp mark 0}{(5.502,4.837)}
\gppoint{gp mark 0}{(5.502,5.614)}
\gppoint{gp mark 0}{(5.502,4.786)}
\gppoint{gp mark 0}{(5.502,4.398)}
\gppoint{gp mark 0}{(5.502,5.650)}
\gppoint{gp mark 0}{(5.502,5.317)}
\gppoint{gp mark 0}{(5.502,5.271)}
\gppoint{gp mark 0}{(5.502,4.433)}
\gppoint{gp mark 0}{(5.502,5.040)}
\gppoint{gp mark 0}{(5.502,5.532)}
\gppoint{gp mark 0}{(5.502,5.483)}
\gppoint{gp mark 0}{(5.502,5.209)}
\gppoint{gp mark 0}{(5.502,4.433)}
\gppoint{gp mark 0}{(5.502,5.227)}
\gppoint{gp mark 0}{(5.502,5.264)}
\gppoint{gp mark 0}{(5.502,4.372)}
\gppoint{gp mark 0}{(5.502,5.434)}
\gppoint{gp mark 0}{(5.502,4.618)}
\gppoint{gp mark 0}{(5.513,5.639)}
\gppoint{gp mark 0}{(5.513,5.270)}
\gppoint{gp mark 0}{(5.513,5.009)}
\gppoint{gp mark 0}{(5.513,5.657)}
\gppoint{gp mark 0}{(5.513,5.393)}
\gppoint{gp mark 0}{(5.513,5.560)}
\gppoint{gp mark 0}{(5.513,5.430)}
\gppoint{gp mark 0}{(5.513,5.430)}
\gppoint{gp mark 0}{(5.513,5.823)}
\gppoint{gp mark 0}{(5.513,5.430)}
\gppoint{gp mark 0}{(5.513,5.046)}
\gppoint{gp mark 0}{(5.513,4.592)}
\gppoint{gp mark 0}{(5.513,5.441)}
\gppoint{gp mark 0}{(5.513,6.254)}
\gppoint{gp mark 0}{(5.513,5.007)}
\gppoint{gp mark 0}{(5.513,6.284)}
\gppoint{gp mark 0}{(5.513,5.679)}
\gppoint{gp mark 0}{(5.513,5.079)}
\gppoint{gp mark 0}{(5.513,5.079)}
\gppoint{gp mark 0}{(5.513,4.966)}
\gppoint{gp mark 0}{(5.513,5.150)}
\gppoint{gp mark 0}{(5.513,5.594)}
\gppoint{gp mark 0}{(5.513,5.441)}
\gppoint{gp mark 0}{(5.513,5.034)}
\gppoint{gp mark 0}{(5.513,5.349)}
\gppoint{gp mark 0}{(5.513,5.556)}
\gppoint{gp mark 0}{(5.513,4.289)}
\gppoint{gp mark 0}{(5.513,4.758)}
\gppoint{gp mark 0}{(5.513,5.189)}
\gppoint{gp mark 0}{(5.513,4.730)}
\gppoint{gp mark 0}{(5.513,5.705)}
\gppoint{gp mark 0}{(5.513,4.297)}
\gppoint{gp mark 0}{(5.513,5.742)}
\gppoint{gp mark 0}{(5.513,4.592)}
\gppoint{gp mark 0}{(5.513,4.676)}
\gppoint{gp mark 0}{(5.513,5.913)}
\gppoint{gp mark 0}{(5.513,5.103)}
\gppoint{gp mark 0}{(5.513,5.181)}
\gppoint{gp mark 0}{(5.513,4.918)}
\gppoint{gp mark 0}{(5.513,5.451)}
\gppoint{gp mark 0}{(5.513,5.787)}
\gppoint{gp mark 0}{(5.513,4.466)}
\gppoint{gp mark 0}{(5.513,5.645)}
\gppoint{gp mark 0}{(5.513,5.594)}
\gppoint{gp mark 0}{(5.524,5.358)}
\gppoint{gp mark 0}{(5.524,4.676)}
\gppoint{gp mark 0}{(5.524,5.108)}
\gppoint{gp mark 0}{(5.524,5.555)}
\gppoint{gp mark 0}{(5.524,5.297)}
\gppoint{gp mark 0}{(5.524,4.676)}
\gppoint{gp mark 0}{(5.524,5.292)}
\gppoint{gp mark 0}{(5.524,4.422)}
\gppoint{gp mark 0}{(5.524,5.450)}
\gppoint{gp mark 0}{(5.524,5.088)}
\gppoint{gp mark 0}{(5.524,4.766)}
\gppoint{gp mark 0}{(5.524,5.188)}
\gppoint{gp mark 0}{(5.524,4.623)}
\gppoint{gp mark 0}{(5.524,5.079)}
\gppoint{gp mark 0}{(5.524,5.640)}
\gppoint{gp mark 0}{(5.524,4.902)}
\gppoint{gp mark 0}{(5.524,4.676)}
\gppoint{gp mark 0}{(5.524,5.079)}
\gppoint{gp mark 0}{(5.524,4.762)}
\gppoint{gp mark 0}{(5.524,4.693)}
\gppoint{gp mark 0}{(5.524,5.028)}
\gppoint{gp mark 0}{(5.524,4.550)}
\gppoint{gp mark 0}{(5.524,4.762)}
\gppoint{gp mark 0}{(5.524,5.201)}
\gppoint{gp mark 0}{(5.524,4.189)}
\gppoint{gp mark 0}{(5.524,5.724)}
\gppoint{gp mark 0}{(5.524,4.433)}
\gppoint{gp mark 0}{(5.524,4.548)}
\gppoint{gp mark 0}{(5.524,5.421)}
\gppoint{gp mark 0}{(5.524,5.291)}
\gppoint{gp mark 0}{(5.524,5.058)}
\gppoint{gp mark 0}{(5.524,4.903)}
\gppoint{gp mark 0}{(5.524,4.800)}
\gppoint{gp mark 0}{(5.524,4.372)}
\gppoint{gp mark 0}{(5.524,5.685)}
\gppoint{gp mark 0}{(5.524,5.384)}
\gppoint{gp mark 0}{(5.524,4.584)}
\gppoint{gp mark 0}{(5.524,4.711)}
\gppoint{gp mark 0}{(5.524,4.986)}
\gppoint{gp mark 0}{(5.524,5.486)}
\gppoint{gp mark 0}{(5.534,5.239)}
\gppoint{gp mark 0}{(5.534,5.392)}
\gppoint{gp mark 0}{(5.534,4.584)}
\gppoint{gp mark 0}{(5.534,5.315)}
\gppoint{gp mark 0}{(5.534,5.754)}
\gppoint{gp mark 0}{(5.534,5.317)}
\gppoint{gp mark 0}{(5.534,5.275)}
\gppoint{gp mark 0}{(5.534,5.384)}
\gppoint{gp mark 0}{(5.534,5.148)}
\gppoint{gp mark 0}{(5.534,4.971)}
\gppoint{gp mark 0}{(5.534,4.297)}
\gppoint{gp mark 0}{(5.534,6.044)}
\gppoint{gp mark 0}{(5.534,5.459)}
\gppoint{gp mark 0}{(5.534,5.281)}
\gppoint{gp mark 0}{(5.534,5.079)}
\gppoint{gp mark 0}{(5.534,5.359)}
\gppoint{gp mark 0}{(5.534,5.070)}
\gppoint{gp mark 0}{(5.534,5.091)}
\gppoint{gp mark 0}{(5.534,4.841)}
\gppoint{gp mark 0}{(5.534,4.892)}
\gppoint{gp mark 0}{(5.534,5.097)}
\gppoint{gp mark 0}{(5.534,5.614)}
\gppoint{gp mark 0}{(5.534,4.297)}
\gppoint{gp mark 0}{(5.534,5.406)}
\gppoint{gp mark 0}{(5.534,4.630)}
\gppoint{gp mark 0}{(5.534,5.603)}
\gppoint{gp mark 0}{(5.534,4.974)}
\gppoint{gp mark 0}{(5.534,4.667)}
\gppoint{gp mark 0}{(5.534,5.058)}
\gppoint{gp mark 0}{(5.534,5.743)}
\gppoint{gp mark 0}{(5.534,4.297)}
\gppoint{gp mark 0}{(5.534,5.201)}
\gppoint{gp mark 0}{(5.534,4.645)}
\gppoint{gp mark 0}{(5.534,5.219)}
\gppoint{gp mark 0}{(5.534,4.890)}
\gppoint{gp mark 0}{(5.534,5.284)}
\gppoint{gp mark 0}{(5.545,5.484)}
\gppoint{gp mark 0}{(5.545,5.195)}
\gppoint{gp mark 0}{(5.545,5.294)}
\gppoint{gp mark 0}{(5.545,5.294)}
\gppoint{gp mark 0}{(5.545,5.294)}
\gppoint{gp mark 0}{(5.545,5.291)}
\gppoint{gp mark 0}{(5.545,5.294)}
\gppoint{gp mark 0}{(5.545,5.421)}
\gppoint{gp mark 0}{(5.545,5.294)}
\gppoint{gp mark 0}{(5.545,4.357)}
\gppoint{gp mark 0}{(5.545,5.231)}
\gppoint{gp mark 0}{(5.545,5.091)}
\gppoint{gp mark 0}{(5.545,4.674)}
\gppoint{gp mark 0}{(5.545,5.790)}
\gppoint{gp mark 0}{(5.545,4.446)}
\gppoint{gp mark 0}{(5.545,5.650)}
\gppoint{gp mark 0}{(5.545,5.060)}
\gppoint{gp mark 0}{(5.545,4.953)}
\gppoint{gp mark 0}{(5.545,5.013)}
\gppoint{gp mark 0}{(5.545,5.032)}
\gppoint{gp mark 0}{(5.545,5.594)}
\gppoint{gp mark 0}{(5.545,5.435)}
\gppoint{gp mark 0}{(5.545,4.592)}
\gppoint{gp mark 0}{(5.545,5.606)}
\gppoint{gp mark 0}{(5.545,5.079)}
\gppoint{gp mark 0}{(5.545,5.224)}
\gppoint{gp mark 0}{(5.545,5.079)}
\gppoint{gp mark 0}{(5.545,4.669)}
\gppoint{gp mark 0}{(5.545,5.035)}
\gppoint{gp mark 0}{(5.545,5.438)}
\gppoint{gp mark 0}{(5.545,5.824)}
\gppoint{gp mark 0}{(5.545,5.358)}
\gppoint{gp mark 0}{(5.545,5.007)}
\gppoint{gp mark 0}{(5.545,5.309)}
\gppoint{gp mark 0}{(5.545,5.104)}
\gppoint{gp mark 0}{(5.545,5.054)}
\gppoint{gp mark 0}{(5.545,4.533)}
\gppoint{gp mark 0}{(5.545,5.454)}
\gppoint{gp mark 0}{(5.556,5.088)}
\gppoint{gp mark 0}{(5.556,5.459)}
\gppoint{gp mark 0}{(5.556,5.568)}
\gppoint{gp mark 0}{(5.556,5.687)}
\gppoint{gp mark 0}{(5.556,4.449)}
\gppoint{gp mark 0}{(5.556,4.772)}
\gppoint{gp mark 0}{(5.556,5.088)}
\gppoint{gp mark 0}{(5.556,5.819)}
\gppoint{gp mark 0}{(5.556,5.142)}
\gppoint{gp mark 0}{(5.556,5.597)}
\gppoint{gp mark 0}{(5.556,4.121)}
\gppoint{gp mark 0}{(5.556,5.459)}
\gppoint{gp mark 0}{(5.556,4.662)}
\gppoint{gp mark 0}{(5.556,5.299)}
\gppoint{gp mark 0}{(5.556,5.577)}
\gppoint{gp mark 0}{(5.556,5.393)}
\gppoint{gp mark 0}{(5.556,4.189)}
\gppoint{gp mark 0}{(5.556,5.393)}
\gppoint{gp mark 0}{(5.556,4.895)}
\gppoint{gp mark 0}{(5.556,5.293)}
\gppoint{gp mark 0}{(5.556,5.230)}
\gppoint{gp mark 0}{(5.556,4.898)}
\gppoint{gp mark 0}{(5.556,5.459)}
\gppoint{gp mark 0}{(5.556,4.972)}
\gppoint{gp mark 0}{(5.556,5.098)}
\gppoint{gp mark 0}{(5.556,5.157)}
\gppoint{gp mark 0}{(5.556,5.076)}
\gppoint{gp mark 0}{(5.556,4.784)}
\gppoint{gp mark 0}{(5.556,4.928)}
\gppoint{gp mark 0}{(5.556,5.578)}
\gppoint{gp mark 0}{(5.556,5.484)}
\gppoint{gp mark 0}{(5.556,4.584)}
\gppoint{gp mark 0}{(5.556,5.616)}
\gppoint{gp mark 0}{(5.556,5.812)}
\gppoint{gp mark 0}{(5.556,4.832)}
\gppoint{gp mark 0}{(5.556,5.366)}
\gppoint{gp mark 0}{(5.556,4.788)}
\gppoint{gp mark 0}{(5.566,5.370)}
\gppoint{gp mark 0}{(5.566,5.215)}
\gppoint{gp mark 0}{(5.566,4.804)}
\gppoint{gp mark 0}{(5.566,5.772)}
\gppoint{gp mark 0}{(5.566,5.226)}
\gppoint{gp mark 0}{(5.566,6.119)}
\gppoint{gp mark 0}{(5.566,4.618)}
\gppoint{gp mark 0}{(5.566,5.216)}
\gppoint{gp mark 0}{(5.566,4.981)}
\gppoint{gp mark 0}{(5.566,4.405)}
\gppoint{gp mark 0}{(5.566,4.391)}
\gppoint{gp mark 0}{(5.566,5.210)}
\gppoint{gp mark 0}{(5.566,4.615)}
\gppoint{gp mark 0}{(5.566,5.458)}
\gppoint{gp mark 0}{(5.566,5.134)}
\gppoint{gp mark 0}{(5.566,5.577)}
\gppoint{gp mark 0}{(5.566,4.766)}
\gppoint{gp mark 0}{(5.566,5.458)}
\gppoint{gp mark 0}{(5.566,5.760)}
\gppoint{gp mark 0}{(5.566,5.337)}
\gppoint{gp mark 0}{(5.566,5.625)}
\gppoint{gp mark 0}{(5.566,5.341)}
\gppoint{gp mark 0}{(5.566,5.047)}
\gppoint{gp mark 0}{(5.566,5.496)}
\gppoint{gp mark 0}{(5.566,5.740)}
\gppoint{gp mark 0}{(5.566,5.072)}
\gppoint{gp mark 0}{(5.566,5.611)}
\gppoint{gp mark 0}{(5.566,5.832)}
\gppoint{gp mark 0}{(5.566,5.760)}
\gppoint{gp mark 0}{(5.566,5.047)}
\gppoint{gp mark 0}{(5.566,4.936)}
\gppoint{gp mark 0}{(5.566,5.859)}
\gppoint{gp mark 0}{(5.566,5.047)}
\gppoint{gp mark 0}{(5.566,5.065)}
\gppoint{gp mark 0}{(5.566,5.626)}
\gppoint{gp mark 0}{(5.566,5.548)}
\gppoint{gp mark 0}{(5.566,5.638)}
\gppoint{gp mark 0}{(5.566,5.047)}
\gppoint{gp mark 0}{(5.566,4.741)}
\gppoint{gp mark 0}{(5.576,5.294)}
\gppoint{gp mark 0}{(5.576,4.811)}
\gppoint{gp mark 0}{(5.576,4.713)}
\gppoint{gp mark 0}{(5.576,4.376)}
\gppoint{gp mark 0}{(5.576,5.784)}
\gppoint{gp mark 0}{(5.576,5.405)}
\gppoint{gp mark 0}{(5.576,4.695)}
\gppoint{gp mark 0}{(5.576,5.674)}
\gppoint{gp mark 0}{(5.576,4.521)}
\gppoint{gp mark 0}{(5.576,5.292)}
\gppoint{gp mark 0}{(5.576,5.367)}
\gppoint{gp mark 0}{(5.576,5.405)}
\gppoint{gp mark 0}{(5.576,5.484)}
\gppoint{gp mark 0}{(5.576,5.234)}
\gppoint{gp mark 0}{(5.576,4.672)}
\gppoint{gp mark 0}{(5.576,5.058)}
\gppoint{gp mark 0}{(5.576,4.515)}
\gppoint{gp mark 0}{(5.576,4.267)}
\gppoint{gp mark 0}{(5.576,4.974)}
\gppoint{gp mark 0}{(5.576,4.945)}
\gppoint{gp mark 0}{(5.576,5.737)}
\gppoint{gp mark 0}{(5.576,4.811)}
\gppoint{gp mark 0}{(5.576,5.050)}
\gppoint{gp mark 0}{(5.576,4.494)}
\gppoint{gp mark 0}{(5.576,5.475)}
\gppoint{gp mark 0}{(5.576,5.310)}
\gppoint{gp mark 0}{(5.576,4.947)}
\gppoint{gp mark 0}{(5.576,5.646)}
\gppoint{gp mark 0}{(5.576,5.210)}
\gppoint{gp mark 0}{(5.576,6.192)}
\gppoint{gp mark 0}{(5.576,5.768)}
\gppoint{gp mark 0}{(5.576,4.804)}
\gppoint{gp mark 0}{(5.576,5.043)}
\gppoint{gp mark 0}{(5.576,5.772)}
\gppoint{gp mark 0}{(5.576,5.050)}
\gppoint{gp mark 0}{(5.576,4.391)}
\gppoint{gp mark 0}{(5.576,4.890)}
\gppoint{gp mark 0}{(5.576,4.533)}
\gppoint{gp mark 0}{(5.576,5.292)}
\gppoint{gp mark 0}{(5.587,5.404)}
\gppoint{gp mark 0}{(5.587,5.614)}
\gppoint{gp mark 0}{(5.587,5.466)}
\gppoint{gp mark 0}{(5.587,5.347)}
\gppoint{gp mark 0}{(5.587,5.821)}
\gppoint{gp mark 0}{(5.587,4.786)}
\gppoint{gp mark 0}{(5.587,5.697)}
\gppoint{gp mark 0}{(5.587,4.890)}
\gppoint{gp mark 0}{(5.587,4.917)}
\gppoint{gp mark 0}{(5.587,5.857)}
\gppoint{gp mark 0}{(5.587,5.294)}
\gppoint{gp mark 0}{(5.587,5.560)}
\gppoint{gp mark 0}{(5.587,5.416)}
\gppoint{gp mark 0}{(5.587,5.193)}
\gppoint{gp mark 0}{(5.587,5.729)}
\gppoint{gp mark 0}{(5.587,5.016)}
\gppoint{gp mark 0}{(5.587,5.714)}
\gppoint{gp mark 0}{(5.587,4.711)}
\gppoint{gp mark 0}{(5.587,4.735)}
\gppoint{gp mark 0}{(5.587,4.957)}
\gppoint{gp mark 0}{(5.587,5.637)}
\gppoint{gp mark 0}{(5.587,4.888)}
\gppoint{gp mark 0}{(5.587,5.023)}
\gppoint{gp mark 0}{(5.587,5.366)}
\gppoint{gp mark 0}{(5.597,5.483)}
\gppoint{gp mark 0}{(5.597,6.149)}
\gppoint{gp mark 0}{(5.597,4.376)}
\gppoint{gp mark 0}{(5.597,5.666)}
\gppoint{gp mark 0}{(5.597,5.475)}
\gppoint{gp mark 0}{(5.597,5.510)}
\gppoint{gp mark 0}{(5.597,5.448)}
\gppoint{gp mark 0}{(5.597,4.937)}
\gppoint{gp mark 0}{(5.597,5.611)}
\gppoint{gp mark 0}{(5.597,5.332)}
\gppoint{gp mark 0}{(5.597,6.097)}
\gppoint{gp mark 0}{(5.597,5.640)}
\gppoint{gp mark 0}{(5.597,5.089)}
\gppoint{gp mark 0}{(5.597,4.841)}
\gppoint{gp mark 0}{(5.597,4.959)}
\gppoint{gp mark 0}{(5.597,5.475)}
\gppoint{gp mark 0}{(5.597,5.628)}
\gppoint{gp mark 0}{(5.597,5.452)}
\gppoint{gp mark 0}{(5.597,5.475)}
\gppoint{gp mark 0}{(5.597,5.794)}
\gppoint{gp mark 0}{(5.597,5.825)}
\gppoint{gp mark 0}{(5.597,5.463)}
\gppoint{gp mark 0}{(5.597,4.815)}
\gppoint{gp mark 0}{(5.597,4.830)}
\gppoint{gp mark 0}{(5.597,5.675)}
\gppoint{gp mark 0}{(5.597,4.883)}
\gppoint{gp mark 0}{(5.597,4.453)}
\gppoint{gp mark 0}{(5.597,4.876)}
\gppoint{gp mark 0}{(5.597,5.121)}
\gppoint{gp mark 0}{(5.597,5.564)}
\gppoint{gp mark 0}{(5.597,5.748)}
\gppoint{gp mark 0}{(5.597,5.783)}
\gppoint{gp mark 0}{(5.597,4.643)}
\gppoint{gp mark 0}{(5.597,5.344)}
\gppoint{gp mark 0}{(5.597,4.780)}
\gppoint{gp mark 0}{(5.597,5.675)}
\gppoint{gp mark 0}{(5.607,5.912)}
\gppoint{gp mark 0}{(5.607,5.435)}
\gppoint{gp mark 0}{(5.607,5.702)}
\gppoint{gp mark 0}{(5.607,5.055)}
\gppoint{gp mark 0}{(5.607,5.538)}
\gppoint{gp mark 0}{(5.607,4.839)}
\gppoint{gp mark 0}{(5.607,5.620)}
\gppoint{gp mark 0}{(5.607,5.300)}
\gppoint{gp mark 0}{(5.607,5.252)}
\gppoint{gp mark 0}{(5.607,5.859)}
\gppoint{gp mark 0}{(5.607,4.940)}
\gppoint{gp mark 0}{(5.607,5.654)}
\gppoint{gp mark 0}{(5.607,5.690)}
\gppoint{gp mark 0}{(5.607,5.435)}
\gppoint{gp mark 0}{(5.607,4.542)}
\gppoint{gp mark 0}{(5.607,5.435)}
\gppoint{gp mark 0}{(5.607,4.110)}
\gppoint{gp mark 0}{(5.607,5.067)}
\gppoint{gp mark 0}{(5.607,4.674)}
\gppoint{gp mark 0}{(5.607,4.697)}
\gppoint{gp mark 0}{(5.607,4.726)}
\gppoint{gp mark 0}{(5.607,5.087)}
\gppoint{gp mark 0}{(5.607,5.359)}
\gppoint{gp mark 0}{(5.607,5.635)}
\gppoint{gp mark 0}{(5.607,5.432)}
\gppoint{gp mark 0}{(5.607,4.971)}
\gppoint{gp mark 0}{(5.607,5.552)}
\gppoint{gp mark 0}{(5.607,4.643)}
\gppoint{gp mark 0}{(5.607,4.699)}
\gppoint{gp mark 0}{(5.607,5.060)}
\gppoint{gp mark 0}{(5.607,5.297)}
\gppoint{gp mark 0}{(5.607,5.498)}
\gppoint{gp mark 0}{(5.607,4.830)}
\gppoint{gp mark 0}{(5.607,4.903)}
\gppoint{gp mark 0}{(5.616,5.367)}
\gppoint{gp mark 0}{(5.616,5.804)}
\gppoint{gp mark 0}{(5.616,4.915)}
\gppoint{gp mark 0}{(5.616,5.367)}
\gppoint{gp mark 0}{(5.616,5.506)}
\gppoint{gp mark 0}{(5.616,5.367)}
\gppoint{gp mark 0}{(5.616,5.712)}
\gppoint{gp mark 0}{(5.616,5.367)}
\gppoint{gp mark 0}{(5.616,5.943)}
\gppoint{gp mark 0}{(5.616,5.847)}
\gppoint{gp mark 0}{(5.616,5.367)}
\gppoint{gp mark 0}{(5.616,4.648)}
\gppoint{gp mark 0}{(5.616,5.477)}
\gppoint{gp mark 0}{(5.616,5.367)}
\gppoint{gp mark 0}{(5.616,5.367)}
\gppoint{gp mark 0}{(5.616,5.367)}
\gppoint{gp mark 0}{(5.616,5.367)}
\gppoint{gp mark 0}{(5.616,6.195)}
\gppoint{gp mark 0}{(5.616,5.367)}
\gppoint{gp mark 0}{(5.616,5.367)}
\gppoint{gp mark 0}{(5.616,5.519)}
\gppoint{gp mark 0}{(5.616,5.412)}
\gppoint{gp mark 0}{(5.616,4.655)}
\gppoint{gp mark 0}{(5.616,5.569)}
\gppoint{gp mark 0}{(5.616,5.367)}
\gppoint{gp mark 0}{(5.616,5.058)}
\gppoint{gp mark 0}{(5.616,4.722)}
\gppoint{gp mark 0}{(5.616,5.186)}
\gppoint{gp mark 0}{(5.616,5.321)}
\gppoint{gp mark 0}{(5.616,5.367)}
\gppoint{gp mark 0}{(5.616,5.367)}
\gppoint{gp mark 0}{(5.616,5.367)}
\gppoint{gp mark 0}{(5.616,5.847)}
\gppoint{gp mark 0}{(5.616,5.367)}
\gppoint{gp mark 0}{(5.616,5.367)}
\gppoint{gp mark 0}{(5.616,5.367)}
\gppoint{gp mark 0}{(5.616,5.610)}
\gppoint{gp mark 0}{(5.616,5.367)}
\gppoint{gp mark 0}{(5.616,5.367)}
\gppoint{gp mark 0}{(5.616,4.314)}
\gppoint{gp mark 0}{(5.616,5.367)}
\gppoint{gp mark 0}{(5.616,5.702)}
\gppoint{gp mark 0}{(5.616,4.679)}
\gppoint{gp mark 0}{(5.616,5.703)}
\gppoint{gp mark 0}{(5.616,4.643)}
\gppoint{gp mark 0}{(5.616,5.310)}
\gppoint{gp mark 0}{(5.616,5.591)}
\gppoint{gp mark 0}{(5.616,5.601)}
\gppoint{gp mark 0}{(5.616,4.870)}
\gppoint{gp mark 0}{(5.616,5.367)}
\gppoint{gp mark 0}{(5.616,4.548)}
\gppoint{gp mark 0}{(5.616,5.636)}
\gppoint{gp mark 0}{(5.616,5.367)}
\gppoint{gp mark 0}{(5.616,5.367)}
\gppoint{gp mark 0}{(5.616,4.898)}
\gppoint{gp mark 0}{(5.626,5.042)}
\gppoint{gp mark 0}{(5.626,5.590)}
\gppoint{gp mark 0}{(5.626,5.610)}
\gppoint{gp mark 0}{(5.626,5.567)}
\gppoint{gp mark 0}{(5.626,5.659)}
\gppoint{gp mark 0}{(5.626,5.213)}
\gppoint{gp mark 0}{(5.626,5.616)}
\gppoint{gp mark 0}{(5.626,5.153)}
\gppoint{gp mark 0}{(5.626,4.708)}
\gppoint{gp mark 0}{(5.626,5.472)}
\gppoint{gp mark 0}{(5.626,4.530)}
\gppoint{gp mark 0}{(5.626,5.472)}
\gppoint{gp mark 0}{(5.626,5.472)}
\gppoint{gp mark 0}{(5.626,4.218)}
\gppoint{gp mark 0}{(5.626,5.538)}
\gppoint{gp mark 0}{(5.626,5.535)}
\gppoint{gp mark 0}{(5.626,5.472)}
\gppoint{gp mark 0}{(5.626,5.535)}
\gppoint{gp mark 0}{(5.626,5.472)}
\gppoint{gp mark 0}{(5.626,6.300)}
\gppoint{gp mark 0}{(5.626,4.401)}
\gppoint{gp mark 0}{(5.626,5.371)}
\gppoint{gp mark 0}{(5.626,5.518)}
\gppoint{gp mark 0}{(5.626,5.447)}
\gppoint{gp mark 0}{(5.626,5.787)}
\gppoint{gp mark 0}{(5.626,5.134)}
\gppoint{gp mark 0}{(5.626,5.306)}
\gppoint{gp mark 0}{(5.626,5.605)}
\gppoint{gp mark 0}{(5.626,5.091)}
\gppoint{gp mark 0}{(5.626,5.178)}
\gppoint{gp mark 0}{(5.626,5.802)}
\gppoint{gp mark 0}{(5.626,5.103)}
\gppoint{gp mark 0}{(5.626,5.671)}
\gppoint{gp mark 0}{(5.626,5.421)}
\gppoint{gp mark 0}{(5.636,5.477)}
\gppoint{gp mark 0}{(5.636,5.492)}
\gppoint{gp mark 0}{(5.636,4.506)}
\gppoint{gp mark 0}{(5.636,4.890)}
\gppoint{gp mark 0}{(5.636,5.605)}
\gppoint{gp mark 0}{(5.636,5.389)}
\gppoint{gp mark 0}{(5.636,5.063)}
\gppoint{gp mark 0}{(5.636,5.580)}
\gppoint{gp mark 0}{(5.636,4.751)}
\gppoint{gp mark 0}{(5.636,5.415)}
\gppoint{gp mark 0}{(5.636,5.044)}
\gppoint{gp mark 0}{(5.636,5.589)}
\gppoint{gp mark 0}{(5.636,5.674)}
\gppoint{gp mark 0}{(5.636,5.152)}
\gppoint{gp mark 0}{(5.636,5.859)}
\gppoint{gp mark 0}{(5.636,5.838)}
\gppoint{gp mark 0}{(5.636,5.655)}
\gppoint{gp mark 0}{(5.636,5.645)}
\gppoint{gp mark 0}{(5.636,4.638)}
\gppoint{gp mark 0}{(5.636,5.152)}
\gppoint{gp mark 0}{(5.636,5.655)}
\gppoint{gp mark 0}{(5.636,4.986)}
\gppoint{gp mark 0}{(5.636,5.704)}
\gppoint{gp mark 0}{(5.636,5.838)}
\gppoint{gp mark 0}{(5.636,5.639)}
\gppoint{gp mark 0}{(5.636,5.565)}
\gppoint{gp mark 0}{(5.636,5.872)}
\gppoint{gp mark 0}{(5.636,5.430)}
\gppoint{gp mark 0}{(5.636,4.981)}
\gppoint{gp mark 0}{(5.645,5.331)}
\gppoint{gp mark 0}{(5.645,5.172)}
\gppoint{gp mark 0}{(5.645,6.031)}
\gppoint{gp mark 0}{(5.645,5.553)}
\gppoint{gp mark 0}{(5.645,4.912)}
\gppoint{gp mark 0}{(5.645,5.814)}
\gppoint{gp mark 0}{(5.645,5.219)}
\gppoint{gp mark 0}{(5.645,4.605)}
\gppoint{gp mark 0}{(5.645,4.944)}
\gppoint{gp mark 0}{(5.645,4.776)}
\gppoint{gp mark 0}{(5.645,4.871)}
\gppoint{gp mark 0}{(5.645,4.581)}
\gppoint{gp mark 0}{(5.645,5.415)}
\gppoint{gp mark 0}{(5.645,5.653)}
\gppoint{gp mark 0}{(5.645,6.031)}
\gppoint{gp mark 0}{(5.645,5.042)}
\gppoint{gp mark 0}{(5.645,5.394)}
\gppoint{gp mark 0}{(5.645,5.726)}
\gppoint{gp mark 0}{(5.645,4.695)}
\gppoint{gp mark 0}{(5.645,5.379)}
\gppoint{gp mark 0}{(5.645,3.864)}
\gppoint{gp mark 0}{(5.645,4.715)}
\gppoint{gp mark 0}{(5.645,5.331)}
\gppoint{gp mark 0}{(5.645,4.640)}
\gppoint{gp mark 0}{(5.645,5.858)}
\gppoint{gp mark 0}{(5.645,4.981)}
\gppoint{gp mark 0}{(5.645,5.386)}
\gppoint{gp mark 0}{(5.645,5.379)}
\gppoint{gp mark 0}{(5.645,5.185)}
\gppoint{gp mark 0}{(5.655,5.169)}
\gppoint{gp mark 0}{(5.655,4.564)}
\gppoint{gp mark 0}{(5.655,5.604)}
\gppoint{gp mark 0}{(5.655,5.511)}
\gppoint{gp mark 0}{(5.655,5.723)}
\gppoint{gp mark 0}{(5.655,3.975)}
\gppoint{gp mark 0}{(5.655,4.897)}
\gppoint{gp mark 0}{(5.655,5.201)}
\gppoint{gp mark 0}{(5.655,5.702)}
\gppoint{gp mark 0}{(5.655,5.387)}
\gppoint{gp mark 0}{(5.655,5.404)}
\gppoint{gp mark 0}{(5.655,5.511)}
\gppoint{gp mark 0}{(5.655,5.408)}
\gppoint{gp mark 0}{(5.655,5.782)}
\gppoint{gp mark 0}{(5.655,5.136)}
\gppoint{gp mark 0}{(5.655,5.317)}
\gppoint{gp mark 0}{(5.655,5.893)}
\gppoint{gp mark 0}{(5.655,5.511)}
\gppoint{gp mark 0}{(5.655,5.296)}
\gppoint{gp mark 0}{(5.655,4.942)}
\gppoint{gp mark 0}{(5.655,5.537)}
\gppoint{gp mark 0}{(5.655,5.510)}
\gppoint{gp mark 0}{(5.655,5.714)}
\gppoint{gp mark 0}{(5.655,5.718)}
\gppoint{gp mark 0}{(5.655,5.402)}
\gppoint{gp mark 0}{(5.655,5.519)}
\gppoint{gp mark 0}{(5.655,4.602)}
\gppoint{gp mark 0}{(5.655,4.743)}
\gppoint{gp mark 0}{(5.655,5.441)}
\gppoint{gp mark 0}{(5.655,5.301)}
\gppoint{gp mark 0}{(5.655,5.408)}
\gppoint{gp mark 0}{(5.655,5.328)}
\gppoint{gp mark 0}{(5.664,4.655)}
\gppoint{gp mark 0}{(5.664,5.536)}
\gppoint{gp mark 0}{(5.664,4.778)}
\gppoint{gp mark 0}{(5.664,5.228)}
\gppoint{gp mark 0}{(5.664,5.195)}
\gppoint{gp mark 0}{(5.664,5.744)}
\gppoint{gp mark 0}{(5.664,4.674)}
\gppoint{gp mark 0}{(5.664,5.510)}
\gppoint{gp mark 0}{(5.664,5.247)}
\gppoint{gp mark 0}{(5.664,5.375)}
\gppoint{gp mark 0}{(5.664,5.638)}
\gppoint{gp mark 0}{(5.664,5.331)}
\gppoint{gp mark 0}{(5.664,5.266)}
\gppoint{gp mark 0}{(5.664,5.408)}
\gppoint{gp mark 0}{(5.664,4.699)}
\gppoint{gp mark 0}{(5.664,5.899)}
\gppoint{gp mark 0}{(5.664,5.754)}
\gppoint{gp mark 0}{(5.664,5.533)}
\gppoint{gp mark 0}{(5.664,5.166)}
\gppoint{gp mark 0}{(5.664,5.664)}
\gppoint{gp mark 0}{(5.664,5.533)}
\gppoint{gp mark 0}{(5.664,5.170)}
\gppoint{gp mark 0}{(5.664,5.746)}
\gppoint{gp mark 0}{(5.664,5.744)}
\gppoint{gp mark 0}{(5.664,5.691)}
\gppoint{gp mark 0}{(5.664,4.704)}
\gppoint{gp mark 0}{(5.664,5.189)}
\gppoint{gp mark 0}{(5.664,5.547)}
\gppoint{gp mark 0}{(5.673,4.667)}
\gppoint{gp mark 0}{(5.673,5.253)}
\gppoint{gp mark 0}{(5.673,5.743)}
\gppoint{gp mark 0}{(5.673,5.188)}
\gppoint{gp mark 0}{(5.673,5.872)}
\gppoint{gp mark 0}{(5.673,5.815)}
\gppoint{gp mark 0}{(5.673,4.512)}
\gppoint{gp mark 0}{(5.673,5.892)}
\gppoint{gp mark 0}{(5.673,5.689)}
\gppoint{gp mark 0}{(5.673,5.756)}
\gppoint{gp mark 0}{(5.673,4.640)}
\gppoint{gp mark 0}{(5.673,4.857)}
\gppoint{gp mark 0}{(5.673,5.531)}
\gppoint{gp mark 0}{(5.673,5.170)}
\gppoint{gp mark 0}{(5.673,4.676)}
\gppoint{gp mark 0}{(5.673,5.701)}
\gppoint{gp mark 0}{(5.673,5.441)}
\gppoint{gp mark 0}{(5.673,5.531)}
\gppoint{gp mark 0}{(5.673,5.515)}
\gppoint{gp mark 0}{(5.673,5.683)}
\gppoint{gp mark 0}{(5.673,5.770)}
\gppoint{gp mark 0}{(5.673,5.478)}
\gppoint{gp mark 0}{(5.673,5.462)}
\gppoint{gp mark 0}{(5.673,5.462)}
\gppoint{gp mark 0}{(5.673,5.091)}
\gppoint{gp mark 0}{(5.673,4.802)}
\gppoint{gp mark 0}{(5.673,5.491)}
\gppoint{gp mark 0}{(5.673,4.802)}
\gppoint{gp mark 0}{(5.673,5.615)}
\gppoint{gp mark 0}{(5.673,5.363)}
\gppoint{gp mark 0}{(5.673,5.851)}
\gppoint{gp mark 0}{(5.673,4.929)}
\gppoint{gp mark 0}{(5.673,5.061)}
\gppoint{gp mark 0}{(5.673,5.618)}
\gppoint{gp mark 0}{(5.673,5.186)}
\gppoint{gp mark 0}{(5.673,5.626)}
\gppoint{gp mark 0}{(5.673,4.732)}
\gppoint{gp mark 0}{(5.673,5.088)}
\gppoint{gp mark 0}{(5.673,5.253)}
\gppoint{gp mark 0}{(5.673,5.488)}
\gppoint{gp mark 0}{(5.673,5.464)}
\gppoint{gp mark 0}{(5.673,5.589)}
\gppoint{gp mark 0}{(5.683,5.432)}
\gppoint{gp mark 0}{(5.683,5.044)}
\gppoint{gp mark 0}{(5.683,5.467)}
\gppoint{gp mark 0}{(5.683,5.583)}
\gppoint{gp mark 0}{(5.683,5.866)}
\gppoint{gp mark 0}{(5.683,5.948)}
\gppoint{gp mark 0}{(5.683,5.149)}
\gppoint{gp mark 0}{(5.683,6.466)}
\gppoint{gp mark 0}{(5.683,5.472)}
\gppoint{gp mark 0}{(5.683,4.667)}
\gppoint{gp mark 0}{(5.683,5.569)}
\gppoint{gp mark 0}{(5.683,6.091)}
\gppoint{gp mark 0}{(5.683,5.827)}
\gppoint{gp mark 0}{(5.683,5.457)}
\gppoint{gp mark 0}{(5.683,5.612)}
\gppoint{gp mark 0}{(5.683,5.034)}
\gppoint{gp mark 0}{(5.683,4.110)}
\gppoint{gp mark 0}{(5.683,6.291)}
\gppoint{gp mark 0}{(5.683,5.461)}
\gppoint{gp mark 0}{(5.683,4.807)}
\gppoint{gp mark 0}{(5.683,5.432)}
\gppoint{gp mark 0}{(5.683,4.110)}
\gppoint{gp mark 0}{(5.683,5.586)}
\gppoint{gp mark 0}{(5.683,4.110)}
\gppoint{gp mark 0}{(5.683,5.150)}
\gppoint{gp mark 0}{(5.683,5.272)}
\gppoint{gp mark 0}{(5.683,4.890)}
\gppoint{gp mark 0}{(5.683,5.418)}
\gppoint{gp mark 0}{(5.683,4.925)}
\gppoint{gp mark 0}{(5.683,4.110)}
\gppoint{gp mark 0}{(5.683,4.436)}
\gppoint{gp mark 0}{(5.683,5.733)}
\gppoint{gp mark 0}{(5.683,4.751)}
\gppoint{gp mark 0}{(5.683,5.233)}
\gppoint{gp mark 0}{(5.683,5.750)}
\gppoint{gp mark 0}{(5.683,4.110)}
\gppoint{gp mark 0}{(5.683,5.805)}
\gppoint{gp mark 0}{(5.683,4.110)}
\gppoint{gp mark 0}{(5.683,4.905)}
\gppoint{gp mark 0}{(5.683,4.110)}
\gppoint{gp mark 0}{(5.683,4.254)}
\gppoint{gp mark 0}{(5.683,5.822)}
\gppoint{gp mark 0}{(5.683,5.400)}
\gppoint{gp mark 0}{(5.683,4.110)}
\gppoint{gp mark 0}{(5.683,4.110)}
\gppoint{gp mark 0}{(5.683,4.110)}
\gppoint{gp mark 0}{(5.683,4.913)}
\gppoint{gp mark 0}{(5.683,4.110)}
\gppoint{gp mark 0}{(5.683,4.110)}
\gppoint{gp mark 0}{(5.683,4.110)}
\gppoint{gp mark 0}{(5.683,5.651)}
\gppoint{gp mark 0}{(5.683,5.272)}
\gppoint{gp mark 0}{(5.692,4.868)}
\gppoint{gp mark 0}{(5.692,5.490)}
\gppoint{gp mark 0}{(5.692,5.515)}
\gppoint{gp mark 0}{(5.692,6.105)}
\gppoint{gp mark 0}{(5.692,5.577)}
\gppoint{gp mark 0}{(5.692,4.485)}
\gppoint{gp mark 0}{(5.692,6.009)}
\gppoint{gp mark 0}{(5.692,5.756)}
\gppoint{gp mark 0}{(5.692,4.811)}
\gppoint{gp mark 0}{(5.692,5.904)}
\gppoint{gp mark 0}{(5.692,5.635)}
\gppoint{gp mark 0}{(5.692,5.904)}
\gppoint{gp mark 0}{(5.692,5.491)}
\gppoint{gp mark 0}{(5.692,4.760)}
\gppoint{gp mark 0}{(5.692,5.077)}
\gppoint{gp mark 0}{(5.692,5.794)}
\gppoint{gp mark 0}{(5.692,4.620)}
\gppoint{gp mark 0}{(5.692,5.514)}
\gppoint{gp mark 0}{(5.692,4.737)}
\gppoint{gp mark 0}{(5.692,5.783)}
\gppoint{gp mark 0}{(5.692,5.105)}
\gppoint{gp mark 0}{(5.692,5.464)}
\gppoint{gp mark 0}{(5.692,5.219)}
\gppoint{gp mark 0}{(5.692,4.338)}
\gppoint{gp mark 0}{(5.692,5.261)}
\gppoint{gp mark 0}{(5.692,5.323)}
\gppoint{gp mark 0}{(5.692,4.657)}
\gppoint{gp mark 0}{(5.692,5.441)}
\gppoint{gp mark 0}{(5.692,4.527)}
\gppoint{gp mark 0}{(5.692,5.822)}
\gppoint{gp mark 0}{(5.692,5.742)}
\gppoint{gp mark 0}{(5.692,5.478)}
\gppoint{gp mark 0}{(5.692,4.683)}
\gppoint{gp mark 0}{(5.701,5.793)}
\gppoint{gp mark 0}{(5.701,5.335)}
\gppoint{gp mark 0}{(5.701,5.927)}
\gppoint{gp mark 0}{(5.701,4.559)}
\gppoint{gp mark 0}{(5.701,5.043)}
\gppoint{gp mark 0}{(5.701,5.067)}
\gppoint{gp mark 0}{(5.701,5.775)}
\gppoint{gp mark 0}{(5.701,4.683)}
\gppoint{gp mark 0}{(5.701,5.391)}
\gppoint{gp mark 0}{(5.701,5.839)}
\gppoint{gp mark 0}{(5.701,4.597)}
\gppoint{gp mark 0}{(5.701,5.564)}
\gppoint{gp mark 0}{(5.701,5.699)}
\gppoint{gp mark 0}{(5.701,5.481)}
\gppoint{gp mark 0}{(5.701,5.475)}
\gppoint{gp mark 0}{(5.701,5.877)}
\gppoint{gp mark 0}{(5.701,5.023)}
\gppoint{gp mark 0}{(5.701,6.221)}
\gppoint{gp mark 0}{(5.701,6.263)}
\gppoint{gp mark 0}{(5.701,5.203)}
\gppoint{gp mark 0}{(5.701,4.826)}
\gppoint{gp mark 0}{(5.701,5.869)}
\gppoint{gp mark 0}{(5.701,5.003)}
\gppoint{gp mark 0}{(5.701,5.869)}
\gppoint{gp mark 0}{(5.701,5.344)}
\gppoint{gp mark 0}{(5.701,5.092)}
\gppoint{gp mark 0}{(5.701,5.076)}
\gppoint{gp mark 0}{(5.701,4.897)}
\gppoint{gp mark 0}{(5.701,4.758)}
\gppoint{gp mark 0}{(5.710,5.248)}
\gppoint{gp mark 0}{(5.710,4.662)}
\gppoint{gp mark 0}{(5.710,5.492)}
\gppoint{gp mark 0}{(5.710,5.876)}
\gppoint{gp mark 0}{(5.710,5.234)}
\gppoint{gp mark 0}{(5.710,5.614)}
\gppoint{gp mark 0}{(5.710,5.812)}
\gppoint{gp mark 0}{(5.710,5.291)}
\gppoint{gp mark 0}{(5.710,5.492)}
\gppoint{gp mark 0}{(5.710,5.536)}
\gppoint{gp mark 0}{(5.710,5.354)}
\gppoint{gp mark 0}{(5.710,5.515)}
\gppoint{gp mark 0}{(5.710,5.851)}
\gppoint{gp mark 0}{(5.710,5.014)}
\gppoint{gp mark 0}{(5.710,5.169)}
\gppoint{gp mark 0}{(5.710,6.131)}
\gppoint{gp mark 0}{(5.710,5.528)}
\gppoint{gp mark 0}{(5.710,5.830)}
\gppoint{gp mark 0}{(5.710,4.992)}
\gppoint{gp mark 0}{(5.710,4.524)}
\gppoint{gp mark 0}{(5.710,5.329)}
\gppoint{gp mark 0}{(5.710,5.888)}
\gppoint{gp mark 0}{(5.710,5.353)}
\gppoint{gp mark 0}{(5.710,5.409)}
\gppoint{gp mark 0}{(5.710,5.478)}
\gppoint{gp mark 0}{(5.710,5.442)}
\gppoint{gp mark 0}{(5.710,5.101)}
\gppoint{gp mark 0}{(5.710,6.036)}
\gppoint{gp mark 0}{(5.710,4.690)}
\gppoint{gp mark 0}{(5.710,5.712)}
\gppoint{gp mark 0}{(5.710,5.747)}
\gppoint{gp mark 0}{(5.710,6.127)}
\gppoint{gp mark 0}{(5.710,6.003)}
\gppoint{gp mark 0}{(5.719,5.263)}
\gppoint{gp mark 0}{(5.719,5.692)}
\gppoint{gp mark 0}{(5.719,5.618)}
\gppoint{gp mark 0}{(5.719,5.963)}
\gppoint{gp mark 0}{(5.719,5.125)}
\gppoint{gp mark 0}{(5.719,5.892)}
\gppoint{gp mark 0}{(5.719,4.732)}
\gppoint{gp mark 0}{(5.719,4.981)}
\gppoint{gp mark 0}{(5.719,5.751)}
\gppoint{gp mark 0}{(5.719,4.963)}
\gppoint{gp mark 0}{(5.719,5.342)}
\gppoint{gp mark 0}{(5.719,5.093)}
\gppoint{gp mark 0}{(5.719,5.541)}
\gppoint{gp mark 0}{(5.719,4.368)}
\gppoint{gp mark 0}{(5.719,4.530)}
\gppoint{gp mark 0}{(5.719,4.475)}
\gppoint{gp mark 0}{(5.719,5.734)}
\gppoint{gp mark 0}{(5.719,5.282)}
\gppoint{gp mark 0}{(5.719,5.662)}
\gppoint{gp mark 0}{(5.719,5.011)}
\gppoint{gp mark 0}{(5.719,5.219)}
\gppoint{gp mark 0}{(5.719,4.158)}
\gppoint{gp mark 0}{(5.719,5.424)}
\gppoint{gp mark 0}{(5.719,5.054)}
\gppoint{gp mark 0}{(5.719,5.190)}
\gppoint{gp mark 0}{(5.719,5.702)}
\gppoint{gp mark 0}{(5.719,5.228)}
\gppoint{gp mark 0}{(5.719,5.731)}
\gppoint{gp mark 0}{(5.719,4.518)}
\gppoint{gp mark 0}{(5.719,5.230)}
\gppoint{gp mark 0}{(5.719,5.921)}
\gppoint{gp mark 0}{(5.719,5.432)}
\gppoint{gp mark 0}{(5.719,5.356)}
\gppoint{gp mark 0}{(5.727,5.656)}
\gppoint{gp mark 0}{(5.727,5.829)}
\gppoint{gp mark 0}{(5.727,4.939)}
\gppoint{gp mark 0}{(5.727,5.396)}
\gppoint{gp mark 0}{(5.727,5.172)}
\gppoint{gp mark 0}{(5.727,4.449)}
\gppoint{gp mark 0}{(5.727,4.545)}
\gppoint{gp mark 0}{(5.727,4.900)}
\gppoint{gp mark 0}{(5.727,5.241)}
\gppoint{gp mark 0}{(5.727,5.749)}
\gppoint{gp mark 0}{(5.727,5.153)}
\gppoint{gp mark 0}{(5.727,5.864)}
\gppoint{gp mark 0}{(5.727,5.341)}
\gppoint{gp mark 0}{(5.727,5.881)}
\gppoint{gp mark 0}{(5.727,5.537)}
\gppoint{gp mark 0}{(5.727,5.521)}
\gppoint{gp mark 0}{(5.727,5.463)}
\gppoint{gp mark 0}{(5.727,5.441)}
\gppoint{gp mark 0}{(5.727,5.446)}
\gppoint{gp mark 0}{(5.727,5.558)}
\gppoint{gp mark 0}{(5.727,6.038)}
\gppoint{gp mark 0}{(5.727,5.740)}
\gppoint{gp mark 0}{(5.727,6.004)}
\gppoint{gp mark 0}{(5.727,5.268)}
\gppoint{gp mark 0}{(5.727,5.810)}
\gppoint{gp mark 0}{(5.727,5.746)}
\gppoint{gp mark 0}{(5.727,5.829)}
\gppoint{gp mark 0}{(5.727,5.042)}
\gppoint{gp mark 0}{(5.736,4.855)}
\gppoint{gp mark 0}{(5.736,5.289)}
\gppoint{gp mark 0}{(5.736,5.490)}
\gppoint{gp mark 0}{(5.736,5.590)}
\gppoint{gp mark 0}{(5.736,5.367)}
\gppoint{gp mark 0}{(5.736,5.824)}
\gppoint{gp mark 0}{(5.736,6.054)}
\gppoint{gp mark 0}{(5.736,6.131)}
\gppoint{gp mark 0}{(5.736,5.497)}
\gppoint{gp mark 0}{(5.736,5.517)}
\gppoint{gp mark 0}{(5.736,4.887)}
\gppoint{gp mark 0}{(5.736,4.665)}
\gppoint{gp mark 0}{(5.736,4.828)}
\gppoint{gp mark 0}{(5.736,5.772)}
\gppoint{gp mark 0}{(5.736,5.701)}
\gppoint{gp mark 0}{(5.736,5.531)}
\gppoint{gp mark 0}{(5.736,5.141)}
\gppoint{gp mark 0}{(5.736,6.014)}
\gppoint{gp mark 0}{(5.736,6.006)}
\gppoint{gp mark 0}{(5.736,6.131)}
\gppoint{gp mark 0}{(5.736,5.706)}
\gppoint{gp mark 0}{(5.736,4.887)}
\gppoint{gp mark 0}{(5.745,5.241)}
\gppoint{gp mark 0}{(5.745,5.562)}
\gppoint{gp mark 0}{(5.745,5.043)}
\gppoint{gp mark 0}{(5.745,4.737)}
\gppoint{gp mark 0}{(5.745,5.278)}
\gppoint{gp mark 0}{(5.745,5.029)}
\gppoint{gp mark 0}{(5.745,4.559)}
\gppoint{gp mark 0}{(5.745,5.835)}
\gppoint{gp mark 0}{(5.745,5.751)}
\gppoint{gp mark 0}{(5.745,5.133)}
\gppoint{gp mark 0}{(5.745,5.184)}
\gppoint{gp mark 0}{(5.745,5.456)}
\gppoint{gp mark 0}{(5.745,4.633)}
\gppoint{gp mark 0}{(5.745,5.902)}
\gppoint{gp mark 0}{(5.745,5.712)}
\gppoint{gp mark 0}{(5.745,5.399)}
\gppoint{gp mark 0}{(5.745,5.733)}
\gppoint{gp mark 0}{(5.745,6.124)}
\gppoint{gp mark 0}{(5.745,6.040)}
\gppoint{gp mark 0}{(5.745,5.277)}
\gppoint{gp mark 0}{(5.745,5.446)}
\gppoint{gp mark 0}{(5.745,5.060)}
\gppoint{gp mark 0}{(5.745,5.649)}
\gppoint{gp mark 0}{(5.745,5.383)}
\gppoint{gp mark 0}{(5.745,5.627)}
\gppoint{gp mark 0}{(5.745,4.782)}
\gppoint{gp mark 0}{(5.753,6.119)}
\gppoint{gp mark 0}{(5.753,5.517)}
\gppoint{gp mark 0}{(5.753,5.839)}
\gppoint{gp mark 0}{(5.753,5.764)}
\gppoint{gp mark 0}{(5.753,5.181)}
\gppoint{gp mark 0}{(5.753,5.839)}
\gppoint{gp mark 0}{(5.753,6.175)}
\gppoint{gp mark 0}{(5.753,5.443)}
\gppoint{gp mark 0}{(5.753,5.642)}
\gppoint{gp mark 0}{(5.753,5.537)}
\gppoint{gp mark 0}{(5.753,4.472)}
\gppoint{gp mark 0}{(5.753,5.563)}
\gppoint{gp mark 0}{(5.753,5.915)}
\gppoint{gp mark 0}{(5.753,5.454)}
\gppoint{gp mark 0}{(5.753,4.726)}
\gppoint{gp mark 0}{(5.753,4.662)}
\gppoint{gp mark 0}{(5.753,6.150)}
\gppoint{gp mark 0}{(5.753,4.633)}
\gppoint{gp mark 0}{(5.753,5.360)}
\gppoint{gp mark 0}{(5.753,5.857)}
\gppoint{gp mark 0}{(5.753,4.394)}
\gppoint{gp mark 0}{(5.753,5.361)}
\gppoint{gp mark 0}{(5.753,5.768)}
\gppoint{gp mark 0}{(5.753,4.693)}
\gppoint{gp mark 0}{(5.753,5.791)}
\gppoint{gp mark 0}{(5.753,5.289)}
\gppoint{gp mark 0}{(5.753,5.784)}
\gppoint{gp mark 0}{(5.753,4.956)}
\gppoint{gp mark 0}{(5.753,5.941)}
\gppoint{gp mark 0}{(5.753,5.244)}
\gppoint{gp mark 0}{(5.753,4.965)}
\gppoint{gp mark 0}{(5.753,5.710)}
\gppoint{gp mark 0}{(5.761,5.800)}
\gppoint{gp mark 0}{(5.761,6.129)}
\gppoint{gp mark 0}{(5.761,5.713)}
\gppoint{gp mark 0}{(5.761,5.933)}
\gppoint{gp mark 0}{(5.761,5.165)}
\gppoint{gp mark 0}{(5.761,5.962)}
\gppoint{gp mark 0}{(5.761,5.821)}
\gppoint{gp mark 0}{(5.761,5.441)}
\gppoint{gp mark 0}{(5.761,5.083)}
\gppoint{gp mark 0}{(5.761,5.613)}
\gppoint{gp mark 0}{(5.761,5.821)}
\gppoint{gp mark 0}{(5.761,5.898)}
\gppoint{gp mark 0}{(5.761,5.402)}
\gppoint{gp mark 0}{(5.761,4.953)}
\gppoint{gp mark 0}{(5.761,5.560)}
\gppoint{gp mark 0}{(5.761,5.375)}
\gppoint{gp mark 0}{(5.761,5.929)}
\gppoint{gp mark 0}{(5.761,5.782)}
\gppoint{gp mark 0}{(5.761,5.544)}
\gppoint{gp mark 0}{(5.770,5.868)}
\gppoint{gp mark 0}{(5.770,5.091)}
\gppoint{gp mark 0}{(5.770,5.902)}
\gppoint{gp mark 0}{(5.770,5.305)}
\gppoint{gp mark 0}{(5.770,5.750)}
\gppoint{gp mark 0}{(5.770,4.948)}
\gppoint{gp mark 0}{(5.770,4.792)}
\gppoint{gp mark 0}{(5.770,4.898)}
\gppoint{gp mark 0}{(5.770,5.575)}
\gppoint{gp mark 0}{(5.770,5.278)}
\gppoint{gp mark 0}{(5.770,4.913)}
\gppoint{gp mark 0}{(5.770,5.388)}
\gppoint{gp mark 0}{(5.770,4.819)}
\gppoint{gp mark 0}{(5.770,5.807)}
\gppoint{gp mark 0}{(5.770,4.963)}
\gppoint{gp mark 0}{(5.770,5.332)}
\gppoint{gp mark 0}{(5.770,5.232)}
\gppoint{gp mark 0}{(5.770,5.979)}
\gppoint{gp mark 0}{(5.770,5.197)}
\gppoint{gp mark 0}{(5.770,5.404)}
\gppoint{gp mark 0}{(5.770,5.759)}
\gppoint{gp mark 0}{(5.770,5.073)}
\gppoint{gp mark 0}{(5.770,4.645)}
\gppoint{gp mark 0}{(5.770,5.248)}
\gppoint{gp mark 0}{(5.770,5.533)}
\gppoint{gp mark 0}{(5.770,5.637)}
\gppoint{gp mark 0}{(5.770,4.169)}
\gppoint{gp mark 0}{(5.770,5.577)}
\gppoint{gp mark 0}{(5.770,5.729)}
\gppoint{gp mark 0}{(5.778,4.790)}
\gppoint{gp mark 0}{(5.778,5.303)}
\gppoint{gp mark 0}{(5.778,5.591)}
\gppoint{gp mark 0}{(5.778,5.348)}
\gppoint{gp mark 0}{(5.778,5.042)}
\gppoint{gp mark 0}{(5.778,5.077)}
\gppoint{gp mark 0}{(5.778,6.035)}
\gppoint{gp mark 0}{(5.778,5.631)}
\gppoint{gp mark 0}{(5.778,5.190)}
\gppoint{gp mark 0}{(5.778,5.284)}
\gppoint{gp mark 0}{(5.778,5.202)}
\gppoint{gp mark 0}{(5.778,4.939)}
\gppoint{gp mark 0}{(5.778,5.556)}
\gppoint{gp mark 0}{(5.778,5.337)}
\gppoint{gp mark 0}{(5.778,6.006)}
\gppoint{gp mark 0}{(5.778,5.690)}
\gppoint{gp mark 0}{(5.778,5.454)}
\gppoint{gp mark 0}{(5.778,5.348)}
\gppoint{gp mark 0}{(5.778,5.379)}
\gppoint{gp mark 0}{(5.778,5.146)}
\gppoint{gp mark 0}{(5.778,6.035)}
\gppoint{gp mark 0}{(5.778,6.436)}
\gppoint{gp mark 0}{(5.778,5.587)}
\gppoint{gp mark 0}{(5.778,5.610)}
\gppoint{gp mark 0}{(5.778,4.868)}
\gppoint{gp mark 0}{(5.778,6.170)}
\gppoint{gp mark 0}{(5.778,5.348)}
\gppoint{gp mark 0}{(5.778,6.303)}
\gppoint{gp mark 0}{(5.778,6.086)}
\gppoint{gp mark 0}{(5.778,5.539)}
\gppoint{gp mark 0}{(5.778,5.348)}
\gppoint{gp mark 0}{(5.778,5.303)}
\gppoint{gp mark 0}{(5.778,5.303)}
\gppoint{gp mark 0}{(5.778,4.885)}
\gppoint{gp mark 0}{(5.778,5.851)}
\gppoint{gp mark 0}{(5.786,6.290)}
\gppoint{gp mark 0}{(5.786,5.158)}
\gppoint{gp mark 0}{(5.786,6.130)}
\gppoint{gp mark 0}{(5.786,5.854)}
\gppoint{gp mark 0}{(5.786,5.701)}
\gppoint{gp mark 0}{(5.786,5.177)}
\gppoint{gp mark 0}{(5.786,5.456)}
\gppoint{gp mark 0}{(5.786,5.655)}
\gppoint{gp mark 0}{(5.786,5.655)}
\gppoint{gp mark 0}{(5.786,5.624)}
\gppoint{gp mark 0}{(5.786,5.308)}
\gppoint{gp mark 0}{(5.786,5.695)}
\gppoint{gp mark 0}{(5.786,5.829)}
\gppoint{gp mark 0}{(5.786,5.992)}
\gppoint{gp mark 0}{(5.786,5.716)}
\gppoint{gp mark 0}{(5.786,5.308)}
\gppoint{gp mark 0}{(5.786,5.314)}
\gppoint{gp mark 0}{(5.786,5.615)}
\gppoint{gp mark 0}{(5.786,4.573)}
\gppoint{gp mark 0}{(5.786,5.485)}
\gppoint{gp mark 0}{(5.786,4.926)}
\gppoint{gp mark 0}{(5.786,5.587)}
\gppoint{gp mark 0}{(5.786,6.066)}
\gppoint{gp mark 0}{(5.786,6.151)}
\gppoint{gp mark 0}{(5.786,5.586)}
\gppoint{gp mark 0}{(5.786,5.660)}
\gppoint{gp mark 0}{(5.786,5.172)}
\gppoint{gp mark 0}{(5.786,5.271)}
\gppoint{gp mark 0}{(5.786,5.921)}
\gppoint{gp mark 0}{(5.786,5.418)}
\gppoint{gp mark 0}{(5.786,5.535)}
\gppoint{gp mark 0}{(5.795,5.404)}
\gppoint{gp mark 0}{(5.795,4.293)}
\gppoint{gp mark 0}{(5.795,5.262)}
\gppoint{gp mark 0}{(5.795,5.872)}
\gppoint{gp mark 0}{(5.795,5.438)}
\gppoint{gp mark 0}{(5.795,5.938)}
\gppoint{gp mark 0}{(5.795,5.517)}
\gppoint{gp mark 0}{(5.795,5.902)}
\gppoint{gp mark 0}{(5.795,5.778)}
\gppoint{gp mark 0}{(5.795,5.262)}
\gppoint{gp mark 0}{(5.795,4.857)}
\gppoint{gp mark 0}{(5.795,5.093)}
\gppoint{gp mark 0}{(5.795,5.590)}
\gppoint{gp mark 0}{(5.795,4.660)}
\gppoint{gp mark 0}{(5.795,5.238)}
\gppoint{gp mark 0}{(5.795,4.837)}
\gppoint{gp mark 0}{(5.795,4.600)}
\gppoint{gp mark 0}{(5.795,5.523)}
\gppoint{gp mark 0}{(5.795,5.465)}
\gppoint{gp mark 0}{(5.795,4.960)}
\gppoint{gp mark 0}{(5.795,5.712)}
\gppoint{gp mark 0}{(5.795,5.756)}
\gppoint{gp mark 0}{(5.795,5.554)}
\gppoint{gp mark 0}{(5.795,5.327)}
\gppoint{gp mark 0}{(5.795,5.490)}
\gppoint{gp mark 0}{(5.795,5.554)}
\gppoint{gp mark 0}{(5.795,5.432)}
\gppoint{gp mark 0}{(5.795,4.966)}
\gppoint{gp mark 0}{(5.795,5.554)}
\gppoint{gp mark 0}{(5.795,5.312)}
\gppoint{gp mark 0}{(5.795,4.870)}
\gppoint{gp mark 0}{(5.795,6.052)}
\gppoint{gp mark 0}{(5.795,4.925)}
\gppoint{gp mark 0}{(5.803,5.854)}
\gppoint{gp mark 0}{(5.803,4.965)}
\gppoint{gp mark 0}{(5.803,5.804)}
\gppoint{gp mark 0}{(5.803,5.667)}
\gppoint{gp mark 0}{(5.803,5.615)}
\gppoint{gp mark 0}{(5.803,5.772)}
\gppoint{gp mark 0}{(5.803,5.598)}
\gppoint{gp mark 0}{(5.803,5.803)}
\gppoint{gp mark 0}{(5.803,5.911)}
\gppoint{gp mark 0}{(5.803,5.772)}
\gppoint{gp mark 0}{(5.803,5.092)}
\gppoint{gp mark 0}{(5.803,4.951)}
\gppoint{gp mark 0}{(5.803,5.673)}
\gppoint{gp mark 0}{(5.803,5.412)}
\gppoint{gp mark 0}{(5.803,4.762)}
\gppoint{gp mark 0}{(5.803,5.418)}
\gppoint{gp mark 0}{(5.803,5.016)}
\gppoint{gp mark 0}{(5.803,5.387)}
\gppoint{gp mark 0}{(5.803,5.673)}
\gppoint{gp mark 0}{(5.811,6.096)}
\gppoint{gp mark 0}{(5.811,5.179)}
\gppoint{gp mark 0}{(5.811,5.508)}
\gppoint{gp mark 0}{(5.811,5.477)}
\gppoint{gp mark 0}{(5.811,5.427)}
\gppoint{gp mark 0}{(5.811,6.005)}
\gppoint{gp mark 0}{(5.811,5.427)}
\gppoint{gp mark 0}{(5.811,5.797)}
\gppoint{gp mark 0}{(5.811,5.869)}
\gppoint{gp mark 0}{(5.811,5.617)}
\gppoint{gp mark 0}{(5.811,4.699)}
\gppoint{gp mark 0}{(5.811,5.241)}
\gppoint{gp mark 0}{(5.811,5.342)}
\gppoint{gp mark 0}{(5.811,4.954)}
\gppoint{gp mark 0}{(5.811,5.233)}
\gppoint{gp mark 0}{(5.811,3.856)}
\gppoint{gp mark 0}{(5.811,5.908)}
\gppoint{gp mark 0}{(5.811,5.252)}
\gppoint{gp mark 0}{(5.811,6.025)}
\gppoint{gp mark 0}{(5.811,5.661)}
\gppoint{gp mark 0}{(5.811,5.658)}
\gppoint{gp mark 0}{(5.811,4.365)}
\gppoint{gp mark 0}{(5.811,5.014)}
\gppoint{gp mark 0}{(5.811,5.673)}
\gppoint{gp mark 0}{(5.811,5.761)}
\gppoint{gp mark 0}{(5.811,6.130)}
\gppoint{gp mark 0}{(5.811,5.667)}
\gppoint{gp mark 0}{(5.811,5.800)}
\gppoint{gp mark 0}{(5.811,6.095)}
\gppoint{gp mark 0}{(5.811,4.987)}
\gppoint{gp mark 0}{(5.819,5.713)}
\gppoint{gp mark 0}{(5.819,5.815)}
\gppoint{gp mark 0}{(5.819,4.878)}
\gppoint{gp mark 0}{(5.819,5.376)}
\gppoint{gp mark 0}{(5.819,5.499)}
\gppoint{gp mark 0}{(5.819,5.048)}
\gppoint{gp mark 0}{(5.819,5.412)}
\gppoint{gp mark 0}{(5.819,4.756)}
\gppoint{gp mark 0}{(5.819,4.878)}
\gppoint{gp mark 0}{(5.819,5.752)}
\gppoint{gp mark 0}{(5.819,5.833)}
\gppoint{gp mark 0}{(5.819,5.362)}
\gppoint{gp mark 0}{(5.819,5.673)}
\gppoint{gp mark 0}{(5.819,5.299)}
\gppoint{gp mark 0}{(5.819,4.957)}
\gppoint{gp mark 0}{(5.819,5.983)}
\gppoint{gp mark 0}{(5.819,5.210)}
\gppoint{gp mark 0}{(5.819,5.171)}
\gppoint{gp mark 0}{(5.819,6.156)}
\gppoint{gp mark 0}{(5.819,5.171)}
\gppoint{gp mark 0}{(5.819,4.996)}
\gppoint{gp mark 0}{(5.819,5.038)}
\gppoint{gp mark 0}{(5.819,5.074)}
\gppoint{gp mark 0}{(5.819,5.850)}
\gppoint{gp mark 0}{(5.819,4.981)}
\gppoint{gp mark 0}{(5.819,5.109)}
\gppoint{gp mark 0}{(5.826,5.792)}
\gppoint{gp mark 0}{(5.826,5.810)}
\gppoint{gp mark 0}{(5.826,6.006)}
\gppoint{gp mark 0}{(5.826,5.117)}
\gppoint{gp mark 0}{(5.826,5.293)}
\gppoint{gp mark 0}{(5.826,5.902)}
\gppoint{gp mark 0}{(5.826,4.868)}
\gppoint{gp mark 0}{(5.826,4.954)}
\gppoint{gp mark 0}{(5.826,5.441)}
\gppoint{gp mark 0}{(5.826,4.845)}
\gppoint{gp mark 0}{(5.826,6.064)}
\gppoint{gp mark 0}{(5.826,6.045)}
\gppoint{gp mark 0}{(5.826,5.802)}
\gppoint{gp mark 0}{(5.826,5.990)}
\gppoint{gp mark 0}{(5.826,5.802)}
\gppoint{gp mark 0}{(5.826,5.295)}
\gppoint{gp mark 0}{(5.826,5.392)}
\gppoint{gp mark 0}{(5.826,5.544)}
\gppoint{gp mark 0}{(5.826,5.098)}
\gppoint{gp mark 0}{(5.826,6.237)}
\gppoint{gp mark 0}{(5.826,6.135)}
\gppoint{gp mark 0}{(5.826,5.651)}
\gppoint{gp mark 0}{(5.826,5.105)}
\gppoint{gp mark 0}{(5.826,4.888)}
\gppoint{gp mark 0}{(5.826,5.510)}
\gppoint{gp mark 0}{(5.826,5.411)}
\gppoint{gp mark 0}{(5.826,5.937)}
\gppoint{gp mark 0}{(5.834,5.582)}
\gppoint{gp mark 0}{(5.834,5.717)}
\gppoint{gp mark 0}{(5.834,5.046)}
\gppoint{gp mark 0}{(5.834,5.083)}
\gppoint{gp mark 0}{(5.834,5.291)}
\gppoint{gp mark 0}{(5.834,5.424)}
\gppoint{gp mark 0}{(5.834,4.846)}
\gppoint{gp mark 0}{(5.834,5.991)}
\gppoint{gp mark 0}{(5.834,5.357)}
\gppoint{gp mark 0}{(5.834,6.406)}
\gppoint{gp mark 0}{(5.834,5.878)}
\gppoint{gp mark 0}{(5.834,4.798)}
\gppoint{gp mark 0}{(5.834,6.406)}
\gppoint{gp mark 0}{(5.834,6.009)}
\gppoint{gp mark 0}{(5.834,6.406)}
\gppoint{gp mark 0}{(5.834,5.805)}
\gppoint{gp mark 0}{(5.834,5.559)}
\gppoint{gp mark 0}{(5.834,4.903)}
\gppoint{gp mark 0}{(5.834,5.342)}
\gppoint{gp mark 0}{(5.834,5.498)}
\gppoint{gp mark 0}{(5.834,6.180)}
\gppoint{gp mark 0}{(5.834,5.717)}
\gppoint{gp mark 0}{(5.834,6.104)}
\gppoint{gp mark 0}{(5.834,5.357)}
\gppoint{gp mark 0}{(5.834,5.825)}
\gppoint{gp mark 0}{(5.834,5.878)}
\gppoint{gp mark 0}{(5.834,6.024)}
\gppoint{gp mark 0}{(5.834,5.365)}
\gppoint{gp mark 0}{(5.842,5.563)}
\gppoint{gp mark 0}{(5.842,5.443)}
\gppoint{gp mark 0}{(5.842,5.935)}
\gppoint{gp mark 0}{(5.842,5.840)}
\gppoint{gp mark 0}{(5.842,5.576)}
\gppoint{gp mark 0}{(5.842,6.016)}
\gppoint{gp mark 0}{(5.842,5.358)}
\gppoint{gp mark 0}{(5.842,5.443)}
\gppoint{gp mark 0}{(5.842,5.279)}
\gppoint{gp mark 0}{(5.842,5.443)}
\gppoint{gp mark 0}{(5.842,5.358)}
\gppoint{gp mark 0}{(5.842,5.443)}
\gppoint{gp mark 0}{(5.842,4.951)}
\gppoint{gp mark 0}{(5.842,5.229)}
\gppoint{gp mark 0}{(5.842,5.443)}
\gppoint{gp mark 0}{(5.842,5.293)}
\gppoint{gp mark 0}{(5.842,5.991)}
\gppoint{gp mark 0}{(5.842,5.676)}
\gppoint{gp mark 0}{(5.842,4.586)}
\gppoint{gp mark 0}{(5.842,5.443)}
\gppoint{gp mark 0}{(5.842,4.623)}
\gppoint{gp mark 0}{(5.842,5.443)}
\gppoint{gp mark 0}{(5.842,6.026)}
\gppoint{gp mark 0}{(5.842,5.709)}
\gppoint{gp mark 0}{(5.842,5.443)}
\gppoint{gp mark 0}{(5.842,5.443)}
\gppoint{gp mark 0}{(5.842,5.402)}
\gppoint{gp mark 0}{(5.842,5.625)}
\gppoint{gp mark 0}{(5.842,5.069)}
\gppoint{gp mark 0}{(5.842,5.425)}
\gppoint{gp mark 0}{(5.842,5.322)}
\gppoint{gp mark 0}{(5.842,5.514)}
\gppoint{gp mark 0}{(5.842,5.443)}
\gppoint{gp mark 0}{(5.842,6.097)}
\gppoint{gp mark 0}{(5.842,5.809)}
\gppoint{gp mark 0}{(5.842,4.933)}
\gppoint{gp mark 0}{(5.842,5.910)}
\gppoint{gp mark 0}{(5.842,5.770)}
\gppoint{gp mark 0}{(5.842,5.443)}
\gppoint{gp mark 0}{(5.842,5.731)}
\gppoint{gp mark 0}{(5.850,5.572)}
\gppoint{gp mark 0}{(5.850,5.444)}
\gppoint{gp mark 0}{(5.850,5.638)}
\gppoint{gp mark 0}{(5.850,4.863)}
\gppoint{gp mark 0}{(5.850,5.618)}
\gppoint{gp mark 0}{(5.850,5.836)}
\gppoint{gp mark 0}{(5.850,5.905)}
\gppoint{gp mark 0}{(5.850,4.491)}
\gppoint{gp mark 0}{(5.850,5.624)}
\gppoint{gp mark 0}{(5.850,4.882)}
\gppoint{gp mark 0}{(5.850,5.804)}
\gppoint{gp mark 0}{(5.850,5.823)}
\gppoint{gp mark 0}{(5.850,5.142)}
\gppoint{gp mark 0}{(5.850,5.835)}
\gppoint{gp mark 0}{(5.850,5.418)}
\gppoint{gp mark 0}{(5.850,4.754)}
\gppoint{gp mark 0}{(5.850,5.418)}
\gppoint{gp mark 0}{(5.850,4.883)}
\gppoint{gp mark 0}{(5.850,5.779)}
\gppoint{gp mark 0}{(5.850,5.577)}
\gppoint{gp mark 0}{(5.850,5.434)}
\gppoint{gp mark 0}{(5.850,6.020)}
\gppoint{gp mark 0}{(5.850,5.972)}
\gppoint{gp mark 0}{(5.850,4.683)}
\gppoint{gp mark 0}{(5.850,5.666)}
\gppoint{gp mark 0}{(5.857,5.728)}
\gppoint{gp mark 0}{(5.857,6.081)}
\gppoint{gp mark 0}{(5.857,4.940)}
\gppoint{gp mark 0}{(5.857,5.637)}
\gppoint{gp mark 0}{(5.857,6.130)}
\gppoint{gp mark 0}{(5.857,6.084)}
\gppoint{gp mark 0}{(5.857,5.728)}
\gppoint{gp mark 0}{(5.857,5.383)}
\gppoint{gp mark 0}{(5.857,5.569)}
\gppoint{gp mark 0}{(5.857,4.169)}
\gppoint{gp mark 0}{(5.857,5.728)}
\gppoint{gp mark 0}{(5.857,6.126)}
\gppoint{gp mark 0}{(5.857,5.220)}
\gppoint{gp mark 0}{(5.857,5.418)}
\gppoint{gp mark 0}{(5.857,3.720)}
\gppoint{gp mark 0}{(5.857,5.301)}
\gppoint{gp mark 0}{(5.857,5.768)}
\gppoint{gp mark 0}{(5.857,5.383)}
\gppoint{gp mark 0}{(5.857,6.075)}
\gppoint{gp mark 0}{(5.857,4.915)}
\gppoint{gp mark 0}{(5.857,4.854)}
\gppoint{gp mark 0}{(5.857,5.892)}
\gppoint{gp mark 0}{(5.857,6.238)}
\gppoint{gp mark 0}{(5.857,5.722)}
\gppoint{gp mark 0}{(5.857,6.097)}
\gppoint{gp mark 0}{(5.857,5.980)}
\gppoint{gp mark 0}{(5.865,5.805)}
\gppoint{gp mark 0}{(5.865,6.010)}
\gppoint{gp mark 0}{(5.865,6.051)}
\gppoint{gp mark 0}{(5.865,5.519)}
\gppoint{gp mark 0}{(5.865,5.732)}
\gppoint{gp mark 0}{(5.865,4.672)}
\gppoint{gp mark 0}{(5.865,5.732)}
\gppoint{gp mark 0}{(5.865,5.654)}
\gppoint{gp mark 0}{(5.865,5.389)}
\gppoint{gp mark 0}{(5.865,5.707)}
\gppoint{gp mark 0}{(5.865,5.328)}
\gppoint{gp mark 0}{(5.865,4.028)}
\gppoint{gp mark 0}{(5.865,5.796)}
\gppoint{gp mark 0}{(5.865,5.852)}
\gppoint{gp mark 0}{(5.865,5.304)}
\gppoint{gp mark 0}{(5.865,4.533)}
\gppoint{gp mark 0}{(5.865,5.953)}
\gppoint{gp mark 0}{(5.865,6.223)}
\gppoint{gp mark 0}{(5.865,5.694)}
\gppoint{gp mark 0}{(5.872,5.424)}
\gppoint{gp mark 0}{(5.872,5.448)}
\gppoint{gp mark 0}{(5.872,5.104)}
\gppoint{gp mark 0}{(5.872,5.394)}
\gppoint{gp mark 0}{(5.872,5.394)}
\gppoint{gp mark 0}{(5.872,5.448)}
\gppoint{gp mark 0}{(5.872,5.579)}
\gppoint{gp mark 0}{(5.872,5.318)}
\gppoint{gp mark 0}{(5.872,6.081)}
\gppoint{gp mark 0}{(5.872,5.448)}
\gppoint{gp mark 0}{(5.872,5.521)}
\gppoint{gp mark 0}{(5.872,5.250)}
\gppoint{gp mark 0}{(5.872,5.450)}
\gppoint{gp mark 0}{(5.872,5.708)}
\gppoint{gp mark 0}{(5.872,6.119)}
\gppoint{gp mark 0}{(5.872,5.988)}
\gppoint{gp mark 0}{(5.872,6.127)}
\gppoint{gp mark 0}{(5.872,6.124)}
\gppoint{gp mark 0}{(5.872,5.448)}
\gppoint{gp mark 0}{(5.872,5.811)}
\gppoint{gp mark 0}{(5.872,5.448)}
\gppoint{gp mark 0}{(5.872,5.691)}
\gppoint{gp mark 0}{(5.872,4.564)}
\gppoint{gp mark 0}{(5.872,5.699)}
\gppoint{gp mark 0}{(5.872,5.353)}
\gppoint{gp mark 0}{(5.872,6.076)}
\gppoint{gp mark 0}{(5.872,5.448)}
\gppoint{gp mark 0}{(5.872,5.448)}
\gppoint{gp mark 0}{(5.872,6.189)}
\gppoint{gp mark 0}{(5.872,5.448)}
\gppoint{gp mark 0}{(5.872,5.193)}
\gppoint{gp mark 0}{(5.872,5.256)}
\gppoint{gp mark 0}{(5.872,5.448)}
\gppoint{gp mark 0}{(5.872,5.555)}
\gppoint{gp mark 0}{(5.872,6.104)}
\gppoint{gp mark 0}{(5.872,4.743)}
\gppoint{gp mark 0}{(5.872,5.796)}
\gppoint{gp mark 0}{(5.872,6.203)}
\gppoint{gp mark 0}{(5.880,5.448)}
\gppoint{gp mark 0}{(5.880,5.271)}
\gppoint{gp mark 0}{(5.880,5.592)}
\gppoint{gp mark 0}{(5.880,5.345)}
\gppoint{gp mark 0}{(5.880,5.754)}
\gppoint{gp mark 0}{(5.880,5.785)}
\gppoint{gp mark 0}{(5.880,6.106)}
\gppoint{gp mark 0}{(5.880,4.832)}
\gppoint{gp mark 0}{(5.880,6.221)}
\gppoint{gp mark 0}{(5.880,5.331)}
\gppoint{gp mark 0}{(5.880,5.448)}
\gppoint{gp mark 0}{(5.880,5.972)}
\gppoint{gp mark 0}{(5.880,4.745)}
\gppoint{gp mark 0}{(5.880,5.358)}
\gppoint{gp mark 0}{(5.880,6.137)}
\gppoint{gp mark 0}{(5.880,5.818)}
\gppoint{gp mark 0}{(5.880,5.448)}
\gppoint{gp mark 0}{(5.880,4.983)}
\gppoint{gp mark 0}{(5.880,5.920)}
\gppoint{gp mark 0}{(5.880,5.667)}
\gppoint{gp mark 0}{(5.880,5.775)}
\gppoint{gp mark 0}{(5.880,5.448)}
\gppoint{gp mark 0}{(5.880,6.160)}
\gppoint{gp mark 0}{(5.880,5.943)}
\gppoint{gp mark 0}{(5.880,5.818)}
\gppoint{gp mark 0}{(5.880,5.448)}
\gppoint{gp mark 0}{(5.880,5.896)}
\gppoint{gp mark 0}{(5.880,5.143)}
\gppoint{gp mark 0}{(5.880,5.476)}
\gppoint{gp mark 0}{(5.880,5.496)}
\gppoint{gp mark 0}{(5.880,5.941)}
\gppoint{gp mark 0}{(5.880,5.977)}
\gppoint{gp mark 0}{(5.880,5.782)}
\gppoint{gp mark 0}{(5.887,4.978)}
\gppoint{gp mark 0}{(5.887,5.448)}
\gppoint{gp mark 0}{(5.887,5.737)}
\gppoint{gp mark 0}{(5.887,5.028)}
\gppoint{gp mark 0}{(5.887,4.807)}
\gppoint{gp mark 0}{(5.887,5.729)}
\gppoint{gp mark 0}{(5.887,4.905)}
\gppoint{gp mark 0}{(5.887,6.743)}
\gppoint{gp mark 0}{(5.887,5.603)}
\gppoint{gp mark 0}{(5.887,4.683)}
\gppoint{gp mark 0}{(5.887,6.126)}
\gppoint{gp mark 0}{(5.887,5.982)}
\gppoint{gp mark 0}{(5.887,6.126)}
\gppoint{gp mark 0}{(5.887,4.581)}
\gppoint{gp mark 0}{(5.887,6.743)}
\gppoint{gp mark 0}{(5.887,4.937)}
\gppoint{gp mark 0}{(5.887,5.786)}
\gppoint{gp mark 0}{(5.887,5.867)}
\gppoint{gp mark 0}{(5.887,5.704)}
\gppoint{gp mark 0}{(5.887,6.381)}
\gppoint{gp mark 0}{(5.887,5.448)}
\gppoint{gp mark 0}{(5.887,4.807)}
\gppoint{gp mark 0}{(5.887,5.248)}
\gppoint{gp mark 0}{(5.887,5.448)}
\gppoint{gp mark 0}{(5.887,6.133)}
\gppoint{gp mark 0}{(5.887,5.023)}
\gppoint{gp mark 0}{(5.887,5.035)}
\gppoint{gp mark 0}{(5.887,5.101)}
\gppoint{gp mark 0}{(5.887,5.422)}
\gppoint{gp mark 0}{(5.887,6.172)}
\gppoint{gp mark 0}{(5.887,5.725)}
\gppoint{gp mark 0}{(5.887,5.870)}
\gppoint{gp mark 0}{(5.895,6.348)}
\gppoint{gp mark 0}{(5.895,5.547)}
\gppoint{gp mark 0}{(5.895,6.348)}
\gppoint{gp mark 0}{(5.895,4.947)}
\gppoint{gp mark 0}{(5.895,6.348)}
\gppoint{gp mark 0}{(5.895,5.713)}
\gppoint{gp mark 0}{(5.895,5.949)}
\gppoint{gp mark 0}{(5.895,5.661)}
\gppoint{gp mark 0}{(5.895,6.348)}
\gppoint{gp mark 0}{(5.895,6.093)}
\gppoint{gp mark 0}{(5.895,6.348)}
\gppoint{gp mark 0}{(5.895,5.282)}
\gppoint{gp mark 0}{(5.895,5.282)}
\gppoint{gp mark 0}{(5.895,6.303)}
\gppoint{gp mark 0}{(5.895,5.587)}
\gppoint{gp mark 0}{(5.895,6.348)}
\gppoint{gp mark 0}{(5.895,5.055)}
\gppoint{gp mark 0}{(5.895,5.596)}
\gppoint{gp mark 0}{(5.895,5.604)}
\gppoint{gp mark 0}{(5.895,5.393)}
\gppoint{gp mark 0}{(5.895,6.348)}
\gppoint{gp mark 0}{(5.895,6.348)}
\gppoint{gp mark 0}{(5.895,5.508)}
\gppoint{gp mark 0}{(5.895,5.497)}
\gppoint{gp mark 0}{(5.895,6.348)}
\gppoint{gp mark 0}{(5.895,5.892)}
\gppoint{gp mark 0}{(5.895,5.760)}
\gppoint{gp mark 0}{(5.895,6.348)}
\gppoint{gp mark 0}{(5.895,6.348)}
\gppoint{gp mark 0}{(5.895,6.348)}
\gppoint{gp mark 0}{(5.895,6.348)}
\gppoint{gp mark 0}{(5.895,6.348)}
\gppoint{gp mark 0}{(5.895,5.096)}
\gppoint{gp mark 0}{(5.895,6.348)}
\gppoint{gp mark 0}{(5.895,6.348)}
\gppoint{gp mark 0}{(5.895,5.228)}
\gppoint{gp mark 0}{(5.895,6.348)}
\gppoint{gp mark 0}{(5.895,6.348)}
\gppoint{gp mark 0}{(5.895,6.348)}
\gppoint{gp mark 0}{(5.895,5.731)}
\gppoint{gp mark 0}{(5.895,5.238)}
\gppoint{gp mark 0}{(5.895,6.348)}
\gppoint{gp mark 0}{(5.895,6.348)}
\gppoint{gp mark 0}{(5.895,5.730)}
\gppoint{gp mark 0}{(5.895,6.348)}
\gppoint{gp mark 0}{(5.895,5.842)}
\gppoint{gp mark 0}{(5.895,6.348)}
\gppoint{gp mark 0}{(5.895,5.768)}
\gppoint{gp mark 0}{(5.895,5.099)}
\gppoint{gp mark 0}{(5.895,5.264)}
\gppoint{gp mark 0}{(5.902,6.077)}
\gppoint{gp mark 0}{(5.902,5.725)}
\gppoint{gp mark 0}{(5.902,6.201)}
\gppoint{gp mark 0}{(5.902,5.760)}
\gppoint{gp mark 0}{(5.902,5.826)}
\gppoint{gp mark 0}{(5.902,5.185)}
\gppoint{gp mark 0}{(5.902,5.185)}
\gppoint{gp mark 0}{(5.902,6.083)}
\gppoint{gp mark 0}{(5.902,5.908)}
\gppoint{gp mark 0}{(5.902,5.977)}
\gppoint{gp mark 0}{(5.902,4.898)}
\gppoint{gp mark 0}{(5.902,4.933)}
\gppoint{gp mark 0}{(5.902,5.698)}
\gppoint{gp mark 0}{(5.902,6.007)}
\gppoint{gp mark 0}{(5.902,6.201)}
\gppoint{gp mark 0}{(5.902,6.077)}
\gppoint{gp mark 0}{(5.902,5.649)}
\gppoint{gp mark 0}{(5.902,5.979)}
\gppoint{gp mark 0}{(5.902,5.589)}
\gppoint{gp mark 0}{(5.902,6.030)}
\gppoint{gp mark 0}{(5.902,5.436)}
\gppoint{gp mark 0}{(5.909,5.779)}
\gppoint{gp mark 0}{(5.909,6.182)}
\gppoint{gp mark 0}{(5.909,6.182)}
\gppoint{gp mark 0}{(5.909,6.182)}
\gppoint{gp mark 0}{(5.909,6.182)}
\gppoint{gp mark 0}{(5.909,6.182)}
\gppoint{gp mark 0}{(5.909,6.182)}
\gppoint{gp mark 0}{(5.909,6.182)}
\gppoint{gp mark 0}{(5.909,6.182)}
\gppoint{gp mark 0}{(5.909,6.182)}
\gppoint{gp mark 0}{(5.909,6.182)}
\gppoint{gp mark 0}{(5.909,5.357)}
\gppoint{gp mark 0}{(5.909,6.182)}
\gppoint{gp mark 0}{(5.909,6.182)}
\gppoint{gp mark 0}{(5.909,5.360)}
\gppoint{gp mark 0}{(5.909,6.182)}
\gppoint{gp mark 0}{(5.909,6.182)}
\gppoint{gp mark 0}{(5.909,5.405)}
\gppoint{gp mark 0}{(5.909,5.833)}
\gppoint{gp mark 0}{(5.909,5.679)}
\gppoint{gp mark 0}{(5.909,4.923)}
\gppoint{gp mark 0}{(5.909,5.128)}
\gppoint{gp mark 0}{(5.909,5.923)}
\gppoint{gp mark 0}{(5.909,5.613)}
\gppoint{gp mark 0}{(5.909,5.371)}
\gppoint{gp mark 0}{(5.909,5.682)}
\gppoint{gp mark 0}{(5.909,5.879)}
\gppoint{gp mark 0}{(5.909,5.925)}
\gppoint{gp mark 0}{(5.909,5.516)}
\gppoint{gp mark 0}{(5.909,5.042)}
\gppoint{gp mark 0}{(5.909,5.832)}
\gppoint{gp mark 0}{(5.909,5.454)}
\gppoint{gp mark 0}{(5.909,6.539)}
\gppoint{gp mark 0}{(5.909,5.756)}
\gppoint{gp mark 0}{(5.909,5.647)}
\gppoint{gp mark 0}{(5.909,4.633)}
\gppoint{gp mark 0}{(5.909,6.142)}
\gppoint{gp mark 0}{(5.909,6.060)}
\gppoint{gp mark 0}{(5.909,5.412)}
\gppoint{gp mark 0}{(5.909,6.182)}
\gppoint{gp mark 0}{(5.909,5.880)}
\gppoint{gp mark 0}{(5.916,6.438)}
\gppoint{gp mark 0}{(5.916,5.766)}
\gppoint{gp mark 0}{(5.916,5.893)}
\gppoint{gp mark 0}{(5.916,6.085)}
\gppoint{gp mark 0}{(5.916,4.743)}
\gppoint{gp mark 0}{(5.916,5.412)}
\gppoint{gp mark 0}{(5.916,5.684)}
\gppoint{gp mark 0}{(5.916,5.172)}
\gppoint{gp mark 0}{(5.916,5.421)}
\gppoint{gp mark 0}{(5.916,6.376)}
\gppoint{gp mark 0}{(5.916,5.792)}
\gppoint{gp mark 0}{(5.916,4.898)}
\gppoint{gp mark 0}{(5.916,6.244)}
\gppoint{gp mark 0}{(5.916,5.470)}
\gppoint{gp mark 0}{(5.916,5.606)}
\gppoint{gp mark 0}{(5.916,5.789)}
\gppoint{gp mark 0}{(5.916,5.484)}
\gppoint{gp mark 0}{(5.916,6.114)}
\gppoint{gp mark 0}{(5.916,5.140)}
\gppoint{gp mark 0}{(5.916,5.402)}
\gppoint{gp mark 0}{(5.916,6.154)}
\gppoint{gp mark 0}{(5.916,5.402)}
\gppoint{gp mark 0}{(5.916,6.068)}
\gppoint{gp mark 0}{(5.916,5.783)}
\gppoint{gp mark 0}{(5.916,6.211)}
\gppoint{gp mark 0}{(5.916,6.050)}
\gppoint{gp mark 0}{(5.916,5.449)}
\gppoint{gp mark 0}{(5.916,5.799)}
\gppoint{gp mark 0}{(5.916,4.567)}
\gppoint{gp mark 0}{(5.916,5.317)}
\gppoint{gp mark 0}{(5.916,5.802)}
\gppoint{gp mark 0}{(5.916,6.201)}
\gppoint{gp mark 0}{(5.916,6.511)}
\gppoint{gp mark 0}{(5.916,5.777)}
\gppoint{gp mark 0}{(5.923,4.880)}
\gppoint{gp mark 0}{(5.923,6.109)}
\gppoint{gp mark 0}{(5.923,5.629)}
\gppoint{gp mark 0}{(5.923,5.701)}
\gppoint{gp mark 0}{(5.923,5.722)}
\gppoint{gp mark 0}{(5.923,5.302)}
\gppoint{gp mark 0}{(5.923,5.704)}
\gppoint{gp mark 0}{(5.923,5.505)}
\gppoint{gp mark 0}{(5.923,5.765)}
\gppoint{gp mark 0}{(5.923,5.827)}
\gppoint{gp mark 0}{(5.923,5.695)}
\gppoint{gp mark 0}{(5.923,5.215)}
\gppoint{gp mark 0}{(5.923,4.925)}
\gppoint{gp mark 0}{(5.923,6.528)}
\gppoint{gp mark 0}{(5.923,4.686)}
\gppoint{gp mark 0}{(5.923,6.162)}
\gppoint{gp mark 0}{(5.923,5.458)}
\gppoint{gp mark 0}{(5.923,5.117)}
\gppoint{gp mark 0}{(5.923,5.378)}
\gppoint{gp mark 0}{(5.923,5.870)}
\gppoint{gp mark 0}{(5.923,5.582)}
\gppoint{gp mark 0}{(5.923,5.748)}
\gppoint{gp mark 0}{(5.923,5.748)}
\gppoint{gp mark 0}{(5.930,6.717)}
\gppoint{gp mark 0}{(5.930,6.471)}
\gppoint{gp mark 0}{(5.930,6.072)}
\gppoint{gp mark 0}{(5.930,5.554)}
\gppoint{gp mark 0}{(5.930,5.845)}
\gppoint{gp mark 0}{(5.930,5.820)}
\gppoint{gp mark 0}{(5.930,5.845)}
\gppoint{gp mark 0}{(5.930,5.719)}
\gppoint{gp mark 0}{(5.930,5.378)}
\gppoint{gp mark 0}{(5.930,5.686)}
\gppoint{gp mark 0}{(5.930,5.007)}
\gppoint{gp mark 0}{(5.930,5.682)}
\gppoint{gp mark 0}{(5.930,5.405)}
\gppoint{gp mark 0}{(5.930,4.732)}
\gppoint{gp mark 0}{(5.930,6.902)}
\gppoint{gp mark 0}{(5.930,6.196)}
\gppoint{gp mark 0}{(5.930,5.665)}
\gppoint{gp mark 0}{(5.937,5.656)}
\gppoint{gp mark 0}{(5.937,5.436)}
\gppoint{gp mark 0}{(5.937,5.436)}
\gppoint{gp mark 0}{(5.937,5.498)}
\gppoint{gp mark 0}{(5.937,5.409)}
\gppoint{gp mark 0}{(5.937,5.076)}
\gppoint{gp mark 0}{(5.937,5.472)}
\gppoint{gp mark 0}{(5.937,5.387)}
\gppoint{gp mark 0}{(5.937,5.436)}
\gppoint{gp mark 0}{(5.937,5.074)}
\gppoint{gp mark 0}{(5.937,6.055)}
\gppoint{gp mark 0}{(5.937,5.620)}
\gppoint{gp mark 0}{(5.937,5.361)}
\gppoint{gp mark 0}{(5.937,5.436)}
\gppoint{gp mark 0}{(5.937,6.020)}
\gppoint{gp mark 0}{(5.937,5.386)}
\gppoint{gp mark 0}{(5.937,6.172)}
\gppoint{gp mark 0}{(5.937,5.383)}
\gppoint{gp mark 0}{(5.937,6.172)}
\gppoint{gp mark 0}{(5.937,5.512)}
\gppoint{gp mark 0}{(5.937,5.852)}
\gppoint{gp mark 0}{(5.937,6.032)}
\gppoint{gp mark 0}{(5.937,5.000)}
\gppoint{gp mark 0}{(5.937,4.706)}
\gppoint{gp mark 0}{(5.937,5.411)}
\gppoint{gp mark 0}{(5.944,5.992)}
\gppoint{gp mark 0}{(5.944,5.800)}
\gppoint{gp mark 0}{(5.944,5.676)}
\gppoint{gp mark 0}{(5.944,5.625)}
\gppoint{gp mark 0}{(5.944,5.800)}
\gppoint{gp mark 0}{(5.944,5.446)}
\gppoint{gp mark 0}{(5.944,5.792)}
\gppoint{gp mark 0}{(5.944,6.242)}
\gppoint{gp mark 0}{(5.944,5.989)}
\gppoint{gp mark 0}{(5.944,4.950)}
\gppoint{gp mark 0}{(5.944,6.299)}
\gppoint{gp mark 0}{(5.944,5.471)}
\gppoint{gp mark 0}{(5.944,5.252)}
\gppoint{gp mark 0}{(5.944,5.640)}
\gppoint{gp mark 0}{(5.944,5.220)}
\gppoint{gp mark 0}{(5.944,5.084)}
\gppoint{gp mark 0}{(5.951,6.468)}
\gppoint{gp mark 0}{(5.951,6.390)}
\gppoint{gp mark 0}{(5.951,6.078)}
\gppoint{gp mark 0}{(5.951,6.214)}
\gppoint{gp mark 0}{(5.951,5.605)}
\gppoint{gp mark 0}{(5.951,6.390)}
\gppoint{gp mark 0}{(5.951,6.035)}
\gppoint{gp mark 0}{(5.951,5.563)}
\gppoint{gp mark 0}{(5.951,6.352)}
\gppoint{gp mark 0}{(5.951,6.020)}
\gppoint{gp mark 0}{(5.951,6.390)}
\gppoint{gp mark 0}{(5.951,5.415)}
\gppoint{gp mark 0}{(5.951,4.895)}
\gppoint{gp mark 0}{(5.951,5.415)}
\gppoint{gp mark 0}{(5.951,5.837)}
\gppoint{gp mark 0}{(5.951,6.244)}
\gppoint{gp mark 0}{(5.951,5.431)}
\gppoint{gp mark 0}{(5.951,5.415)}
\gppoint{gp mark 0}{(5.951,5.648)}
\gppoint{gp mark 0}{(5.951,5.853)}
\gppoint{gp mark 0}{(5.951,5.807)}
\gppoint{gp mark 0}{(5.951,5.013)}
\gppoint{gp mark 0}{(5.951,5.795)}
\gppoint{gp mark 0}{(5.951,5.823)}
\gppoint{gp mark 0}{(5.951,5.548)}
\gppoint{gp mark 0}{(5.951,6.068)}
\gppoint{gp mark 0}{(5.951,5.351)}
\gppoint{gp mark 0}{(5.951,4.937)}
\gppoint{gp mark 0}{(5.951,5.611)}
\gppoint{gp mark 0}{(5.951,5.505)}
\gppoint{gp mark 0}{(5.951,6.390)}
\gppoint{gp mark 0}{(5.951,6.390)}
\gppoint{gp mark 0}{(5.951,4.346)}
\gppoint{gp mark 0}{(5.958,5.384)}
\gppoint{gp mark 0}{(5.958,6.012)}
\gppoint{gp mark 0}{(5.958,5.751)}
\gppoint{gp mark 0}{(5.958,4.724)}
\gppoint{gp mark 0}{(5.958,5.023)}
\gppoint{gp mark 0}{(5.958,5.668)}
\gppoint{gp mark 0}{(5.958,5.748)}
\gppoint{gp mark 0}{(5.958,4.724)}
\gppoint{gp mark 0}{(5.958,5.159)}
\gppoint{gp mark 0}{(5.958,4.978)}
\gppoint{gp mark 0}{(5.958,6.078)}
\gppoint{gp mark 0}{(5.958,5.882)}
\gppoint{gp mark 0}{(5.958,6.170)}
\gppoint{gp mark 0}{(5.958,6.285)}
\gppoint{gp mark 0}{(5.958,5.054)}
\gppoint{gp mark 0}{(5.958,5.218)}
\gppoint{gp mark 0}{(5.958,6.285)}
\gppoint{gp mark 0}{(5.958,4.813)}
\gppoint{gp mark 0}{(5.958,5.496)}
\gppoint{gp mark 0}{(5.958,6.338)}
\gppoint{gp mark 0}{(5.958,4.539)}
\gppoint{gp mark 0}{(5.965,5.593)}
\gppoint{gp mark 0}{(5.965,5.625)}
\gppoint{gp mark 0}{(5.965,5.313)}
\gppoint{gp mark 0}{(5.965,5.591)}
\gppoint{gp mark 0}{(5.965,6.174)}
\gppoint{gp mark 0}{(5.965,6.234)}
\gppoint{gp mark 0}{(5.965,5.705)}
\gppoint{gp mark 0}{(5.965,5.150)}
\gppoint{gp mark 0}{(5.965,5.591)}
\gppoint{gp mark 0}{(5.965,5.464)}
\gppoint{gp mark 0}{(5.965,4.980)}
\gppoint{gp mark 0}{(5.965,6.243)}
\gppoint{gp mark 0}{(5.965,4.638)}
\gppoint{gp mark 0}{(5.965,5.731)}
\gppoint{gp mark 0}{(5.965,5.464)}
\gppoint{gp mark 0}{(5.965,5.464)}
\gppoint{gp mark 0}{(5.965,5.321)}
\gppoint{gp mark 0}{(5.965,5.316)}
\gppoint{gp mark 0}{(5.965,5.923)}
\gppoint{gp mark 0}{(5.965,5.765)}
\gppoint{gp mark 0}{(5.972,6.184)}
\gppoint{gp mark 0}{(5.972,5.688)}
\gppoint{gp mark 0}{(5.972,5.980)}
\gppoint{gp mark 0}{(5.972,5.198)}
\gppoint{gp mark 0}{(5.972,5.330)}
\gppoint{gp mark 0}{(5.972,6.441)}
\gppoint{gp mark 0}{(5.972,5.913)}
\gppoint{gp mark 0}{(5.972,6.102)}
\gppoint{gp mark 0}{(5.972,5.593)}
\gppoint{gp mark 0}{(5.972,5.101)}
\gppoint{gp mark 0}{(5.972,6.081)}
\gppoint{gp mark 0}{(5.972,5.533)}
\gppoint{gp mark 0}{(5.972,5.991)}
\gppoint{gp mark 0}{(5.972,5.913)}
\gppoint{gp mark 0}{(5.972,4.905)}
\gppoint{gp mark 0}{(5.972,6.259)}
\gppoint{gp mark 0}{(5.972,6.445)}
\gppoint{gp mark 0}{(5.972,5.614)}
\gppoint{gp mark 0}{(5.972,5.855)}
\gppoint{gp mark 0}{(5.972,5.877)}
\gppoint{gp mark 0}{(5.972,5.387)}
\gppoint{gp mark 0}{(5.972,5.593)}
\gppoint{gp mark 0}{(5.972,5.760)}
\gppoint{gp mark 0}{(5.978,5.436)}
\gppoint{gp mark 0}{(5.978,6.016)}
\gppoint{gp mark 0}{(5.978,6.012)}
\gppoint{gp mark 0}{(5.978,4.843)}
\gppoint{gp mark 0}{(5.978,6.050)}
\gppoint{gp mark 0}{(5.978,5.532)}
\gppoint{gp mark 0}{(5.978,5.274)}
\gppoint{gp mark 0}{(5.978,5.173)}
\gppoint{gp mark 0}{(5.978,5.856)}
\gppoint{gp mark 0}{(5.978,6.500)}
\gppoint{gp mark 0}{(5.978,6.063)}
\gppoint{gp mark 0}{(5.978,6.063)}
\gppoint{gp mark 0}{(5.978,5.275)}
\gppoint{gp mark 0}{(5.978,5.638)}
\gppoint{gp mark 0}{(5.985,5.532)}
\gppoint{gp mark 0}{(5.985,6.256)}
\gppoint{gp mark 0}{(5.985,5.073)}
\gppoint{gp mark 0}{(5.985,5.892)}
\gppoint{gp mark 0}{(5.985,5.528)}
\gppoint{gp mark 0}{(5.985,5.839)}
\gppoint{gp mark 0}{(5.985,6.256)}
\gppoint{gp mark 0}{(5.985,4.994)}
\gppoint{gp mark 0}{(5.985,5.225)}
\gppoint{gp mark 0}{(5.985,5.900)}
\gppoint{gp mark 0}{(5.985,5.667)}
\gppoint{gp mark 0}{(5.985,6.103)}
\gppoint{gp mark 0}{(5.985,6.135)}
\gppoint{gp mark 0}{(5.985,6.231)}
\gppoint{gp mark 0}{(5.985,5.225)}
\gppoint{gp mark 0}{(5.985,6.180)}
\gppoint{gp mark 0}{(5.985,5.437)}
\gppoint{gp mark 0}{(5.985,5.884)}
\gppoint{gp mark 0}{(5.985,5.816)}
\gppoint{gp mark 0}{(5.985,6.044)}
\gppoint{gp mark 0}{(5.985,5.374)}
\gppoint{gp mark 0}{(5.985,6.190)}
\gppoint{gp mark 0}{(5.985,5.653)}
\gppoint{gp mark 0}{(5.985,5.507)}
\gppoint{gp mark 0}{(5.985,6.027)}
\gppoint{gp mark 0}{(5.985,5.575)}
\gppoint{gp mark 0}{(5.985,6.085)}
\gppoint{gp mark 0}{(5.985,4.937)}
\gppoint{gp mark 0}{(5.991,5.020)}
\gppoint{gp mark 0}{(5.991,6.024)}
\gppoint{gp mark 0}{(5.991,5.651)}
\gppoint{gp mark 0}{(5.991,6.016)}
\gppoint{gp mark 0}{(5.991,6.208)}
\gppoint{gp mark 0}{(5.991,4.241)}
\gppoint{gp mark 0}{(5.991,6.172)}
\gppoint{gp mark 0}{(5.991,6.064)}
\gppoint{gp mark 0}{(5.991,6.347)}
\gppoint{gp mark 0}{(5.991,4.978)}
\gppoint{gp mark 0}{(5.991,4.819)}
\gppoint{gp mark 0}{(5.991,5.121)}
\gppoint{gp mark 0}{(5.991,6.094)}
\gppoint{gp mark 0}{(5.991,5.764)}
\gppoint{gp mark 0}{(5.991,5.238)}
\gppoint{gp mark 0}{(5.991,5.351)}
\gppoint{gp mark 0}{(5.991,6.217)}
\gppoint{gp mark 0}{(5.991,5.698)}
\gppoint{gp mark 0}{(5.998,6.095)}
\gppoint{gp mark 0}{(5.998,5.010)}
\gppoint{gp mark 0}{(5.998,5.573)}
\gppoint{gp mark 0}{(5.998,5.900)}
\gppoint{gp mark 0}{(5.998,6.252)}
\gppoint{gp mark 0}{(5.998,5.482)}
\gppoint{gp mark 0}{(5.998,6.045)}
\gppoint{gp mark 0}{(5.998,5.862)}
\gppoint{gp mark 0}{(5.998,6.173)}
\gppoint{gp mark 0}{(5.998,5.768)}
\gppoint{gp mark 0}{(5.998,5.567)}
\gppoint{gp mark 0}{(5.998,5.847)}
\gppoint{gp mark 0}{(5.998,4.674)}
\gppoint{gp mark 0}{(5.998,5.691)}
\gppoint{gp mark 0}{(5.998,5.759)}
\gppoint{gp mark 0}{(5.998,5.191)}
\gppoint{gp mark 0}{(5.998,5.119)}
\gppoint{gp mark 0}{(5.998,6.021)}
\gppoint{gp mark 0}{(5.998,5.489)}
\gppoint{gp mark 0}{(5.998,5.317)}
\gppoint{gp mark 0}{(6.005,5.646)}
\gppoint{gp mark 0}{(6.005,6.607)}
\gppoint{gp mark 0}{(6.005,5.478)}
\gppoint{gp mark 0}{(6.005,4.992)}
\gppoint{gp mark 0}{(6.005,5.405)}
\gppoint{gp mark 0}{(6.005,5.536)}
\gppoint{gp mark 0}{(6.005,6.034)}
\gppoint{gp mark 0}{(6.005,5.864)}
\gppoint{gp mark 0}{(6.005,4.857)}
\gppoint{gp mark 0}{(6.005,5.844)}
\gppoint{gp mark 0}{(6.005,6.024)}
\gppoint{gp mark 0}{(6.005,5.844)}
\gppoint{gp mark 0}{(6.005,5.648)}
\gppoint{gp mark 0}{(6.011,6.350)}
\gppoint{gp mark 0}{(6.011,5.839)}
\gppoint{gp mark 0}{(6.011,6.765)}
\gppoint{gp mark 0}{(6.011,5.727)}
\gppoint{gp mark 0}{(6.011,5.803)}
\gppoint{gp mark 0}{(6.011,5.658)}
\gppoint{gp mark 0}{(6.011,5.920)}
\gppoint{gp mark 0}{(6.011,6.444)}
\gppoint{gp mark 0}{(6.011,5.139)}
\gppoint{gp mark 0}{(6.011,5.548)}
\gppoint{gp mark 0}{(6.011,6.441)}
\gppoint{gp mark 0}{(6.011,6.123)}
\gppoint{gp mark 0}{(6.011,4.908)}
\gppoint{gp mark 0}{(6.011,5.705)}
\gppoint{gp mark 0}{(6.011,5.249)}
\gppoint{gp mark 0}{(6.011,5.732)}
\gppoint{gp mark 0}{(6.011,5.550)}
\gppoint{gp mark 0}{(6.011,5.327)}
\gppoint{gp mark 0}{(6.011,5.524)}
\gppoint{gp mark 0}{(6.011,5.017)}
\gppoint{gp mark 0}{(6.011,5.707)}
\gppoint{gp mark 0}{(6.011,5.490)}
\gppoint{gp mark 0}{(6.011,6.204)}
\gppoint{gp mark 0}{(6.011,5.816)}
\gppoint{gp mark 0}{(6.011,5.524)}
\gppoint{gp mark 0}{(6.011,6.211)}
\gppoint{gp mark 0}{(6.017,6.025)}
\gppoint{gp mark 0}{(6.017,6.025)}
\gppoint{gp mark 0}{(6.017,5.645)}
\gppoint{gp mark 0}{(6.017,5.893)}
\gppoint{gp mark 0}{(6.017,5.303)}
\gppoint{gp mark 0}{(6.017,5.317)}
\gppoint{gp mark 0}{(6.017,5.376)}
\gppoint{gp mark 0}{(6.017,4.778)}
\gppoint{gp mark 0}{(6.017,5.804)}
\gppoint{gp mark 0}{(6.017,6.197)}
\gppoint{gp mark 0}{(6.017,5.237)}
\gppoint{gp mark 0}{(6.017,5.362)}
\gppoint{gp mark 0}{(6.017,5.213)}
\gppoint{gp mark 0}{(6.017,5.048)}
\gppoint{gp mark 0}{(6.017,6.065)}
\gppoint{gp mark 0}{(6.017,5.858)}
\gppoint{gp mark 0}{(6.017,4.936)}
\gppoint{gp mark 0}{(6.017,6.250)}
\gppoint{gp mark 0}{(6.017,5.681)}
\gppoint{gp mark 0}{(6.017,5.902)}
\gppoint{gp mark 0}{(6.024,5.838)}
\gppoint{gp mark 0}{(6.024,6.951)}
\gppoint{gp mark 0}{(6.024,5.765)}
\gppoint{gp mark 0}{(6.024,4.802)}
\gppoint{gp mark 0}{(6.024,5.561)}
\gppoint{gp mark 0}{(6.024,4.948)}
\gppoint{gp mark 0}{(6.024,4.920)}
\gppoint{gp mark 0}{(6.024,5.863)}
\gppoint{gp mark 0}{(6.024,5.285)}
\gppoint{gp mark 0}{(6.024,6.414)}
\gppoint{gp mark 0}{(6.024,6.289)}
\gppoint{gp mark 0}{(6.024,5.320)}
\gppoint{gp mark 0}{(6.024,6.052)}
\gppoint{gp mark 0}{(6.030,5.806)}
\gppoint{gp mark 0}{(6.030,6.377)}
\gppoint{gp mark 0}{(6.030,5.803)}
\gppoint{gp mark 0}{(6.030,5.926)}
\gppoint{gp mark 0}{(6.030,5.864)}
\gppoint{gp mark 0}{(6.030,5.803)}
\gppoint{gp mark 0}{(6.030,6.263)}
\gppoint{gp mark 0}{(6.030,6.002)}
\gppoint{gp mark 0}{(6.030,5.473)}
\gppoint{gp mark 0}{(6.030,5.841)}
\gppoint{gp mark 0}{(6.030,5.803)}
\gppoint{gp mark 0}{(6.030,6.168)}
\gppoint{gp mark 0}{(6.030,6.139)}
\gppoint{gp mark 0}{(6.030,5.510)}
\gppoint{gp mark 0}{(6.030,6.117)}
\gppoint{gp mark 0}{(6.030,5.021)}
\gppoint{gp mark 0}{(6.030,6.002)}
\gppoint{gp mark 0}{(6.030,5.371)}
\gppoint{gp mark 0}{(6.030,5.162)}
\gppoint{gp mark 0}{(6.030,6.321)}
\gppoint{gp mark 0}{(6.030,5.271)}
\gppoint{gp mark 0}{(6.030,6.050)}
\gppoint{gp mark 0}{(6.030,6.065)}
\gppoint{gp mark 0}{(6.030,6.164)}
\gppoint{gp mark 0}{(6.036,6.097)}
\gppoint{gp mark 0}{(6.036,5.351)}
\gppoint{gp mark 0}{(6.036,4.863)}
\gppoint{gp mark 0}{(6.036,5.285)}
\gppoint{gp mark 0}{(6.036,5.862)}
\gppoint{gp mark 0}{(6.036,6.773)}
\gppoint{gp mark 0}{(6.036,4.676)}
\gppoint{gp mark 0}{(6.036,5.638)}
\gppoint{gp mark 0}{(6.036,4.456)}
\gppoint{gp mark 0}{(6.036,6.651)}
\gppoint{gp mark 0}{(6.036,5.337)}
\gppoint{gp mark 0}{(6.036,4.676)}
\gppoint{gp mark 0}{(6.036,5.577)}
\gppoint{gp mark 0}{(6.036,5.013)}
\gppoint{gp mark 0}{(6.036,5.250)}
\gppoint{gp mark 0}{(6.036,4.676)}
\gppoint{gp mark 0}{(6.036,5.489)}
\gppoint{gp mark 0}{(6.036,6.766)}
\gppoint{gp mark 0}{(6.036,4.923)}
\gppoint{gp mark 0}{(6.036,6.296)}
\gppoint{gp mark 0}{(6.036,6.293)}
\gppoint{gp mark 0}{(6.036,5.677)}
\gppoint{gp mark 0}{(6.036,6.079)}
\gppoint{gp mark 0}{(6.036,6.079)}
\gppoint{gp mark 0}{(6.036,6.079)}
\gppoint{gp mark 0}{(6.036,6.055)}
\gppoint{gp mark 0}{(6.036,5.402)}
\gppoint{gp mark 0}{(6.036,5.710)}
\gppoint{gp mark 0}{(6.036,4.711)}
\gppoint{gp mark 0}{(6.043,6.584)}
\gppoint{gp mark 0}{(6.043,5.907)}
\gppoint{gp mark 0}{(6.043,6.045)}
\gppoint{gp mark 0}{(6.043,5.687)}
\gppoint{gp mark 0}{(6.043,6.082)}
\gppoint{gp mark 0}{(6.043,5.310)}
\gppoint{gp mark 0}{(6.043,5.564)}
\gppoint{gp mark 0}{(6.043,5.643)}
\gppoint{gp mark 0}{(6.043,6.006)}
\gppoint{gp mark 0}{(6.043,5.970)}
\gppoint{gp mark 0}{(6.043,4.676)}
\gppoint{gp mark 0}{(6.043,5.504)}
\gppoint{gp mark 0}{(6.043,5.487)}
\gppoint{gp mark 0}{(6.043,5.924)}
\gppoint{gp mark 0}{(6.043,5.864)}
\gppoint{gp mark 0}{(6.043,5.148)}
\gppoint{gp mark 0}{(6.049,5.883)}
\gppoint{gp mark 0}{(6.049,5.878)}
\gppoint{gp mark 0}{(6.049,5.454)}
\gppoint{gp mark 0}{(6.049,4.676)}
\gppoint{gp mark 0}{(6.049,6.849)}
\gppoint{gp mark 0}{(6.049,6.034)}
\gppoint{gp mark 0}{(6.049,6.416)}
\gppoint{gp mark 0}{(6.049,6.064)}
\gppoint{gp mark 0}{(6.049,6.412)}
\gppoint{gp mark 0}{(6.049,6.139)}
\gppoint{gp mark 0}{(6.049,5.811)}
\gppoint{gp mark 0}{(6.049,5.627)}
\gppoint{gp mark 0}{(6.049,6.133)}
\gppoint{gp mark 0}{(6.049,6.306)}
\gppoint{gp mark 0}{(6.049,6.092)}
\gppoint{gp mark 0}{(6.055,4.954)}
\gppoint{gp mark 0}{(6.055,6.311)}
\gppoint{gp mark 0}{(6.055,6.143)}
\gppoint{gp mark 0}{(6.055,5.477)}
\gppoint{gp mark 0}{(6.055,5.292)}
\gppoint{gp mark 0}{(6.055,4.954)}
\gppoint{gp mark 0}{(6.055,5.268)}
\gppoint{gp mark 0}{(6.055,5.881)}
\gppoint{gp mark 0}{(6.055,4.954)}
\gppoint{gp mark 0}{(6.055,6.010)}
\gppoint{gp mark 0}{(6.055,5.482)}
\gppoint{gp mark 0}{(6.055,4.954)}
\gppoint{gp mark 0}{(6.055,6.341)}
\gppoint{gp mark 0}{(6.055,4.954)}
\gppoint{gp mark 0}{(6.055,6.515)}
\gppoint{gp mark 0}{(6.055,5.786)}
\gppoint{gp mark 0}{(6.055,4.954)}
\gppoint{gp mark 0}{(6.055,4.954)}
\gppoint{gp mark 0}{(6.055,6.143)}
\gppoint{gp mark 0}{(6.055,5.760)}
\gppoint{gp mark 0}{(6.055,6.028)}
\gppoint{gp mark 0}{(6.055,4.954)}
\gppoint{gp mark 0}{(6.061,5.714)}
\gppoint{gp mark 0}{(6.061,5.820)}
\gppoint{gp mark 0}{(6.061,5.835)}
\gppoint{gp mark 0}{(6.061,6.342)}
\gppoint{gp mark 0}{(6.061,5.677)}
\gppoint{gp mark 0}{(6.061,5.800)}
\gppoint{gp mark 0}{(6.061,5.488)}
\gppoint{gp mark 0}{(6.061,5.931)}
\gppoint{gp mark 0}{(6.061,5.959)}
\gppoint{gp mark 0}{(6.061,5.919)}
\gppoint{gp mark 0}{(6.061,5.671)}
\gppoint{gp mark 0}{(6.061,6.522)}
\gppoint{gp mark 0}{(6.061,5.157)}
\gppoint{gp mark 0}{(6.061,5.672)}
\gppoint{gp mark 0}{(6.061,5.820)}
\gppoint{gp mark 0}{(6.061,5.820)}
\gppoint{gp mark 0}{(6.061,4.819)}
\gppoint{gp mark 0}{(6.061,5.741)}
\gppoint{gp mark 0}{(6.061,5.078)}
\gppoint{gp mark 0}{(6.061,4.823)}
\gppoint{gp mark 0}{(6.061,5.393)}
\gppoint{gp mark 0}{(6.061,5.820)}
\gppoint{gp mark 0}{(6.061,5.192)}
\gppoint{gp mark 0}{(6.061,5.464)}
\gppoint{gp mark 0}{(6.061,5.835)}
\gppoint{gp mark 0}{(6.067,6.046)}
\gppoint{gp mark 0}{(6.067,5.667)}
\gppoint{gp mark 0}{(6.067,5.094)}
\gppoint{gp mark 0}{(6.067,5.332)}
\gppoint{gp mark 0}{(6.067,4.921)}
\gppoint{gp mark 0}{(6.067,5.727)}
\gppoint{gp mark 0}{(6.067,5.567)}
\gppoint{gp mark 0}{(6.067,5.519)}
\gppoint{gp mark 0}{(6.067,6.375)}
\gppoint{gp mark 0}{(6.067,4.695)}
\gppoint{gp mark 0}{(6.067,5.018)}
\gppoint{gp mark 0}{(6.067,6.603)}
\gppoint{gp mark 0}{(6.067,6.291)}
\gppoint{gp mark 0}{(6.067,5.846)}
\gppoint{gp mark 0}{(6.067,5.274)}
\gppoint{gp mark 0}{(6.067,5.317)}
\gppoint{gp mark 0}{(6.073,5.382)}
\gppoint{gp mark 0}{(6.073,5.285)}
\gppoint{gp mark 0}{(6.073,5.093)}
\gppoint{gp mark 0}{(6.073,6.011)}
\gppoint{gp mark 0}{(6.073,5.548)}
\gppoint{gp mark 0}{(6.073,5.910)}
\gppoint{gp mark 0}{(6.073,6.073)}
\gppoint{gp mark 0}{(6.073,5.409)}
\gppoint{gp mark 0}{(6.073,6.184)}
\gppoint{gp mark 0}{(6.073,5.968)}
\gppoint{gp mark 0}{(6.073,5.884)}
\gppoint{gp mark 0}{(6.073,5.625)}
\gppoint{gp mark 0}{(6.073,6.365)}
\gppoint{gp mark 0}{(6.073,6.201)}
\gppoint{gp mark 0}{(6.073,6.252)}
\gppoint{gp mark 0}{(6.073,5.920)}
\gppoint{gp mark 0}{(6.073,5.957)}
\gppoint{gp mark 0}{(6.073,5.935)}
\gppoint{gp mark 0}{(6.073,5.709)}
\gppoint{gp mark 0}{(6.073,5.884)}
\gppoint{gp mark 0}{(6.073,6.010)}
\gppoint{gp mark 0}{(6.073,5.874)}
\gppoint{gp mark 0}{(6.079,5.780)}
\gppoint{gp mark 0}{(6.079,6.424)}
\gppoint{gp mark 0}{(6.079,5.808)}
\gppoint{gp mark 0}{(6.079,5.134)}
\gppoint{gp mark 0}{(6.079,5.394)}
\gppoint{gp mark 0}{(6.079,4.693)}
\gppoint{gp mark 0}{(6.079,4.939)}
\gppoint{gp mark 0}{(6.079,5.575)}
\gppoint{gp mark 0}{(6.079,5.954)}
\gppoint{gp mark 0}{(6.079,5.366)}
\gppoint{gp mark 0}{(6.079,6.126)}
\gppoint{gp mark 0}{(6.079,5.749)}
\gppoint{gp mark 0}{(6.079,6.538)}
\gppoint{gp mark 0}{(6.079,5.151)}
\gppoint{gp mark 0}{(6.079,6.151)}
\gppoint{gp mark 0}{(6.079,5.991)}
\gppoint{gp mark 0}{(6.079,5.965)}
\gppoint{gp mark 0}{(6.079,5.266)}
\gppoint{gp mark 0}{(6.085,5.955)}
\gppoint{gp mark 0}{(6.085,5.683)}
\gppoint{gp mark 0}{(6.085,5.794)}
\gppoint{gp mark 0}{(6.085,5.408)}
\gppoint{gp mark 0}{(6.085,5.128)}
\gppoint{gp mark 0}{(6.085,5.697)}
\gppoint{gp mark 0}{(6.085,6.346)}
\gppoint{gp mark 0}{(6.085,5.972)}
\gppoint{gp mark 0}{(6.085,6.269)}
\gppoint{gp mark 0}{(6.085,6.014)}
\gppoint{gp mark 0}{(6.085,5.888)}
\gppoint{gp mark 0}{(6.091,5.155)}
\gppoint{gp mark 0}{(6.091,5.561)}
\gppoint{gp mark 0}{(6.091,5.726)}
\gppoint{gp mark 0}{(6.091,5.924)}
\gppoint{gp mark 0}{(6.091,5.300)}
\gppoint{gp mark 0}{(6.091,5.703)}
\gppoint{gp mark 0}{(6.091,5.606)}
\gppoint{gp mark 0}{(6.091,5.643)}
\gppoint{gp mark 0}{(6.091,5.523)}
\gppoint{gp mark 0}{(6.091,5.708)}
\gppoint{gp mark 0}{(6.091,5.849)}
\gppoint{gp mark 0}{(6.091,6.174)}
\gppoint{gp mark 0}{(6.091,5.779)}
\gppoint{gp mark 0}{(6.091,6.052)}
\gppoint{gp mark 0}{(6.091,4.713)}
\gppoint{gp mark 0}{(6.091,5.926)}
\gppoint{gp mark 0}{(6.091,6.237)}
\gppoint{gp mark 0}{(6.097,5.949)}
\gppoint{gp mark 0}{(6.097,6.490)}
\gppoint{gp mark 0}{(6.097,5.828)}
\gppoint{gp mark 0}{(6.097,5.188)}
\gppoint{gp mark 0}{(6.097,4.993)}
\gppoint{gp mark 0}{(6.097,5.081)}
\gppoint{gp mark 0}{(6.097,5.978)}
\gppoint{gp mark 0}{(6.097,5.908)}
\gppoint{gp mark 0}{(6.097,5.472)}
\gppoint{gp mark 0}{(6.097,5.492)}
\gppoint{gp mark 0}{(6.097,6.227)}
\gppoint{gp mark 0}{(6.097,6.406)}
\gppoint{gp mark 0}{(6.097,6.369)}
\gppoint{gp mark 0}{(6.097,6.544)}
\gppoint{gp mark 0}{(6.097,6.367)}
\gppoint{gp mark 0}{(6.097,5.615)}
\gppoint{gp mark 0}{(6.097,5.771)}
\gppoint{gp mark 0}{(6.097,4.518)}
\gppoint{gp mark 0}{(6.103,5.675)}
\gppoint{gp mark 0}{(6.103,6.121)}
\gppoint{gp mark 0}{(6.103,6.096)}
\gppoint{gp mark 0}{(6.103,5.948)}
\gppoint{gp mark 0}{(6.103,6.884)}
\gppoint{gp mark 0}{(6.103,6.369)}
\gppoint{gp mark 0}{(6.103,6.256)}
\gppoint{gp mark 0}{(6.103,4.509)}
\gppoint{gp mark 0}{(6.103,5.878)}
\gppoint{gp mark 0}{(6.103,6.501)}
\gppoint{gp mark 0}{(6.103,5.871)}
\gppoint{gp mark 0}{(6.103,6.494)}
\gppoint{gp mark 0}{(6.103,6.991)}
\gppoint{gp mark 0}{(6.103,6.503)}
\gppoint{gp mark 0}{(6.103,5.664)}
\gppoint{gp mark 0}{(6.103,6.282)}
\gppoint{gp mark 0}{(6.103,5.822)}
\gppoint{gp mark 0}{(6.109,5.878)}
\gppoint{gp mark 0}{(6.109,5.601)}
\gppoint{gp mark 0}{(6.109,5.093)}
\gppoint{gp mark 0}{(6.109,6.251)}
\gppoint{gp mark 0}{(6.109,5.793)}
\gppoint{gp mark 0}{(6.109,6.068)}
\gppoint{gp mark 0}{(6.109,5.605)}
\gppoint{gp mark 0}{(6.109,5.902)}
\gppoint{gp mark 0}{(6.109,5.672)}
\gppoint{gp mark 0}{(6.109,5.902)}
\gppoint{gp mark 0}{(6.109,6.415)}
\gppoint{gp mark 0}{(6.109,6.349)}
\gppoint{gp mark 0}{(6.109,5.446)}
\gppoint{gp mark 0}{(6.109,6.321)}
\gppoint{gp mark 0}{(6.109,5.673)}
\gppoint{gp mark 0}{(6.115,5.437)}
\gppoint{gp mark 0}{(6.115,6.714)}
\gppoint{gp mark 0}{(6.115,6.154)}
\gppoint{gp mark 0}{(6.115,6.212)}
\gppoint{gp mark 0}{(6.115,5.783)}
\gppoint{gp mark 0}{(6.115,6.072)}
\gppoint{gp mark 0}{(6.115,6.212)}
\gppoint{gp mark 0}{(6.115,5.307)}
\gppoint{gp mark 0}{(6.115,6.030)}
\gppoint{gp mark 0}{(6.115,6.243)}
\gppoint{gp mark 0}{(6.115,6.212)}
\gppoint{gp mark 0}{(6.115,5.916)}
\gppoint{gp mark 0}{(6.115,6.154)}
\gppoint{gp mark 0}{(6.115,6.286)}
\gppoint{gp mark 0}{(6.115,5.927)}
\gppoint{gp mark 0}{(6.120,5.912)}
\gppoint{gp mark 0}{(6.120,5.505)}
\gppoint{gp mark 0}{(6.120,5.974)}
\gppoint{gp mark 0}{(6.120,5.984)}
\gppoint{gp mark 0}{(6.120,4.790)}
\gppoint{gp mark 0}{(6.120,6.035)}
\gppoint{gp mark 0}{(6.120,5.125)}
\gppoint{gp mark 0}{(6.120,6.521)}
\gppoint{gp mark 0}{(6.120,6.163)}
\gppoint{gp mark 0}{(6.120,5.857)}
\gppoint{gp mark 0}{(6.120,6.003)}
\gppoint{gp mark 0}{(6.120,5.830)}
\gppoint{gp mark 0}{(6.120,4.794)}
\gppoint{gp mark 0}{(6.120,5.943)}
\gppoint{gp mark 0}{(6.120,6.315)}
\gppoint{gp mark 0}{(6.120,6.495)}
\gppoint{gp mark 0}{(6.120,5.002)}
\gppoint{gp mark 0}{(6.120,6.487)}
\gppoint{gp mark 0}{(6.120,5.991)}
\gppoint{gp mark 0}{(6.120,4.870)}
\gppoint{gp mark 0}{(6.120,5.985)}
\gppoint{gp mark 0}{(6.120,5.151)}
\gppoint{gp mark 0}{(6.120,6.347)}
\gppoint{gp mark 0}{(6.126,6.429)}
\gppoint{gp mark 0}{(6.126,5.972)}
\gppoint{gp mark 0}{(6.126,6.293)}
\gppoint{gp mark 0}{(6.126,6.276)}
\gppoint{gp mark 0}{(6.126,6.494)}
\gppoint{gp mark 0}{(6.126,6.149)}
\gppoint{gp mark 0}{(6.126,5.125)}
\gppoint{gp mark 0}{(6.126,6.067)}
\gppoint{gp mark 0}{(6.126,5.776)}
\gppoint{gp mark 0}{(6.126,6.273)}
\gppoint{gp mark 0}{(6.126,5.961)}
\gppoint{gp mark 0}{(6.126,6.422)}
\gppoint{gp mark 0}{(6.126,6.226)}
\gppoint{gp mark 0}{(6.126,6.160)}
\gppoint{gp mark 0}{(6.126,6.160)}
\gppoint{gp mark 0}{(6.126,5.777)}
\gppoint{gp mark 0}{(6.126,5.452)}
\gppoint{gp mark 0}{(6.126,6.094)}
\gppoint{gp mark 0}{(6.126,5.944)}
\gppoint{gp mark 0}{(6.126,6.280)}
\gppoint{gp mark 0}{(6.126,6.896)}
\gppoint{gp mark 0}{(6.126,6.270)}
\gppoint{gp mark 0}{(6.126,6.270)}
\gppoint{gp mark 0}{(6.126,6.074)}
\gppoint{gp mark 0}{(6.132,5.782)}
\gppoint{gp mark 0}{(6.132,6.202)}
\gppoint{gp mark 0}{(6.132,6.245)}
\gppoint{gp mark 0}{(6.132,5.721)}
\gppoint{gp mark 0}{(6.132,6.186)}
\gppoint{gp mark 0}{(6.132,5.496)}
\gppoint{gp mark 0}{(6.132,4.983)}
\gppoint{gp mark 0}{(6.132,4.739)}
\gppoint{gp mark 0}{(6.132,6.410)}
\gppoint{gp mark 0}{(6.132,6.323)}
\gppoint{gp mark 0}{(6.132,6.155)}
\gppoint{gp mark 0}{(6.132,5.142)}
\gppoint{gp mark 0}{(6.132,6.475)}
\gppoint{gp mark 0}{(6.132,5.466)}
\gppoint{gp mark 0}{(6.132,6.370)}
\gppoint{gp mark 0}{(6.132,5.618)}
\gppoint{gp mark 0}{(6.132,6.370)}
\gppoint{gp mark 0}{(6.132,5.787)}
\gppoint{gp mark 0}{(6.132,6.155)}
\gppoint{gp mark 0}{(6.132,6.140)}
\gppoint{gp mark 0}{(6.132,6.502)}
\gppoint{gp mark 0}{(6.132,5.885)}
\gppoint{gp mark 0}{(6.132,5.129)}
\gppoint{gp mark 0}{(6.132,6.580)}
\gppoint{gp mark 0}{(6.132,5.649)}
\gppoint{gp mark 0}{(6.138,6.704)}
\gppoint{gp mark 0}{(6.138,6.582)}
\gppoint{gp mark 0}{(6.138,5.009)}
\gppoint{gp mark 0}{(6.138,5.928)}
\gppoint{gp mark 0}{(6.138,5.810)}
\gppoint{gp mark 0}{(6.138,6.269)}
\gppoint{gp mark 0}{(6.138,5.469)}
\gppoint{gp mark 0}{(6.138,6.271)}
\gppoint{gp mark 0}{(6.138,6.220)}
\gppoint{gp mark 0}{(6.138,5.972)}
\gppoint{gp mark 0}{(6.138,6.239)}
\gppoint{gp mark 0}{(6.138,5.928)}
\gppoint{gp mark 0}{(6.138,6.516)}
\gppoint{gp mark 0}{(6.138,6.411)}
\gppoint{gp mark 0}{(6.138,5.887)}
\gppoint{gp mark 0}{(6.143,6.108)}
\gppoint{gp mark 0}{(6.143,6.265)}
\gppoint{gp mark 0}{(6.143,5.812)}
\gppoint{gp mark 0}{(6.143,6.751)}
\gppoint{gp mark 0}{(6.143,5.901)}
\gppoint{gp mark 0}{(6.143,5.999)}
\gppoint{gp mark 0}{(6.143,6.121)}
\gppoint{gp mark 0}{(6.143,6.145)}
\gppoint{gp mark 0}{(6.143,6.355)}
\gppoint{gp mark 0}{(6.143,6.121)}
\gppoint{gp mark 0}{(6.143,6.154)}
\gppoint{gp mark 0}{(6.143,5.706)}
\gppoint{gp mark 0}{(6.143,5.709)}
\gppoint{gp mark 0}{(6.143,5.789)}
\gppoint{gp mark 0}{(6.149,5.841)}
\gppoint{gp mark 0}{(6.149,6.819)}
\gppoint{gp mark 0}{(6.149,6.008)}
\gppoint{gp mark 0}{(6.149,6.038)}
\gppoint{gp mark 0}{(6.149,6.366)}
\gppoint{gp mark 0}{(6.149,4.401)}
\gppoint{gp mark 0}{(6.149,5.825)}
\gppoint{gp mark 0}{(6.149,5.383)}
\gppoint{gp mark 0}{(6.149,5.025)}
\gppoint{gp mark 0}{(6.149,5.025)}
\gppoint{gp mark 0}{(6.149,6.312)}
\gppoint{gp mark 0}{(6.149,5.025)}
\gppoint{gp mark 0}{(6.149,5.556)}
\gppoint{gp mark 0}{(6.149,5.686)}
\gppoint{gp mark 0}{(6.149,5.383)}
\gppoint{gp mark 0}{(6.149,6.312)}
\gppoint{gp mark 0}{(6.149,5.610)}
\gppoint{gp mark 0}{(6.149,5.003)}
\gppoint{gp mark 0}{(6.149,5.025)}
\gppoint{gp mark 0}{(6.149,6.296)}
\gppoint{gp mark 0}{(6.149,6.108)}
\gppoint{gp mark 0}{(6.149,6.441)}
\gppoint{gp mark 0}{(6.149,6.219)}
\gppoint{gp mark 0}{(6.149,5.025)}
\gppoint{gp mark 0}{(6.154,5.183)}
\gppoint{gp mark 0}{(6.154,6.226)}
\gppoint{gp mark 0}{(6.154,5.021)}
\gppoint{gp mark 0}{(6.154,5.331)}
\gppoint{gp mark 0}{(6.154,6.400)}
\gppoint{gp mark 0}{(6.154,5.730)}
\gppoint{gp mark 0}{(6.154,5.528)}
\gppoint{gp mark 0}{(6.154,5.626)}
\gppoint{gp mark 0}{(6.154,5.784)}
\gppoint{gp mark 0}{(6.154,6.226)}
\gppoint{gp mark 0}{(6.154,6.030)}
\gppoint{gp mark 0}{(6.154,6.181)}
\gppoint{gp mark 0}{(6.154,4.984)}
\gppoint{gp mark 0}{(6.154,6.230)}
\gppoint{gp mark 0}{(6.154,6.015)}
\gppoint{gp mark 0}{(6.154,6.226)}
\gppoint{gp mark 0}{(6.154,5.437)}
\gppoint{gp mark 0}{(6.154,5.863)}
\gppoint{gp mark 0}{(6.154,5.662)}
\gppoint{gp mark 0}{(6.154,5.618)}
\gppoint{gp mark 0}{(6.154,6.794)}
\gppoint{gp mark 0}{(6.154,6.226)}
\gppoint{gp mark 0}{(6.154,6.046)}
\gppoint{gp mark 0}{(6.154,6.069)}
\gppoint{gp mark 0}{(6.154,6.575)}
\gppoint{gp mark 0}{(6.154,6.226)}
\gppoint{gp mark 0}{(6.154,6.226)}
\gppoint{gp mark 0}{(6.160,6.550)}
\gppoint{gp mark 0}{(6.160,6.629)}
\gppoint{gp mark 0}{(6.160,5.013)}
\gppoint{gp mark 0}{(6.160,5.477)}
\gppoint{gp mark 0}{(6.160,7.008)}
\gppoint{gp mark 0}{(6.160,6.004)}
\gppoint{gp mark 0}{(6.160,5.993)}
\gppoint{gp mark 0}{(6.160,5.694)}
\gppoint{gp mark 0}{(6.160,5.699)}
\gppoint{gp mark 0}{(6.160,6.858)}
\gppoint{gp mark 0}{(6.160,6.284)}
\gppoint{gp mark 0}{(6.160,5.716)}
\gppoint{gp mark 0}{(6.160,6.166)}
\gppoint{gp mark 0}{(6.160,5.378)}
\gppoint{gp mark 0}{(6.160,5.027)}
\gppoint{gp mark 0}{(6.160,5.027)}
\gppoint{gp mark 0}{(6.165,6.296)}
\gppoint{gp mark 0}{(6.165,6.252)}
\gppoint{gp mark 0}{(6.165,6.894)}
\gppoint{gp mark 0}{(6.165,5.160)}
\gppoint{gp mark 0}{(6.165,5.434)}
\gppoint{gp mark 0}{(6.165,6.102)}
\gppoint{gp mark 0}{(6.165,5.459)}
\gppoint{gp mark 0}{(6.165,5.884)}
\gppoint{gp mark 0}{(6.165,5.613)}
\gppoint{gp mark 0}{(6.165,6.357)}
\gppoint{gp mark 0}{(6.165,6.555)}
\gppoint{gp mark 0}{(6.165,5.976)}
\gppoint{gp mark 0}{(6.165,5.243)}
\gppoint{gp mark 0}{(6.165,5.631)}
\gppoint{gp mark 0}{(6.165,5.633)}
\gppoint{gp mark 0}{(6.165,4.828)}
\gppoint{gp mark 0}{(6.165,6.443)}
\gppoint{gp mark 0}{(6.165,5.778)}
\gppoint{gp mark 0}{(6.165,5.626)}
\gppoint{gp mark 0}{(6.165,6.195)}
\gppoint{gp mark 0}{(6.165,6.043)}
\gppoint{gp mark 0}{(6.165,5.856)}
\gppoint{gp mark 0}{(6.165,6.471)}
\gppoint{gp mark 0}{(6.165,5.354)}
\gppoint{gp mark 0}{(6.165,4.813)}
\gppoint{gp mark 0}{(6.171,6.486)}
\gppoint{gp mark 0}{(6.171,6.305)}
\gppoint{gp mark 0}{(6.171,6.338)}
\gppoint{gp mark 0}{(6.171,6.758)}
\gppoint{gp mark 0}{(6.171,5.359)}
\gppoint{gp mark 0}{(6.171,6.081)}
\gppoint{gp mark 0}{(6.171,5.970)}
\gppoint{gp mark 0}{(6.171,6.178)}
\gppoint{gp mark 0}{(6.171,5.679)}
\gppoint{gp mark 0}{(6.171,5.618)}
\gppoint{gp mark 0}{(6.171,5.479)}
\gppoint{gp mark 0}{(6.171,5.126)}
\gppoint{gp mark 0}{(6.171,6.347)}
\gppoint{gp mark 0}{(6.171,6.305)}
\gppoint{gp mark 0}{(6.171,6.305)}
\gppoint{gp mark 0}{(6.171,5.076)}
\gppoint{gp mark 0}{(6.171,5.799)}
\gppoint{gp mark 0}{(6.171,5.773)}
\gppoint{gp mark 0}{(6.171,5.343)}
\gppoint{gp mark 0}{(6.171,5.531)}
\gppoint{gp mark 0}{(6.171,5.774)}
\gppoint{gp mark 0}{(6.171,6.305)}
\gppoint{gp mark 0}{(6.171,5.281)}
\gppoint{gp mark 0}{(6.171,5.859)}
\gppoint{gp mark 0}{(6.176,5.556)}
\gppoint{gp mark 0}{(6.176,5.738)}
\gppoint{gp mark 0}{(6.176,5.867)}
\gppoint{gp mark 0}{(6.176,6.027)}
\gppoint{gp mark 0}{(6.176,5.046)}
\gppoint{gp mark 0}{(6.176,5.291)}
\gppoint{gp mark 0}{(6.176,6.185)}
\gppoint{gp mark 0}{(6.176,5.524)}
\gppoint{gp mark 0}{(6.176,5.046)}
\gppoint{gp mark 0}{(6.176,6.326)}
\gppoint{gp mark 0}{(6.176,5.308)}
\gppoint{gp mark 0}{(6.176,5.176)}
\gppoint{gp mark 0}{(6.176,5.614)}
\gppoint{gp mark 0}{(6.176,6.237)}
\gppoint{gp mark 0}{(6.176,6.128)}
\gppoint{gp mark 0}{(6.176,5.755)}
\gppoint{gp mark 0}{(6.176,4.892)}
\gppoint{gp mark 0}{(6.176,6.097)}
\gppoint{gp mark 0}{(6.182,5.196)}
\gppoint{gp mark 0}{(6.182,5.211)}
\gppoint{gp mark 0}{(6.182,6.051)}
\gppoint{gp mark 0}{(6.182,5.909)}
\gppoint{gp mark 0}{(6.182,5.723)}
\gppoint{gp mark 0}{(6.182,5.794)}
\gppoint{gp mark 0}{(6.182,5.867)}
\gppoint{gp mark 0}{(6.182,6.369)}
\gppoint{gp mark 0}{(6.182,5.909)}
\gppoint{gp mark 0}{(6.182,5.710)}
\gppoint{gp mark 0}{(6.182,5.909)}
\gppoint{gp mark 0}{(6.182,5.909)}
\gppoint{gp mark 0}{(6.182,6.581)}
\gppoint{gp mark 0}{(6.182,6.338)}
\gppoint{gp mark 0}{(6.182,6.004)}
\gppoint{gp mark 0}{(6.182,6.351)}
\gppoint{gp mark 0}{(6.182,5.265)}
\gppoint{gp mark 0}{(6.182,5.506)}
\gppoint{gp mark 0}{(6.182,6.176)}
\gppoint{gp mark 0}{(6.182,5.884)}
\gppoint{gp mark 0}{(6.182,5.038)}
\gppoint{gp mark 0}{(6.182,5.981)}
\gppoint{gp mark 0}{(6.182,5.038)}
\gppoint{gp mark 0}{(6.182,6.670)}
\gppoint{gp mark 0}{(6.182,5.354)}
\gppoint{gp mark 0}{(6.187,4.948)}
\gppoint{gp mark 0}{(6.187,6.068)}
\gppoint{gp mark 0}{(6.187,6.449)}
\gppoint{gp mark 0}{(6.187,5.746)}
\gppoint{gp mark 0}{(6.187,5.832)}
\gppoint{gp mark 0}{(6.187,6.457)}
\gppoint{gp mark 0}{(6.187,6.180)}
\gppoint{gp mark 0}{(6.187,5.109)}
\gppoint{gp mark 0}{(6.187,6.295)}
\gppoint{gp mark 0}{(6.187,6.046)}
\gppoint{gp mark 0}{(6.187,6.007)}
\gppoint{gp mark 0}{(6.187,6.105)}
\gppoint{gp mark 0}{(6.187,6.154)}
\gppoint{gp mark 0}{(6.187,5.470)}
\gppoint{gp mark 0}{(6.187,5.152)}
\gppoint{gp mark 0}{(6.187,6.050)}
\gppoint{gp mark 0}{(6.187,6.476)}
\gppoint{gp mark 0}{(6.187,6.604)}
\gppoint{gp mark 0}{(6.187,5.521)}
\gppoint{gp mark 0}{(6.192,5.772)}
\gppoint{gp mark 0}{(6.192,5.949)}
\gppoint{gp mark 0}{(6.192,5.772)}
\gppoint{gp mark 0}{(6.192,5.772)}
\gppoint{gp mark 0}{(6.192,5.772)}
\gppoint{gp mark 0}{(6.192,5.772)}
\gppoint{gp mark 0}{(6.192,5.772)}
\gppoint{gp mark 0}{(6.192,6.341)}
\gppoint{gp mark 0}{(6.192,5.103)}
\gppoint{gp mark 0}{(6.192,5.772)}
\gppoint{gp mark 0}{(6.192,5.772)}
\gppoint{gp mark 0}{(6.192,5.772)}
\gppoint{gp mark 0}{(6.192,6.075)}
\gppoint{gp mark 0}{(6.192,5.772)}
\gppoint{gp mark 0}{(6.192,5.511)}
\gppoint{gp mark 0}{(6.192,5.772)}
\gppoint{gp mark 0}{(6.192,6.046)}
\gppoint{gp mark 0}{(6.192,5.772)}
\gppoint{gp mark 0}{(6.192,4.984)}
\gppoint{gp mark 0}{(6.192,5.772)}
\gppoint{gp mark 0}{(6.192,5.772)}
\gppoint{gp mark 0}{(6.192,5.794)}
\gppoint{gp mark 0}{(6.192,5.772)}
\gppoint{gp mark 0}{(6.192,5.772)}
\gppoint{gp mark 0}{(6.192,5.454)}
\gppoint{gp mark 0}{(6.192,5.772)}
\gppoint{gp mark 0}{(6.192,5.953)}
\gppoint{gp mark 0}{(6.192,5.772)}
\gppoint{gp mark 0}{(6.192,5.772)}
\gppoint{gp mark 0}{(6.192,5.772)}
\gppoint{gp mark 0}{(6.192,4.984)}
\gppoint{gp mark 0}{(6.198,6.188)}
\gppoint{gp mark 0}{(6.198,5.683)}
\gppoint{gp mark 0}{(6.198,5.402)}
\gppoint{gp mark 0}{(6.198,5.770)}
\gppoint{gp mark 0}{(6.198,6.394)}
\gppoint{gp mark 0}{(6.198,6.173)}
\gppoint{gp mark 0}{(6.198,6.115)}
\gppoint{gp mark 0}{(6.198,6.584)}
\gppoint{gp mark 0}{(6.198,5.182)}
\gppoint{gp mark 0}{(6.198,6.188)}
\gppoint{gp mark 0}{(6.198,6.700)}
\gppoint{gp mark 0}{(6.198,6.074)}
\gppoint{gp mark 0}{(6.198,5.783)}
\gppoint{gp mark 0}{(6.203,5.432)}
\gppoint{gp mark 0}{(6.203,6.494)}
\gppoint{gp mark 0}{(6.203,6.271)}
\gppoint{gp mark 0}{(6.203,6.456)}
\gppoint{gp mark 0}{(6.203,6.514)}
\gppoint{gp mark 0}{(6.203,5.660)}
\gppoint{gp mark 0}{(6.203,6.319)}
\gppoint{gp mark 0}{(6.203,6.051)}
\gppoint{gp mark 0}{(6.203,5.750)}
\gppoint{gp mark 0}{(6.203,6.321)}
\gppoint{gp mark 0}{(6.203,5.195)}
\gppoint{gp mark 0}{(6.203,6.938)}
\gppoint{gp mark 0}{(6.203,6.773)}
\gppoint{gp mark 0}{(6.203,6.174)}
\gppoint{gp mark 0}{(6.203,5.791)}
\gppoint{gp mark 0}{(6.203,6.234)}
\gppoint{gp mark 0}{(6.203,5.159)}
\gppoint{gp mark 0}{(6.203,6.074)}
\gppoint{gp mark 0}{(6.203,5.853)}
\gppoint{gp mark 0}{(6.203,4.859)}
\gppoint{gp mark 0}{(6.208,5.258)}
\gppoint{gp mark 0}{(6.208,5.916)}
\gppoint{gp mark 0}{(6.208,5.556)}
\gppoint{gp mark 0}{(6.208,6.558)}
\gppoint{gp mark 0}{(6.208,5.842)}
\gppoint{gp mark 0}{(6.208,5.502)}
\gppoint{gp mark 0}{(6.208,7.101)}
\gppoint{gp mark 0}{(6.208,5.142)}
\gppoint{gp mark 0}{(6.208,6.962)}
\gppoint{gp mark 0}{(6.208,6.285)}
\gppoint{gp mark 0}{(6.208,6.134)}
\gppoint{gp mark 0}{(6.208,6.017)}
\gppoint{gp mark 0}{(6.208,5.656)}
\gppoint{gp mark 0}{(6.208,6.775)}
\gppoint{gp mark 0}{(6.208,6.562)}
\gppoint{gp mark 0}{(6.208,5.769)}
\gppoint{gp mark 0}{(6.208,5.492)}
\gppoint{gp mark 0}{(6.208,5.871)}
\gppoint{gp mark 0}{(6.208,5.419)}
\gppoint{gp mark 0}{(6.208,5.575)}
\gppoint{gp mark 0}{(6.208,6.030)}
\gppoint{gp mark 0}{(6.208,6.109)}
\gppoint{gp mark 0}{(6.213,5.296)}
\gppoint{gp mark 0}{(6.213,5.074)}
\gppoint{gp mark 0}{(6.213,6.128)}
\gppoint{gp mark 0}{(6.213,6.226)}
\gppoint{gp mark 0}{(6.213,5.871)}
\gppoint{gp mark 0}{(6.213,6.044)}
\gppoint{gp mark 0}{(6.213,6.282)}
\gppoint{gp mark 0}{(6.213,6.646)}
\gppoint{gp mark 0}{(6.213,6.471)}
\gppoint{gp mark 0}{(6.213,5.802)}
\gppoint{gp mark 0}{(6.213,5.258)}
\gppoint{gp mark 0}{(6.213,5.799)}
\gppoint{gp mark 0}{(6.213,5.799)}
\gppoint{gp mark 0}{(6.213,6.411)}
\gppoint{gp mark 0}{(6.213,5.802)}
\gppoint{gp mark 0}{(6.213,5.552)}
\gppoint{gp mark 0}{(6.213,5.013)}
\gppoint{gp mark 0}{(6.213,5.261)}
\gppoint{gp mark 0}{(6.213,5.258)}
\gppoint{gp mark 0}{(6.213,6.390)}
\gppoint{gp mark 0}{(6.218,5.827)}
\gppoint{gp mark 0}{(6.218,6.227)}
\gppoint{gp mark 0}{(6.218,6.133)}
\gppoint{gp mark 0}{(6.218,6.427)}
\gppoint{gp mark 0}{(6.218,6.418)}
\gppoint{gp mark 0}{(6.218,5.982)}
\gppoint{gp mark 0}{(6.218,5.881)}
\gppoint{gp mark 0}{(6.218,5.929)}
\gppoint{gp mark 0}{(6.218,5.528)}
\gppoint{gp mark 0}{(6.218,5.993)}
\gppoint{gp mark 0}{(6.218,5.658)}
\gppoint{gp mark 0}{(6.218,5.060)}
\gppoint{gp mark 0}{(6.218,6.393)}
\gppoint{gp mark 0}{(6.218,5.613)}
\gppoint{gp mark 0}{(6.218,5.884)}
\gppoint{gp mark 0}{(6.218,5.547)}
\gppoint{gp mark 0}{(6.218,5.164)}
\gppoint{gp mark 0}{(6.224,6.506)}
\gppoint{gp mark 0}{(6.224,5.817)}
\gppoint{gp mark 0}{(6.224,5.820)}
\gppoint{gp mark 0}{(6.224,6.252)}
\gppoint{gp mark 0}{(6.224,6.430)}
\gppoint{gp mark 0}{(6.224,5.093)}
\gppoint{gp mark 0}{(6.224,6.068)}
\gppoint{gp mark 0}{(6.224,6.137)}
\gppoint{gp mark 0}{(6.224,6.522)}
\gppoint{gp mark 0}{(6.224,5.291)}
\gppoint{gp mark 0}{(6.224,6.420)}
\gppoint{gp mark 0}{(6.224,4.945)}
\gppoint{gp mark 0}{(6.224,6.238)}
\gppoint{gp mark 0}{(6.224,5.857)}
\gppoint{gp mark 0}{(6.224,5.151)}
\gppoint{gp mark 0}{(6.224,7.030)}
\gppoint{gp mark 0}{(6.224,6.500)}
\gppoint{gp mark 0}{(6.224,5.585)}
\gppoint{gp mark 0}{(6.224,5.887)}
\gppoint{gp mark 0}{(6.229,5.160)}
\gppoint{gp mark 0}{(6.229,4.885)}
\gppoint{gp mark 0}{(6.229,5.487)}
\gppoint{gp mark 0}{(6.229,6.507)}
\gppoint{gp mark 0}{(6.229,5.860)}
\gppoint{gp mark 0}{(6.229,6.238)}
\gppoint{gp mark 0}{(6.229,6.360)}
\gppoint{gp mark 0}{(6.229,5.439)}
\gppoint{gp mark 0}{(6.229,5.439)}
\gppoint{gp mark 0}{(6.229,5.637)}
\gppoint{gp mark 0}{(6.229,5.532)}
\gppoint{gp mark 0}{(6.229,6.429)}
\gppoint{gp mark 0}{(6.229,5.104)}
\gppoint{gp mark 0}{(6.229,6.466)}
\gppoint{gp mark 0}{(6.229,4.730)}
\gppoint{gp mark 0}{(6.229,6.591)}
\gppoint{gp mark 0}{(6.229,5.747)}
\gppoint{gp mark 0}{(6.229,5.691)}
\gppoint{gp mark 0}{(6.229,5.708)}
\gppoint{gp mark 0}{(6.229,6.135)}
\gppoint{gp mark 0}{(6.234,6.742)}
\gppoint{gp mark 0}{(6.234,5.903)}
\gppoint{gp mark 0}{(6.234,5.673)}
\gppoint{gp mark 0}{(6.234,5.801)}
\gppoint{gp mark 0}{(6.234,6.573)}
\gppoint{gp mark 0}{(6.234,5.166)}
\gppoint{gp mark 0}{(6.234,6.166)}
\gppoint{gp mark 0}{(6.234,6.465)}
\gppoint{gp mark 0}{(6.234,6.590)}
\gppoint{gp mark 0}{(6.234,5.446)}
\gppoint{gp mark 0}{(6.234,6.187)}
\gppoint{gp mark 0}{(6.234,6.147)}
\gppoint{gp mark 0}{(6.234,5.086)}
\gppoint{gp mark 0}{(6.234,6.671)}
\gppoint{gp mark 0}{(6.234,6.462)}
\gppoint{gp mark 0}{(6.234,6.203)}
\gppoint{gp mark 0}{(6.239,6.861)}
\gppoint{gp mark 0}{(6.239,6.375)}
\gppoint{gp mark 0}{(6.239,6.119)}
\gppoint{gp mark 0}{(6.239,5.184)}
\gppoint{gp mark 0}{(6.239,5.447)}
\gppoint{gp mark 0}{(6.239,6.228)}
\gppoint{gp mark 0}{(6.239,5.177)}
\gppoint{gp mark 0}{(6.239,6.034)}
\gppoint{gp mark 0}{(6.239,5.528)}
\gppoint{gp mark 0}{(6.239,5.881)}
\gppoint{gp mark 0}{(6.239,4.843)}
\gppoint{gp mark 0}{(6.239,6.718)}
\gppoint{gp mark 0}{(6.239,6.389)}
\gppoint{gp mark 0}{(6.239,6.146)}
\gppoint{gp mark 0}{(6.239,5.135)}
\gppoint{gp mark 0}{(6.239,6.884)}
\gppoint{gp mark 0}{(6.239,5.184)}
\gppoint{gp mark 0}{(6.244,6.619)}
\gppoint{gp mark 0}{(6.244,6.634)}
\gppoint{gp mark 0}{(6.244,4.918)}
\gppoint{gp mark 0}{(6.244,5.532)}
\gppoint{gp mark 0}{(6.244,6.479)}
\gppoint{gp mark 0}{(6.244,5.977)}
\gppoint{gp mark 0}{(6.244,6.304)}
\gppoint{gp mark 0}{(6.244,4.766)}
\gppoint{gp mark 0}{(6.244,5.993)}
\gppoint{gp mark 0}{(6.244,5.482)}
\gppoint{gp mark 0}{(6.244,4.756)}
\gppoint{gp mark 0}{(6.249,5.185)}
\gppoint{gp mark 0}{(6.249,6.513)}
\gppoint{gp mark 0}{(6.249,6.018)}
\gppoint{gp mark 0}{(6.249,5.890)}
\gppoint{gp mark 0}{(6.249,5.454)}
\gppoint{gp mark 0}{(6.249,5.835)}
\gppoint{gp mark 0}{(6.249,5.136)}
\gppoint{gp mark 0}{(6.249,5.188)}
\gppoint{gp mark 0}{(6.249,6.421)}
\gppoint{gp mark 0}{(6.249,6.058)}
\gppoint{gp mark 0}{(6.249,5.930)}
\gppoint{gp mark 0}{(6.249,6.519)}
\gppoint{gp mark 0}{(6.249,5.997)}
\gppoint{gp mark 0}{(6.249,5.750)}
\gppoint{gp mark 0}{(6.249,5.903)}
\gppoint{gp mark 0}{(6.249,5.900)}
\gppoint{gp mark 0}{(6.254,6.331)}
\gppoint{gp mark 0}{(6.254,5.675)}
\gppoint{gp mark 0}{(6.254,6.426)}
\gppoint{gp mark 0}{(6.254,6.194)}
\gppoint{gp mark 0}{(6.254,5.432)}
\gppoint{gp mark 0}{(6.254,5.773)}
\gppoint{gp mark 0}{(6.254,6.221)}
\gppoint{gp mark 0}{(6.254,6.532)}
\gppoint{gp mark 0}{(6.254,5.111)}
\gppoint{gp mark 0}{(6.254,6.558)}
\gppoint{gp mark 0}{(6.254,6.476)}
\gppoint{gp mark 0}{(6.254,6.476)}
\gppoint{gp mark 0}{(6.254,6.411)}
\gppoint{gp mark 0}{(6.254,5.599)}
\gppoint{gp mark 0}{(6.254,5.625)}
\gppoint{gp mark 0}{(6.254,6.476)}
\gppoint{gp mark 0}{(6.254,6.526)}
\gppoint{gp mark 0}{(6.254,6.462)}
\gppoint{gp mark 0}{(6.254,6.089)}
\gppoint{gp mark 0}{(6.254,6.069)}
\gppoint{gp mark 0}{(6.259,5.865)}
\gppoint{gp mark 0}{(6.259,6.351)}
\gppoint{gp mark 0}{(6.259,5.728)}
\gppoint{gp mark 0}{(6.259,6.164)}
\gppoint{gp mark 0}{(6.259,5.196)}
\gppoint{gp mark 0}{(6.259,6.521)}
\gppoint{gp mark 0}{(6.259,6.327)}
\gppoint{gp mark 0}{(6.259,6.273)}
\gppoint{gp mark 0}{(6.259,5.754)}
\gppoint{gp mark 0}{(6.259,6.185)}
\gppoint{gp mark 0}{(6.259,4.832)}
\gppoint{gp mark 0}{(6.259,6.164)}
\gppoint{gp mark 0}{(6.259,6.185)}
\gppoint{gp mark 0}{(6.259,6.198)}
\gppoint{gp mark 0}{(6.259,4.762)}
\gppoint{gp mark 0}{(6.259,6.234)}
\gppoint{gp mark 0}{(6.259,6.590)}
\gppoint{gp mark 0}{(6.259,6.408)}
\gppoint{gp mark 0}{(6.259,5.155)}
\gppoint{gp mark 0}{(6.259,5.615)}
\gppoint{gp mark 0}{(6.264,6.313)}
\gppoint{gp mark 0}{(6.264,6.158)}
\gppoint{gp mark 0}{(6.264,6.425)}
\gppoint{gp mark 0}{(6.264,5.783)}
\gppoint{gp mark 0}{(6.264,5.770)}
\gppoint{gp mark 0}{(6.264,5.205)}
\gppoint{gp mark 0}{(6.264,6.512)}
\gppoint{gp mark 0}{(6.264,5.939)}
\gppoint{gp mark 0}{(6.264,6.133)}
\gppoint{gp mark 0}{(6.264,6.224)}
\gppoint{gp mark 0}{(6.264,6.542)}
\gppoint{gp mark 0}{(6.264,5.596)}
\gppoint{gp mark 0}{(6.264,6.002)}
\gppoint{gp mark 0}{(6.264,5.993)}
\gppoint{gp mark 0}{(6.264,5.945)}
\gppoint{gp mark 0}{(6.264,5.457)}
\gppoint{gp mark 0}{(6.264,6.022)}
\gppoint{gp mark 0}{(6.264,5.436)}
\gppoint{gp mark 0}{(6.264,6.240)}
\gppoint{gp mark 0}{(6.264,6.605)}
\gppoint{gp mark 0}{(6.264,5.418)}
\gppoint{gp mark 0}{(6.264,6.240)}
\gppoint{gp mark 0}{(6.264,5.200)}
\gppoint{gp mark 0}{(6.269,5.883)}
\gppoint{gp mark 0}{(6.269,6.522)}
\gppoint{gp mark 0}{(6.269,6.136)}
\gppoint{gp mark 0}{(6.269,6.038)}
\gppoint{gp mark 0}{(6.269,6.215)}
\gppoint{gp mark 0}{(6.269,5.887)}
\gppoint{gp mark 0}{(6.269,5.194)}
\gppoint{gp mark 0}{(6.269,5.994)}
\gppoint{gp mark 0}{(6.269,6.361)}
\gppoint{gp mark 0}{(6.269,6.090)}
\gppoint{gp mark 0}{(6.269,6.261)}
\gppoint{gp mark 0}{(6.269,6.434)}
\gppoint{gp mark 0}{(6.269,6.115)}
\gppoint{gp mark 0}{(6.269,4.542)}
\gppoint{gp mark 0}{(6.269,6.428)}
\gppoint{gp mark 0}{(6.269,4.843)}
\gppoint{gp mark 0}{(6.269,5.212)}
\gppoint{gp mark 0}{(6.269,5.920)}
\gppoint{gp mark 0}{(6.269,5.942)}
\gppoint{gp mark 0}{(6.269,6.119)}
\gppoint{gp mark 0}{(6.269,6.227)}
\gppoint{gp mark 0}{(6.274,5.181)}
\gppoint{gp mark 0}{(6.274,6.602)}
\gppoint{gp mark 0}{(6.274,6.695)}
\gppoint{gp mark 0}{(6.274,6.483)}
\gppoint{gp mark 0}{(6.274,6.533)}
\gppoint{gp mark 0}{(6.274,6.265)}
\gppoint{gp mark 0}{(6.274,6.488)}
\gppoint{gp mark 0}{(6.274,6.239)}
\gppoint{gp mark 0}{(6.274,6.004)}
\gppoint{gp mark 0}{(6.274,5.713)}
\gppoint{gp mark 0}{(6.274,5.043)}
\gppoint{gp mark 0}{(6.274,6.876)}
\gppoint{gp mark 0}{(6.274,5.525)}
\gppoint{gp mark 0}{(6.274,5.831)}
\gppoint{gp mark 0}{(6.274,6.160)}
\gppoint{gp mark 0}{(6.274,6.147)}
\gppoint{gp mark 0}{(6.279,6.070)}
\gppoint{gp mark 0}{(6.279,6.521)}
\gppoint{gp mark 0}{(6.279,5.208)}
\gppoint{gp mark 0}{(6.279,6.109)}
\gppoint{gp mark 0}{(6.279,4.968)}
\gppoint{gp mark 0}{(6.279,6.483)}
\gppoint{gp mark 0}{(6.279,6.259)}
\gppoint{gp mark 0}{(6.279,6.107)}
\gppoint{gp mark 0}{(6.279,6.340)}
\gppoint{gp mark 0}{(6.279,5.946)}
\gppoint{gp mark 0}{(6.279,6.231)}
\gppoint{gp mark 0}{(6.279,5.952)}
\gppoint{gp mark 0}{(6.279,6.298)}
\gppoint{gp mark 0}{(6.279,4.796)}
\gppoint{gp mark 0}{(6.279,6.007)}
\gppoint{gp mark 0}{(6.283,6.213)}
\gppoint{gp mark 0}{(6.283,5.383)}
\gppoint{gp mark 0}{(6.283,5.770)}
\gppoint{gp mark 0}{(6.283,6.110)}
\gppoint{gp mark 0}{(6.283,5.769)}
\gppoint{gp mark 0}{(6.283,6.049)}
\gppoint{gp mark 0}{(6.283,5.439)}
\gppoint{gp mark 0}{(6.283,6.210)}
\gppoint{gp mark 0}{(6.283,4.843)}
\gppoint{gp mark 0}{(6.283,6.400)}
\gppoint{gp mark 0}{(6.283,6.779)}
\gppoint{gp mark 0}{(6.283,5.582)}
\gppoint{gp mark 0}{(6.283,6.326)}
\gppoint{gp mark 0}{(6.283,6.494)}
\gppoint{gp mark 0}{(6.283,5.802)}
\gppoint{gp mark 0}{(6.283,6.326)}
\gppoint{gp mark 0}{(6.288,6.154)}
\gppoint{gp mark 0}{(6.288,6.492)}
\gppoint{gp mark 0}{(6.288,5.665)}
\gppoint{gp mark 0}{(6.288,6.551)}
\gppoint{gp mark 0}{(6.288,6.990)}
\gppoint{gp mark 0}{(6.288,6.364)}
\gppoint{gp mark 0}{(6.288,6.354)}
\gppoint{gp mark 0}{(6.288,5.863)}
\gppoint{gp mark 0}{(6.288,6.374)}
\gppoint{gp mark 0}{(6.288,6.447)}
\gppoint{gp mark 0}{(6.288,6.404)}
\gppoint{gp mark 0}{(6.288,6.919)}
\gppoint{gp mark 0}{(6.288,6.295)}
\gppoint{gp mark 0}{(6.288,6.283)}
\gppoint{gp mark 0}{(6.288,5.818)}
\gppoint{gp mark 0}{(6.288,5.740)}
\gppoint{gp mark 0}{(6.293,7.003)}
\gppoint{gp mark 0}{(6.293,6.500)}
\gppoint{gp mark 0}{(6.293,6.312)}
\gppoint{gp mark 0}{(6.293,6.107)}
\gppoint{gp mark 0}{(6.293,5.680)}
\gppoint{gp mark 0}{(6.293,6.694)}
\gppoint{gp mark 0}{(6.293,6.547)}
\gppoint{gp mark 0}{(6.293,6.075)}
\gppoint{gp mark 0}{(6.298,6.274)}
\gppoint{gp mark 0}{(6.298,6.024)}
\gppoint{gp mark 0}{(6.298,5.414)}
\gppoint{gp mark 0}{(6.298,6.631)}
\gppoint{gp mark 0}{(6.298,5.789)}
\gppoint{gp mark 0}{(6.298,6.153)}
\gppoint{gp mark 0}{(6.298,6.843)}
\gppoint{gp mark 0}{(6.303,5.677)}
\gppoint{gp mark 0}{(6.303,5.513)}
\gppoint{gp mark 0}{(6.303,5.582)}
\gppoint{gp mark 0}{(6.303,5.502)}
\gppoint{gp mark 0}{(6.303,6.268)}
\gppoint{gp mark 0}{(6.303,5.243)}
\gppoint{gp mark 0}{(6.303,5.223)}
\gppoint{gp mark 0}{(6.303,6.134)}
\gppoint{gp mark 0}{(6.303,5.654)}
\gppoint{gp mark 0}{(6.303,5.213)}
\gppoint{gp mark 0}{(6.303,6.634)}
\gppoint{gp mark 0}{(6.303,5.646)}
\gppoint{gp mark 0}{(6.303,4.944)}
\gppoint{gp mark 0}{(6.303,5.502)}
\gppoint{gp mark 0}{(6.303,5.677)}
\gppoint{gp mark 0}{(6.303,6.175)}
\gppoint{gp mark 0}{(6.303,5.083)}
\gppoint{gp mark 0}{(6.303,5.502)}
\gppoint{gp mark 0}{(6.303,6.189)}
\gppoint{gp mark 0}{(6.307,6.025)}
\gppoint{gp mark 0}{(6.307,6.140)}
\gppoint{gp mark 0}{(6.307,6.256)}
\gppoint{gp mark 0}{(6.307,5.945)}
\gppoint{gp mark 0}{(6.307,6.256)}
\gppoint{gp mark 0}{(6.307,6.999)}
\gppoint{gp mark 0}{(6.307,6.423)}
\gppoint{gp mark 0}{(6.307,4.586)}
\gppoint{gp mark 0}{(6.307,6.225)}
\gppoint{gp mark 0}{(6.307,5.655)}
\gppoint{gp mark 0}{(6.307,5.830)}
\gppoint{gp mark 0}{(6.307,6.009)}
\gppoint{gp mark 0}{(6.307,6.099)}
\gppoint{gp mark 0}{(6.307,6.177)}
\gppoint{gp mark 0}{(6.307,5.521)}
\gppoint{gp mark 0}{(6.307,6.786)}
\gppoint{gp mark 0}{(6.312,6.660)}
\gppoint{gp mark 0}{(6.312,6.692)}
\gppoint{gp mark 0}{(6.312,5.179)}
\gppoint{gp mark 0}{(6.312,6.443)}
\gppoint{gp mark 0}{(6.312,5.672)}
\gppoint{gp mark 0}{(6.312,6.321)}
\gppoint{gp mark 0}{(6.312,6.198)}
\gppoint{gp mark 0}{(6.312,5.834)}
\gppoint{gp mark 0}{(6.312,6.694)}
\gppoint{gp mark 0}{(6.312,6.424)}
\gppoint{gp mark 0}{(6.312,6.169)}
\gppoint{gp mark 0}{(6.312,5.789)}
\gppoint{gp mark 0}{(6.312,5.254)}
\gppoint{gp mark 0}{(6.312,6.110)}
\gppoint{gp mark 0}{(6.312,6.177)}
\gppoint{gp mark 0}{(6.312,6.685)}
\gppoint{gp mark 0}{(6.317,6.884)}
\gppoint{gp mark 0}{(6.317,5.474)}
\gppoint{gp mark 0}{(6.317,4.859)}
\gppoint{gp mark 0}{(6.317,6.039)}
\gppoint{gp mark 0}{(6.317,6.873)}
\gppoint{gp mark 0}{(6.317,6.033)}
\gppoint{gp mark 0}{(6.317,5.822)}
\gppoint{gp mark 0}{(6.317,6.298)}
\gppoint{gp mark 0}{(6.317,6.847)}
\gppoint{gp mark 0}{(6.317,5.905)}
\gppoint{gp mark 0}{(6.317,6.523)}
\gppoint{gp mark 0}{(6.317,5.509)}
\gppoint{gp mark 0}{(6.321,6.059)}
\gppoint{gp mark 0}{(6.321,6.211)}
\gppoint{gp mark 0}{(6.321,6.426)}
\gppoint{gp mark 0}{(6.321,6.426)}
\gppoint{gp mark 0}{(6.321,5.747)}
\gppoint{gp mark 0}{(6.321,6.068)}
\gppoint{gp mark 0}{(6.321,6.426)}
\gppoint{gp mark 0}{(6.321,6.556)}
\gppoint{gp mark 0}{(6.321,5.853)}
\gppoint{gp mark 0}{(6.321,6.426)}
\gppoint{gp mark 0}{(6.321,6.207)}
\gppoint{gp mark 0}{(6.321,6.133)}
\gppoint{gp mark 0}{(6.321,6.426)}
\gppoint{gp mark 0}{(6.321,5.736)}
\gppoint{gp mark 0}{(6.321,5.833)}
\gppoint{gp mark 0}{(6.326,6.041)}
\gppoint{gp mark 0}{(6.326,6.624)}
\gppoint{gp mark 0}{(6.326,6.629)}
\gppoint{gp mark 0}{(6.326,6.176)}
\gppoint{gp mark 0}{(6.326,6.162)}
\gppoint{gp mark 0}{(6.326,6.198)}
\gppoint{gp mark 0}{(6.326,6.321)}
\gppoint{gp mark 0}{(6.326,6.428)}
\gppoint{gp mark 0}{(6.326,6.337)}
\gppoint{gp mark 0}{(6.326,5.631)}
\gppoint{gp mark 0}{(6.326,5.676)}
\gppoint{gp mark 0}{(6.326,6.783)}
\gppoint{gp mark 0}{(6.326,6.219)}
\gppoint{gp mark 0}{(6.326,5.067)}
\gppoint{gp mark 0}{(6.326,4.798)}
\gppoint{gp mark 0}{(6.326,6.353)}
\gppoint{gp mark 0}{(6.326,7.034)}
\gppoint{gp mark 0}{(6.326,6.052)}
\gppoint{gp mark 0}{(6.326,6.570)}
\gppoint{gp mark 0}{(6.326,6.742)}
\gppoint{gp mark 0}{(6.331,6.325)}
\gppoint{gp mark 0}{(6.331,6.217)}
\gppoint{gp mark 0}{(6.331,5.579)}
\gppoint{gp mark 0}{(6.331,6.198)}
\gppoint{gp mark 0}{(6.331,6.136)}
\gppoint{gp mark 0}{(6.331,5.626)}
\gppoint{gp mark 0}{(6.331,5.172)}
\gppoint{gp mark 0}{(6.331,6.071)}
\gppoint{gp mark 0}{(6.331,5.848)}
\gppoint{gp mark 0}{(6.331,5.881)}
\gppoint{gp mark 0}{(6.331,6.076)}
\gppoint{gp mark 0}{(6.331,5.982)}
\gppoint{gp mark 0}{(6.335,4.895)}
\gppoint{gp mark 0}{(6.335,6.811)}
\gppoint{gp mark 0}{(6.335,6.514)}
\gppoint{gp mark 0}{(6.335,6.227)}
\gppoint{gp mark 0}{(6.335,5.885)}
\gppoint{gp mark 0}{(6.335,5.318)}
\gppoint{gp mark 0}{(6.335,5.826)}
\gppoint{gp mark 0}{(6.335,6.152)}
\gppoint{gp mark 0}{(6.335,6.286)}
\gppoint{gp mark 0}{(6.335,6.234)}
\gppoint{gp mark 0}{(6.335,6.168)}
\gppoint{gp mark 0}{(6.335,6.456)}
\gppoint{gp mark 0}{(6.335,6.318)}
\gppoint{gp mark 0}{(6.335,6.145)}
\gppoint{gp mark 0}{(6.340,5.900)}
\gppoint{gp mark 0}{(6.340,6.513)}
\gppoint{gp mark 0}{(6.340,6.487)}
\gppoint{gp mark 0}{(6.340,6.619)}
\gppoint{gp mark 0}{(6.340,6.619)}
\gppoint{gp mark 0}{(6.340,4.875)}
\gppoint{gp mark 0}{(6.340,6.661)}
\gppoint{gp mark 0}{(6.340,5.758)}
\gppoint{gp mark 0}{(6.340,6.168)}
\gppoint{gp mark 0}{(6.340,6.382)}
\gppoint{gp mark 0}{(6.340,5.383)}
\gppoint{gp mark 0}{(6.340,5.875)}
\gppoint{gp mark 0}{(6.340,6.491)}
\gppoint{gp mark 0}{(6.344,6.163)}
\gppoint{gp mark 0}{(6.344,6.515)}
\gppoint{gp mark 0}{(6.344,6.407)}
\gppoint{gp mark 0}{(6.344,5.935)}
\gppoint{gp mark 0}{(6.344,6.306)}
\gppoint{gp mark 0}{(6.344,6.609)}
\gppoint{gp mark 0}{(6.344,5.881)}
\gppoint{gp mark 0}{(6.344,6.309)}
\gppoint{gp mark 0}{(6.344,6.579)}
\gppoint{gp mark 0}{(6.344,5.880)}
\gppoint{gp mark 0}{(6.344,6.431)}
\gppoint{gp mark 0}{(6.344,5.916)}
\gppoint{gp mark 0}{(6.344,6.452)}
\gppoint{gp mark 0}{(6.349,5.800)}
\gppoint{gp mark 0}{(6.349,6.470)}
\gppoint{gp mark 0}{(6.349,6.450)}
\gppoint{gp mark 0}{(6.349,5.136)}
\gppoint{gp mark 0}{(6.349,6.591)}
\gppoint{gp mark 0}{(6.349,6.470)}
\gppoint{gp mark 0}{(6.349,6.381)}
\gppoint{gp mark 0}{(6.349,5.687)}
\gppoint{gp mark 0}{(6.349,5.544)}
\gppoint{gp mark 0}{(6.349,5.391)}
\gppoint{gp mark 0}{(6.349,6.235)}
\gppoint{gp mark 0}{(6.349,5.732)}
\gppoint{gp mark 0}{(6.349,5.158)}
\gppoint{gp mark 0}{(6.349,5.867)}
\gppoint{gp mark 0}{(6.349,5.867)}
\gppoint{gp mark 0}{(6.349,6.521)}
\gppoint{gp mark 0}{(6.353,5.219)}
\gppoint{gp mark 0}{(6.353,6.377)}
\gppoint{gp mark 0}{(6.353,6.576)}
\gppoint{gp mark 0}{(6.353,5.865)}
\gppoint{gp mark 0}{(6.353,6.026)}
\gppoint{gp mark 0}{(6.353,5.157)}
\gppoint{gp mark 0}{(6.353,6.475)}
\gppoint{gp mark 0}{(6.353,5.898)}
\gppoint{gp mark 0}{(6.353,6.434)}
\gppoint{gp mark 0}{(6.353,5.917)}
\gppoint{gp mark 0}{(6.353,5.021)}
\gppoint{gp mark 0}{(6.353,6.299)}
\gppoint{gp mark 0}{(6.353,5.812)}
\gppoint{gp mark 0}{(6.353,6.480)}
\gppoint{gp mark 0}{(6.353,6.433)}
\gppoint{gp mark 0}{(6.358,6.569)}
\gppoint{gp mark 0}{(6.358,6.130)}
\gppoint{gp mark 0}{(6.358,6.832)}
\gppoint{gp mark 0}{(6.358,6.610)}
\gppoint{gp mark 0}{(6.358,6.512)}
\gppoint{gp mark 0}{(6.358,5.940)}
\gppoint{gp mark 0}{(6.358,6.061)}
\gppoint{gp mark 0}{(6.358,6.045)}
\gppoint{gp mark 0}{(6.358,5.765)}
\gppoint{gp mark 0}{(6.358,6.338)}
\gppoint{gp mark 0}{(6.358,5.775)}
\gppoint{gp mark 0}{(6.358,6.437)}
\gppoint{gp mark 0}{(6.358,5.837)}
\gppoint{gp mark 0}{(6.358,6.595)}
\gppoint{gp mark 0}{(6.358,6.731)}
\gppoint{gp mark 0}{(6.358,5.836)}
\gppoint{gp mark 0}{(6.358,7.067)}
\gppoint{gp mark 0}{(6.362,6.257)}
\gppoint{gp mark 0}{(6.362,6.257)}
\gppoint{gp mark 0}{(6.362,6.548)}
\gppoint{gp mark 0}{(6.362,6.441)}
\gppoint{gp mark 0}{(6.362,5.859)}
\gppoint{gp mark 0}{(6.362,6.146)}
\gppoint{gp mark 0}{(6.362,6.694)}
\gppoint{gp mark 0}{(6.362,5.237)}
\gppoint{gp mark 0}{(6.362,6.354)}
\gppoint{gp mark 0}{(6.362,6.516)}
\gppoint{gp mark 0}{(6.367,5.911)}
\gppoint{gp mark 0}{(6.367,5.658)}
\gppoint{gp mark 0}{(6.367,6.413)}
\gppoint{gp mark 0}{(6.367,5.698)}
\gppoint{gp mark 0}{(6.367,6.908)}
\gppoint{gp mark 0}{(6.367,6.679)}
\gppoint{gp mark 0}{(6.367,6.004)}
\gppoint{gp mark 0}{(6.367,5.072)}
\gppoint{gp mark 0}{(6.367,6.270)}
\gppoint{gp mark 0}{(6.367,6.413)}
\gppoint{gp mark 0}{(6.367,6.692)}
\gppoint{gp mark 0}{(6.371,6.690)}
\gppoint{gp mark 0}{(6.371,5.494)}
\gppoint{gp mark 0}{(6.371,6.750)}
\gppoint{gp mark 0}{(6.371,6.549)}
\gppoint{gp mark 0}{(6.371,6.854)}
\gppoint{gp mark 0}{(6.371,5.150)}
\gppoint{gp mark 0}{(6.371,6.668)}
\gppoint{gp mark 0}{(6.371,6.107)}
\gppoint{gp mark 0}{(6.375,6.597)}
\gppoint{gp mark 0}{(6.375,5.977)}
\gppoint{gp mark 0}{(6.375,5.654)}
\gppoint{gp mark 0}{(6.375,5.488)}
\gppoint{gp mark 0}{(6.375,6.141)}
\gppoint{gp mark 0}{(6.375,6.416)}
\gppoint{gp mark 0}{(6.375,6.358)}
\gppoint{gp mark 0}{(6.375,4.942)}
\gppoint{gp mark 0}{(6.375,6.393)}
\gppoint{gp mark 0}{(6.375,6.746)}
\gppoint{gp mark 0}{(6.375,6.191)}
\gppoint{gp mark 0}{(6.375,5.917)}
\gppoint{gp mark 0}{(6.375,6.176)}
\gppoint{gp mark 0}{(6.375,4.913)}
\gppoint{gp mark 0}{(6.375,6.393)}
\gppoint{gp mark 0}{(6.375,6.521)}
\gppoint{gp mark 0}{(6.380,5.681)}
\gppoint{gp mark 0}{(6.380,6.510)}
\gppoint{gp mark 0}{(6.380,6.459)}
\gppoint{gp mark 0}{(6.380,5.856)}
\gppoint{gp mark 0}{(6.380,5.717)}
\gppoint{gp mark 0}{(6.380,6.372)}
\gppoint{gp mark 0}{(6.380,6.586)}
\gppoint{gp mark 0}{(6.380,6.476)}
\gppoint{gp mark 0}{(6.380,6.464)}
\gppoint{gp mark 0}{(6.380,6.744)}
\gppoint{gp mark 0}{(6.380,6.602)}
\gppoint{gp mark 0}{(6.380,6.443)}
\gppoint{gp mark 0}{(6.380,5.988)}
\gppoint{gp mark 0}{(6.384,5.894)}
\gppoint{gp mark 0}{(6.384,5.694)}
\gppoint{gp mark 0}{(6.384,7.463)}
\gppoint{gp mark 0}{(6.384,5.697)}
\gppoint{gp mark 0}{(6.384,5.091)}
\gppoint{gp mark 0}{(6.388,6.287)}
\gppoint{gp mark 0}{(6.388,5.836)}
\gppoint{gp mark 0}{(6.388,6.280)}
\gppoint{gp mark 0}{(6.388,6.084)}
\gppoint{gp mark 0}{(6.388,5.802)}
\gppoint{gp mark 0}{(6.388,6.287)}
\gppoint{gp mark 0}{(6.388,6.287)}
\gppoint{gp mark 0}{(6.388,5.716)}
\gppoint{gp mark 0}{(6.388,6.155)}
\gppoint{gp mark 0}{(6.388,6.099)}
\gppoint{gp mark 0}{(6.393,7.196)}
\gppoint{gp mark 0}{(6.393,6.685)}
\gppoint{gp mark 0}{(6.393,6.074)}
\gppoint{gp mark 0}{(6.393,6.018)}
\gppoint{gp mark 0}{(6.393,5.383)}
\gppoint{gp mark 0}{(6.393,5.152)}
\gppoint{gp mark 0}{(6.393,5.751)}
\gppoint{gp mark 0}{(6.393,6.152)}
\gppoint{gp mark 0}{(6.393,6.000)}
\gppoint{gp mark 0}{(6.393,6.286)}
\gppoint{gp mark 0}{(6.393,6.325)}
\gppoint{gp mark 0}{(6.393,6.315)}
\gppoint{gp mark 0}{(6.393,5.526)}
\gppoint{gp mark 0}{(6.397,6.548)}
\gppoint{gp mark 0}{(6.397,5.630)}
\gppoint{gp mark 0}{(6.397,6.734)}
\gppoint{gp mark 0}{(6.397,6.734)}
\gppoint{gp mark 0}{(6.397,5.292)}
\gppoint{gp mark 0}{(6.397,5.500)}
\gppoint{gp mark 0}{(6.397,4.921)}
\gppoint{gp mark 0}{(6.397,6.255)}
\gppoint{gp mark 0}{(6.397,4.921)}
\gppoint{gp mark 0}{(6.397,6.389)}
\gppoint{gp mark 0}{(6.397,6.333)}
\gppoint{gp mark 0}{(6.397,5.002)}
\gppoint{gp mark 0}{(6.397,5.564)}
\gppoint{gp mark 0}{(6.397,6.125)}
\gppoint{gp mark 0}{(6.397,5.997)}
\gppoint{gp mark 0}{(6.401,4.993)}
\gppoint{gp mark 0}{(6.401,6.698)}
\gppoint{gp mark 0}{(6.401,6.388)}
\gppoint{gp mark 0}{(6.401,4.997)}
\gppoint{gp mark 0}{(6.401,6.421)}
\gppoint{gp mark 0}{(6.401,6.692)}
\gppoint{gp mark 0}{(6.401,6.702)}
\gppoint{gp mark 0}{(6.401,6.692)}
\gppoint{gp mark 0}{(6.401,7.056)}
\gppoint{gp mark 0}{(6.401,6.421)}
\gppoint{gp mark 0}{(6.401,6.220)}
\gppoint{gp mark 0}{(6.401,6.388)}
\gppoint{gp mark 0}{(6.401,5.817)}
\gppoint{gp mark 0}{(6.401,5.811)}
\gppoint{gp mark 0}{(6.401,6.388)}
\gppoint{gp mark 0}{(6.401,6.204)}
\gppoint{gp mark 0}{(6.401,6.388)}
\gppoint{gp mark 0}{(6.401,6.388)}
\gppoint{gp mark 0}{(6.406,6.707)}
\gppoint{gp mark 0}{(6.406,6.379)}
\gppoint{gp mark 0}{(6.406,6.549)}
\gppoint{gp mark 0}{(6.406,4.950)}
\gppoint{gp mark 0}{(6.406,7.052)}
\gppoint{gp mark 0}{(6.406,6.434)}
\gppoint{gp mark 0}{(6.406,6.707)}
\gppoint{gp mark 0}{(6.406,6.263)}
\gppoint{gp mark 0}{(6.406,6.382)}
\gppoint{gp mark 0}{(6.406,6.420)}
\gppoint{gp mark 0}{(6.406,6.263)}
\gppoint{gp mark 0}{(6.406,6.379)}
\gppoint{gp mark 0}{(6.406,6.865)}
\gppoint{gp mark 0}{(6.410,4.959)}
\gppoint{gp mark 0}{(6.410,6.993)}
\gppoint{gp mark 0}{(6.410,5.310)}
\gppoint{gp mark 0}{(6.410,6.435)}
\gppoint{gp mark 0}{(6.410,5.662)}
\gppoint{gp mark 0}{(6.410,6.251)}
\gppoint{gp mark 0}{(6.410,6.306)}
\gppoint{gp mark 0}{(6.410,5.462)}
\gppoint{gp mark 0}{(6.410,5.890)}
\gppoint{gp mark 0}{(6.410,6.321)}
\gppoint{gp mark 0}{(6.410,6.896)}
\gppoint{gp mark 0}{(6.410,6.014)}
\gppoint{gp mark 0}{(6.414,6.838)}
\gppoint{gp mark 0}{(6.414,5.654)}
\gppoint{gp mark 0}{(6.414,6.228)}
\gppoint{gp mark 0}{(6.414,5.799)}
\gppoint{gp mark 0}{(6.414,4.921)}
\gppoint{gp mark 0}{(6.414,6.617)}
\gppoint{gp mark 0}{(6.414,6.763)}
\gppoint{gp mark 0}{(6.414,6.731)}
\gppoint{gp mark 0}{(6.414,6.228)}
\gppoint{gp mark 0}{(6.414,5.819)}
\gppoint{gp mark 0}{(6.414,5.468)}
\gppoint{gp mark 0}{(6.414,6.344)}
\gppoint{gp mark 0}{(6.418,6.014)}
\gppoint{gp mark 0}{(6.418,5.838)}
\gppoint{gp mark 0}{(6.418,5.661)}
\gppoint{gp mark 0}{(6.422,6.747)}
\gppoint{gp mark 0}{(6.422,6.550)}
\gppoint{gp mark 0}{(6.422,5.956)}
\gppoint{gp mark 0}{(6.422,6.629)}
\gppoint{gp mark 0}{(6.422,5.716)}
\gppoint{gp mark 0}{(6.422,6.422)}
\gppoint{gp mark 0}{(6.422,6.472)}
\gppoint{gp mark 0}{(6.426,5.634)}
\gppoint{gp mark 0}{(6.426,6.413)}
\gppoint{gp mark 0}{(6.426,6.043)}
\gppoint{gp mark 0}{(6.426,6.173)}
\gppoint{gp mark 0}{(6.426,5.883)}
\gppoint{gp mark 0}{(6.426,5.814)}
\gppoint{gp mark 0}{(6.426,5.956)}
\gppoint{gp mark 0}{(6.426,6.575)}
\gppoint{gp mark 0}{(6.426,5.117)}
\gppoint{gp mark 0}{(6.426,6.658)}
\gppoint{gp mark 0}{(6.431,6.258)}
\gppoint{gp mark 0}{(6.431,5.939)}
\gppoint{gp mark 0}{(6.431,6.332)}
\gppoint{gp mark 0}{(6.431,6.602)}
\gppoint{gp mark 0}{(6.431,6.422)}
\gppoint{gp mark 0}{(6.431,6.474)}
\gppoint{gp mark 0}{(6.431,5.790)}
\gppoint{gp mark 0}{(6.435,5.651)}
\gppoint{gp mark 0}{(6.435,5.749)}
\gppoint{gp mark 0}{(6.435,6.027)}
\gppoint{gp mark 0}{(6.435,6.308)}
\gppoint{gp mark 0}{(6.435,5.372)}
\gppoint{gp mark 0}{(6.435,5.929)}
\gppoint{gp mark 0}{(6.435,6.221)}
\gppoint{gp mark 0}{(6.435,5.912)}
\gppoint{gp mark 0}{(6.435,5.758)}
\gppoint{gp mark 0}{(6.435,6.183)}
\gppoint{gp mark 0}{(6.435,5.388)}
\gppoint{gp mark 0}{(6.435,5.628)}
\gppoint{gp mark 0}{(6.439,6.537)}
\gppoint{gp mark 0}{(6.439,6.538)}
\gppoint{gp mark 0}{(6.439,6.843)}
\gppoint{gp mark 0}{(6.439,6.851)}
\gppoint{gp mark 0}{(6.439,6.273)}
\gppoint{gp mark 0}{(6.439,6.147)}
\gppoint{gp mark 0}{(6.439,5.812)}
\gppoint{gp mark 0}{(6.443,5.799)}
\gppoint{gp mark 0}{(6.443,6.633)}
\gppoint{gp mark 0}{(6.443,6.898)}
\gppoint{gp mark 0}{(6.443,6.230)}
\gppoint{gp mark 0}{(6.443,6.601)}
\gppoint{gp mark 0}{(6.447,6.409)}
\gppoint{gp mark 0}{(6.447,6.579)}
\gppoint{gp mark 0}{(6.447,5.010)}
\gppoint{gp mark 0}{(6.447,5.843)}
\gppoint{gp mark 0}{(6.447,6.419)}
\gppoint{gp mark 0}{(6.447,5.272)}
\gppoint{gp mark 0}{(6.447,6.665)}
\gppoint{gp mark 0}{(6.447,5.412)}
\gppoint{gp mark 0}{(6.447,6.399)}
\gppoint{gp mark 0}{(6.451,7.207)}
\gppoint{gp mark 0}{(6.451,6.164)}
\gppoint{gp mark 0}{(6.451,6.576)}
\gppoint{gp mark 0}{(6.451,5.413)}
\gppoint{gp mark 0}{(6.451,6.014)}
\gppoint{gp mark 0}{(6.451,6.393)}
\gppoint{gp mark 0}{(6.451,6.342)}
\gppoint{gp mark 0}{(6.455,5.524)}
\gppoint{gp mark 0}{(6.455,5.513)}
\gppoint{gp mark 0}{(6.455,5.913)}
\gppoint{gp mark 0}{(6.455,5.413)}
\gppoint{gp mark 0}{(6.455,6.489)}
\gppoint{gp mark 0}{(6.455,4.828)}
\gppoint{gp mark 0}{(6.455,6.017)}
\gppoint{gp mark 0}{(6.455,6.175)}
\gppoint{gp mark 0}{(6.455,6.181)}
\gppoint{gp mark 0}{(6.455,6.319)}
\gppoint{gp mark 0}{(6.459,5.565)}
\gppoint{gp mark 0}{(6.459,5.890)}
\gppoint{gp mark 0}{(6.459,5.750)}
\gppoint{gp mark 0}{(6.459,5.823)}
\gppoint{gp mark 0}{(6.463,6.535)}
\gppoint{gp mark 0}{(6.463,6.432)}
\gppoint{gp mark 0}{(6.463,6.090)}
\gppoint{gp mark 0}{(6.463,7.083)}
\gppoint{gp mark 0}{(6.463,6.783)}
\gppoint{gp mark 0}{(6.463,6.099)}
\gppoint{gp mark 0}{(6.463,6.099)}
\gppoint{gp mark 0}{(6.463,6.229)}
\gppoint{gp mark 0}{(6.463,6.264)}
\gppoint{gp mark 0}{(6.463,6.028)}
\gppoint{gp mark 0}{(6.467,5.585)}
\gppoint{gp mark 0}{(6.467,6.396)}
\gppoint{gp mark 0}{(6.467,6.088)}
\gppoint{gp mark 0}{(6.467,6.139)}
\gppoint{gp mark 0}{(6.467,6.092)}
\gppoint{gp mark 0}{(6.467,6.921)}
\gppoint{gp mark 0}{(6.467,6.783)}
\gppoint{gp mark 0}{(6.467,6.015)}
\gppoint{gp mark 0}{(6.467,6.818)}
\gppoint{gp mark 0}{(6.467,6.843)}
\gppoint{gp mark 0}{(6.471,5.985)}
\gppoint{gp mark 0}{(6.471,6.810)}
\gppoint{gp mark 0}{(6.471,6.510)}
\gppoint{gp mark 0}{(6.471,6.741)}
\gppoint{gp mark 0}{(6.471,6.538)}
\gppoint{gp mark 0}{(6.471,6.762)}
\gppoint{gp mark 0}{(6.475,6.462)}
\gppoint{gp mark 0}{(6.475,6.356)}
\gppoint{gp mark 0}{(6.475,6.769)}
\gppoint{gp mark 0}{(6.475,6.580)}
\gppoint{gp mark 0}{(6.475,5.826)}
\gppoint{gp mark 0}{(6.475,6.161)}
\gppoint{gp mark 0}{(6.475,5.588)}
\gppoint{gp mark 0}{(6.475,5.918)}
\gppoint{gp mark 0}{(6.479,6.267)}
\gppoint{gp mark 0}{(6.479,5.776)}
\gppoint{gp mark 0}{(6.479,5.700)}
\gppoint{gp mark 0}{(6.479,6.562)}
\gppoint{gp mark 0}{(6.483,6.363)}
\gppoint{gp mark 0}{(6.483,6.284)}
\gppoint{gp mark 0}{(6.483,6.736)}
\gppoint{gp mark 0}{(6.483,6.791)}
\gppoint{gp mark 0}{(6.483,6.370)}
\gppoint{gp mark 0}{(6.483,5.890)}
\gppoint{gp mark 0}{(6.483,6.350)}
\gppoint{gp mark 0}{(6.487,6.400)}
\gppoint{gp mark 0}{(6.487,6.205)}
\gppoint{gp mark 0}{(6.487,6.699)}
\gppoint{gp mark 0}{(6.487,5.662)}
\gppoint{gp mark 0}{(6.487,5.748)}
\gppoint{gp mark 0}{(6.487,5.743)}
\gppoint{gp mark 0}{(6.491,6.602)}
\gppoint{gp mark 0}{(6.491,6.057)}
\gppoint{gp mark 0}{(6.491,6.444)}
\gppoint{gp mark 0}{(6.494,6.465)}
\gppoint{gp mark 0}{(6.494,6.583)}
\gppoint{gp mark 0}{(6.494,6.811)}
\gppoint{gp mark 0}{(6.494,6.489)}
\gppoint{gp mark 0}{(6.494,7.097)}
\gppoint{gp mark 0}{(6.494,6.057)}
\gppoint{gp mark 0}{(6.498,6.302)}
\gppoint{gp mark 0}{(6.498,6.090)}
\gppoint{gp mark 0}{(6.498,6.572)}
\gppoint{gp mark 0}{(6.498,5.791)}
\gppoint{gp mark 0}{(6.502,6.492)}
\gppoint{gp mark 0}{(6.502,6.263)}
\gppoint{gp mark 0}{(6.502,5.892)}
\gppoint{gp mark 0}{(6.502,6.465)}
\gppoint{gp mark 0}{(6.502,5.994)}
\gppoint{gp mark 0}{(6.506,5.349)}
\gppoint{gp mark 0}{(6.506,6.688)}
\gppoint{gp mark 0}{(6.506,6.140)}
\gppoint{gp mark 0}{(6.506,6.548)}
\gppoint{gp mark 0}{(6.506,5.252)}
\gppoint{gp mark 0}{(6.506,6.917)}
\gppoint{gp mark 0}{(6.510,6.665)}
\gppoint{gp mark 0}{(6.510,5.239)}
\gppoint{gp mark 0}{(6.510,5.814)}
\gppoint{gp mark 0}{(6.510,6.066)}
\gppoint{gp mark 0}{(6.510,5.861)}
\gppoint{gp mark 0}{(6.510,6.568)}
\gppoint{gp mark 0}{(6.510,5.731)}
\gppoint{gp mark 0}{(6.514,6.177)}
\gppoint{gp mark 0}{(6.514,5.752)}
\gppoint{gp mark 0}{(6.514,6.579)}
\gppoint{gp mark 0}{(6.514,6.554)}
\gppoint{gp mark 0}{(6.514,6.424)}
\gppoint{gp mark 0}{(6.517,5.331)}
\gppoint{gp mark 0}{(6.517,6.063)}
\gppoint{gp mark 0}{(6.517,6.582)}
\gppoint{gp mark 0}{(6.517,6.336)}
\gppoint{gp mark 0}{(6.521,5.883)}
\gppoint{gp mark 0}{(6.521,5.830)}
\gppoint{gp mark 0}{(6.521,6.648)}
\gppoint{gp mark 0}{(6.521,6.565)}
\gppoint{gp mark 0}{(6.525,6.091)}
\gppoint{gp mark 0}{(6.525,6.484)}
\gppoint{gp mark 0}{(6.525,6.349)}
\gppoint{gp mark 0}{(6.529,6.164)}
\gppoint{gp mark 0}{(6.529,5.808)}
\gppoint{gp mark 0}{(6.529,6.414)}
\gppoint{gp mark 0}{(6.529,6.882)}
\gppoint{gp mark 0}{(6.529,5.693)}
\gppoint{gp mark 0}{(6.529,5.718)}
\gppoint{gp mark 0}{(6.532,5.938)}
\gppoint{gp mark 0}{(6.532,5.938)}
\gppoint{gp mark 0}{(6.532,5.938)}
\gppoint{gp mark 0}{(6.532,5.938)}
\gppoint{gp mark 0}{(6.532,6.522)}
\gppoint{gp mark 0}{(6.532,5.938)}
\gppoint{gp mark 0}{(6.532,5.302)}
\gppoint{gp mark 0}{(6.536,6.622)}
\gppoint{gp mark 0}{(6.536,6.549)}
\gppoint{gp mark 0}{(6.536,6.432)}
\gppoint{gp mark 0}{(6.536,6.950)}
\gppoint{gp mark 0}{(6.536,6.831)}
\gppoint{gp mark 0}{(6.536,5.327)}
\gppoint{gp mark 0}{(6.536,6.285)}
\gppoint{gp mark 0}{(6.536,6.066)}
\gppoint{gp mark 0}{(6.536,5.797)}
\gppoint{gp mark 0}{(6.536,6.235)}
\gppoint{gp mark 0}{(6.536,6.649)}
\gppoint{gp mark 0}{(6.540,7.057)}
\gppoint{gp mark 0}{(6.540,6.181)}
\gppoint{gp mark 0}{(6.540,6.771)}
\gppoint{gp mark 0}{(6.543,6.207)}
\gppoint{gp mark 0}{(6.543,5.899)}
\gppoint{gp mark 0}{(6.543,6.504)}
\gppoint{gp mark 0}{(6.543,6.369)}
\gppoint{gp mark 0}{(6.547,5.763)}
\gppoint{gp mark 0}{(6.547,6.889)}
\gppoint{gp mark 0}{(6.547,6.235)}
\gppoint{gp mark 0}{(6.547,6.540)}
\gppoint{gp mark 0}{(6.547,6.251)}
\gppoint{gp mark 0}{(6.551,6.166)}
\gppoint{gp mark 0}{(6.551,5.696)}
\gppoint{gp mark 0}{(6.551,5.854)}
\gppoint{gp mark 0}{(6.551,4.883)}
\gppoint{gp mark 0}{(6.551,5.474)}
\gppoint{gp mark 0}{(6.551,6.898)}
\gppoint{gp mark 0}{(6.554,5.614)}
\gppoint{gp mark 0}{(6.558,6.753)}
\gppoint{gp mark 0}{(6.558,6.778)}
\gppoint{gp mark 0}{(6.558,5.689)}
\gppoint{gp mark 0}{(6.558,5.764)}
\gppoint{gp mark 0}{(6.558,5.943)}
\gppoint{gp mark 0}{(6.561,6.420)}
\gppoint{gp mark 0}{(6.561,6.581)}
\gppoint{gp mark 0}{(6.561,5.788)}
\gppoint{gp mark 0}{(6.561,5.788)}
\gppoint{gp mark 0}{(6.561,6.881)}
\gppoint{gp mark 0}{(6.565,5.850)}
\gppoint{gp mark 0}{(6.565,5.974)}
\gppoint{gp mark 0}{(6.565,5.924)}
\gppoint{gp mark 0}{(6.565,6.570)}
\gppoint{gp mark 0}{(6.565,6.665)}
\gppoint{gp mark 0}{(6.565,6.544)}
\gppoint{gp mark 0}{(6.565,6.614)}
\gppoint{gp mark 0}{(6.569,6.804)}
\gppoint{gp mark 0}{(6.569,5.880)}
\gppoint{gp mark 0}{(6.569,6.462)}
\gppoint{gp mark 0}{(6.569,5.854)}
\gppoint{gp mark 0}{(6.569,5.900)}
\gppoint{gp mark 0}{(6.572,5.863)}
\gppoint{gp mark 0}{(6.572,6.209)}
\gppoint{gp mark 0}{(6.572,5.016)}
\gppoint{gp mark 0}{(6.572,5.773)}
\gppoint{gp mark 0}{(6.572,6.673)}
\gppoint{gp mark 0}{(6.572,6.118)}
\gppoint{gp mark 0}{(6.572,6.878)}
\gppoint{gp mark 0}{(6.572,5.773)}
\gppoint{gp mark 0}{(6.576,5.016)}
\gppoint{gp mark 0}{(6.576,5.340)}
\gppoint{gp mark 0}{(6.576,5.340)}
\gppoint{gp mark 0}{(6.576,6.944)}
\gppoint{gp mark 0}{(6.576,6.739)}
\gppoint{gp mark 0}{(6.576,5.340)}
\gppoint{gp mark 0}{(6.576,6.540)}
\gppoint{gp mark 0}{(6.576,6.160)}
\gppoint{gp mark 0}{(6.579,5.350)}
\gppoint{gp mark 0}{(6.579,6.686)}
\gppoint{gp mark 0}{(6.579,5.320)}
\gppoint{gp mark 0}{(6.579,6.856)}
\gppoint{gp mark 0}{(6.579,5.884)}
\gppoint{gp mark 0}{(6.579,6.358)}
\gppoint{gp mark 0}{(6.579,6.668)}
\gppoint{gp mark 0}{(6.579,6.579)}
\gppoint{gp mark 0}{(6.579,6.539)}
\gppoint{gp mark 0}{(6.579,6.814)}
\gppoint{gp mark 0}{(6.583,6.182)}
\gppoint{gp mark 0}{(6.583,6.872)}
\gppoint{gp mark 0}{(6.583,6.182)}
\gppoint{gp mark 0}{(6.583,7.187)}
\gppoint{gp mark 0}{(6.583,6.709)}
\gppoint{gp mark 0}{(6.583,5.605)}
\gppoint{gp mark 0}{(6.586,5.731)}
\gppoint{gp mark 0}{(6.586,6.432)}
\gppoint{gp mark 0}{(6.586,6.598)}
\gppoint{gp mark 0}{(6.590,5.956)}
\gppoint{gp mark 0}{(6.590,6.507)}
\gppoint{gp mark 0}{(6.593,6.097)}
\gppoint{gp mark 0}{(6.593,7.092)}
\gppoint{gp mark 0}{(6.593,6.708)}
\gppoint{gp mark 0}{(6.597,5.945)}
\gppoint{gp mark 0}{(6.597,6.848)}
\gppoint{gp mark 0}{(6.600,5.625)}
\gppoint{gp mark 0}{(6.600,6.350)}
\gppoint{gp mark 0}{(6.604,6.858)}
\gppoint{gp mark 0}{(6.604,5.824)}
\gppoint{gp mark 0}{(6.604,7.040)}
\gppoint{gp mark 0}{(6.604,6.026)}
\gppoint{gp mark 0}{(6.604,6.568)}
\gppoint{gp mark 0}{(6.604,6.497)}
\gppoint{gp mark 0}{(6.604,6.626)}
\gppoint{gp mark 0}{(6.604,7.212)}
\gppoint{gp mark 0}{(6.604,6.655)}
\gppoint{gp mark 0}{(6.604,6.448)}
\gppoint{gp mark 0}{(6.604,5.569)}
\gppoint{gp mark 0}{(6.607,5.675)}
\gppoint{gp mark 0}{(6.607,6.110)}
\gppoint{gp mark 0}{(6.607,5.136)}
\gppoint{gp mark 0}{(6.607,5.028)}
\gppoint{gp mark 0}{(6.607,6.753)}
\gppoint{gp mark 0}{(6.610,6.209)}
\gppoint{gp mark 0}{(6.610,6.418)}
\gppoint{gp mark 0}{(6.610,6.470)}
\gppoint{gp mark 0}{(6.610,6.274)}
\gppoint{gp mark 0}{(6.610,5.779)}
\gppoint{gp mark 0}{(6.614,6.817)}
\gppoint{gp mark 0}{(6.614,5.640)}
\gppoint{gp mark 0}{(6.614,6.198)}
\gppoint{gp mark 0}{(6.617,6.681)}
\gppoint{gp mark 0}{(6.617,6.519)}
\gppoint{gp mark 0}{(6.617,6.400)}
\gppoint{gp mark 0}{(6.621,5.293)}
\gppoint{gp mark 0}{(6.621,6.430)}
\gppoint{gp mark 0}{(6.621,6.496)}
\gppoint{gp mark 0}{(6.624,6.187)}
\gppoint{gp mark 0}{(6.624,6.776)}
\gppoint{gp mark 0}{(6.627,6.275)}
\gppoint{gp mark 0}{(6.627,6.536)}
\gppoint{gp mark 0}{(6.627,5.903)}
\gppoint{gp mark 0}{(6.627,6.789)}
\gppoint{gp mark 0}{(6.627,5.337)}
\gppoint{gp mark 0}{(6.627,6.812)}
\gppoint{gp mark 0}{(6.627,7.442)}
\gppoint{gp mark 0}{(6.627,5.957)}
\gppoint{gp mark 0}{(6.631,6.749)}
\gppoint{gp mark 0}{(6.631,7.170)}
\gppoint{gp mark 0}{(6.631,7.088)}
\gppoint{gp mark 0}{(6.631,7.012)}
\gppoint{gp mark 0}{(6.634,6.928)}
\gppoint{gp mark 0}{(6.634,6.697)}
\gppoint{gp mark 0}{(6.634,6.201)}
\gppoint{gp mark 0}{(6.634,5.466)}
\gppoint{gp mark 0}{(6.634,6.119)}
\gppoint{gp mark 0}{(6.634,6.544)}
\gppoint{gp mark 0}{(6.637,5.403)}
\gppoint{gp mark 0}{(6.637,6.487)}
\gppoint{gp mark 0}{(6.637,5.454)}
\gppoint{gp mark 0}{(6.637,5.685)}
\gppoint{gp mark 0}{(6.637,6.342)}
\gppoint{gp mark 0}{(6.637,6.474)}
\gppoint{gp mark 0}{(6.641,6.635)}
\gppoint{gp mark 0}{(6.641,6.563)}
\gppoint{gp mark 0}{(6.644,6.707)}
\gppoint{gp mark 0}{(6.644,5.816)}
\gppoint{gp mark 0}{(6.644,5.881)}
\gppoint{gp mark 0}{(6.644,6.666)}
\gppoint{gp mark 0}{(6.644,6.176)}
\gppoint{gp mark 0}{(6.644,6.175)}
\gppoint{gp mark 0}{(6.647,6.529)}
\gppoint{gp mark 0}{(6.647,6.836)}
\gppoint{gp mark 0}{(6.647,5.676)}
\gppoint{gp mark 0}{(6.647,6.476)}
\gppoint{gp mark 0}{(6.651,6.507)}
\gppoint{gp mark 0}{(6.651,6.951)}
\gppoint{gp mark 0}{(6.651,6.117)}
\gppoint{gp mark 0}{(6.651,6.176)}
\gppoint{gp mark 0}{(6.654,6.594)}
\gppoint{gp mark 0}{(6.654,6.770)}
\gppoint{gp mark 0}{(6.654,7.098)}
\gppoint{gp mark 0}{(6.654,6.619)}
\gppoint{gp mark 0}{(6.654,5.988)}
\gppoint{gp mark 0}{(6.654,5.259)}
\gppoint{gp mark 0}{(6.657,6.453)}
\gppoint{gp mark 0}{(6.657,6.289)}
\gppoint{gp mark 0}{(6.657,6.286)}
\gppoint{gp mark 0}{(6.657,7.174)}
\gppoint{gp mark 0}{(6.660,5.717)}
\gppoint{gp mark 0}{(6.660,6.913)}
\gppoint{gp mark 0}{(6.660,6.408)}
\gppoint{gp mark 0}{(6.660,5.905)}
\gppoint{gp mark 0}{(6.663,6.244)}
\gppoint{gp mark 0}{(6.663,6.401)}
\gppoint{gp mark 0}{(6.663,6.318)}
\gppoint{gp mark 0}{(6.667,6.965)}
\gppoint{gp mark 0}{(6.667,5.803)}
\gppoint{gp mark 0}{(6.667,6.826)}
\gppoint{gp mark 0}{(6.670,6.318)}
\gppoint{gp mark 0}{(6.670,5.122)}
\gppoint{gp mark 0}{(6.670,6.093)}
\gppoint{gp mark 0}{(6.670,5.297)}
\gppoint{gp mark 0}{(6.670,5.923)}
\gppoint{gp mark 0}{(6.670,5.014)}
\gppoint{gp mark 0}{(6.670,6.813)}
\gppoint{gp mark 0}{(6.670,6.755)}
\gppoint{gp mark 0}{(6.670,6.243)}
\gppoint{gp mark 0}{(6.670,6.792)}
\gppoint{gp mark 0}{(6.673,5.444)}
\gppoint{gp mark 0}{(6.673,5.800)}
\gppoint{gp mark 0}{(6.673,5.917)}
\gppoint{gp mark 0}{(6.673,7.116)}
\gppoint{gp mark 0}{(6.673,5.805)}
\gppoint{gp mark 0}{(6.673,6.204)}
\gppoint{gp mark 0}{(6.673,6.819)}
\gppoint{gp mark 0}{(6.676,6.663)}
\gppoint{gp mark 0}{(6.676,5.946)}
\gppoint{gp mark 0}{(6.676,6.663)}
\gppoint{gp mark 0}{(6.676,6.587)}
\gppoint{gp mark 0}{(6.676,6.873)}
\gppoint{gp mark 0}{(6.679,7.146)}
\gppoint{gp mark 0}{(6.679,5.345)}
\gppoint{gp mark 0}{(6.679,6.575)}
\gppoint{gp mark 0}{(6.679,7.146)}
\gppoint{gp mark 0}{(6.679,5.887)}
\gppoint{gp mark 0}{(6.679,6.225)}
\gppoint{gp mark 0}{(6.679,5.592)}
\gppoint{gp mark 0}{(6.679,5.768)}
\gppoint{gp mark 0}{(6.679,7.021)}
\gppoint{gp mark 0}{(6.683,6.489)}
\gppoint{gp mark 0}{(6.683,6.331)}
\gppoint{gp mark 0}{(6.683,6.382)}
\gppoint{gp mark 0}{(6.683,6.525)}
\gppoint{gp mark 0}{(6.683,6.572)}
\gppoint{gp mark 0}{(6.683,6.454)}
\gppoint{gp mark 0}{(6.686,7.118)}
\gppoint{gp mark 0}{(6.686,5.658)}
\gppoint{gp mark 0}{(6.686,5.947)}
\gppoint{gp mark 0}{(6.689,6.550)}
\gppoint{gp mark 0}{(6.689,5.805)}
\gppoint{gp mark 0}{(6.689,6.411)}
\gppoint{gp mark 0}{(6.689,6.775)}
\gppoint{gp mark 0}{(6.689,6.758)}
\gppoint{gp mark 0}{(6.689,6.790)}
\gppoint{gp mark 0}{(6.692,5.902)}
\gppoint{gp mark 0}{(6.692,5.919)}
\gppoint{gp mark 0}{(6.692,5.902)}
\gppoint{gp mark 0}{(6.692,5.359)}
\gppoint{gp mark 0}{(6.695,5.716)}
\gppoint{gp mark 0}{(6.698,5.814)}
\gppoint{gp mark 0}{(6.698,5.752)}
\gppoint{gp mark 0}{(6.698,5.632)}
\gppoint{gp mark 0}{(6.698,6.722)}
\gppoint{gp mark 0}{(6.698,5.994)}
\gppoint{gp mark 0}{(6.698,5.904)}
\gppoint{gp mark 0}{(6.698,6.495)}
\gppoint{gp mark 0}{(6.701,6.045)}
\gppoint{gp mark 0}{(6.701,6.658)}
\gppoint{gp mark 0}{(6.701,6.336)}
\gppoint{gp mark 0}{(6.701,6.124)}
\gppoint{gp mark 0}{(6.701,6.119)}
\gppoint{gp mark 0}{(6.704,6.515)}
\gppoint{gp mark 0}{(6.704,6.720)}
\gppoint{gp mark 0}{(6.704,6.188)}
\gppoint{gp mark 0}{(6.704,6.053)}
\gppoint{gp mark 0}{(6.708,6.813)}
\gppoint{gp mark 0}{(6.708,6.548)}
\gppoint{gp mark 0}{(6.708,6.316)}
\gppoint{gp mark 0}{(6.711,6.729)}
\gppoint{gp mark 0}{(6.711,6.839)}
\gppoint{gp mark 0}{(6.711,6.859)}
\gppoint{gp mark 0}{(6.711,6.446)}
\gppoint{gp mark 0}{(6.711,5.832)}
\gppoint{gp mark 0}{(6.711,6.881)}
\gppoint{gp mark 0}{(6.711,6.916)}
\gppoint{gp mark 0}{(6.711,5.901)}
\gppoint{gp mark 0}{(6.711,6.729)}
\gppoint{gp mark 0}{(6.714,6.784)}
\gppoint{gp mark 0}{(6.714,6.487)}
\gppoint{gp mark 0}{(6.717,7.138)}
\gppoint{gp mark 0}{(6.717,5.121)}
\gppoint{gp mark 0}{(6.717,6.574)}
\gppoint{gp mark 0}{(6.720,6.110)}
\gppoint{gp mark 0}{(6.720,5.458)}
\gppoint{gp mark 0}{(6.723,6.821)}
\gppoint{gp mark 0}{(6.723,5.946)}
\gppoint{gp mark 0}{(6.723,7.041)}
\gppoint{gp mark 0}{(6.723,5.721)}
\gppoint{gp mark 0}{(6.723,6.749)}
\gppoint{gp mark 0}{(6.726,5.207)}
\gppoint{gp mark 0}{(6.726,6.590)}
\gppoint{gp mark 0}{(6.729,6.319)}
\gppoint{gp mark 0}{(6.732,6.895)}
\gppoint{gp mark 0}{(6.732,6.649)}
\gppoint{gp mark 0}{(6.732,7.413)}
\gppoint{gp mark 0}{(6.732,6.479)}
\gppoint{gp mark 0}{(6.735,7.253)}
\gppoint{gp mark 0}{(6.735,5.363)}
\gppoint{gp mark 0}{(6.738,6.848)}
\gppoint{gp mark 0}{(6.738,5.874)}
\gppoint{gp mark 0}{(6.741,5.429)}
\gppoint{gp mark 0}{(6.744,5.994)}
\gppoint{gp mark 0}{(6.744,5.375)}
\gppoint{gp mark 0}{(6.744,6.407)}
\gppoint{gp mark 0}{(6.744,6.953)}
\gppoint{gp mark 0}{(6.747,6.648)}
\gppoint{gp mark 0}{(6.747,5.851)}
\gppoint{gp mark 0}{(6.750,5.245)}
\gppoint{gp mark 0}{(6.756,7.085)}
\gppoint{gp mark 0}{(6.756,6.653)}
\gppoint{gp mark 0}{(6.756,6.030)}
\gppoint{gp mark 0}{(6.756,6.178)}
\gppoint{gp mark 0}{(6.756,7.082)}
\gppoint{gp mark 0}{(6.764,5.907)}
\gppoint{gp mark 0}{(6.764,6.469)}
\gppoint{gp mark 0}{(6.767,6.333)}
\gppoint{gp mark 0}{(6.770,7.276)}
\gppoint{gp mark 0}{(6.770,6.087)}
\gppoint{gp mark 0}{(6.773,5.348)}
\gppoint{gp mark 0}{(6.776,7.175)}
\gppoint{gp mark 0}{(6.776,6.131)}
\gppoint{gp mark 0}{(6.781,6.482)}
\gppoint{gp mark 0}{(6.784,5.582)}
\gppoint{gp mark 0}{(6.784,6.594)}
\gppoint{gp mark 0}{(6.784,6.902)}
\gppoint{gp mark 0}{(6.787,6.360)}
\gppoint{gp mark 0}{(6.787,6.430)}
\gppoint{gp mark 0}{(6.790,7.286)}
\gppoint{gp mark 0}{(6.793,5.565)}
\gppoint{gp mark 0}{(6.793,6.536)}
\gppoint{gp mark 0}{(6.793,5.669)}
\gppoint{gp mark 0}{(6.796,5.918)}
\gppoint{gp mark 0}{(6.798,7.218)}
\gppoint{gp mark 0}{(6.801,6.420)}
\gppoint{gp mark 0}{(6.804,6.614)}
\gppoint{gp mark 0}{(6.804,7.071)}
\gppoint{gp mark 0}{(6.807,7.139)}
\gppoint{gp mark 0}{(6.810,6.781)}
\gppoint{gp mark 0}{(6.812,5.456)}
\gppoint{gp mark 0}{(6.812,5.457)}
\gppoint{gp mark 0}{(6.815,5.446)}
\gppoint{gp mark 0}{(6.815,5.766)}
\gppoint{gp mark 0}{(6.837,7.245)}
\gppoint{gp mark 0}{(6.839,6.022)}
\gppoint{gp mark 0}{(6.842,7.196)}
\gppoint{gp mark 0}{(6.842,7.367)}
\gppoint{gp mark 0}{(6.842,5.785)}
\gppoint{gp mark 0}{(6.845,5.342)}
\gppoint{gp mark 0}{(6.847,6.804)}
\gppoint{gp mark 0}{(6.850,6.832)}
\gppoint{gp mark 0}{(6.850,7.000)}
\gppoint{gp mark 0}{(6.853,6.939)}
\gppoint{gp mark 0}{(6.855,6.039)}
\gppoint{gp mark 0}{(6.858,7.028)}
\gppoint{gp mark 0}{(6.858,5.515)}
\gppoint{gp mark 0}{(6.863,7.151)}
\gppoint{gp mark 0}{(6.866,5.895)}
\gppoint{gp mark 0}{(6.868,5.754)}
\gppoint{gp mark 0}{(6.871,5.834)}
\gppoint{gp mark 0}{(6.874,5.927)}
\gppoint{gp mark 0}{(6.876,6.887)}
\gppoint{gp mark 0}{(6.879,6.183)}
\gppoint{gp mark 0}{(6.884,7.449)}
\gppoint{gp mark 0}{(6.886,7.087)}
\gppoint{gp mark 0}{(6.889,7.008)}
\gppoint{gp mark 0}{(6.891,5.691)}
\gppoint{gp mark 0}{(6.894,6.910)}
\gppoint{gp mark 0}{(6.894,5.380)}
\gppoint{gp mark 0}{(6.896,6.927)}
\gppoint{gp mark 0}{(6.899,5.492)}
\gppoint{gp mark 0}{(6.901,7.088)}
\gppoint{gp mark 0}{(6.906,6.322)}
\gppoint{gp mark 0}{(6.914,6.196)}
\gppoint{gp mark 0}{(6.919,5.835)}
\gppoint{gp mark 0}{(6.926,7.362)}
\gppoint{gp mark 0}{(6.929,7.460)}
\gppoint{gp mark 0}{(6.936,6.484)}
\gppoint{gp mark 0}{(6.936,6.049)}
\gppoint{gp mark 0}{(6.936,6.975)}
\gppoint{gp mark 0}{(6.936,7.255)}
\gppoint{gp mark 0}{(6.938,7.023)}
\gppoint{gp mark 0}{(6.941,5.477)}
\gppoint{gp mark 0}{(6.948,5.980)}
\gppoint{gp mark 0}{(6.969,7.020)}
\gppoint{gp mark 0}{(6.976,7.335)}
\gppoint{gp mark 0}{(6.978,7.200)}
\gppoint{gp mark 0}{(6.985,6.954)}
\gppoint{gp mark 0}{(6.988,7.055)}
\gppoint{gp mark 0}{(6.990,7.345)}
\gppoint{gp mark 0}{(6.999,7.040)}
\gppoint{gp mark 0}{(6.999,5.641)}
\gppoint{gp mark 0}{(7.006,5.829)}
\gppoint{gp mark 0}{(7.015,6.112)}
\gppoint{gp mark 0}{(7.026,6.539)}
\gppoint{gp mark 0}{(7.030,5.249)}
\gppoint{gp mark 0}{(7.032,7.005)}
\gppoint{gp mark 0}{(7.034,6.205)}
\gppoint{gp mark 0}{(7.047,5.790)}
\gppoint{gp mark 0}{(7.049,6.917)}
\gppoint{gp mark 0}{(7.052,7.307)}
\gppoint{gp mark 0}{(7.056,7.532)}
\gppoint{gp mark 0}{(7.060,7.111)}
\gppoint{gp mark 0}{(7.062,5.351)}
\gppoint{gp mark 0}{(7.079,7.579)}
\gppoint{gp mark 0}{(7.087,7.163)}
\gppoint{gp mark 0}{(7.091,5.621)}
\gppoint{gp mark 0}{(7.093,7.446)}
\gppoint{gp mark 0}{(7.095,7.196)}
\gppoint{gp mark 0}{(7.099,5.353)}
\gppoint{gp mark 0}{(7.118,6.882)}
\gppoint{gp mark 0}{(7.129,5.491)}
\gppoint{gp mark 0}{(7.145,4.713)}
\gppoint{gp mark 0}{(7.166,6.268)}
\gppoint{gp mark 0}{(7.175,5.506)}
\gppoint{gp mark 0}{(7.183,6.124)}
\gppoint{gp mark 0}{(7.233,5.919)}
\gppoint{gp mark 0}{(7.239,6.074)}
\gppoint{gp mark 0}{(7.247,6.393)}
\gppoint{gp mark 0}{(7.283,7.532)}
\gppoint{gp mark 0}{(7.301,6.319)}
\gppoint{gp mark 0}{(7.321,7.816)}
\gppoint{gp mark 0}{(7.374,6.075)}
\gppoint{gp mark 0}{(7.402,7.260)}
\gppoint{gp mark 0}{(7.418,7.844)}
\gppoint{gp mark 0}{(7.476,7.338)}
\gppoint{gp mark 0}{(7.495,7.470)}
\gppoint{gp mark 0}{(7.505,6.128)}
\gppoint{gp mark 0}{(7.546,7.495)}
\gppoint{gp mark 0}{(7.596,6.124)}
\gppoint{gp mark 0}{(7.639,7.206)}
\gppoint{gp mark 0}{(7.833,5.524)}
\gppoint{gp mark 0}{(7.945,6.509)}
\gppoint{gp mark 0}{(8.104,8.265)}
\gppoint{gp mark 0}{(8.214,8.253)}
\node[gp node left] at (9.927,1.935) {$N$ };
\gpcolor{color=gp lt color 2}
\gpsetlinetype{gp lt plot 0}
\draw[gp path] (10.663,1.935)--(11.579,1.935);
\draw[gp path] (1.320,1.705)--(1.427,1.779)--(1.535,1.854)--(1.642,1.929)--(1.749,2.004)%
  --(1.857,2.078)--(1.964,2.153)--(2.071,2.228)--(2.179,2.302)--(2.286,2.377)--(2.393,2.452)%
  --(2.501,2.526)--(2.608,2.601)--(2.715,2.676)--(2.823,2.751)--(2.930,2.825)--(3.037,2.900)%
  --(3.145,2.975)--(3.252,3.049)--(3.360,3.124)--(3.467,3.199)--(3.574,3.274)--(3.682,3.348)%
  --(3.789,3.423)--(3.896,3.498)--(4.004,3.572)--(4.111,3.647)--(4.218,3.722)--(4.326,3.797)%
  --(4.433,3.871)--(4.540,3.946)--(4.648,4.021)--(4.755,4.095)--(4.862,4.170)--(4.970,4.245)%
  --(5.077,4.319)--(5.184,4.394)--(5.292,4.469)--(5.399,4.544)--(5.506,4.618)--(5.614,4.693)%
  --(5.721,4.768)--(5.828,4.842)--(5.936,4.917)--(6.043,4.992)--(6.150,5.067)--(6.258,5.141)%
  --(6.365,5.216)--(6.472,5.291)--(6.580,5.365)--(6.687,5.440)--(6.795,5.515)--(6.902,5.589)%
  --(7.009,5.664)--(7.117,5.739)--(7.224,5.814)--(7.331,5.888)--(7.439,5.963)--(7.546,6.038)%
  --(7.653,6.112)--(7.761,6.187)--(7.868,6.262)--(7.975,6.337)--(8.083,6.411)--(8.190,6.486)%
  --(8.297,6.561)--(8.405,6.635)--(8.512,6.710)--(8.619,6.785)--(8.727,6.859)--(8.834,6.934)%
  --(8.941,7.009)--(9.049,7.084)--(9.156,7.158)--(9.263,7.233)--(9.371,7.308)--(9.478,7.382)%
  --(9.585,7.457)--(9.693,7.532)--(9.800,7.607)--(9.907,7.681)--(10.015,7.756)--(10.122,7.831)%
  --(10.230,7.905)--(10.337,7.980)--(10.444,8.055)--(10.552,8.130)--(10.659,8.204)--(10.766,8.279)%
  --(10.874,8.354)--(10.913,8.381);
\gpcolor{color=gp lt color border}
\node[gp node left] at (9.927,1.627) {$N^2$ };
\gpcolor{color=gp lt color 3}
\draw[gp path] (10.663,1.627)--(11.579,1.627);
\draw[gp path] (2.033,0.985)--(2.071,1.039)--(2.179,1.188)--(2.286,1.338)--(2.393,1.487)%
  --(2.501,1.637)--(2.608,1.786)--(2.715,1.935)--(2.823,2.085)--(2.930,2.234)--(3.037,2.384)%
  --(3.145,2.533)--(3.252,2.683)--(3.360,2.832)--(3.467,2.981)--(3.574,3.131)--(3.682,3.280)%
  --(3.789,3.430)--(3.896,3.579)--(4.004,3.728)--(4.111,3.878)--(4.218,4.027)--(4.326,4.177)%
  --(4.433,4.326)--(4.540,4.476)--(4.648,4.625)--(4.755,4.774)--(4.862,4.924)--(4.970,5.073)%
  --(5.077,5.223)--(5.184,5.372)--(5.292,5.521)--(5.399,5.671)--(5.506,5.820)--(5.614,5.970)%
  --(5.721,6.119)--(5.828,6.268)--(5.936,6.418)--(6.043,6.567)--(6.150,6.717)--(6.258,6.866)%
  --(6.365,7.016)--(6.472,7.165)--(6.580,7.314)--(6.687,7.464)--(6.795,7.613)--(6.902,7.763)%
  --(7.009,7.912)--(7.117,8.061)--(7.224,8.211)--(7.331,8.360)--(7.346,8.381);
\gpcolor{color=gp lt color border}
\node[gp node left] at (9.927,1.319) {$N^3$ };
\gpcolor{color=gp lt color 6}
\draw[gp path] (10.663,1.319)--(11.579,1.319);
\draw[gp path] (2.504,0.985)--(2.608,1.203)--(2.715,1.427)--(2.823,1.651)--(2.930,1.876)%
  --(3.037,2.100)--(3.145,2.324)--(3.252,2.548)--(3.360,2.772)--(3.467,2.996)--(3.574,3.220)%
  --(3.682,3.444)--(3.789,3.669)--(3.896,3.893)--(4.004,4.117)--(4.111,4.341)--(4.218,4.565)%
  --(4.326,4.789)--(4.433,5.013)--(4.540,5.237)--(4.648,5.462)--(4.755,5.686)--(4.862,5.910)%
  --(4.970,6.134)--(5.077,6.358)--(5.184,6.582)--(5.292,6.806)--(5.399,7.030)--(5.506,7.254)%
  --(5.614,7.479)--(5.721,7.703)--(5.828,7.927)--(5.936,8.151)--(6.043,8.375)--(6.046,8.381);
\gpcolor{color=gp lt color border}
\gpsetlinetype{gp lt border}
\gpsetlinewidth{1.00}
\draw[gp path] (1.320,8.381)--(1.320,0.985)--(11.947,0.985)--(11.947,8.381)--cycle;
%% coordinates of the plot area
\gpdefrectangularnode{gp plot 1}{\pgfpoint{1.320cm}{0.985cm}}{\pgfpoint{11.947cm}{8.381cm}}
\end{tikzpicture}
%% gnuplot variables

      \end{myplot}

      Больший интерес представляет количество операций объединения множеств и
      суммарное время, затраченное на эти операции\footnote{
        Объединение двух множеств проводится за время, пропорциональное их
        размеру.
      }. Их зависимость от размеров программы представлена на
      рис.~\ref{plot:merge_ops}, \ref{plot:simple_ops}. Видно, что количество
      операций объединения множеств линейно зависит от размеров метода, а
      время, затраченное на эти операции, близко к квадратичному.

      Получается, что на практике алгоритм работает за время, пропорциональное
      квадрату размера метода. Это согласуется с теоретически расчитанной
      временной сложностью алгоритма, которая есть $O(N^3)$ в худшем случае,
      причем видно, что на практике <<худший случай>> не достигается.

    \subsection{Потребление памяти}

      \begin{myplot}%
        {Зависимость объема множеств целей от размера метода}%
        {plot:mem_used}
        \begin{tikzpicture}[gnuplot]
%% generated with GNUPLOT 4.5p0 (Lua 5.1; terminal rev. 99, script rev. 98)
%% 27.05.2011 12:41:07
\path (0.000,0.000) rectangle (12.500,8.750);
\gpcolor{color=gp lt color border}
\gpsetlinetype{gp lt border}
\gpsetlinewidth{1.00}
\draw[gp path] (1.320,0.985)--(1.500,0.985);
\draw[gp path] (11.947,0.985)--(11.767,0.985);
\node[gp node right] at (1.136,0.985) {$10^{0}$};
\draw[gp path] (1.320,1.542)--(1.410,1.542);
\draw[gp path] (11.947,1.542)--(11.857,1.542);
\draw[gp path] (1.320,1.867)--(1.410,1.867);
\draw[gp path] (11.947,1.867)--(11.857,1.867);
\draw[gp path] (1.320,2.098)--(1.410,2.098);
\draw[gp path] (11.947,2.098)--(11.857,2.098);
\draw[gp path] (1.320,2.277)--(1.410,2.277);
\draw[gp path] (11.947,2.277)--(11.857,2.277);
\draw[gp path] (1.320,2.424)--(1.410,2.424);
\draw[gp path] (11.947,2.424)--(11.857,2.424);
\draw[gp path] (1.320,2.548)--(1.410,2.548);
\draw[gp path] (11.947,2.548)--(11.857,2.548);
\draw[gp path] (1.320,2.655)--(1.410,2.655);
\draw[gp path] (11.947,2.655)--(11.857,2.655);
\draw[gp path] (1.320,2.749)--(1.410,2.749);
\draw[gp path] (11.947,2.749)--(11.857,2.749);
\draw[gp path] (1.320,2.834)--(1.500,2.834);
\draw[gp path] (11.947,2.834)--(11.767,2.834);
\node[gp node right] at (1.136,2.834) {$10^{1}$};
\draw[gp path] (1.320,3.391)--(1.410,3.391);
\draw[gp path] (11.947,3.391)--(11.857,3.391);
\draw[gp path] (1.320,3.716)--(1.410,3.716);
\draw[gp path] (11.947,3.716)--(11.857,3.716);
\draw[gp path] (1.320,3.947)--(1.410,3.947);
\draw[gp path] (11.947,3.947)--(11.857,3.947);
\draw[gp path] (1.320,4.126)--(1.410,4.126);
\draw[gp path] (11.947,4.126)--(11.857,4.126);
\draw[gp path] (1.320,4.273)--(1.410,4.273);
\draw[gp path] (11.947,4.273)--(11.857,4.273);
\draw[gp path] (1.320,4.397)--(1.410,4.397);
\draw[gp path] (11.947,4.397)--(11.857,4.397);
\draw[gp path] (1.320,4.504)--(1.410,4.504);
\draw[gp path] (11.947,4.504)--(11.857,4.504);
\draw[gp path] (1.320,4.598)--(1.410,4.598);
\draw[gp path] (11.947,4.598)--(11.857,4.598);
\draw[gp path] (1.320,4.683)--(1.500,4.683);
\draw[gp path] (11.947,4.683)--(11.767,4.683);
\node[gp node right] at (1.136,4.683) {$10^{2}$};
\draw[gp path] (1.320,5.240)--(1.410,5.240);
\draw[gp path] (11.947,5.240)--(11.857,5.240);
\draw[gp path] (1.320,5.565)--(1.410,5.565);
\draw[gp path] (11.947,5.565)--(11.857,5.565);
\draw[gp path] (1.320,5.796)--(1.410,5.796);
\draw[gp path] (11.947,5.796)--(11.857,5.796);
\draw[gp path] (1.320,5.975)--(1.410,5.975);
\draw[gp path] (11.947,5.975)--(11.857,5.975);
\draw[gp path] (1.320,6.122)--(1.410,6.122);
\draw[gp path] (11.947,6.122)--(11.857,6.122);
\draw[gp path] (1.320,6.246)--(1.410,6.246);
\draw[gp path] (11.947,6.246)--(11.857,6.246);
\draw[gp path] (1.320,6.353)--(1.410,6.353);
\draw[gp path] (11.947,6.353)--(11.857,6.353);
\draw[gp path] (1.320,6.447)--(1.410,6.447);
\draw[gp path] (11.947,6.447)--(11.857,6.447);
\draw[gp path] (1.320,6.532)--(1.500,6.532);
\draw[gp path] (11.947,6.532)--(11.767,6.532);
\node[gp node right] at (1.136,6.532) {$10^{3}$};
\draw[gp path] (1.320,7.089)--(1.410,7.089);
\draw[gp path] (11.947,7.089)--(11.857,7.089);
\draw[gp path] (1.320,7.414)--(1.410,7.414);
\draw[gp path] (11.947,7.414)--(11.857,7.414);
\draw[gp path] (1.320,7.645)--(1.410,7.645);
\draw[gp path] (11.947,7.645)--(11.857,7.645);
\draw[gp path] (1.320,7.824)--(1.410,7.824);
\draw[gp path] (11.947,7.824)--(11.857,7.824);
\draw[gp path] (1.320,7.971)--(1.410,7.971);
\draw[gp path] (11.947,7.971)--(11.857,7.971);
\draw[gp path] (1.320,8.095)--(1.410,8.095);
\draw[gp path] (11.947,8.095)--(11.857,8.095);
\draw[gp path] (1.320,8.202)--(1.410,8.202);
\draw[gp path] (11.947,8.202)--(11.857,8.202);
\draw[gp path] (1.320,8.296)--(1.410,8.296);
\draw[gp path] (11.947,8.296)--(11.857,8.296);
\draw[gp path] (1.320,8.381)--(1.500,8.381);
\draw[gp path] (11.947,8.381)--(11.767,8.381);
\node[gp node right] at (1.136,8.381) {$10^{4}$};
\draw[gp path] (1.320,0.985)--(1.320,1.165);
\draw[gp path] (1.320,8.381)--(1.320,8.201);
\node[gp node center] at (1.320,0.677) {$10^{0}$};
\draw[gp path] (2.120,0.985)--(2.120,1.075);
\draw[gp path] (2.120,8.381)--(2.120,8.291);
\draw[gp path] (2.588,0.985)--(2.588,1.075);
\draw[gp path] (2.588,8.381)--(2.588,8.291);
\draw[gp path] (2.920,0.985)--(2.920,1.075);
\draw[gp path] (2.920,8.381)--(2.920,8.291);
\draw[gp path] (3.177,0.985)--(3.177,1.075);
\draw[gp path] (3.177,8.381)--(3.177,8.291);
\draw[gp path] (3.387,0.985)--(3.387,1.075);
\draw[gp path] (3.387,8.381)--(3.387,8.291);
\draw[gp path] (3.565,0.985)--(3.565,1.075);
\draw[gp path] (3.565,8.381)--(3.565,8.291);
\draw[gp path] (3.719,0.985)--(3.719,1.075);
\draw[gp path] (3.719,8.381)--(3.719,8.291);
\draw[gp path] (3.855,0.985)--(3.855,1.075);
\draw[gp path] (3.855,8.381)--(3.855,8.291);
\draw[gp path] (3.977,0.985)--(3.977,1.165);
\draw[gp path] (3.977,8.381)--(3.977,8.201);
\node[gp node center] at (3.977,0.677) {$10^{1}$};
\draw[gp path] (4.777,0.985)--(4.777,1.075);
\draw[gp path] (4.777,8.381)--(4.777,8.291);
\draw[gp path] (5.244,0.985)--(5.244,1.075);
\draw[gp path] (5.244,8.381)--(5.244,8.291);
\draw[gp path] (5.576,0.985)--(5.576,1.075);
\draw[gp path] (5.576,8.381)--(5.576,8.291);
\draw[gp path] (5.834,0.985)--(5.834,1.075);
\draw[gp path] (5.834,8.381)--(5.834,8.291);
\draw[gp path] (6.044,0.985)--(6.044,1.075);
\draw[gp path] (6.044,8.381)--(6.044,8.291);
\draw[gp path] (6.222,0.985)--(6.222,1.075);
\draw[gp path] (6.222,8.381)--(6.222,8.291);
\draw[gp path] (6.376,0.985)--(6.376,1.075);
\draw[gp path] (6.376,8.381)--(6.376,8.291);
\draw[gp path] (6.512,0.985)--(6.512,1.075);
\draw[gp path] (6.512,8.381)--(6.512,8.291);
\draw[gp path] (6.634,0.985)--(6.634,1.165);
\draw[gp path] (6.634,8.381)--(6.634,8.201);
\node[gp node center] at (6.634,0.677) {$10^{2}$};
\draw[gp path] (7.433,0.985)--(7.433,1.075);
\draw[gp path] (7.433,8.381)--(7.433,8.291);
\draw[gp path] (7.901,0.985)--(7.901,1.075);
\draw[gp path] (7.901,8.381)--(7.901,8.291);
\draw[gp path] (8.233,0.985)--(8.233,1.075);
\draw[gp path] (8.233,8.381)--(8.233,8.291);
\draw[gp path] (8.490,0.985)--(8.490,1.075);
\draw[gp path] (8.490,8.381)--(8.490,8.291);
\draw[gp path] (8.701,0.985)--(8.701,1.075);
\draw[gp path] (8.701,8.381)--(8.701,8.291);
\draw[gp path] (8.879,0.985)--(8.879,1.075);
\draw[gp path] (8.879,8.381)--(8.879,8.291);
\draw[gp path] (9.033,0.985)--(9.033,1.075);
\draw[gp path] (9.033,8.381)--(9.033,8.291);
\draw[gp path] (9.169,0.985)--(9.169,1.075);
\draw[gp path] (9.169,8.381)--(9.169,8.291);
\draw[gp path] (9.290,0.985)--(9.290,1.165);
\draw[gp path] (9.290,8.381)--(9.290,8.201);
\node[gp node center] at (9.290,0.677) {$10^{3}$};
\draw[gp path] (10.090,0.985)--(10.090,1.075);
\draw[gp path] (10.090,8.381)--(10.090,8.291);
\draw[gp path] (10.558,0.985)--(10.558,1.075);
\draw[gp path] (10.558,8.381)--(10.558,8.291);
\draw[gp path] (10.890,0.985)--(10.890,1.075);
\draw[gp path] (10.890,8.381)--(10.890,8.291);
\draw[gp path] (11.147,0.985)--(11.147,1.075);
\draw[gp path] (11.147,8.381)--(11.147,8.291);
\draw[gp path] (11.358,0.985)--(11.358,1.075);
\draw[gp path] (11.358,8.381)--(11.358,8.291);
\draw[gp path] (11.535,0.985)--(11.535,1.075);
\draw[gp path] (11.535,8.381)--(11.535,8.291);
\draw[gp path] (11.690,0.985)--(11.690,1.075);
\draw[gp path] (11.690,8.381)--(11.690,8.291);
\draw[gp path] (11.825,0.985)--(11.825,1.075);
\draw[gp path] (11.825,8.381)--(11.825,8.291);
\draw[gp path] (11.947,0.985)--(11.947,1.165);
\draw[gp path] (11.947,8.381)--(11.947,8.201);
\node[gp node center] at (11.947,0.677) {$10^{4}$};
\draw[gp path] (1.320,8.381)--(1.320,0.985)--(11.947,0.985)--(11.947,8.381)--cycle;
\node[gp node center,rotate=-270] at (0.246,4.683) {Объем памяти, в условных единицах};
\node[gp node center] at (6.633,0.215) {Количество присваиваний};
\gpsetlinewidth{2.00}
\gpsetpointsize{4.00}
\gppoint{gp mark 0}{(3.565,2.911)}
\gppoint{gp mark 0}{(5.738,4.589)}
\gppoint{gp mark 0}{(6.690,5.298)}
\gppoint{gp mark 0}{(4.445,3.632)}
\gppoint{gp mark 0}{(3.977,3.260)}
\gppoint{gp mark 0}{(3.387,2.749)}
\gppoint{gp mark 0}{(6.063,4.875)}
\gppoint{gp mark 0}{(5.944,4.737)}
\gppoint{gp mark 0}{(7.337,5.740)}
\gppoint{gp mark 0}{(6.537,5.194)}
\gppoint{gp mark 0}{(4.777,4.042)}
\gppoint{gp mark 0}{(6.499,5.168)}
\gppoint{gp mark 0}{(4.655,3.793)}
\gppoint{gp mark 0}{(6.405,5.065)}
\gppoint{gp mark 0}{(4.717,3.840)}
\gppoint{gp mark 0}{(4.987,4.173)}
\gppoint{gp mark 0}{(5.165,4.286)}
\gppoint{gp mark 0}{(5.319,4.385)}
\gppoint{gp mark 0}{(5.455,4.473)}
\gppoint{gp mark 0}{(7.078,5.541)}
\gppoint{gp mark 0}{(6.154,4.887)}
\gppoint{gp mark 0}{(3.977,3.211)}
\gppoint{gp mark 0}{(3.565,2.834)}
\gppoint{gp mark 0}{(4.445,3.601)}
\gppoint{gp mark 0}{(6.622,5.215)}
\gppoint{gp mark 0}{(5.762,4.598)}
\gppoint{gp mark 0}{(6.872,5.393)}
\gppoint{gp mark 0}{(7.101,5.554)}
\gppoint{gp mark 0}{(4.886,3.947)}
\gppoint{gp mark 0}{(6.432,5.118)}
\gppoint{gp mark 0}{(6.656,5.240)}
\gppoint{gp mark 0}{(6.005,4.774)}
\gppoint{gp mark 0}{(4.087,3.306)}
\gppoint{gp mark 0}{(4.365,3.537)}
\gppoint{gp mark 0}{(4.589,3.906)}
\gppoint{gp mark 0}{(6.954,5.475)}
\gppoint{gp mark 0}{(4.445,3.570)}
\gppoint{gp mark 0}{(4.589,3.885)}
\gppoint{gp mark 0}{(4.087,3.260)}
\gppoint{gp mark 0}{(7.648,5.951)}
\gppoint{gp mark 0}{(3.977,3.349)}
\gppoint{gp mark 0}{(5.517,4.452)}
\gppoint{gp mark 0}{(6.586,5.236)}
\gppoint{gp mark 0}{(5.605,4.552)}
\gppoint{gp mark 0}{(4.886,4.005)}
\gppoint{gp mark 0}{(6.154,4.900)}
\gppoint{gp mark 0}{(5.985,4.809)}
\gppoint{gp mark 0}{(3.565,2.749)}
\gppoint{gp mark 0}{(4.365,3.632)}
\gppoint{gp mark 0}{(4.717,3.793)}
\gppoint{gp mark 0}{(6.082,4.849)}
\gppoint{gp mark 0}{(7.176,5.634)}
\gppoint{gp mark 0}{(6.254,5.014)}
\gppoint{gp mark 0}{(4.279,3.689)}
\gppoint{gp mark 0}{(6.550,5.194)}
\gppoint{gp mark 0}{(5.834,4.675)}
\gppoint{gp mark 0}{(4.987,4.203)}
\gppoint{gp mark 0}{(6.882,5.432)}
\gppoint{gp mark 0}{(4.087,3.211)}
\gppoint{gp mark 0}{(7.538,5.858)}
\gppoint{gp mark 0}{(4.445,3.537)}
\gppoint{gp mark 0}{(3.565,2.655)}
\gppoint{gp mark 0}{(6.668,5.240)}
\gppoint{gp mark 0}{(4.365,3.601)}
\gppoint{gp mark 0}{(5.079,4.126)}
\gppoint{gp mark 0}{(5.712,4.616)}
\gppoint{gp mark 0}{(5.633,4.524)}
\gppoint{gp mark 0}{(6.119,4.918)}
\gppoint{gp mark 0}{(4.886,3.986)}
\gppoint{gp mark 0}{(6.656,5.248)}
\gppoint{gp mark 0}{(4.777,4.059)}
\gppoint{gp mark 0}{(7.318,5.720)}
\gppoint{gp mark 0}{(6.205,4.953)}
\gppoint{gp mark 0}{(6.473,5.100)}
\gppoint{gp mark 0}{(4.589,3.863)}
\gppoint{gp mark 0}{(6.025,4.774)}
\gppoint{gp mark 0}{(6.512,5.146)}
\gppoint{gp mark 0}{(7.548,5.880)}
\gppoint{gp mark 0}{(4.717,3.768)}
\gppoint{gp mark 0}{(6.005,4.788)}
\gppoint{gp mark 0}{(5.389,4.385)}
\gppoint{gp mark 0}{(5.319,4.337)}
\gppoint{gp mark 0}{(4.655,3.885)}
\gppoint{gp mark 0}{(7.456,5.814)}
\gppoint{gp mark 0}{(6.754,5.327)}
\gppoint{gp mark 0}{(4.279,3.632)}
\gppoint{gp mark 0}{(6.909,5.435)}
\gppoint{gp mark 0}{(6.900,5.428)}
\gppoint{gp mark 0}{(7.462,5.818)}
\gppoint{gp mark 0}{(4.777,3.967)}
\gppoint{gp mark 0}{(6.063,4.849)}
\gppoint{gp mark 0}{(6.332,5.035)}
\gppoint{gp mark 0}{(7.183,5.625)}
\gppoint{gp mark 0}{(5.810,4.675)}
\gppoint{gp mark 0}{(5.712,4.607)}
\gppoint{gp mark 0}{(6.863,5.403)}
\gppoint{gp mark 0}{(5.923,4.752)}
\gppoint{gp mark 0}{(5.486,4.452)}
\gppoint{gp mark 0}{(6.286,5.003)}
\gppoint{gp mark 0}{(6.574,5.203)}
\gppoint{gp mark 0}{(4.886,4.042)}
\gppoint{gp mark 0}{(6.562,5.194)}
\gppoint{gp mark 0}{(6.205,4.947)}
\gppoint{gp mark 0}{(7.253,5.673)}
\gppoint{gp mark 0}{(6.136,4.894)}
\gppoint{gp mark 0}{(4.833,3.986)}
\gppoint{gp mark 0}{(4.519,3.768)}
\gppoint{gp mark 0}{(5.319,4.325)}
\gppoint{gp mark 0}{(4.087,3.467)}
\gppoint{gp mark 0}{(4.279,3.601)}
\gppoint{gp mark 0}{(5.712,4.598)}
\gppoint{gp mark 0}{(5.857,4.699)}
\gppoint{gp mark 0}{(4.938,4.059)}
\gppoint{gp mark 0}{(7.259,5.675)}
\gppoint{gp mark 0}{(4.187,3.537)}
\gppoint{gp mark 0}{(6.679,5.271)}
\gppoint{gp mark 0}{(4.987,4.094)}
\gppoint{gp mark 0}{(3.977,3.391)}
\gppoint{gp mark 0}{(6.270,4.987)}
\gppoint{gp mark 0}{(4.886,4.024)}
\gppoint{gp mark 0}{(6.171,4.912)}
\gppoint{gp mark 0}{(5.547,4.473)}
\gppoint{gp mark 0}{(5.738,4.642)}
\gppoint{gp mark 0}{(4.279,3.570)}
\gppoint{gp mark 0}{(7.456,5.818)}
\gppoint{gp mark 0}{(7.109,5.568)}
\gppoint{gp mark 0}{(6.025,4.809)}
\gppoint{gp mark 0}{(4.717,3.885)}
\gppoint{gp mark 0}{(6.610,5.236)}
\gppoint{gp mark 0}{(4.938,4.042)}
\gppoint{gp mark 0}{(6.512,5.168)}
\gppoint{gp mark 0}{(7.197,5.629)}
\gppoint{gp mark 0}{(6.824,5.383)}
\gppoint{gp mark 0}{(5.282,4.273)}
\gppoint{gp mark 0}{(5.834,4.699)}
\gppoint{gp mark 0}{(5.354,4.325)}
\gppoint{gp mark 0}{(4.445,3.661)}
\gppoint{gp mark 0}{(6.419,5.080)}
\gppoint{gp mark 0}{(6.574,5.190)}
\gppoint{gp mark 0}{(5.712,4.580)}
\gppoint{gp mark 0}{(4.365,3.716)}
\gppoint{gp mark 0}{(3.855,3.211)}
\gppoint{gp mark 0}{(4.279,3.537)}
\gppoint{gp mark 0}{(6.317,5.030)}
\gppoint{gp mark 0}{(7.796,6.052)}
\gppoint{gp mark 0}{(4.777,3.986)}
\gppoint{gp mark 0}{(5.576,4.524)}
\gppoint{gp mark 0}{(4.987,4.126)}
\gppoint{gp mark 0}{(5.985,4.774)}
\gppoint{gp mark 0}{(6.405,5.090)}
\gppoint{gp mark 0}{(6.254,4.987)}
\gppoint{gp mark 0}{(6.980,5.475)}
\gppoint{gp mark 0}{(6.189,4.942)}
\gppoint{gp mark 0}{(6.446,5.095)}
\gppoint{gp mark 0}{(4.519,4.042)}
\gppoint{gp mark 0}{(5.517,4.337)}
\gppoint{gp mark 0}{(3.565,1.867)}
\gppoint{gp mark 0}{(5.079,4.232)}
\gppoint{gp mark 0}{(5.712,4.571)}
\gppoint{gp mark 0}{(5.547,4.361)}
\gppoint{gp mark 0}{(4.886,3.793)}
\gppoint{gp mark 0}{(5.633,4.589)}
\gppoint{gp mark 0}{(7.086,5.546)}
\gppoint{gp mark 0}{(4.365,3.260)}
\gppoint{gp mark 0}{(4.445,3.349)}
\gppoint{gp mark 0}{(5.923,4.659)}
\gppoint{gp mark 0}{(5.244,4.110)}
\gppoint{gp mark 0}{(6.222,4.887)}
\gppoint{gp mark 0}{(7.147,5.589)}
\gppoint{gp mark 0}{(5.901,4.707)}
\gppoint{gp mark 0}{(4.777,3.885)}
\gppoint{gp mark 0}{(4.187,3.430)}
\gppoint{gp mark 0}{(5.205,4.203)}
\gppoint{gp mark 0}{(4.938,3.840)}
\gppoint{gp mark 0}{(5.389,4.430)}
\gppoint{gp mark 0}{(4.833,3.927)}
\gppoint{gp mark 0}{(5.633,4.580)}
\gppoint{gp mark 0}{(3.177,2.424)}
\gppoint{gp mark 0}{(6.598,5.263)}
\gppoint{gp mark 0}{(7.078,5.560)}
\gppoint{gp mark 0}{(4.365,3.211)}
\gppoint{gp mark 0}{(5.738,4.504)}
\gppoint{gp mark 0}{(4.445,3.306)}
\gppoint{gp mark 0}{(4.886,3.768)}
\gppoint{gp mark 0}{(4.777,3.863)}
\gppoint{gp mark 0}{(7.070,5.554)}
\gppoint{gp mark 0}{(5.455,4.373)}
\gppoint{gp mark 0}{(5.762,4.524)}
\gppoint{gp mark 0}{(6.512,5.173)}
\gppoint{gp mark 0}{(6.656,5.207)}
\gppoint{gp mark 0}{(5.517,4.361)}
\gppoint{gp mark 0}{(3.387,1.867)}
\gppoint{gp mark 0}{(3.177,2.277)}
\gppoint{gp mark 0}{(4.445,3.260)}
\gppoint{gp mark 0}{(5.923,4.675)}
\gppoint{gp mark 0}{(4.365,3.349)}
\gppoint{gp mark 0}{(4.187,3.503)}
\gppoint{gp mark 0}{(6.656,5.219)}
\gppoint{gp mark 0}{(4.279,3.430)}
\gppoint{gp mark 0}{(5.547,4.337)}
\gppoint{gp mark 0}{(4.938,3.793)}
\gppoint{gp mark 0}{(5.282,4.110)}
\gppoint{gp mark 0}{(6.254,5.055)}
\gppoint{gp mark 0}{(4.589,4.042)}
\gppoint{gp mark 0}{(7.456,5.810)}
\gppoint{gp mark 0}{(6.459,5.075)}
\gppoint{gp mark 0}{(6.376,5.132)}
\gppoint{gp mark 0}{(7.584,5.900)}
\gppoint{gp mark 0}{(4.777,3.927)}
\gppoint{gp mark 0}{(4.886,3.840)}
\gppoint{gp mark 0}{(6.834,5.334)}
\gppoint{gp mark 0}{(5.787,4.767)}
\gppoint{gp mark 0}{(6.645,5.227)}
\gppoint{gp mark 0}{(5.034,4.273)}
\gppoint{gp mark 0}{(4.938,3.768)}
\gppoint{gp mark 0}{(3.177,2.098)}
\gppoint{gp mark 0}{(7.854,6.108)}
\gppoint{gp mark 0}{(5.455,4.397)}
\gppoint{gp mark 0}{(5.389,4.441)}
\gppoint{gp mark 0}{(5.486,4.373)}
\gppoint{gp mark 0}{(7.211,5.627)}
\gppoint{gp mark 0}{(4.365,3.306)}
\gppoint{gp mark 0}{(4.445,3.211)}
\gppoint{gp mark 0}{(4.886,3.817)}
\gppoint{gp mark 0}{(6.598,5.256)}
\gppoint{gp mark 0}{(4.717,3.947)}
\gppoint{gp mark 0}{(5.944,4.650)}
\gppoint{gp mark 0}{(5.123,4.217)}
\gppoint{gp mark 0}{(4.279,3.391)}
\gppoint{gp mark 0}{(5.576,4.633)}
\gppoint{gp mark 0}{(6.610,5.248)}
\gppoint{gp mark 0}{(5.686,4.562)}
\gppoint{gp mark 0}{(6.936,5.469)}
\gppoint{gp mark 0}{(5.762,4.504)}
\gppoint{gp mark 0}{(6.872,5.435)}
\gppoint{gp mark 0}{(3.387,2.277)}
\gppoint{gp mark 0}{(6.525,5.203)}
\gppoint{gp mark 0}{(4.445,3.503)}
\gppoint{gp mark 0}{(5.455,4.337)}
\gppoint{gp mark 0}{(5.282,4.203)}
\gppoint{gp mark 0}{(4.187,3.260)}
\gppoint{gp mark 0}{(3.565,2.548)}
\gppoint{gp mark 0}{(4.886,3.885)}
\gppoint{gp mark 0}{(5.762,4.571)}
\gppoint{gp mark 0}{(6.205,4.900)}
\gppoint{gp mark 0}{(4.279,3.349)}
\gppoint{gp mark 0}{(4.365,3.430)}
\gppoint{gp mark 0}{(5.738,4.552)}
\gppoint{gp mark 0}{(3.177,1.867)}
\gppoint{gp mark 0}{(6.634,5.211)}
\gppoint{gp mark 0}{(5.389,4.473)}
\gppoint{gp mark 0}{(7.343,5.753)}
\gppoint{gp mark 0}{(4.833,3.840)}
\gppoint{gp mark 0}{(4.777,3.793)}
\gppoint{gp mark 0}{(6.743,5.290)}
\gppoint{gp mark 0}{(4.938,3.927)}
\gppoint{gp mark 0}{(7.101,5.546)}
\gppoint{gp mark 0}{(6.286,5.040)}
\gppoint{gp mark 0}{(4.833,3.817)}
\gppoint{gp mark 0}{(6.562,5.155)}
\gppoint{gp mark 0}{(6.459,5.080)}
\gppoint{gp mark 0}{(4.445,3.467)}
\gppoint{gp mark 0}{(5.034,4.246)}
\gppoint{gp mark 0}{(5.923,4.683)}
\gppoint{gp mark 0}{(4.279,3.306)}
\gppoint{gp mark 0}{(3.565,2.424)}
\gppoint{gp mark 0}{(5.660,4.633)}
\gppoint{gp mark 0}{(6.512,5.190)}
\gppoint{gp mark 0}{(6.063,4.788)}
\gppoint{gp mark 0}{(4.938,3.906)}
\gppoint{gp mark 0}{(4.187,3.211)}
\gppoint{gp mark 0}{(6.909,5.406)}
\gppoint{gp mark 0}{(5.810,4.730)}
\gppoint{gp mark 0}{(4.987,4.217)}
\gppoint{gp mark 0}{(3.387,2.098)}
\gppoint{gp mark 0}{(4.886,3.863)}
\gppoint{gp mark 0}{(5.244,4.158)}
\gppoint{gp mark 0}{(6.082,4.802)}
\gppoint{gp mark 0}{(5.547,4.397)}
\gppoint{gp mark 0}{(6.100,4.816)}
\gppoint{gp mark 0}{(5.354,4.441)}
\gppoint{gp mark 0}{(7.154,5.581)}
\gppoint{gp mark 0}{(6.100,4.809)}
\gppoint{gp mark 0}{(5.857,4.752)}
\gppoint{gp mark 0}{(4.365,3.503)}
\gppoint{gp mark 0}{(4.279,3.260)}
\gppoint{gp mark 0}{(4.777,3.840)}
\gppoint{gp mark 0}{(7.046,5.501)}
\gppoint{gp mark 0}{(3.565,2.277)}
\gppoint{gp mark 0}{(6.063,4.781)}
\gppoint{gp mark 0}{(3.387,2.548)}
\gppoint{gp mark 0}{(4.445,3.430)}
\gppoint{gp mark 0}{(6.622,5.259)}
\gppoint{gp mark 0}{(5.244,4.203)}
\gppoint{gp mark 0}{(6.574,5.159)}
\gppoint{gp mark 0}{(6.634,5.219)}
\gppoint{gp mark 0}{(5.834,4.767)}
\gppoint{gp mark 0}{(4.187,3.349)}
\gppoint{gp mark 0}{(6.082,4.823)}
\gppoint{gp mark 0}{(6.815,5.334)}
\gppoint{gp mark 0}{(6.205,4.887)}
\gppoint{gp mark 0}{(4.187,3.306)}
\gppoint{gp mark 0}{(4.589,3.947)}
\gppoint{gp mark 0}{(5.205,4.094)}
\gppoint{gp mark 0}{(5.517,4.397)}
\gppoint{gp mark 0}{(6.486,5.181)}
\gppoint{gp mark 0}{(4.279,3.211)}
\gppoint{gp mark 0}{(5.738,4.562)}
\gppoint{gp mark 0}{(4.365,3.467)}
\gppoint{gp mark 0}{(4.777,3.817)}
\gppoint{gp mark 0}{(7.094,5.532)}
\gppoint{gp mark 0}{(6.432,5.070)}
\gppoint{gp mark 0}{(4.445,3.391)}
\gppoint{gp mark 0}{(4.833,3.768)}
\gppoint{gp mark 0}{(3.565,2.098)}
\gppoint{gp mark 0}{(6.332,4.976)}
\gppoint{gp mark 0}{(5.686,4.524)}
\gppoint{gp mark 0}{(6.853,5.412)}
\gppoint{gp mark 0}{(4.938,3.863)}
\gppoint{gp mark 0}{(6.785,5.379)}
\gppoint{gp mark 0}{(6.622,5.256)}
\gppoint{gp mark 0}{(5.165,4.126)}
\gppoint{gp mark 0}{(5.660,4.616)}
\gppoint{gp mark 0}{(5.282,4.158)}
\gppoint{gp mark 0}{(4.717,4.024)}
\gppoint{gp mark 0}{(5.034,4.217)}
\gppoint{gp mark 0}{(5.123,4.273)}
\gppoint{gp mark 0}{(7.154,5.576)}
\gppoint{gp mark 0}{(7.767,5.999)}
\gppoint{gp mark 0}{(6.537,5.123)}
\gppoint{gp mark 0}{(6.486,5.085)}
\gppoint{gp mark 0}{(6.044,4.752)}
\gppoint{gp mark 0}{(6.119,5.003)}
\gppoint{gp mark 0}{(5.738,4.430)}
\gppoint{gp mark 0}{(4.987,3.885)}
\gppoint{gp mark 0}{(5.205,4.077)}
\gppoint{gp mark 0}{(5.123,3.840)}
\gppoint{gp mark 0}{(5.165,4.042)}
\gppoint{gp mark 0}{(5.901,4.875)}
\gppoint{gp mark 0}{(4.445,2.911)}
\gppoint{gp mark 0}{(4.365,2.749)}
\gppoint{gp mark 0}{(3.855,2.548)}
\gppoint{gp mark 0}{(4.279,3.160)}
\gppoint{gp mark 0}{(6.690,5.236)}
\gppoint{gp mark 0}{(4.187,3.045)}
\gppoint{gp mark 0}{(3.719,2.277)}
\gppoint{gp mark 0}{(4.717,4.142)}
\gppoint{gp mark 0}{(6.701,5.211)}
\gppoint{gp mark 0}{(7.225,5.682)}
\gppoint{gp mark 0}{(7.161,5.557)}
\gppoint{gp mark 0}{(6.562,5.104)}
\gppoint{gp mark 0}{(7.246,5.620)}
\gppoint{gp mark 0}{(6.853,5.355)}
\gppoint{gp mark 0}{(5.244,3.967)}
\gppoint{gp mark 0}{(7.356,5.698)}
\gppoint{gp mark 0}{(5.985,4.699)}
\gppoint{gp mark 0}{(5.034,3.906)}
\gppoint{gp mark 0}{(5.738,4.419)}
\gppoint{gp mark 0}{(5.762,4.441)}
\gppoint{gp mark 0}{(6.743,5.240)}
\gppoint{gp mark 0}{(4.365,2.655)}
\gppoint{gp mark 0}{(4.445,2.834)}
\gppoint{gp mark 0}{(3.855,2.424)}
\gppoint{gp mark 0}{(4.279,3.104)}
\gppoint{gp mark 0}{(4.187,2.980)}
\gppoint{gp mark 0}{(3.719,2.098)}
\gppoint{gp mark 0}{(6.347,4.930)}
\gppoint{gp mark 0}{(7.305,5.734)}
\gppoint{gp mark 0}{(5.165,4.024)}
\gppoint{gp mark 0}{(6.805,5.425)}
\gppoint{gp mark 0}{(5.633,4.325)}
\gppoint{gp mark 0}{(4.987,3.863)}
\gppoint{gp mark 0}{(6.286,4.881)}
\gppoint{gp mark 0}{(6.005,4.650)}
\gppoint{gp mark 0}{(6.863,5.331)}
\gppoint{gp mark 0}{(5.660,4.349)}
\gppoint{gp mark 0}{(6.918,5.498)}
\gppoint{gp mark 0}{(6.834,5.419)}
\gppoint{gp mark 0}{(6.432,5.190)}
\gppoint{gp mark 0}{(6.499,5.090)}
\gppoint{gp mark 0}{(7.663,5.993)}
\gppoint{gp mark 0}{(3.855,2.277)}
\gppoint{gp mark 0}{(5.079,3.840)}
\gppoint{gp mark 0}{(5.244,4.005)}
\gppoint{gp mark 0}{(4.187,3.160)}
\gppoint{gp mark 0}{(3.719,2.548)}
\gppoint{gp mark 0}{(4.445,2.749)}
\gppoint{gp mark 0}{(5.282,3.967)}
\gppoint{gp mark 0}{(4.279,3.045)}
\gppoint{gp mark 0}{(4.365,2.911)}
\gppoint{gp mark 0}{(4.987,3.927)}
\gppoint{gp mark 0}{(5.205,4.042)}
\gppoint{gp mark 0}{(6.405,5.151)}
\gppoint{gp mark 0}{(5.123,3.793)}
\gppoint{gp mark 0}{(5.633,4.361)}
\gppoint{gp mark 0}{(6.574,5.104)}
\gppoint{gp mark 0}{(5.660,4.337)}
\gppoint{gp mark 0}{(6.586,5.298)}
\gppoint{gp mark 0}{(6.082,4.737)}
\gppoint{gp mark 0}{(5.686,4.494)}
\gppoint{gp mark 0}{(5.165,4.077)}
\gppoint{gp mark 0}{(5.576,4.408)}
\gppoint{gp mark 0}{(6.954,5.530)}
\gppoint{gp mark 0}{(6.562,5.109)}
\gppoint{gp mark 0}{(5.660,4.325)}
\gppoint{gp mark 0}{(6.701,5.215)}
\gppoint{gp mark 0}{(5.282,3.947)}
\gppoint{gp mark 0}{(5.205,4.024)}
\gppoint{gp mark 0}{(3.719,2.424)}
\gppoint{gp mark 0}{(5.389,4.524)}
\gppoint{gp mark 0}{(6.347,4.942)}
\gppoint{gp mark 0}{(7.054,5.592)}
\gppoint{gp mark 0}{(5.244,3.986)}
\gppoint{gp mark 0}{(5.686,4.483)}
\gppoint{gp mark 0}{(4.279,2.980)}
\gppoint{gp mark 0}{(4.445,2.655)}
\gppoint{gp mark 0}{(4.365,2.834)}
\gppoint{gp mark 0}{(4.717,4.094)}
\gppoint{gp mark 0}{(4.187,3.104)}
\gppoint{gp mark 0}{(5.079,3.817)}
\gppoint{gp mark 0}{(5.034,3.863)}
\gppoint{gp mark 0}{(5.123,3.768)}
\gppoint{gp mark 0}{(6.690,5.223)}
\gppoint{gp mark 0}{(7.299,5.734)}
\gppoint{gp mark 0}{(3.855,2.098)}
\gppoint{gp mark 0}{(7.500,5.884)}
\gppoint{gp mark 0}{(7.624,5.912)}
\gppoint{gp mark 0}{(7.005,5.444)}
\gppoint{gp mark 0}{(7.517,5.885)}
\gppoint{gp mark 0}{(6.775,5.396)}
\gppoint{gp mark 0}{(7.312,5.749)}
\gppoint{gp mark 0}{(6.005,4.691)}
\gppoint{gp mark 0}{(5.123,3.927)}
\gppoint{gp mark 0}{(7.484,5.864)}
\gppoint{gp mark 0}{(6.100,4.767)}
\gppoint{gp mark 0}{(4.087,2.548)}
\gppoint{gp mark 0}{(4.445,3.160)}
\gppoint{gp mark 0}{(4.279,2.911)}
\gppoint{gp mark 0}{(4.987,3.793)}
\gppoint{gp mark 0}{(4.187,2.749)}
\gppoint{gp mark 0}{(4.365,3.045)}
\gppoint{gp mark 0}{(5.205,4.005)}
\gppoint{gp mark 0}{(6.044,4.722)}
\gppoint{gp mark 0}{(5.034,3.840)}
\gppoint{gp mark 0}{(5.964,4.659)}
\gppoint{gp mark 0}{(6.610,5.290)}
\gppoint{gp mark 0}{(5.762,4.494)}
\gppoint{gp mark 0}{(5.079,3.885)}
\gppoint{gp mark 0}{(5.282,4.077)}
\gppoint{gp mark 0}{(3.977,2.277)}
\gppoint{gp mark 0}{(6.082,4.752)}
\gppoint{gp mark 0}{(6.222,5.045)}
\gppoint{gp mark 0}{(7.253,5.634)}
\gppoint{gp mark 0}{(6.754,5.267)}
\gppoint{gp mark 0}{(6.634,5.305)}
\gppoint{gp mark 0}{(5.165,3.967)}
\gppoint{gp mark 0}{(6.927,5.495)}
\gppoint{gp mark 0}{(6.025,4.707)}
\gppoint{gp mark 0}{(6.459,5.190)}
\gppoint{gp mark 0}{(4.987,3.768)}
\gppoint{gp mark 0}{(4.445,3.104)}
\gppoint{gp mark 0}{(4.187,2.655)}
\gppoint{gp mark 0}{(4.365,2.980)}
\gppoint{gp mark 0}{(4.279,2.834)}
\gppoint{gp mark 0}{(4.087,2.424)}
\gppoint{gp mark 0}{(5.319,4.504)}
\gppoint{gp mark 0}{(7.259,5.637)}
\gppoint{gp mark 0}{(5.034,3.817)}
\gppoint{gp mark 0}{(5.686,4.419)}
\gppoint{gp mark 0}{(5.205,3.986)}
\gppoint{gp mark 0}{(6.025,4.699)}
\gppoint{gp mark 0}{(3.977,2.098)}
\gppoint{gp mark 0}{(6.562,5.118)}
\gppoint{gp mark 0}{(6.733,5.248)}
\gppoint{gp mark 0}{(5.244,4.024)}
\gppoint{gp mark 0}{(6.900,5.373)}
\gppoint{gp mark 0}{(5.576,4.361)}
\gppoint{gp mark 0}{(5.633,4.408)}
\gppoint{gp mark 0}{(5.605,4.337)}
\gppoint{gp mark 0}{(4.938,4.286)}
\gppoint{gp mark 0}{(5.034,3.793)}
\gppoint{gp mark 0}{(4.365,3.160)}
\gppoint{gp mark 0}{(6.722,5.252)}
\gppoint{gp mark 0}{(4.187,2.911)}
\gppoint{gp mark 0}{(6.668,5.320)}
\gppoint{gp mark 0}{(3.977,2.548)}
\gppoint{gp mark 0}{(4.445,3.045)}
\gppoint{gp mark 0}{(4.279,2.749)}
\gppoint{gp mark 0}{(4.987,3.840)}
\gppoint{gp mark 0}{(6.254,4.900)}
\gppoint{gp mark 0}{(6.446,5.177)}
\gppoint{gp mark 0}{(4.087,2.277)}
\gppoint{gp mark 0}{(7.094,5.599)}
\gppoint{gp mark 0}{(5.205,3.967)}
\gppoint{gp mark 0}{(5.660,4.385)}
\gppoint{gp mark 0}{(7.517,5.882)}
\gppoint{gp mark 0}{(7.422,5.826)}
\gppoint{gp mark 0}{(7.038,5.484)}
\gppoint{gp mark 0}{(6.025,4.691)}
\gppoint{gp mark 0}{(6.512,5.095)}
\gppoint{gp mark 0}{(6.171,5.003)}
\gppoint{gp mark 0}{(6.656,5.327)}
\gppoint{gp mark 0}{(7.714,5.964)}
\gppoint{gp mark 0}{(6.722,5.248)}
\gppoint{gp mark 0}{(6.222,5.050)}
\gppoint{gp mark 0}{(5.034,3.768)}
\gppoint{gp mark 0}{(5.547,4.616)}
\gppoint{gp mark 0}{(5.244,4.059)}
\gppoint{gp mark 0}{(5.123,3.863)}
\gppoint{gp mark 0}{(7.416,5.812)}
\gppoint{gp mark 0}{(6.419,5.155)}
\gppoint{gp mark 0}{(4.187,2.834)}
\gppoint{gp mark 0}{(4.279,2.655)}
\gppoint{gp mark 0}{(4.365,3.104)}
\gppoint{gp mark 0}{(4.445,2.980)}
\gppoint{gp mark 0}{(7.490,5.869)}
\gppoint{gp mark 0}{(5.079,3.906)}
\gppoint{gp mark 0}{(6.270,4.881)}
\gppoint{gp mark 0}{(5.633,4.397)}
\gppoint{gp mark 0}{(3.977,2.424)}
\gppoint{gp mark 0}{(6.347,4.965)}
\gppoint{gp mark 0}{(7.211,5.675)}
\gppoint{gp mark 0}{(7.410,5.816)}
\gppoint{gp mark 0}{(5.282,4.024)}
\gppoint{gp mark 0}{(7.462,5.780)}
\gppoint{gp mark 0}{(7.054,5.581)}
\gppoint{gp mark 0}{(5.738,4.483)}
\gppoint{gp mark 0}{(5.205,3.947)}
\gppoint{gp mark 0}{(5.762,4.463)}
\gppoint{gp mark 0}{(6.733,5.240)}
\gppoint{gp mark 0}{(4.187,2.277)}
\gppoint{gp mark 0}{(6.927,5.530)}
\gppoint{gp mark 0}{(7.846,6.045)}
\gppoint{gp mark 0}{(5.123,4.005)}
\gppoint{gp mark 0}{(6.743,5.211)}
\gppoint{gp mark 0}{(5.244,3.793)}
\gppoint{gp mark 0}{(7.038,5.441)}
\gppoint{gp mark 0}{(4.087,2.911)}
\gppoint{gp mark 0}{(3.977,2.749)}
\gppoint{gp mark 0}{(5.985,4.767)}
\gppoint{gp mark 0}{(5.686,4.385)}
\gppoint{gp mark 0}{(3.855,3.160)}
\gppoint{gp mark 0}{(4.279,2.548)}
\gppoint{gp mark 0}{(5.712,4.408)}
\gppoint{gp mark 0}{(5.576,4.473)}
\gppoint{gp mark 0}{(5.660,4.452)}
\gppoint{gp mark 0}{(5.787,4.862)}
\gppoint{gp mark 0}{(4.987,4.042)}
\gppoint{gp mark 0}{(7.022,5.453)}
\gppoint{gp mark 0}{(7.775,6.005)}
\gppoint{gp mark 0}{(6.082,4.659)}
\gppoint{gp mark 0}{(5.282,3.840)}
\gppoint{gp mark 0}{(5.205,3.927)}
\gppoint{gp mark 0}{(5.165,3.885)}
\gppoint{gp mark 0}{(5.944,4.795)}
\gppoint{gp mark 0}{(4.886,4.110)}
\gppoint{gp mark 0}{(7.686,5.966)}
\gppoint{gp mark 0}{(5.123,3.986)}
\gppoint{gp mark 0}{(5.034,4.059)}
\gppoint{gp mark 0}{(6.954,5.498)}
\gppoint{gp mark 0}{(6.512,5.100)}
\gppoint{gp mark 0}{(4.886,4.094)}
\gppoint{gp mark 0}{(6.332,4.918)}
\gppoint{gp mark 0}{(7.211,5.670)}
\gppoint{gp mark 0}{(7.374,5.694)}
\gppoint{gp mark 0}{(3.719,2.980)}
\gppoint{gp mark 0}{(5.205,3.906)}
\gppoint{gp mark 0}{(4.279,2.424)}
\gppoint{gp mark 0}{(5.282,3.817)}
\gppoint{gp mark 0}{(5.686,4.373)}
\gppoint{gp mark 0}{(3.977,2.655)}
\gppoint{gp mark 0}{(4.087,2.834)}
\gppoint{gp mark 0}{(6.764,5.425)}
\gppoint{gp mark 0}{(5.244,3.768)}
\gppoint{gp mark 0}{(5.517,4.504)}
\gppoint{gp mark 0}{(7.038,5.438)}
\gppoint{gp mark 0}{(3.855,3.104)}
\gppoint{gp mark 0}{(7.054,5.607)}
\gppoint{gp mark 0}{(4.187,2.098)}
\gppoint{gp mark 0}{(5.762,4.349)}
\gppoint{gp mark 0}{(6.701,5.240)}
\gppoint{gp mark 0}{(6.805,5.399)}
\gppoint{gp mark 0}{(6.668,5.279)}
\gppoint{gp mark 0}{(5.165,3.863)}
\gppoint{gp mark 0}{(11.757,7.920)}
\gppoint{gp mark 0}{(5.205,3.885)}
\gppoint{gp mark 0}{(5.282,3.793)}
\gppoint{gp mark 0}{(7.273,5.625)}
\gppoint{gp mark 0}{(5.686,4.408)}
\gppoint{gp mark 0}{(5.123,3.967)}
\gppoint{gp mark 0}{(6.525,5.104)}
\gppoint{gp mark 0}{(4.187,2.548)}
\gppoint{gp mark 0}{(5.455,4.571)}
\gppoint{gp mark 0}{(3.977,2.911)}
\gppoint{gp mark 0}{(3.855,3.045)}
\gppoint{gp mark 0}{(4.087,2.749)}
\gppoint{gp mark 0}{(5.034,4.042)}
\gppoint{gp mark 0}{(5.165,3.927)}
\gppoint{gp mark 0}{(5.762,4.337)}
\gppoint{gp mark 0}{(5.712,4.385)}
\gppoint{gp mark 0}{(6.362,4.887)}
\gppoint{gp mark 0}{(5.354,4.625)}
\gppoint{gp mark 0}{(5.244,3.840)}
\gppoint{gp mark 0}{(6.286,4.942)}
\gppoint{gp mark 0}{(5.123,3.947)}
\gppoint{gp mark 0}{(6.918,5.527)}
\gppoint{gp mark 0}{(5.165,3.906)}
\gppoint{gp mark 0}{(3.977,2.834)}
\gppoint{gp mark 0}{(4.187,2.424)}
\gppoint{gp mark 0}{(5.282,3.768)}
\gppoint{gp mark 0}{(3.855,2.980)}
\gppoint{gp mark 0}{(6.537,5.090)}
\gppoint{gp mark 0}{(5.660,4.419)}
\gppoint{gp mark 0}{(4.087,2.655)}
\gppoint{gp mark 0}{(6.486,5.128)}
\gppoint{gp mark 0}{(5.901,4.802)}
\gppoint{gp mark 0}{(5.079,3.986)}
\gppoint{gp mark 0}{(5.205,3.863)}
\gppoint{gp mark 0}{(5.244,3.817)}
\gppoint{gp mark 0}{(7.368,5.703)}
\gppoint{gp mark 0}{(5.517,4.524)}
\gppoint{gp mark 0}{(5.547,4.504)}
\gppoint{gp mark 0}{(5.034,4.024)}
\gppoint{gp mark 0}{(5.787,4.869)}
\gppoint{gp mark 0}{(7.574,5.933)}
\gppoint{gp mark 0}{(6.815,5.399)}
\gppoint{gp mark 0}{(7.117,5.571)}
\gppoint{gp mark 0}{(6.679,5.263)}
\gppoint{gp mark 0}{(7.204,5.670)}
\gppoint{gp mark 0}{(8.889,5.826)}
\gppoint{gp mark 0}{(5.123,4.077)}
\gppoint{gp mark 0}{(3.719,2.749)}
\gppoint{gp mark 0}{(5.205,3.840)}
\gppoint{gp mark 0}{(5.901,4.795)}
\gppoint{gp mark 0}{(5.486,4.533)}
\gppoint{gp mark 0}{(5.738,4.385)}
\gppoint{gp mark 0}{(4.445,2.548)}
\gppoint{gp mark 0}{(5.165,3.793)}
\gppoint{gp mark 0}{(3.977,3.045)}
\gppoint{gp mark 0}{(6.254,4.936)}
\gppoint{gp mark 0}{(6.100,4.707)}
\gppoint{gp mark 0}{(3.855,2.911)}
\gppoint{gp mark 0}{(7.605,5.903)}
\gppoint{gp mark 0}{(6.238,5.014)}
\gppoint{gp mark 0}{(7.046,5.459)}
\gppoint{gp mark 0}{(4.087,3.160)}
\gppoint{gp mark 0}{(5.686,4.337)}
\gppoint{gp mark 0}{(7.154,5.541)}
\gppoint{gp mark 0}{(6.656,5.290)}
\gppoint{gp mark 0}{(7.005,5.478)}
\gppoint{gp mark 0}{(6.722,5.211)}
\gppoint{gp mark 0}{(5.244,3.885)}
\gppoint{gp mark 0}{(8.042,6.248)}
\gppoint{gp mark 0}{(6.254,4.930)}
\gppoint{gp mark 0}{(5.738,4.373)}
\gppoint{gp mark 0}{(4.833,4.126)}
\gppoint{gp mark 0}{(7.404,5.824)}
\gppoint{gp mark 0}{(5.244,3.863)}
\gppoint{gp mark 0}{(3.719,2.655)}
\gppoint{gp mark 0}{(5.712,4.349)}
\gppoint{gp mark 0}{(5.762,4.397)}
\gppoint{gp mark 0}{(5.901,4.788)}
\gppoint{gp mark 0}{(3.855,2.834)}
\gppoint{gp mark 0}{(6.656,5.286)}
\gppoint{gp mark 0}{(6.459,5.155)}
\gppoint{gp mark 0}{(4.087,3.104)}
\gppoint{gp mark 0}{(3.977,2.980)}
\gppoint{gp mark 0}{(5.205,3.817)}
\gppoint{gp mark 0}{(8.039,6.246)}
\gppoint{gp mark 0}{(6.044,4.650)}
\gppoint{gp mark 0}{(5.633,4.463)}
\gppoint{gp mark 0}{(5.985,4.730)}
\gppoint{gp mark 0}{(5.034,3.986)}
\gppoint{gp mark 0}{(5.660,4.483)}
\gppoint{gp mark 0}{(5.282,3.906)}
\gppoint{gp mark 0}{(7.117,5.554)}
\gppoint{gp mark 0}{(7.204,5.656)}
\gppoint{gp mark 0}{(5.944,4.816)}
\gppoint{gp mark 0}{(6.302,4.965)}
\gppoint{gp mark 0}{(4.445,2.424)}
\gppoint{gp mark 0}{(5.123,4.059)}
\gppoint{gp mark 0}{(6.690,5.244)}
\gppoint{gp mark 0}{(5.165,3.840)}
\gppoint{gp mark 0}{(5.319,4.607)}
\gppoint{gp mark 0}{(6.025,4.752)}
\gppoint{gp mark 0}{(4.987,4.005)}
\gppoint{gp mark 0}{(5.985,4.722)}
\gppoint{gp mark 0}{(4.365,2.548)}
\gppoint{gp mark 0}{(4.087,3.045)}
\gppoint{gp mark 0}{(3.855,2.749)}
\gppoint{gp mark 0}{(6.376,5.186)}
\gppoint{gp mark 0}{(3.719,2.911)}
\gppoint{gp mark 0}{(5.517,4.571)}
\gppoint{gp mark 0}{(5.282,3.885)}
\gppoint{gp mark 0}{(6.189,4.992)}
\gppoint{gp mark 0}{(6.063,4.659)}
\gppoint{gp mark 0}{(5.244,3.927)}
\gppoint{gp mark 0}{(7.225,5.668)}
\gppoint{gp mark 0}{(5.633,4.494)}
\gppoint{gp mark 0}{(5.034,3.967)}
\gppoint{gp mark 0}{(5.964,4.737)}
\gppoint{gp mark 0}{(6.499,5.104)}
\gppoint{gp mark 0}{(6.918,5.518)}
\gppoint{gp mark 0}{(6.634,5.283)}
\gppoint{gp mark 0}{(6.656,5.298)}
\gppoint{gp mark 0}{(6.872,5.376)}
\gppoint{gp mark 0}{(6.863,5.383)}
\gppoint{gp mark 0}{(5.605,4.430)}
\gppoint{gp mark 0}{(8.046,6.947)}
\gppoint{gp mark 0}{(5.605,4.419)}
\gppoint{gp mark 0}{(6.286,4.965)}
\gppoint{gp mark 0}{(5.547,4.543)}
\gppoint{gp mark 0}{(5.455,4.524)}
\gppoint{gp mark 0}{(5.762,4.373)}
\gppoint{gp mark 0}{(7.931,6.156)}
\gppoint{gp mark 0}{(6.574,5.080)}
\gppoint{gp mark 0}{(5.205,3.768)}
\gppoint{gp mark 0}{(6.679,5.248)}
\gppoint{gp mark 0}{(5.282,3.863)}
\gppoint{gp mark 0}{(5.244,3.906)}
\gppoint{gp mark 0}{(4.987,3.986)}
\gppoint{gp mark 0}{(3.719,2.834)}
\gppoint{gp mark 0}{(4.087,2.980)}
\gppoint{gp mark 0}{(6.863,5.379)}
\gppoint{gp mark 0}{(6.189,4.987)}
\gppoint{gp mark 0}{(3.855,2.655)}
\gppoint{gp mark 0}{(6.044,4.667)}
\gppoint{gp mark 0}{(3.977,3.104)}
\gppoint{gp mark 0}{(5.165,3.817)}
\gppoint{gp mark 0}{(6.063,4.650)}
\gppoint{gp mark 0}{(4.365,2.424)}
\gppoint{gp mark 0}{(5.123,4.024)}
\gppoint{gp mark 0}{(6.722,5.215)}
\gppoint{gp mark 0}{(5.576,4.441)}
\gppoint{gp mark 0}{(5.712,4.325)}
\gppoint{gp mark 0}{(4.833,4.094)}
\gppoint{gp mark 0}{(4.938,4.158)}
\gppoint{gp mark 0}{(5.834,4.337)}
\gppoint{gp mark 0}{(6.063,4.642)}
\gppoint{gp mark 0}{(5.079,3.260)}
\gppoint{gp mark 0}{(5.787,4.385)}
\gppoint{gp mark 0}{(6.222,5.104)}
\gppoint{gp mark 0}{(5.034,3.503)}
\gppoint{gp mark 0}{(6.419,4.900)}
\gppoint{gp mark 0}{(6.189,5.123)}
\gppoint{gp mark 0}{(4.777,3.045)}
\gppoint{gp mark 0}{(4.833,3.160)}
\gppoint{gp mark 0}{(5.123,3.349)}
\gppoint{gp mark 0}{(5.282,3.632)}
\gppoint{gp mark 0}{(4.987,3.430)}
\gppoint{gp mark 0}{(5.244,3.570)}
\gppoint{gp mark 0}{(5.165,3.689)}
\gppoint{gp mark 0}{(5.205,3.743)}
\gppoint{gp mark 0}{(7.062,5.459)}
\gppoint{gp mark 0}{(4.589,2.548)}
\gppoint{gp mark 0}{(4.938,2.911)}
\gppoint{gp mark 0}{(7.147,5.524)}
\gppoint{gp mark 0}{(5.923,4.430)}
\gppoint{gp mark 0}{(6.574,5.035)}
\gppoint{gp mark 0}{(4.886,2.749)}
\gppoint{gp mark 0}{(5.282,3.601)}
\gppoint{gp mark 0}{(5.547,4.730)}
\gppoint{gp mark 0}{(5.034,3.467)}
\gppoint{gp mark 0}{(5.857,4.349)}
\gppoint{gp mark 0}{(4.833,3.104)}
\gppoint{gp mark 0}{(6.100,4.598)}
\gppoint{gp mark 0}{(4.777,2.980)}
\gppoint{gp mark 0}{(6.936,5.532)}
\gppoint{gp mark 0}{(5.165,3.661)}
\gppoint{gp mark 0}{(4.987,3.391)}
\gppoint{gp mark 0}{(5.123,3.306)}
\gppoint{gp mark 0}{(5.079,3.211)}
\gppoint{gp mark 0}{(5.205,3.716)}
\gppoint{gp mark 0}{(5.244,3.537)}
\gppoint{gp mark 0}{(6.005,4.504)}
\gppoint{gp mark 0}{(7.850,6.034)}
\gppoint{gp mark 0}{(6.891,5.323)}
\gppoint{gp mark 0}{(7.183,5.707)}
\gppoint{gp mark 0}{(7.484,5.776)}
\gppoint{gp mark 0}{(5.810,4.397)}
\gppoint{gp mark 0}{(6.459,4.930)}
\gppoint{gp mark 0}{(5.834,4.325)}
\gppoint{gp mark 0}{(4.589,2.424)}
\gppoint{gp mark 0}{(6.512,4.976)}
\gppoint{gp mark 0}{(5.985,4.562)}
\gppoint{gp mark 0}{(5.944,4.441)}
\gppoint{gp mark 0}{(7.046,5.602)}
\gppoint{gp mark 0}{(5.879,4.463)}
\gppoint{gp mark 0}{(6.795,5.215)}
\gppoint{gp mark 0}{(6.390,4.912)}
\gppoint{gp mark 0}{(7.299,5.625)}
\gppoint{gp mark 0}{(7.054,5.459)}
\gppoint{gp mark 0}{(4.519,2.548)}
\gppoint{gp mark 0}{(4.886,2.911)}
\gppoint{gp mark 0}{(6.005,4.533)}
\gppoint{gp mark 0}{(5.282,3.570)}
\gppoint{gp mark 0}{(5.123,3.260)}
\gppoint{gp mark 0}{(6.622,5.334)}
\gppoint{gp mark 0}{(4.987,3.503)}
\gppoint{gp mark 0}{(5.244,3.632)}
\gppoint{gp mark 0}{(4.777,3.160)}
\gppoint{gp mark 0}{(5.079,3.349)}
\gppoint{gp mark 0}{(5.205,3.689)}
\gppoint{gp mark 0}{(5.034,3.430)}
\gppoint{gp mark 0}{(4.833,3.045)}
\gppoint{gp mark 0}{(4.938,2.749)}
\gppoint{gp mark 0}{(5.923,4.452)}
\gppoint{gp mark 0}{(6.775,5.227)}
\gppoint{gp mark 0}{(5.762,4.836)}
\gppoint{gp mark 0}{(5.787,4.408)}
\gppoint{gp mark 0}{(6.100,4.589)}
\gppoint{gp mark 0}{(6.419,4.887)}
\gppoint{gp mark 0}{(4.987,3.467)}
\gppoint{gp mark 0}{(6.044,4.633)}
\gppoint{gp mark 0}{(5.605,4.802)}
\gppoint{gp mark 0}{(6.853,5.286)}
\gppoint{gp mark 0}{(4.886,2.834)}
\gppoint{gp mark 0}{(5.944,4.419)}
\gppoint{gp mark 0}{(5.205,3.661)}
\gppoint{gp mark 0}{(4.833,2.980)}
\gppoint{gp mark 0}{(5.165,3.716)}
\gppoint{gp mark 0}{(5.034,3.391)}
\gppoint{gp mark 0}{(5.079,3.306)}
\gppoint{gp mark 0}{(4.777,3.104)}
\gppoint{gp mark 0}{(5.244,3.601)}
\gppoint{gp mark 0}{(5.686,4.869)}
\gppoint{gp mark 0}{(5.123,3.211)}
\gppoint{gp mark 0}{(5.787,4.397)}
\gppoint{gp mark 0}{(5.282,3.537)}
\gppoint{gp mark 0}{(5.857,4.325)}
\gppoint{gp mark 0}{(6.446,4.953)}
\gppoint{gp mark 0}{(4.519,2.424)}
\gppoint{gp mark 0}{(7.013,5.576)}
\gppoint{gp mark 0}{(6.005,4.524)}
\gppoint{gp mark 0}{(6.824,5.248)}
\gppoint{gp mark 0}{(7.380,5.675)}
\gppoint{gp mark 0}{(7.374,5.687)}
\gppoint{gp mark 0}{(6.063,4.607)}
\gppoint{gp mark 0}{(7.292,5.625)}
\gppoint{gp mark 0}{(6.100,4.642)}
\gppoint{gp mark 0}{(4.833,2.911)}
\gppoint{gp mark 0}{(5.165,3.570)}
\gppoint{gp mark 0}{(5.034,3.349)}
\gppoint{gp mark 0}{(5.205,3.632)}
\gppoint{gp mark 0}{(7.086,5.466)}
\gppoint{gp mark 0}{(4.886,3.045)}
\gppoint{gp mark 0}{(4.987,3.260)}
\gppoint{gp mark 0}{(5.923,4.473)}
\gppoint{gp mark 0}{(5.123,3.503)}
\gppoint{gp mark 0}{(4.938,3.160)}
\gppoint{gp mark 0}{(5.079,3.430)}
\gppoint{gp mark 0}{(5.244,3.689)}
\gppoint{gp mark 0}{(5.282,3.743)}
\gppoint{gp mark 0}{(7.094,5.472)}
\gppoint{gp mark 0}{(7.595,5.964)}
\gppoint{gp mark 0}{(4.777,2.749)}
\gppoint{gp mark 0}{(6.082,4.625)}
\gppoint{gp mark 0}{(6.473,4.970)}
\gppoint{gp mark 0}{(5.944,4.494)}
\gppoint{gp mark 0}{(4.717,2.548)}
\gppoint{gp mark 0}{(5.964,4.514)}
\gppoint{gp mark 0}{(7.197,5.709)}
\gppoint{gp mark 0}{(7.038,5.609)}
\gppoint{gp mark 0}{(6.668,5.383)}
\gppoint{gp mark 0}{(7.239,5.736)}
\gppoint{gp mark 0}{(7.218,5.723)}
\gppoint{gp mark 0}{(6.025,4.571)}
\gppoint{gp mark 0}{(7.629,6.872)}
\gppoint{gp mark 0}{(4.833,2.834)}
\gppoint{gp mark 0}{(6.574,5.050)}
\gppoint{gp mark 0}{(5.738,4.856)}
\gppoint{gp mark 0}{(5.282,3.716)}
\gppoint{gp mark 0}{(6.005,4.543)}
\gppoint{gp mark 0}{(5.165,3.537)}
\gppoint{gp mark 0}{(5.244,3.661)}
\gppoint{gp mark 0}{(7.522,5.796)}
\gppoint{gp mark 0}{(5.079,3.391)}
\gppoint{gp mark 0}{(5.034,3.306)}
\gppoint{gp mark 0}{(4.938,3.104)}
\gppoint{gp mark 0}{(4.987,3.211)}
\gppoint{gp mark 0}{(5.123,3.467)}
\gppoint{gp mark 0}{(4.886,2.980)}
\gppoint{gp mark 0}{(5.205,3.601)}
\gppoint{gp mark 0}{(5.787,4.325)}
\gppoint{gp mark 0}{(6.405,4.906)}
\gppoint{gp mark 0}{(5.923,4.463)}
\gppoint{gp mark 0}{(5.964,4.504)}
\gppoint{gp mark 0}{(7.392,5.832)}
\gppoint{gp mark 0}{(6.598,5.338)}
\gppoint{gp mark 0}{(6.100,4.633)}
\gppoint{gp mark 0}{(6.785,5.223)}
\gppoint{gp mark 0}{(7.279,5.757)}
\gppoint{gp mark 0}{(6.171,5.085)}
\gppoint{gp mark 0}{(5.787,4.361)}
\gppoint{gp mark 0}{(5.123,3.430)}
\gppoint{gp mark 0}{(5.205,3.570)}
\gppoint{gp mark 0}{(4.886,3.160)}
\gppoint{gp mark 0}{(4.938,3.045)}
\gppoint{gp mark 0}{(5.944,4.473)}
\gppoint{gp mark 0}{(5.165,3.632)}
\gppoint{gp mark 0}{(5.244,3.743)}
\gppoint{gp mark 0}{(5.079,3.503)}
\gppoint{gp mark 0}{(4.987,3.349)}
\gppoint{gp mark 0}{(4.777,2.911)}
\gppoint{gp mark 0}{(5.034,3.260)}
\gppoint{gp mark 0}{(6.574,5.045)}
\gppoint{gp mark 0}{(6.044,4.607)}
\gppoint{gp mark 0}{(5.282,3.689)}
\gppoint{gp mark 0}{(7.842,6.033)}
\gppoint{gp mark 0}{(4.833,2.749)}
\gppoint{gp mark 0}{(5.633,4.823)}
\gppoint{gp mark 0}{(7.427,5.852)}
\gppoint{gp mark 0}{(6.645,5.362)}
\gppoint{gp mark 0}{(5.205,3.537)}
\gppoint{gp mark 0}{(4.777,2.834)}
\gppoint{gp mark 0}{(7.147,5.516)}
\gppoint{gp mark 0}{(6.025,4.543)}
\gppoint{gp mark 0}{(7.062,5.438)}
\gppoint{gp mark 0}{(6.764,5.215)}
\gppoint{gp mark 0}{(6.082,4.633)}
\gppoint{gp mark 0}{(5.787,4.349)}
\gppoint{gp mark 0}{(5.834,4.397)}
\gppoint{gp mark 0}{(4.987,3.306)}
\gppoint{gp mark 0}{(5.282,3.661)}
\gppoint{gp mark 0}{(5.738,4.869)}
\gppoint{gp mark 0}{(5.244,3.716)}
\gppoint{gp mark 0}{(5.123,3.391)}
\gppoint{gp mark 0}{(4.938,2.980)}
\gppoint{gp mark 0}{(4.886,3.104)}
\gppoint{gp mark 0}{(5.034,3.211)}
\gppoint{gp mark 0}{(6.171,5.080)}
\gppoint{gp mark 0}{(5.079,3.467)}
\gppoint{gp mark 0}{(5.165,3.601)}
\gppoint{gp mark 0}{(6.332,5.173)}
\gppoint{gp mark 0}{(6.486,4.987)}
\gppoint{gp mark 0}{(6.988,5.581)}
\gppoint{gp mark 0}{(6.954,5.560)}
\gppoint{gp mark 0}{(5.879,4.441)}
\gppoint{gp mark 0}{(4.833,2.655)}
\gppoint{gp mark 0}{(6.690,5.386)}
\gppoint{gp mark 0}{(6.376,4.894)}
\gppoint{gp mark 0}{(9.770,6.539)}
\gppoint{gp mark 0}{(4.655,2.749)}
\gppoint{gp mark 0}{(5.282,3.349)}
\gppoint{gp mark 0}{(7.667,5.900)}
\gppoint{gp mark 0}{(6.082,4.514)}
\gppoint{gp mark 0}{(6.005,4.589)}
\gppoint{gp mark 0}{(5.079,3.570)}
\gppoint{gp mark 0}{(4.987,3.689)}
\gppoint{gp mark 0}{(6.844,5.320)}
\gppoint{gp mark 0}{(5.165,3.430)}
\gppoint{gp mark 0}{(5.205,3.503)}
\gppoint{gp mark 0}{(6.785,5.244)}
\gppoint{gp mark 0}{(4.589,3.160)}
\gppoint{gp mark 0}{(5.834,4.430)}
\gppoint{gp mark 0}{(4.717,2.911)}
\gppoint{gp mark 0}{(4.519,3.045)}
\gppoint{gp mark 0}{(5.034,3.743)}
\gppoint{gp mark 0}{(5.244,3.260)}
\gppoint{gp mark 0}{(6.025,4.607)}
\gppoint{gp mark 0}{(6.863,5.305)}
\gppoint{gp mark 0}{(5.123,3.632)}
\gppoint{gp mark 0}{(5.985,4.642)}
\gppoint{gp mark 0}{(7.286,5.649)}
\gppoint{gp mark 0}{(5.944,4.361)}
\gppoint{gp mark 0}{(5.964,4.625)}
\gppoint{gp mark 0}{(6.302,5.186)}
\gppoint{gp mark 0}{(7.362,5.668)}
\gppoint{gp mark 0}{(6.376,4.953)}
\gppoint{gp mark 0}{(5.576,4.856)}
\gppoint{gp mark 0}{(5.810,4.483)}
\gppoint{gp mark 0}{(5.282,3.306)}
\gppoint{gp mark 0}{(7.614,5.982)}
\gppoint{gp mark 0}{(4.987,3.661)}
\gppoint{gp mark 0}{(5.985,4.633)}
\gppoint{gp mark 0}{(6.785,5.240)}
\gppoint{gp mark 0}{(4.655,2.655)}
\gppoint{gp mark 0}{(5.964,4.616)}
\gppoint{gp mark 0}{(5.547,4.667)}
\gppoint{gp mark 0}{(4.519,2.980)}
\gppoint{gp mark 0}{(5.205,3.467)}
\gppoint{gp mark 0}{(5.165,3.391)}
\gppoint{gp mark 0}{(4.717,2.834)}
\gppoint{gp mark 0}{(5.605,4.869)}
\gppoint{gp mark 0}{(5.079,3.537)}
\gppoint{gp mark 0}{(5.787,4.463)}
\gppoint{gp mark 0}{(5.879,4.373)}
\gppoint{gp mark 0}{(4.589,3.104)}
\gppoint{gp mark 0}{(5.034,3.716)}
\gppoint{gp mark 0}{(5.123,3.601)}
\gppoint{gp mark 0}{(7.517,5.768)}
\gppoint{gp mark 0}{(5.244,3.211)}
\gppoint{gp mark 0}{(7.218,5.694)}
\gppoint{gp mark 0}{(8.356,6.405)}
\gppoint{gp mark 0}{(7.299,5.637)}
\gppoint{gp mark 0}{(6.238,5.070)}
\gppoint{gp mark 0}{(6.044,4.543)}
\gppoint{gp mark 0}{(6.882,5.298)}
\gppoint{gp mark 0}{(6.550,5.003)}
\gppoint{gp mark 0}{(6.082,4.533)}
\gppoint{gp mark 0}{(7.754,6.088)}
\gppoint{gp mark 0}{(5.079,3.632)}
\gppoint{gp mark 0}{(5.165,3.503)}
\gppoint{gp mark 0}{(4.519,3.160)}
\gppoint{gp mark 0}{(7.433,5.885)}
\gppoint{gp mark 0}{(4.655,2.911)}
\gppoint{gp mark 0}{(5.123,3.570)}
\gppoint{gp mark 0}{(5.244,3.349)}
\gppoint{gp mark 0}{(4.589,3.045)}
\gppoint{gp mark 0}{(5.205,3.430)}
\gppoint{gp mark 0}{(5.282,3.260)}
\gppoint{gp mark 0}{(4.987,3.743)}
\gppoint{gp mark 0}{(5.787,4.494)}
\gppoint{gp mark 0}{(6.119,5.132)}
\gppoint{gp mark 0}{(5.034,3.689)}
\gppoint{gp mark 0}{(5.810,4.473)}
\gppoint{gp mark 0}{(4.717,2.749)}
\gppoint{gp mark 0}{(6.238,5.065)}
\gppoint{gp mark 0}{(6.390,4.959)}
\gppoint{gp mark 0}{(5.923,4.361)}
\gppoint{gp mark 0}{(6.317,5.168)}
\gppoint{gp mark 0}{(6.063,4.543)}
\gppoint{gp mark 0}{(7.410,5.839)}
\gppoint{gp mark 0}{(5.576,4.869)}
\gppoint{gp mark 0}{(5.205,3.391)}
\gppoint{gp mark 0}{(5.244,3.306)}
\gppoint{gp mark 0}{(5.165,3.467)}
\gppoint{gp mark 0}{(5.282,3.211)}
\gppoint{gp mark 0}{(5.034,3.661)}
\gppoint{gp mark 0}{(4.655,2.834)}
\gppoint{gp mark 0}{(5.079,3.601)}
\gppoint{gp mark 0}{(6.834,5.207)}
\gppoint{gp mark 0}{(4.589,2.980)}
\gppoint{gp mark 0}{(4.519,3.104)}
\gppoint{gp mark 0}{(6.044,4.562)}
\gppoint{gp mark 0}{(5.123,3.537)}
\gppoint{gp mark 0}{(6.082,4.524)}
\gppoint{gp mark 0}{(4.987,3.716)}
\gppoint{gp mark 0}{(4.717,2.655)}
\gppoint{gp mark 0}{(5.455,4.699)}
\gppoint{gp mark 0}{(6.390,4.953)}
\gppoint{gp mark 0}{(6.154,5.109)}
\gppoint{gp mark 0}{(6.679,5.432)}
\gppoint{gp mark 0}{(6.100,4.504)}
\gppoint{gp mark 0}{(5.834,4.441)}
\gppoint{gp mark 0}{(6.419,4.930)}
\gppoint{gp mark 0}{(8.018,6.233)}
\gppoint{gp mark 0}{(6.270,5.186)}
\gppoint{gp mark 0}{(5.123,3.743)}
\gppoint{gp mark 0}{(5.879,4.337)}
\gppoint{gp mark 0}{(6.044,4.514)}
\gppoint{gp mark 0}{(5.762,4.823)}
\gppoint{gp mark 0}{(4.717,3.160)}
\gppoint{gp mark 0}{(5.964,4.589)}
\gppoint{gp mark 0}{(4.987,3.570)}
\gppoint{gp mark 0}{(5.165,3.260)}
\gppoint{gp mark 0}{(5.244,3.430)}
\gppoint{gp mark 0}{(4.655,3.045)}
\gppoint{gp mark 0}{(4.589,2.911)}
\gppoint{gp mark 0}{(5.034,3.632)}
\gppoint{gp mark 0}{(5.205,3.349)}
\gppoint{gp mark 0}{(4.519,2.749)}
\gppoint{gp mark 0}{(5.282,3.503)}
\gppoint{gp mark 0}{(5.079,3.689)}
\gppoint{gp mark 0}{(6.473,4.924)}
\gppoint{gp mark 0}{(7.672,5.896)}
\gppoint{gp mark 0}{(6.082,4.552)}
\gppoint{gp mark 0}{(4.938,2.548)}
\gppoint{gp mark 0}{(6.537,4.981)}
\gppoint{gp mark 0}{(6.499,5.035)}
\gppoint{gp mark 0}{(6.063,4.533)}
\gppoint{gp mark 0}{(6.754,5.409)}
\gppoint{gp mark 0}{(7.169,5.507)}
\gppoint{gp mark 0}{(6.025,4.633)}
\gppoint{gp mark 0}{(5.901,4.349)}
\gppoint{gp mark 0}{(5.576,4.829)}
\gppoint{gp mark 0}{(7.404,5.854)}
\gppoint{gp mark 0}{(5.034,3.601)}
\gppoint{gp mark 0}{(4.519,2.655)}
\gppoint{gp mark 0}{(5.282,3.467)}
\gppoint{gp mark 0}{(4.655,2.980)}
\gppoint{gp mark 0}{(4.987,3.537)}
\gppoint{gp mark 0}{(5.079,3.661)}
\gppoint{gp mark 0}{(4.589,2.834)}
\gppoint{gp mark 0}{(5.165,3.211)}
\gppoint{gp mark 0}{(4.717,3.104)}
\gppoint{gp mark 0}{(5.244,3.391)}
\gppoint{gp mark 0}{(5.205,3.306)}
\gppoint{gp mark 0}{(7.022,5.576)}
\gppoint{gp mark 0}{(7.132,5.521)}
\gppoint{gp mark 0}{(6.332,5.146)}
\gppoint{gp mark 0}{(6.499,5.030)}
\gppoint{gp mark 0}{(6.082,4.543)}
\gppoint{gp mark 0}{(6.562,4.998)}
\gppoint{gp mark 0}{(6.963,5.538)}
\gppoint{gp mark 0}{(5.944,4.397)}
\gppoint{gp mark 0}{(5.123,3.716)}
\gppoint{gp mark 0}{(5.787,4.419)}
\gppoint{gp mark 0}{(5.879,4.361)}
\gppoint{gp mark 0}{(6.025,4.625)}
\gppoint{gp mark 0}{(6.332,5.142)}
\gppoint{gp mark 0}{(6.063,4.514)}
\gppoint{gp mark 0}{(4.886,2.548)}
\gppoint{gp mark 0}{(5.123,3.689)}
\gppoint{gp mark 0}{(5.205,3.260)}
\gppoint{gp mark 0}{(5.923,4.408)}
\gppoint{gp mark 0}{(6.512,5.055)}
\gppoint{gp mark 0}{(5.079,3.743)}
\gppoint{gp mark 0}{(4.717,3.045)}
\gppoint{gp mark 0}{(4.589,2.749)}
\gppoint{gp mark 0}{(4.519,2.911)}
\gppoint{gp mark 0}{(5.034,3.570)}
\gppoint{gp mark 0}{(6.082,4.571)}
\gppoint{gp mark 0}{(5.244,3.503)}
\gppoint{gp mark 0}{(5.282,3.430)}
\gppoint{gp mark 0}{(4.655,3.160)}
\gppoint{gp mark 0}{(5.787,4.452)}
\gppoint{gp mark 0}{(6.100,4.552)}
\gppoint{gp mark 0}{(6.044,4.533)}
\gppoint{gp mark 0}{(5.165,3.349)}
\gppoint{gp mark 0}{(6.302,5.194)}
\gppoint{gp mark 0}{(7.218,5.709)}
\gppoint{gp mark 0}{(5.834,4.494)}
\gppoint{gp mark 0}{(4.987,3.632)}
\gppoint{gp mark 0}{(7.161,5.507)}
\gppoint{gp mark 0}{(7.433,5.878)}
\gppoint{gp mark 0}{(7.473,5.867)}
\gppoint{gp mark 0}{(5.079,3.716)}
\gppoint{gp mark 0}{(4.987,3.601)}
\gppoint{gp mark 0}{(5.857,4.463)}
\gppoint{gp mark 0}{(6.063,4.504)}
\gppoint{gp mark 0}{(5.547,4.683)}
\gppoint{gp mark 0}{(5.165,3.306)}
\gppoint{gp mark 0}{(4.519,2.834)}
\gppoint{gp mark 0}{(4.717,2.980)}
\gppoint{gp mark 0}{(5.282,3.391)}
\gppoint{gp mark 0}{(6.405,4.965)}
\gppoint{gp mark 0}{(7.038,5.592)}
\gppoint{gp mark 0}{(4.655,3.104)}
\gppoint{gp mark 0}{(4.589,2.655)}
\gppoint{gp mark 0}{(5.123,3.661)}
\gppoint{gp mark 0}{(5.244,3.467)}
\gppoint{gp mark 0}{(5.205,3.211)}
\gppoint{gp mark 0}{(5.834,4.483)}
\gppoint{gp mark 0}{(5.034,3.537)}
\gppoint{gp mark 0}{(5.879,4.349)}
\gppoint{gp mark 0}{(6.044,4.524)}
\gppoint{gp mark 0}{(7.046,5.586)}
\gppoint{gp mark 0}{(5.944,4.373)}
\gppoint{gp mark 0}{(6.701,5.432)}
\gppoint{gp mark 0}{(7.022,5.581)}
\gppoint{gp mark 0}{(5.964,4.598)}
\gppoint{gp mark 0}{(5.923,4.397)}
\gppoint{gp mark 0}{(6.754,5.399)}
\gppoint{gp mark 0}{(6.815,5.207)}
\gppoint{gp mark 0}{(7.398,5.854)}
\gppoint{gp mark 0}{(6.473,4.906)}
\gppoint{gp mark 0}{(6.743,5.406)}
\gppoint{gp mark 0}{(6.900,5.294)}
\gppoint{gp mark 0}{(5.738,4.722)}
\gppoint{gp mark 0}{(5.547,4.849)}
\gppoint{gp mark 0}{(6.376,5.003)}
\gppoint{gp mark 0}{(7.013,5.541)}
\gppoint{gp mark 0}{(5.389,4.781)}
\gppoint{gp mark 0}{(7.124,5.459)}
\gppoint{gp mark 0}{(4.589,3.503)}
\gppoint{gp mark 0}{(5.165,3.045)}
\gppoint{gp mark 0}{(4.886,3.570)}
\gppoint{gp mark 0}{(4.938,3.632)}
\gppoint{gp mark 0}{(6.362,5.114)}
\gppoint{gp mark 0}{(5.810,4.571)}
\gppoint{gp mark 0}{(4.777,3.689)}
\gppoint{gp mark 0}{(4.519,3.430)}
\gppoint{gp mark 0}{(4.655,3.260)}
\gppoint{gp mark 0}{(5.205,3.160)}
\gppoint{gp mark 0}{(4.717,3.349)}
\gppoint{gp mark 0}{(6.302,5.075)}
\gppoint{gp mark 0}{(6.005,4.337)}
\gppoint{gp mark 0}{(6.082,4.430)}
\gppoint{gp mark 0}{(5.282,2.911)}
\gppoint{gp mark 0}{(7.564,5.977)}
\gppoint{gp mark 0}{(7.574,5.971)}
\gppoint{gp mark 0}{(6.844,5.227)}
\gppoint{gp mark 0}{(4.833,3.743)}
\gppoint{gp mark 0}{(6.499,4.924)}
\gppoint{gp mark 0}{(5.879,4.625)}
\gppoint{gp mark 0}{(7.691,5.889)}
\gppoint{gp mark 0}{(6.044,4.463)}
\gppoint{gp mark 0}{(5.923,4.580)}
\gppoint{gp mark 0}{(5.205,3.104)}
\gppoint{gp mark 0}{(6.025,4.349)}
\gppoint{gp mark 0}{(4.589,3.467)}
\gppoint{gp mark 0}{(6.586,5.399)}
\gppoint{gp mark 0}{(5.810,4.562)}
\gppoint{gp mark 0}{(4.519,3.391)}
\gppoint{gp mark 0}{(4.938,3.601)}
\gppoint{gp mark 0}{(4.655,3.211)}
\gppoint{gp mark 0}{(6.390,5.009)}
\gppoint{gp mark 0}{(4.717,3.306)}
\gppoint{gp mark 0}{(4.886,3.537)}
\gppoint{gp mark 0}{(6.376,4.998)}
\gppoint{gp mark 0}{(6.082,4.419)}
\gppoint{gp mark 0}{(6.100,4.441)}
\gppoint{gp mark 0}{(7.154,5.481)}
\gppoint{gp mark 0}{(7.324,5.675)}
\gppoint{gp mark 0}{(6.863,5.207)}
\gppoint{gp mark 0}{(5.165,2.980)}
\gppoint{gp mark 0}{(6.945,5.581)}
\gppoint{gp mark 0}{(6.574,4.936)}
\gppoint{gp mark 0}{(7.943,6.123)}
\gppoint{gp mark 0}{(6.286,5.075)}
\gppoint{gp mark 0}{(4.519,3.503)}
\gppoint{gp mark 0}{(6.459,5.035)}
\gppoint{gp mark 0}{(5.244,2.911)}
\gppoint{gp mark 0}{(6.025,4.337)}
\gppoint{gp mark 0}{(6.005,4.361)}
\gppoint{gp mark 0}{(5.605,4.691)}
\gppoint{gp mark 0}{(4.589,3.430)}
\gppoint{gp mark 0}{(4.655,3.349)}
\gppoint{gp mark 0}{(4.717,3.260)}
\gppoint{gp mark 0}{(6.971,5.604)}
\gppoint{gp mark 0}{(6.824,5.312)}
\gppoint{gp mark 0}{(4.833,3.689)}
\gppoint{gp mark 0}{(4.777,3.743)}
\gppoint{gp mark 0}{(6.853,5.227)}
\gppoint{gp mark 0}{(5.810,4.552)}
\gppoint{gp mark 0}{(6.189,5.203)}
\gppoint{gp mark 0}{(5.165,3.160)}
\gppoint{gp mark 0}{(4.938,3.570)}
\gppoint{gp mark 0}{(7.239,5.705)}
\gppoint{gp mark 0}{(4.886,3.632)}
\gppoint{gp mark 0}{(5.712,4.752)}
\gppoint{gp mark 0}{(6.390,5.003)}
\gppoint{gp mark 0}{(6.775,5.290)}
\gppoint{gp mark 0}{(4.833,3.661)}
\gppoint{gp mark 0}{(4.655,3.306)}
\gppoint{gp mark 0}{(6.805,5.323)}
\gppoint{gp mark 0}{(6.743,5.366)}
\gppoint{gp mark 0}{(4.589,3.391)}
\gppoint{gp mark 0}{(4.717,3.211)}
\gppoint{gp mark 0}{(4.777,3.716)}
\gppoint{gp mark 0}{(5.762,4.714)}
\gppoint{gp mark 0}{(6.362,5.100)}
\gppoint{gp mark 0}{(4.886,3.601)}
\gppoint{gp mark 0}{(4.938,3.537)}
\gppoint{gp mark 0}{(5.985,4.373)}
\gppoint{gp mark 0}{(4.519,3.467)}
\gppoint{gp mark 0}{(5.165,3.104)}
\gppoint{gp mark 0}{(5.901,4.616)}
\gppoint{gp mark 0}{(5.205,2.980)}
\gppoint{gp mark 0}{(5.834,4.524)}
\gppoint{gp mark 0}{(7.511,5.816)}
\gppoint{gp mark 0}{(7.062,5.498)}
\gppoint{gp mark 0}{(5.455,4.836)}
\gppoint{gp mark 0}{(6.918,5.578)}
\gppoint{gp mark 0}{(4.886,3.689)}
\gppoint{gp mark 0}{(4.717,3.503)}
\gppoint{gp mark 0}{(6.063,4.452)}
\gppoint{gp mark 0}{(4.777,3.570)}
\gppoint{gp mark 0}{(4.655,3.430)}
\gppoint{gp mark 0}{(4.519,3.260)}
\gppoint{gp mark 0}{(4.938,3.743)}
\gppoint{gp mark 0}{(4.589,3.349)}
\gppoint{gp mark 0}{(5.686,4.722)}
\gppoint{gp mark 0}{(5.282,3.160)}
\gppoint{gp mark 0}{(6.317,5.104)}
\gppoint{gp mark 0}{(6.499,4.900)}
\gppoint{gp mark 0}{(6.005,4.385)}
\gppoint{gp mark 0}{(5.244,3.045)}
\gppoint{gp mark 0}{(6.550,4.947)}
\gppoint{gp mark 0}{(5.901,4.607)}
\gppoint{gp mark 0}{(4.833,3.632)}
\gppoint{gp mark 0}{(6.082,4.473)}
\gppoint{gp mark 0}{(6.025,4.408)}
\gppoint{gp mark 0}{(6.815,5.312)}
\gppoint{gp mark 0}{(5.923,4.625)}
\gppoint{gp mark 0}{(6.405,4.998)}
\gppoint{gp mark 0}{(5.165,2.655)}
\gppoint{gp mark 0}{(7.714,5.912)}
\gppoint{gp mark 0}{(4.938,3.716)}
\gppoint{gp mark 0}{(4.886,3.661)}
\gppoint{gp mark 0}{(4.655,3.391)}
\gppoint{gp mark 0}{(4.519,3.211)}
\gppoint{gp mark 0}{(4.777,3.537)}
\gppoint{gp mark 0}{(5.660,4.699)}
\gppoint{gp mark 0}{(4.717,3.467)}
\gppoint{gp mark 0}{(4.589,3.306)}
\gppoint{gp mark 0}{(4.833,3.601)}
\gppoint{gp mark 0}{(6.764,5.271)}
\gppoint{gp mark 0}{(6.390,4.987)}
\gppoint{gp mark 0}{(5.810,4.524)}
\gppoint{gp mark 0}{(5.964,4.325)}
\gppoint{gp mark 0}{(7.356,5.632)}
\gppoint{gp mark 0}{(6.100,4.483)}
\gppoint{gp mark 0}{(5.879,4.580)}
\gppoint{gp mark 0}{(5.787,4.504)}
\gppoint{gp mark 0}{(4.655,3.503)}
\gppoint{gp mark 0}{(4.833,3.570)}
\gppoint{gp mark 0}{(6.063,4.430)}
\gppoint{gp mark 0}{(4.777,3.632)}
\gppoint{gp mark 0}{(4.589,3.260)}
\gppoint{gp mark 0}{(4.717,3.430)}
\gppoint{gp mark 0}{(4.519,3.349)}
\gppoint{gp mark 0}{(5.244,3.160)}
\gppoint{gp mark 0}{(5.282,3.045)}
\gppoint{gp mark 0}{(6.390,4.981)}
\gppoint{gp mark 0}{(4.886,3.743)}
\gppoint{gp mark 0}{(7.169,5.478)}
\gppoint{gp mark 0}{(5.985,4.337)}
\gppoint{gp mark 0}{(6.882,5.252)}
\gppoint{gp mark 0}{(6.100,4.473)}
\gppoint{gp mark 0}{(6.537,4.947)}
\gppoint{gp mark 0}{(5.964,4.361)}
\gppoint{gp mark 0}{(6.872,5.227)}
\gppoint{gp mark 0}{(6.082,4.494)}
\gppoint{gp mark 0}{(4.938,3.689)}
\gppoint{gp mark 0}{(6.891,5.244)}
\gppoint{gp mark 0}{(6.610,5.406)}
\gppoint{gp mark 0}{(4.777,3.601)}
\gppoint{gp mark 0}{(5.901,4.580)}
\gppoint{gp mark 0}{(4.655,3.467)}
\gppoint{gp mark 0}{(6.891,5.240)}
\gppoint{gp mark 0}{(4.717,3.391)}
\gppoint{gp mark 0}{(4.886,3.716)}
\gppoint{gp mark 0}{(4.833,3.537)}
\gppoint{gp mark 0}{(6.405,5.009)}
\gppoint{gp mark 0}{(5.923,4.633)}
\gppoint{gp mark 0}{(7.109,5.521)}
\gppoint{gp mark 0}{(4.519,3.306)}
\gppoint{gp mark 0}{(6.005,4.397)}
\gppoint{gp mark 0}{(4.589,3.211)}
\gppoint{gp mark 0}{(6.189,5.181)}
\gppoint{gp mark 0}{(7.954,6.097)}
\gppoint{gp mark 0}{(7.350,5.632)}
\gppoint{gp mark 0}{(7.433,5.832)}
\gppoint{gp mark 0}{(6.512,4.918)}
\gppoint{gp mark 0}{(5.857,4.543)}
\gppoint{gp mark 0}{(6.882,5.248)}
\gppoint{gp mark 0}{(4.938,3.661)}
\gppoint{gp mark 0}{(7.030,5.554)}
\gppoint{gp mark 0}{(7.078,5.498)}
\gppoint{gp mark 0}{(7.046,5.565)}
\gppoint{gp mark 0}{(6.082,4.337)}
\gppoint{gp mark 0}{(5.079,2.749)}
\gppoint{gp mark 0}{(4.833,3.503)}
\gppoint{gp mark 0}{(4.717,3.632)}
\gppoint{gp mark 0}{(4.938,3.349)}
\gppoint{gp mark 0}{(4.777,3.430)}
\gppoint{gp mark 0}{(7.005,5.557)}
\gppoint{gp mark 0}{(4.987,3.045)}
\gppoint{gp mark 0}{(4.886,3.260)}
\gppoint{gp mark 0}{(5.034,3.160)}
\gppoint{gp mark 0}{(6.044,4.385)}
\gppoint{gp mark 0}{(6.025,4.452)}
\gppoint{gp mark 0}{(6.100,4.361)}
\gppoint{gp mark 0}{(6.459,4.981)}
\gppoint{gp mark 0}{(7.094,5.507)}
\gppoint{gp mark 0}{(7.380,5.625)}
\gppoint{gp mark 0}{(4.655,3.570)}
\gppoint{gp mark 0}{(5.879,4.552)}
\gppoint{gp mark 0}{(6.550,4.924)}
\gppoint{gp mark 0}{(6.136,5.203)}
\gppoint{gp mark 0}{(7.439,5.856)}
\gppoint{gp mark 0}{(5.787,4.625)}
\gppoint{gp mark 0}{(6.795,5.312)}
\gppoint{gp mark 0}{(6.815,5.298)}
\gppoint{gp mark 0}{(7.805,6.025)}
\gppoint{gp mark 0}{(6.005,4.419)}
\gppoint{gp mark 0}{(6.432,4.998)}
\gppoint{gp mark 0}{(6.082,4.325)}
\gppoint{gp mark 0}{(4.655,3.537)}
\gppoint{gp mark 0}{(5.034,3.104)}
\gppoint{gp mark 0}{(6.882,5.223)}
\gppoint{gp mark 0}{(6.189,5.155)}
\gppoint{gp mark 0}{(4.886,3.211)}
\gppoint{gp mark 0}{(4.833,3.467)}
\gppoint{gp mark 0}{(4.777,3.391)}
\gppoint{gp mark 0}{(4.938,3.306)}
\gppoint{gp mark 0}{(6.044,4.373)}
\gppoint{gp mark 0}{(4.987,2.980)}
\gppoint{gp mark 0}{(6.100,4.349)}
\gppoint{gp mark 0}{(7.517,5.800)}
\gppoint{gp mark 0}{(4.589,3.716)}
\gppoint{gp mark 0}{(5.964,4.463)}
\gppoint{gp mark 0}{(4.519,3.661)}
\gppoint{gp mark 0}{(6.668,5.393)}
\gppoint{gp mark 0}{(7.718,6.088)}
\gppoint{gp mark 0}{(6.473,4.981)}
\gppoint{gp mark 0}{(5.633,4.737)}
\gppoint{gp mark 0}{(5.034,3.045)}
\gppoint{gp mark 0}{(4.655,3.632)}
\gppoint{gp mark 0}{(4.938,3.260)}
\gppoint{gp mark 0}{(4.833,3.430)}
\gppoint{gp mark 0}{(4.886,3.349)}
\gppoint{gp mark 0}{(4.987,3.160)}
\gppoint{gp mark 0}{(4.717,3.570)}
\gppoint{gp mark 0}{(6.795,5.305)}
\gppoint{gp mark 0}{(7.609,5.964)}
\gppoint{gp mark 0}{(6.834,5.275)}
\gppoint{gp mark 0}{(4.777,3.503)}
\gppoint{gp mark 0}{(5.857,4.589)}
\gppoint{gp mark 0}{(7.046,5.535)}
\gppoint{gp mark 0}{(6.446,5.003)}
\gppoint{gp mark 0}{(6.598,5.428)}
\gppoint{gp mark 0}{(7.600,5.971)}
\gppoint{gp mark 0}{(5.810,4.625)}
\gppoint{gp mark 0}{(5.834,4.607)}
\gppoint{gp mark 0}{(6.459,4.992)}
\gppoint{gp mark 0}{(7.246,5.718)}
\gppoint{gp mark 0}{(4.519,3.743)}
\gppoint{gp mark 0}{(6.891,5.223)}
\gppoint{gp mark 0}{(4.987,3.104)}
\gppoint{gp mark 0}{(4.777,3.467)}
\gppoint{gp mark 0}{(4.655,3.601)}
\gppoint{gp mark 0}{(6.419,5.019)}
\gppoint{gp mark 0}{(4.886,3.306)}
\gppoint{gp mark 0}{(4.833,3.391)}
\gppoint{gp mark 0}{(4.717,3.537)}
\gppoint{gp mark 0}{(6.459,4.987)}
\gppoint{gp mark 0}{(4.938,3.211)}
\gppoint{gp mark 0}{(5.517,4.788)}
\gppoint{gp mark 0}{(5.633,4.730)}
\gppoint{gp mark 0}{(6.044,4.397)}
\gppoint{gp mark 0}{(6.512,4.942)}
\gppoint{gp mark 0}{(4.445,3.768)}
\gppoint{gp mark 0}{(5.944,4.504)}
\gppoint{gp mark 0}{(5.034,2.980)}
\gppoint{gp mark 0}{(6.945,5.597)}
\gppoint{gp mark 0}{(6.025,4.419)}
\gppoint{gp mark 0}{(6.525,4.930)}
\gppoint{gp mark 0}{(6.100,4.325)}
\gppoint{gp mark 0}{(5.123,3.160)}
\gppoint{gp mark 0}{(4.655,3.689)}
\gppoint{gp mark 0}{(4.589,3.632)}
\gppoint{gp mark 0}{(6.063,4.361)}
\gppoint{gp mark 0}{(5.810,4.607)}
\gppoint{gp mark 0}{(5.034,2.911)}
\gppoint{gp mark 0}{(5.079,3.045)}
\gppoint{gp mark 0}{(4.886,3.430)}
\gppoint{gp mark 0}{(4.777,3.260)}
\gppoint{gp mark 0}{(6.171,5.203)}
\gppoint{gp mark 0}{(4.833,3.349)}
\gppoint{gp mark 0}{(4.938,3.503)}
\gppoint{gp mark 0}{(5.787,4.589)}
\gppoint{gp mark 0}{(4.717,3.743)}
\gppoint{gp mark 0}{(5.964,4.430)}
\gppoint{gp mark 0}{(7.147,5.441)}
\gppoint{gp mark 0}{(6.044,4.337)}
\gppoint{gp mark 0}{(5.282,2.548)}
\gppoint{gp mark 0}{(5.762,4.707)}
\gppoint{gp mark 0}{(4.445,3.927)}
\gppoint{gp mark 0}{(6.936,5.609)}
\gppoint{gp mark 0}{(6.419,5.055)}
\gppoint{gp mark 0}{(7.030,5.541)}
\gppoint{gp mark 0}{(6.376,5.024)}
\gppoint{gp mark 0}{(7.767,6.064)}
\gppoint{gp mark 0}{(7.197,5.759)}
\gppoint{gp mark 0}{(5.857,4.633)}
\gppoint{gp mark 0}{(4.445,3.906)}
\gppoint{gp mark 0}{(5.901,4.524)}
\gppoint{gp mark 0}{(6.562,4.906)}
\gppoint{gp mark 0}{(5.123,3.104)}
\gppoint{gp mark 0}{(6.082,4.373)}
\gppoint{gp mark 0}{(4.886,3.391)}
\gppoint{gp mark 0}{(6.063,4.349)}
\gppoint{gp mark 0}{(4.717,3.716)}
\gppoint{gp mark 0}{(4.938,3.467)}
\gppoint{gp mark 0}{(4.777,3.211)}
\gppoint{gp mark 0}{(4.833,3.306)}
\gppoint{gp mark 0}{(4.655,3.661)}
\gppoint{gp mark 0}{(4.519,3.537)}
\gppoint{gp mark 0}{(5.985,4.441)}
\gppoint{gp mark 0}{(6.005,4.463)}
\gppoint{gp mark 0}{(6.432,4.976)}
\gppoint{gp mark 0}{(6.754,5.352)}
\gppoint{gp mark 0}{(7.467,5.835)}
\gppoint{gp mark 0}{(4.589,3.601)}
\gppoint{gp mark 0}{(5.964,4.419)}
\gppoint{gp mark 0}{(6.100,4.397)}
\gppoint{gp mark 0}{(5.762,4.699)}
\gppoint{gp mark 0}{(5.034,2.834)}
\gppoint{gp mark 0}{(5.079,2.980)}
\gppoint{gp mark 0}{(6.882,5.219)}
\gppoint{gp mark 0}{(4.589,3.570)}
\gppoint{gp mark 0}{(5.605,4.722)}
\gppoint{gp mark 0}{(5.079,3.160)}
\gppoint{gp mark 0}{(5.547,4.809)}
\gppoint{gp mark 0}{(5.354,4.836)}
\gppoint{gp mark 0}{(7.775,6.045)}
\gppoint{gp mark 0}{(5.901,4.514)}
\gppoint{gp mark 0}{(7.176,5.753)}
\gppoint{gp mark 0}{(4.987,2.911)}
\gppoint{gp mark 0}{(6.775,5.305)}
\gppoint{gp mark 0}{(4.777,3.349)}
\gppoint{gp mark 0}{(4.833,3.260)}
\gppoint{gp mark 0}{(6.645,5.389)}
\gppoint{gp mark 0}{(4.938,3.430)}
\gppoint{gp mark 0}{(4.717,3.689)}
\gppoint{gp mark 0}{(4.886,3.503)}
\gppoint{gp mark 0}{(6.499,4.936)}
\gppoint{gp mark 0}{(5.123,3.045)}
\gppoint{gp mark 0}{(5.944,4.552)}
\gppoint{gp mark 0}{(6.005,4.494)}
\gppoint{gp mark 0}{(7.605,5.961)}
\gppoint{gp mark 0}{(7.169,5.453)}
\gppoint{gp mark 0}{(6.063,4.337)}
\gppoint{gp mark 0}{(6.082,4.408)}
\gppoint{gp mark 0}{(4.655,3.743)}
\gppoint{gp mark 0}{(4.519,3.632)}
\gppoint{gp mark 0}{(6.824,5.298)}
\gppoint{gp mark 0}{(5.633,4.767)}
\gppoint{gp mark 0}{(6.025,4.463)}
\gppoint{gp mark 0}{(5.985,4.419)}
\gppoint{gp mark 0}{(6.100,4.373)}
\gppoint{gp mark 0}{(4.589,3.537)}
\gppoint{gp mark 0}{(6.562,4.918)}
\gppoint{gp mark 0}{(4.833,3.211)}
\gppoint{gp mark 0}{(4.886,3.467)}
\gppoint{gp mark 0}{(5.123,2.980)}
\gppoint{gp mark 0}{(4.938,3.391)}
\gppoint{gp mark 0}{(4.777,3.306)}
\gppoint{gp mark 0}{(7.190,5.759)}
\gppoint{gp mark 0}{(5.944,4.543)}
\gppoint{gp mark 0}{(5.079,3.104)}
\gppoint{gp mark 0}{(6.063,4.325)}
\gppoint{gp mark 0}{(6.044,4.349)}
\gppoint{gp mark 0}{(7.792,6.031)}
\gppoint{gp mark 0}{(6.722,5.338)}
\gppoint{gp mark 0}{(7.211,5.729)}
\gppoint{gp mark 0}{(6.525,4.953)}
\gppoint{gp mark 0}{(5.712,4.650)}
\gppoint{gp mark 0}{(4.655,3.716)}
\gppoint{gp mark 0}{(4.717,3.661)}
\gppoint{gp mark 0}{(4.445,3.863)}
\gppoint{gp mark 0}{(6.005,4.483)}
\gppoint{gp mark 0}{(5.857,3.840)}
\gppoint{gp mark 0}{(5.576,3.430)}
\gppoint{gp mark 0}{(5.834,3.793)}
\gppoint{gp mark 0}{(5.605,3.503)}
\gppoint{gp mark 0}{(5.738,3.570)}
\gppoint{gp mark 0}{(5.686,3.689)}
\gppoint{gp mark 0}{(5.923,3.967)}
\gppoint{gp mark 0}{(5.985,4.203)}
\gppoint{gp mark 0}{(5.762,3.632)}
\gppoint{gp mark 0}{(6.005,4.110)}
\gppoint{gp mark 0}{(6.486,5.403)}
\gppoint{gp mark 0}{(5.810,3.927)}
\gppoint{gp mark 0}{(5.712,3.743)}
\gppoint{gp mark 0}{(5.787,3.885)}
\gppoint{gp mark 0}{(5.901,4.077)}
\gppoint{gp mark 0}{(6.701,4.981)}
\gppoint{gp mark 0}{(5.944,4.005)}
\gppoint{gp mark 0}{(6.025,4.142)}
\gppoint{gp mark 0}{(6.882,5.194)}
\gppoint{gp mark 0}{(7.495,5.948)}
\gppoint{gp mark 0}{(6.082,4.232)}
\gppoint{gp mark 0}{(6.634,4.959)}
\gppoint{gp mark 0}{(7.109,6.793)}
\gppoint{gp mark 0}{(5.686,3.661)}
\gppoint{gp mark 0}{(5.964,4.158)}
\gppoint{gp mark 0}{(7.398,5.887)}
\gppoint{gp mark 0}{(6.634,4.953)}
\gppoint{gp mark 0}{(5.712,3.716)}
\gppoint{gp mark 0}{(6.082,4.217)}
\gppoint{gp mark 0}{(5.944,3.986)}
\gppoint{gp mark 0}{(5.834,3.768)}
\gppoint{gp mark 0}{(5.901,4.059)}
\gppoint{gp mark 0}{(7.767,6.114)}
\gppoint{gp mark 0}{(6.044,4.273)}
\gppoint{gp mark 0}{(5.923,3.947)}
\gppoint{gp mark 0}{(5.605,3.467)}
\gppoint{gp mark 0}{(5.810,3.906)}
\gppoint{gp mark 0}{(5.787,3.863)}
\gppoint{gp mark 0}{(5.455,2.980)}
\gppoint{gp mark 0}{(6.668,4.942)}
\gppoint{gp mark 0}{(6.005,4.094)}
\gppoint{gp mark 0}{(5.738,3.537)}
\gppoint{gp mark 0}{(5.762,3.601)}
\gppoint{gp mark 0}{(7.467,5.929)}
\gppoint{gp mark 0}{(7.538,5.971)}
\gppoint{gp mark 0}{(5.834,3.840)}
\gppoint{gp mark 0}{(6.005,4.142)}
\gppoint{gp mark 0}{(5.964,4.203)}
\gppoint{gp mark 0}{(6.044,4.312)}
\gppoint{gp mark 0}{(5.762,3.570)}
\gppoint{gp mark 0}{(7.305,5.535)}
\gppoint{gp mark 0}{(5.712,3.689)}
\gppoint{gp mark 0}{(6.764,5.095)}
\gppoint{gp mark 0}{(5.810,3.885)}
\gppoint{gp mark 0}{(5.944,3.967)}
\gppoint{gp mark 0}{(5.633,3.349)}
\gppoint{gp mark 0}{(5.787,3.927)}
\gppoint{gp mark 0}{(5.576,3.503)}
\gppoint{gp mark 0}{(7.279,5.513)}
\gppoint{gp mark 0}{(5.901,4.042)}
\gppoint{gp mark 0}{(5.738,3.632)}
\gppoint{gp mark 0}{(5.879,4.077)}
\gppoint{gp mark 0}{(6.795,5.065)}
\gppoint{gp mark 0}{(6.025,4.110)}
\gppoint{gp mark 0}{(5.605,3.430)}
\gppoint{gp mark 0}{(5.857,3.793)}
\gppoint{gp mark 0}{(5.985,4.173)}
\gppoint{gp mark 0}{(6.082,4.259)}
\gppoint{gp mark 0}{(5.686,3.743)}
\gppoint{gp mark 0}{(7.495,5.944)}
\gppoint{gp mark 0}{(5.660,3.260)}
\gppoint{gp mark 0}{(5.762,3.537)}
\gppoint{gp mark 0}{(5.605,3.391)}
\gppoint{gp mark 0}{(6.082,4.246)}
\gppoint{gp mark 0}{(5.686,3.716)}
\gppoint{gp mark 0}{(5.633,3.306)}
\gppoint{gp mark 0}{(6.100,4.217)}
\gppoint{gp mark 0}{(5.985,4.158)}
\gppoint{gp mark 0}{(5.787,3.906)}
\gppoint{gp mark 0}{(5.923,3.986)}
\gppoint{gp mark 0}{(6.044,4.299)}
\gppoint{gp mark 0}{(6.701,4.987)}
\gppoint{gp mark 0}{(5.738,3.601)}
\gppoint{gp mark 0}{(5.834,3.817)}
\gppoint{gp mark 0}{(6.005,4.126)}
\gppoint{gp mark 0}{(5.964,4.188)}
\gppoint{gp mark 0}{(5.810,3.863)}
\gppoint{gp mark 0}{(6.656,4.942)}
\gppoint{gp mark 0}{(5.576,3.467)}
\gppoint{gp mark 0}{(5.712,3.661)}
\gppoint{gp mark 0}{(5.857,3.768)}
\gppoint{gp mark 0}{(7.584,5.792)}
\gppoint{gp mark 0}{(6.754,5.019)}
\gppoint{gp mark 0}{(7.826,6.158)}
\gppoint{gp mark 0}{(6.025,4.094)}
\gppoint{gp mark 0}{(5.879,4.059)}
\gppoint{gp mark 0}{(6.834,5.100)}
\gppoint{gp mark 0}{(6.610,4.894)}
\gppoint{gp mark 0}{(7.232,5.504)}
\gppoint{gp mark 0}{(6.909,5.173)}
\gppoint{gp mark 0}{(6.189,5.263)}
\gppoint{gp mark 0}{(5.712,3.632)}
\gppoint{gp mark 0}{(5.738,3.689)}
\gppoint{gp mark 0}{(6.082,4.286)}
\gppoint{gp mark 0}{(5.633,3.430)}
\gppoint{gp mark 0}{(6.446,5.369)}
\gppoint{gp mark 0}{(5.576,3.260)}
\gppoint{gp mark 0}{(5.944,4.077)}
\gppoint{gp mark 0}{(5.660,3.503)}
\gppoint{gp mark 0}{(5.686,3.570)}
\gppoint{gp mark 0}{(6.005,4.173)}
\gppoint{gp mark 0}{(6.063,4.259)}
\gppoint{gp mark 0}{(6.270,5.283)}
\gppoint{gp mark 0}{(5.787,3.793)}
\gppoint{gp mark 0}{(5.762,3.743)}
\gppoint{gp mark 0}{(5.810,3.840)}
\gppoint{gp mark 0}{(6.668,4.970)}
\gppoint{gp mark 0}{(6.025,4.203)}
\gppoint{gp mark 0}{(6.586,4.887)}
\gppoint{gp mark 0}{(5.879,3.967)}
\gppoint{gp mark 0}{(5.857,3.927)}
\gppoint{gp mark 0}{(5.834,3.885)}
\gppoint{gp mark 0}{(5.422,2.548)}
\gppoint{gp mark 0}{(5.901,4.005)}
\gppoint{gp mark 0}{(5.923,4.042)}
\gppoint{gp mark 0}{(7.211,5.472)}
\gppoint{gp mark 0}{(6.044,4.232)}
\gppoint{gp mark 0}{(5.964,4.110)}
\gppoint{gp mark 0}{(5.605,3.349)}
\gppoint{gp mark 0}{(6.900,5.194)}
\gppoint{gp mark 0}{(6.909,5.203)}
\gppoint{gp mark 0}{(5.547,3.160)}
\gppoint{gp mark 0}{(5.787,3.768)}
\gppoint{gp mark 0}{(5.879,3.947)}
\gppoint{gp mark 0}{(6.550,5.419)}
\gppoint{gp mark 0}{(5.633,3.391)}
\gppoint{gp mark 0}{(5.944,4.059)}
\gppoint{gp mark 0}{(6.063,4.246)}
\gppoint{gp mark 0}{(5.810,3.817)}
\gppoint{gp mark 0}{(5.834,3.863)}
\gppoint{gp mark 0}{(6.222,5.256)}
\gppoint{gp mark 0}{(5.738,3.661)}
\gppoint{gp mark 0}{(5.660,3.467)}
\gppoint{gp mark 0}{(5.686,3.537)}
\gppoint{gp mark 0}{(7.796,6.138)}
\gppoint{gp mark 0}{(5.762,3.716)}
\gppoint{gp mark 0}{(5.857,3.906)}
\gppoint{gp mark 0}{(5.712,3.601)}
\gppoint{gp mark 0}{(6.044,4.217)}
\gppoint{gp mark 0}{(5.985,4.126)}
\gppoint{gp mark 0}{(6.634,4.930)}
\gppoint{gp mark 0}{(6.764,5.075)}
\gppoint{gp mark 0}{(5.762,3.689)}
\gppoint{gp mark 0}{(5.738,3.743)}
\gppoint{gp mark 0}{(5.944,4.042)}
\gppoint{gp mark 0}{(5.712,3.570)}
\gppoint{gp mark 0}{(5.923,4.077)}
\gppoint{gp mark 0}{(5.517,3.160)}
\gppoint{gp mark 0}{(5.879,4.005)}
\gppoint{gp mark 0}{(6.473,5.376)}
\gppoint{gp mark 0}{(6.805,5.114)}
\gppoint{gp mark 0}{(6.025,4.173)}
\gppoint{gp mark 0}{(5.810,3.793)}
\gppoint{gp mark 0}{(6.063,4.232)}
\gppoint{gp mark 0}{(5.633,3.503)}
\gppoint{gp mark 0}{(5.660,3.430)}
\gppoint{gp mark 0}{(6.005,4.203)}
\gppoint{gp mark 0}{(5.857,3.885)}
\gppoint{gp mark 0}{(5.576,3.349)}
\gppoint{gp mark 0}{(5.787,3.840)}
\gppoint{gp mark 0}{(5.985,4.110)}
\gppoint{gp mark 0}{(5.686,3.632)}
\gppoint{gp mark 0}{(5.834,3.927)}
\gppoint{gp mark 0}{(5.964,4.142)}
\gppoint{gp mark 0}{(6.100,4.286)}
\gppoint{gp mark 0}{(5.834,3.906)}
\gppoint{gp mark 0}{(6.419,5.345)}
\gppoint{gp mark 0}{(6.063,4.217)}
\gppoint{gp mark 0}{(5.964,4.126)}
\gppoint{gp mark 0}{(6.082,4.299)}
\gppoint{gp mark 0}{(5.686,3.601)}
\gppoint{gp mark 0}{(5.879,3.986)}
\gppoint{gp mark 0}{(5.762,3.661)}
\gppoint{gp mark 0}{(5.787,3.817)}
\gppoint{gp mark 0}{(6.044,4.246)}
\gppoint{gp mark 0}{(5.738,3.716)}
\gppoint{gp mark 0}{(6.872,5.155)}
\gppoint{gp mark 0}{(5.605,3.211)}
\gppoint{gp mark 0}{(5.660,3.391)}
\gppoint{gp mark 0}{(5.901,3.947)}
\gppoint{gp mark 0}{(5.576,3.306)}
\gppoint{gp mark 0}{(5.944,4.024)}
\gppoint{gp mark 0}{(5.923,4.059)}
\gppoint{gp mark 0}{(5.985,4.094)}
\gppoint{gp mark 0}{(5.857,3.863)}
\gppoint{gp mark 0}{(6.980,5.647)}
\gppoint{gp mark 0}{(5.712,3.537)}
\gppoint{gp mark 0}{(7.211,5.459)}
\gppoint{gp mark 0}{(6.764,5.123)}
\gppoint{gp mark 0}{(5.712,3.503)}
\gppoint{gp mark 0}{(6.005,4.232)}
\gppoint{gp mark 0}{(6.025,4.259)}
\gppoint{gp mark 0}{(7.239,5.530)}
\gppoint{gp mark 0}{(6.634,4.912)}
\gppoint{gp mark 0}{(6.082,4.110)}
\gppoint{gp mark 0}{(5.879,3.885)}
\gppoint{gp mark 0}{(6.063,4.203)}
\gppoint{gp mark 0}{(6.927,5.654)}
\gppoint{gp mark 0}{(7.745,6.090)}
\gppoint{gp mark 0}{(6.805,5.085)}
\gppoint{gp mark 0}{(5.660,3.632)}
\gppoint{gp mark 0}{(5.944,3.840)}
\gppoint{gp mark 0}{(6.598,4.970)}
\gppoint{gp mark 0}{(5.901,3.927)}
\gppoint{gp mark 0}{(7.038,5.659)}
\gppoint{gp mark 0}{(5.605,3.743)}
\gppoint{gp mark 0}{(5.576,3.689)}
\gppoint{gp mark 0}{(5.319,3.045)}
\gppoint{gp mark 0}{(5.857,4.005)}
\gppoint{gp mark 0}{(6.100,4.142)}
\gppoint{gp mark 0}{(5.354,3.160)}
\gppoint{gp mark 0}{(5.422,2.911)}
\gppoint{gp mark 0}{(5.985,4.312)}
\gppoint{gp mark 0}{(5.834,3.967)}
\gppoint{gp mark 0}{(6.954,5.629)}
\gppoint{gp mark 0}{(5.810,4.077)}
\gppoint{gp mark 0}{(5.944,3.817)}
\gppoint{gp mark 0}{(5.633,3.537)}
\gppoint{gp mark 0}{(7.559,5.972)}
\gppoint{gp mark 0}{(5.319,2.980)}
\gppoint{gp mark 0}{(5.686,3.391)}
\gppoint{gp mark 0}{(5.712,3.467)}
\gppoint{gp mark 0}{(6.100,4.126)}
\gppoint{gp mark 0}{(5.901,3.906)}
\gppoint{gp mark 0}{(7.801,6.146)}
\gppoint{gp mark 0}{(6.025,4.246)}
\gppoint{gp mark 0}{(5.857,3.986)}
\gppoint{gp mark 0}{(5.576,3.661)}
\gppoint{gp mark 0}{(5.834,3.947)}
\gppoint{gp mark 0}{(5.787,4.024)}
\gppoint{gp mark 0}{(5.605,3.716)}
\gppoint{gp mark 0}{(6.082,4.094)}
\gppoint{gp mark 0}{(5.810,4.059)}
\gppoint{gp mark 0}{(5.660,3.601)}
\gppoint{gp mark 0}{(5.879,3.863)}
\gppoint{gp mark 0}{(6.775,5.128)}
\gppoint{gp mark 0}{(6.063,4.188)}
\gppoint{gp mark 0}{(7.380,5.581)}
\gppoint{gp mark 0}{(5.923,3.768)}
\gppoint{gp mark 0}{(5.712,3.430)}
\gppoint{gp mark 0}{(7.218,5.447)}
\gppoint{gp mark 0}{(6.063,4.173)}
\gppoint{gp mark 0}{(5.319,3.160)}
\gppoint{gp mark 0}{(6.044,4.203)}
\gppoint{gp mark 0}{(5.686,3.503)}
\gppoint{gp mark 0}{(7.350,5.604)}
\gppoint{gp mark 0}{(5.787,4.077)}
\gppoint{gp mark 0}{(5.605,3.689)}
\gppoint{gp mark 0}{(5.923,3.840)}
\gppoint{gp mark 0}{(5.901,3.885)}
\gppoint{gp mark 0}{(5.633,3.632)}
\gppoint{gp mark 0}{(6.743,4.992)}
\gppoint{gp mark 0}{(6.100,4.110)}
\gppoint{gp mark 0}{(6.824,5.075)}
\gppoint{gp mark 0}{(5.576,3.743)}
\gppoint{gp mark 0}{(6.754,4.981)}
\gppoint{gp mark 0}{(5.857,3.967)}
\gppoint{gp mark 0}{(6.853,5.194)}
\gppoint{gp mark 0}{(5.834,4.005)}
\gppoint{gp mark 0}{(5.944,3.793)}
\gppoint{gp mark 0}{(7.614,5.818)}
\gppoint{gp mark 0}{(7.624,5.810)}
\gppoint{gp mark 0}{(5.810,4.042)}
\gppoint{gp mark 0}{(5.879,3.927)}
\gppoint{gp mark 0}{(6.082,4.142)}
\gppoint{gp mark 0}{(5.660,3.570)}
\gppoint{gp mark 0}{(7.259,5.507)}
\gppoint{gp mark 0}{(5.985,4.286)}
\gppoint{gp mark 0}{(5.762,3.260)}
\gppoint{gp mark 0}{(5.901,3.863)}
\gppoint{gp mark 0}{(5.660,3.537)}
\gppoint{gp mark 0}{(5.879,3.906)}
\gppoint{gp mark 0}{(6.405,5.366)}
\gppoint{gp mark 0}{(6.634,4.918)}
\gppoint{gp mark 0}{(6.100,4.094)}
\gppoint{gp mark 0}{(5.633,3.601)}
\gppoint{gp mark 0}{(5.605,3.661)}
\gppoint{gp mark 0}{(6.909,5.137)}
\gppoint{gp mark 0}{(5.857,3.947)}
\gppoint{gp mark 0}{(5.944,3.768)}
\gppoint{gp mark 0}{(5.576,3.716)}
\gppoint{gp mark 0}{(7.279,5.487)}
\gppoint{gp mark 0}{(5.923,3.817)}
\gppoint{gp mark 0}{(5.834,3.986)}
\gppoint{gp mark 0}{(7.176,5.481)}
\gppoint{gp mark 0}{(7.190,5.469)}
\gppoint{gp mark 0}{(6.025,4.217)}
\gppoint{gp mark 0}{(5.810,4.024)}
\gppoint{gp mark 0}{(5.686,3.467)}
\gppoint{gp mark 0}{(5.354,2.980)}
\gppoint{gp mark 0}{(6.362,5.271)}
\gppoint{gp mark 0}{(5.985,4.273)}
\gppoint{gp mark 0}{(6.082,4.126)}
\gppoint{gp mark 0}{(6.063,4.142)}
\gppoint{gp mark 0}{(5.762,3.503)}
\gppoint{gp mark 0}{(5.879,3.793)}
\gppoint{gp mark 0}{(5.354,2.911)}
\gppoint{gp mark 0}{(5.605,3.632)}
\gppoint{gp mark 0}{(6.512,5.428)}
\gppoint{gp mark 0}{(5.633,3.689)}
\gppoint{gp mark 0}{(6.598,4.947)}
\gppoint{gp mark 0}{(5.576,3.570)}
\gppoint{gp mark 0}{(6.044,4.110)}
\gppoint{gp mark 0}{(7.392,5.907)}
\gppoint{gp mark 0}{(5.787,3.967)}
\gppoint{gp mark 0}{(5.810,4.005)}
\gppoint{gp mark 0}{(5.901,3.840)}
\gppoint{gp mark 0}{(5.834,4.042)}
\gppoint{gp mark 0}{(5.660,3.743)}
\gppoint{gp mark 0}{(5.857,4.077)}
\gppoint{gp mark 0}{(6.005,4.286)}
\gppoint{gp mark 0}{(6.656,4.912)}
\gppoint{gp mark 0}{(6.025,4.312)}
\gppoint{gp mark 0}{(5.923,3.885)}
\gppoint{gp mark 0}{(6.988,5.677)}
\gppoint{gp mark 0}{(5.964,4.232)}
\gppoint{gp mark 0}{(5.879,3.768)}
\gppoint{gp mark 0}{(6.100,4.188)}
\gppoint{gp mark 0}{(5.810,3.986)}
\gppoint{gp mark 0}{(5.576,3.537)}
\gppoint{gp mark 0}{(7.292,5.560)}
\gppoint{gp mark 0}{(6.082,4.158)}
\gppoint{gp mark 0}{(5.857,4.059)}
\gppoint{gp mark 0}{(6.005,4.273)}
\gppoint{gp mark 0}{(5.787,3.947)}
\gppoint{gp mark 0}{(5.923,3.863)}
\gppoint{gp mark 0}{(5.633,3.661)}
\gppoint{gp mark 0}{(5.964,4.217)}
\gppoint{gp mark 0}{(5.605,3.601)}
\gppoint{gp mark 0}{(5.944,3.906)}
\gppoint{gp mark 0}{(6.063,4.126)}
\gppoint{gp mark 0}{(5.834,4.024)}
\gppoint{gp mark 0}{(5.660,3.716)}
\gppoint{gp mark 0}{(5.901,3.817)}
\gppoint{gp mark 0}{(6.025,4.299)}
\gppoint{gp mark 0}{(8.902,5.889)}
\gppoint{gp mark 0}{(5.422,3.045)}
\gppoint{gp mark 0}{(7.404,5.910)}
\gppoint{gp mark 0}{(7.013,5.687)}
\gppoint{gp mark 0}{(5.633,3.743)}
\gppoint{gp mark 0}{(5.810,3.967)}
\gppoint{gp mark 0}{(7.259,5.495)}
\gppoint{gp mark 0}{(7.279,5.501)}
\gppoint{gp mark 0}{(6.205,5.211)}
\gppoint{gp mark 0}{(5.923,3.927)}
\gppoint{gp mark 0}{(5.576,3.632)}
\gppoint{gp mark 0}{(5.901,3.793)}
\gppoint{gp mark 0}{(5.660,3.689)}
\gppoint{gp mark 0}{(6.679,5.035)}
\gppoint{gp mark 0}{(5.787,4.005)}
\gppoint{gp mark 0}{(6.025,4.286)}
\gppoint{gp mark 0}{(7.147,5.736)}
\gppoint{gp mark 0}{(6.044,4.142)}
\gppoint{gp mark 0}{(6.562,5.409)}
\gppoint{gp mark 0}{(5.834,4.077)}
\gppoint{gp mark 0}{(5.879,3.840)}
\gppoint{gp mark 0}{(6.082,4.203)}
\gppoint{gp mark 0}{(5.738,3.503)}
\gppoint{gp mark 0}{(6.005,4.312)}
\gppoint{gp mark 0}{(5.605,3.570)}
\gppoint{gp mark 0}{(5.944,3.885)}
\gppoint{gp mark 0}{(5.985,4.232)}
\gppoint{gp mark 0}{(5.633,3.716)}
\gppoint{gp mark 0}{(5.879,3.817)}
\gppoint{gp mark 0}{(5.944,3.863)}
\gppoint{gp mark 0}{(6.063,4.094)}
\gppoint{gp mark 0}{(6.005,4.299)}
\gppoint{gp mark 0}{(5.985,4.217)}
\gppoint{gp mark 0}{(5.964,4.246)}
\gppoint{gp mark 0}{(5.660,3.661)}
\gppoint{gp mark 0}{(5.576,3.601)}
\gppoint{gp mark 0}{(6.025,4.273)}
\gppoint{gp mark 0}{(5.810,3.947)}
\gppoint{gp mark 0}{(5.605,3.537)}
\gppoint{gp mark 0}{(6.656,4.918)}
\gppoint{gp mark 0}{(5.923,3.906)}
\gppoint{gp mark 0}{(5.901,3.768)}
\gppoint{gp mark 0}{(5.857,4.024)}
\gppoint{gp mark 0}{(6.473,5.345)}
\gppoint{gp mark 0}{(7.609,5.812)}
\gppoint{gp mark 0}{(6.100,4.158)}
\gppoint{gp mark 0}{(7.386,5.905)}
\gppoint{gp mark 0}{(5.787,3.986)}
\gppoint{gp mark 0}{(6.347,5.294)}
\gppoint{gp mark 0}{(5.389,3.104)}
\gppoint{gp mark 0}{(7.362,5.568)}
\gppoint{gp mark 0}{(5.964,3.885)}
\gppoint{gp mark 0}{(6.610,4.981)}
\gppoint{gp mark 0}{(7.225,5.518)}
\gppoint{gp mark 0}{(6.722,4.959)}
\gppoint{gp mark 0}{(7.740,6.131)}
\gppoint{gp mark 0}{(5.455,3.689)}
\gppoint{gp mark 0}{(6.100,4.005)}
\gppoint{gp mark 0}{(5.517,3.570)}
\gppoint{gp mark 0}{(5.547,3.632)}
\gppoint{gp mark 0}{(6.082,3.967)}
\gppoint{gp mark 0}{(5.354,3.503)}
\gppoint{gp mark 0}{(5.810,4.203)}
\gppoint{gp mark 0}{(5.319,3.430)}
\gppoint{gp mark 0}{(5.834,4.110)}
\gppoint{gp mark 0}{(6.891,5.132)}
\gppoint{gp mark 0}{(5.486,3.743)}
\gppoint{gp mark 0}{(5.901,4.312)}
\gppoint{gp mark 0}{(5.857,4.142)}
\gppoint{gp mark 0}{(5.389,3.260)}
\gppoint{gp mark 0}{(6.286,5.211)}
\gppoint{gp mark 0}{(6.025,3.840)}
\gppoint{gp mark 0}{(6.063,4.077)}
\gppoint{gp mark 0}{(6.550,5.383)}
\gppoint{gp mark 0}{(5.422,3.349)}
\gppoint{gp mark 0}{(7.648,5.871)}
\gppoint{gp mark 0}{(6.598,5.014)}
\gppoint{gp mark 0}{(7.350,5.535)}
\gppoint{gp mark 0}{(6.622,4.992)}
\gppoint{gp mark 0}{(6.679,4.912)}
\gppoint{gp mark 0}{(5.123,4.675)}
\gppoint{gp mark 0}{(6.286,5.207)}
\gppoint{gp mark 0}{(5.389,3.211)}
\gppoint{gp mark 0}{(5.547,3.601)}
\gppoint{gp mark 0}{(5.834,4.094)}
\gppoint{gp mark 0}{(5.486,3.716)}
\gppoint{gp mark 0}{(5.354,3.467)}
\gppoint{gp mark 0}{(5.901,4.299)}
\gppoint{gp mark 0}{(5.985,3.906)}
\gppoint{gp mark 0}{(5.787,4.158)}
\gppoint{gp mark 0}{(5.319,3.391)}
\gppoint{gp mark 0}{(5.455,3.661)}
\gppoint{gp mark 0}{(6.082,3.947)}
\gppoint{gp mark 0}{(5.964,3.863)}
\gppoint{gp mark 0}{(6.909,5.109)}
\gppoint{gp mark 0}{(5.857,4.126)}
\gppoint{gp mark 0}{(5.517,3.537)}
\gppoint{gp mark 0}{(7.279,5.469)}
\gppoint{gp mark 0}{(7.022,5.647)}
\gppoint{gp mark 0}{(7.886,6.001)}
\gppoint{gp mark 0}{(6.044,4.024)}
\gppoint{gp mark 0}{(6.005,3.768)}
\gppoint{gp mark 0}{(6.610,4.976)}
\gppoint{gp mark 0}{(6.100,3.986)}
\gppoint{gp mark 0}{(6.063,4.059)}
\gppoint{gp mark 0}{(7.456,5.908)}
\gppoint{gp mark 0}{(5.422,3.306)}
\gppoint{gp mark 0}{(7.169,5.716)}
\gppoint{gp mark 0}{(5.389,3.349)}
\gppoint{gp mark 0}{(6.863,5.075)}
\gppoint{gp mark 0}{(5.923,4.259)}
\gppoint{gp mark 0}{(7.554,5.961)}
\gppoint{gp mark 0}{(5.422,3.260)}
\gppoint{gp mark 0}{(5.517,3.632)}
\gppoint{gp mark 0}{(5.964,3.927)}
\gppoint{gp mark 0}{(6.044,4.077)}
\gppoint{gp mark 0}{(5.486,3.689)}
\gppoint{gp mark 0}{(5.354,3.430)}
\gppoint{gp mark 0}{(5.455,3.743)}
\gppoint{gp mark 0}{(5.319,3.503)}
\gppoint{gp mark 0}{(5.787,4.203)}
\gppoint{gp mark 0}{(5.810,4.173)}
\gppoint{gp mark 0}{(7.490,5.974)}
\gppoint{gp mark 0}{(6.872,5.065)}
\gppoint{gp mark 0}{(6.574,5.362)}
\gppoint{gp mark 0}{(5.547,3.570)}
\gppoint{gp mark 0}{(6.082,4.005)}
\gppoint{gp mark 0}{(7.197,5.489)}
\gppoint{gp mark 0}{(6.005,3.840)}
\gppoint{gp mark 0}{(6.025,3.793)}
\gppoint{gp mark 0}{(5.857,4.110)}
\gppoint{gp mark 0}{(7.467,5.907)}
\gppoint{gp mark 0}{(6.063,4.042)}
\gppoint{gp mark 0}{(5.879,4.312)}
\gppoint{gp mark 0}{(6.844,5.095)}
\gppoint{gp mark 0}{(5.547,3.537)}
\gppoint{gp mark 0}{(5.079,4.667)}
\gppoint{gp mark 0}{(5.389,3.306)}
\gppoint{gp mark 0}{(5.486,3.661)}
\gppoint{gp mark 0}{(5.354,3.391)}
\gppoint{gp mark 0}{(6.063,4.024)}
\gppoint{gp mark 0}{(6.824,5.181)}
\gppoint{gp mark 0}{(6.025,3.768)}
\gppoint{gp mark 0}{(5.834,4.126)}
\gppoint{gp mark 0}{(5.319,3.467)}
\gppoint{gp mark 0}{(5.686,3.104)}
\gppoint{gp mark 0}{(6.679,4.918)}
\gppoint{gp mark 0}{(5.944,4.217)}
\gppoint{gp mark 0}{(6.100,3.947)}
\gppoint{gp mark 0}{(6.586,5.009)}
\gppoint{gp mark 0}{(5.964,3.906)}
\gppoint{gp mark 0}{(5.517,3.601)}
\gppoint{gp mark 0}{(5.985,3.863)}
\gppoint{gp mark 0}{(6.598,4.998)}
\gppoint{gp mark 0}{(5.455,3.716)}
\gppoint{gp mark 0}{(6.044,4.059)}
\gppoint{gp mark 0}{(7.232,5.456)}
\gppoint{gp mark 0}{(5.901,4.273)}
\gppoint{gp mark 0}{(5.389,3.430)}
\gppoint{gp mark 0}{(5.547,3.743)}
\gppoint{gp mark 0}{(5.834,4.173)}
\gppoint{gp mark 0}{(5.857,4.203)}
\gppoint{gp mark 0}{(6.690,4.900)}
\gppoint{gp mark 0}{(5.944,4.312)}
\gppoint{gp mark 0}{(5.901,4.259)}
\gppoint{gp mark 0}{(5.517,3.689)}
\gppoint{gp mark 0}{(6.044,3.967)}
\gppoint{gp mark 0}{(5.422,3.503)}
\gppoint{gp mark 0}{(6.775,5.151)}
\gppoint{gp mark 0}{(6.025,3.927)}
\gppoint{gp mark 0}{(5.486,3.632)}
\gppoint{gp mark 0}{(5.455,3.570)}
\gppoint{gp mark 0}{(5.319,3.260)}
\gppoint{gp mark 0}{(5.964,3.793)}
\gppoint{gp mark 0}{(6.005,3.885)}
\gppoint{gp mark 0}{(8.256,6.416)}
\gppoint{gp mark 0}{(5.354,3.349)}
\gppoint{gp mark 0}{(6.082,4.042)}
\gppoint{gp mark 0}{(6.063,4.005)}
\gppoint{gp mark 0}{(7.286,5.578)}
\gppoint{gp mark 0}{(6.764,5.142)}
\gppoint{gp mark 0}{(6.909,5.132)}
\gppoint{gp mark 0}{(5.985,3.840)}
\gppoint{gp mark 0}{(6.586,4.981)}
\gppoint{gp mark 0}{(6.795,5.164)}
\gppoint{gp mark 0}{(5.547,3.716)}
\gppoint{gp mark 0}{(5.455,3.537)}
\gppoint{gp mark 0}{(6.005,3.863)}
\gppoint{gp mark 0}{(5.834,4.158)}
\gppoint{gp mark 0}{(5.389,3.391)}
\gppoint{gp mark 0}{(5.901,4.246)}
\gppoint{gp mark 0}{(7.094,5.751)}
\gppoint{gp mark 0}{(5.354,3.306)}
\gppoint{gp mark 0}{(6.562,5.373)}
\gppoint{gp mark 0}{(5.486,3.601)}
\gppoint{gp mark 0}{(6.332,5.248)}
\gppoint{gp mark 0}{(5.923,4.273)}
\gppoint{gp mark 0}{(5.422,3.467)}
\gppoint{gp mark 0}{(5.787,4.094)}
\gppoint{gp mark 0}{(6.063,3.986)}
\gppoint{gp mark 0}{(5.517,3.661)}
\gppoint{gp mark 0}{(7.392,5.919)}
\gppoint{gp mark 0}{(5.319,3.211)}
\gppoint{gp mark 0}{(5.857,4.188)}
\gppoint{gp mark 0}{(5.879,4.217)}
\gppoint{gp mark 0}{(5.985,3.817)}
\gppoint{gp mark 0}{(5.964,3.768)}
\gppoint{gp mark 0}{(6.044,3.947)}
\gppoint{gp mark 0}{(6.785,5.155)}
\gppoint{gp mark 0}{(5.354,3.260)}
\gppoint{gp mark 0}{(5.547,3.689)}
\gppoint{gp mark 0}{(6.712,4.912)}
\gppoint{gp mark 0}{(5.787,4.142)}
\gppoint{gp mark 0}{(5.834,4.203)}
\gppoint{gp mark 0}{(5.422,3.430)}
\gppoint{gp mark 0}{(5.857,4.173)}
\gppoint{gp mark 0}{(5.319,3.349)}
\gppoint{gp mark 0}{(6.550,5.362)}
\gppoint{gp mark 0}{(6.005,3.927)}
\gppoint{gp mark 0}{(5.486,3.570)}
\gppoint{gp mark 0}{(5.964,3.840)}
\gppoint{gp mark 0}{(6.044,4.005)}
\gppoint{gp mark 0}{(5.455,3.632)}
\gppoint{gp mark 0}{(5.810,4.110)}
\gppoint{gp mark 0}{(5.389,3.503)}
\gppoint{gp mark 0}{(5.985,3.793)}
\gppoint{gp mark 0}{(6.795,5.159)}
\gppoint{gp mark 0}{(5.517,3.743)}
\gppoint{gp mark 0}{(6.025,3.885)}
\gppoint{gp mark 0}{(5.923,4.312)}
\gppoint{gp mark 0}{(5.633,2.548)}
\gppoint{gp mark 0}{(6.622,5.003)}
\gppoint{gp mark 0}{(6.205,5.305)}
\gppoint{gp mark 0}{(6.701,4.924)}
\gppoint{gp mark 0}{(6.863,5.090)}
\gppoint{gp mark 0}{(5.389,3.467)}
\gppoint{gp mark 0}{(5.879,4.246)}
\gppoint{gp mark 0}{(7.094,5.747)}
\gppoint{gp mark 0}{(6.918,5.661)}
\gppoint{gp mark 0}{(6.668,5.040)}
\gppoint{gp mark 0}{(5.944,4.273)}
\gppoint{gp mark 0}{(5.319,3.306)}
\gppoint{gp mark 0}{(5.422,3.391)}
\gppoint{gp mark 0}{(6.824,5.198)}
\gppoint{gp mark 0}{(5.455,3.601)}
\gppoint{gp mark 0}{(5.964,3.817)}
\gppoint{gp mark 0}{(5.486,3.537)}
\gppoint{gp mark 0}{(6.525,5.345)}
\gppoint{gp mark 0}{(5.547,3.661)}
\gppoint{gp mark 0}{(5.810,4.094)}
\gppoint{gp mark 0}{(5.517,3.716)}
\gppoint{gp mark 0}{(5.923,4.299)}
\gppoint{gp mark 0}{(6.044,3.986)}
\gppoint{gp mark 0}{(6.100,4.024)}
\gppoint{gp mark 0}{(6.005,3.906)}
\gppoint{gp mark 0}{(5.857,4.158)}
\gppoint{gp mark 0}{(6.645,5.019)}
\gppoint{gp mark 0}{(6.882,5.109)}
\gppoint{gp mark 0}{(6.645,5.014)}
\gppoint{gp mark 0}{(6.025,4.005)}
\gppoint{gp mark 0}{(5.985,4.077)}
\gppoint{gp mark 0}{(5.964,4.042)}
\gppoint{gp mark 0}{(6.872,5.114)}
\gppoint{gp mark 0}{(5.486,3.503)}
\gppoint{gp mark 0}{(5.455,3.430)}
\gppoint{gp mark 0}{(5.319,3.689)}
\gppoint{gp mark 0}{(5.422,3.632)}
\gppoint{gp mark 0}{(5.389,3.570)}
\gppoint{gp mark 0}{(5.354,3.743)}
\gppoint{gp mark 0}{(5.834,4.232)}
\gppoint{gp mark 0}{(5.923,4.110)}
\gppoint{gp mark 0}{(5.944,4.142)}
\gppoint{gp mark 0}{(5.810,4.312)}
\gppoint{gp mark 0}{(5.547,3.349)}
\gppoint{gp mark 0}{(5.517,3.260)}
\gppoint{gp mark 0}{(6.679,4.959)}
\gppoint{gp mark 0}{(6.005,3.967)}
\gppoint{gp mark 0}{(6.063,3.927)}
\gppoint{gp mark 0}{(5.660,2.911)}
\gppoint{gp mark 0}{(6.044,3.885)}
\gppoint{gp mark 0}{(6.332,5.236)}
\gppoint{gp mark 0}{(7.935,7.302)}
\gppoint{gp mark 0}{(6.063,3.906)}
\gppoint{gp mark 0}{(7.266,5.456)}
\gppoint{gp mark 0}{(5.422,3.601)}
\gppoint{gp mark 0}{(6.610,5.019)}
\gppoint{gp mark 0}{(5.834,4.217)}
\gppoint{gp mark 0}{(6.005,3.947)}
\gppoint{gp mark 0}{(5.923,4.094)}
\gppoint{gp mark 0}{(5.857,4.246)}
\gppoint{gp mark 0}{(6.754,4.894)}
\gppoint{gp mark 0}{(7.318,5.592)}
\gppoint{gp mark 0}{(5.901,4.188)}
\gppoint{gp mark 0}{(5.944,4.126)}
\gppoint{gp mark 0}{(5.354,3.716)}
\gppoint{gp mark 0}{(5.455,3.391)}
\gppoint{gp mark 0}{(5.486,3.467)}
\gppoint{gp mark 0}{(5.389,3.537)}
\gppoint{gp mark 0}{(6.550,5.352)}
\gppoint{gp mark 0}{(5.879,4.158)}
\gppoint{gp mark 0}{(6.863,5.100)}
\gppoint{gp mark 0}{(5.964,4.024)}
\gppoint{gp mark 0}{(5.985,4.059)}
\gppoint{gp mark 0}{(5.810,4.299)}
\gppoint{gp mark 0}{(6.025,3.986)}
\gppoint{gp mark 0}{(6.044,3.863)}
\gppoint{gp mark 0}{(6.586,5.040)}
\gppoint{gp mark 0}{(5.319,3.661)}
\gppoint{gp mark 0}{(7.239,5.481)}
\gppoint{gp mark 0}{(7.094,5.742)}
\gppoint{gp mark 0}{(6.634,5.014)}
\gppoint{gp mark 0}{(6.562,5.341)}
\gppoint{gp mark 0}{(5.985,4.042)}
\gppoint{gp mark 0}{(5.879,4.203)}
\gppoint{gp mark 0}{(5.857,4.232)}
\gppoint{gp mark 0}{(5.389,3.632)}
\gppoint{gp mark 0}{(6.100,3.793)}
\gppoint{gp mark 0}{(5.455,3.503)}
\gppoint{gp mark 0}{(6.909,5.065)}
\gppoint{gp mark 0}{(5.354,3.689)}
\gppoint{gp mark 0}{(5.422,3.570)}
\gppoint{gp mark 0}{(5.964,4.077)}
\gppoint{gp mark 0}{(6.044,3.927)}
\gppoint{gp mark 0}{(7.266,5.453)}
\gppoint{gp mark 0}{(5.319,3.743)}
\gppoint{gp mark 0}{(5.486,3.430)}
\gppoint{gp mark 0}{(6.891,5.085)}
\gppoint{gp mark 0}{(5.923,4.142)}
\gppoint{gp mark 0}{(6.754,4.887)}
\gppoint{gp mark 0}{(6.645,5.003)}
\gppoint{gp mark 0}{(5.517,3.349)}
\gppoint{gp mark 0}{(5.901,4.173)}
\gppoint{gp mark 0}{(5.686,2.548)}
\gppoint{gp mark 0}{(6.063,3.885)}
\gppoint{gp mark 0}{(6.302,5.244)}
\gppoint{gp mark 0}{(6.598,5.045)}
\gppoint{gp mark 0}{(5.944,4.110)}
\gppoint{gp mark 0}{(7.312,5.594)}
\gppoint{gp mark 0}{(5.455,3.467)}
\gppoint{gp mark 0}{(5.944,4.094)}
\gppoint{gp mark 0}{(5.810,4.273)}
\gppoint{gp mark 0}{(5.517,3.306)}
\gppoint{gp mark 0}{(6.562,5.338)}
\gppoint{gp mark 0}{(5.901,4.158)}
\gppoint{gp mark 0}{(5.422,3.537)}
\gppoint{gp mark 0}{(5.389,3.601)}
\gppoint{gp mark 0}{(6.044,3.906)}
\gppoint{gp mark 0}{(5.244,4.667)}
\gppoint{gp mark 0}{(5.354,3.661)}
\gppoint{gp mark 0}{(6.005,3.986)}
\gppoint{gp mark 0}{(6.656,4.987)}
\gppoint{gp mark 0}{(6.775,5.190)}
\gppoint{gp mark 0}{(5.319,3.716)}
\gppoint{gp mark 0}{(5.985,4.024)}
\gppoint{gp mark 0}{(6.082,3.817)}
\gppoint{gp mark 0}{(6.063,3.863)}
\gppoint{gp mark 0}{(7.176,5.527)}
\gppoint{gp mark 0}{(5.923,4.126)}
\gppoint{gp mark 0}{(6.025,3.947)}
\gppoint{gp mark 0}{(7.356,5.554)}
\gppoint{gp mark 0}{(5.787,4.299)}
\gppoint{gp mark 0}{(6.785,5.181)}
\gppoint{gp mark 0}{(5.964,4.059)}
\gppoint{gp mark 0}{(7.246,5.469)}
\gppoint{gp mark 0}{(7.595,6.820)}
\gppoint{gp mark 0}{(5.389,3.689)}
\gppoint{gp mark 0}{(6.598,5.035)}
\gppoint{gp mark 0}{(5.787,4.232)}
\gppoint{gp mark 0}{(6.679,4.936)}
\gppoint{gp mark 0}{(5.354,3.632)}
\gppoint{gp mark 0}{(5.547,3.503)}
\gppoint{gp mark 0}{(5.486,3.349)}
\gppoint{gp mark 0}{(6.025,4.077)}
\gppoint{gp mark 0}{(7.273,5.453)}
\gppoint{gp mark 0}{(5.319,3.570)}
\gppoint{gp mark 0}{(6.743,4.912)}
\gppoint{gp mark 0}{(6.100,3.927)}
\gppoint{gp mark 0}{(5.964,3.967)}
\gppoint{gp mark 0}{(8.267,6.416)}
\gppoint{gp mark 0}{(5.879,4.110)}
\gppoint{gp mark 0}{(6.044,3.793)}
\gppoint{gp mark 0}{(6.622,5.055)}
\gppoint{gp mark 0}{(7.343,5.563)}
\gppoint{gp mark 0}{(6.005,4.042)}
\gppoint{gp mark 0}{(6.063,3.840)}
\gppoint{gp mark 0}{(6.997,5.644)}
\gppoint{gp mark 0}{(5.422,3.743)}
\gppoint{gp mark 0}{(6.645,4.992)}
\gppoint{gp mark 0}{(6.082,3.885)}
\gppoint{gp mark 0}{(5.944,4.203)}
\gppoint{gp mark 0}{(7.318,5.584)}
\gppoint{gp mark 0}{(5.985,3.986)}
\gppoint{gp mark 0}{(5.964,3.947)}
\gppoint{gp mark 0}{(5.547,3.467)}
\gppoint{gp mark 0}{(5.834,4.273)}
\gppoint{gp mark 0}{(5.517,3.391)}
\gppoint{gp mark 0}{(6.100,3.906)}
\gppoint{gp mark 0}{(6.025,4.059)}
\gppoint{gp mark 0}{(5.319,3.537)}
\gppoint{gp mark 0}{(6.722,4.881)}
\gppoint{gp mark 0}{(5.422,3.716)}
\gppoint{gp mark 0}{(5.389,3.661)}
\gppoint{gp mark 0}{(6.622,5.050)}
\gppoint{gp mark 0}{(6.679,4.930)}
\gppoint{gp mark 0}{(5.354,3.601)}
\gppoint{gp mark 0}{(7.893,6.013)}
\gppoint{gp mark 0}{(5.944,4.188)}
\gppoint{gp mark 0}{(5.810,4.246)}
\gppoint{gp mark 0}{(5.857,4.299)}
\gppoint{gp mark 0}{(7.605,5.780)}
\gppoint{gp mark 0}{(6.844,5.100)}
\gppoint{gp mark 0}{(5.486,3.306)}
\gppoint{gp mark 0}{(5.455,3.211)}
\gppoint{gp mark 0}{(5.879,4.094)}
\gppoint{gp mark 0}{(6.805,5.137)}
\gppoint{gp mark 0}{(6.005,4.024)}
\gppoint{gp mark 0}{(6.432,5.396)}
\gppoint{gp mark 0}{(6.690,4.936)}
\gppoint{gp mark 0}{(6.872,5.123)}
\gppoint{gp mark 0}{(5.901,4.110)}
\gppoint{gp mark 0}{(5.834,4.312)}
\gppoint{gp mark 0}{(5.923,4.203)}
\gppoint{gp mark 0}{(5.455,3.349)}
\gppoint{gp mark 0}{(5.422,3.689)}
\gppoint{gp mark 0}{(6.025,4.042)}
\gppoint{gp mark 0}{(6.754,4.912)}
\gppoint{gp mark 0}{(5.354,3.570)}
\gppoint{gp mark 0}{(7.368,5.535)}
\gppoint{gp mark 0}{(5.517,3.503)}
\gppoint{gp mark 0}{(5.985,3.967)}
\gppoint{gp mark 0}{(5.547,3.430)}
\gppoint{gp mark 0}{(5.319,3.632)}
\gppoint{gp mark 0}{(5.944,4.173)}
\gppoint{gp mark 0}{(5.879,4.142)}
\gppoint{gp mark 0}{(6.900,5.095)}
\gppoint{gp mark 0}{(5.857,4.286)}
\gppoint{gp mark 0}{(6.701,4.970)}
\gppoint{gp mark 0}{(6.044,3.840)}
\gppoint{gp mark 0}{(5.389,3.743)}
\gppoint{gp mark 0}{(5.787,4.259)}
\gppoint{gp mark 0}{(6.997,5.639)}
\gppoint{gp mark 0}{(6.082,3.927)}
\gppoint{gp mark 0}{(6.712,4.959)}
\gppoint{gp mark 0}{(5.857,4.273)}
\gppoint{gp mark 0}{(5.455,3.306)}
\gppoint{gp mark 0}{(5.985,3.947)}
\gppoint{gp mark 0}{(5.389,3.716)}
\gppoint{gp mark 0}{(5.319,3.601)}
\gppoint{gp mark 0}{(6.347,5.232)}
\gppoint{gp mark 0}{(5.354,3.537)}
\gppoint{gp mark 0}{(6.722,4.894)}
\gppoint{gp mark 0}{(6.785,5.198)}
\gppoint{gp mark 0}{(5.964,3.986)}
\gppoint{gp mark 0}{(7.286,5.602)}
\gppoint{gp mark 0}{(5.517,3.467)}
\gppoint{gp mark 0}{(5.422,3.661)}
\gppoint{gp mark 0}{(7.312,5.581)}
\gppoint{gp mark 0}{(6.025,4.024)}
\gppoint{gp mark 0}{(6.100,3.863)}
\gppoint{gp mark 0}{(5.901,4.094)}
\gppoint{gp mark 0}{(5.834,4.299)}
\gppoint{gp mark 0}{(5.547,3.391)}
\gppoint{gp mark 0}{(5.944,4.158)}
\gppoint{gp mark 0}{(5.810,4.217)}
\gppoint{gp mark 0}{(5.879,4.126)}
\gppoint{gp mark 0}{(7.624,5.768)}
\gppoint{gp mark 0}{(7.190,5.527)}
\gppoint{gp mark 0}{(5.354,3.927)}
\gppoint{gp mark 0}{(6.900,5.024)}
\gppoint{gp mark 0}{(5.517,3.967)}
\gppoint{gp mark 0}{(5.605,4.203)}
\gppoint{gp mark 0}{(6.419,5.219)}
\gppoint{gp mark 0}{(5.422,3.840)}
\gppoint{gp mark 0}{(5.660,4.142)}
\gppoint{gp mark 0}{(5.547,4.005)}
\gppoint{gp mark 0}{(6.390,5.236)}
\gppoint{gp mark 0}{(5.633,4.110)}
\gppoint{gp mark 0}{(5.455,4.042)}
\gppoint{gp mark 0}{(6.712,5.151)}
\gppoint{gp mark 0}{(5.389,3.793)}
\gppoint{gp mark 0}{(6.844,5.003)}
\gppoint{gp mark 0}{(5.738,4.232)}
\gppoint{gp mark 0}{(7.124,5.673)}
\gppoint{gp mark 0}{(5.319,3.885)}
\gppoint{gp mark 0}{(5.633,4.094)}
\gppoint{gp mark 0}{(5.455,4.024)}
\gppoint{gp mark 0}{(5.686,4.273)}
\gppoint{gp mark 0}{(5.762,4.246)}
\gppoint{gp mark 0}{(5.660,4.126)}
\gppoint{gp mark 0}{(7.456,5.988)}
\gppoint{gp mark 0}{(6.712,5.146)}
\gppoint{gp mark 0}{(5.605,4.188)}
\gppoint{gp mark 0}{(5.547,3.986)}
\gppoint{gp mark 0}{(5.389,3.768)}
\gppoint{gp mark 0}{(5.517,3.947)}
\gppoint{gp mark 0}{(5.738,4.217)}
\gppoint{gp mark 0}{(5.354,3.906)}
\gppoint{gp mark 0}{(5.486,4.059)}
\gppoint{gp mark 0}{(5.422,3.817)}
\gppoint{gp mark 0}{(5.576,4.158)}
\gppoint{gp mark 0}{(6.525,5.279)}
\gppoint{gp mark 0}{(6.754,5.181)}
\gppoint{gp mark 0}{(6.459,5.240)}
\gppoint{gp mark 0}{(7.286,5.450)}
\gppoint{gp mark 0}{(6.610,5.060)}
\gppoint{gp mark 0}{(5.576,4.203)}
\gppoint{gp mark 0}{(5.422,3.793)}
\gppoint{gp mark 0}{(5.738,4.259)}
\gppoint{gp mark 0}{(5.354,3.885)}
\gppoint{gp mark 0}{(6.446,5.259)}
\gppoint{gp mark 0}{(5.686,4.312)}
\gppoint{gp mark 0}{(5.517,4.005)}
\gppoint{gp mark 0}{(5.319,3.927)}
\gppoint{gp mark 0}{(5.547,3.967)}
\gppoint{gp mark 0}{(5.455,4.077)}
\gppoint{gp mark 0}{(6.863,4.992)}
\gppoint{gp mark 0}{(7.893,6.071)}
\gppoint{gp mark 0}{(7.279,5.599)}
\gppoint{gp mark 0}{(5.605,4.173)}
\gppoint{gp mark 0}{(6.824,4.947)}
\gppoint{gp mark 0}{(7.022,5.761)}
\gppoint{gp mark 0}{(5.486,4.042)}
\gppoint{gp mark 0}{(6.733,5.194)}
\gppoint{gp mark 0}{(6.656,5.114)}
\gppoint{gp mark 0}{(6.795,4.881)}
\gppoint{gp mark 0}{(8.389,6.540)}
\gppoint{gp mark 0}{(5.422,3.768)}
\gppoint{gp mark 0}{(7.392,5.949)}
\gppoint{gp mark 0}{(5.712,4.273)}
\gppoint{gp mark 0}{(5.486,4.024)}
\gppoint{gp mark 0}{(6.785,4.894)}
\gppoint{gp mark 0}{(6.824,4.942)}
\gppoint{gp mark 0}{(5.660,4.094)}
\gppoint{gp mark 0}{(5.389,3.817)}
\gppoint{gp mark 0}{(5.354,3.863)}
\gppoint{gp mark 0}{(5.319,3.906)}
\gppoint{gp mark 0}{(5.633,4.126)}
\gppoint{gp mark 0}{(5.455,4.059)}
\gppoint{gp mark 0}{(7.005,5.734)}
\gppoint{gp mark 0}{(6.525,5.271)}
\gppoint{gp mark 0}{(5.547,3.947)}
\gppoint{gp mark 0}{(6.119,5.352)}
\gppoint{gp mark 0}{(6.063,3.661)}
\gppoint{gp mark 0}{(7.554,5.933)}
\gppoint{gp mark 0}{(6.082,3.689)}
\gppoint{gp mark 0}{(5.762,4.312)}
\gppoint{gp mark 0}{(5.517,4.042)}
\gppoint{gp mark 0}{(5.576,4.110)}
\gppoint{gp mark 0}{(5.738,4.286)}
\gppoint{gp mark 0}{(5.547,4.077)}
\gppoint{gp mark 0}{(5.319,3.793)}
\gppoint{gp mark 0}{(5.354,3.840)}
\gppoint{gp mark 0}{(5.605,4.142)}
\gppoint{gp mark 0}{(6.980,5.727)}
\gppoint{gp mark 0}{(5.389,3.885)}
\gppoint{gp mark 0}{(5.486,4.005)}
\gppoint{gp mark 0}{(5.857,2.548)}
\gppoint{gp mark 0}{(5.686,4.232)}
\gppoint{gp mark 0}{(5.422,3.927)}
\gppoint{gp mark 0}{(5.486,3.986)}
\gppoint{gp mark 0}{(5.605,4.126)}
\gppoint{gp mark 0}{(7.312,5.463)}
\gppoint{gp mark 0}{(7.433,5.969)}
\gppoint{gp mark 0}{(5.422,3.906)}
\gppoint{gp mark 0}{(5.576,4.094)}
\gppoint{gp mark 0}{(5.354,3.817)}
\gppoint{gp mark 0}{(5.762,4.299)}
\gppoint{gp mark 0}{(5.389,3.863)}
\gppoint{gp mark 0}{(6.082,3.661)}
\gppoint{gp mark 0}{(5.857,2.424)}
\gppoint{gp mark 0}{(6.785,4.906)}
\gppoint{gp mark 0}{(7.211,5.560)}
\gppoint{gp mark 0}{(6.690,5.146)}
\gppoint{gp mark 0}{(6.733,5.181)}
\gppoint{gp mark 0}{(7.404,5.951)}
\gppoint{gp mark 0}{(5.486,3.967)}
\gppoint{gp mark 0}{(5.034,4.514)}
\gppoint{gp mark 0}{(6.764,4.900)}
\gppoint{gp mark 0}{(5.738,4.312)}
\gppoint{gp mark 0}{(5.762,4.286)}
\gppoint{gp mark 0}{(5.605,4.110)}
\gppoint{gp mark 0}{(5.576,4.142)}
\gppoint{gp mark 0}{(6.376,5.219)}
\gppoint{gp mark 0}{(5.354,3.793)}
\gppoint{gp mark 0}{(5.422,3.885)}
\gppoint{gp mark 0}{(5.319,3.840)}
\gppoint{gp mark 0}{(5.547,4.042)}
\gppoint{gp mark 0}{(7.253,5.589)}
\gppoint{gp mark 0}{(5.517,4.077)}
\gppoint{gp mark 0}{(5.455,4.005)}
\gppoint{gp mark 0}{(5.389,3.927)}
\gppoint{gp mark 0}{(5.834,2.548)}
\gppoint{gp mark 0}{(5.633,4.203)}
\gppoint{gp mark 0}{(5.686,4.259)}
\gppoint{gp mark 0}{(6.853,4.981)}
\gppoint{gp mark 0}{(6.701,5.168)}
\gppoint{gp mark 0}{(6.634,5.114)}
\gppoint{gp mark 0}{(7.479,5.893)}
\gppoint{gp mark 0}{(7.324,5.481)}
\gppoint{gp mark 0}{(5.422,3.863)}
\gppoint{gp mark 0}{(5.205,4.580)}
\gppoint{gp mark 0}{(5.633,4.188)}
\gppoint{gp mark 0}{(6.525,5.286)}
\gppoint{gp mark 0}{(5.319,3.817)}
\gppoint{gp mark 0}{(5.486,3.947)}
\gppoint{gp mark 0}{(5.686,4.246)}
\gppoint{gp mark 0}{(6.634,5.109)}
\gppoint{gp mark 0}{(5.455,3.986)}
\gppoint{gp mark 0}{(5.517,4.059)}
\gppoint{gp mark 0}{(5.389,3.906)}
\gppoint{gp mark 0}{(5.354,3.768)}
\gppoint{gp mark 0}{(6.645,5.100)}
\gppoint{gp mark 0}{(5.738,4.110)}
\gppoint{gp mark 0}{(7.299,5.466)}
\gppoint{gp mark 0}{(7.239,5.615)}
\gppoint{gp mark 0}{(5.517,3.793)}
\gppoint{gp mark 0}{(5.486,3.927)}
\gppoint{gp mark 0}{(5.605,4.312)}
\gppoint{gp mark 0}{(5.422,4.005)}
\gppoint{gp mark 0}{(6.936,5.714)}
\gppoint{gp mark 0}{(6.824,4.887)}
\gppoint{gp mark 0}{(6.900,4.981)}
\gppoint{gp mark 0}{(7.467,5.971)}
\gppoint{gp mark 0}{(5.547,3.840)}
\gppoint{gp mark 0}{(6.863,5.024)}
\gppoint{gp mark 0}{(5.389,3.967)}
\gppoint{gp mark 0}{(5.712,4.203)}
\gppoint{gp mark 0}{(5.633,4.232)}
\gppoint{gp mark 0}{(6.610,5.104)}
\gppoint{gp mark 0}{(7.124,5.689)}
\gppoint{gp mark 0}{(6.795,4.942)}
\gppoint{gp mark 0}{(6.824,4.881)}
\gppoint{gp mark 0}{(6.733,5.164)}
\gppoint{gp mark 0}{(5.455,3.863)}
\gppoint{gp mark 0}{(5.517,3.768)}
\gppoint{gp mark 0}{(5.389,3.947)}
\gppoint{gp mark 0}{(5.486,3.906)}
\gppoint{gp mark 0}{(5.319,4.024)}
\gppoint{gp mark 0}{(5.354,4.059)}
\gppoint{gp mark 0}{(6.815,4.918)}
\gppoint{gp mark 0}{(6.844,5.040)}
\gppoint{gp mark 0}{(5.762,4.126)}
\gppoint{gp mark 0}{(6.063,3.467)}
\gppoint{gp mark 0}{(7.538,5.933)}
\gppoint{gp mark 0}{(5.547,3.817)}
\gppoint{gp mark 0}{(5.605,4.299)}
\gppoint{gp mark 0}{(6.754,5.146)}
\gppoint{gp mark 0}{(6.459,5.207)}
\gppoint{gp mark 0}{(5.964,3.661)}
\gppoint{gp mark 0}{(5.605,4.286)}
\gppoint{gp mark 0}{(5.964,3.743)}
\gppoint{gp mark 0}{(7.239,5.609)}
\gppoint{gp mark 0}{(7.117,5.691)}
\gppoint{gp mark 0}{(5.455,3.927)}
\gppoint{gp mark 0}{(5.517,3.840)}
\gppoint{gp mark 0}{(6.712,5.177)}
\gppoint{gp mark 0}{(5.547,3.793)}
\gppoint{gp mark 0}{(5.686,4.203)}
\gppoint{gp mark 0}{(5.486,3.885)}
\gppoint{gp mark 0}{(6.754,5.142)}
\gppoint{gp mark 0}{(5.985,3.689)}
\gppoint{gp mark 0}{(7.273,5.584)}
\gppoint{gp mark 0}{(6.634,5.095)}
\gppoint{gp mark 0}{(5.389,4.005)}
\gppoint{gp mark 0}{(6.795,4.936)}
\gppoint{gp mark 0}{(6.764,4.970)}
\gppoint{gp mark 0}{(6.733,5.159)}
\gppoint{gp mark 0}{(5.762,4.110)}
\gppoint{gp mark 0}{(5.712,4.173)}
\gppoint{gp mark 0}{(5.686,4.188)}
\gppoint{gp mark 0}{(7.246,5.602)}
\gppoint{gp mark 0}{(6.586,5.128)}
\gppoint{gp mark 0}{(5.633,4.246)}
\gppoint{gp mark 0}{(5.486,3.863)}
\gppoint{gp mark 0}{(5.354,4.024)}
\gppoint{gp mark 0}{(5.547,3.768)}
\gppoint{gp mark 0}{(5.738,4.126)}
\gppoint{gp mark 0}{(6.459,5.215)}
\gppoint{gp mark 0}{(5.422,3.947)}
\gppoint{gp mark 0}{(6.824,4.894)}
\gppoint{gp mark 0}{(5.389,3.986)}
\gppoint{gp mark 0}{(5.517,3.817)}
\gppoint{gp mark 0}{(6.795,4.930)}
\gppoint{gp mark 0}{(7.273,5.581)}
\gppoint{gp mark 0}{(6.525,5.301)}
\gppoint{gp mark 0}{(5.455,3.906)}
\gppoint{gp mark 0}{(5.712,4.158)}
\gppoint{gp mark 0}{(6.997,5.755)}
\gppoint{gp mark 0}{(6.668,5.060)}
\gppoint{gp mark 0}{(5.605,4.259)}
\gppoint{gp mark 0}{(6.238,5.355)}
\gppoint{gp mark 0}{(5.422,4.077)}
\gppoint{gp mark 0}{(5.547,3.927)}
\gppoint{gp mark 0}{(5.576,4.232)}
\gppoint{gp mark 0}{(7.161,5.668)}
\gppoint{gp mark 0}{(6.610,5.123)}
\gppoint{gp mark 0}{(7.324,5.453)}
\gppoint{gp mark 0}{(6.743,5.159)}
\gppoint{gp mark 0}{(5.486,3.840)}
\gppoint{gp mark 0}{(6.486,5.305)}
\gppoint{gp mark 0}{(5.455,3.793)}
\gppoint{gp mark 0}{(5.354,4.005)}
\gppoint{gp mark 0}{(5.389,4.042)}
\gppoint{gp mark 0}{(5.517,3.885)}
\gppoint{gp mark 0}{(5.712,4.142)}
\gppoint{gp mark 0}{(6.754,5.168)}
\gppoint{gp mark 0}{(4.938,4.408)}
\gppoint{gp mark 0}{(6.701,5.194)}
\gppoint{gp mark 0}{(5.319,3.967)}
\gppoint{gp mark 0}{(7.312,5.441)}
\gppoint{gp mark 0}{(6.574,5.298)}
\gppoint{gp mark 0}{(7.490,5.910)}
\gppoint{gp mark 0}{(5.319,3.947)}
\gppoint{gp mark 0}{(5.576,4.217)}
\gppoint{gp mark 0}{(5.547,3.906)}
\gppoint{gp mark 0}{(5.517,3.863)}
\gppoint{gp mark 0}{(5.354,3.986)}
\gppoint{gp mark 0}{(6.795,4.965)}
\gppoint{gp mark 0}{(5.422,4.059)}
\gppoint{gp mark 0}{(5.455,3.768)}
\gppoint{gp mark 0}{(6.405,5.256)}
\gppoint{gp mark 0}{(5.762,4.188)}
\gppoint{gp mark 0}{(7.204,5.532)}
\gppoint{gp mark 0}{(6.733,5.146)}
\gppoint{gp mark 0}{(7.331,5.456)}
\gppoint{gp mark 0}{(5.633,4.273)}
\gppoint{gp mark 0}{(5.985,3.601)}
\gppoint{gp mark 0}{(5.486,3.817)}
\gppoint{gp mark 0}{(6.634,5.060)}
\gppoint{gp mark 0}{(6.610,5.118)}
\gppoint{gp mark 0}{(6.775,4.942)}
\gppoint{gp mark 0}{(5.738,4.158)}
\gppoint{gp mark 0}{(5.738,4.203)}
\gppoint{gp mark 0}{(5.486,3.793)}
\gppoint{gp mark 0}{(5.686,4.142)}
\gppoint{gp mark 0}{(6.525,5.320)}
\gppoint{gp mark 0}{(5.389,4.077)}
\gppoint{gp mark 0}{(6.586,5.114)}
\gppoint{gp mark 0}{(7.218,5.552)}
\gppoint{gp mark 0}{(5.762,4.173)}
\gppoint{gp mark 0}{(5.422,4.042)}
\gppoint{gp mark 0}{(5.319,4.005)}
\gppoint{gp mark 0}{(7.663,5.770)}
\gppoint{gp mark 0}{(5.517,3.927)}
\gppoint{gp mark 0}{(7.318,5.441)}
\gppoint{gp mark 0}{(6.900,5.014)}
\gppoint{gp mark 0}{(5.455,3.840)}
\gppoint{gp mark 0}{(5.547,3.885)}
\gppoint{gp mark 0}{(6.679,5.186)}
\gppoint{gp mark 0}{(5.712,4.110)}
\gppoint{gp mark 0}{(6.805,4.900)}
\gppoint{gp mark 0}{(5.762,4.158)}
\gppoint{gp mark 0}{(6.775,4.930)}
\gppoint{gp mark 0}{(5.486,3.768)}
\gppoint{gp mark 0}{(5.605,4.217)}
\gppoint{gp mark 0}{(5.034,4.580)}
\gppoint{gp mark 0}{(5.712,4.094)}
\gppoint{gp mark 0}{(5.517,3.906)}
\gppoint{gp mark 0}{(6.824,4.918)}
\gppoint{gp mark 0}{(6.005,3.716)}
\gppoint{gp mark 0}{(6.025,3.661)}
\gppoint{gp mark 0}{(6.785,4.965)}
\gppoint{gp mark 0}{(6.634,5.070)}
\gppoint{gp mark 0}{(5.633,4.299)}
\gppoint{gp mark 0}{(7.154,5.656)}
\gppoint{gp mark 0}{(5.576,4.246)}
\gppoint{gp mark 0}{(5.389,4.059)}
\gppoint{gp mark 0}{(5.455,3.817)}
\gppoint{gp mark 0}{(5.547,3.863)}
\gppoint{gp mark 0}{(5.422,4.024)}
\gppoint{gp mark 0}{(5.354,3.947)}
\gppoint{gp mark 0}{(4.987,4.385)}
\gppoint{gp mark 0}{(6.701,5.065)}
\gppoint{gp mark 0}{(7.490,5.917)}
\gppoint{gp mark 0}{(5.389,4.110)}
\gppoint{gp mark 0}{(6.598,5.168)}
\gppoint{gp mark 0}{(7.691,5.766)}
\gppoint{gp mark 0}{(5.633,3.793)}
\gppoint{gp mark 0}{(7.292,5.507)}
\gppoint{gp mark 0}{(5.660,3.840)}
\gppoint{gp mark 0}{(6.525,5.219)}
\gppoint{gp mark 0}{(5.810,3.503)}
\gppoint{gp mark 0}{(5.923,3.570)}
\gppoint{gp mark 0}{(5.319,4.173)}
\gppoint{gp mark 0}{(6.805,5.045)}
\gppoint{gp mark 0}{(5.576,3.885)}
\gppoint{gp mark 0}{(5.605,3.927)}
\gppoint{gp mark 0}{(5.547,4.259)}
\gppoint{gp mark 0}{(5.738,3.967)}
\gppoint{gp mark 0}{(5.686,4.042)}
\gppoint{gp mark 0}{(5.985,2.548)}
\gppoint{gp mark 0}{(6.844,4.912)}
\gppoint{gp mark 0}{(5.901,3.743)}
\gppoint{gp mark 0}{(5.762,4.005)}
\gppoint{gp mark 0}{(6.656,5.177)}
\gppoint{gp mark 0}{(6.971,5.749)}
\gppoint{gp mark 0}{(7.362,5.478)}
\gppoint{gp mark 0}{(5.712,4.077)}
\gppoint{gp mark 0}{(5.901,3.716)}
\gppoint{gp mark 0}{(5.079,4.325)}
\gppoint{gp mark 0}{(6.997,5.707)}
\gppoint{gp mark 0}{(5.787,3.391)}
\gppoint{gp mark 0}{(5.576,3.863)}
\gppoint{gp mark 0}{(5.282,4.441)}
\gppoint{gp mark 0}{(5.738,3.947)}
\gppoint{gp mark 0}{(5.605,3.906)}
\gppoint{gp mark 0}{(5.712,4.059)}
\gppoint{gp mark 0}{(7.286,5.498)}
\gppoint{gp mark 0}{(6.785,4.976)}
\gppoint{gp mark 0}{(5.660,3.817)}
\gppoint{gp mark 0}{(6.764,4.998)}
\gppoint{gp mark 0}{(5.354,4.188)}
\gppoint{gp mark 0}{(7.054,5.666)}
\gppoint{gp mark 0}{(7.380,5.469)}
\gppoint{gp mark 0}{(5.633,3.768)}
\gppoint{gp mark 0}{(5.517,4.217)}
\gppoint{gp mark 0}{(5.422,4.126)}
\gppoint{gp mark 0}{(5.879,3.661)}
\gppoint{gp mark 0}{(5.547,4.246)}
\gppoint{gp mark 0}{(5.762,3.986)}
\gppoint{gp mark 0}{(6.733,5.128)}
\gppoint{gp mark 0}{(5.319,4.158)}
\gppoint{gp mark 0}{(7.038,5.712)}
\gppoint{gp mark 0}{(6.598,5.164)}
\gppoint{gp mark 0}{(5.389,4.094)}
\gppoint{gp mark 0}{(6.775,5.003)}
\gppoint{gp mark 0}{(6.690,5.085)}
\gppoint{gp mark 0}{(5.455,4.312)}
\gppoint{gp mark 0}{(7.559,5.903)}
\gppoint{gp mark 0}{(5.319,4.203)}
\gppoint{gp mark 0}{(7.380,5.466)}
\gppoint{gp mark 0}{(5.633,3.840)}
\gppoint{gp mark 0}{(5.547,4.232)}
\gppoint{gp mark 0}{(5.354,4.173)}
\gppoint{gp mark 0}{(7.292,5.501)}
\gppoint{gp mark 0}{(5.660,3.793)}
\gppoint{gp mark 0}{(5.738,4.005)}
\gppoint{gp mark 0}{(5.389,4.142)}
\gppoint{gp mark 0}{(5.486,4.286)}
\gppoint{gp mark 0}{(6.863,4.900)}
\gppoint{gp mark 0}{(7.796,6.115)}
\gppoint{gp mark 0}{(6.586,5.168)}
\gppoint{gp mark 0}{(5.879,3.743)}
\gppoint{gp mark 0}{(5.901,3.689)}
\gppoint{gp mark 0}{(5.923,3.632)}
\gppoint{gp mark 0}{(6.900,4.947)}
\gppoint{gp mark 0}{(5.857,3.260)}
\gppoint{gp mark 0}{(5.686,4.077)}
\gppoint{gp mark 0}{(6.063,3.045)}
\gppoint{gp mark 0}{(5.712,4.042)}
\gppoint{gp mark 0}{(6.853,4.912)}
\gppoint{gp mark 0}{(6.712,5.065)}
\gppoint{gp mark 0}{(7.183,5.589)}
\gppoint{gp mark 0}{(6.795,4.981)}
\gppoint{gp mark 0}{(7.439,5.951)}
\gppoint{gp mark 0}{(5.738,3.986)}
\gppoint{gp mark 0}{(5.605,3.863)}
\gppoint{gp mark 0}{(5.633,3.817)}
\gppoint{gp mark 0}{(5.389,4.126)}
\gppoint{gp mark 0}{(5.712,4.024)}
\gppoint{gp mark 0}{(6.254,5.352)}
\gppoint{gp mark 0}{(5.787,3.467)}
\gppoint{gp mark 0}{(5.517,4.246)}
\gppoint{gp mark 0}{(5.686,4.059)}
\gppoint{gp mark 0}{(5.762,3.947)}
\gppoint{gp mark 0}{(5.660,3.768)}
\gppoint{gp mark 0}{(6.722,5.128)}
\gppoint{gp mark 0}{(5.576,3.906)}
\gppoint{gp mark 0}{(5.879,3.716)}
\gppoint{gp mark 0}{(7.279,5.554)}
\gppoint{gp mark 0}{(5.901,3.661)}
\gppoint{gp mark 0}{(6.656,5.181)}
\gppoint{gp mark 0}{(5.165,4.483)}
\gppoint{gp mark 0}{(5.389,4.173)}
\gppoint{gp mark 0}{(5.660,3.927)}
\gppoint{gp mark 0}{(5.686,3.967)}
\gppoint{gp mark 0}{(7.362,5.466)}
\gppoint{gp mark 0}{(7.374,5.478)}
\gppoint{gp mark 0}{(5.712,4.005)}
\gppoint{gp mark 0}{(5.762,4.077)}
\gppoint{gp mark 0}{(5.633,3.885)}
\gppoint{gp mark 0}{(6.963,5.753)}
\gppoint{gp mark 0}{(5.605,3.840)}
\gppoint{gp mark 0}{(5.422,4.203)}
\gppoint{gp mark 0}{(5.738,4.042)}
\gppoint{gp mark 0}{(5.576,3.793)}
\gppoint{gp mark 0}{(5.901,3.632)}
\gppoint{gp mark 0}{(7.109,5.691)}
\gppoint{gp mark 0}{(5.834,3.430)}
\gppoint{gp mark 0}{(6.954,5.749)}
\gppoint{gp mark 0}{(6.872,4.924)}
\gppoint{gp mark 0}{(5.633,3.863)}
\gppoint{gp mark 0}{(5.576,3.768)}
\gppoint{gp mark 0}{(5.738,4.024)}
\gppoint{gp mark 0}{(5.605,3.817)}
\gppoint{gp mark 0}{(5.389,4.158)}
\gppoint{gp mark 0}{(5.547,4.299)}
\gppoint{gp mark 0}{(5.517,4.273)}
\gppoint{gp mark 0}{(5.686,3.947)}
\gppoint{gp mark 0}{(6.743,5.118)}
\gppoint{gp mark 0}{(5.319,4.094)}
\gppoint{gp mark 0}{(7.538,5.898)}
\gppoint{gp mark 0}{(5.834,3.391)}
\gppoint{gp mark 0}{(7.392,5.972)}
\gppoint{gp mark 0}{(6.634,5.173)}
\gppoint{gp mark 0}{(5.923,3.661)}
\gppoint{gp mark 0}{(5.660,3.906)}
\gppoint{gp mark 0}{(5.712,3.986)}
\gppoint{gp mark 0}{(7.924,6.025)}
\gppoint{gp mark 0}{(7.259,5.563)}
\gppoint{gp mark 0}{(5.605,3.793)}
\gppoint{gp mark 0}{(6.795,5.003)}
\gppoint{gp mark 0}{(5.944,3.689)}
\gppoint{gp mark 0}{(5.762,4.042)}
\gppoint{gp mark 0}{(7.197,5.589)}
\gppoint{gp mark 0}{(5.660,3.885)}
\gppoint{gp mark 0}{(5.576,3.840)}
\gppoint{gp mark 0}{(7.517,5.938)}
\gppoint{gp mark 0}{(5.389,4.203)}
\gppoint{gp mark 0}{(5.712,3.967)}
\gppoint{gp mark 0}{(5.923,3.743)}
\gppoint{gp mark 0}{(6.679,5.075)}
\gppoint{gp mark 0}{(5.517,4.312)}
\gppoint{gp mark 0}{(6.834,5.045)}
\gppoint{gp mark 0}{(7.232,5.541)}
\gppoint{gp mark 0}{(5.633,3.927)}
\gppoint{gp mark 0}{(7.762,6.142)}
\gppoint{gp mark 0}{(8.082,6.972)}
\gppoint{gp mark 0}{(7.456,5.959)}
\gppoint{gp mark 0}{(5.738,4.059)}
\gppoint{gp mark 0}{(6.785,5.009)}
\gppoint{gp mark 0}{(6.701,5.090)}
\gppoint{gp mark 0}{(7.614,5.839)}
\gppoint{gp mark 0}{(5.422,4.158)}
\gppoint{gp mark 0}{(5.547,4.273)}
\gppoint{gp mark 0}{(5.576,3.817)}
\gppoint{gp mark 0}{(5.712,3.947)}
\gppoint{gp mark 0}{(6.645,5.173)}
\gppoint{gp mark 0}{(5.605,3.768)}
\gppoint{gp mark 0}{(5.389,4.188)}
\gppoint{gp mark 0}{(5.319,4.126)}
\gppoint{gp mark 0}{(7.658,5.804)}
\gppoint{gp mark 0}{(5.660,3.863)}
\gppoint{gp mark 0}{(7.204,5.602)}
\gppoint{gp mark 0}{(5.633,3.906)}
\gppoint{gp mark 0}{(7.564,5.862)}
\gppoint{gp mark 0}{(5.787,3.306)}
\gppoint{gp mark 0}{(5.686,3.986)}
\gppoint{gp mark 0}{(6.610,5.164)}
\gppoint{gp mark 0}{(5.354,4.094)}
\gppoint{gp mark 0}{(5.517,4.299)}
\gppoint{gp mark 0}{(7.374,5.481)}
\gppoint{gp mark 0}{(7.427,5.988)}
\gppoint{gp mark 0}{(5.762,4.024)}
\gppoint{gp mark 0}{(6.610,5.177)}
\gppoint{gp mark 0}{(5.787,3.689)}
\gppoint{gp mark 0}{(5.762,3.840)}
\gppoint{gp mark 0}{(5.319,4.286)}
\gppoint{gp mark 0}{(5.547,4.142)}
\gppoint{gp mark 0}{(5.738,3.793)}
\gppoint{gp mark 0}{(8.322,6.350)}
\gppoint{gp mark 0}{(5.605,4.077)}
\gppoint{gp mark 0}{(5.486,4.203)}
\gppoint{gp mark 0}{(7.246,5.557)}
\gppoint{gp mark 0}{(5.422,4.259)}
\gppoint{gp mark 0}{(5.834,3.570)}
\gppoint{gp mark 0}{(5.576,4.042)}
\gppoint{gp mark 0}{(5.660,4.005)}
\gppoint{gp mark 0}{(5.686,3.885)}
\gppoint{gp mark 0}{(5.857,3.632)}
\gppoint{gp mark 0}{(5.810,3.743)}
\gppoint{gp mark 0}{(5.205,4.408)}
\gppoint{gp mark 0}{(5.389,4.232)}
\gppoint{gp mark 0}{(6.954,5.740)}
\gppoint{gp mark 0}{(5.244,4.337)}
\gppoint{gp mark 0}{(5.686,3.863)}
\gppoint{gp mark 0}{(5.810,3.716)}
\gppoint{gp mark 0}{(6.562,5.207)}
\gppoint{gp mark 0}{(5.354,4.299)}
\gppoint{gp mark 0}{(5.576,4.024)}
\gppoint{gp mark 0}{(5.486,4.188)}
\gppoint{gp mark 0}{(6.954,5.738)}
\gppoint{gp mark 0}{(5.738,3.768)}
\gppoint{gp mark 0}{(5.762,3.817)}
\gppoint{gp mark 0}{(6.891,4.918)}
\gppoint{gp mark 0}{(6.668,5.146)}
\gppoint{gp mark 0}{(5.633,3.947)}
\gppoint{gp mark 0}{(6.815,5.009)}
\gppoint{gp mark 0}{(7.767,6.146)}
\gppoint{gp mark 0}{(5.605,4.059)}
\gppoint{gp mark 0}{(5.712,3.906)}
\gppoint{gp mark 0}{(6.656,5.137)}
\gppoint{gp mark 0}{(5.455,4.158)}
\gppoint{gp mark 0}{(5.879,3.391)}
\gppoint{gp mark 0}{(5.517,4.094)}
\gppoint{gp mark 0}{(5.660,3.986)}
\gppoint{gp mark 0}{(5.857,3.601)}
\gppoint{gp mark 0}{(5.787,3.661)}
\gppoint{gp mark 0}{(5.422,4.246)}
\gppoint{gp mark 0}{(5.319,4.273)}
\gppoint{gp mark 0}{(5.834,3.632)}
\gppoint{gp mark 0}{(5.810,3.689)}
\gppoint{gp mark 0}{(5.517,4.142)}
\gppoint{gp mark 0}{(5.319,4.312)}
\gppoint{gp mark 0}{(5.857,3.570)}
\gppoint{gp mark 0}{(5.712,3.885)}
\gppoint{gp mark 0}{(5.787,3.743)}
\gppoint{gp mark 0}{(5.762,3.793)}
\gppoint{gp mark 0}{(5.738,3.840)}
\gppoint{gp mark 0}{(6.712,5.104)}
\gppoint{gp mark 0}{(7.101,5.663)}
\gppoint{gp mark 0}{(7.197,5.599)}
\gppoint{gp mark 0}{(5.633,4.005)}
\gppoint{gp mark 0}{(5.605,4.042)}
\gppoint{gp mark 0}{(5.576,4.077)}
\gppoint{gp mark 0}{(5.389,4.259)}
\gppoint{gp mark 0}{(5.547,4.110)}
\gppoint{gp mark 0}{(7.246,5.563)}
\gppoint{gp mark 0}{(7.467,5.948)}
\gppoint{gp mark 0}{(6.562,5.215)}
\gppoint{gp mark 0}{(5.660,3.947)}
\gppoint{gp mark 0}{(6.645,5.155)}
\gppoint{gp mark 0}{(5.633,3.986)}
\gppoint{gp mark 0}{(5.422,4.217)}
\gppoint{gp mark 0}{(5.486,4.158)}
\gppoint{gp mark 0}{(6.574,5.207)}
\gppoint{gp mark 0}{(5.576,4.059)}
\gppoint{gp mark 0}{(5.810,3.661)}
\gppoint{gp mark 0}{(5.712,3.863)}
\gppoint{gp mark 0}{(5.738,3.817)}
\gppoint{gp mark 0}{(6.815,4.998)}
\gppoint{gp mark 0}{(6.891,4.906)}
\gppoint{gp mark 0}{(5.282,4.325)}
\gppoint{gp mark 0}{(5.354,4.273)}
\gppoint{gp mark 0}{(5.244,4.349)}
\gppoint{gp mark 0}{(5.319,4.299)}
\gppoint{gp mark 0}{(5.787,3.716)}
\gppoint{gp mark 0}{(5.605,4.024)}
\gppoint{gp mark 0}{(6.586,5.198)}
\gppoint{gp mark 0}{(5.686,3.906)}
\gppoint{gp mark 0}{(5.686,3.793)}
\gppoint{gp mark 0}{(6.722,5.065)}
\gppoint{gp mark 0}{(7.362,5.441)}
\gppoint{gp mark 0}{(6.302,5.383)}
\gppoint{gp mark 0}{(6.882,4.887)}
\gppoint{gp mark 0}{(6.486,5.244)}
\gppoint{gp mark 0}{(6.853,4.947)}
\gppoint{gp mark 0}{(5.787,3.570)}
\gppoint{gp mark 0}{(5.455,4.110)}
\gppoint{gp mark 0}{(5.810,3.632)}
\gppoint{gp mark 0}{(5.762,3.927)}
\gppoint{gp mark 0}{(5.834,3.689)}
\gppoint{gp mark 0}{(5.738,3.885)}
\gppoint{gp mark 0}{(5.857,3.743)}
\gppoint{gp mark 0}{(5.605,4.005)}
\gppoint{gp mark 0}{(7.070,5.687)}
\gppoint{gp mark 0}{(6.419,5.327)}
\gppoint{gp mark 0}{(5.633,4.042)}
\gppoint{gp mark 0}{(5.944,3.503)}
\gppoint{gp mark 0}{(5.576,3.967)}
\gppoint{gp mark 0}{(5.486,4.142)}
\gppoint{gp mark 0}{(6.863,4.959)}
\gppoint{gp mark 0}{(5.422,4.312)}
\gppoint{gp mark 0}{(6.656,5.159)}
\gppoint{gp mark 0}{(6.743,5.085)}
\gppoint{gp mark 0}{(5.633,4.024)}
\gppoint{gp mark 0}{(5.660,4.059)}
\gppoint{gp mark 0}{(7.259,5.532)}
\gppoint{gp mark 0}{(5.857,3.716)}
\gppoint{gp mark 0}{(5.605,3.986)}
\gppoint{gp mark 0}{(8.099,6.232)}
\gppoint{gp mark 0}{(5.576,3.947)}
\gppoint{gp mark 0}{(5.517,4.158)}
\gppoint{gp mark 0}{(5.165,4.325)}
\gppoint{gp mark 0}{(5.762,3.906)}
\gppoint{gp mark 0}{(5.486,4.126)}
\gppoint{gp mark 0}{(5.686,3.768)}
\gppoint{gp mark 0}{(5.738,3.863)}
\gppoint{gp mark 0}{(5.810,3.601)}
\gppoint{gp mark 0}{(6.712,5.128)}
\gppoint{gp mark 0}{(5.944,3.467)}
\gppoint{gp mark 0}{(7.749,6.161)}
\gppoint{gp mark 0}{(5.712,3.817)}
\gppoint{gp mark 0}{(6.863,4.953)}
\gppoint{gp mark 0}{(6.971,5.738)}
\gppoint{gp mark 0}{(6.668,5.164)}
\gppoint{gp mark 0}{(7.109,5.670)}
\gppoint{gp mark 0}{(6.562,5.223)}
\gppoint{gp mark 0}{(5.455,4.142)}
\gppoint{gp mark 0}{(6.872,4.959)}
\gppoint{gp mark 0}{(5.834,3.743)}
\gppoint{gp mark 0}{(5.605,3.967)}
\gppoint{gp mark 0}{(6.598,5.177)}
\gppoint{gp mark 0}{(5.547,4.173)}
\gppoint{gp mark 0}{(7.054,5.682)}
\gppoint{gp mark 0}{(5.205,4.337)}
\gppoint{gp mark 0}{(5.686,3.840)}
\gppoint{gp mark 0}{(6.432,5.283)}
\gppoint{gp mark 0}{(7.038,5.709)}
\gppoint{gp mark 0}{(7.727,6.172)}
\gppoint{gp mark 0}{(5.486,4.110)}
\gppoint{gp mark 0}{(5.810,3.570)}
\gppoint{gp mark 0}{(6.082,2.548)}
\gppoint{gp mark 0}{(5.787,3.632)}
\gppoint{gp mark 0}{(5.857,3.689)}
\gppoint{gp mark 0}{(6.537,5.219)}
\gppoint{gp mark 0}{(6.824,5.009)}
\gppoint{gp mark 0}{(5.762,3.863)}
\gppoint{gp mark 0}{(5.686,3.817)}
\gppoint{gp mark 0}{(5.660,4.024)}
\gppoint{gp mark 0}{(7.204,5.581)}
\gppoint{gp mark 0}{(5.712,3.768)}
\gppoint{gp mark 0}{(7.305,5.521)}
\gppoint{gp mark 0}{(5.738,3.906)}
\gppoint{gp mark 0}{(5.422,4.273)}
\gppoint{gp mark 0}{(6.668,5.155)}
\gppoint{gp mark 0}{(5.834,3.716)}
\gppoint{gp mark 0}{(5.205,4.325)}
\gppoint{gp mark 0}{(5.605,3.947)}
\gppoint{gp mark 0}{(5.165,4.349)}
\gppoint{gp mark 0}{(5.547,4.158)}
\gppoint{gp mark 0}{(5.455,4.126)}
\gppoint{gp mark 0}{(7.398,5.991)}
\gppoint{gp mark 0}{(5.517,4.188)}
\gppoint{gp mark 0}{(6.390,3.927)}
\gppoint{gp mark 0}{(6.909,4.849)}
\gppoint{gp mark 0}{(6.805,4.752)}
\gppoint{gp mark 0}{(6.610,4.337)}
\gppoint{gp mark 0}{(7.858,6.258)}
\gppoint{gp mark 0}{(6.882,4.862)}
\gppoint{gp mark 0}{(6.525,4.142)}
\gppoint{gp mark 0}{(6.622,4.361)}
\gppoint{gp mark 0}{(6.936,5.766)}
\gppoint{gp mark 0}{(6.863,4.781)}
\gppoint{gp mark 0}{(6.645,4.494)}
\gppoint{gp mark 0}{(6.690,4.571)}
\gppoint{gp mark 0}{(7.046,5.818)}
\gppoint{gp mark 0}{(6.622,4.349)}
\gppoint{gp mark 0}{(6.512,4.094)}
\gppoint{gp mark 0}{(6.586,4.373)}
\gppoint{gp mark 0}{(6.286,3.211)}
\gppoint{gp mark 0}{(6.909,4.843)}
\gppoint{gp mark 0}{(6.900,4.849)}
\gppoint{gp mark 0}{(6.656,4.452)}
\gppoint{gp mark 0}{(6.499,4.173)}
\gppoint{gp mark 0}{(6.525,4.110)}
\gppoint{gp mark 0}{(6.668,4.430)}
\gppoint{gp mark 0}{(6.586,4.408)}
\gppoint{gp mark 0}{(6.537,4.312)}
\gppoint{gp mark 0}{(7.445,5.492)}
\gppoint{gp mark 0}{(7.624,5.689)}
\gppoint{gp mark 0}{(6.844,4.816)}
\gppoint{gp mark 0}{(8.152,6.148)}
\gppoint{gp mark 0}{(6.795,4.650)}
\gppoint{gp mark 0}{(7.030,5.820)}
\gppoint{gp mark 0}{(6.679,4.562)}
\gppoint{gp mark 0}{(6.432,4.059)}
\gppoint{gp mark 0}{(7.239,5.922)}
\gppoint{gp mark 0}{(6.805,4.760)}
\gppoint{gp mark 0}{(6.499,4.158)}
\gppoint{gp mark 0}{(6.900,4.843)}
\gppoint{gp mark 0}{(6.690,4.543)}
\gppoint{gp mark 0}{(6.537,4.299)}
\gppoint{gp mark 0}{(6.586,4.397)}
\gppoint{gp mark 0}{(7.337,5.979)}
\gppoint{gp mark 0}{(6.754,4.580)}
\gppoint{gp mark 0}{(7.286,5.953)}
\gppoint{gp mark 0}{(6.754,4.642)}
\gppoint{gp mark 0}{(6.690,4.533)}
\gppoint{gp mark 0}{(5.985,5.396)}
\gppoint{gp mark 0}{(6.473,4.077)}
\gppoint{gp mark 0}{(6.656,4.473)}
\gppoint{gp mark 0}{(7.404,5.459)}
\gppoint{gp mark 0}{(6.844,4.781)}
\gppoint{gp mark 0}{(6.376,3.793)}
\gppoint{gp mark 0}{(7.559,5.615)}
\gppoint{gp mark 0}{(6.537,4.232)}
\gppoint{gp mark 0}{(6.537,4.217)}
\gppoint{gp mark 0}{(6.900,4.856)}
\gppoint{gp mark 0}{(6.712,4.562)}
\gppoint{gp mark 0}{(6.586,4.325)}
\gppoint{gp mark 0}{(6.733,4.598)}
\gppoint{gp mark 0}{(6.645,4.441)}
\gppoint{gp mark 0}{(6.743,4.616)}
\gppoint{gp mark 0}{(7.579,5.632)}
\gppoint{gp mark 0}{(6.432,3.947)}
\gppoint{gp mark 0}{(7.780,6.208)}
\gppoint{gp mark 0}{(6.550,4.246)}
\gppoint{gp mark 0}{(6.668,4.483)}
\gppoint{gp mark 0}{(6.499,4.126)}
\gppoint{gp mark 0}{(6.525,4.188)}
\gppoint{gp mark 0}{(6.598,4.349)}
\gppoint{gp mark 0}{(6.785,4.683)}
\gppoint{gp mark 0}{(6.486,4.094)}
\gppoint{gp mark 0}{(6.722,4.607)}
\gppoint{gp mark 0}{(6.834,4.752)}
\gppoint{gp mark 0}{(6.815,4.722)}
\gppoint{gp mark 0}{(6.473,4.042)}
\gppoint{gp mark 0}{(6.610,4.408)}
\gppoint{gp mark 0}{(6.562,4.312)}
\gppoint{gp mark 0}{(6.634,4.452)}
\gppoint{gp mark 0}{(6.785,4.707)}
\gppoint{gp mark 0}{(6.900,4.875)}
\gppoint{gp mark 0}{(6.853,4.774)}
\gppoint{gp mark 0}{(6.100,5.425)}
\gppoint{gp mark 0}{(6.872,4.802)}
\gppoint{gp mark 0}{(6.562,4.299)}
\gppoint{gp mark 0}{(6.574,4.273)}
\gppoint{gp mark 0}{(6.610,4.397)}
\gppoint{gp mark 0}{(6.473,4.024)}
\gppoint{gp mark 0}{(6.537,4.246)}
\gppoint{gp mark 0}{(7.410,5.469)}
\gppoint{gp mark 0}{(6.286,3.467)}
\gppoint{gp mark 0}{(6.432,3.986)}
\gppoint{gp mark 0}{(6.512,4.188)}
\gppoint{gp mark 0}{(6.733,4.580)}
\gppoint{gp mark 0}{(6.722,4.598)}
\gppoint{gp mark 0}{(6.668,4.361)}
\gppoint{gp mark 0}{(6.419,4.005)}
\gppoint{gp mark 0}{(7.517,5.541)}
\gppoint{gp mark 0}{(6.486,4.286)}
\gppoint{gp mark 0}{(7.386,5.478)}
\gppoint{gp mark 0}{(7.983,6.008)}
\gppoint{gp mark 0}{(6.712,4.607)}
\gppoint{gp mark 0}{(6.446,3.927)}
\gppoint{gp mark 0}{(7.374,5.971)}
\gppoint{gp mark 0}{(6.270,3.743)}
\gppoint{gp mark 0}{(6.701,4.589)}
\gppoint{gp mark 0}{(6.743,4.514)}
\gppoint{gp mark 0}{(6.376,4.042)}
\gppoint{gp mark 0}{(6.622,4.452)}
\gppoint{gp mark 0}{(6.690,4.642)}
\gppoint{gp mark 0}{(7.331,5.948)}
\gppoint{gp mark 0}{(6.622,4.441)}
\gppoint{gp mark 0}{(6.743,4.504)}
\gppoint{gp mark 0}{(6.834,4.667)}
\gppoint{gp mark 0}{(6.712,4.598)}
\gppoint{gp mark 0}{(6.997,5.824)}
\gppoint{gp mark 0}{(6.853,4.869)}
\gppoint{gp mark 0}{(6.863,4.829)}
\gppoint{gp mark 0}{(7.700,5.738)}
\gppoint{gp mark 0}{(6.499,4.299)}
\gppoint{gp mark 0}{(6.909,4.788)}
\gppoint{gp mark 0}{(6.872,4.843)}
\gppoint{gp mark 0}{(6.645,4.397)}
\gppoint{gp mark 0}{(6.550,4.188)}
\gppoint{gp mark 0}{(6.486,4.273)}
\gppoint{gp mark 0}{(6.733,4.562)}
\gppoint{gp mark 0}{(7.433,5.521)}
\gppoint{gp mark 0}{(6.785,4.737)}
\gppoint{gp mark 0}{(6.656,4.361)}
\gppoint{gp mark 0}{(7.410,5.459)}
\gppoint{gp mark 0}{(7.312,5.954)}
\gppoint{gp mark 0}{(7.653,5.718)}
\gppoint{gp mark 0}{(6.390,4.042)}
\gppoint{gp mark 0}{(6.512,4.259)}
\gppoint{gp mark 0}{(6.446,3.885)}
\gppoint{gp mark 0}{(6.795,4.722)}
\gppoint{gp mark 0}{(6.668,4.337)}
\gppoint{gp mark 0}{(6.376,4.077)}
\gppoint{gp mark 0}{(6.863,4.849)}
\gppoint{gp mark 0}{(7.762,6.954)}
\gppoint{gp mark 0}{(7.589,5.627)}
\gppoint{gp mark 0}{(6.722,4.562)}
\gppoint{gp mark 0}{(6.936,5.784)}
\gppoint{gp mark 0}{(6.805,4.699)}
\gppoint{gp mark 0}{(6.254,3.716)}
\gppoint{gp mark 0}{(6.512,4.246)}
\gppoint{gp mark 0}{(6.537,4.188)}
\gppoint{gp mark 0}{(6.390,4.024)}
\gppoint{gp mark 0}{(6.405,3.986)}
\gppoint{gp mark 0}{(6.805,4.659)}
\gppoint{gp mark 0}{(7.522,5.599)}
\gppoint{gp mark 0}{(6.634,4.337)}
\gppoint{gp mark 0}{(6.390,4.005)}
\gppoint{gp mark 0}{(7.533,5.609)}
\gppoint{gp mark 0}{(6.537,4.110)}
\gppoint{gp mark 0}{(6.610,4.473)}
\gppoint{gp mark 0}{(6.690,4.607)}
\gppoint{gp mark 0}{(6.645,4.361)}
\gppoint{gp mark 0}{(6.980,5.776)}
\gppoint{gp mark 0}{(6.562,4.158)}
\gppoint{gp mark 0}{(6.815,4.667)}
\gppoint{gp mark 0}{(7.506,5.538)}
\gppoint{gp mark 0}{(6.824,4.683)}
\gppoint{gp mark 0}{(6.499,4.246)}
\gppoint{gp mark 0}{(6.622,4.483)}
\gppoint{gp mark 0}{(7.374,5.975)}
\gppoint{gp mark 0}{(6.376,3.947)}
\gppoint{gp mark 0}{(6.775,4.730)}
\gppoint{gp mark 0}{(6.656,4.373)}
\gppoint{gp mark 0}{(7.232,5.929)}
\gppoint{gp mark 0}{(7.554,5.594)}
\gppoint{gp mark 0}{(7.479,5.563)}
\gppoint{gp mark 0}{(6.486,4.259)}
\gppoint{gp mark 0}{(6.722,4.533)}
\gppoint{gp mark 0}{(6.824,4.707)}
\gppoint{gp mark 0}{(6.712,4.625)}
\gppoint{gp mark 0}{(6.610,4.494)}
\gppoint{gp mark 0}{(6.645,4.337)}
\gppoint{gp mark 0}{(6.909,4.809)}
\gppoint{gp mark 0}{(6.586,4.452)}
\gppoint{gp mark 0}{(7.070,5.856)}
\gppoint{gp mark 0}{(6.872,4.862)}
\gppoint{gp mark 0}{(6.775,4.722)}
\gppoint{gp mark 0}{(6.743,4.562)}
\gppoint{gp mark 0}{(6.795,4.745)}
\gppoint{gp mark 0}{(7.993,5.994)}
\gppoint{gp mark 0}{(7.901,6.270)}
\gppoint{gp mark 0}{(6.525,4.273)}
\gppoint{gp mark 0}{(6.853,4.829)}
\gppoint{gp mark 0}{(7.543,5.581)}
\gppoint{gp mark 0}{(6.574,4.158)}
\gppoint{gp mark 0}{(6.712,4.616)}
\gppoint{gp mark 0}{(6.785,4.760)}
\gppoint{gp mark 0}{(6.486,4.246)}
\gppoint{gp mark 0}{(6.622,4.463)}
\gppoint{gp mark 0}{(6.834,4.683)}
\gppoint{gp mark 0}{(7.433,5.453)}
\gppoint{gp mark 0}{(7.538,5.541)}
\gppoint{gp mark 0}{(7.386,5.501)}
\gppoint{gp mark 0}{(6.473,4.259)}
\gppoint{gp mark 0}{(6.909,4.737)}
\gppoint{gp mark 0}{(6.679,4.385)}
\gppoint{gp mark 0}{(7.584,5.687)}
\gppoint{gp mark 0}{(7.495,5.584)}
\gppoint{gp mark 0}{(6.701,4.337)}
\gppoint{gp mark 0}{(6.376,4.173)}
\gppoint{gp mark 0}{(6.171,3.349)}
\gppoint{gp mark 0}{(6.690,4.408)}
\gppoint{gp mark 0}{(6.390,4.203)}
\gppoint{gp mark 0}{(6.446,4.312)}
\gppoint{gp mark 0}{(7.398,5.489)}
\gppoint{gp mark 0}{(7.456,5.478)}
\gppoint{gp mark 0}{(7.473,5.472)}
\gppoint{gp mark 0}{(6.390,4.188)}
\gppoint{gp mark 0}{(6.900,4.714)}
\gppoint{gp mark 0}{(7.495,5.581)}
\gppoint{gp mark 0}{(6.405,4.094)}
\gppoint{gp mark 0}{(7.386,5.498)}
\gppoint{gp mark 0}{(6.610,4.504)}
\gppoint{gp mark 0}{(6.733,4.483)}
\gppoint{gp mark 0}{(6.586,4.543)}
\gppoint{gp mark 0}{(6.405,4.142)}
\gppoint{gp mark 0}{(7.490,5.584)}
\gppoint{gp mark 0}{(6.764,4.823)}
\gppoint{gp mark 0}{(6.754,4.430)}
\gppoint{gp mark 0}{(6.432,4.312)}
\gppoint{gp mark 0}{(6.701,4.361)}
\gppoint{gp mark 0}{(6.690,4.385)}
\gppoint{gp mark 0}{(7.117,5.841)}
\gppoint{gp mark 0}{(6.805,4.875)}
\gppoint{gp mark 0}{(6.679,4.397)}
\gppoint{gp mark 0}{(6.473,4.217)}
\gppoint{gp mark 0}{(6.785,4.788)}
\gppoint{gp mark 0}{(6.900,4.730)}
\gppoint{gp mark 0}{(6.656,4.598)}
\gppoint{gp mark 0}{(7.368,5.962)}
\gppoint{gp mark 0}{(6.764,4.816)}
\gppoint{gp mark 0}{(6.645,4.616)}
\gppoint{gp mark 0}{(6.390,4.158)}
\gppoint{gp mark 0}{(6.743,4.441)}
\gppoint{gp mark 0}{(7.046,5.780)}
\gppoint{gp mark 0}{(7.714,5.712)}
\gppoint{gp mark 0}{(6.712,4.408)}
\gppoint{gp mark 0}{(7.614,5.629)}
\gppoint{gp mark 0}{(6.446,4.259)}
\gppoint{gp mark 0}{(6.656,4.625)}
\gppoint{gp mark 0}{(6.473,4.312)}
\gppoint{gp mark 0}{(6.405,4.173)}
\gppoint{gp mark 0}{(6.390,4.142)}
\gppoint{gp mark 0}{(6.586,4.514)}
\gppoint{gp mark 0}{(6.419,4.203)}
\gppoint{gp mark 0}{(6.432,4.232)}
\gppoint{gp mark 0}{(6.598,4.533)}
\gppoint{gp mark 0}{(6.775,4.795)}
\gppoint{gp mark 0}{(6.679,4.325)}
\gppoint{gp mark 0}{(6.690,4.349)}
\gppoint{gp mark 0}{(6.805,4.829)}
\gppoint{gp mark 0}{(6.376,4.094)}
\gppoint{gp mark 0}{(6.432,4.217)}
\gppoint{gp mark 0}{(6.405,4.158)}
\gppoint{gp mark 0}{(6.459,4.273)}
\gppoint{gp mark 0}{(6.154,3.391)}
\gppoint{gp mark 0}{(6.622,4.562)}
\gppoint{gp mark 0}{(6.918,5.796)}
\gppoint{gp mark 0}{(6.598,4.524)}
\gppoint{gp mark 0}{(7.533,5.552)}
\gppoint{gp mark 0}{(6.473,4.286)}
\gppoint{gp mark 0}{(6.171,3.430)}
\gppoint{gp mark 0}{(6.376,4.142)}
\gppoint{gp mark 0}{(6.645,4.589)}
\gppoint{gp mark 0}{(6.459,4.312)}
\gppoint{gp mark 0}{(6.376,4.126)}
\gppoint{gp mark 0}{(6.645,4.580)}
\gppoint{gp mark 0}{(6.945,5.804)}
\gppoint{gp mark 0}{(6.679,4.349)}
\gppoint{gp mark 0}{(6.419,4.158)}
\gppoint{gp mark 0}{(6.712,4.373)}
\gppoint{gp mark 0}{(7.386,5.492)}
\gppoint{gp mark 0}{(6.668,4.616)}
\gppoint{gp mark 0}{(6.390,4.094)}
\gppoint{gp mark 0}{(6.432,4.246)}
\gppoint{gp mark 0}{(7.667,5.744)}
\gppoint{gp mark 0}{(6.376,4.286)}
\gppoint{gp mark 0}{(6.891,4.707)}
\gppoint{gp mark 0}{(6.390,4.312)}
\gppoint{gp mark 0}{(6.432,4.173)}
\gppoint{gp mark 0}{(6.634,4.552)}
\gppoint{gp mark 0}{(6.446,4.203)}
\gppoint{gp mark 0}{(6.988,5.788)}
\gppoint{gp mark 0}{(6.512,3.947)}
\gppoint{gp mark 0}{(6.980,5.800)}
\gppoint{gp mark 0}{(6.405,4.217)}
\gppoint{gp mark 0}{(6.419,4.246)}
\gppoint{gp mark 0}{(6.882,4.683)}
\gppoint{gp mark 0}{(6.909,4.667)}
\gppoint{gp mark 0}{(6.775,4.869)}
\gppoint{gp mark 0}{(6.722,4.373)}
\gppoint{gp mark 0}{(6.432,4.158)}
\gppoint{gp mark 0}{(6.690,4.483)}
\gppoint{gp mark 0}{(6.733,4.397)}
\gppoint{gp mark 0}{(6.005,5.369)}
\gppoint{gp mark 0}{(6.119,3.743)}
\gppoint{gp mark 0}{(6.743,4.361)}
\gppoint{gp mark 0}{(7.101,5.864)}
\gppoint{gp mark 0}{(6.775,4.862)}
\gppoint{gp mark 0}{(6.668,4.514)}
\gppoint{gp mark 0}{(6.459,4.142)}
\gppoint{gp mark 0}{(6.405,4.259)}
\gppoint{gp mark 0}{(6.390,4.286)}
\gppoint{gp mark 0}{(6.376,4.312)}
\gppoint{gp mark 0}{(6.900,4.675)}
\gppoint{gp mark 0}{(6.419,4.232)}
\gppoint{gp mark 0}{(6.834,4.781)}
\gppoint{gp mark 0}{(7.495,5.599)}
\gppoint{gp mark 0}{(6.459,4.126)}
\gppoint{gp mark 0}{(7.639,5.617)}
\gppoint{gp mark 0}{(6.376,4.299)}
\gppoint{gp mark 0}{(6.656,4.524)}
\gppoint{gp mark 0}{(6.668,4.504)}
\gppoint{gp mark 0}{(6.100,5.331)}
\gppoint{gp mark 0}{(6.622,4.580)}
\gppoint{gp mark 0}{(6.795,4.829)}
\gppoint{gp mark 0}{(6.598,4.616)}
\gppoint{gp mark 0}{(6.679,4.483)}
\gppoint{gp mark 0}{(7.266,5.893)}
\gppoint{gp mark 0}{(7.433,5.466)}
\gppoint{gp mark 0}{(7.183,5.934)}
\gppoint{gp mark 0}{(6.473,4.203)}
\gppoint{gp mark 0}{(6.805,4.781)}
\gppoint{gp mark 0}{(6.512,4.042)}
\gppoint{gp mark 0}{(6.446,4.142)}
\gppoint{gp mark 0}{(6.574,3.927)}
\gppoint{gp mark 0}{(6.586,4.589)}
\gppoint{gp mark 0}{(6.743,4.385)}
\gppoint{gp mark 0}{(6.419,4.312)}
\gppoint{gp mark 0}{(6.844,4.722)}
\gppoint{gp mark 0}{(6.690,4.452)}
\gppoint{gp mark 0}{(6.795,4.875)}
\gppoint{gp mark 0}{(6.690,4.441)}
\gppoint{gp mark 0}{(6.954,5.796)}
\gppoint{gp mark 0}{(6.473,4.188)}
\gppoint{gp mark 0}{(8.343,6.532)}
\gppoint{gp mark 0}{(6.656,4.543)}
\gppoint{gp mark 0}{(6.390,4.246)}
\gppoint{gp mark 0}{(6.701,4.463)}
\gppoint{gp mark 0}{(6.376,4.217)}
\gppoint{gp mark 0}{(6.405,4.273)}
\gppoint{gp mark 0}{(6.419,4.299)}
\gppoint{gp mark 0}{(6.863,4.745)}
\gppoint{gp mark 0}{(7.517,5.589)}
\gppoint{gp mark 0}{(7.374,5.954)}
\gppoint{gp mark 0}{(6.376,4.259)}
\gppoint{gp mark 0}{(6.853,4.722)}
\gppoint{gp mark 0}{(6.679,4.452)}
\gppoint{gp mark 0}{(6.598,4.589)}
\gppoint{gp mark 0}{(6.743,4.408)}
\gppoint{gp mark 0}{(7.331,5.977)}
\gppoint{gp mark 0}{(6.900,4.707)}
\gppoint{gp mark 0}{(6.834,4.809)}
\gppoint{gp mark 0}{(6.459,4.203)}
\gppoint{gp mark 0}{(6.376,4.246)}
\gppoint{gp mark 0}{(6.668,4.543)}
\gppoint{gp mark 0}{(6.473,4.158)}
\gppoint{gp mark 0}{(5.985,5.359)}
\gppoint{gp mark 0}{(6.900,4.699)}
\gppoint{gp mark 0}{(6.834,4.802)}
\gppoint{gp mark 0}{(6.754,4.373)}
\gppoint{gp mark 0}{(6.446,4.094)}
\gppoint{gp mark 0}{(6.909,4.683)}
\gppoint{gp mark 0}{(6.238,4.005)}
\gppoint{gp mark 0}{(6.136,3.927)}
\gppoint{gp mark 0}{(7.500,5.478)}
\gppoint{gp mark 0}{(6.743,4.836)}
\gppoint{gp mark 0}{(6.332,4.312)}
\gppoint{gp mark 0}{(6.775,4.408)}
\gppoint{gp mark 0}{(6.634,4.752)}
\gppoint{gp mark 0}{(6.119,3.885)}
\gppoint{gp mark 0}{(6.805,4.473)}
\gppoint{gp mark 0}{(7.422,5.554)}
\gppoint{gp mark 0}{(7.533,5.487)}
\gppoint{gp mark 0}{(6.805,4.463)}
\gppoint{gp mark 0}{(6.238,3.986)}
\gppoint{gp mark 0}{(6.622,4.667)}
\gppoint{gp mark 0}{(6.844,4.543)}
\gppoint{gp mark 0}{(6.136,3.906)}
\gppoint{gp mark 0}{(6.254,4.158)}
\gppoint{gp mark 0}{(6.882,4.616)}
\gppoint{gp mark 0}{(6.119,3.863)}
\gppoint{gp mark 0}{(6.347,4.217)}
\gppoint{gp mark 0}{(6.238,3.967)}
\gppoint{gp mark 0}{(6.171,3.793)}
\gppoint{gp mark 0}{(6.154,3.840)}
\gppoint{gp mark 0}{(7.543,5.530)}
\gppoint{gp mark 0}{(6.668,4.722)}
\gppoint{gp mark 0}{(6.254,4.203)}
\gppoint{gp mark 0}{(6.362,4.232)}
\gppoint{gp mark 0}{(6.205,4.042)}
\gppoint{gp mark 0}{(7.147,5.826)}
\gppoint{gp mark 0}{(6.189,4.077)}
\gppoint{gp mark 0}{(6.136,3.885)}
\gppoint{gp mark 0}{(6.805,4.494)}
\gppoint{gp mark 0}{(6.775,4.385)}
\gppoint{gp mark 0}{(6.656,4.737)}
\gppoint{gp mark 0}{(7.533,5.495)}
\gppoint{gp mark 0}{(6.701,4.795)}
\gppoint{gp mark 0}{(6.222,4.005)}
\gppoint{gp mark 0}{(6.119,3.906)}
\gppoint{gp mark 0}{(6.254,4.188)}
\gppoint{gp mark 0}{(6.270,4.158)}
\gppoint{gp mark 0}{(6.189,4.059)}
\gppoint{gp mark 0}{(7.094,5.788)}
\gppoint{gp mark 0}{(6.622,4.650)}
\gppoint{gp mark 0}{(6.834,4.419)}
\gppoint{gp mark 0}{(6.154,3.817)}
\gppoint{gp mark 0}{(6.222,3.986)}
\gppoint{gp mark 0}{(6.853,4.543)}
\gppoint{gp mark 0}{(6.205,4.024)}
\gppoint{gp mark 0}{(6.909,4.580)}
\gppoint{gp mark 0}{(6.815,4.463)}
\gppoint{gp mark 0}{(7.124,5.804)}
\gppoint{gp mark 0}{(6.302,4.094)}
\gppoint{gp mark 0}{(6.805,4.483)}
\gppoint{gp mark 0}{(6.634,4.760)}
\gppoint{gp mark 0}{(6.171,3.927)}
\gppoint{gp mark 0}{(6.154,3.885)}
\gppoint{gp mark 0}{(6.622,4.707)}
\gppoint{gp mark 0}{(6.286,4.173)}
\gppoint{gp mark 0}{(6.332,4.259)}
\gppoint{gp mark 0}{(6.270,4.142)}
\gppoint{gp mark 0}{(7.467,5.609)}
\gppoint{gp mark 0}{(6.679,4.781)}
\gppoint{gp mark 0}{(6.815,4.452)}
\gppoint{gp mark 0}{(6.586,4.659)}
\gppoint{gp mark 0}{(6.701,4.809)}
\gppoint{gp mark 0}{(6.136,3.840)}
\gppoint{gp mark 0}{(7.639,5.761)}
\gppoint{gp mark 0}{(6.733,4.849)}
\gppoint{gp mark 0}{(7.005,5.865)}
\gppoint{gp mark 0}{(6.286,4.158)}
\gppoint{gp mark 0}{(6.795,4.397)}
\gppoint{gp mark 0}{(6.805,4.419)}
\gppoint{gp mark 0}{(7.574,5.703)}
\gppoint{gp mark 0}{(6.598,4.667)}
\gppoint{gp mark 0}{(6.222,4.024)}
\gppoint{gp mark 0}{(6.844,4.504)}
\gppoint{gp mark 0}{(6.302,4.188)}
\gppoint{gp mark 0}{(6.863,4.543)}
\gppoint{gp mark 0}{(6.171,3.906)}
\gppoint{gp mark 0}{(6.270,4.126)}
\gppoint{gp mark 0}{(7.796,6.945)}
\gppoint{gp mark 0}{(7.639,5.757)}
\gppoint{gp mark 0}{(6.189,4.005)}
\gppoint{gp mark 0}{(6.222,4.077)}
\gppoint{gp mark 0}{(6.205,3.967)}
\gppoint{gp mark 0}{(7.299,5.900)}
\gppoint{gp mark 0}{(6.286,4.203)}
\gppoint{gp mark 0}{(6.863,4.571)}
\gppoint{gp mark 0}{(6.317,4.259)}
\gppoint{gp mark 0}{(6.872,4.552)}
\gppoint{gp mark 0}{(6.805,4.452)}
\gppoint{gp mark 0}{(7.318,5.903)}
\gppoint{gp mark 0}{(6.238,4.042)}
\gppoint{gp mark 0}{(7.416,6.754)}
\gppoint{gp mark 0}{(6.945,5.835)}
\gppoint{gp mark 0}{(6.222,4.059)}
\gppoint{gp mark 0}{(6.610,4.699)}
\gppoint{gp mark 0}{(6.598,4.650)}
\gppoint{gp mark 0}{(6.254,4.126)}
\gppoint{gp mark 0}{(6.347,4.299)}
\gppoint{gp mark 0}{(6.754,4.856)}
\gppoint{gp mark 0}{(6.332,4.217)}
\gppoint{gp mark 0}{(7.386,5.538)}
\gppoint{gp mark 0}{(6.171,3.863)}
\gppoint{gp mark 0}{(6.362,4.273)}
\gppoint{gp mark 0}{(6.785,4.397)}
\gppoint{gp mark 0}{(6.189,3.986)}
\gppoint{gp mark 0}{(7.826,6.212)}
\gppoint{gp mark 0}{(6.634,4.691)}
\gppoint{gp mark 0}{(6.332,4.203)}
\gppoint{gp mark 0}{(6.598,4.767)}
\gppoint{gp mark 0}{(6.154,3.967)}
\gppoint{gp mark 0}{(6.286,4.232)}
\gppoint{gp mark 0}{(6.853,4.642)}
\gppoint{gp mark 0}{(6.656,4.659)}
\gppoint{gp mark 0}{(6.347,4.110)}
\gppoint{gp mark 0}{(7.527,5.530)}
\gppoint{gp mark 0}{(6.171,4.005)}
\gppoint{gp mark 0}{(7.392,5.573)}
\gppoint{gp mark 0}{(6.270,4.312)}
\gppoint{gp mark 0}{(6.815,4.408)}
\gppoint{gp mark 0}{(6.645,4.707)}
\gppoint{gp mark 0}{(6.205,3.927)}
\gppoint{gp mark 0}{(6.317,4.158)}
\gppoint{gp mark 0}{(6.171,3.986)}
\gppoint{gp mark 0}{(6.656,4.650)}
\gppoint{gp mark 0}{(6.882,4.543)}
\gppoint{gp mark 0}{(6.270,4.299)}
\gppoint{gp mark 0}{(6.286,4.217)}
\gppoint{gp mark 0}{(6.254,4.273)}
\gppoint{gp mark 0}{(6.136,4.059)}
\gppoint{gp mark 0}{(6.189,3.863)}
\gppoint{gp mark 0}{(6.154,3.947)}
\gppoint{gp mark 0}{(6.645,4.699)}
\gppoint{gp mark 0}{(6.785,4.419)}
\gppoint{gp mark 0}{(6.872,4.598)}
\gppoint{gp mark 0}{(6.645,4.691)}
\gppoint{gp mark 0}{(6.775,4.473)}
\gppoint{gp mark 0}{(6.882,4.571)}
\gppoint{gp mark 0}{(6.598,4.752)}
\gppoint{gp mark 0}{(6.136,4.042)}
\gppoint{gp mark 0}{(6.317,4.203)}
\gppoint{gp mark 0}{(6.254,4.312)}
\gppoint{gp mark 0}{(6.154,4.005)}
\gppoint{gp mark 0}{(6.362,4.110)}
\gppoint{gp mark 0}{(6.270,4.286)}
\gppoint{gp mark 0}{(6.844,4.642)}
\gppoint{gp mark 0}{(7.386,5.573)}
\gppoint{gp mark 0}{(6.154,3.986)}
\gppoint{gp mark 0}{(6.205,3.863)}
\gppoint{gp mark 0}{(6.872,4.580)}
\gppoint{gp mark 0}{(6.286,4.246)}
\gppoint{gp mark 0}{(7.169,5.796)}
\gppoint{gp mark 0}{(6.302,4.217)}
\gppoint{gp mark 0}{(7.368,5.922)}
\gppoint{gp mark 0}{(6.679,4.869)}
\gppoint{gp mark 0}{(6.785,4.441)}
\gppoint{gp mark 0}{(6.119,4.059)}
\gppoint{gp mark 0}{(6.136,4.024)}
\gppoint{gp mark 0}{(6.882,4.562)}
\gppoint{gp mark 0}{(6.317,4.188)}
\gppoint{gp mark 0}{(6.189,3.906)}
\gppoint{gp mark 0}{(6.712,4.829)}
\gppoint{gp mark 0}{(6.610,4.752)}
\gppoint{gp mark 0}{(6.136,4.005)}
\gppoint{gp mark 0}{(6.171,4.077)}
\gppoint{gp mark 0}{(6.586,4.722)}
\gppoint{gp mark 0}{(6.775,4.452)}
\gppoint{gp mark 0}{(6.286,4.286)}
\gppoint{gp mark 0}{(6.362,4.203)}
\gppoint{gp mark 0}{(6.622,4.767)}
\gppoint{gp mark 0}{(7.479,5.466)}
\gppoint{gp mark 0}{(6.909,4.571)}
\gppoint{gp mark 0}{(6.154,4.042)}
\gppoint{gp mark 0}{(6.317,4.110)}
\gppoint{gp mark 0}{(6.332,4.142)}
\gppoint{gp mark 0}{(7.305,5.912)}
\gppoint{gp mark 0}{(6.254,4.217)}
\gppoint{gp mark 0}{(6.302,4.299)}
\gppoint{gp mark 0}{(6.347,4.158)}
\gppoint{gp mark 0}{(7.439,5.602)}
\gppoint{gp mark 0}{(6.171,4.059)}
\gppoint{gp mark 0}{(7.286,5.901)}
\gppoint{gp mark 0}{(6.362,4.188)}
\gppoint{gp mark 0}{(6.824,4.373)}
\gppoint{gp mark 0}{(6.154,4.024)}
\gppoint{gp mark 0}{(6.238,3.906)}
\gppoint{gp mark 0}{(6.844,4.580)}
\gppoint{gp mark 0}{(6.119,3.947)}
\gppoint{gp mark 0}{(7.147,5.802)}
\gppoint{gp mark 0}{(6.909,4.552)}
\gppoint{gp mark 0}{(7.500,5.447)}
\gppoint{gp mark 0}{(6.645,4.659)}
\gppoint{gp mark 0}{(6.332,4.110)}
\gppoint{gp mark 0}{(6.775,4.430)}
\gppoint{gp mark 0}{(6.136,3.967)}
\gppoint{gp mark 0}{(7.305,5.910)}
\gppoint{gp mark 0}{(7.337,5.934)}
\gppoint{gp mark 0}{(6.302,4.286)}
\gppoint{gp mark 0}{(6.154,4.077)}
\gppoint{gp mark 0}{(6.598,4.722)}
\gppoint{gp mark 0}{(6.872,4.625)}
\gppoint{gp mark 0}{(6.733,4.781)}
\gppoint{gp mark 0}{(6.668,4.691)}
\gppoint{gp mark 0}{(6.171,4.042)}
\gppoint{gp mark 0}{(6.634,4.675)}
\gppoint{gp mark 0}{(7.368,5.917)}
\gppoint{gp mark 0}{(6.656,4.707)}
\gppoint{gp mark 0}{(6.222,3.927)}
\gppoint{gp mark 0}{(7.456,5.581)}
\gppoint{gp mark 0}{(7.901,6.305)}
\gppoint{gp mark 0}{(6.701,4.869)}
\gppoint{gp mark 0}{(6.171,4.024)}
\gppoint{gp mark 0}{(6.154,4.059)}
\gppoint{gp mark 0}{(6.743,4.816)}
\gppoint{gp mark 0}{(7.749,6.218)}
\gppoint{gp mark 0}{(6.119,3.986)}
\gppoint{gp mark 0}{(6.679,4.843)}
\gppoint{gp mark 0}{(6.690,4.829)}
\gppoint{gp mark 0}{(6.270,4.217)}
\gppoint{gp mark 0}{(7.218,5.953)}
\gppoint{gp mark 0}{(6.154,4.110)}
\gppoint{gp mark 0}{(6.222,4.232)}
\gppoint{gp mark 0}{(6.171,4.142)}
\gppoint{gp mark 0}{(6.136,4.203)}
\gppoint{gp mark 0}{(6.891,4.494)}
\gppoint{gp mark 0}{(7.410,5.609)}
\gppoint{gp mark 0}{(6.872,4.361)}
\gppoint{gp mark 0}{(6.189,4.286)}
\gppoint{gp mark 0}{(6.764,4.552)}
\gppoint{gp mark 0}{(6.834,4.607)}
\gppoint{gp mark 0}{(6.844,4.373)}
\gppoint{gp mark 0}{(6.764,4.543)}
\gppoint{gp mark 0}{(6.136,4.188)}
\gppoint{gp mark 0}{(6.302,3.817)}
\gppoint{gp mark 0}{(6.645,4.869)}
\gppoint{gp mark 0}{(6.154,4.094)}
\gppoint{gp mark 0}{(6.119,4.158)}
\gppoint{gp mark 0}{(6.171,4.126)}
\gppoint{gp mark 0}{(6.317,4.024)}
\gppoint{gp mark 0}{(6.795,4.524)}
\gppoint{gp mark 0}{(6.712,4.667)}
\gppoint{gp mark 0}{(6.270,3.906)}
\gppoint{gp mark 0}{(6.332,4.059)}
\gppoint{gp mark 0}{(6.222,4.217)}
\gppoint{gp mark 0}{(6.668,4.843)}
\gppoint{gp mark 0}{(6.586,4.823)}
\gppoint{gp mark 0}{(6.005,5.219)}
\gppoint{gp mark 0}{(7.070,5.802)}
\gppoint{gp mark 0}{(6.222,4.259)}
\gppoint{gp mark 0}{(6.171,4.110)}
\gppoint{gp mark 0}{(6.286,3.840)}
\gppoint{gp mark 0}{(6.722,4.767)}
\gppoint{gp mark 0}{(6.332,4.042)}
\gppoint{gp mark 0}{(6.189,4.312)}
\gppoint{gp mark 0}{(6.743,4.737)}
\gppoint{gp mark 0}{(6.119,4.203)}
\gppoint{gp mark 0}{(6.775,4.552)}
\gppoint{gp mark 0}{(6.063,5.256)}
\gppoint{gp mark 0}{(6.171,4.094)}
\gppoint{gp mark 0}{(6.205,4.273)}
\gppoint{gp mark 0}{(6.154,4.126)}
\gppoint{gp mark 0}{(6.712,4.650)}
\gppoint{gp mark 0}{(6.238,4.217)}
\gppoint{gp mark 0}{(6.286,3.817)}
\gppoint{gp mark 0}{(6.136,4.158)}
\gppoint{gp mark 0}{(6.254,3.906)}
\gppoint{gp mark 0}{(7.490,5.492)}
\gppoint{gp mark 0}{(6.679,4.699)}
\gppoint{gp mark 0}{(7.422,5.602)}
\gppoint{gp mark 0}{(6.317,4.059)}
\gppoint{gp mark 0}{(6.302,3.927)}
\gppoint{gp mark 0}{(7.086,5.814)}
\gppoint{gp mark 0}{(6.712,4.707)}
\gppoint{gp mark 0}{(7.569,5.736)}
\gppoint{gp mark 0}{(6.136,4.142)}
\gppoint{gp mark 0}{(7.427,5.615)}
\gppoint{gp mark 0}{(6.189,4.232)}
\gppoint{gp mark 0}{(6.119,4.110)}
\gppoint{gp mark 0}{(7.169,5.794)}
\gppoint{gp mark 0}{(6.824,4.625)}
\gppoint{gp mark 0}{(6.805,4.580)}
\gppoint{gp mark 0}{(6.136,4.126)}
\gppoint{gp mark 0}{(7.386,5.576)}
\gppoint{gp mark 0}{(6.302,3.906)}
\gppoint{gp mark 0}{(7.225,5.991)}
\gppoint{gp mark 0}{(6.844,4.325)}
\gppoint{gp mark 0}{(6.119,4.094)}
\gppoint{gp mark 0}{(6.254,3.768)}
\gppoint{gp mark 0}{(6.317,3.947)}
\gppoint{gp mark 0}{(6.622,4.816)}
\gppoint{gp mark 0}{(6.005,5.223)}
\gppoint{gp mark 0}{(6.205,4.246)}
\gppoint{gp mark 0}{(6.302,3.885)}
\gppoint{gp mark 0}{(6.712,4.691)}
\gppoint{gp mark 0}{(6.668,4.862)}
\gppoint{gp mark 0}{(6.154,4.203)}
\gppoint{gp mark 0}{(6.690,4.659)}
\gppoint{gp mark 0}{(6.598,4.781)}
\gppoint{gp mark 0}{(6.238,4.286)}
\gppoint{gp mark 0}{(7.462,5.557)}
\gppoint{gp mark 0}{(6.082,5.267)}
\gppoint{gp mark 0}{(6.834,4.625)}
\gppoint{gp mark 0}{(6.205,4.232)}
\gppoint{gp mark 0}{(6.222,4.312)}
\gppoint{gp mark 0}{(6.900,4.494)}
\gppoint{gp mark 0}{(6.872,4.385)}
\gppoint{gp mark 0}{(6.286,3.927)}
\gppoint{gp mark 0}{(6.863,4.408)}
\gppoint{gp mark 0}{(7.343,5.889)}
\gppoint{gp mark 0}{(6.645,4.836)}
\gppoint{gp mark 0}{(7.527,5.438)}
\gppoint{gp mark 0}{(6.743,4.760)}
\gppoint{gp mark 0}{(6.775,4.504)}
\gppoint{gp mark 0}{(6.119,4.126)}
\gppoint{gp mark 0}{(6.712,4.683)}
\gppoint{gp mark 0}{(6.189,4.246)}
\gppoint{gp mark 0}{(6.222,4.299)}
\gppoint{gp mark 0}{(7.331,5.936)}
\gppoint{gp mark 0}{(6.834,4.616)}
\gppoint{gp mark 0}{(7.889,6.992)}
\gppoint{gp mark 0}{(6.238,4.142)}
\gppoint{gp mark 0}{(6.205,4.203)}
\gppoint{gp mark 0}{(6.302,4.005)}
\gppoint{gp mark 0}{(7.589,5.744)}
\gppoint{gp mark 0}{(6.722,4.691)}
\gppoint{gp mark 0}{(6.805,4.552)}
\gppoint{gp mark 0}{(7.913,6.289)}
\gppoint{gp mark 0}{(6.189,4.173)}
\gppoint{gp mark 0}{(6.390,3.743)}
\gppoint{gp mark 0}{(6.270,4.077)}
\gppoint{gp mark 0}{(6.119,4.286)}
\gppoint{gp mark 0}{(6.863,4.430)}
\gppoint{gp mark 0}{(6.775,4.642)}
\gppoint{gp mark 0}{(6.853,4.494)}
\gppoint{gp mark 0}{(6.286,3.967)}
\gppoint{gp mark 0}{(7.094,5.810)}
\gppoint{gp mark 0}{(7.495,5.518)}
\gppoint{gp mark 0}{(6.332,3.927)}
\gppoint{gp mark 0}{(6.815,4.562)}
\gppoint{gp mark 0}{(6.205,4.188)}
\gppoint{gp mark 0}{(6.586,4.856)}
\gppoint{gp mark 0}{(6.270,4.059)}
\gppoint{gp mark 0}{(6.222,4.094)}
\gppoint{gp mark 0}{(6.238,4.126)}
\gppoint{gp mark 0}{(6.317,3.863)}
\gppoint{gp mark 0}{(6.254,4.024)}
\gppoint{gp mark 0}{(6.656,4.774)}
\gppoint{gp mark 0}{(6.598,4.869)}
\gppoint{gp mark 0}{(6.189,4.158)}
\gppoint{gp mark 0}{(6.347,3.840)}
\gppoint{gp mark 0}{(6.189,4.203)}
\gppoint{gp mark 0}{(6.302,3.967)}
\gppoint{gp mark 0}{(6.222,4.142)}
\gppoint{gp mark 0}{(6.119,4.312)}
\gppoint{gp mark 0}{(6.679,4.767)}
\gppoint{gp mark 0}{(6.863,4.452)}
\gppoint{gp mark 0}{(6.701,4.737)}
\gppoint{gp mark 0}{(7.584,5.744)}
\gppoint{gp mark 0}{(6.171,4.232)}
\gppoint{gp mark 0}{(6.805,4.571)}
\gppoint{gp mark 0}{(6.690,4.752)}
\gppoint{gp mark 0}{(6.668,4.781)}
\gppoint{gp mark 0}{(7.416,5.589)}
\gppoint{gp mark 0}{(6.286,3.986)}
\gppoint{gp mark 0}{(6.222,4.126)}
\gppoint{gp mark 0}{(6.119,4.299)}
\gppoint{gp mark 0}{(7.286,5.939)}
\gppoint{gp mark 0}{(6.610,4.843)}
\gppoint{gp mark 0}{(6.743,4.667)}
\gppoint{gp mark 0}{(6.701,4.730)}
\gppoint{gp mark 0}{(6.891,4.373)}
\gppoint{gp mark 0}{(6.171,4.217)}
\gppoint{gp mark 0}{(6.754,4.650)}
\gppoint{gp mark 0}{(6.189,4.188)}
\gppoint{gp mark 0}{(6.154,4.246)}
\gppoint{gp mark 0}{(6.764,4.589)}
\gppoint{gp mark 0}{(6.136,4.259)}
\gppoint{gp mark 0}{(5.787,5.305)}
\gppoint{gp mark 0}{(6.238,4.203)}
\gppoint{gp mark 0}{(6.332,3.840)}
\gppoint{gp mark 0}{(6.302,4.077)}
\gppoint{gp mark 0}{(6.785,4.625)}
\gppoint{gp mark 0}{(7.644,5.677)}
\gppoint{gp mark 0}{(6.119,4.232)}
\gppoint{gp mark 0}{(7.225,5.980)}
\gppoint{gp mark 0}{(6.189,4.110)}
\gppoint{gp mark 0}{(7.517,5.507)}
\gppoint{gp mark 0}{(6.154,4.286)}
\gppoint{gp mark 0}{(6.834,4.571)}
\gppoint{gp mark 0}{(6.853,4.452)}
\gppoint{gp mark 0}{(6.872,4.494)}
\gppoint{gp mark 0}{(6.286,4.024)}
\gppoint{gp mark 0}{(7.070,5.820)}
\gppoint{gp mark 0}{(6.634,4.774)}
\gppoint{gp mark 0}{(6.690,4.730)}
\gppoint{gp mark 0}{(6.154,4.273)}
\gppoint{gp mark 0}{(6.733,4.667)}
\gppoint{gp mark 0}{(7.462,5.538)}
\gppoint{gp mark 0}{(6.805,4.504)}
\gppoint{gp mark 0}{(6.119,4.217)}
\gppoint{gp mark 0}{(6.834,4.562)}
\gppoint{gp mark 0}{(6.136,4.246)}
\gppoint{gp mark 0}{(6.302,4.059)}
\gppoint{gp mark 0}{(6.205,4.126)}
\gppoint{gp mark 0}{(6.656,4.802)}
\gppoint{gp mark 0}{(6.844,4.419)}
\gppoint{gp mark 0}{(7.117,5.780)}
\gppoint{gp mark 0}{(6.317,3.840)}
\gppoint{gp mark 0}{(6.270,3.967)}
\gppoint{gp mark 0}{(7.324,5.927)}
\gppoint{gp mark 0}{(6.238,4.173)}
\gppoint{gp mark 0}{(6.785,4.642)}
\gppoint{gp mark 0}{(6.154,4.312)}
\gppoint{gp mark 0}{(6.254,4.005)}
\gppoint{gp mark 0}{(6.222,4.203)}
\gppoint{gp mark 0}{(6.712,4.752)}
\gppoint{gp mark 0}{(6.824,4.571)}
\gppoint{gp mark 0}{(6.205,4.110)}
\gppoint{gp mark 0}{(7.094,5.798)}
\gppoint{gp mark 0}{(6.645,4.774)}
\gppoint{gp mark 0}{(6.586,4.843)}
\gppoint{gp mark 0}{(6.222,4.188)}
\gppoint{gp mark 0}{(6.824,4.562)}
\gppoint{gp mark 0}{(6.119,4.246)}
\gppoint{gp mark 0}{(6.205,4.094)}
\gppoint{gp mark 0}{(7.398,5.612)}
\gppoint{gp mark 0}{(7.648,5.675)}
\gppoint{gp mark 0}{(6.171,4.273)}
\gppoint{gp mark 0}{(6.270,3.947)}
\gppoint{gp mark 0}{(7.704,5.617)}
\gppoint{gp mark 0}{(6.775,4.580)}
\gppoint{gp mark 0}{(6.997,5.927)}
\gppoint{gp mark 0}{(6.562,4.836)}
\gppoint{gp mark 0}{(7.086,5.964)}
\gppoint{gp mark 0}{(6.171,4.361)}
\gppoint{gp mark 0}{(6.119,4.385)}
\gppoint{gp mark 0}{(6.824,3.967)}
\gppoint{gp mark 0}{(6.459,4.722)}
\gppoint{gp mark 0}{(6.136,4.408)}
\gppoint{gp mark 0}{(6.432,4.752)}
\gppoint{gp mark 0}{(6.376,4.691)}
\gppoint{gp mark 0}{(6.405,4.659)}
\gppoint{gp mark 0}{(7.273,5.814)}
\gppoint{gp mark 0}{(6.390,4.707)}
\gppoint{gp mark 0}{(7.677,5.563)}
\gppoint{gp mark 0}{(7.484,5.709)}
\gppoint{gp mark 0}{(6.302,4.533)}
\gppoint{gp mark 0}{(7.495,6.788)}
\gppoint{gp mark 0}{(6.347,4.580)}
\gppoint{gp mark 0}{(6.136,4.397)}
\gppoint{gp mark 0}{(6.537,4.856)}
\gppoint{gp mark 0}{(6.119,4.373)}
\gppoint{gp mark 0}{(6.332,4.633)}
\gppoint{gp mark 0}{(6.286,4.504)}
\gppoint{gp mark 0}{(6.302,4.524)}
\gppoint{gp mark 0}{(6.376,4.683)}
\gppoint{gp mark 0}{(6.499,4.816)}
\gppoint{gp mark 0}{(6.317,4.616)}
\gppoint{gp mark 0}{(5.660,4.987)}
\gppoint{gp mark 0}{(6.419,4.667)}
\gppoint{gp mark 0}{(6.270,4.562)}
\gppoint{gp mark 0}{(7.579,5.444)}
\gppoint{gp mark 0}{(6.189,4.494)}
\gppoint{gp mark 0}{(6.205,4.473)}
\gppoint{gp mark 0}{(6.171,4.337)}
\gppoint{gp mark 0}{(6.222,4.452)}
\gppoint{gp mark 0}{(6.405,4.675)}
\gppoint{gp mark 0}{(5.738,5.035)}
\gppoint{gp mark 0}{(6.317,4.642)}
\gppoint{gp mark 0}{(6.332,4.625)}
\gppoint{gp mark 0}{(6.270,4.552)}
\gppoint{gp mark 0}{(6.238,4.430)}
\gppoint{gp mark 0}{(7.368,5.882)}
\gppoint{gp mark 0}{(6.525,4.781)}
\gppoint{gp mark 0}{(6.432,4.760)}
\gppoint{gp mark 0}{(6.550,4.856)}
\gppoint{gp mark 0}{(6.119,4.397)}
\gppoint{gp mark 0}{(7.038,5.933)}
\gppoint{gp mark 0}{(6.222,4.441)}
\gppoint{gp mark 0}{(5.944,5.100)}
\gppoint{gp mark 0}{(6.171,4.325)}
\gppoint{gp mark 0}{(7.574,5.444)}
\gppoint{gp mark 0}{(7.784,6.277)}
\gppoint{gp mark 0}{(6.238,4.419)}
\gppoint{gp mark 0}{(6.238,4.494)}
\gppoint{gp mark 0}{(7.259,5.814)}
\gppoint{gp mark 0}{(6.286,4.552)}
\gppoint{gp mark 0}{(6.189,4.430)}
\gppoint{gp mark 0}{(5.686,5.024)}
\gppoint{gp mark 0}{(6.254,4.514)}
\gppoint{gp mark 0}{(6.376,4.659)}
\gppoint{gp mark 0}{(7.253,6.818)}
\gppoint{gp mark 0}{(6.254,4.504)}
\gppoint{gp mark 0}{(6.136,4.349)}
\gppoint{gp mark 0}{(6.171,4.397)}
\gppoint{gp mark 0}{(6.205,4.441)}
\gppoint{gp mark 0}{(6.082,5.190)}
\gppoint{gp mark 0}{(6.390,4.667)}
\gppoint{gp mark 0}{(6.270,4.524)}
\gppoint{gp mark 0}{(6.362,4.633)}
\gppoint{gp mark 0}{(6.332,4.598)}
\gppoint{gp mark 0}{(7.204,5.780)}
\gppoint{gp mark 0}{(6.205,4.430)}
\gppoint{gp mark 0}{(7.273,5.826)}
\gppoint{gp mark 0}{(6.419,4.691)}
\gppoint{gp mark 0}{(6.119,4.361)}
\gppoint{gp mark 0}{(6.254,4.533)}
\gppoint{gp mark 0}{(6.222,4.494)}
\gppoint{gp mark 0}{(6.189,4.452)}
\gppoint{gp mark 0}{(6.063,5.177)}
\gppoint{gp mark 0}{(8.569,5.338)}
\gppoint{gp mark 0}{(6.537,4.843)}
\gppoint{gp mark 0}{(7.924,6.220)}
\gppoint{gp mark 0}{(6.270,4.504)}
\gppoint{gp mark 0}{(6.459,4.760)}
\gppoint{gp mark 0}{(7.579,5.450)}
\gppoint{gp mark 0}{(7.404,5.627)}
\gppoint{gp mark 0}{(5.712,5.019)}
\gppoint{gp mark 0}{(6.446,4.714)}
\gppoint{gp mark 0}{(5.944,5.118)}
\gppoint{gp mark 0}{(7.266,5.812)}
\gppoint{gp mark 0}{(6.136,4.325)}
\gppoint{gp mark 0}{(6.189,4.441)}
\gppoint{gp mark 0}{(6.432,4.730)}
\gppoint{gp mark 0}{(6.238,4.463)}
\gppoint{gp mark 0}{(6.405,4.699)}
\gppoint{gp mark 0}{(6.222,4.483)}
\gppoint{gp mark 0}{(6.473,4.745)}
\gppoint{gp mark 0}{(6.376,4.667)}
\gppoint{gp mark 0}{(6.270,4.642)}
\gppoint{gp mark 0}{(7.569,5.484)}
\gppoint{gp mark 0}{(6.332,4.571)}
\gppoint{gp mark 0}{(6.362,4.533)}
\gppoint{gp mark 0}{(6.238,4.361)}
\gppoint{gp mark 0}{(6.222,4.337)}
\gppoint{gp mark 0}{(6.189,4.385)}
\gppoint{gp mark 0}{(6.025,5.186)}
\gppoint{gp mark 0}{(7.218,5.766)}
\gppoint{gp mark 0}{(7.266,5.808)}
\gppoint{gp mark 0}{(5.633,5.019)}
\gppoint{gp mark 0}{(6.119,4.463)}
\gppoint{gp mark 0}{(5.576,5.040)}
\gppoint{gp mark 0}{(6.362,4.524)}
\gppoint{gp mark 0}{(6.270,4.633)}
\gppoint{gp mark 0}{(6.562,4.774)}
\gppoint{gp mark 0}{(6.512,4.829)}
\gppoint{gp mark 0}{(7.318,5.837)}
\gppoint{gp mark 0}{(6.238,4.337)}
\gppoint{gp mark 0}{(6.205,4.385)}
\gppoint{gp mark 0}{(6.171,4.430)}
\gppoint{gp mark 0}{(7.398,5.644)}
\gppoint{gp mark 0}{(6.486,4.875)}
\gppoint{gp mark 0}{(7.211,5.774)}
\gppoint{gp mark 0}{(6.525,4.829)}
\gppoint{gp mark 0}{(7.211,5.772)}
\gppoint{gp mark 0}{(7.204,5.776)}
\gppoint{gp mark 0}{(6.473,4.650)}
\gppoint{gp mark 0}{(6.499,4.856)}
\gppoint{gp mark 0}{(6.189,4.397)}
\gppoint{gp mark 0}{(6.254,4.633)}
\gppoint{gp mark 0}{(6.376,4.760)}
\gppoint{gp mark 0}{(6.205,4.373)}
\gppoint{gp mark 0}{(5.486,4.900)}
\gppoint{gp mark 0}{(5.738,5.003)}
\gppoint{gp mark 0}{(6.189,4.337)}
\gppoint{gp mark 0}{(6.459,4.691)}
\gppoint{gp mark 0}{(6.347,4.552)}
\gppoint{gp mark 0}{(6.171,4.494)}
\gppoint{gp mark 0}{(6.317,4.514)}
\gppoint{gp mark 0}{(6.499,4.849)}
\gppoint{gp mark 0}{(6.205,4.361)}
\gppoint{gp mark 0}{(6.775,3.986)}
\gppoint{gp mark 0}{(6.222,4.373)}
\gppoint{gp mark 0}{(6.317,4.504)}
\gppoint{gp mark 0}{(6.189,4.325)}
\gppoint{gp mark 0}{(6.238,4.397)}
\gppoint{gp mark 0}{(7.101,5.949)}
\gppoint{gp mark 0}{(6.270,4.598)}
\gppoint{gp mark 0}{(7.629,5.492)}
\gppoint{gp mark 0}{(6.171,4.483)}
\gppoint{gp mark 0}{(7.331,5.839)}
\gppoint{gp mark 0}{(7.559,5.742)}
\gppoint{gp mark 0}{(6.347,4.543)}
\gppoint{gp mark 0}{(6.574,4.816)}
\gppoint{gp mark 0}{(6.347,4.571)}
\gppoint{gp mark 0}{(6.205,4.337)}
\gppoint{gp mark 0}{(6.154,4.494)}
\gppoint{gp mark 0}{(6.238,4.385)}
\gppoint{gp mark 0}{(6.189,4.361)}
\gppoint{gp mark 0}{(7.279,5.806)}
\gppoint{gp mark 0}{(6.390,4.722)}
\gppoint{gp mark 0}{(6.550,4.781)}
\gppoint{gp mark 0}{(6.171,4.473)}
\gppoint{gp mark 0}{(7.500,5.700)}
\gppoint{gp mark 0}{(6.900,4.188)}
\gppoint{gp mark 0}{(6.171,4.463)}
\gppoint{gp mark 0}{(5.547,4.906)}
\gppoint{gp mark 0}{(6.270,4.580)}
\gppoint{gp mark 0}{(6.362,4.543)}
\gppoint{gp mark 0}{(6.189,4.349)}
\gppoint{gp mark 0}{(7.398,5.651)}
\gppoint{gp mark 0}{(6.347,4.562)}
\gppoint{gp mark 0}{(6.119,4.552)}
\gppoint{gp mark 0}{(6.082,5.104)}
\gppoint{gp mark 0}{(6.512,4.659)}
\gppoint{gp mark 0}{(5.605,4.924)}
\gppoint{gp mark 0}{(6.446,4.875)}
\gppoint{gp mark 0}{(6.171,4.533)}
\gppoint{gp mark 0}{(5.985,5.095)}
\gppoint{gp mark 0}{(6.537,4.752)}
\gppoint{gp mark 0}{(6.205,4.642)}
\gppoint{gp mark 0}{(6.446,4.869)}
\gppoint{gp mark 0}{(6.154,4.504)}
\gppoint{gp mark 0}{(6.537,4.745)}
\gppoint{gp mark 0}{(6.432,4.856)}
\gppoint{gp mark 0}{(6.286,4.325)}
\gppoint{gp mark 0}{(6.270,4.397)}
\gppoint{gp mark 0}{(6.419,4.788)}
\gppoint{gp mark 0}{(6.550,4.752)}
\gppoint{gp mark 0}{(7.700,5.573)}
\gppoint{gp mark 0}{(6.537,4.767)}
\gppoint{gp mark 0}{(6.136,4.552)}
\gppoint{gp mark 0}{(5.605,4.912)}
\gppoint{gp mark 0}{(6.270,4.385)}
\gppoint{gp mark 0}{(6.634,3.743)}
\gppoint{gp mark 0}{(6.025,5.065)}
\gppoint{gp mark 0}{(6.189,4.642)}
\gppoint{gp mark 0}{(7.246,5.770)}
\gppoint{gp mark 0}{(6.574,4.722)}
\gppoint{gp mark 0}{(6.044,5.132)}
\gppoint{gp mark 0}{(6.286,4.361)}
\gppoint{gp mark 0}{(8.146,6.904)}
\gppoint{gp mark 0}{(6.362,4.419)}
\gppoint{gp mark 0}{(6.238,4.580)}
\gppoint{gp mark 0}{(6.332,4.463)}
\gppoint{gp mark 0}{(6.154,4.524)}
\gppoint{gp mark 0}{(7.629,5.475)}
\gppoint{gp mark 0}{(6.171,4.504)}
\gppoint{gp mark 0}{(6.302,4.325)}
\gppoint{gp mark 0}{(7.422,5.680)}
\gppoint{gp mark 0}{(6.525,4.650)}
\gppoint{gp mark 0}{(6.459,4.843)}
\gppoint{gp mark 0}{(6.499,4.675)}
\gppoint{gp mark 0}{(6.136,4.533)}
\gppoint{gp mark 0}{(6.286,4.385)}
\gppoint{gp mark 0}{(6.486,4.659)}
\gppoint{gp mark 0}{(6.362,4.494)}
\gppoint{gp mark 0}{(6.405,4.809)}
\gppoint{gp mark 0}{(6.550,4.737)}
\gppoint{gp mark 0}{(6.222,4.625)}
\gppoint{gp mark 0}{(6.100,5.128)}
\gppoint{gp mark 0}{(5.787,5.137)}
\gppoint{gp mark 0}{(6.238,4.633)}
\gppoint{gp mark 0}{(6.254,4.325)}
\gppoint{gp mark 0}{(6.119,4.504)}
\gppoint{gp mark 0}{(5.660,4.918)}
\gppoint{gp mark 0}{(6.390,4.788)}
\gppoint{gp mark 0}{(6.082,5.118)}
\gppoint{gp mark 0}{(6.286,4.373)}
\gppoint{gp mark 0}{(7.374,5.851)}
\gppoint{gp mark 0}{(6.574,4.760)}
\gppoint{gp mark 0}{(6.136,4.524)}
\gppoint{gp mark 0}{(6.486,4.675)}
\gppoint{gp mark 0}{(6.286,4.408)}
\gppoint{gp mark 0}{(6.332,4.430)}
\gppoint{gp mark 0}{(6.362,4.473)}
\gppoint{gp mark 0}{(6.432,4.849)}
\gppoint{gp mark 0}{(6.499,4.659)}
\gppoint{gp mark 0}{(6.347,4.494)}
\gppoint{gp mark 0}{(5.547,5.045)}
\gppoint{gp mark 0}{(7.218,5.826)}
\gppoint{gp mark 0}{(6.189,4.598)}
\gppoint{gp mark 0}{(6.486,4.667)}
\gppoint{gp mark 0}{(6.222,4.633)}
\gppoint{gp mark 0}{(6.512,4.699)}
\gppoint{gp mark 0}{(6.171,4.543)}
\gppoint{gp mark 0}{(6.362,4.463)}
\gppoint{gp mark 0}{(6.302,4.373)}
\gppoint{gp mark 0}{(7.331,5.880)}
\gppoint{gp mark 0}{(6.205,4.580)}
\gppoint{gp mark 0}{(6.254,4.349)}
\gppoint{gp mark 0}{(6.317,4.441)}
\gppoint{gp mark 0}{(7.579,5.498)}
\gppoint{gp mark 0}{(6.286,4.397)}
\gppoint{gp mark 0}{(6.390,4.774)}
\gppoint{gp mark 0}{(6.882,3.986)}
\gppoint{gp mark 0}{(7.078,5.987)}
\gppoint{gp mark 0}{(6.189,4.552)}
\gppoint{gp mark 0}{(5.422,5.035)}
\gppoint{gp mark 0}{(6.347,4.337)}
\gppoint{gp mark 0}{(6.446,4.823)}
\gppoint{gp mark 0}{(6.432,4.809)}
\gppoint{gp mark 0}{(6.270,4.494)}
\gppoint{gp mark 0}{(6.254,4.473)}
\gppoint{gp mark 0}{(6.574,4.675)}
\gppoint{gp mark 0}{(7.176,5.822)}
\gppoint{gp mark 0}{(6.622,3.601)}
\gppoint{gp mark 0}{(6.254,4.463)}
\gppoint{gp mark 0}{(6.537,4.683)}
\gppoint{gp mark 0}{(7.589,5.504)}
\gppoint{gp mark 0}{(6.390,4.869)}
\gppoint{gp mark 0}{(6.205,4.562)}
\gppoint{gp mark 0}{(7.398,5.675)}
\gppoint{gp mark 0}{(6.317,4.373)}
\gppoint{gp mark 0}{(6.362,4.337)}
\gppoint{gp mark 0}{(6.270,4.473)}
\gppoint{gp mark 0}{(6.473,4.781)}
\gppoint{gp mark 0}{(6.347,4.361)}
\gppoint{gp mark 0}{(6.459,4.795)}
\gppoint{gp mark 0}{(6.100,5.065)}
\gppoint{gp mark 0}{(6.136,4.625)}
\gppoint{gp mark 0}{(6.512,4.737)}
\gppoint{gp mark 0}{(6.238,4.514)}
\gppoint{gp mark 0}{(6.499,4.752)}
\gppoint{gp mark 0}{(7.809,6.308)}
\gppoint{gp mark 0}{(6.574,4.659)}
\gppoint{gp mark 0}{(6.419,4.836)}
\gppoint{gp mark 0}{(7.589,5.501)}
\gppoint{gp mark 0}{(5.762,4.881)}
\gppoint{gp mark 0}{(6.270,4.463)}
\gppoint{gp mark 0}{(7.619,5.463)}
\gppoint{gp mark 0}{(6.486,4.760)}
\gppoint{gp mark 0}{(6.238,4.504)}
\gppoint{gp mark 0}{(6.332,4.373)}
\gppoint{gp mark 0}{(6.222,4.524)}
\gppoint{gp mark 0}{(7.253,5.780)}
\gppoint{gp mark 0}{(7.380,5.828)}
\gppoint{gp mark 0}{(6.390,4.849)}
\gppoint{gp mark 0}{(6.317,4.337)}
\gppoint{gp mark 0}{(6.222,4.552)}
\gppoint{gp mark 0}{(6.136,4.607)}
\gppoint{gp mark 0}{(6.805,4.110)}
\gppoint{gp mark 0}{(7.456,5.620)}
\gppoint{gp mark 0}{(6.189,4.514)}
\gppoint{gp mark 0}{(6.254,4.430)}
\gppoint{gp mark 0}{(6.270,4.452)}
\gppoint{gp mark 0}{(7.232,5.782)}
\gppoint{gp mark 0}{(6.537,4.659)}
\gppoint{gp mark 0}{(6.205,4.533)}
\gppoint{gp mark 0}{(6.486,4.722)}
\gppoint{gp mark 0}{(6.205,4.524)}
\gppoint{gp mark 0}{(5.686,4.881)}
\gppoint{gp mark 0}{(6.332,4.349)}
\gppoint{gp mark 0}{(6.171,4.633)}
\gppoint{gp mark 0}{(6.499,4.730)}
\gppoint{gp mark 0}{(6.486,4.714)}
\gppoint{gp mark 0}{(6.154,4.616)}
\gppoint{gp mark 0}{(7.392,5.680)}
\gppoint{gp mark 0}{(5.686,4.900)}
\gppoint{gp mark 0}{(6.189,4.533)}
\gppoint{gp mark 0}{(6.376,4.849)}
\gppoint{gp mark 0}{(5.738,4.924)}
\gppoint{gp mark 0}{(6.270,4.430)}
\gppoint{gp mark 0}{(6.136,4.589)}
\gppoint{gp mark 0}{(6.891,3.793)}
\gppoint{gp mark 0}{(6.486,4.737)}
\gppoint{gp mark 0}{(6.302,4.473)}
\gppoint{gp mark 0}{(6.119,4.607)}
\gppoint{gp mark 0}{(6.332,4.337)}
\gppoint{gp mark 0}{(6.254,4.452)}
\gppoint{gp mark 0}{(6.317,4.361)}
\gppoint{gp mark 0}{(6.525,4.752)}
\gppoint{gp mark 0}{(6.537,4.667)}
\gppoint{gp mark 0}{(6.512,4.760)}
\gppoint{gp mark 0}{(6.286,4.483)}
\gppoint{gp mark 0}{(6.136,4.580)}
\gppoint{gp mark 0}{(5.712,4.881)}
\gppoint{gp mark 0}{(6.082,5.090)}
\gppoint{gp mark 0}{(6.154,4.633)}
\gppoint{gp mark 0}{(7.758,6.259)}
\gppoint{gp mark 0}{(6.446,4.774)}
\gppoint{gp mark 0}{(6.362,4.373)}
\gppoint{gp mark 0}{(6.525,4.745)}
\gppoint{gp mark 0}{(6.459,4.816)}
\gppoint{gp mark 0}{(6.347,4.397)}
\gppoint{gp mark 0}{(6.317,4.349)}
\gppoint{gp mark 0}{(6.005,5.128)}
\gppoint{gp mark 0}{(6.432,4.473)}
\gppoint{gp mark 0}{(6.459,4.430)}
\gppoint{gp mark 0}{(6.562,4.589)}
\gppoint{gp mark 0}{(6.154,4.659)}
\gppoint{gp mark 0}{(6.317,4.862)}
\gppoint{gp mark 0}{(6.722,4.286)}
\gppoint{gp mark 0}{(6.390,4.408)}
\gppoint{gp mark 0}{(7.386,5.705)}
\gppoint{gp mark 0}{(6.419,4.361)}
\gppoint{gp mark 0}{(6.525,4.533)}
\gppoint{gp mark 0}{(8.865,5.792)}
\gppoint{gp mark 0}{(6.317,4.856)}
\gppoint{gp mark 0}{(6.712,4.126)}
\gppoint{gp mark 0}{(6.537,4.616)}
\gppoint{gp mark 0}{(6.205,4.760)}
\gppoint{gp mark 0}{(6.499,4.562)}
\gppoint{gp mark 0}{(6.254,4.802)}
\gppoint{gp mark 0}{(7.889,6.239)}
\gppoint{gp mark 0}{(6.270,4.816)}
\gppoint{gp mark 0}{(6.332,4.869)}
\gppoint{gp mark 0}{(6.419,4.349)}
\gppoint{gp mark 0}{(6.238,4.730)}
\gppoint{gp mark 0}{(6.347,4.849)}
\gppoint{gp mark 0}{(6.537,4.642)}
\gppoint{gp mark 0}{(6.154,4.675)}
\gppoint{gp mark 0}{(6.222,4.737)}
\gppoint{gp mark 0}{(6.512,4.533)}
\gppoint{gp mark 0}{(6.119,4.707)}
\gppoint{gp mark 0}{(6.405,4.361)}
\gppoint{gp mark 0}{(7.312,5.794)}
\gppoint{gp mark 0}{(7.619,5.578)}
\gppoint{gp mark 0}{(6.473,4.430)}
\gppoint{gp mark 0}{(6.362,4.836)}
\gppoint{gp mark 0}{(6.419,4.337)}
\gppoint{gp mark 0}{(6.722,4.312)}
\gppoint{gp mark 0}{(6.302,4.781)}
\gppoint{gp mark 0}{(6.302,4.774)}
\gppoint{gp mark 0}{(6.405,4.349)}
\gppoint{gp mark 0}{(6.154,4.667)}
\gppoint{gp mark 0}{(6.044,5.050)}
\gppoint{gp mark 0}{(6.419,4.325)}
\gppoint{gp mark 0}{(6.222,4.730)}
\gppoint{gp mark 0}{(6.486,4.562)}
\gppoint{gp mark 0}{(7.533,5.661)}
\gppoint{gp mark 0}{(7.259,5.884)}
\gppoint{gp mark 0}{(7.117,5.926)}
\gppoint{gp mark 0}{(6.119,4.659)}
\gppoint{gp mark 0}{(6.499,4.533)}
\gppoint{gp mark 0}{(6.701,4.173)}
\gppoint{gp mark 0}{(6.332,4.849)}
\gppoint{gp mark 0}{(6.376,4.337)}
\gppoint{gp mark 0}{(7.190,5.837)}
\gppoint{gp mark 0}{(6.154,4.683)}
\gppoint{gp mark 0}{(7.117,5.915)}
\gppoint{gp mark 0}{(6.390,4.349)}
\gppoint{gp mark 0}{(6.376,4.325)}
\gppoint{gp mark 0}{(6.286,4.802)}
\gppoint{gp mark 0}{(6.754,4.299)}
\gppoint{gp mark 0}{(5.857,4.918)}
\gppoint{gp mark 0}{(6.679,4.094)}
\gppoint{gp mark 0}{(6.189,4.714)}
\gppoint{gp mark 0}{(6.119,4.650)}
\gppoint{gp mark 0}{(6.473,4.483)}
\gppoint{gp mark 0}{(6.254,4.795)}
\gppoint{gp mark 0}{(6.390,4.337)}
\gppoint{gp mark 0}{(6.405,4.408)}
\gppoint{gp mark 0}{(6.486,4.533)}
\gppoint{gp mark 0}{(6.446,4.430)}
\gppoint{gp mark 0}{(6.459,4.494)}
\gppoint{gp mark 0}{(6.537,4.607)}
\gppoint{gp mark 0}{(6.119,4.675)}
\gppoint{gp mark 0}{(6.473,4.463)}
\gppoint{gp mark 0}{(6.459,4.483)}
\gppoint{gp mark 0}{(6.376,4.349)}
\gppoint{gp mark 0}{(6.238,4.745)}
\gppoint{gp mark 0}{(6.525,4.543)}
\gppoint{gp mark 0}{(6.512,4.562)}
\gppoint{gp mark 0}{(7.533,5.670)}
\gppoint{gp mark 0}{(6.390,4.325)}
\gppoint{gp mark 0}{(6.347,4.869)}
\gppoint{gp mark 0}{(6.486,4.524)}
\gppoint{gp mark 0}{(7.117,5.919)}
\gppoint{gp mark 0}{(6.154,4.699)}
\gppoint{gp mark 0}{(6.473,4.361)}
\gppoint{gp mark 0}{(6.525,4.607)}
\gppoint{gp mark 0}{(6.362,4.795)}
\gppoint{gp mark 0}{(6.562,4.514)}
\gppoint{gp mark 0}{(6.238,4.675)}
\gppoint{gp mark 0}{(7.350,5.814)}
\gppoint{gp mark 0}{(6.754,4.142)}
\gppoint{gp mark 0}{(6.376,4.473)}
\gppoint{gp mark 0}{(6.446,4.408)}
\gppoint{gp mark 0}{(7.362,5.806)}
\gppoint{gp mark 0}{(6.136,4.767)}
\gppoint{gp mark 0}{(7.554,5.656)}
\gppoint{gp mark 0}{(6.205,4.699)}
\gppoint{gp mark 0}{(7.386,5.720)}
\gppoint{gp mark 0}{(6.486,4.616)}
\gppoint{gp mark 0}{(6.302,4.843)}
\gppoint{gp mark 0}{(6.189,4.683)}
\gppoint{gp mark 0}{(6.317,4.802)}
\gppoint{gp mark 0}{(6.222,4.650)}
\gppoint{gp mark 0}{(7.467,5.729)}
\gppoint{gp mark 0}{(6.598,4.059)}
\gppoint{gp mark 0}{(6.754,4.126)}
\gppoint{gp mark 0}{(6.525,4.598)}
\gppoint{gp mark 0}{(7.858,6.961)}
\gppoint{gp mark 0}{(6.222,4.675)}
\gppoint{gp mark 0}{(5.923,4.900)}
\gppoint{gp mark 0}{(6.562,4.533)}
\gppoint{gp mark 0}{(6.550,4.552)}
\gppoint{gp mark 0}{(6.286,4.849)}
\gppoint{gp mark 0}{(7.350,5.818)}
\gppoint{gp mark 0}{(6.317,4.823)}
\gppoint{gp mark 0}{(6.376,4.494)}
\gppoint{gp mark 0}{(7.337,5.824)}
\gppoint{gp mark 0}{(6.537,4.562)}
\gppoint{gp mark 0}{(6.405,4.441)}
\gppoint{gp mark 0}{(6.512,4.598)}
\gppoint{gp mark 0}{(6.419,4.419)}
\gppoint{gp mark 0}{(6.499,4.616)}
\gppoint{gp mark 0}{(6.238,4.650)}
\gppoint{gp mark 0}{(6.586,4.059)}
\gppoint{gp mark 0}{(6.432,4.397)}
\gppoint{gp mark 0}{(6.459,4.349)}
\gppoint{gp mark 0}{(6.332,4.802)}
\gppoint{gp mark 0}{(6.512,4.625)}
\gppoint{gp mark 0}{(6.525,4.642)}
\gppoint{gp mark 0}{(5.810,4.947)}
\gppoint{gp mark 0}{(6.390,4.452)}
\gppoint{gp mark 0}{(6.286,4.862)}
\gppoint{gp mark 0}{(6.574,4.571)}
\gppoint{gp mark 0}{(7.331,5.778)}
\gppoint{gp mark 0}{(7.147,5.917)}
\gppoint{gp mark 0}{(6.668,3.927)}
\gppoint{gp mark 0}{(6.459,4.385)}
\gppoint{gp mark 0}{(7.078,5.914)}
\gppoint{gp mark 0}{(6.205,4.667)}
\gppoint{gp mark 0}{(6.222,4.683)}
\gppoint{gp mark 0}{(7.374,5.804)}
\gppoint{gp mark 0}{(6.486,4.580)}
\gppoint{gp mark 0}{(7.190,5.851)}
\gppoint{gp mark 0}{(6.459,4.373)}
\gppoint{gp mark 0}{(6.405,4.463)}
\gppoint{gp mark 0}{(6.499,4.598)}
\gppoint{gp mark 0}{(6.537,4.504)}
\gppoint{gp mark 0}{(6.025,5.050)}
\gppoint{gp mark 0}{(6.362,4.816)}
\gppoint{gp mark 0}{(6.525,4.633)}
\gppoint{gp mark 0}{(5.857,4.965)}
\gppoint{gp mark 0}{(6.432,4.325)}
\gppoint{gp mark 0}{(6.446,4.349)}
\gppoint{gp mark 0}{(6.712,4.299)}
\gppoint{gp mark 0}{(6.954,5.943)}
\gppoint{gp mark 0}{(6.376,4.419)}
\gppoint{gp mark 0}{(6.171,4.760)}
\gppoint{gp mark 0}{(6.119,4.714)}
\gppoint{gp mark 0}{(6.317,4.774)}
\gppoint{gp mark 0}{(7.292,5.784)}
\gppoint{gp mark 0}{(6.390,4.441)}
\gppoint{gp mark 0}{(6.222,4.707)}
\gppoint{gp mark 0}{(6.025,5.045)}
\gppoint{gp mark 0}{(6.473,4.385)}
\gppoint{gp mark 0}{(6.405,4.494)}
\gppoint{gp mark 0}{(6.171,4.752)}
\gppoint{gp mark 0}{(6.317,4.795)}
\gppoint{gp mark 0}{(7.433,5.753)}
\gppoint{gp mark 0}{(6.376,4.452)}
\gppoint{gp mark 0}{(6.432,4.361)}
\gppoint{gp mark 0}{(6.459,4.408)}
\gppoint{gp mark 0}{(7.380,5.806)}
\gppoint{gp mark 0}{(7.392,5.714)}
\gppoint{gp mark 0}{(7.232,5.878)}
\gppoint{gp mark 0}{(6.446,4.325)}
\gppoint{gp mark 0}{(6.499,4.580)}
\gppoint{gp mark 0}{(6.679,4.246)}
\gppoint{gp mark 0}{(7.331,5.772)}
\gppoint{gp mark 0}{(6.432,4.349)}
\gppoint{gp mark 0}{(7.686,5.510)}
\gppoint{gp mark 0}{(7.211,5.828)}
\gppoint{gp mark 0}{(7.467,5.742)}
\gppoint{gp mark 0}{(6.205,4.650)}
\gppoint{gp mark 0}{(6.562,4.430)}
\gppoint{gp mark 0}{(6.063,4.970)}
\gppoint{gp mark 0}{(6.238,4.849)}
\gppoint{gp mark 0}{(6.332,4.767)}
\gppoint{gp mark 0}{(6.254,4.691)}
\gppoint{gp mark 0}{(7.266,5.856)}
\gppoint{gp mark 0}{(7.450,5.700)}
\gppoint{gp mark 0}{(7.624,5.565)}
\gppoint{gp mark 0}{(6.405,4.504)}
\gppoint{gp mark 0}{(7.197,5.862)}
\gppoint{gp mark 0}{(6.376,4.543)}
\gppoint{gp mark 0}{(6.317,4.745)}
\gppoint{gp mark 0}{(6.634,4.273)}
\gppoint{gp mark 0}{(6.189,4.856)}
\gppoint{gp mark 0}{(6.562,4.419)}
\gppoint{gp mark 0}{(6.574,4.441)}
\gppoint{gp mark 0}{(6.656,4.217)}
\gppoint{gp mark 0}{(6.254,4.683)}
\gppoint{gp mark 0}{(7.183,5.867)}
\gppoint{gp mark 0}{(5.985,4.912)}
\gppoint{gp mark 0}{(6.347,4.737)}
\gppoint{gp mark 0}{(6.238,4.836)}
\gppoint{gp mark 0}{(6.025,4.887)}
\gppoint{gp mark 0}{(5.901,5.045)}
\gppoint{gp mark 0}{(6.486,4.408)}
\gppoint{gp mark 0}{(6.432,4.642)}
\gppoint{gp mark 0}{(6.668,4.232)}
\gppoint{gp mark 0}{(6.512,4.361)}
\gppoint{gp mark 0}{(6.550,4.473)}
\gppoint{gp mark 0}{(6.362,4.722)}
\gppoint{gp mark 0}{(6.222,4.849)}
\gppoint{gp mark 0}{(6.254,4.699)}
\gppoint{gp mark 0}{(6.286,4.667)}
\gppoint{gp mark 0}{(6.432,4.633)}
\gppoint{gp mark 0}{(7.543,5.651)}
\gppoint{gp mark 0}{(6.882,3.104)}
\gppoint{gp mark 0}{(7.920,6.198)}
\gppoint{gp mark 0}{(6.574,4.419)}
\gppoint{gp mark 0}{(6.499,4.373)}
\gppoint{gp mark 0}{(6.562,4.441)}
\gppoint{gp mark 0}{(6.486,4.397)}
\gppoint{gp mark 0}{(6.419,4.504)}
\gppoint{gp mark 0}{(6.405,4.524)}
\gppoint{gp mark 0}{(6.473,4.580)}
\gppoint{gp mark 0}{(6.550,4.463)}
\gppoint{gp mark 0}{(6.459,4.598)}
\gppoint{gp mark 0}{(6.119,4.816)}
\gppoint{gp mark 0}{(6.347,4.730)}
\gppoint{gp mark 0}{(8.586,5.259)}
\gppoint{gp mark 0}{(6.270,4.675)}
\gppoint{gp mark 0}{(6.154,4.809)}
\gppoint{gp mark 0}{(7.218,5.882)}
\gppoint{gp mark 0}{(6.525,4.408)}
\gppoint{gp mark 0}{(6.622,4.203)}
\gppoint{gp mark 0}{(6.610,4.173)}
\gppoint{gp mark 0}{(7.312,5.814)}
\gppoint{gp mark 0}{(7.259,5.845)}
\gppoint{gp mark 0}{(6.446,4.607)}
\gppoint{gp mark 0}{(6.270,4.667)}
\gppoint{gp mark 0}{(6.222,4.856)}
\gppoint{gp mark 0}{(7.005,5.949)}
\gppoint{gp mark 0}{(6.317,4.714)}
\gppoint{gp mark 0}{(5.834,4.998)}
\gppoint{gp mark 0}{(5.901,5.030)}
\gppoint{gp mark 0}{(6.390,4.524)}
\gppoint{gp mark 0}{(6.302,4.699)}
\gppoint{gp mark 0}{(6.512,4.373)}
\gppoint{gp mark 0}{(7.183,5.862)}
\gppoint{gp mark 0}{(6.550,4.441)}
\gppoint{gp mark 0}{(6.044,4.930)}
\gppoint{gp mark 0}{(6.254,4.650)}
\gppoint{gp mark 0}{(6.459,4.616)}
\gppoint{gp mark 0}{(6.205,4.843)}
\gppoint{gp mark 0}{(6.473,4.633)}
\gppoint{gp mark 0}{(6.376,4.504)}
\gppoint{gp mark 0}{(7.456,5.712)}
\gppoint{gp mark 0}{(6.025,4.918)}
\gppoint{gp mark 0}{(6.362,4.752)}
\gppoint{gp mark 0}{(7.368,5.782)}
\gppoint{gp mark 0}{(7.190,5.871)}
\gppoint{gp mark 0}{(5.660,5.085)}
\gppoint{gp mark 0}{(5.857,5.003)}
\gppoint{gp mark 0}{(6.668,4.286)}
\gppoint{gp mark 0}{(6.419,4.552)}
\gppoint{gp mark 0}{(6.390,4.514)}
\gppoint{gp mark 0}{(6.063,4.936)}
\gppoint{gp mark 0}{(6.270,4.659)}
\gppoint{gp mark 0}{(5.923,5.055)}
\gppoint{gp mark 0}{(6.254,4.675)}
\gppoint{gp mark 0}{(6.550,4.430)}
\gppoint{gp mark 0}{(6.512,4.408)}
\gppoint{gp mark 0}{(6.332,4.722)}
\gppoint{gp mark 0}{(6.525,4.385)}
\gppoint{gp mark 0}{(6.656,4.312)}
\gppoint{gp mark 0}{(6.171,4.809)}
\gppoint{gp mark 0}{(7.286,5.802)}
\gppoint{gp mark 0}{(7.433,5.700)}
\gppoint{gp mark 0}{(6.574,4.463)}
\gppoint{gp mark 0}{(6.376,4.524)}
\gppoint{gp mark 0}{(7.681,5.444)}
\gppoint{gp mark 0}{(5.944,5.040)}
\gppoint{gp mark 0}{(6.405,4.562)}
\gppoint{gp mark 0}{(6.473,4.616)}
\gppoint{gp mark 0}{(6.332,4.714)}
\gppoint{gp mark 0}{(6.512,4.397)}
\gppoint{gp mark 0}{(7.445,5.707)}
\gppoint{gp mark 0}{(7.629,5.552)}
\gppoint{gp mark 0}{(6.574,4.361)}
\gppoint{gp mark 0}{(7.404,5.753)}
\gppoint{gp mark 0}{(7.858,6.249)}
\gppoint{gp mark 0}{(6.419,4.607)}
\gppoint{gp mark 0}{(6.610,4.232)}
\gppoint{gp mark 0}{(6.136,4.875)}
\gppoint{gp mark 0}{(6.550,4.408)}
\gppoint{gp mark 0}{(6.473,4.533)}
\gppoint{gp mark 0}{(6.332,4.707)}
\gppoint{gp mark 0}{(6.362,4.675)}
\gppoint{gp mark 0}{(6.562,4.337)}
\gppoint{gp mark 0}{(5.879,4.998)}
\gppoint{gp mark 0}{(6.610,4.217)}
\gppoint{gp mark 0}{(6.205,4.816)}
\gppoint{gp mark 0}{(6.512,4.419)}
\gppoint{gp mark 0}{(7.305,5.816)}
\gppoint{gp mark 0}{(6.586,4.273)}
\gppoint{gp mark 0}{(6.473,4.524)}
\gppoint{gp mark 0}{(6.712,3.986)}
\gppoint{gp mark 0}{(6.432,4.543)}
\gppoint{gp mark 0}{(6.376,4.616)}
\gppoint{gp mark 0}{(6.332,4.699)}
\gppoint{gp mark 0}{(6.254,4.745)}
\gppoint{gp mark 0}{(7.386,5.755)}
\gppoint{gp mark 0}{(6.302,4.730)}
\gppoint{gp mark 0}{(6.598,4.286)}
\gppoint{gp mark 0}{(6.525,4.430)}
\gppoint{gp mark 0}{(5.857,5.024)}
\gppoint{gp mark 0}{(5.738,5.075)}
\gppoint{gp mark 0}{(6.574,4.337)}
\gppoint{gp mark 0}{(6.317,4.707)}
\gppoint{gp mark 0}{(6.362,4.659)}
\gppoint{gp mark 0}{(6.286,4.737)}
\gppoint{gp mark 0}{(6.205,4.809)}
\gppoint{gp mark 0}{(6.486,4.494)}
\gppoint{gp mark 0}{(6.254,4.767)}
\gppoint{gp mark 0}{(6.622,4.232)}
\gppoint{gp mark 0}{(7.609,5.568)}
\gppoint{gp mark 0}{(6.586,4.312)}
\gppoint{gp mark 0}{(6.005,4.947)}
\gppoint{gp mark 0}{(6.332,4.691)}
\gppoint{gp mark 0}{(6.347,4.667)}
\gppoint{gp mark 0}{(6.537,4.397)}
\gppoint{gp mark 0}{(7.639,5.532)}
\gppoint{gp mark 0}{(6.512,4.441)}
\gppoint{gp mark 0}{(6.254,4.760)}
\gppoint{gp mark 0}{(6.446,4.543)}
\gppoint{gp mark 0}{(5.944,4.976)}
\gppoint{gp mark 0}{(6.238,4.774)}
\gppoint{gp mark 0}{(7.246,5.847)}
\gppoint{gp mark 0}{(6.668,4.094)}
\gppoint{gp mark 0}{(6.459,4.524)}
\gppoint{gp mark 0}{(5.834,5.030)}
\gppoint{gp mark 0}{(6.574,4.325)}
\gppoint{gp mark 0}{(7.648,5.521)}
\gppoint{gp mark 0}{(6.432,4.562)}
\gppoint{gp mark 0}{(7.714,5.459)}
\gppoint{gp mark 0}{(6.537,4.337)}
\gppoint{gp mark 0}{(7.343,5.786)}
\gppoint{gp mark 0}{(6.254,4.722)}
\gppoint{gp mark 0}{(6.390,4.607)}
\gppoint{gp mark 0}{(6.525,4.494)}
\gppoint{gp mark 0}{(6.459,4.552)}
\gppoint{gp mark 0}{(6.586,4.232)}
\gppoint{gp mark 0}{(6.362,4.707)}
\gppoint{gp mark 0}{(6.574,4.408)}
\gppoint{gp mark 0}{(6.286,4.752)}
\gppoint{gp mark 0}{(6.222,4.809)}
\gppoint{gp mark 0}{(7.154,5.893)}
\gppoint{gp mark 0}{(7.462,5.698)}
\gppoint{gp mark 0}{(6.376,4.580)}
\gppoint{gp mark 0}{(6.537,4.325)}
\gppoint{gp mark 0}{(5.686,5.060)}
\gppoint{gp mark 0}{(6.419,4.633)}
\gppoint{gp mark 0}{(6.486,4.419)}
\gppoint{gp mark 0}{(6.405,4.616)}
\gppoint{gp mark 0}{(6.238,4.816)}
\gppoint{gp mark 0}{(7.589,5.581)}
\gppoint{gp mark 0}{(6.286,4.745)}
\gppoint{gp mark 0}{(7.548,5.622)}
\gppoint{gp mark 0}{(6.499,4.441)}
\gppoint{gp mark 0}{(6.473,4.552)}
\gppoint{gp mark 0}{(6.598,4.232)}
\gppoint{gp mark 0}{(6.574,4.385)}
\gppoint{gp mark 0}{(7.600,5.589)}
\gppoint{gp mark 0}{(6.270,4.722)}
\gppoint{gp mark 0}{(6.824,3.503)}
\gppoint{gp mark 0}{(6.562,4.408)}
\gppoint{gp mark 0}{(7.343,5.782)}
\gppoint{gp mark 0}{(6.622,4.286)}
\gppoint{gp mark 0}{(7.834,6.265)}
\gppoint{gp mark 0}{(5.762,5.080)}
\gppoint{gp mark 0}{(7.176,5.876)}
\gppoint{gp mark 0}{(6.459,4.562)}
\gppoint{gp mark 0}{(6.499,4.419)}
\gppoint{gp mark 0}{(7.473,5.703)}
\gppoint{gp mark 0}{(6.446,4.504)}
\gppoint{gp mark 0}{(7.416,5.729)}
\gppoint{gp mark 0}{(6.063,4.881)}
\gppoint{gp mark 0}{(6.525,4.463)}
\gppoint{gp mark 0}{(6.432,4.524)}
\gppoint{gp mark 0}{(6.550,4.325)}
\gppoint{gp mark 0}{(6.390,4.580)}
\gppoint{gp mark 0}{(6.405,4.633)}
\gppoint{gp mark 0}{(6.537,4.349)}
\gppoint{gp mark 0}{(6.562,4.397)}
\gppoint{gp mark 0}{(6.286,4.760)}
\gppoint{gp mark 0}{(7.479,5.680)}
\gppoint{gp mark 0}{(7.704,5.435)}
\gppoint{gp mark 0}{(7.484,5.095)}
\gppoint{gp mark 0}{(7.176,4.385)}
\gppoint{gp mark 0}{(7.273,4.589)}
\gppoint{gp mark 0}{(7.362,4.862)}
\gppoint{gp mark 0}{(7.299,4.650)}
\gppoint{gp mark 0}{(6.971,6.561)}
\gppoint{gp mark 0}{(7.579,5.211)}
\gppoint{gp mark 0}{(7.427,4.936)}
\gppoint{gp mark 0}{(7.292,4.691)}
\gppoint{gp mark 0}{(7.259,4.633)}
\gppoint{gp mark 0}{(7.362,4.869)}
\gppoint{gp mark 0}{(7.286,4.659)}
\gppoint{gp mark 0}{(7.427,4.970)}
\gppoint{gp mark 0}{(7.439,4.987)}
\gppoint{gp mark 0}{(7.337,4.774)}
\gppoint{gp mark 0}{(7.386,4.881)}
\gppoint{gp mark 0}{(7.312,4.737)}
\gppoint{gp mark 0}{(7.362,4.849)}
\gppoint{gp mark 0}{(7.324,4.760)}
\gppoint{gp mark 0}{(7.350,4.816)}
\gppoint{gp mark 0}{(7.386,4.894)}
\gppoint{gp mark 0}{(7.445,6.664)}
\gppoint{gp mark 0}{(7.225,4.361)}
\gppoint{gp mark 0}{(7.771,6.342)}
\gppoint{gp mark 0}{(7.427,4.900)}
\gppoint{gp mark 0}{(7.462,5.014)}
\gppoint{gp mark 0}{(7.614,5.301)}
\gppoint{gp mark 0}{(7.595,5.219)}
\gppoint{gp mark 0}{(6.459,5.833)}
\gppoint{gp mark 0}{(7.427,4.881)}
\gppoint{gp mark 0}{(7.667,5.345)}
\gppoint{gp mark 0}{(7.522,5.198)}
\gppoint{gp mark 0}{(7.350,4.862)}
\gppoint{gp mark 0}{(7.343,4.849)}
\gppoint{gp mark 0}{(7.337,4.836)}
\gppoint{gp mark 0}{(7.292,4.730)}
\gppoint{gp mark 0}{(7.337,4.829)}
\gppoint{gp mark 0}{(7.350,4.856)}
\gppoint{gp mark 0}{(7.548,6.678)}
\gppoint{gp mark 0}{(7.374,4.823)}
\gppoint{gp mark 0}{(7.754,6.348)}
\gppoint{gp mark 0}{(7.183,4.419)}
\gppoint{gp mark 0}{(7.878,6.390)}
\gppoint{gp mark 0}{(7.511,5.090)}
\gppoint{gp mark 0}{(7.484,5.100)}
\gppoint{gp mark 0}{(7.522,5.181)}
\gppoint{gp mark 0}{(7.672,5.352)}
\gppoint{gp mark 0}{(7.427,4.906)}
\gppoint{gp mark 0}{(7.318,4.875)}
\gppoint{gp mark 0}{(7.324,4.836)}
\gppoint{gp mark 0}{(7.600,6.705)}
\gppoint{gp mark 0}{(7.467,4.930)}
\gppoint{gp mark 0}{(7.312,4.856)}
\gppoint{gp mark 0}{(7.672,5.422)}
\gppoint{gp mark 0}{(7.266,4.473)}
\gppoint{gp mark 0}{(7.374,4.730)}
\gppoint{gp mark 0}{(7.866,6.406)}
\gppoint{gp mark 0}{(7.218,4.625)}
\gppoint{gp mark 0}{(7.473,4.970)}
\gppoint{gp mark 0}{(7.318,4.843)}
\gppoint{gp mark 0}{(7.350,4.683)}
\gppoint{gp mark 0}{(7.584,5.312)}
\gppoint{gp mark 0}{(7.439,4.887)}
\gppoint{gp mark 0}{(7.850,6.366)}
\gppoint{gp mark 0}{(8.140,7.432)}
\gppoint{gp mark 0}{(7.784,7.001)}
\gppoint{gp mark 0}{(9.726,6.939)}
\gppoint{gp mark 0}{(6.834,5.988)}
\gppoint{gp mark 0}{(7.543,5.109)}
\gppoint{gp mark 0}{(7.368,4.707)}
\gppoint{gp mark 0}{(7.331,4.795)}
\gppoint{gp mark 0}{(7.445,4.930)}
\gppoint{gp mark 0}{(7.404,5.030)}
\gppoint{gp mark 0}{(7.506,5.159)}
\gppoint{gp mark 0}{(7.605,5.267)}
\gppoint{gp mark 0}{(7.356,4.722)}
\gppoint{gp mark 0}{(7.589,5.290)}
\gppoint{gp mark 0}{(7.331,4.781)}
\gppoint{gp mark 0}{(7.695,5.362)}
\gppoint{gp mark 0}{(7.299,4.849)}
\gppoint{gp mark 0}{(8.795,7.127)}
\gppoint{gp mark 0}{(7.331,4.774)}
\gppoint{gp mark 0}{(7.527,5.118)}
\gppoint{gp mark 0}{(7.305,4.829)}
\gppoint{gp mark 0}{(7.204,4.562)}
\gppoint{gp mark 0}{(8.131,6.501)}
\gppoint{gp mark 0}{(7.197,4.642)}
\gppoint{gp mark 0}{(7.462,4.894)}
\gppoint{gp mark 0}{(7.318,4.788)}
\gppoint{gp mark 0}{(7.305,4.869)}
\gppoint{gp mark 0}{(7.619,5.259)}
\gppoint{gp mark 0}{(7.176,4.607)}
\gppoint{gp mark 0}{(7.569,5.305)}
\gppoint{gp mark 0}{(7.506,5.137)}
\gppoint{gp mark 0}{(7.667,5.386)}
\gppoint{gp mark 0}{(7.538,5.118)}
\gppoint{gp mark 0}{(7.273,4.836)}
\gppoint{gp mark 0}{(7.543,5.045)}
\gppoint{gp mark 0}{(7.559,5.035)}
\gppoint{gp mark 0}{(7.279,4.849)}
\gppoint{gp mark 0}{(7.422,5.104)}
\gppoint{gp mark 0}{(7.511,6.653)}
\gppoint{gp mark 0}{(8.195,7.567)}
\gppoint{gp mark 0}{(7.858,6.420)}
\gppoint{gp mark 0}{(7.490,4.881)}
\gppoint{gp mark 0}{(6.891,5.939)}
\gppoint{gp mark 0}{(7.433,5.168)}
\gppoint{gp mark 0}{(7.709,5.309)}
\gppoint{gp mark 0}{(7.176,4.699)}
\gppoint{gp mark 0}{(7.218,4.730)}
\gppoint{gp mark 0}{(7.266,4.849)}
\gppoint{gp mark 0}{(7.204,4.714)}
\gppoint{gp mark 0}{(7.279,4.869)}
\gppoint{gp mark 0}{(7.218,4.745)}
\gppoint{gp mark 0}{(7.392,5.070)}
\gppoint{gp mark 0}{(7.374,4.642)}
\gppoint{gp mark 0}{(7.427,5.123)}
\gppoint{gp mark 0}{(7.246,4.823)}
\gppoint{gp mark 0}{(7.445,5.168)}
\gppoint{gp mark 0}{(7.473,5.190)}
\gppoint{gp mark 0}{(7.484,4.881)}
\gppoint{gp mark 0}{(7.273,4.869)}
\gppoint{gp mark 0}{(7.564,5.338)}
\gppoint{gp mark 0}{(7.259,4.809)}
\gppoint{gp mark 0}{(7.337,4.625)}
\gppoint{gp mark 0}{(7.356,4.607)}
\gppoint{gp mark 0}{(7.548,5.014)}
\gppoint{gp mark 0}{(7.253,4.843)}
\gppoint{gp mark 0}{(7.511,4.881)}
\gppoint{gp mark 0}{(7.579,5.366)}
\gppoint{gp mark 0}{(7.410,5.095)}
\gppoint{gp mark 0}{(7.259,4.823)}
\gppoint{gp mark 0}{(7.392,6.674)}
\gppoint{gp mark 0}{(7.495,4.930)}
\gppoint{gp mark 0}{(7.771,6.375)}
\gppoint{gp mark 0}{(7.046,4.094)}
\gppoint{gp mark 0}{(7.714,5.298)}
\gppoint{gp mark 0}{(7.648,5.252)}
\gppoint{gp mark 0}{(7.704,5.275)}
\gppoint{gp mark 0}{(7.176,4.737)}
\gppoint{gp mark 0}{(7.462,5.142)}
\gppoint{gp mark 0}{(9.012,6.337)}
\gppoint{gp mark 0}{(7.204,4.667)}
\gppoint{gp mark 0}{(6.997,4.217)}
\gppoint{gp mark 0}{(7.533,5.050)}
\gppoint{gp mark 0}{(7.456,5.146)}
\gppoint{gp mark 0}{(7.239,4.707)}
\gppoint{gp mark 0}{(7.538,4.900)}
\gppoint{gp mark 0}{(7.456,5.123)}
\gppoint{gp mark 0}{(7.506,5.055)}
\gppoint{gp mark 0}{(7.176,4.802)}
\gppoint{gp mark 0}{(7.681,5.223)}
\gppoint{gp mark 0}{(6.525,5.768)}
\gppoint{gp mark 0}{(7.672,5.301)}
\gppoint{gp mark 0}{(7.644,5.286)}
\gppoint{gp mark 0}{(7.653,5.271)}
\gppoint{gp mark 0}{(7.629,6.717)}
\gppoint{gp mark 0}{(8.025,6.481)}
\gppoint{gp mark 0}{(7.259,4.767)}
\gppoint{gp mark 0}{(7.286,4.562)}
\gppoint{gp mark 0}{(7.190,4.788)}
\gppoint{gp mark 0}{(7.266,4.745)}
\gppoint{gp mark 0}{(7.259,4.760)}
\gppoint{gp mark 0}{(7.624,6.717)}
\gppoint{gp mark 0}{(7.681,5.211)}
\gppoint{gp mark 0}{(7.305,4.571)}
\gppoint{gp mark 0}{(7.564,5.386)}
\gppoint{gp mark 0}{(7.273,4.745)}
\gppoint{gp mark 0}{(6.971,6.589)}
\gppoint{gp mark 0}{(7.495,5.003)}
\gppoint{gp mark 0}{(7.543,4.947)}
\gppoint{gp mark 0}{(7.517,5.045)}
\gppoint{gp mark 0}{(7.279,4.752)}
\gppoint{gp mark 0}{(7.253,4.683)}
\gppoint{gp mark 0}{(7.629,5.359)}
\gppoint{gp mark 0}{(7.279,4.745)}
\gppoint{gp mark 0}{(7.183,6.644)}
\gppoint{gp mark 0}{(7.266,4.707)}
\gppoint{gp mark 0}{(7.584,5.403)}
\gppoint{gp mark 0}{(7.445,5.104)}
\gppoint{gp mark 0}{(7.218,4.781)}
\gppoint{gp mark 0}{(6.918,4.273)}
\gppoint{gp mark 0}{(7.246,4.714)}
\gppoint{gp mark 0}{(7.253,4.730)}
\gppoint{gp mark 0}{(7.589,5.403)}
\gppoint{gp mark 0}{(7.259,4.707)}
\gppoint{gp mark 0}{(7.559,4.887)}
\gppoint{gp mark 0}{(7.404,5.173)}
\gppoint{gp mark 0}{(7.232,4.722)}
\gppoint{gp mark 0}{(7.246,4.752)}
\gppoint{gp mark 0}{(7.450,5.132)}
\gppoint{gp mark 0}{(6.936,4.273)}
\gppoint{gp mark 0}{(7.439,5.109)}
\gppoint{gp mark 0}{(7.433,5.100)}
\gppoint{gp mark 0}{(7.204,4.774)}
\gppoint{gp mark 0}{(7.500,4.976)}
\gppoint{gp mark 0}{(7.356,4.483)}
\gppoint{gp mark 0}{(7.595,5.399)}
\gppoint{gp mark 0}{(7.462,5.065)}
\gppoint{gp mark 0}{(7.559,4.912)}
\gppoint{gp mark 0}{(7.239,4.722)}
\gppoint{gp mark 0}{(7.211,4.774)}
\gppoint{gp mark 0}{(8.652,5.599)}
\gppoint{gp mark 0}{(7.704,5.203)}
\gppoint{gp mark 0}{(7.101,4.722)}
\gppoint{gp mark 0}{(7.589,4.970)}
\gppoint{gp mark 0}{(7.605,5.003)}
\gppoint{gp mark 0}{(7.681,5.155)}
\gppoint{gp mark 0}{(6.988,4.543)}
\gppoint{gp mark 0}{(7.538,5.393)}
\gppoint{gp mark 0}{(7.005,4.533)}
\gppoint{gp mark 0}{(7.054,4.707)}
\gppoint{gp mark 0}{(7.312,6.582)}
\gppoint{gp mark 0}{(7.169,4.829)}
\gppoint{gp mark 0}{(7.101,4.730)}
\gppoint{gp mark 0}{(7.078,4.650)}
\gppoint{gp mark 0}{(7.450,5.271)}
\gppoint{gp mark 0}{(7.038,4.598)}
\gppoint{gp mark 0}{(6.945,4.408)}
\gppoint{gp mark 0}{(7.109,4.767)}
\gppoint{gp mark 0}{(6.997,4.533)}
\gppoint{gp mark 0}{(7.109,4.760)}
\gppoint{gp mark 0}{(7.094,4.730)}
\gppoint{gp mark 0}{(6.954,4.419)}
\gppoint{gp mark 0}{(7.473,5.323)}
\gppoint{gp mark 0}{(7.030,4.589)}
\gppoint{gp mark 0}{(7.046,4.625)}
\gppoint{gp mark 0}{(7.109,4.752)}
\gppoint{gp mark 0}{(7.022,4.607)}
\gppoint{gp mark 0}{(7.147,4.849)}
\gppoint{gp mark 0}{(7.038,4.642)}
\gppoint{gp mark 0}{(8.108,6.455)}
\gppoint{gp mark 0}{(7.005,4.562)}
\gppoint{gp mark 0}{(7.139,4.802)}
\gppoint{gp mark 0}{(7.450,5.286)}
\gppoint{gp mark 0}{(6.918,4.473)}
\gppoint{gp mark 0}{(7.427,5.219)}
\gppoint{gp mark 0}{(7.490,5.362)}
\gppoint{gp mark 0}{(7.030,4.571)}
\gppoint{gp mark 0}{(7.101,4.659)}
\gppoint{gp mark 0}{(6.679,5.822)}
\gppoint{gp mark 0}{(7.386,5.256)}
\gppoint{gp mark 0}{(7.062,4.760)}
\gppoint{gp mark 0}{(7.070,4.714)}
\gppoint{gp mark 0}{(8.042,6.534)}
\gppoint{gp mark 0}{(7.398,5.240)}
\gppoint{gp mark 0}{(6.154,5.901)}
\gppoint{gp mark 0}{(7.132,4.829)}
\gppoint{gp mark 0}{(6.872,5.876)}
\gppoint{gp mark 0}{(7.559,5.393)}
\gppoint{gp mark 0}{(7.821,6.443)}
\gppoint{gp mark 0}{(7.117,4.875)}
\gppoint{gp mark 0}{(7.467,5.283)}
\gppoint{gp mark 0}{(7.147,4.823)}
\gppoint{gp mark 0}{(6.918,4.494)}
\gppoint{gp mark 0}{(7.022,4.562)}
\gppoint{gp mark 0}{(7.479,5.379)}
\gppoint{gp mark 0}{(7.124,4.856)}
\gppoint{gp mark 0}{(7.416,5.223)}
\gppoint{gp mark 0}{(7.161,4.788)}
\gppoint{gp mark 0}{(7.101,4.667)}
\gppoint{gp mark 0}{(7.548,5.396)}
\gppoint{gp mark 0}{(6.690,5.818)}
\gppoint{gp mark 0}{(6.945,4.494)}
\gppoint{gp mark 0}{(8.049,6.528)}
\gppoint{gp mark 0}{(7.614,5.040)}
\gppoint{gp mark 0}{(7.450,5.323)}
\gppoint{gp mark 0}{(7.392,5.248)}
\gppoint{gp mark 0}{(7.479,5.359)}
\gppoint{gp mark 0}{(7.046,4.552)}
\gppoint{gp mark 0}{(7.605,5.035)}
\gppoint{gp mark 0}{(7.614,5.055)}
\gppoint{gp mark 0}{(7.070,4.767)}
\gppoint{gp mark 0}{(7.653,5.132)}
\gppoint{gp mark 0}{(7.714,5.159)}
\gppoint{gp mark 0}{(6.980,4.373)}
\gppoint{gp mark 0}{(7.086,4.667)}
\gppoint{gp mark 0}{(6.844,5.876)}
\gppoint{gp mark 0}{(7.139,4.856)}
\gppoint{gp mark 0}{(7.005,4.633)}
\gppoint{gp mark 0}{(7.070,4.760)}
\gppoint{gp mark 0}{(7.559,5.399)}
\gppoint{gp mark 0}{(7.695,5.075)}
\gppoint{gp mark 0}{(7.410,5.320)}
\gppoint{gp mark 0}{(6.954,4.625)}
\gppoint{gp mark 0}{(6.980,4.607)}
\gppoint{gp mark 0}{(7.070,4.781)}
\gppoint{gp mark 0}{(6.918,4.552)}
\gppoint{gp mark 0}{(7.054,4.802)}
\gppoint{gp mark 0}{(7.101,4.829)}
\gppoint{gp mark 0}{(6.963,4.633)}
\gppoint{gp mark 0}{(7.005,4.325)}
\gppoint{gp mark 0}{(7.579,6.662)}
\gppoint{gp mark 0}{(7.038,4.452)}
\gppoint{gp mark 0}{(7.928,6.370)}
\gppoint{gp mark 0}{(6.927,4.552)}
\gppoint{gp mark 0}{(7.147,4.760)}
\gppoint{gp mark 0}{(6.598,5.804)}
\gppoint{gp mark 0}{(7.522,5.352)}
\gppoint{gp mark 0}{(7.886,6.321)}
\gppoint{gp mark 0}{(7.490,5.403)}
\gppoint{gp mark 0}{(7.078,4.823)}
\gppoint{gp mark 0}{(7.154,4.737)}
\gppoint{gp mark 0}{(7.462,5.252)}
\gppoint{gp mark 0}{(7.013,4.408)}
\gppoint{gp mark 0}{(6.844,5.830)}
\gppoint{gp mark 0}{(8.575,5.707)}
\gppoint{gp mark 0}{(7.653,5.155)}
\gppoint{gp mark 0}{(6.963,4.589)}
\gppoint{gp mark 0}{(7.462,5.244)}
\gppoint{gp mark 0}{(7.013,4.385)}
\gppoint{gp mark 0}{(7.727,6.406)}
\gppoint{gp mark 0}{(7.154,4.722)}
\gppoint{gp mark 0}{(6.971,4.642)}
\gppoint{gp mark 0}{(6.918,4.533)}
\gppoint{gp mark 0}{(7.139,4.683)}
\gppoint{gp mark 0}{(6.927,4.504)}
\gppoint{gp mark 0}{(7.132,4.699)}
\gppoint{gp mark 0}{(8.035,7.106)}
\gppoint{gp mark 0}{(7.658,5.186)}
\gppoint{gp mark 0}{(7.579,5.035)}
\gppoint{gp mark 0}{(7.548,5.355)}
\gppoint{gp mark 0}{(7.154,4.707)}
\gppoint{gp mark 0}{(6.954,4.543)}
\gppoint{gp mark 0}{(7.522,5.373)}
\gppoint{gp mark 0}{(7.101,4.774)}
\gppoint{gp mark 0}{(7.109,4.788)}
\gppoint{gp mark 0}{(7.564,6.663)}
\gppoint{gp mark 0}{(7.169,4.659)}
\gppoint{gp mark 0}{(6.927,4.625)}
\gppoint{gp mark 0}{(7.062,4.862)}
\gppoint{gp mark 0}{(7.086,4.823)}
\gppoint{gp mark 0}{(7.639,4.887)}
\gppoint{gp mark 0}{(7.078,4.836)}
\gppoint{gp mark 0}{(6.945,4.589)}
\gppoint{gp mark 0}{(7.070,4.843)}
\gppoint{gp mark 0}{(6.936,4.598)}
\gppoint{gp mark 0}{(7.139,4.714)}
\gppoint{gp mark 0}{(7.456,6.691)}
\gppoint{gp mark 0}{(6.927,4.607)}
\gppoint{gp mark 0}{(6.918,4.589)}
\gppoint{gp mark 0}{(6.936,4.625)}
\gppoint{gp mark 0}{(7.132,4.752)}
\gppoint{gp mark 0}{(7.124,4.730)}
\gppoint{gp mark 0}{(6.971,4.543)}
\gppoint{gp mark 0}{(6.988,4.452)}
\gppoint{gp mark 0}{(6.286,5.914)}
\gppoint{gp mark 0}{(6.701,5.794)}
\gppoint{gp mark 0}{(7.078,4.862)}
\gppoint{gp mark 0}{(6.954,4.533)}
\gppoint{gp mark 0}{(7.439,5.244)}
\gppoint{gp mark 0}{(7.030,4.325)}
\gppoint{gp mark 0}{(6.997,4.419)}
\gppoint{gp mark 0}{(7.644,4.912)}
\gppoint{gp mark 0}{(7.433,5.403)}
\gppoint{gp mark 0}{(6.927,4.699)}
\gppoint{gp mark 0}{(7.078,4.349)}
\gppoint{gp mark 0}{(6.918,4.683)}
\gppoint{gp mark 0}{(6.419,5.891)}
\gppoint{gp mark 0}{(7.132,4.533)}
\gppoint{gp mark 0}{(7.101,4.452)}
\gppoint{gp mark 0}{(6.988,4.823)}
\gppoint{gp mark 0}{(7.147,4.642)}
\gppoint{gp mark 0}{(6.918,4.707)}
\gppoint{gp mark 0}{(7.538,5.271)}
\gppoint{gp mark 0}{(7.634,5.181)}
\gppoint{gp mark 0}{(7.564,5.090)}
\gppoint{gp mark 0}{(7.538,6.703)}
\gppoint{gp mark 0}{(6.927,4.667)}
\gppoint{gp mark 0}{(7.709,5.040)}
\gppoint{gp mark 0}{(7.653,6.652)}
\gppoint{gp mark 0}{(8.253,6.911)}
\gppoint{gp mark 0}{(7.686,4.981)}
\gppoint{gp mark 0}{(7.484,5.211)}
\gppoint{gp mark 0}{(6.963,4.722)}
\gppoint{gp mark 0}{(7.022,4.849)}
\gppoint{gp mark 0}{(7.691,5.014)}
\gppoint{gp mark 0}{(7.554,7.186)}
\gppoint{gp mark 0}{(6.927,4.650)}
\gppoint{gp mark 0}{(7.506,5.240)}
\gppoint{gp mark 0}{(7.005,4.816)}
\gppoint{gp mark 0}{(7.589,5.100)}
\gppoint{gp mark 0}{(7.517,5.256)}
\gppoint{gp mark 0}{(7.392,5.383)}
\gppoint{gp mark 0}{(7.124,4.642)}
\gppoint{gp mark 0}{(7.589,5.090)}
\gppoint{gp mark 0}{(6.963,4.699)}
\gppoint{gp mark 0}{(7.614,5.173)}
\gppoint{gp mark 0}{(7.559,6.704)}
\gppoint{gp mark 0}{(7.239,4.286)}
\gppoint{gp mark 0}{(7.013,4.829)}
\gppoint{gp mark 0}{(7.054,4.483)}
\gppoint{gp mark 0}{(6.997,4.849)}
\gppoint{gp mark 0}{(6.927,4.737)}
\gppoint{gp mark 0}{(7.101,4.373)}
\gppoint{gp mark 0}{(7.169,4.562)}
\gppoint{gp mark 0}{(7.484,5.248)}
\gppoint{gp mark 0}{(6.980,4.691)}
\gppoint{gp mark 0}{(7.013,4.862)}
\gppoint{gp mark 0}{(7.139,4.625)}
\gppoint{gp mark 0}{(7.392,5.362)}
\gppoint{gp mark 0}{(6.785,5.794)}
\gppoint{gp mark 0}{(7.404,5.373)}
\gppoint{gp mark 0}{(7.046,4.802)}
\gppoint{gp mark 0}{(7.109,4.373)}
\gppoint{gp mark 0}{(7.398,5.379)}
\gppoint{gp mark 0}{(6.980,4.683)}
\gppoint{gp mark 0}{(7.154,4.494)}
\gppoint{gp mark 0}{(7.629,5.132)}
\gppoint{gp mark 0}{(7.078,4.533)}
\gppoint{gp mark 0}{(7.013,4.675)}
\gppoint{gp mark 0}{(7.038,4.722)}
\gppoint{gp mark 0}{(7.954,6.320)}
\gppoint{gp mark 0}{(7.117,4.373)}
\gppoint{gp mark 0}{(6.945,4.788)}
\gppoint{gp mark 0}{(7.386,5.399)}
\gppoint{gp mark 0}{(7.086,4.616)}
\gppoint{gp mark 0}{(7.439,5.352)}
\gppoint{gp mark 0}{(7.046,4.730)}
\gppoint{gp mark 0}{(7.495,5.275)}
\gppoint{gp mark 0}{(7.543,5.267)}
\gppoint{gp mark 0}{(7.101,4.607)}
\gppoint{gp mark 0}{(7.533,5.219)}
\gppoint{gp mark 0}{(7.046,4.722)}
\gppoint{gp mark 0}{(7.450,5.334)}
\gppoint{gp mark 0}{(7.259,6.582)}
\gppoint{gp mark 0}{(7.584,5.177)}
\gppoint{gp mark 0}{(6.936,4.809)}
\gppoint{gp mark 0}{(7.101,4.625)}
\gppoint{gp mark 0}{(6.980,4.875)}
\gppoint{gp mark 0}{(7.386,5.389)}
\gppoint{gp mark 0}{(6.795,5.808)}
\gppoint{gp mark 0}{(7.490,5.286)}
\gppoint{gp mark 0}{(7.600,5.198)}
\gppoint{gp mark 0}{(7.139,4.397)}
\gppoint{gp mark 0}{(7.639,5.128)}
\gppoint{gp mark 0}{(7.070,4.543)}
\gppoint{gp mark 0}{(7.450,5.352)}
\gppoint{gp mark 0}{(8.741,5.841)}
\gppoint{gp mark 0}{(8.207,6.170)}
\gppoint{gp mark 0}{(6.954,4.849)}
\gppoint{gp mark 0}{(6.971,4.875)}
\gppoint{gp mark 0}{(7.094,4.589)}
\gppoint{gp mark 0}{(7.495,5.290)}
\gppoint{gp mark 0}{(7.086,4.607)}
\gppoint{gp mark 0}{(6.459,5.939)}
\gppoint{gp mark 0}{(6.963,4.829)}
\gppoint{gp mark 0}{(7.564,5.146)}
\gppoint{gp mark 0}{(7.619,5.080)}
\gppoint{gp mark 0}{(6.927,4.875)}
\gppoint{gp mark 0}{(7.350,6.559)}
\gppoint{gp mark 0}{(6.834,5.800)}
\gppoint{gp mark 0}{(7.386,5.425)}
\gppoint{gp mark 0}{(6.900,5.763)}
\gppoint{gp mark 0}{(6.988,4.767)}
\gppoint{gp mark 0}{(7.078,4.589)}
\gppoint{gp mark 0}{(7.054,4.642)}
\gppoint{gp mark 0}{(7.005,4.737)}
\gppoint{gp mark 0}{(7.005,4.730)}
\gppoint{gp mark 0}{(7.154,4.373)}
\gppoint{gp mark 0}{(7.681,4.965)}
\gppoint{gp mark 0}{(6.954,4.816)}
\gppoint{gp mark 0}{(7.054,4.633)}
\gppoint{gp mark 0}{(7.609,5.114)}
\gppoint{gp mark 0}{(7.484,5.312)}
\gppoint{gp mark 0}{(7.132,4.473)}
\gppoint{gp mark 0}{(7.005,4.752)}
\gppoint{gp mark 0}{(7.022,4.659)}
\gppoint{gp mark 0}{(7.677,5.014)}
\gppoint{gp mark 0}{(7.101,4.552)}
\gppoint{gp mark 0}{(7.410,5.386)}
\gppoint{gp mark 0}{(6.980,4.816)}
\gppoint{gp mark 0}{(7.605,5.114)}
\gppoint{gp mark 0}{(7.677,5.003)}
\gppoint{gp mark 0}{(6.918,4.849)}
\gppoint{gp mark 0}{(7.427,5.399)}
\gppoint{gp mark 0}{(7.005,4.760)}
\gppoint{gp mark 0}{(6.971,4.816)}
\gppoint{gp mark 0}{(6.980,4.802)}
\gppoint{gp mark 0}{(7.445,5.379)}
\gppoint{gp mark 0}{(7.062,4.580)}
\gppoint{gp mark 0}{(7.124,5.168)}
\gppoint{gp mark 0}{(6.997,5.014)}
\gppoint{gp mark 0}{(7.086,5.123)}
\gppoint{gp mark 0}{(7.183,5.236)}
\gppoint{gp mark 0}{(6.963,4.970)}
\gppoint{gp mark 0}{(7.762,6.461)}
\gppoint{gp mark 0}{(7.013,4.981)}
\gppoint{gp mark 0}{(6.954,4.970)}
\gppoint{gp mark 0}{(7.078,5.065)}
\gppoint{gp mark 0}{(7.368,5.428)}
\gppoint{gp mark 0}{(6.936,4.894)}
\gppoint{gp mark 0}{(6.918,4.918)}
\gppoint{gp mark 0}{(7.183,5.223)}
\gppoint{gp mark 0}{(7.380,5.412)}
\gppoint{gp mark 0}{(7.380,5.435)}
\gppoint{gp mark 0}{(7.101,5.123)}
\gppoint{gp mark 0}{(6.997,4.992)}
\gppoint{gp mark 0}{(6.775,5.700)}
\gppoint{gp mark 0}{(6.945,4.918)}
\gppoint{gp mark 0}{(7.356,5.406)}
\gppoint{gp mark 0}{(7.190,5.223)}
\gppoint{gp mark 0}{(7.139,5.164)}
\gppoint{gp mark 0}{(7.380,5.432)}
\gppoint{gp mark 0}{(7.204,5.252)}
\gppoint{gp mark 0}{(7.305,5.348)}
\gppoint{gp mark 0}{(7.830,6.492)}
\gppoint{gp mark 0}{(6.405,5.552)}
\gppoint{gp mark 0}{(6.679,5.661)}
\gppoint{gp mark 0}{(7.889,6.512)}
\gppoint{gp mark 0}{(6.119,5.478)}
\gppoint{gp mark 0}{(7.232,5.320)}
\gppoint{gp mark 0}{(7.070,5.104)}
\gppoint{gp mark 0}{(7.292,5.383)}
\gppoint{gp mark 0}{(7.154,5.168)}
\gppoint{gp mark 0}{(6.853,5.761)}
\gppoint{gp mark 0}{(7.070,5.100)}
\gppoint{gp mark 0}{(7.161,5.137)}
\gppoint{gp mark 0}{(7.380,5.393)}
\gppoint{gp mark 0}{(7.362,5.399)}
\gppoint{gp mark 0}{(7.169,5.146)}
\gppoint{gp mark 0}{(7.312,5.345)}
\gppoint{gp mark 0}{(6.963,4.912)}
\gppoint{gp mark 0}{(7.968,6.541)}
\gppoint{gp mark 0}{(6.954,4.924)}
\gppoint{gp mark 0}{(7.054,5.132)}
\gppoint{gp mark 0}{(6.945,4.930)}
\gppoint{gp mark 0}{(6.668,5.617)}
\gppoint{gp mark 0}{(7.362,5.406)}
\gppoint{gp mark 0}{(7.882,6.505)}
\gppoint{gp mark 0}{(6.405,5.568)}
\gppoint{gp mark 0}{(7.022,4.981)}
\gppoint{gp mark 0}{(6.918,4.936)}
\gppoint{gp mark 0}{(6.537,5.576)}
\gppoint{gp mark 0}{(6.954,4.881)}
\gppoint{gp mark 0}{(6.945,4.965)}
\gppoint{gp mark 0}{(7.266,5.279)}
\gppoint{gp mark 0}{(7.331,5.352)}
\gppoint{gp mark 0}{(7.176,5.252)}
\gppoint{gp mark 0}{(8.314,6.269)}
\gppoint{gp mark 0}{(6.909,5.740)}
\gppoint{gp mark 0}{(6.980,4.912)}
\gppoint{gp mark 0}{(7.204,5.215)}
\gppoint{gp mark 0}{(6.980,4.906)}
\gppoint{gp mark 0}{(7.324,5.352)}
\gppoint{gp mark 0}{(6.927,4.930)}
\gppoint{gp mark 0}{(6.945,4.953)}
\gppoint{gp mark 0}{(7.013,5.040)}
\gppoint{gp mark 0}{(6.997,4.924)}
\gppoint{gp mark 0}{(7.147,5.123)}
\gppoint{gp mark 0}{(6.882,5.720)}
\gppoint{gp mark 0}{(7.132,5.060)}
\gppoint{gp mark 0}{(7.169,5.109)}
\gppoint{gp mark 0}{(7.225,5.309)}
\gppoint{gp mark 0}{(7.324,5.422)}
\gppoint{gp mark 0}{(7.101,5.186)}
\gppoint{gp mark 0}{(7.038,4.942)}
\gppoint{gp mark 0}{(6.997,4.906)}
\gppoint{gp mark 0}{(7.101,5.181)}
\gppoint{gp mark 0}{(7.117,5.065)}
\gppoint{gp mark 0}{(7.030,4.947)}
\gppoint{gp mark 0}{(6.927,4.992)}
\gppoint{gp mark 0}{(7.286,5.386)}
\gppoint{gp mark 0}{(7.331,5.432)}
\gppoint{gp mark 0}{(6.562,5.565)}
\gppoint{gp mark 0}{(7.878,6.525)}
\gppoint{gp mark 0}{(7.038,4.970)}
\gppoint{gp mark 0}{(6.954,5.035)}
\gppoint{gp mark 0}{(6.853,5.696)}
\gppoint{gp mark 0}{(6.844,5.698)}
\gppoint{gp mark 0}{(6.909,5.720)}
\gppoint{gp mark 0}{(7.154,5.100)}
\gppoint{gp mark 0}{(7.218,5.275)}
\gppoint{gp mark 0}{(7.038,4.887)}
\gppoint{gp mark 0}{(6.980,4.992)}
\gppoint{gp mark 0}{(6.997,4.965)}
\gppoint{gp mark 0}{(7.147,5.080)}
\gppoint{gp mark 0}{(7.062,5.198)}
\gppoint{gp mark 0}{(7.046,4.894)}
\gppoint{gp mark 0}{(7.054,5.190)}
\gppoint{gp mark 0}{(7.324,5.386)}
\gppoint{gp mark 0}{(7.246,5.240)}
\gppoint{gp mark 0}{(7.299,5.412)}
\gppoint{gp mark 0}{(6.668,5.661)}
\gppoint{gp mark 0}{(6.754,5.620)}
\gppoint{gp mark 0}{(7.094,5.159)}
\gppoint{gp mark 0}{(7.305,5.416)}
\gppoint{gp mark 0}{(7.030,4.912)}
\gppoint{gp mark 0}{(7.046,4.887)}
\gppoint{gp mark 0}{(6.824,5.734)}
\gppoint{gp mark 0}{(6.927,5.040)}
\gppoint{gp mark 0}{(6.909,5.694)}
\gppoint{gp mark 0}{(6.997,4.953)}
\gppoint{gp mark 0}{(7.132,5.109)}
\gppoint{gp mark 0}{(6.171,5.510)}
\gppoint{gp mark 0}{(6.971,5.003)}
\gppoint{gp mark 0}{(8.143,6.441)}
\gppoint{gp mark 0}{(6.927,5.035)}
\gppoint{gp mark 0}{(7.204,5.275)}
\gppoint{gp mark 0}{(7.595,6.616)}
\gppoint{gp mark 0}{(6.376,5.597)}
\gppoint{gp mark 0}{(7.337,5.359)}
\gppoint{gp mark 0}{(6.971,4.998)}
\gppoint{gp mark 0}{(6.918,5.019)}
\gppoint{gp mark 0}{(6.988,4.930)}
\gppoint{gp mark 0}{(7.013,4.965)}
\gppoint{gp mark 0}{(7.183,5.309)}
\gppoint{gp mark 0}{(6.754,5.632)}
\gppoint{gp mark 0}{(7.070,5.190)}
\gppoint{gp mark 0}{(7.299,5.435)}
\gppoint{gp mark 0}{(6.562,5.552)}
\gppoint{gp mark 0}{(6.988,4.947)}
\gppoint{gp mark 0}{(6.347,5.459)}
\gppoint{gp mark 0}{(7.022,4.894)}
\gppoint{gp mark 0}{(7.204,5.279)}
\gppoint{gp mark 0}{(7.225,5.286)}
\gppoint{gp mark 0}{(7.190,5.334)}
\gppoint{gp mark 0}{(6.963,5.128)}
\gppoint{gp mark 0}{(7.211,5.379)}
\gppoint{gp mark 0}{(7.374,5.301)}
\gppoint{gp mark 0}{(6.980,5.109)}
\gppoint{gp mark 0}{(7.054,4.906)}
\gppoint{gp mark 0}{(6.971,5.100)}
\gppoint{gp mark 0}{(7.046,5.181)}
\gppoint{gp mark 0}{(7.030,5.198)}
\gppoint{gp mark 0}{(7.078,4.887)}
\gppoint{gp mark 0}{(7.094,4.959)}
\gppoint{gp mark 0}{(7.062,4.912)}
\gppoint{gp mark 0}{(6.945,5.065)}
\gppoint{gp mark 0}{(7.337,5.298)}
\gppoint{gp mark 0}{(7.279,5.416)}
\gppoint{gp mark 0}{(6.844,5.673)}
\gppoint{gp mark 0}{(7.038,5.181)}
\gppoint{gp mark 0}{(6.918,5.090)}
\gppoint{gp mark 0}{(6.900,5.680)}
\gppoint{gp mark 0}{(6.980,5.100)}
\gppoint{gp mark 0}{(7.070,4.894)}
\gppoint{gp mark 0}{(7.094,4.947)}
\gppoint{gp mark 0}{(7.169,5.055)}
\gppoint{gp mark 0}{(7.013,5.168)}
\gppoint{gp mark 0}{(6.963,5.114)}
\gppoint{gp mark 0}{(7.337,5.275)}
\gppoint{gp mark 0}{(7.101,4.959)}
\gppoint{gp mark 0}{(6.446,5.469)}
\gppoint{gp mark 0}{(6.936,5.080)}
\gppoint{gp mark 0}{(6.882,5.675)}
\gppoint{gp mark 0}{(7.022,5.173)}
\gppoint{gp mark 0}{(6.954,5.100)}
\gppoint{gp mark 0}{(7.337,5.271)}
\gppoint{gp mark 0}{(6.795,5.632)}
\gppoint{gp mark 0}{(7.564,4.675)}
\gppoint{gp mark 0}{(6.971,5.132)}
\gppoint{gp mark 0}{(6.997,5.142)}
\gppoint{gp mark 0}{(6.927,5.065)}
\gppoint{gp mark 0}{(7.054,4.894)}
\gppoint{gp mark 0}{(6.936,5.090)}
\gppoint{gp mark 0}{(7.078,4.906)}
\gppoint{gp mark 0}{(7.305,5.223)}
\gppoint{gp mark 0}{(6.963,5.100)}
\gppoint{gp mark 0}{(7.086,7.157)}
\gppoint{gp mark 0}{(7.239,5.435)}
\gppoint{gp mark 0}{(7.246,5.416)}
\gppoint{gp mark 0}{(7.380,5.283)}
\gppoint{gp mark 0}{(7.062,4.970)}
\gppoint{gp mark 0}{(6.419,5.469)}
\gppoint{gp mark 0}{(7.147,4.998)}
\gppoint{gp mark 0}{(7.117,5.040)}
\gppoint{gp mark 0}{(6.945,5.109)}
\gppoint{gp mark 0}{(7.232,5.425)}
\gppoint{gp mark 0}{(6.971,5.060)}
\gppoint{gp mark 0}{(6.963,5.085)}
\gppoint{gp mark 0}{(7.318,5.227)}
\gppoint{gp mark 0}{(7.139,5.024)}
\gppoint{gp mark 0}{(7.337,5.323)}
\gppoint{gp mark 0}{(6.971,5.070)}
\gppoint{gp mark 0}{(7.324,5.227)}
\gppoint{gp mark 0}{(6.997,5.186)}
\gppoint{gp mark 0}{(7.374,5.290)}
\gppoint{gp mark 0}{(7.154,4.992)}
\gppoint{gp mark 0}{(6.945,5.132)}
\gppoint{gp mark 0}{(7.183,5.369)}
\gppoint{gp mark 0}{(7.559,6.567)}
\gppoint{gp mark 0}{(7.086,4.881)}
\gppoint{gp mark 0}{(6.980,5.090)}
\gppoint{gp mark 0}{(7.506,6.575)}
\gppoint{gp mark 0}{(6.446,5.441)}
\gppoint{gp mark 0}{(6.499,5.513)}
\gppoint{gp mark 0}{(7.139,5.045)}
\gppoint{gp mark 0}{(6.459,5.459)}
\gppoint{gp mark 0}{(6.775,5.637)}
\gppoint{gp mark 0}{(7.124,5.019)}
\gppoint{gp mark 0}{(6.562,5.504)}
\gppoint{gp mark 0}{(7.609,6.627)}
\gppoint{gp mark 0}{(6.997,5.095)}
\gppoint{gp mark 0}{(7.124,4.924)}
\gppoint{gp mark 0}{(6.853,5.634)}
\gppoint{gp mark 0}{(7.070,4.976)}
\gppoint{gp mark 0}{(7.318,5.323)}
\gppoint{gp mark 0}{(6.405,5.487)}
\gppoint{gp mark 0}{(7.211,5.432)}
\gppoint{gp mark 0}{(7.154,4.965)}
\gppoint{gp mark 0}{(6.909,5.642)}
\gppoint{gp mark 0}{(7.305,6.704)}
\gppoint{gp mark 0}{(6.712,5.696)}
\gppoint{gp mark 0}{(7.495,6.568)}
\gppoint{gp mark 0}{(7.579,4.774)}
\gppoint{gp mark 0}{(7.139,4.881)}
\gppoint{gp mark 0}{(7.246,5.338)}
\gppoint{gp mark 0}{(7.677,6.621)}
\gppoint{gp mark 0}{(7.239,5.341)}
\gppoint{gp mark 0}{(7.292,5.283)}
\gppoint{gp mark 0}{(7.928,6.518)}
\gppoint{gp mark 0}{(7.305,5.298)}
\gppoint{gp mark 0}{(6.980,5.198)}
\gppoint{gp mark 0}{(7.299,5.286)}
\gppoint{gp mark 0}{(7.094,5.030)}
\gppoint{gp mark 0}{(7.013,5.090)}
\gppoint{gp mark 0}{(6.900,5.647)}
\gppoint{gp mark 0}{(7.147,4.947)}
\gppoint{gp mark 0}{(6.795,5.668)}
\gppoint{gp mark 0}{(7.324,5.327)}
\gppoint{gp mark 0}{(7.139,4.912)}
\gppoint{gp mark 0}{(7.038,5.132)}
\gppoint{gp mark 0}{(6.574,5.475)}
\gppoint{gp mark 0}{(7.614,4.699)}
\gppoint{gp mark 0}{(6.988,5.123)}
\gppoint{gp mark 0}{(7.038,5.065)}
\gppoint{gp mark 0}{(7.054,5.045)}
\gppoint{gp mark 0}{(7.022,5.085)}
\gppoint{gp mark 0}{(7.183,5.435)}
\gppoint{gp mark 0}{(7.554,6.553)}
\gppoint{gp mark 0}{(7.527,6.560)}
\gppoint{gp mark 0}{(6.550,5.456)}
\gppoint{gp mark 0}{(6.988,5.118)}
\gppoint{gp mark 0}{(6.945,5.177)}
\gppoint{gp mark 0}{(7.362,5.232)}
\gppoint{gp mark 0}{(7.273,5.338)}
\gppoint{gp mark 0}{(7.211,5.399)}
\gppoint{gp mark 0}{(7.368,5.223)}
\gppoint{gp mark 0}{(7.161,4.894)}
\gppoint{gp mark 0}{(7.078,5.019)}
\gppoint{gp mark 0}{(6.997,5.118)}
\gppoint{gp mark 0}{(7.109,4.976)}
\gppoint{gp mark 0}{(6.954,5.164)}
\gppoint{gp mark 0}{(6.945,5.173)}
\gppoint{gp mark 0}{(6.997,5.114)}
\gppoint{gp mark 0}{(7.211,5.396)}
\gppoint{gp mark 0}{(6.963,5.146)}
\gppoint{gp mark 0}{(6.405,5.521)}
\gppoint{gp mark 0}{(6.945,5.198)}
\gppoint{gp mark 0}{(6.988,5.100)}
\gppoint{gp mark 0}{(7.005,5.118)}
\gppoint{gp mark 0}{(7.070,5.055)}
\gppoint{gp mark 0}{(6.668,5.740)}
\gppoint{gp mark 0}{(7.132,4.970)}
\gppoint{gp mark 0}{(6.988,5.114)}
\gppoint{gp mark 0}{(6.882,5.625)}
\gppoint{gp mark 0}{(6.980,5.155)}
\gppoint{gp mark 0}{(7.266,5.331)}
\gppoint{gp mark 0}{(6.844,5.642)}
\gppoint{gp mark 0}{(7.183,6.649)}
\gppoint{gp mark 0}{(7.218,4.936)}
\gppoint{gp mark 0}{(7.169,5.422)}
\gppoint{gp mark 0}{(7.197,4.900)}
\gppoint{gp mark 0}{(7.484,4.707)}
\gppoint{gp mark 0}{(7.374,5.173)}
\gppoint{gp mark 0}{(6.971,5.240)}
\gppoint{gp mark 0}{(6.733,5.527)}
\gppoint{gp mark 0}{(7.204,4.970)}
\gppoint{gp mark 0}{(7.273,5.035)}
\gppoint{gp mark 0}{(6.668,5.466)}
\gppoint{gp mark 0}{(6.785,5.541)}
\gppoint{gp mark 0}{(6.872,5.578)}
\gppoint{gp mark 0}{(6.988,5.298)}
\gppoint{gp mark 0}{(7.878,6.454)}
\gppoint{gp mark 0}{(6.927,5.227)}
\gppoint{gp mark 0}{(8.264,6.303)}
\gppoint{gp mark 0}{(7.109,5.359)}
\gppoint{gp mark 0}{(7.767,6.512)}
\gppoint{gp mark 0}{(6.963,5.256)}
\gppoint{gp mark 0}{(6.610,5.444)}
\gppoint{gp mark 0}{(7.538,4.823)}
\gppoint{gp mark 0}{(7.239,4.992)}
\gppoint{gp mark 0}{(7.030,5.312)}
\gppoint{gp mark 0}{(7.086,5.362)}
\gppoint{gp mark 0}{(7.312,5.104)}
\gppoint{gp mark 0}{(7.147,5.416)}
\gppoint{gp mark 0}{(6.645,5.472)}
\gppoint{gp mark 0}{(6.963,5.252)}
\gppoint{gp mark 0}{(7.286,5.060)}
\gppoint{gp mark 0}{(7.337,5.137)}
\gppoint{gp mark 0}{(7.380,5.198)}
\gppoint{gp mark 0}{(6.668,5.481)}
\gppoint{gp mark 0}{(6.954,5.240)}
\gppoint{gp mark 0}{(7.368,5.181)}
\gppoint{gp mark 0}{(7.473,4.633)}
\gppoint{gp mark 0}{(6.936,5.223)}
\gppoint{gp mark 0}{(9.221,7.941)}
\gppoint{gp mark 0}{(7.490,4.707)}
\gppoint{gp mark 0}{(7.139,5.403)}
\gppoint{gp mark 0}{(7.124,5.389)}
\gppoint{gp mark 0}{(7.286,5.075)}
\gppoint{gp mark 0}{(6.988,5.283)}
\gppoint{gp mark 0}{(7.350,5.168)}
\gppoint{gp mark 0}{(7.225,4.953)}
\gppoint{gp mark 0}{(6.824,5.571)}
\gppoint{gp mark 0}{(6.945,5.223)}
\gppoint{gp mark 0}{(7.343,5.137)}
\gppoint{gp mark 0}{(6.927,5.207)}
\gppoint{gp mark 0}{(7.086,5.366)}
\gppoint{gp mark 0}{(7.286,5.070)}
\gppoint{gp mark 0}{(7.279,5.040)}
\gppoint{gp mark 0}{(7.109,5.341)}
\gppoint{gp mark 0}{(7.954,6.493)}
\gppoint{gp mark 0}{(6.909,5.584)}
\gppoint{gp mark 0}{(6.754,5.495)}
\gppoint{gp mark 0}{(7.259,5.003)}
\gppoint{gp mark 0}{(7.312,5.080)}
\gppoint{gp mark 0}{(6.656,5.438)}
\gppoint{gp mark 0}{(6.690,5.527)}
\gppoint{gp mark 0}{(7.078,5.366)}
\gppoint{gp mark 0}{(7.368,5.164)}
\gppoint{gp mark 0}{(6.824,5.532)}
\gppoint{gp mark 0}{(7.176,4.970)}
\gppoint{gp mark 0}{(7.190,4.947)}
\gppoint{gp mark 0}{(7.253,5.024)}
\gppoint{gp mark 0}{(6.634,5.459)}
\gppoint{gp mark 0}{(7.022,5.294)}
\gppoint{gp mark 0}{(6.945,5.240)}
\gppoint{gp mark 0}{(7.259,5.009)}
\gppoint{gp mark 0}{(7.629,6.564)}
\gppoint{gp mark 0}{(7.356,5.203)}
\gppoint{gp mark 0}{(6.936,5.259)}
\gppoint{gp mark 0}{(7.101,5.345)}
\gppoint{gp mark 0}{(6.954,5.207)}
\gppoint{gp mark 0}{(6.622,5.481)}
\gppoint{gp mark 0}{(7.266,4.981)}
\gppoint{gp mark 0}{(6.668,5.453)}
\gppoint{gp mark 0}{(6.844,5.604)}
\gppoint{gp mark 0}{(7.070,5.383)}
\gppoint{gp mark 0}{(7.386,4.452)}
\gppoint{gp mark 0}{(7.211,4.887)}
\gppoint{gp mark 0}{(6.733,5.489)}
\gppoint{gp mark 0}{(6.764,5.560)}
\gppoint{gp mark 0}{(7.094,5.331)}
\gppoint{gp mark 0}{(7.527,4.829)}
\gppoint{gp mark 0}{(6.927,5.240)}
\gppoint{gp mark 0}{(6.918,5.248)}
\gppoint{gp mark 0}{(6.712,5.521)}
\gppoint{gp mark 0}{(6.853,5.552)}
\gppoint{gp mark 0}{(6.927,5.298)}
\gppoint{gp mark 0}{(7.273,4.936)}
\gppoint{gp mark 0}{(7.022,5.259)}
\gppoint{gp mark 0}{(7.259,4.959)}
\gppoint{gp mark 0}{(7.517,4.849)}
\gppoint{gp mark 0}{(7.543,4.745)}
\gppoint{gp mark 0}{(6.891,5.571)}
\gppoint{gp mark 0}{(7.161,5.359)}
\gppoint{gp mark 0}{(7.169,5.366)}
\gppoint{gp mark 0}{(7.548,4.760)}
\gppoint{gp mark 0}{(7.259,4.953)}
\gppoint{gp mark 0}{(7.266,4.965)}
\gppoint{gp mark 0}{(6.900,5.563)}
\gppoint{gp mark 0}{(7.796,6.541)}
\gppoint{gp mark 0}{(7.943,6.493)}
\gppoint{gp mark 0}{(6.891,5.568)}
\gppoint{gp mark 0}{(6.997,5.227)}
\gppoint{gp mark 0}{(6.764,5.592)}
\gppoint{gp mark 0}{(7.030,5.256)}
\gppoint{gp mark 0}{(7.398,4.524)}
\gppoint{gp mark 0}{(7.246,4.894)}
\gppoint{gp mark 0}{(7.266,4.953)}
\gppoint{gp mark 0}{(7.124,5.345)}
\gppoint{gp mark 0}{(7.147,5.379)}
\gppoint{gp mark 0}{(7.279,4.970)}
\gppoint{gp mark 0}{(6.954,5.305)}
\gppoint{gp mark 0}{(7.312,5.177)}
\gppoint{gp mark 0}{(7.132,5.348)}
\gppoint{gp mark 0}{(6.701,5.453)}
\gppoint{gp mark 0}{(6.722,5.466)}
\gppoint{gp mark 0}{(7.246,4.912)}
\gppoint{gp mark 0}{(7.054,5.386)}
\gppoint{gp mark 0}{(6.936,5.286)}
\gppoint{gp mark 0}{(7.005,5.223)}
\gppoint{gp mark 0}{(6.154,5.666)}
\gppoint{gp mark 0}{(7.266,4.942)}
\gppoint{gp mark 0}{(6.863,5.543)}
\gppoint{gp mark 0}{(7.554,4.767)}
\gppoint{gp mark 0}{(7.826,6.526)}
\gppoint{gp mark 0}{(7.161,5.383)}
\gppoint{gp mark 0}{(6.722,5.472)}
\gppoint{gp mark 0}{(7.331,5.194)}
\gppoint{gp mark 0}{(7.517,4.862)}
\gppoint{gp mark 0}{(6.963,5.301)}
\gppoint{gp mark 0}{(6.900,5.571)}
\gppoint{gp mark 0}{(7.292,5.137)}
\gppoint{gp mark 0}{(7.022,5.227)}
\gppoint{gp mark 0}{(6.963,5.298)}
\gppoint{gp mark 0}{(7.147,5.348)}
\gppoint{gp mark 0}{(7.331,6.678)}
\gppoint{gp mark 0}{(6.805,5.586)}
\gppoint{gp mark 0}{(7.484,4.869)}
\gppoint{gp mark 0}{(7.312,5.155)}
\gppoint{gp mark 0}{(7.246,4.930)}
\gppoint{gp mark 0}{(6.909,5.538)}
\gppoint{gp mark 0}{(6.690,5.481)}
\gppoint{gp mark 0}{(7.246,4.947)}
\gppoint{gp mark 0}{(7.132,5.369)}
\gppoint{gp mark 0}{(7.239,4.959)}
\gppoint{gp mark 0}{(7.337,5.132)}
\gppoint{gp mark 0}{(6.927,5.320)}
\gppoint{gp mark 0}{(7.433,4.494)}
\gppoint{gp mark 0}{(7.506,4.809)}
\gppoint{gp mark 0}{(7.013,5.244)}
\gppoint{gp mark 0}{(7.190,5.030)}
\gppoint{gp mark 0}{(7.273,4.894)}
\gppoint{gp mark 0}{(6.997,5.256)}
\gppoint{gp mark 0}{(6.701,5.469)}
\gppoint{gp mark 0}{(6.634,5.489)}
\gppoint{gp mark 0}{(7.169,5.355)}
\gppoint{gp mark 0}{(7.070,5.428)}
\gppoint{gp mark 0}{(7.380,5.095)}
\gppoint{gp mark 0}{(6.945,5.327)}
\gppoint{gp mark 0}{(7.324,5.159)}
\gppoint{gp mark 0}{(6.882,5.535)}
\gppoint{gp mark 0}{(6.419,5.759)}
\gppoint{gp mark 0}{(7.266,4.894)}
\gppoint{gp mark 0}{(7.380,5.090)}
\gppoint{gp mark 0}{(7.246,4.953)}
\gppoint{gp mark 0}{(7.943,6.491)}
\gppoint{gp mark 0}{(7.253,4.965)}
\gppoint{gp mark 0}{(7.862,7.070)}
\gppoint{gp mark 0}{(7.543,4.675)}
\gppoint{gp mark 0}{(7.078,5.428)}
\gppoint{gp mark 0}{(6.900,5.552)}
\gppoint{gp mark 0}{(7.490,4.875)}
\gppoint{gp mark 0}{(7.312,6.654)}
\gppoint{gp mark 0}{(6.668,5.498)}
\gppoint{gp mark 0}{(6.805,5.581)}
\gppoint{gp mark 0}{(6.722,5.444)}
\gppoint{gp mark 0}{(6.954,5.376)}
\gppoint{gp mark 0}{(7.218,5.104)}
\gppoint{gp mark 0}{(7.239,5.164)}
\gppoint{gp mark 0}{(7.356,4.987)}
\gppoint{gp mark 0}{(6.853,5.504)}
\gppoint{gp mark 0}{(7.218,5.100)}
\gppoint{gp mark 0}{(6.927,5.348)}
\gppoint{gp mark 0}{(7.312,4.970)}
\gppoint{gp mark 0}{(6.891,5.524)}
\gppoint{gp mark 0}{(6.954,5.383)}
\gppoint{gp mark 0}{(7.013,5.389)}
\gppoint{gp mark 0}{(7.124,5.290)}
\gppoint{gp mark 0}{(7.253,5.137)}
\gppoint{gp mark 0}{(6.733,5.607)}
\gppoint{gp mark 0}{(6.918,5.352)}
\gppoint{gp mark 0}{(7.190,5.070)}
\gppoint{gp mark 0}{(7.273,5.181)}
\gppoint{gp mark 0}{(6.954,5.379)}
\gppoint{gp mark 0}{(7.337,5.009)}
\gppoint{gp mark 0}{(7.038,5.419)}
\gppoint{gp mark 0}{(7.368,5.040)}
\gppoint{gp mark 0}{(6.909,5.510)}
\gppoint{gp mark 0}{(7.161,5.320)}
\gppoint{gp mark 0}{(7.147,5.305)}
\gppoint{gp mark 0}{(7.732,6.531)}
\gppoint{gp mark 0}{(7.386,4.659)}
\gppoint{gp mark 0}{(6.754,5.615)}
\gppoint{gp mark 0}{(7.239,5.151)}
\gppoint{gp mark 0}{(7.410,6.630)}
\gppoint{gp mark 0}{(7.398,4.683)}
\gppoint{gp mark 0}{(6.701,5.586)}
\gppoint{gp mark 0}{(6.891,5.516)}
\gppoint{gp mark 0}{(7.445,4.802)}
\gppoint{gp mark 0}{(7.232,5.137)}
\gppoint{gp mark 0}{(7.286,4.881)}
\gppoint{gp mark 0}{(7.183,5.065)}
\gppoint{gp mark 0}{(7.022,5.422)}
\gppoint{gp mark 0}{(7.218,5.132)}
\gppoint{gp mark 0}{(6.954,5.369)}
\gppoint{gp mark 0}{(7.038,5.435)}
\gppoint{gp mark 0}{(7.132,5.294)}
\gppoint{gp mark 0}{(7.117,5.279)}
\gppoint{gp mark 0}{(7.279,5.190)}
\gppoint{gp mark 0}{(6.872,5.498)}
\gppoint{gp mark 0}{(7.350,5.009)}
\gppoint{gp mark 0}{(7.253,5.155)}
\gppoint{gp mark 0}{(7.070,5.232)}
\gppoint{gp mark 0}{(6.997,5.435)}
\gppoint{gp mark 0}{(6.882,5.501)}
\gppoint{gp mark 0}{(6.473,5.622)}
\gppoint{gp mark 0}{(6.656,5.532)}
\gppoint{gp mark 0}{(7.070,5.240)}
\gppoint{gp mark 0}{(7.331,4.894)}
\gppoint{gp mark 0}{(7.154,5.294)}
\gppoint{gp mark 0}{(7.343,5.050)}
\gppoint{gp mark 0}{(6.909,5.492)}
\gppoint{gp mark 0}{(6.815,5.453)}
\gppoint{gp mark 0}{(7.190,5.114)}
\gppoint{gp mark 0}{(7.456,4.823)}
\gppoint{gp mark 0}{(7.312,4.924)}
\gppoint{gp mark 0}{(6.900,5.495)}
\gppoint{gp mark 0}{(6.936,5.366)}
\gppoint{gp mark 0}{(7.109,5.207)}
\gppoint{gp mark 0}{(7.362,5.009)}
\gppoint{gp mark 0}{(7.259,5.164)}
\gppoint{gp mark 0}{(7.404,4.714)}
\gppoint{gp mark 0}{(7.343,5.035)}
\gppoint{gp mark 0}{(7.312,4.887)}
\gppoint{gp mark 0}{(6.863,5.524)}
\gppoint{gp mark 0}{(6.795,5.481)}
\gppoint{gp mark 0}{(7.109,5.232)}
\gppoint{gp mark 0}{(6.805,5.438)}
\gppoint{gp mark 0}{(7.147,5.271)}
\gppoint{gp mark 0}{(7.318,4.894)}
\gppoint{gp mark 0}{(7.246,5.190)}
\gppoint{gp mark 0}{(6.963,5.334)}
\gppoint{gp mark 0}{(6.891,5.489)}
\gppoint{gp mark 0}{(6.936,5.383)}
\gppoint{gp mark 0}{(7.318,4.887)}
\gppoint{gp mark 0}{(6.795,5.478)}
\gppoint{gp mark 0}{(7.022,5.396)}
\gppoint{gp mark 0}{(6.743,5.592)}
\gppoint{gp mark 0}{(6.656,5.549)}
\gppoint{gp mark 0}{(7.109,5.223)}
\gppoint{gp mark 0}{(6.863,5.527)}
\gppoint{gp mark 0}{(7.204,5.070)}
\gppoint{gp mark 0}{(6.909,5.498)}
\gppoint{gp mark 0}{(7.318,4.881)}
\gppoint{gp mark 0}{(7.404,4.795)}
\gppoint{gp mark 0}{(6.459,5.677)}
\gppoint{gp mark 0}{(7.312,5.045)}
\gppoint{gp mark 0}{(7.070,5.275)}
\gppoint{gp mark 0}{(7.279,5.114)}
\gppoint{gp mark 0}{(6.550,5.651)}
\gppoint{gp mark 0}{(6.815,5.527)}
\gppoint{gp mark 0}{(7.467,4.714)}
\gppoint{gp mark 0}{(7.132,5.207)}
\gppoint{gp mark 0}{(6.945,5.393)}
\gppoint{gp mark 0}{(7.046,5.366)}
\gppoint{gp mark 0}{(7.176,5.155)}
\gppoint{gp mark 0}{(7.204,5.190)}
\gppoint{gp mark 0}{(7.005,5.341)}
\gppoint{gp mark 0}{(6.795,5.489)}
\gppoint{gp mark 0}{(7.456,4.767)}
\gppoint{gp mark 0}{(7.324,5.035)}
\gppoint{gp mark 0}{(7.273,5.114)}
\gppoint{gp mark 0}{(7.380,4.936)}
\gppoint{gp mark 0}{(6.963,5.428)}
\gppoint{gp mark 0}{(7.239,5.080)}
\gppoint{gp mark 0}{(7.054,5.294)}
\gppoint{gp mark 0}{(7.101,5.309)}
\gppoint{gp mark 0}{(7.117,5.232)}
\gppoint{gp mark 0}{(7.062,5.286)}
\gppoint{gp mark 0}{(7.410,6.642)}
\gppoint{gp mark 0}{(7.368,4.947)}
\gppoint{gp mark 0}{(7.259,5.100)}
\gppoint{gp mark 0}{(7.286,4.976)}
\gppoint{gp mark 0}{(6.733,5.560)}
\gppoint{gp mark 0}{(7.324,5.040)}
\gppoint{gp mark 0}{(7.266,5.109)}
\gppoint{gp mark 0}{(6.863,5.450)}
\gppoint{gp mark 0}{(7.624,6.580)}
\gppoint{gp mark 0}{(6.945,5.403)}
\gppoint{gp mark 0}{(6.844,5.447)}
\gppoint{gp mark 0}{(7.225,5.194)}
\gppoint{gp mark 0}{(7.062,5.275)}
\gppoint{gp mark 0}{(6.918,5.396)}
\gppoint{gp mark 0}{(7.305,5.003)}
\gppoint{gp mark 0}{(7.132,5.236)}
\gppoint{gp mark 0}{(7.292,4.976)}
\gppoint{gp mark 0}{(7.169,5.256)}
\gppoint{gp mark 0}{(7.343,4.881)}
\gppoint{gp mark 0}{(7.132,5.232)}
\gppoint{gp mark 0}{(7.197,5.155)}
\gppoint{gp mark 0}{(7.022,5.348)}
\gppoint{gp mark 0}{(7.279,5.075)}
\gppoint{gp mark 0}{(7.374,4.887)}
\gppoint{gp mark 0}{(6.824,5.487)}
\gppoint{gp mark 0}{(7.094,5.294)}
\gppoint{gp mark 0}{(7.259,5.080)}
\gppoint{gp mark 0}{(7.427,4.788)}
\gppoint{gp mark 0}{(6.824,5.495)}
\gppoint{gp mark 0}{(7.954,6.454)}
\gppoint{gp mark 0}{(7.380,4.887)}
\gppoint{gp mark 0}{(6.712,5.557)}
\gppoint{gp mark 0}{(7.038,5.341)}
\gppoint{gp mark 0}{(7.392,4.862)}
\gppoint{gp mark 0}{(6.863,5.469)}
\gppoint{gp mark 0}{(7.022,5.352)}
\gppoint{gp mark 0}{(7.211,5.155)}
\gppoint{gp mark 0}{(7.398,4.843)}
\gppoint{gp mark 0}{(7.380,4.881)}
\gppoint{gp mark 0}{(7.362,4.887)}
\gppoint{gp mark 0}{(7.054,5.305)}
\gppoint{gp mark 0}{(7.331,5.014)}
\gppoint{gp mark 0}{(7.218,5.159)}
\gppoint{gp mark 0}{(7.427,4.823)}
\gppoint{gp mark 0}{(7.094,5.279)}
\gppoint{gp mark 0}{(6.844,5.463)}
\gppoint{gp mark 0}{(7.218,5.155)}
\gppoint{gp mark 0}{(7.404,4.869)}
\gppoint{gp mark 0}{(6.918,5.412)}
\gppoint{gp mark 0}{(6.954,5.386)}
\gppoint{gp mark 0}{(6.824,5.498)}
\gppoint{gp mark 0}{(6.785,5.521)}
\gppoint{gp mark 0}{(7.286,5.035)}
\gppoint{gp mark 0}{(6.909,5.453)}
\gppoint{gp mark 0}{(7.038,5.355)}
\gppoint{gp mark 0}{(7.427,4.809)}
\gppoint{gp mark 0}{(6.927,5.412)}
\gppoint{gp mark 0}{(6.863,5.481)}
\gppoint{gp mark 0}{(7.392,4.829)}
\gppoint{gp mark 0}{(8.192,5.441)}
\gppoint{gp mark 0}{(7.070,6.355)}
\gppoint{gp mark 0}{(7.826,7.123)}
\gppoint{gp mark 0}{(7.225,6.390)}
\gppoint{gp mark 0}{(7.714,6.543)}
\gppoint{gp mark 0}{(7.343,6.421)}
\gppoint{gp mark 0}{(7.445,6.444)}
\gppoint{gp mark 0}{(7.681,6.531)}
\gppoint{gp mark 0}{(7.101,6.369)}
\gppoint{gp mark 0}{(8.089,6.712)}
\gppoint{gp mark 0}{(8.018,5.198)}
\gppoint{gp mark 0}{(8.001,6.676)}
\gppoint{gp mark 0}{(7.976,5.065)}
\gppoint{gp mark 0}{(8.092,5.345)}
\gppoint{gp mark 0}{(7.266,6.436)}
\gppoint{gp mark 0}{(8.089,5.376)}
\gppoint{gp mark 0}{(8.397,5.858)}
\gppoint{gp mark 0}{(7.239,6.421)}
\gppoint{gp mark 0}{(8.426,5.774)}
\gppoint{gp mark 0}{(8.105,5.428)}
\gppoint{gp mark 0}{(7.639,6.538)}
\gppoint{gp mark 0}{(8.204,6.768)}
\gppoint{gp mark 0}{(8.121,5.352)}
\gppoint{gp mark 0}{(7.830,6.560)}
\gppoint{gp mark 0}{(8.775,6.378)}
\gppoint{gp mark 0}{(8.124,5.355)}
\gppoint{gp mark 0}{(8.159,5.271)}
\gppoint{gp mark 0}{(8.089,6.726)}
\gppoint{gp mark 0}{(7.062,6.348)}
\gppoint{gp mark 0}{(7.993,5.267)}
\gppoint{gp mark 0}{(7.740,6.606)}
\gppoint{gp mark 0}{(7.253,6.343)}
\gppoint{gp mark 0}{(8.289,5.463)}
\gppoint{gp mark 0}{(7.928,6.583)}
\gppoint{gp mark 0}{(7.997,5.263)}
\gppoint{gp mark 0}{(6.733,6.202)}
\gppoint{gp mark 0}{(7.350,6.379)}
\gppoint{gp mark 0}{(8.004,5.327)}
\gppoint{gp mark 0}{(7.318,6.358)}
\gppoint{gp mark 0}{(8.008,6.706)}
\gppoint{gp mark 0}{(7.648,6.481)}
\gppoint{gp mark 0}{(8.152,5.194)}
\gppoint{gp mark 0}{(8.351,5.549)}
\gppoint{gp mark 0}{(8.108,5.045)}
\gppoint{gp mark 0}{(7.253,6.324)}
\gppoint{gp mark 0}{(7.714,6.473)}
\gppoint{gp mark 0}{(8.204,5.663)}
\gppoint{gp mark 0}{(7.745,6.614)}
\gppoint{gp mark 0}{(8.324,5.597)}
\gppoint{gp mark 0}{(7.279,6.378)}
\gppoint{gp mark 0}{(7.101,6.392)}
\gppoint{gp mark 0}{(7.749,4.730)}
\gppoint{gp mark 0}{(7.762,4.788)}
\gppoint{gp mark 0}{(8.052,5.259)}
\gppoint{gp mark 0}{(7.758,4.788)}
\gppoint{gp mark 0}{(7.070,6.388)}
\gppoint{gp mark 0}{(8.056,5.263)}
\gppoint{gp mark 0}{(6.900,6.197)}
\gppoint{gp mark 0}{(8.052,5.215)}
\gppoint{gp mark 0}{(7.886,6.581)}
\gppoint{gp mark 0}{(7.957,7.110)}
\gppoint{gp mark 0}{(9.440,6.198)}
\gppoint{gp mark 0}{(7.771,6.629)}
\gppoint{gp mark 0}{(8.028,5.359)}
\gppoint{gp mark 0}{(7.732,4.849)}
\gppoint{gp mark 0}{(8.629,6.888)}
\gppoint{gp mark 0}{(7.677,6.460)}
\gppoint{gp mark 0}{(8.028,5.355)}
\gppoint{gp mark 0}{(7.878,5.244)}
\gppoint{gp mark 0}{(7.972,5.422)}
\gppoint{gp mark 0}{(7.830,6.682)}
\gppoint{gp mark 0}{(7.913,5.301)}
\gppoint{gp mark 0}{(7.858,5.223)}
\gppoint{gp mark 0}{(7.889,5.286)}
\gppoint{gp mark 0}{(7.564,6.384)}
\gppoint{gp mark 0}{(7.846,5.203)}
\gppoint{gp mark 0}{(7.838,5.186)}
\gppoint{gp mark 0}{(7.663,6.425)}
\gppoint{gp mark 0}{(7.882,6.690)}
\gppoint{gp mark 0}{(7.433,7.013)}
\gppoint{gp mark 0}{(7.913,5.283)}
\gppoint{gp mark 0}{(7.801,5.109)}
\gppoint{gp mark 0}{(6.701,6.126)}
\gppoint{gp mark 0}{(7.817,5.090)}
\gppoint{gp mark 0}{(7.439,6.347)}
\gppoint{gp mark 0}{(8.394,5.495)}
\gppoint{gp mark 0}{(7.928,5.373)}
\gppoint{gp mark 0}{(7.796,5.142)}
\gppoint{gp mark 0}{(7.030,6.455)}
\gppoint{gp mark 0}{(7.809,5.194)}
\gppoint{gp mark 0}{(7.439,6.323)}
\gppoint{gp mark 0}{(7.943,5.432)}
\gppoint{gp mark 0}{(7.548,6.362)}
\gppoint{gp mark 0}{(7.834,6.674)}
\gppoint{gp mark 0}{(7.522,6.355)}
\gppoint{gp mark 0}{(8.018,6.552)}
\gppoint{gp mark 0}{(7.889,5.259)}
\gppoint{gp mark 0}{(7.801,5.173)}
\gppoint{gp mark 0}{(7.439,6.332)}
\gppoint{gp mark 0}{(7.723,5.040)}
\gppoint{gp mark 0}{(6.785,6.166)}
\gppoint{gp mark 0}{(7.838,5.065)}
\gppoint{gp mark 0}{(7.762,4.959)}
\gppoint{gp mark 0}{(7.858,5.312)}
\gppoint{gp mark 0}{(9.304,5.734)}
\gppoint{gp mark 0}{(8.245,5.858)}
\gppoint{gp mark 0}{(7.495,6.382)}
\gppoint{gp mark 0}{(8.441,5.543)}
\gppoint{gp mark 0}{(7.826,5.100)}
\gppoint{gp mark 0}{(7.834,4.981)}
\gppoint{gp mark 0}{(7.554,6.350)}
\gppoint{gp mark 0}{(7.801,4.906)}
\gppoint{gp mark 0}{(7.909,5.432)}
\gppoint{gp mark 0}{(7.506,6.326)}
\gppoint{gp mark 0}{(8.507,5.680)}
\gppoint{gp mark 0}{(7.727,5.114)}
\gppoint{gp mark 0}{(7.754,5.198)}
\gppoint{gp mark 0}{(7.939,5.215)}
\gppoint{gp mark 0}{(7.511,6.325)}
\gppoint{gp mark 0}{(7.961,5.271)}
\gppoint{gp mark 0}{(7.548,6.338)}
\gppoint{gp mark 0}{(7.740,5.203)}
\gppoint{gp mark 0}{(8.486,5.656)}
\gppoint{gp mark 0}{(7.467,6.364)}
\gppoint{gp mark 0}{(7.740,5.194)}
\gppoint{gp mark 0}{(7.732,5.137)}
\gppoint{gp mark 0}{(7.422,6.380)}
\gppoint{gp mark 0}{(6.586,6.153)}
\gppoint{gp mark 0}{(6.785,6.116)}
\gppoint{gp mark 0}{(7.343,6.500)}
\gppoint{gp mark 0}{(7.878,5.432)}
\gppoint{gp mark 0}{(7.022,6.481)}
\gppoint{gp mark 0}{(7.767,5.080)}
\gppoint{gp mark 0}{(7.897,5.369)}
\gppoint{gp mark 0}{(8.204,5.858)}
\gppoint{gp mark 0}{(7.792,5.050)}
\gppoint{gp mark 0}{(7.718,6.684)}
\gppoint{gp mark 0}{(7.767,5.104)}
\gppoint{gp mark 0}{(7.965,5.219)}
\gppoint{gp mark 0}{(7.736,5.137)}
\gppoint{gp mark 0}{(6.997,6.481)}
\gppoint{gp mark 0}{(7.745,5.244)}
\gppoint{gp mark 0}{(7.511,6.423)}
\gppoint{gp mark 0}{(8.270,6.875)}
\gppoint{gp mark 0}{(7.813,5.366)}
\gppoint{gp mark 0}{(7.968,5.181)}
\gppoint{gp mark 0}{(7.897,6.660)}
\gppoint{gp mark 0}{(7.506,6.424)}
\gppoint{gp mark 0}{(7.905,5.030)}
\gppoint{gp mark 0}{(7.788,5.338)}
\gppoint{gp mark 0}{(7.796,6.717)}
\gppoint{gp mark 0}{(7.771,5.286)}
\gppoint{gp mark 0}{(7.732,5.267)}
\gppoint{gp mark 0}{(7.456,6.401)}
\gppoint{gp mark 0}{(7.368,6.488)}
\gppoint{gp mark 0}{(7.758,5.309)}
\gppoint{gp mark 0}{(7.732,5.259)}
\gppoint{gp mark 0}{(7.788,5.366)}
\gppoint{gp mark 0}{(7.740,5.327)}
\gppoint{gp mark 0}{(8.102,6.573)}
\gppoint{gp mark 0}{(7.961,6.674)}
\gppoint{gp mark 0}{(8.204,5.934)}
\gppoint{gp mark 0}{(7.838,5.379)}
\gppoint{gp mark 0}{(7.968,5.109)}
\gppoint{gp mark 0}{(7.805,5.432)}
\gppoint{gp mark 0}{(7.749,5.301)}
\gppoint{gp mark 0}{(8.399,6.778)}
\gppoint{gp mark 0}{(7.522,6.416)}
\gppoint{gp mark 0}{(7.878,5.040)}
\gppoint{gp mark 0}{(6.918,6.522)}
\gppoint{gp mark 0}{(7.749,5.279)}
\gppoint{gp mark 0}{(8.497,5.578)}
\gppoint{gp mark 0}{(7.605,6.335)}
\gppoint{gp mark 0}{(7.957,5.100)}
\gppoint{gp mark 0}{(7.826,5.373)}
\gppoint{gp mark 0}{(7.439,6.395)}
\gppoint{gp mark 0}{(7.878,4.998)}
\gppoint{gp mark 0}{(8.379,6.794)}
\gppoint{gp mark 0}{(7.445,6.430)}
\gppoint{gp mark 0}{(7.356,6.461)}
\gppoint{gp mark 0}{(7.404,6.417)}
\gppoint{gp mark 0}{(7.564,6.360)}
\gppoint{gp mark 0}{(7.878,5.132)}
\gppoint{gp mark 0}{(7.946,4.992)}
\gppoint{gp mark 0}{(7.935,4.912)}
\gppoint{gp mark 0}{(8.281,5.852)}
\gppoint{gp mark 0}{(7.767,5.412)}
\gppoint{gp mark 0}{(7.350,6.469)}
\gppoint{gp mark 0}{(6.824,6.001)}
\gppoint{gp mark 0}{(7.796,5.263)}
\gppoint{gp mark 0}{(7.838,5.259)}
\gppoint{gp mark 0}{(7.889,5.080)}
\gppoint{gp mark 0}{(7.132,6.503)}
\gppoint{gp mark 0}{(7.874,6.684)}
\gppoint{gp mark 0}{(7.754,5.331)}
\gppoint{gp mark 0}{(7.495,6.404)}
\gppoint{gp mark 0}{(7.850,5.259)}
\gppoint{gp mark 0}{(7.983,7.155)}
\gppoint{gp mark 0}{(8.155,6.566)}
\gppoint{gp mark 0}{(7.780,5.334)}
\gppoint{gp mark 0}{(6.882,6.004)}
\gppoint{gp mark 0}{(7.882,5.168)}
\gppoint{gp mark 0}{(7.754,5.369)}
\gppoint{gp mark 0}{(7.714,6.314)}
\gppoint{gp mark 0}{(7.677,6.295)}
\gppoint{gp mark 0}{(7.522,6.239)}
\gppoint{gp mark 0}{(7.826,5.584)}
\gppoint{gp mark 0}{(8.015,5.810)}
\gppoint{gp mark 0}{(7.862,5.627)}
\gppoint{gp mark 0}{(7.232,6.111)}
\gppoint{gp mark 0}{(7.704,6.314)}
\gppoint{gp mark 0}{(7.695,6.310)}
\gppoint{gp mark 0}{(7.767,6.746)}
\gppoint{gp mark 0}{(7.862,5.632)}
\gppoint{gp mark 0}{(7.681,6.316)}
\gppoint{gp mark 0}{(7.886,5.684)}
\gppoint{gp mark 0}{(7.479,6.231)}
\gppoint{gp mark 0}{(7.609,6.281)}
\gppoint{gp mark 0}{(7.629,6.272)}
\gppoint{gp mark 0}{(6.918,6.012)}
\gppoint{gp mark 0}{(7.392,6.185)}
\gppoint{gp mark 0}{(7.931,5.757)}
\gppoint{gp mark 0}{(7.038,6.007)}
\gppoint{gp mark 0}{(7.758,5.456)}
\gppoint{gp mark 0}{(7.634,6.263)}
\gppoint{gp mark 0}{(7.456,6.192)}
\gppoint{gp mark 0}{(7.916,5.744)}
\gppoint{gp mark 0}{(7.312,6.171)}
\gppoint{gp mark 0}{(7.479,6.239)}
\gppoint{gp mark 0}{(7.924,5.734)}
\gppoint{gp mark 0}{(7.605,6.251)}
\gppoint{gp mark 0}{(7.854,5.659)}
\gppoint{gp mark 0}{(7.511,6.247)}
\gppoint{gp mark 0}{(7.305,6.160)}
\gppoint{gp mark 0}{(7.732,5.501)}
\gppoint{gp mark 0}{(7.634,6.266)}
\gppoint{gp mark 0}{(7.745,5.527)}
\gppoint{gp mark 0}{(7.939,5.759)}
\gppoint{gp mark 0}{(7.416,6.203)}
\gppoint{gp mark 0}{(8.404,5.146)}
\gppoint{gp mark 0}{(8.183,5.946)}
\gppoint{gp mark 0}{(8.008,5.792)}
\gppoint{gp mark 0}{(7.253,6.101)}
\gppoint{gp mark 0}{(7.161,6.050)}
\gppoint{gp mark 0}{(7.723,5.516)}
\gppoint{gp mark 0}{(7.775,5.444)}
\gppoint{gp mark 0}{(7.878,5.670)}
\gppoint{gp mark 0}{(7.905,5.617)}
\gppoint{gp mark 0}{(7.718,5.546)}
\gppoint{gp mark 0}{(7.139,6.024)}
\gppoint{gp mark 0}{(7.809,5.481)}
\gppoint{gp mark 0}{(7.337,6.119)}
\gppoint{gp mark 0}{(7.893,5.740)}
\gppoint{gp mark 0}{(7.723,5.541)}
\gppoint{gp mark 0}{(7.225,6.152)}
\gppoint{gp mark 0}{(7.913,5.736)}
\gppoint{gp mark 0}{(7.846,5.489)}
\gppoint{gp mark 0}{(7.374,6.112)}
\gppoint{gp mark 0}{(7.968,5.656)}
\gppoint{gp mark 0}{(7.204,6.138)}
\gppoint{gp mark 0}{(7.473,6.233)}
\gppoint{gp mark 0}{(7.218,6.136)}
\gppoint{gp mark 0}{(7.634,6.306)}
\gppoint{gp mark 0}{(7.500,6.181)}
\gppoint{gp mark 0}{(6.954,6.052)}
\gppoint{gp mark 0}{(7.183,6.148)}
\gppoint{gp mark 0}{(7.305,6.100)}
\gppoint{gp mark 0}{(7.950,5.682)}
\gppoint{gp mark 0}{(7.893,5.757)}
\gppoint{gp mark 0}{(7.758,6.765)}
\gppoint{gp mark 0}{(6.997,6.081)}
\gppoint{gp mark 0}{(7.218,6.138)}
\gppoint{gp mark 0}{(7.554,6.201)}
\gppoint{gp mark 0}{(7.920,5.637)}
\gppoint{gp mark 0}{(7.538,7.328)}
\gppoint{gp mark 0}{(7.609,6.293)}
\gppoint{gp mark 0}{(7.070,6.021)}
\gppoint{gp mark 0}{(7.913,5.718)}
\gppoint{gp mark 0}{(7.279,6.147)}
\gppoint{gp mark 0}{(7.522,6.179)}
\gppoint{gp mark 0}{(7.410,6.246)}
\gppoint{gp mark 0}{(7.022,6.065)}
\gppoint{gp mark 0}{(7.862,5.740)}
\gppoint{gp mark 0}{(7.343,6.090)}
\gppoint{gp mark 0}{(7.801,5.498)}
\gppoint{gp mark 0}{(7.350,6.097)}
\gppoint{gp mark 0}{(8.346,7.136)}
\gppoint{gp mark 0}{(8.095,5.985)}
\gppoint{gp mark 0}{(7.511,6.193)}
\gppoint{gp mark 0}{(6.927,6.085)}
\gppoint{gp mark 0}{(8.028,5.828)}
\gppoint{gp mark 0}{(7.139,6.007)}
\gppoint{gp mark 0}{(8.022,5.835)}
\gppoint{gp mark 0}{(7.279,6.142)}
\gppoint{gp mark 0}{(7.422,6.240)}
\gppoint{gp mark 0}{(7.758,5.557)}
\gppoint{gp mark 0}{(7.094,6.021)}
\gppoint{gp mark 0}{(7.368,6.089)}
\gppoint{gp mark 0}{(7.901,5.698)}
\gppoint{gp mark 0}{(7.183,6.164)}
\gppoint{gp mark 0}{(7.495,6.208)}
\gppoint{gp mark 0}{(7.404,6.246)}
\gppoint{gp mark 0}{(7.749,5.644)}
\gppoint{gp mark 0}{(7.456,6.283)}
\gppoint{gp mark 0}{(7.928,6.750)}
\gppoint{gp mark 0}{(7.866,5.444)}
\gppoint{gp mark 0}{(7.727,5.625)}
\gppoint{gp mark 0}{(7.913,5.513)}
\gppoint{gp mark 0}{(7.928,5.532)}
\gppoint{gp mark 0}{(7.239,6.024)}
\gppoint{gp mark 0}{(7.380,6.085)}
\gppoint{gp mark 0}{(7.882,5.478)}
\gppoint{gp mark 0}{(7.771,5.680)}
\gppoint{gp mark 0}{(7.882,5.475)}
\gppoint{gp mark 0}{(7.495,6.298)}
\gppoint{gp mark 0}{(7.232,6.039)}
\gppoint{gp mark 0}{(7.749,5.622)}
\gppoint{gp mark 0}{(7.427,6.251)}
\gppoint{gp mark 0}{(7.225,5.996)}
\gppoint{gp mark 0}{(7.473,6.268)}
\gppoint{gp mark 0}{(7.834,5.747)}
\gppoint{gp mark 0}{(7.901,5.489)}
\gppoint{gp mark 0}{(7.920,5.560)}
\gppoint{gp mark 0}{(7.318,6.043)}
\gppoint{gp mark 0}{(7.758,5.634)}
\gppoint{gp mark 0}{(7.204,6.039)}
\gppoint{gp mark 0}{(7.579,6.196)}
\gppoint{gp mark 0}{(7.889,5.450)}
\gppoint{gp mark 0}{(7.854,5.489)}
\gppoint{gp mark 0}{(7.821,5.694)}
\gppoint{gp mark 0}{(7.197,6.027)}
\gppoint{gp mark 0}{(7.398,6.278)}
\gppoint{gp mark 0}{(7.905,5.447)}
\gppoint{gp mark 0}{(7.874,5.495)}
\gppoint{gp mark 0}{(7.211,6.022)}
\gppoint{gp mark 0}{(7.897,5.481)}
\gppoint{gp mark 0}{(7.517,6.308)}
\gppoint{gp mark 0}{(7.324,6.008)}
\gppoint{gp mark 0}{(7.916,5.450)}
\gppoint{gp mark 0}{(7.132,6.111)}
\gppoint{gp mark 0}{(7.161,6.122)}
\gppoint{gp mark 0}{(7.416,6.300)}
\gppoint{gp mark 0}{(7.259,6.088)}
\gppoint{gp mark 0}{(7.943,5.466)}
\gppoint{gp mark 0}{(7.983,6.862)}
\gppoint{gp mark 0}{(7.882,5.554)}
\gppoint{gp mark 0}{(7.374,6.034)}
\gppoint{gp mark 0}{(7.517,6.267)}
\gppoint{gp mark 0}{(7.901,5.597)}
\gppoint{gp mark 0}{(7.479,6.250)}
\gppoint{gp mark 0}{(7.211,6.896)}
\gppoint{gp mark 0}{(7.266,6.078)}
\gppoint{gp mark 0}{(7.305,6.001)}
\gppoint{gp mark 0}{(7.846,5.656)}
\gppoint{gp mark 0}{(8.162,6.821)}
\gppoint{gp mark 0}{(7.324,5.999)}
\gppoint{gp mark 0}{(7.117,6.130)}
\gppoint{gp mark 0}{(7.343,6.034)}
\gppoint{gp mark 0}{(7.846,5.673)}
\gppoint{gp mark 0}{(7.714,6.198)}
\gppoint{gp mark 0}{(7.928,5.495)}
\gppoint{gp mark 0}{(7.362,6.015)}
\gppoint{gp mark 0}{(7.380,6.010)}
\gppoint{gp mark 0}{(7.830,6.772)}
\gppoint{gp mark 0}{(7.600,6.242)}
\gppoint{gp mark 0}{(7.495,6.269)}
\gppoint{gp mark 0}{(7.490,6.273)}
\gppoint{gp mark 0}{(7.190,6.072)}
\gppoint{gp mark 0}{(7.305,6.028)}
\gppoint{gp mark 0}{(8.168,5.778)}
\gppoint{gp mark 0}{(7.368,6.008)}
\gppoint{gp mark 0}{(7.913,5.568)}
\gppoint{gp mark 0}{(7.866,5.589)}
\gppoint{gp mark 0}{(7.574,6.240)}
\gppoint{gp mark 0}{(7.013,6.139)}
\gppoint{gp mark 0}{(7.350,6.007)}
\gppoint{gp mark 0}{(7.259,6.045)}
\gppoint{gp mark 0}{(7.225,6.076)}
\gppoint{gp mark 0}{(7.211,6.069)}
\gppoint{gp mark 0}{(8.086,5.629)}
\gppoint{gp mark 0}{(7.548,6.088)}
\gppoint{gp mark 0}{(7.574,6.090)}
\gppoint{gp mark 0}{(7.456,6.037)}
\gppoint{gp mark 0}{(8.056,5.573)}
\gppoint{gp mark 0}{(8.319,5.327)}
\gppoint{gp mark 0}{(7.634,6.127)}
\gppoint{gp mark 0}{(7.101,6.227)}
\gppoint{gp mark 0}{(7.450,6.028)}
\gppoint{gp mark 0}{(7.686,6.155)}
\gppoint{gp mark 0}{(7.286,6.289)}
\gppoint{gp mark 0}{(7.838,5.876)}
\gppoint{gp mark 0}{(7.805,5.837)}
\gppoint{gp mark 0}{(7.427,5.996)}
\gppoint{gp mark 0}{(8.300,5.323)}
\gppoint{gp mark 0}{(7.331,6.293)}
\gppoint{gp mark 0}{(6.945,6.192)}
\gppoint{gp mark 0}{(7.854,6.848)}
\gppoint{gp mark 0}{(7.695,6.136)}
\gppoint{gp mark 0}{(7.506,6.088)}
\gppoint{gp mark 0}{(7.522,6.050)}
\gppoint{gp mark 0}{(7.663,6.170)}
\gppoint{gp mark 0}{(7.538,6.046)}
\gppoint{gp mark 0}{(8.177,5.725)}
\gppoint{gp mark 0}{(7.467,6.007)}
\gppoint{gp mark 0}{(8.281,5.286)}
\gppoint{gp mark 0}{(7.672,6.166)}
\gppoint{gp mark 0}{(7.386,6.021)}
\gppoint{gp mark 0}{(8.149,5.747)}
\gppoint{gp mark 0}{(7.893,5.894)}
\gppoint{gp mark 0}{(7.343,6.288)}
\gppoint{gp mark 0}{(7.410,6.031)}
\gppoint{gp mark 0}{(7.500,6.082)}
\gppoint{gp mark 0}{(7.473,6.013)}
\gppoint{gp mark 0}{(7.445,6.008)}
\gppoint{gp mark 0}{(7.410,6.025)}
\gppoint{gp mark 0}{(8.359,5.341)}
\gppoint{gp mark 0}{(7.176,6.285)}
\gppoint{gp mark 0}{(7.554,6.046)}
\gppoint{gp mark 0}{(7.727,5.816)}
\gppoint{gp mark 0}{(7.410,6.021)}
\gppoint{gp mark 0}{(7.356,6.302)}
\gppoint{gp mark 0}{(7.543,6.045)}
\gppoint{gp mark 0}{(7.484,6.082)}
\gppoint{gp mark 0}{(7.479,6.079)}
\gppoint{gp mark 0}{(7.784,5.776)}
\gppoint{gp mark 0}{(7.972,5.951)}
\gppoint{gp mark 0}{(7.380,6.291)}
\gppoint{gp mark 0}{(7.813,5.871)}
\gppoint{gp mark 0}{(7.439,6.008)}
\gppoint{gp mark 0}{(7.479,6.082)}
\gppoint{gp mark 0}{(7.484,6.079)}
\gppoint{gp mark 0}{(7.473,6.001)}
\gppoint{gp mark 0}{(7.983,5.535)}
\gppoint{gp mark 0}{(7.176,6.290)}
\gppoint{gp mark 0}{(7.564,6.138)}
\gppoint{gp mark 0}{(7.639,6.165)}
\gppoint{gp mark 0}{(7.916,5.898)}
\gppoint{gp mark 0}{(7.299,6.252)}
\gppoint{gp mark 0}{(7.870,5.958)}
\gppoint{gp mark 0}{(7.946,5.915)}
\gppoint{gp mark 0}{(7.805,5.786)}
\gppoint{gp mark 0}{(7.595,6.152)}
\gppoint{gp mark 0}{(7.548,6.033)}
\gppoint{gp mark 0}{(7.433,6.069)}
\gppoint{gp mark 0}{(7.500,6.002)}
\gppoint{gp mark 0}{(8.460,4.675)}
\gppoint{gp mark 0}{(7.386,6.061)}
\gppoint{gp mark 0}{(7.299,6.259)}
\gppoint{gp mark 0}{(7.834,5.816)}
\gppoint{gp mark 0}{(7.866,5.958)}
\gppoint{gp mark 0}{(8.032,5.466)}
\gppoint{gp mark 0}{(8.022,5.584)}
\gppoint{gp mark 0}{(7.686,6.124)}
\gppoint{gp mark 0}{(7.762,5.885)}
\gppoint{gp mark 0}{(7.589,6.134)}
\gppoint{gp mark 0}{(7.548,6.021)}
\gppoint{gp mark 0}{(6.963,6.213)}
\gppoint{gp mark 0}{(7.450,6.083)}
\gppoint{gp mark 0}{(7.614,6.171)}
\gppoint{gp mark 0}{(7.416,6.088)}
\gppoint{gp mark 0}{(7.337,6.256)}
\gppoint{gp mark 0}{(7.139,6.176)}
\gppoint{gp mark 0}{(7.834,5.768)}
\gppoint{gp mark 0}{(7.500,6.042)}
\gppoint{gp mark 0}{(7.218,6.315)}
\gppoint{gp mark 0}{(7.439,6.049)}
\gppoint{gp mark 0}{(7.246,6.293)}
\gppoint{gp mark 0}{(7.445,6.053)}
\gppoint{gp mark 0}{(7.124,6.174)}
\gppoint{gp mark 0}{(7.658,6.124)}
\gppoint{gp mark 0}{(7.574,6.167)}
\gppoint{gp mark 0}{(7.838,5.778)}
\gppoint{gp mark 0}{(7.796,5.816)}
\gppoint{gp mark 0}{(7.983,5.602)}
\gppoint{gp mark 0}{(7.805,5.798)}
\gppoint{gp mark 0}{(7.462,6.048)}
\gppoint{gp mark 0}{(7.543,5.999)}
\gppoint{gp mark 0}{(8.952,6.019)}
\gppoint{gp mark 0}{(9.636,8.345)}
\gppoint{gp mark 0}{(7.644,6.124)}
\gppoint{gp mark 0}{(7.479,6.034)}
\gppoint{gp mark 0}{(7.543,5.998)}
\gppoint{gp mark 0}{(7.554,6.165)}
\gppoint{gp mark 0}{(7.796,5.943)}
\gppoint{gp mark 0}{(7.889,5.808)}
\gppoint{gp mark 0}{(8.001,5.639)}
\gppoint{gp mark 0}{(7.517,6.143)}
\gppoint{gp mark 0}{(7.232,6.201)}
\gppoint{gp mark 0}{(7.211,6.187)}
\gppoint{gp mark 0}{(7.362,6.244)}
\gppoint{gp mark 0}{(7.299,6.221)}
\gppoint{gp mark 0}{(8.127,5.513)}
\gppoint{gp mark 0}{(7.450,6.126)}
\gppoint{gp mark 0}{(6.668,6.323)}
\gppoint{gp mark 0}{(7.771,5.924)}
\gppoint{gp mark 0}{(7.410,6.095)}
\gppoint{gp mark 0}{(8.134,5.507)}
\gppoint{gp mark 0}{(8.118,5.441)}
\gppoint{gp mark 0}{(7.723,5.926)}
\gppoint{gp mark 0}{(7.484,6.160)}
\gppoint{gp mark 0}{(7.846,5.961)}
\gppoint{gp mark 0}{(7.490,6.156)}
\gppoint{gp mark 0}{(7.595,6.034)}
\gppoint{gp mark 0}{(7.473,6.110)}
\gppoint{gp mark 0}{(7.266,6.187)}
\gppoint{gp mark 0}{(7.154,6.298)}
\gppoint{gp mark 0}{(7.629,6.012)}
\gppoint{gp mark 0}{(7.559,6.152)}
\gppoint{gp mark 0}{(7.648,6.069)}
\gppoint{gp mark 0}{(7.450,6.096)}
\gppoint{gp mark 0}{(7.368,6.223)}
\gppoint{gp mark 0}{(8.099,6.743)}
\gppoint{gp mark 0}{(7.704,6.055)}
\gppoint{gp mark 0}{(7.796,5.979)}
\gppoint{gp mark 0}{(7.204,6.200)}
\gppoint{gp mark 0}{(7.246,6.187)}
\gppoint{gp mark 0}{(7.714,6.053)}
\gppoint{gp mark 0}{(7.479,6.166)}
\gppoint{gp mark 0}{(7.663,6.015)}
\gppoint{gp mark 0}{(7.700,6.037)}
\gppoint{gp mark 0}{(7.709,6.031)}
\gppoint{gp mark 0}{(7.410,6.148)}
\gppoint{gp mark 0}{(8.676,7.397)}
\gppoint{gp mark 0}{(7.886,5.871)}
\gppoint{gp mark 0}{(7.624,6.088)}
\gppoint{gp mark 0}{(7.813,5.907)}
\gppoint{gp mark 0}{(7.439,6.160)}
\gppoint{gp mark 0}{(7.874,5.851)}
\gppoint{gp mark 0}{(7.543,6.123)}
\gppoint{gp mark 0}{(9.065,6.815)}
\gppoint{gp mark 0}{(6.927,6.288)}
\gppoint{gp mark 0}{(7.380,6.208)}
\gppoint{gp mark 0}{(7.197,6.226)}
\gppoint{gp mark 0}{(7.691,6.031)}
\gppoint{gp mark 0}{(7.490,6.104)}
\gppoint{gp mark 0}{(7.398,6.146)}
\gppoint{gp mark 0}{(7.124,6.279)}
\gppoint{gp mark 0}{(7.624,6.066)}
\gppoint{gp mark 0}{(8.115,6.754)}
\gppoint{gp mark 0}{(7.935,5.763)}
\gppoint{gp mark 0}{(7.882,5.835)}
\gppoint{gp mark 0}{(7.343,6.203)}
\gppoint{gp mark 0}{(8.340,5.030)}
\gppoint{gp mark 0}{(8.210,5.355)}
\gppoint{gp mark 0}{(7.767,5.944)}
\gppoint{gp mark 0}{(7.634,6.059)}
\gppoint{gp mark 0}{(7.916,5.810)}
\gppoint{gp mark 0}{(7.878,5.876)}
\gppoint{gp mark 0}{(7.920,5.804)}
\gppoint{gp mark 0}{(7.928,5.810)}
\gppoint{gp mark 0}{(7.817,5.939)}
\gppoint{gp mark 0}{(7.870,5.878)}
\gppoint{gp mark 0}{(7.850,5.910)}
\gppoint{gp mark 0}{(7.522,6.093)}
\gppoint{gp mark 0}{(7.870,5.876)}
\gppoint{gp mark 0}{(7.817,5.936)}
\gppoint{gp mark 0}{(8.219,5.369)}
\gppoint{gp mark 0}{(7.522,6.106)}
\gppoint{gp mark 0}{(7.663,6.885)}
\gppoint{gp mark 0}{(7.467,6.136)}
\gppoint{gp mark 0}{(7.653,6.036)}
\gppoint{gp mark 0}{(7.842,5.896)}
\gppoint{gp mark 0}{(7.589,6.074)}
\gppoint{gp mark 0}{(7.901,5.839)}
\gppoint{gp mark 0}{(7.337,6.191)}
\gppoint{gp mark 0}{(7.826,5.905)}
\node[gp node left] at (9.927,1.935) {$N$ };
\gpcolor{color=gp lt color 2}
\gpsetlinetype{gp lt plot 0}
\draw[gp path] (10.663,1.935)--(11.579,1.935);
\draw[gp path] (1.320,0.985)--(1.427,1.060)--(1.535,1.134)--(1.642,1.209)--(1.749,1.284)%
  --(1.857,1.359)--(1.964,1.433)--(2.071,1.508)--(2.179,1.583)--(2.286,1.657)--(2.393,1.732)%
  --(2.501,1.807)--(2.608,1.881)--(2.715,1.956)--(2.823,2.031)--(2.930,2.106)--(3.037,2.180)%
  --(3.145,2.255)--(3.252,2.330)--(3.360,2.404)--(3.467,2.479)--(3.574,2.554)--(3.682,2.629)%
  --(3.789,2.703)--(3.896,2.778)--(4.004,2.853)--(4.111,2.927)--(4.218,3.002)--(4.326,3.077)%
  --(4.433,3.152)--(4.540,3.226)--(4.648,3.301)--(4.755,3.376)--(4.862,3.450)--(4.970,3.525)%
  --(5.077,3.600)--(5.184,3.674)--(5.292,3.749)--(5.399,3.824)--(5.506,3.899)--(5.614,3.973)%
  --(5.721,4.048)--(5.828,4.123)--(5.936,4.197)--(6.043,4.272)--(6.150,4.347)--(6.258,4.422)%
  --(6.365,4.496)--(6.472,4.571)--(6.580,4.646)--(6.687,4.720)--(6.795,4.795)--(6.902,4.870)%
  --(7.009,4.944)--(7.117,5.019)--(7.224,5.094)--(7.331,5.169)--(7.439,5.243)--(7.546,5.318)%
  --(7.653,5.393)--(7.761,5.467)--(7.868,5.542)--(7.975,5.617)--(8.083,5.692)--(8.190,5.766)%
  --(8.297,5.841)--(8.405,5.916)--(8.512,5.990)--(8.619,6.065)--(8.727,6.140)--(8.834,6.214)%
  --(8.941,6.289)--(9.049,6.364)--(9.156,6.439)--(9.263,6.513)--(9.371,6.588)--(9.478,6.663)%
  --(9.585,6.737)--(9.693,6.812)--(9.800,6.887)--(9.907,6.962)--(10.015,7.036)--(10.122,7.111)%
  --(10.230,7.186)--(10.337,7.260)--(10.444,7.335)--(10.552,7.410)--(10.659,7.485)--(10.766,7.559)%
  --(10.874,7.634)--(10.981,7.709)--(11.088,7.783)--(11.196,7.858)--(11.303,7.933)--(11.410,8.007)%
  --(11.518,8.082)--(11.625,8.157)--(11.732,8.232)--(11.840,8.306);
\gpcolor{color=gp lt color border}
\node[gp node left] at (9.927,1.627) {$N^2$ };
\gpcolor{color=gp lt color 3}
\draw[gp path] (10.663,1.627)--(11.579,1.627);
\draw[gp path] (3.048,0.985)--(3.145,1.119)--(3.252,1.269)--(3.360,1.418)--(3.467,1.568)%
  --(3.574,1.717)--(3.682,1.867)--(3.789,2.016)--(3.896,2.165)--(4.004,2.315)--(4.111,2.464)%
  --(4.218,2.614)--(4.326,2.763)--(4.433,2.912)--(4.540,3.062)--(4.648,3.211)--(4.755,3.361)%
  --(4.862,3.510)--(4.970,3.659)--(5.077,3.809)--(5.184,3.958)--(5.292,4.108)--(5.399,4.257)%
  --(5.506,4.407)--(5.614,4.556)--(5.721,4.705)--(5.828,4.855)--(5.936,5.004)--(6.043,5.154)%
  --(6.150,5.303)--(6.258,5.452)--(6.365,5.602)--(6.472,5.751)--(6.580,5.901)--(6.687,6.050)%
  --(6.795,6.200)--(6.902,6.349)--(7.009,6.498)--(7.117,6.648)--(7.224,6.797)--(7.331,6.947)%
  --(7.439,7.096)--(7.546,7.245)--(7.653,7.395)--(7.761,7.544)--(7.868,7.694)--(7.975,7.843)%
  --(8.083,7.992)--(8.190,8.142)--(8.297,8.291)--(8.362,8.381);
\gpcolor{color=gp lt color border}
\node[gp node left] at (9.927,1.319) {$N^3$ };
\gpcolor{color=gp lt color 6}
\draw[gp path] (10.663,1.319)--(11.579,1.319);
\draw[gp path] (3.275,0.985)--(3.360,1.161)--(3.467,1.386)--(3.574,1.610)--(3.682,1.834)%
  --(3.789,2.058)--(3.896,2.282)--(4.004,2.506)--(4.111,2.730)--(4.218,2.954)--(4.326,3.179)%
  --(4.433,3.403)--(4.540,3.627)--(4.648,3.851)--(4.755,4.075)--(4.862,4.299)--(4.970,4.523)%
  --(5.077,4.747)--(5.184,4.971)--(5.292,5.196)--(5.399,5.420)--(5.506,5.644)--(5.614,5.868)%
  --(5.721,6.092)--(5.828,6.316)--(5.936,6.540)--(6.043,6.764)--(6.150,6.989)--(6.258,7.213)%
  --(6.365,7.437)--(6.472,7.661)--(6.580,7.885)--(6.687,8.109)--(6.795,8.333)--(6.817,8.381);
\gpcolor{color=gp lt color border}
\gpsetlinetype{gp lt border}
\gpsetlinewidth{1.00}
\draw[gp path] (1.320,8.381)--(1.320,0.985)--(11.947,0.985)--(11.947,8.381)--cycle;
%% coordinates of the plot area
\gpdefrectangularnode{gp plot 1}{\pgfpoint{1.320cm}{0.985cm}}{\pgfpoint{11.947cm}{8.381cm}}
\end{tikzpicture}
%% gnuplot variables

      \end{myplot}

      Для хранения множеств целей указателей алгоритм использует значительные
      объемы памяти. На рис.~\ref{plot:mem_used} представлена зависимость
      этого объема от размеров программы. Объем потребляемой памяти имеет
      существенный разброс при анализе больших методов, но может быть
      оценен сверху величиной, пропорциональной квадрату размера метода.
      Такой результат является приемлимым для использования в рамках
      оптимизирующего компилятора.

    \subsection{Сравнение точности}

      Нужно понимать, что сравнивать результат работы алгоритма с <<реальными>>
      результатами невозможно, так как получение таких результатов потребует
      исполнения программы на всех возможных входных данных. Поэтому
      сравнивалась относительная точность разных алгоритмов анализа.
      Для сравнения с основным алгоритмом были использованы следующие его
      вариации:
      \begin{itemize}
        \item алгоритм, не учитывающий информацию о типах языка (на графиках
              обозначен как <<\eng{w/o types}>>);
        \item алгоритм, не учитывающий потоки данных в программе (учитывающий
              только информацию о типах) (на графиках обозначен как <<\eng{w/o
              data flow}>>);
        \item алгоритм \eng{equality-based} типа (на графиках обозначен как
              <<\eng{equality-based}>>).
      \end{itemize}
      Работа этих алгоритмов были сымитирована основным алгоритмом, что не дает
      истинных результатов о скорости их работы и потреблении памяти, однако
      предоставляет достоверную информацию об их точности.

      Сравнение четырех алгоритмов анализа проводилось на всех трех тестовых
      приложениях.
      В качестве меры точности результатов анализа использовалось усредненное
      количество синонимов для переменных ссылочного типа, поделенное на
      общее количество переменных.
      Чем ниже это число, тем более точные результаты дает алгоритм анализа.
      Результаты представлены в виде двух типов графиков.
      На первом представлена зависимость количества методов от значения меры
      точности, которая получилась при анализе этих методов
      (см.~рис.~\ref{plot:specjvm_all_aliases_distribution},
                \ref{plot:specjvm2008_all_aliases_distribution} и
                \ref{plot:eclipse_all_aliases_distribution}).
      Второй график является кумулятивной версией первого
      (см.~рис.~\ref{plot:specjvm_all_aliases_distribution_cumulative},
                \ref{plot:specjvm2008_all_aliases_distribution_cumulative} и
                \ref{plot:eclipse_all_aliases_distribution_cumulative}),
      и на нем удобнее сравнивать точность различных алгоритмов:
      среди двух алгоритмов анализа более точным является тот, у которого
      соответствующая кривая проходит выше.

      \begin{myplot}%
        {Зависимость количества методов от значения меры точности для \eng{SPEC~JVM98}}%
        {plot:specjvm_all_aliases_distribution}
        \begin{tikzpicture}[gnuplot]
%% generated with GNUPLOT 4.5p0 (Lua 5.1; terminal rev. 99, script rev. 98)
%% 27.05.2011 11:41:14
\path (0.000,0.000) rectangle (12.500,8.750);
\gpcolor{color=gp lt color border}
\gpsetlinetype{gp lt border}
\gpsetlinewidth{1.00}
\draw[gp path] (1.688,0.985)--(1.868,0.985);
\draw[gp path] (11.947,0.985)--(11.767,0.985);
\node[gp node right] at (1.504,0.985) {\num{0}};
\draw[gp path] (1.688,2.834)--(1.868,2.834);
\draw[gp path] (11.947,2.834)--(11.767,2.834);
\node[gp node right] at (1.504,2.834) {\num{0.05}};
\draw[gp path] (1.688,4.683)--(1.868,4.683);
\draw[gp path] (11.947,4.683)--(11.767,4.683);
\node[gp node right] at (1.504,4.683) {\num{0.1}};
\draw[gp path] (1.688,6.532)--(1.868,6.532);
\draw[gp path] (11.947,6.532)--(11.767,6.532);
\node[gp node right] at (1.504,6.532) {\num{0.15}};
\draw[gp path] (1.688,8.381)--(1.868,8.381);
\draw[gp path] (11.947,8.381)--(11.767,8.381);
\node[gp node right] at (1.504,8.381) {\num{0.2}};
\draw[gp path] (1.688,0.985)--(1.688,1.165);
\draw[gp path] (1.688,8.381)--(1.688,8.201);
\node[gp node center] at (1.688,0.677) {\num{0}};
\draw[gp path] (3.740,0.985)--(3.740,1.165);
\draw[gp path] (3.740,8.381)--(3.740,8.201);
\node[gp node center] at (3.740,0.677) {\num{0.2}};
\draw[gp path] (5.792,0.985)--(5.792,1.165);
\draw[gp path] (5.792,8.381)--(5.792,8.201);
\node[gp node center] at (5.792,0.677) {\num{0.4}};
\draw[gp path] (7.843,0.985)--(7.843,1.165);
\draw[gp path] (7.843,8.381)--(7.843,8.201);
\node[gp node center] at (7.843,0.677) {\num{0.6}};
\draw[gp path] (9.895,0.985)--(9.895,1.165);
\draw[gp path] (9.895,8.381)--(9.895,8.201);
\node[gp node center] at (9.895,0.677) {\num{0.8}};
\draw[gp path] (11.947,0.985)--(11.947,1.165);
\draw[gp path] (11.947,8.381)--(11.947,8.201);
\node[gp node center] at (11.947,0.677) {\num{1}};
\draw[gp path] (1.688,8.381)--(1.688,0.985)--(11.947,0.985)--(11.947,8.381)--cycle;
\node[gp node center,rotate=-270] at (0.246,4.683) {Количество методов, \%};
\node[gp node center] at (6.817,0.215) {Отношение среднего количества синонимов к числу переменных};
\node[gp node right] at (4.448,8.047) {base};
\gpfill{color=gp lt color 0,opacity=0.10} (4.632,7.970)--(5.548,7.970)--(5.548,8.124)--(4.632,8.124)--cycle;
\gpcolor{color=gp lt color 0}
\gpsetlinetype{gp lt plot 0}
\gpsetlinewidth{2.00}
\draw[gp path] (4.632,7.970)--(5.548,7.970)--(5.548,8.124)--(4.632,8.124)--cycle;
\gpfill{color=gp lt color 0,opacity=0.10} (1.688,1.033)--(1.688,1.033)--(1.698,1.032)--(1.709,1.031)%
    --(1.719,1.030)--(1.729,1.028)--(1.739,1.027)--(1.750,1.026)--(1.760,1.025)%
    --(1.770,1.024)--(1.780,1.023)--(1.791,1.022)--(1.801,1.021)--(1.811,1.020)%
    --(1.822,1.019)--(1.832,1.018)--(1.842,1.017)--(1.852,1.016)--(1.863,1.015)%
    --(1.873,1.014)--(1.883,1.013)--(1.893,1.012)--(1.904,1.012)--(1.914,1.011)%
    --(1.924,1.010)--(1.934,1.009)--(1.945,1.009)--(1.955,1.008)--(1.965,1.008)%
    --(1.976,1.007)--(1.986,1.007)--(1.996,1.007)--(2.006,1.007)--(2.017,1.006)%
    --(2.027,1.006)--(2.037,1.006)--(2.047,1.006)--(2.058,1.006)--(2.068,1.006)%
    --(2.078,1.007)--(2.089,1.007)--(2.099,1.007)--(2.109,1.008)--(2.119,1.008)%
    --(2.130,1.009)--(2.140,1.009)--(2.150,1.010)--(2.160,1.011)--(2.171,1.012)%
    --(2.181,1.013)--(2.191,1.014)--(2.201,1.015)--(2.212,1.016)--(2.222,1.017)%
    --(2.232,1.018)--(2.243,1.019)--(2.253,1.021)--(2.263,1.022)--(2.273,1.023)%
    --(2.284,1.025)--(2.294,1.026)--(2.304,1.028)--(2.314,1.029)--(2.325,1.031)%
    --(2.335,1.032)--(2.345,1.034)--(2.356,1.035)--(2.366,1.037)--(2.376,1.039)%
    --(2.386,1.040)--(2.397,1.042)--(2.407,1.044)--(2.417,1.045)--(2.427,1.047)%
    --(2.438,1.048)--(2.448,1.050)--(2.458,1.052)--(2.468,1.053)--(2.479,1.055)%
    --(2.489,1.057)--(2.499,1.058)--(2.510,1.060)--(2.520,1.061)--(2.530,1.063)%
    --(2.540,1.065)--(2.551,1.066)--(2.561,1.067)--(2.571,1.069)--(2.581,1.070)%
    --(2.592,1.072)--(2.602,1.073)--(2.612,1.074)--(2.623,1.075)--(2.633,1.077)%
    --(2.643,1.078)--(2.653,1.079)--(2.664,1.080)--(2.674,1.081)--(2.684,1.082)%
    --(2.694,1.083)--(2.705,1.083)--(2.715,1.084)--(2.725,1.085)--(2.735,1.085)%
    --(2.746,1.086)--(2.756,1.086)--(2.766,1.087)--(2.777,1.087)--(2.787,1.087)%
    --(2.797,1.087)--(2.807,1.087)--(2.818,1.087)--(2.828,1.087)--(2.838,1.087)%
    --(2.848,1.086)--(2.859,1.086)--(2.869,1.085)--(2.879,1.085)--(2.890,1.084)%
    --(2.900,1.083)--(2.910,1.082)--(2.920,1.081)--(2.931,1.080)--(2.941,1.078)%
    --(2.951,1.077)--(2.961,1.076)--(2.972,1.074)--(2.982,1.072)--(2.992,1.071)%
    --(3.002,1.069)--(3.013,1.067)--(3.023,1.066)--(3.033,1.064)--(3.044,1.062)%
    --(3.054,1.061)--(3.064,1.059)--(3.074,1.058)--(3.085,1.056)--(3.095,1.055)%
    --(3.105,1.054)--(3.115,1.053)--(3.126,1.052)--(3.136,1.051)--(3.146,1.050)%
    --(3.157,1.050)--(3.167,1.050)--(3.177,1.050)--(3.187,1.050)--(3.198,1.050)%
    --(3.208,1.051)--(3.218,1.052)--(3.228,1.053)--(3.239,1.054)--(3.249,1.056)%
    --(3.259,1.058)--(3.269,1.060)--(3.280,1.063)--(3.290,1.066)--(3.300,1.069)%
    --(3.311,1.073)--(3.321,1.077)--(3.331,1.082)--(3.341,1.087)--(3.352,1.092)%
    --(3.362,1.098)--(3.372,1.104)--(3.382,1.110)--(3.393,1.117)--(3.403,1.124)%
    --(3.413,1.131)--(3.424,1.139)--(3.434,1.147)--(3.444,1.155)--(3.454,1.163)%
    --(3.465,1.172)--(3.475,1.180)--(3.485,1.189)--(3.495,1.198)--(3.506,1.207)%
    --(3.516,1.217)--(3.526,1.226)--(3.536,1.235)--(3.547,1.245)--(3.557,1.254)%
    --(3.567,1.264)--(3.578,1.274)--(3.588,1.283)--(3.598,1.293)--(3.608,1.302)%
    --(3.619,1.312)--(3.629,1.321)--(3.639,1.331)--(3.649,1.340)--(3.660,1.349)%
    --(3.670,1.358)--(3.680,1.367)--(3.691,1.375)--(3.701,1.384)--(3.711,1.392)%
    --(3.721,1.400)--(3.732,1.408)--(3.742,1.415)--(3.752,1.423)--(3.762,1.430)%
    --(3.773,1.436)--(3.783,1.443)--(3.793,1.449)--(3.803,1.455)--(3.814,1.461)%
    --(3.824,1.466)--(3.834,1.471)--(3.845,1.476)--(3.855,1.480)--(3.865,1.485)%
    --(3.875,1.489)--(3.886,1.492)--(3.896,1.496)--(3.906,1.499)--(3.916,1.501)%
    --(3.927,1.504)--(3.937,1.506)--(3.947,1.508)--(3.958,1.510)--(3.968,1.511)%
    --(3.978,1.512)--(3.988,1.513)--(3.999,1.513)--(4.009,1.513)--(4.019,1.513)%
    --(4.029,1.513)--(4.040,1.512)--(4.050,1.511)--(4.060,1.510)--(4.070,1.508)%
    --(4.081,1.506)--(4.091,1.504)--(4.101,1.501)--(4.112,1.498)--(4.122,1.495)%
    --(4.132,1.491)--(4.142,1.487)--(4.153,1.483)--(4.163,1.479)--(4.173,1.474)%
    --(4.183,1.469)--(4.194,1.464)--(4.204,1.458)--(4.214,1.452)--(4.225,1.446)%
    --(4.235,1.440)--(4.245,1.434)--(4.255,1.428)--(4.266,1.421)--(4.276,1.415)%
    --(4.286,1.409)--(4.296,1.402)--(4.307,1.395)--(4.317,1.389)--(4.327,1.382)%
    --(4.337,1.376)--(4.348,1.370)--(4.358,1.363)--(4.368,1.357)--(4.379,1.351)%
    --(4.389,1.346)--(4.399,1.340)--(4.409,1.335)--(4.420,1.330)--(4.430,1.325)%
    --(4.440,1.320)--(4.450,1.316)--(4.461,1.312)--(4.471,1.309)--(4.481,1.305)%
    --(4.492,1.303)--(4.502,1.300)--(4.512,1.298)--(4.522,1.297)--(4.533,1.296)%
    --(4.543,1.295)--(4.553,1.295)--(4.563,1.296)--(4.574,1.297)--(4.584,1.299)%
    --(4.594,1.301)--(4.604,1.304)--(4.615,1.307)--(4.625,1.310)--(4.635,1.314)%
    --(4.646,1.319)--(4.656,1.323)--(4.666,1.328)--(4.676,1.334)--(4.687,1.339)%
    --(4.697,1.345)--(4.707,1.351)--(4.717,1.358)--(4.728,1.364)--(4.738,1.371)%
    --(4.748,1.378)--(4.759,1.385)--(4.769,1.392)--(4.779,1.399)--(4.789,1.406)%
    --(4.800,1.413)--(4.810,1.420)--(4.820,1.427)--(4.830,1.434)--(4.841,1.441)%
    --(4.851,1.448)--(4.861,1.455)--(4.871,1.461)--(4.882,1.467)--(4.892,1.473)%
    --(4.902,1.479)--(4.913,1.485)--(4.923,1.490)--(4.933,1.495)--(4.943,1.500)%
    --(4.954,1.504)--(4.964,1.508)--(4.974,1.511)--(4.984,1.514)--(4.995,1.517)%
    --(5.005,1.519)--(5.015,1.521)--(5.026,1.522)--(5.036,1.524)--(5.046,1.525)%
    --(5.056,1.525)--(5.067,1.526)--(5.077,1.526)--(5.087,1.526)--(5.097,1.525)%
    --(5.108,1.525)--(5.118,1.525)--(5.128,1.524)--(5.138,1.523)--(5.149,1.522)%
    --(5.159,1.521)--(5.169,1.521)--(5.180,1.520)--(5.190,1.519)--(5.200,1.518)%
    --(5.210,1.517)--(5.221,1.516)--(5.231,1.516)--(5.241,1.515)--(5.251,1.515)%
    --(5.262,1.515)--(5.272,1.515)--(5.282,1.515)--(5.293,1.515)--(5.303,1.516)%
    --(5.313,1.517)--(5.323,1.518)--(5.334,1.520)--(5.344,1.522)--(5.354,1.524)%
    --(5.364,1.527)--(5.375,1.530)--(5.385,1.534)--(5.395,1.538)--(5.405,1.542)%
    --(5.416,1.547)--(5.426,1.552)--(5.436,1.558)--(5.447,1.564)--(5.457,1.570)%
    --(5.467,1.577)--(5.477,1.584)--(5.488,1.592)--(5.498,1.599)--(5.508,1.607)%
    --(5.518,1.616)--(5.529,1.625)--(5.539,1.634)--(5.549,1.643)--(5.560,1.652)%
    --(5.570,1.662)--(5.580,1.672)--(5.590,1.683)--(5.601,1.693)--(5.611,1.704)%
    --(5.621,1.715)--(5.631,1.726)--(5.642,1.737)--(5.652,1.748)--(5.662,1.760)%
    --(5.672,1.771)--(5.683,1.783)--(5.693,1.795)--(5.703,1.807)--(5.714,1.819)%
    --(5.724,1.831)--(5.734,1.844)--(5.744,1.856)--(5.755,1.868)--(5.765,1.881)%
    --(5.775,1.893)--(5.785,1.906)--(5.796,1.918)--(5.806,1.931)--(5.816,1.943)%
    --(5.827,1.955)--(5.837,1.968)--(5.847,1.980)--(5.857,1.992)--(5.868,2.004)%
    --(5.878,2.016)--(5.888,2.028)--(5.898,2.039)--(5.909,2.051)--(5.919,2.062)%
    --(5.929,2.073)--(5.939,2.084)--(5.950,2.094)--(5.960,2.104)--(5.970,2.114)%
    --(5.981,2.124)--(5.991,2.133)--(6.001,2.142)--(6.011,2.151)--(6.022,2.159)%
    --(6.032,2.167)--(6.042,2.174)--(6.052,2.181)--(6.063,2.188)--(6.073,2.194)%
    --(6.083,2.199)--(6.094,2.204)--(6.104,2.209)--(6.114,2.213)--(6.124,2.217)%
    --(6.135,2.220)--(6.145,2.222)--(6.155,2.224)--(6.165,2.225)--(6.176,2.225)%
    --(6.186,2.225)--(6.196,2.225)--(6.206,2.223)--(6.217,2.221)--(6.227,2.218)%
    --(6.237,2.215)--(6.248,2.211)--(6.258,2.207)--(6.268,2.202)--(6.278,2.197)%
    --(6.289,2.191)--(6.299,2.185)--(6.309,2.179)--(6.319,2.173)--(6.330,2.167)%
    --(6.340,2.161)--(6.350,2.154)--(6.361,2.148)--(6.371,2.141)--(6.381,2.135)%
    --(6.391,2.129)--(6.402,2.123)--(6.412,2.117)--(6.422,2.112)--(6.432,2.107)%
    --(6.443,2.102)--(6.453,2.098)--(6.463,2.095)--(6.473,2.091)--(6.484,2.089)%
    --(6.494,2.087)--(6.504,2.086)--(6.515,2.085)--(6.525,2.086)--(6.535,2.087)%
    --(6.545,2.089)--(6.556,2.092)--(6.566,2.096)--(6.576,2.101)--(6.586,2.107)%
    --(6.597,2.114)--(6.607,2.123)--(6.617,2.132)--(6.628,2.143)--(6.638,2.155)%
    --(6.648,2.169)--(6.658,2.183)--(6.669,2.198)--(6.679,2.214)--(6.689,2.231)%
    --(6.699,2.249)--(6.710,2.268)--(6.720,2.287)--(6.730,2.307)--(6.740,2.328)%
    --(6.751,2.349)--(6.761,2.371)--(6.771,2.393)--(6.782,2.415)--(6.792,2.438)%
    --(6.802,2.461)--(6.812,2.484)--(6.823,2.508)--(6.833,2.531)--(6.843,2.555)%
    --(6.853,2.578)--(6.864,2.602)--(6.874,2.625)--(6.884,2.648)--(6.895,2.671)%
    --(6.905,2.693)--(6.915,2.716)--(6.925,2.737)--(6.936,2.759)--(6.946,2.779)%
    --(6.956,2.799)--(6.966,2.819)--(6.977,2.838)--(6.987,2.856)--(6.997,2.873)%
    --(7.007,2.889)--(7.018,2.905)--(7.028,2.919)--(7.038,2.932)--(7.049,2.945)%
    --(7.059,2.956)--(7.069,2.967)--(7.079,2.976)--(7.090,2.985)--(7.100,2.992)%
    --(7.110,2.999)--(7.120,3.005)--(7.131,3.010)--(7.141,3.014)--(7.151,3.018)%
    --(7.162,3.020)--(7.172,3.022)--(7.182,3.023)--(7.192,3.023)--(7.203,3.022)%
    --(7.213,3.021)--(7.223,3.019)--(7.233,3.016)--(7.244,3.012)--(7.254,3.008)%
    --(7.264,3.003)--(7.274,2.998)--(7.285,2.992)--(7.295,2.985)--(7.305,2.978)%
    --(7.316,2.970)--(7.326,2.961)--(7.336,2.952)--(7.346,2.942)--(7.357,2.932)%
    --(7.367,2.921)--(7.377,2.910)--(7.387,2.899)--(7.398,2.886)--(7.408,2.874)%
    --(7.418,2.861)--(7.429,2.847)--(7.439,2.834)--(7.449,2.819)--(7.459,2.805)%
    --(7.470,2.790)--(7.480,2.775)--(7.490,2.761)--(7.500,2.746)--(7.511,2.731)%
    --(7.521,2.716)--(7.531,2.702)--(7.541,2.687)--(7.552,2.674)--(7.562,2.660)%
    --(7.572,2.647)--(7.583,2.635)--(7.593,2.623)--(7.603,2.612)--(7.613,2.602)%
    --(7.624,2.592)--(7.634,2.584)--(7.644,2.576)--(7.654,2.569)--(7.665,2.564)%
    --(7.675,2.560)--(7.685,2.557)--(7.696,2.555)--(7.706,2.555)--(7.716,2.556)%
    --(7.726,2.559)--(7.737,2.564)--(7.747,2.570)--(7.757,2.578)--(7.767,2.587)%
    --(7.778,2.599)--(7.788,2.613)--(7.798,2.628)--(7.808,2.646)--(7.819,2.666)%
    --(7.829,2.688)--(7.839,2.713)--(7.850,2.739)--(7.860,2.769)--(7.870,2.800)%
    --(7.880,2.834)--(7.891,2.869)--(7.901,2.906)--(7.911,2.945)--(7.921,2.986)%
    --(7.932,3.028)--(7.942,3.071)--(7.952,3.115)--(7.963,3.160)--(7.973,3.207)%
    --(7.983,3.254)--(7.993,3.301)--(8.004,3.350)--(8.014,3.398)--(8.024,3.447)%
    --(8.034,3.496)--(8.045,3.544)--(8.055,3.593)--(8.065,3.641)--(8.075,3.689)%
    --(8.086,3.737)--(8.096,3.783)--(8.106,3.829)--(8.117,3.874)--(8.127,3.918)%
    --(8.137,3.960)--(8.147,4.001)--(8.158,4.041)--(8.168,4.079)--(8.178,4.116)%
    --(8.188,4.150)--(8.199,4.183)--(8.209,4.213)--(8.219,4.241)--(8.230,4.267)%
    --(8.240,4.290)--(8.250,4.310)--(8.260,4.328)--(8.271,4.343)--(8.281,4.355)%
    --(8.291,4.365)--(8.301,4.373)--(8.312,4.378)--(8.322,4.382)--(8.332,4.384)%
    --(8.342,4.385)--(8.353,4.384)--(8.363,4.381)--(8.373,4.378)--(8.384,4.374)%
    --(8.394,4.369)--(8.404,4.364)--(8.414,4.358)--(8.425,4.352)--(8.435,4.346)%
    --(8.445,4.340)--(8.455,4.334)--(8.466,4.329)--(8.476,4.325)--(8.486,4.321)%
    --(8.497,4.318)--(8.507,4.316)--(8.517,4.316)--(8.527,4.317)--(8.538,4.320)%
    --(8.548,4.324)--(8.558,4.331)--(8.568,4.340)--(8.579,4.350)--(8.589,4.364)%
    --(8.599,4.380)--(8.609,4.399)--(8.620,4.421)--(8.630,4.446)--(8.640,4.474)%
    --(8.651,4.506)--(8.661,4.541)--(8.671,4.581)--(8.681,4.624)--(8.692,4.670)%
    --(8.702,4.720)--(8.712,4.773)--(8.722,4.828)--(8.733,4.886)--(8.743,4.946)%
    --(8.753,5.009)--(8.764,5.073)--(8.774,5.138)--(8.784,5.205)--(8.794,5.274)%
    --(8.805,5.343)--(8.815,5.412)--(8.825,5.482)--(8.835,5.552)--(8.846,5.622)%
    --(8.856,5.692)--(8.866,5.761)--(8.876,5.829)--(8.887,5.896)--(8.897,5.962)%
    --(8.907,6.026)--(8.918,6.088)--(8.928,6.149)--(8.938,6.207)--(8.948,6.263)%
    --(8.959,6.316)--(8.969,6.366)--(8.979,6.413)--(8.989,6.456)--(9.000,6.496)%
    --(9.010,6.531)--(9.020,6.563)--(9.031,6.590)--(9.041,6.612)--(9.051,6.630)%
    --(9.061,6.642)--(9.072,6.649)--(9.082,6.651)--(9.092,6.647)--(9.102,6.638)%
    --(9.113,6.623)--(9.123,6.605)--(9.133,6.581)--(9.143,6.554)--(9.154,6.522)%
    --(9.164,6.487)--(9.174,6.448)--(9.185,6.406)--(9.195,6.362)--(9.205,6.314)%
    --(9.215,6.264)--(9.226,6.212)--(9.236,6.158)--(9.246,6.103)--(9.256,6.046)%
    --(9.267,5.988)--(9.277,5.929)--(9.287,5.870)--(9.298,5.810)--(9.308,5.750)%
    --(9.318,5.690)--(9.328,5.631)--(9.339,5.572)--(9.349,5.514)--(9.359,5.458)%
    --(9.369,5.403)--(9.380,5.349)--(9.390,5.298)--(9.400,5.249)--(9.410,5.202)%
    --(9.421,5.158)--(9.431,5.117)--(9.441,5.079)--(9.452,5.044)--(9.462,5.014)%
    --(9.472,4.987)--(9.482,4.965)--(9.493,4.947)--(9.503,4.934)--(9.513,4.925)%
    --(9.523,4.919)--(9.534,4.918)--(9.544,4.920)--(9.554,4.925)--(9.565,4.933)%
    --(9.575,4.944)--(9.585,4.957)--(9.595,4.973)--(9.606,4.991)--(9.616,5.010)%
    --(9.626,5.031)--(9.636,5.054)--(9.647,5.078)--(9.657,5.102)--(9.667,5.127)%
    --(9.677,5.153)--(9.688,5.179)--(9.698,5.204)--(9.708,5.230)--(9.719,5.254)%
    --(9.729,5.279)--(9.739,5.302)--(9.749,5.323)--(9.760,5.344)--(9.770,5.363)%
    --(9.780,5.379)--(9.790,5.394)--(9.801,5.406)--(9.811,5.416)--(9.821,5.422)%
    --(9.832,5.426)--(9.842,5.426)--(9.852,5.422)--(9.862,5.415)--(9.873,5.404)%
    --(9.883,5.388)--(9.893,5.368)--(9.903,5.343)--(9.914,5.314)--(9.924,5.280)%
    --(9.934,5.241)--(9.944,5.199)--(9.955,5.153)--(9.965,5.103)--(9.975,5.050)%
    --(9.986,4.994)--(9.996,4.934)--(10.006,4.872)--(10.016,4.807)--(10.027,4.740)%
    --(10.037,4.670)--(10.047,4.598)--(10.057,4.525)--(10.068,4.450)--(10.078,4.373)%
    --(10.088,4.295)--(10.099,4.216)--(10.109,4.137)--(10.119,4.056)--(10.129,3.976)%
    --(10.140,3.895)--(10.150,3.814)--(10.160,3.733)--(10.170,3.652)--(10.181,3.573)%
    --(10.191,3.493)--(10.201,3.415)--(10.211,3.338)--(10.222,3.263)--(10.232,3.189)%
    --(10.242,3.117)--(10.253,3.046)--(10.263,2.978)--(10.273,2.913)--(10.283,2.850)%
    --(10.294,2.789)--(10.304,2.732)--(10.314,2.678)--(10.324,2.627)--(10.335,2.579)%
    --(10.345,2.534)--(10.355,2.492)--(10.366,2.452)--(10.376,2.415)--(10.386,2.381)%
    --(10.396,2.350)--(10.407,2.320)--(10.417,2.293)--(10.427,2.268)--(10.437,2.246)%
    --(10.448,2.225)--(10.458,2.206)--(10.468,2.189)--(10.478,2.174)--(10.489,2.160)%
    --(10.499,2.148)--(10.509,2.137)--(10.520,2.128)--(10.530,2.120)--(10.540,2.113)%
    --(10.550,2.107)--(10.561,2.102)--(10.571,2.098)--(10.581,2.095)--(10.591,2.092)%
    --(10.602,2.090)--(10.612,2.089)--(10.622,2.088)--(10.633,2.087)--(10.643,2.086)%
    --(10.653,2.086)--(10.663,2.085)--(10.674,2.085)--(10.684,2.084)--(10.694,2.083)%
    --(10.704,2.082)--(10.715,2.080)--(10.725,2.078)--(10.735,2.075)--(10.745,2.071)%
    --(10.756,2.067)--(10.766,2.063)--(10.776,2.058)--(10.787,2.053)--(10.797,2.047)%
    --(10.807,2.041)--(10.817,2.035)--(10.828,2.029)--(10.838,2.022)--(10.848,2.015)%
    --(10.858,2.007)--(10.869,2.000)--(10.879,1.992)--(10.889,1.984)--(10.900,1.976)%
    --(10.910,1.968)--(10.920,1.960)--(10.930,1.951)--(10.941,1.943)--(10.951,1.935)%
    --(10.961,1.926)--(10.971,1.918)--(10.982,1.910)--(10.992,1.902)--(11.002,1.894)%
    --(11.012,1.886)--(11.023,1.879)--(11.033,1.871)--(11.043,1.864)--(11.054,1.857)%
    --(11.064,1.850)--(11.074,1.844)--(11.084,1.838)--(11.095,1.832)--(11.105,1.827)%
    --(11.115,1.822)--(11.125,1.817)--(11.136,1.813)--(11.146,1.810)--(11.156,1.807)%
    --(11.167,1.804)--(11.177,1.801)--(11.187,1.799)--(11.197,1.797)--(11.208,1.796)%
    --(11.218,1.795)--(11.228,1.793)--(11.238,1.793)--(11.249,1.792)--(11.259,1.792)%
    --(11.269,1.791)--(11.279,1.791)--(11.290,1.791)--(11.300,1.791)--(11.310,1.791)%
    --(11.321,1.790)--(11.331,1.790)--(11.341,1.790)--(11.351,1.790)--(11.362,1.790)%
    --(11.372,1.789)--(11.382,1.789)--(11.392,1.788)--(11.403,1.787)--(11.413,1.786)%
    --(11.423,1.785)--(11.434,1.783)--(11.444,1.781)--(11.454,1.779)--(11.464,1.777)%
    --(11.475,1.774)--(11.485,1.770)--(11.495,1.767)--(11.505,1.762)--(11.516,1.758)%
    --(11.526,1.753)--(11.536,1.747)--(11.546,1.741)--(11.557,1.734)--(11.567,1.727)%
    --(11.577,1.719)--(11.588,1.711)--(11.598,1.702)--(11.608,1.693)--(11.618,1.684)%
    --(11.629,1.674)--(11.639,1.663)--(11.649,1.652)--(11.659,1.641)--(11.670,1.629)%
    --(11.680,1.617)--(11.690,1.605)--(11.701,1.592)--(11.711,1.579)--(11.721,1.566)%
    --(11.731,1.552)--(11.742,1.539)--(11.752,1.524)--(11.762,1.510)--(11.772,1.495)%
    --(11.783,1.480)--(11.793,1.465)--(11.803,1.450)--(11.813,1.434)--(11.824,1.419)%
    --(11.834,1.403)--(11.844,1.387)--(11.855,1.371)--(11.865,1.354)--(11.875,1.338)%
    --(11.885,1.322)--(11.896,1.305)--(11.906,1.289)--(11.916,1.272)--(11.926,1.255)%
    --(11.937,1.238)--(11.947,1.222)--(11.947,0.985)--(1.688,0.985)--cycle;
\draw[gp path] (1.688,1.033)--(1.698,1.032)--(1.709,1.031)--(1.719,1.030)--(1.729,1.028)%
  --(1.739,1.027)--(1.750,1.026)--(1.760,1.025)--(1.770,1.024)--(1.780,1.023)--(1.791,1.022)%
  --(1.801,1.021)--(1.811,1.020)--(1.822,1.019)--(1.832,1.018)--(1.842,1.017)--(1.852,1.016)%
  --(1.863,1.015)--(1.873,1.014)--(1.883,1.013)--(1.893,1.012)--(1.904,1.012)--(1.914,1.011)%
  --(1.924,1.010)--(1.934,1.009)--(1.945,1.009)--(1.955,1.008)--(1.965,1.008)--(1.976,1.007)%
  --(1.986,1.007)--(1.996,1.007)--(2.006,1.007)--(2.017,1.006)--(2.027,1.006)--(2.037,1.006)%
  --(2.047,1.006)--(2.058,1.006)--(2.068,1.006)--(2.078,1.007)--(2.089,1.007)--(2.099,1.007)%
  --(2.109,1.008)--(2.119,1.008)--(2.130,1.009)--(2.140,1.009)--(2.150,1.010)--(2.160,1.011)%
  --(2.171,1.012)--(2.181,1.013)--(2.191,1.014)--(2.201,1.015)--(2.212,1.016)--(2.222,1.017)%
  --(2.232,1.018)--(2.243,1.019)--(2.253,1.021)--(2.263,1.022)--(2.273,1.023)--(2.284,1.025)%
  --(2.294,1.026)--(2.304,1.028)--(2.314,1.029)--(2.325,1.031)--(2.335,1.032)--(2.345,1.034)%
  --(2.356,1.035)--(2.366,1.037)--(2.376,1.039)--(2.386,1.040)--(2.397,1.042)--(2.407,1.044)%
  --(2.417,1.045)--(2.427,1.047)--(2.438,1.048)--(2.448,1.050)--(2.458,1.052)--(2.468,1.053)%
  --(2.479,1.055)--(2.489,1.057)--(2.499,1.058)--(2.510,1.060)--(2.520,1.061)--(2.530,1.063)%
  --(2.540,1.065)--(2.551,1.066)--(2.561,1.067)--(2.571,1.069)--(2.581,1.070)--(2.592,1.072)%
  --(2.602,1.073)--(2.612,1.074)--(2.623,1.075)--(2.633,1.077)--(2.643,1.078)--(2.653,1.079)%
  --(2.664,1.080)--(2.674,1.081)--(2.684,1.082)--(2.694,1.083)--(2.705,1.083)--(2.715,1.084)%
  --(2.725,1.085)--(2.735,1.085)--(2.746,1.086)--(2.756,1.086)--(2.766,1.087)--(2.777,1.087)%
  --(2.787,1.087)--(2.797,1.087)--(2.807,1.087)--(2.818,1.087)--(2.828,1.087)--(2.838,1.087)%
  --(2.848,1.086)--(2.859,1.086)--(2.869,1.085)--(2.879,1.085)--(2.890,1.084)--(2.900,1.083)%
  --(2.910,1.082)--(2.920,1.081)--(2.931,1.080)--(2.941,1.078)--(2.951,1.077)--(2.961,1.076)%
  --(2.972,1.074)--(2.982,1.072)--(2.992,1.071)--(3.002,1.069)--(3.013,1.067)--(3.023,1.066)%
  --(3.033,1.064)--(3.044,1.062)--(3.054,1.061)--(3.064,1.059)--(3.074,1.058)--(3.085,1.056)%
  --(3.095,1.055)--(3.105,1.054)--(3.115,1.053)--(3.126,1.052)--(3.136,1.051)--(3.146,1.050)%
  --(3.157,1.050)--(3.167,1.050)--(3.177,1.050)--(3.187,1.050)--(3.198,1.050)--(3.208,1.051)%
  --(3.218,1.052)--(3.228,1.053)--(3.239,1.054)--(3.249,1.056)--(3.259,1.058)--(3.269,1.060)%
  --(3.280,1.063)--(3.290,1.066)--(3.300,1.069)--(3.311,1.073)--(3.321,1.077)--(3.331,1.082)%
  --(3.341,1.087)--(3.352,1.092)--(3.362,1.098)--(3.372,1.104)--(3.382,1.110)--(3.393,1.117)%
  --(3.403,1.124)--(3.413,1.131)--(3.424,1.139)--(3.434,1.147)--(3.444,1.155)--(3.454,1.163)%
  --(3.465,1.172)--(3.475,1.180)--(3.485,1.189)--(3.495,1.198)--(3.506,1.207)--(3.516,1.217)%
  --(3.526,1.226)--(3.536,1.235)--(3.547,1.245)--(3.557,1.254)--(3.567,1.264)--(3.578,1.274)%
  --(3.588,1.283)--(3.598,1.293)--(3.608,1.302)--(3.619,1.312)--(3.629,1.321)--(3.639,1.331)%
  --(3.649,1.340)--(3.660,1.349)--(3.670,1.358)--(3.680,1.367)--(3.691,1.375)--(3.701,1.384)%
  --(3.711,1.392)--(3.721,1.400)--(3.732,1.408)--(3.742,1.415)--(3.752,1.423)--(3.762,1.430)%
  --(3.773,1.436)--(3.783,1.443)--(3.793,1.449)--(3.803,1.455)--(3.814,1.461)--(3.824,1.466)%
  --(3.834,1.471)--(3.845,1.476)--(3.855,1.480)--(3.865,1.485)--(3.875,1.489)--(3.886,1.492)%
  --(3.896,1.496)--(3.906,1.499)--(3.916,1.501)--(3.927,1.504)--(3.937,1.506)--(3.947,1.508)%
  --(3.958,1.510)--(3.968,1.511)--(3.978,1.512)--(3.988,1.513)--(3.999,1.513)--(4.009,1.513)%
  --(4.019,1.513)--(4.029,1.513)--(4.040,1.512)--(4.050,1.511)--(4.060,1.510)--(4.070,1.508)%
  --(4.081,1.506)--(4.091,1.504)--(4.101,1.501)--(4.112,1.498)--(4.122,1.495)--(4.132,1.491)%
  --(4.142,1.487)--(4.153,1.483)--(4.163,1.479)--(4.173,1.474)--(4.183,1.469)--(4.194,1.464)%
  --(4.204,1.458)--(4.214,1.452)--(4.225,1.446)--(4.235,1.440)--(4.245,1.434)--(4.255,1.428)%
  --(4.266,1.421)--(4.276,1.415)--(4.286,1.409)--(4.296,1.402)--(4.307,1.395)--(4.317,1.389)%
  --(4.327,1.382)--(4.337,1.376)--(4.348,1.370)--(4.358,1.363)--(4.368,1.357)--(4.379,1.351)%
  --(4.389,1.346)--(4.399,1.340)--(4.409,1.335)--(4.420,1.330)--(4.430,1.325)--(4.440,1.320)%
  --(4.450,1.316)--(4.461,1.312)--(4.471,1.309)--(4.481,1.305)--(4.492,1.303)--(4.502,1.300)%
  --(4.512,1.298)--(4.522,1.297)--(4.533,1.296)--(4.543,1.295)--(4.553,1.295)--(4.563,1.296)%
  --(4.574,1.297)--(4.584,1.299)--(4.594,1.301)--(4.604,1.304)--(4.615,1.307)--(4.625,1.310)%
  --(4.635,1.314)--(4.646,1.319)--(4.656,1.323)--(4.666,1.328)--(4.676,1.334)--(4.687,1.339)%
  --(4.697,1.345)--(4.707,1.351)--(4.717,1.358)--(4.728,1.364)--(4.738,1.371)--(4.748,1.378)%
  --(4.759,1.385)--(4.769,1.392)--(4.779,1.399)--(4.789,1.406)--(4.800,1.413)--(4.810,1.420)%
  --(4.820,1.427)--(4.830,1.434)--(4.841,1.441)--(4.851,1.448)--(4.861,1.455)--(4.871,1.461)%
  --(4.882,1.467)--(4.892,1.473)--(4.902,1.479)--(4.913,1.485)--(4.923,1.490)--(4.933,1.495)%
  --(4.943,1.500)--(4.954,1.504)--(4.964,1.508)--(4.974,1.511)--(4.984,1.514)--(4.995,1.517)%
  --(5.005,1.519)--(5.015,1.521)--(5.026,1.522)--(5.036,1.524)--(5.046,1.525)--(5.056,1.525)%
  --(5.067,1.526)--(5.077,1.526)--(5.087,1.526)--(5.097,1.525)--(5.108,1.525)--(5.118,1.525)%
  --(5.128,1.524)--(5.138,1.523)--(5.149,1.522)--(5.159,1.521)--(5.169,1.521)--(5.180,1.520)%
  --(5.190,1.519)--(5.200,1.518)--(5.210,1.517)--(5.221,1.516)--(5.231,1.516)--(5.241,1.515)%
  --(5.251,1.515)--(5.262,1.515)--(5.272,1.515)--(5.282,1.515)--(5.293,1.515)--(5.303,1.516)%
  --(5.313,1.517)--(5.323,1.518)--(5.334,1.520)--(5.344,1.522)--(5.354,1.524)--(5.364,1.527)%
  --(5.375,1.530)--(5.385,1.534)--(5.395,1.538)--(5.405,1.542)--(5.416,1.547)--(5.426,1.552)%
  --(5.436,1.558)--(5.447,1.564)--(5.457,1.570)--(5.467,1.577)--(5.477,1.584)--(5.488,1.592)%
  --(5.498,1.599)--(5.508,1.607)--(5.518,1.616)--(5.529,1.625)--(5.539,1.634)--(5.549,1.643)%
  --(5.560,1.652)--(5.570,1.662)--(5.580,1.672)--(5.590,1.683)--(5.601,1.693)--(5.611,1.704)%
  --(5.621,1.715)--(5.631,1.726)--(5.642,1.737)--(5.652,1.748)--(5.662,1.760)--(5.672,1.771)%
  --(5.683,1.783)--(5.693,1.795)--(5.703,1.807)--(5.714,1.819)--(5.724,1.831)--(5.734,1.844)%
  --(5.744,1.856)--(5.755,1.868)--(5.765,1.881)--(5.775,1.893)--(5.785,1.906)--(5.796,1.918)%
  --(5.806,1.931)--(5.816,1.943)--(5.827,1.955)--(5.837,1.968)--(5.847,1.980)--(5.857,1.992)%
  --(5.868,2.004)--(5.878,2.016)--(5.888,2.028)--(5.898,2.039)--(5.909,2.051)--(5.919,2.062)%
  --(5.929,2.073)--(5.939,2.084)--(5.950,2.094)--(5.960,2.104)--(5.970,2.114)--(5.981,2.124)%
  --(5.991,2.133)--(6.001,2.142)--(6.011,2.151)--(6.022,2.159)--(6.032,2.167)--(6.042,2.174)%
  --(6.052,2.181)--(6.063,2.188)--(6.073,2.194)--(6.083,2.199)--(6.094,2.204)--(6.104,2.209)%
  --(6.114,2.213)--(6.124,2.217)--(6.135,2.220)--(6.145,2.222)--(6.155,2.224)--(6.165,2.225)%
  --(6.176,2.225)--(6.186,2.225)--(6.196,2.225)--(6.206,2.223)--(6.217,2.221)--(6.227,2.218)%
  --(6.237,2.215)--(6.248,2.211)--(6.258,2.207)--(6.268,2.202)--(6.278,2.197)--(6.289,2.191)%
  --(6.299,2.185)--(6.309,2.179)--(6.319,2.173)--(6.330,2.167)--(6.340,2.161)--(6.350,2.154)%
  --(6.361,2.148)--(6.371,2.141)--(6.381,2.135)--(6.391,2.129)--(6.402,2.123)--(6.412,2.117)%
  --(6.422,2.112)--(6.432,2.107)--(6.443,2.102)--(6.453,2.098)--(6.463,2.095)--(6.473,2.091)%
  --(6.484,2.089)--(6.494,2.087)--(6.504,2.086)--(6.515,2.085)--(6.525,2.086)--(6.535,2.087)%
  --(6.545,2.089)--(6.556,2.092)--(6.566,2.096)--(6.576,2.101)--(6.586,2.107)--(6.597,2.114)%
  --(6.607,2.123)--(6.617,2.132)--(6.628,2.143)--(6.638,2.155)--(6.648,2.169)--(6.658,2.183)%
  --(6.669,2.198)--(6.679,2.214)--(6.689,2.231)--(6.699,2.249)--(6.710,2.268)--(6.720,2.287)%
  --(6.730,2.307)--(6.740,2.328)--(6.751,2.349)--(6.761,2.371)--(6.771,2.393)--(6.782,2.415)%
  --(6.792,2.438)--(6.802,2.461)--(6.812,2.484)--(6.823,2.508)--(6.833,2.531)--(6.843,2.555)%
  --(6.853,2.578)--(6.864,2.602)--(6.874,2.625)--(6.884,2.648)--(6.895,2.671)--(6.905,2.693)%
  --(6.915,2.716)--(6.925,2.737)--(6.936,2.759)--(6.946,2.779)--(6.956,2.799)--(6.966,2.819)%
  --(6.977,2.838)--(6.987,2.856)--(6.997,2.873)--(7.007,2.889)--(7.018,2.905)--(7.028,2.919)%
  --(7.038,2.932)--(7.049,2.945)--(7.059,2.956)--(7.069,2.967)--(7.079,2.976)--(7.090,2.985)%
  --(7.100,2.992)--(7.110,2.999)--(7.120,3.005)--(7.131,3.010)--(7.141,3.014)--(7.151,3.018)%
  --(7.162,3.020)--(7.172,3.022)--(7.182,3.023)--(7.192,3.023)--(7.203,3.022)--(7.213,3.021)%
  --(7.223,3.019)--(7.233,3.016)--(7.244,3.012)--(7.254,3.008)--(7.264,3.003)--(7.274,2.998)%
  --(7.285,2.992)--(7.295,2.985)--(7.305,2.978)--(7.316,2.970)--(7.326,2.961)--(7.336,2.952)%
  --(7.346,2.942)--(7.357,2.932)--(7.367,2.921)--(7.377,2.910)--(7.387,2.899)--(7.398,2.886)%
  --(7.408,2.874)--(7.418,2.861)--(7.429,2.847)--(7.439,2.834)--(7.449,2.819)--(7.459,2.805)%
  --(7.470,2.790)--(7.480,2.775)--(7.490,2.761)--(7.500,2.746)--(7.511,2.731)--(7.521,2.716)%
  --(7.531,2.702)--(7.541,2.687)--(7.552,2.674)--(7.562,2.660)--(7.572,2.647)--(7.583,2.635)%
  --(7.593,2.623)--(7.603,2.612)--(7.613,2.602)--(7.624,2.592)--(7.634,2.584)--(7.644,2.576)%
  --(7.654,2.569)--(7.665,2.564)--(7.675,2.560)--(7.685,2.557)--(7.696,2.555)--(7.706,2.555)%
  --(7.716,2.556)--(7.726,2.559)--(7.737,2.564)--(7.747,2.570)--(7.757,2.578)--(7.767,2.587)%
  --(7.778,2.599)--(7.788,2.613)--(7.798,2.628)--(7.808,2.646)--(7.819,2.666)--(7.829,2.688)%
  --(7.839,2.713)--(7.850,2.739)--(7.860,2.769)--(7.870,2.800)--(7.880,2.834)--(7.891,2.869)%
  --(7.901,2.906)--(7.911,2.945)--(7.921,2.986)--(7.932,3.028)--(7.942,3.071)--(7.952,3.115)%
  --(7.963,3.160)--(7.973,3.207)--(7.983,3.254)--(7.993,3.301)--(8.004,3.350)--(8.014,3.398)%
  --(8.024,3.447)--(8.034,3.496)--(8.045,3.544)--(8.055,3.593)--(8.065,3.641)--(8.075,3.689)%
  --(8.086,3.737)--(8.096,3.783)--(8.106,3.829)--(8.117,3.874)--(8.127,3.918)--(8.137,3.960)%
  --(8.147,4.001)--(8.158,4.041)--(8.168,4.079)--(8.178,4.116)--(8.188,4.150)--(8.199,4.183)%
  --(8.209,4.213)--(8.219,4.241)--(8.230,4.267)--(8.240,4.290)--(8.250,4.310)--(8.260,4.328)%
  --(8.271,4.343)--(8.281,4.355)--(8.291,4.365)--(8.301,4.373)--(8.312,4.378)--(8.322,4.382)%
  --(8.332,4.384)--(8.342,4.385)--(8.353,4.384)--(8.363,4.381)--(8.373,4.378)--(8.384,4.374)%
  --(8.394,4.369)--(8.404,4.364)--(8.414,4.358)--(8.425,4.352)--(8.435,4.346)--(8.445,4.340)%
  --(8.455,4.334)--(8.466,4.329)--(8.476,4.325)--(8.486,4.321)--(8.497,4.318)--(8.507,4.316)%
  --(8.517,4.316)--(8.527,4.317)--(8.538,4.320)--(8.548,4.324)--(8.558,4.331)--(8.568,4.340)%
  --(8.579,4.350)--(8.589,4.364)--(8.599,4.380)--(8.609,4.399)--(8.620,4.421)--(8.630,4.446)%
  --(8.640,4.474)--(8.651,4.506)--(8.661,4.541)--(8.671,4.581)--(8.681,4.624)--(8.692,4.670)%
  --(8.702,4.720)--(8.712,4.773)--(8.722,4.828)--(8.733,4.886)--(8.743,4.946)--(8.753,5.009)%
  --(8.764,5.073)--(8.774,5.138)--(8.784,5.205)--(8.794,5.274)--(8.805,5.343)--(8.815,5.412)%
  --(8.825,5.482)--(8.835,5.552)--(8.846,5.622)--(8.856,5.692)--(8.866,5.761)--(8.876,5.829)%
  --(8.887,5.896)--(8.897,5.962)--(8.907,6.026)--(8.918,6.088)--(8.928,6.149)--(8.938,6.207)%
  --(8.948,6.263)--(8.959,6.316)--(8.969,6.366)--(8.979,6.413)--(8.989,6.456)--(9.000,6.496)%
  --(9.010,6.531)--(9.020,6.563)--(9.031,6.590)--(9.041,6.612)--(9.051,6.630)--(9.061,6.642)%
  --(9.072,6.649)--(9.082,6.651)--(9.092,6.647)--(9.102,6.638)--(9.113,6.623)--(9.123,6.605)%
  --(9.133,6.581)--(9.143,6.554)--(9.154,6.522)--(9.164,6.487)--(9.174,6.448)--(9.185,6.406)%
  --(9.195,6.362)--(9.205,6.314)--(9.215,6.264)--(9.226,6.212)--(9.236,6.158)--(9.246,6.103)%
  --(9.256,6.046)--(9.267,5.988)--(9.277,5.929)--(9.287,5.870)--(9.298,5.810)--(9.308,5.750)%
  --(9.318,5.690)--(9.328,5.631)--(9.339,5.572)--(9.349,5.514)--(9.359,5.458)--(9.369,5.403)%
  --(9.380,5.349)--(9.390,5.298)--(9.400,5.249)--(9.410,5.202)--(9.421,5.158)--(9.431,5.117)%
  --(9.441,5.079)--(9.452,5.044)--(9.462,5.014)--(9.472,4.987)--(9.482,4.965)--(9.493,4.947)%
  --(9.503,4.934)--(9.513,4.925)--(9.523,4.919)--(9.534,4.918)--(9.544,4.920)--(9.554,4.925)%
  --(9.565,4.933)--(9.575,4.944)--(9.585,4.957)--(9.595,4.973)--(9.606,4.991)--(9.616,5.010)%
  --(9.626,5.031)--(9.636,5.054)--(9.647,5.078)--(9.657,5.102)--(9.667,5.127)--(9.677,5.153)%
  --(9.688,5.179)--(9.698,5.204)--(9.708,5.230)--(9.719,5.254)--(9.729,5.279)--(9.739,5.302)%
  --(9.749,5.323)--(9.760,5.344)--(9.770,5.363)--(9.780,5.379)--(9.790,5.394)--(9.801,5.406)%
  --(9.811,5.416)--(9.821,5.422)--(9.832,5.426)--(9.842,5.426)--(9.852,5.422)--(9.862,5.415)%
  --(9.873,5.404)--(9.883,5.388)--(9.893,5.368)--(9.903,5.343)--(9.914,5.314)--(9.924,5.280)%
  --(9.934,5.241)--(9.944,5.199)--(9.955,5.153)--(9.965,5.103)--(9.975,5.050)--(9.986,4.994)%
  --(9.996,4.934)--(10.006,4.872)--(10.016,4.807)--(10.027,4.740)--(10.037,4.670)--(10.047,4.598)%
  --(10.057,4.525)--(10.068,4.450)--(10.078,4.373)--(10.088,4.295)--(10.099,4.216)--(10.109,4.137)%
  --(10.119,4.056)--(10.129,3.976)--(10.140,3.895)--(10.150,3.814)--(10.160,3.733)--(10.170,3.652)%
  --(10.181,3.573)--(10.191,3.493)--(10.201,3.415)--(10.211,3.338)--(10.222,3.263)--(10.232,3.189)%
  --(10.242,3.117)--(10.253,3.046)--(10.263,2.978)--(10.273,2.913)--(10.283,2.850)--(10.294,2.789)%
  --(10.304,2.732)--(10.314,2.678)--(10.324,2.627)--(10.335,2.579)--(10.345,2.534)--(10.355,2.492)%
  --(10.366,2.452)--(10.376,2.415)--(10.386,2.381)--(10.396,2.350)--(10.407,2.320)--(10.417,2.293)%
  --(10.427,2.268)--(10.437,2.246)--(10.448,2.225)--(10.458,2.206)--(10.468,2.189)--(10.478,2.174)%
  --(10.489,2.160)--(10.499,2.148)--(10.509,2.137)--(10.520,2.128)--(10.530,2.120)--(10.540,2.113)%
  --(10.550,2.107)--(10.561,2.102)--(10.571,2.098)--(10.581,2.095)--(10.591,2.092)--(10.602,2.090)%
  --(10.612,2.089)--(10.622,2.088)--(10.633,2.087)--(10.643,2.086)--(10.653,2.086)--(10.663,2.085)%
  --(10.674,2.085)--(10.684,2.084)--(10.694,2.083)--(10.704,2.082)--(10.715,2.080)--(10.725,2.078)%
  --(10.735,2.075)--(10.745,2.071)--(10.756,2.067)--(10.766,2.063)--(10.776,2.058)--(10.787,2.053)%
  --(10.797,2.047)--(10.807,2.041)--(10.817,2.035)--(10.828,2.029)--(10.838,2.022)--(10.848,2.015)%
  --(10.858,2.007)--(10.869,2.000)--(10.879,1.992)--(10.889,1.984)--(10.900,1.976)--(10.910,1.968)%
  --(10.920,1.960)--(10.930,1.951)--(10.941,1.943)--(10.951,1.935)--(10.961,1.926)--(10.971,1.918)%
  --(10.982,1.910)--(10.992,1.902)--(11.002,1.894)--(11.012,1.886)--(11.023,1.879)--(11.033,1.871)%
  --(11.043,1.864)--(11.054,1.857)--(11.064,1.850)--(11.074,1.844)--(11.084,1.838)--(11.095,1.832)%
  --(11.105,1.827)--(11.115,1.822)--(11.125,1.817)--(11.136,1.813)--(11.146,1.810)--(11.156,1.807)%
  --(11.167,1.804)--(11.177,1.801)--(11.187,1.799)--(11.197,1.797)--(11.208,1.796)--(11.218,1.795)%
  --(11.228,1.793)--(11.238,1.793)--(11.249,1.792)--(11.259,1.792)--(11.269,1.791)--(11.279,1.791)%
  --(11.290,1.791)--(11.300,1.791)--(11.310,1.791)--(11.321,1.790)--(11.331,1.790)--(11.341,1.790)%
  --(11.351,1.790)--(11.362,1.790)--(11.372,1.789)--(11.382,1.789)--(11.392,1.788)--(11.403,1.787)%
  --(11.413,1.786)--(11.423,1.785)--(11.434,1.783)--(11.444,1.781)--(11.454,1.779)--(11.464,1.777)%
  --(11.475,1.774)--(11.485,1.770)--(11.495,1.767)--(11.505,1.762)--(11.516,1.758)--(11.526,1.753)%
  --(11.536,1.747)--(11.546,1.741)--(11.557,1.734)--(11.567,1.727)--(11.577,1.719)--(11.588,1.711)%
  --(11.598,1.702)--(11.608,1.693)--(11.618,1.684)--(11.629,1.674)--(11.639,1.663)--(11.649,1.652)%
  --(11.659,1.641)--(11.670,1.629)--(11.680,1.617)--(11.690,1.605)--(11.701,1.592)--(11.711,1.579)%
  --(11.721,1.566)--(11.731,1.552)--(11.742,1.539)--(11.752,1.524)--(11.762,1.510)--(11.772,1.495)%
  --(11.783,1.480)--(11.793,1.465)--(11.803,1.450)--(11.813,1.434)--(11.824,1.419)--(11.834,1.403)%
  --(11.844,1.387)--(11.855,1.371)--(11.865,1.354)--(11.875,1.338)--(11.885,1.322)--(11.896,1.305)%
  --(11.906,1.289)--(11.916,1.272)--(11.926,1.255)--(11.937,1.238)--(11.947,1.222);
\gpcolor{color=gp lt color border}
\node[gp node right] at (4.448,7.739) {equality-based};
\gpfill{color=gp lt color 2,opacity=0.10} (4.632,7.662)--(5.548,7.662)--(5.548,7.816)--(4.632,7.816)--cycle;
\gpcolor{color=gp lt color 2}
\draw[gp path] (4.632,7.662)--(5.548,7.662)--(5.548,7.816)--(4.632,7.816)--cycle;
\gpfill{color=gp lt color 2,opacity=0.10} (1.688,1.033)--(1.688,1.033)--(1.698,1.035)--(1.709,1.037)%
    --(1.719,1.039)--(1.729,1.041)--(1.739,1.043)--(1.750,1.045)--(1.760,1.047)%
    --(1.770,1.049)--(1.780,1.050)--(1.791,1.052)--(1.801,1.054)--(1.811,1.056)%
    --(1.822,1.058)--(1.832,1.060)--(1.842,1.062)--(1.852,1.064)--(1.863,1.066)%
    --(1.873,1.067)--(1.883,1.069)--(1.893,1.071)--(1.904,1.073)--(1.914,1.075)%
    --(1.924,1.076)--(1.934,1.078)--(1.945,1.080)--(1.955,1.082)--(1.965,1.083)%
    --(1.976,1.085)--(1.986,1.087)--(1.996,1.088)--(2.006,1.090)--(2.017,1.092)%
    --(2.027,1.093)--(2.037,1.095)--(2.047,1.096)--(2.058,1.098)--(2.068,1.099)%
    --(2.078,1.101)--(2.089,1.102)--(2.099,1.104)--(2.109,1.105)--(2.119,1.107)%
    --(2.130,1.108)--(2.140,1.109)--(2.150,1.111)--(2.160,1.112)--(2.171,1.113)%
    --(2.181,1.114)--(2.191,1.116)--(2.201,1.117)--(2.212,1.118)--(2.222,1.119)%
    --(2.232,1.120)--(2.243,1.121)--(2.253,1.122)--(2.263,1.123)--(2.273,1.124)%
    --(2.284,1.125)--(2.294,1.126)--(2.304,1.126)--(2.314,1.127)--(2.325,1.128)%
    --(2.335,1.129)--(2.345,1.129)--(2.356,1.130)--(2.366,1.130)--(2.376,1.131)%
    --(2.386,1.131)--(2.397,1.132)--(2.407,1.132)--(2.417,1.132)--(2.427,1.133)%
    --(2.438,1.133)--(2.448,1.133)--(2.458,1.133)--(2.468,1.134)--(2.479,1.134)%
    --(2.489,1.134)--(2.499,1.134)--(2.510,1.133)--(2.520,1.133)--(2.530,1.133)%
    --(2.540,1.133)--(2.551,1.133)--(2.561,1.132)--(2.571,1.132)--(2.581,1.131)%
    --(2.592,1.131)--(2.602,1.130)--(2.612,1.130)--(2.623,1.129)--(2.633,1.128)%
    --(2.643,1.127)--(2.653,1.126)--(2.664,1.126)--(2.674,1.125)--(2.684,1.124)%
    --(2.694,1.122)--(2.705,1.121)--(2.715,1.120)--(2.725,1.119)--(2.735,1.117)%
    --(2.746,1.116)--(2.756,1.114)--(2.766,1.113)--(2.777,1.111)--(2.787,1.109)%
    --(2.797,1.108)--(2.807,1.106)--(2.818,1.104)--(2.828,1.102)--(2.838,1.100)%
    --(2.848,1.098)--(2.859,1.096)--(2.869,1.093)--(2.879,1.091)--(2.890,1.089)%
    --(2.900,1.086)--(2.910,1.083)--(2.920,1.081)--(2.931,1.078)--(2.941,1.075)%
    --(2.951,1.072)--(2.961,1.070)--(2.972,1.067)--(2.982,1.064)--(2.992,1.061)%
    --(3.002,1.058)--(3.013,1.055)--(3.023,1.052)--(3.033,1.050)--(3.044,1.047)%
    --(3.054,1.044)--(3.064,1.042)--(3.074,1.040)--(3.085,1.037)--(3.095,1.035)%
    --(3.105,1.033)--(3.115,1.031)--(3.126,1.030)--(3.136,1.028)--(3.146,1.027)%
    --(3.157,1.026)--(3.167,1.026)--(3.177,1.025)--(3.187,1.025)--(3.198,1.025)%
    --(3.208,1.025)--(3.218,1.026)--(3.228,1.027)--(3.239,1.028)--(3.249,1.029)%
    --(3.259,1.031)--(3.269,1.034)--(3.280,1.036)--(3.290,1.040)--(3.300,1.043)%
    --(3.311,1.047)--(3.321,1.051)--(3.331,1.056)--(3.341,1.061)--(3.352,1.067)%
    --(3.362,1.073)--(3.372,1.080)--(3.382,1.086)--(3.393,1.094)--(3.403,1.101)%
    --(3.413,1.109)--(3.424,1.117)--(3.434,1.125)--(3.444,1.134)--(3.454,1.142)%
    --(3.465,1.151)--(3.475,1.160)--(3.485,1.169)--(3.495,1.178)--(3.506,1.188)%
    --(3.516,1.197)--(3.526,1.206)--(3.536,1.216)--(3.547,1.225)--(3.557,1.234)%
    --(3.567,1.244)--(3.578,1.253)--(3.588,1.262)--(3.598,1.271)--(3.608,1.279)%
    --(3.619,1.288)--(3.629,1.296)--(3.639,1.304)--(3.649,1.312)--(3.660,1.320)%
    --(3.670,1.327)--(3.680,1.334)--(3.691,1.340)--(3.701,1.346)--(3.711,1.352)%
    --(3.721,1.357)--(3.732,1.362)--(3.742,1.367)--(3.752,1.371)--(3.762,1.374)%
    --(3.773,1.377)--(3.783,1.380)--(3.793,1.382)--(3.803,1.383)--(3.814,1.385)%
    --(3.824,1.386)--(3.834,1.386)--(3.845,1.386)--(3.855,1.386)--(3.865,1.386)%
    --(3.875,1.385)--(3.886,1.384)--(3.896,1.383)--(3.906,1.382)--(3.916,1.380)%
    --(3.927,1.378)--(3.937,1.376)--(3.947,1.374)--(3.958,1.371)--(3.968,1.369)%
    --(3.978,1.366)--(3.988,1.363)--(3.999,1.360)--(4.009,1.357)--(4.019,1.354)%
    --(4.029,1.351)--(4.040,1.348)--(4.050,1.345)--(4.060,1.342)--(4.070,1.339)%
    --(4.081,1.336)--(4.091,1.333)--(4.101,1.330)--(4.112,1.327)--(4.122,1.325)%
    --(4.132,1.322)--(4.142,1.320)--(4.153,1.317)--(4.163,1.315)--(4.173,1.313)%
    --(4.183,1.311)--(4.194,1.310)--(4.204,1.308)--(4.214,1.307)--(4.225,1.306)%
    --(4.235,1.304)--(4.245,1.303)--(4.255,1.303)--(4.266,1.302)--(4.276,1.301)%
    --(4.286,1.300)--(4.296,1.300)--(4.307,1.299)--(4.317,1.299)--(4.327,1.299)%
    --(4.337,1.299)--(4.348,1.298)--(4.358,1.298)--(4.368,1.298)--(4.379,1.298)%
    --(4.389,1.298)--(4.399,1.298)--(4.409,1.298)--(4.420,1.298)--(4.430,1.298)%
    --(4.440,1.298)--(4.450,1.298)--(4.461,1.298)--(4.471,1.298)--(4.481,1.298)%
    --(4.492,1.298)--(4.502,1.298)--(4.512,1.297)--(4.522,1.297)--(4.533,1.297)%
    --(4.543,1.296)--(4.553,1.296)--(4.563,1.295)--(4.574,1.295)--(4.584,1.294)%
    --(4.594,1.293)--(4.604,1.293)--(4.615,1.292)--(4.625,1.291)--(4.635,1.290)%
    --(4.646,1.289)--(4.656,1.288)--(4.666,1.287)--(4.676,1.287)--(4.687,1.286)%
    --(4.697,1.285)--(4.707,1.284)--(4.717,1.284)--(4.728,1.283)--(4.738,1.283)%
    --(4.748,1.282)--(4.759,1.282)--(4.769,1.282)--(4.779,1.282)--(4.789,1.282)%
    --(4.800,1.282)--(4.810,1.282)--(4.820,1.283)--(4.830,1.284)--(4.841,1.285)%
    --(4.851,1.286)--(4.861,1.287)--(4.871,1.289)--(4.882,1.291)--(4.892,1.293)%
    --(4.902,1.295)--(4.913,1.298)--(4.923,1.300)--(4.933,1.304)--(4.943,1.307)%
    --(4.954,1.311)--(4.964,1.315)--(4.974,1.319)--(4.984,1.324)--(4.995,1.329)%
    --(5.005,1.334)--(5.015,1.340)--(5.026,1.346)--(5.036,1.352)--(5.046,1.358)%
    --(5.056,1.365)--(5.067,1.372)--(5.077,1.379)--(5.087,1.386)--(5.097,1.393)%
    --(5.108,1.401)--(5.118,1.408)--(5.128,1.416)--(5.138,1.424)--(5.149,1.432)%
    --(5.159,1.440)--(5.169,1.448)--(5.180,1.456)--(5.190,1.464)--(5.200,1.472)%
    --(5.210,1.481)--(5.221,1.489)--(5.231,1.497)--(5.241,1.505)--(5.251,1.513)%
    --(5.262,1.521)--(5.272,1.529)--(5.282,1.536)--(5.293,1.544)--(5.303,1.552)%
    --(5.313,1.559)--(5.323,1.566)--(5.334,1.573)--(5.344,1.580)--(5.354,1.586)%
    --(5.364,1.593)--(5.375,1.599)--(5.385,1.605)--(5.395,1.610)--(5.405,1.616)%
    --(5.416,1.621)--(5.426,1.625)--(5.436,1.630)--(5.447,1.634)--(5.457,1.639)%
    --(5.467,1.642)--(5.477,1.646)--(5.488,1.650)--(5.498,1.653)--(5.508,1.656)%
    --(5.518,1.659)--(5.529,1.662)--(5.539,1.665)--(5.549,1.667)--(5.560,1.669)%
    --(5.570,1.672)--(5.580,1.674)--(5.590,1.676)--(5.601,1.678)--(5.611,1.679)%
    --(5.621,1.681)--(5.631,1.682)--(5.642,1.684)--(5.652,1.685)--(5.662,1.686)%
    --(5.672,1.688)--(5.683,1.689)--(5.693,1.690)--(5.703,1.691)--(5.714,1.692)%
    --(5.724,1.693)--(5.734,1.694)--(5.744,1.695)--(5.755,1.695)--(5.765,1.696)%
    --(5.775,1.697)--(5.785,1.698)--(5.796,1.699)--(5.806,1.700)--(5.816,1.701)%
    --(5.827,1.702)--(5.837,1.703)--(5.847,1.704)--(5.857,1.706)--(5.868,1.707)%
    --(5.878,1.708)--(5.888,1.710)--(5.898,1.712)--(5.909,1.713)--(5.919,1.715)%
    --(5.929,1.717)--(5.939,1.720)--(5.950,1.722)--(5.960,1.725)--(5.970,1.728)%
    --(5.981,1.731)--(5.991,1.734)--(6.001,1.737)--(6.011,1.741)--(6.022,1.745)%
    --(6.032,1.749)--(6.042,1.754)--(6.052,1.758)--(6.063,1.763)--(6.073,1.769)%
    --(6.083,1.774)--(6.094,1.780)--(6.104,1.786)--(6.114,1.793)--(6.124,1.800)%
    --(6.135,1.807)--(6.145,1.815)--(6.155,1.823)--(6.165,1.832)--(6.176,1.840)%
    --(6.186,1.850)--(6.196,1.859)--(6.206,1.870)--(6.217,1.880)--(6.227,1.891)%
    --(6.237,1.903)--(6.248,1.914)--(6.258,1.926)--(6.268,1.939)--(6.278,1.951)%
    --(6.289,1.964)--(6.299,1.978)--(6.309,1.991)--(6.319,2.005)--(6.330,2.019)%
    --(6.340,2.033)--(6.350,2.047)--(6.361,2.061)--(6.371,2.076)--(6.381,2.090)%
    --(6.391,2.105)--(6.402,2.119)--(6.412,2.134)--(6.422,2.148)--(6.432,2.163)%
    --(6.443,2.177)--(6.453,2.192)--(6.463,2.206)--(6.473,2.220)--(6.484,2.235)%
    --(6.494,2.248)--(6.504,2.262)--(6.515,2.276)--(6.525,2.289)--(6.535,2.302)%
    --(6.545,2.315)--(6.556,2.327)--(6.566,2.340)--(6.576,2.351)--(6.586,2.363)%
    --(6.597,2.374)--(6.607,2.385)--(6.617,2.395)--(6.628,2.405)--(6.638,2.414)%
    --(6.648,2.423)--(6.658,2.432)--(6.669,2.440)--(6.679,2.448)--(6.689,2.455)%
    --(6.699,2.462)--(6.710,2.469)--(6.720,2.475)--(6.730,2.481)--(6.740,2.486)%
    --(6.751,2.492)--(6.761,2.496)--(6.771,2.501)--(6.782,2.505)--(6.792,2.508)%
    --(6.802,2.512)--(6.812,2.515)--(6.823,2.517)--(6.833,2.519)--(6.843,2.521)%
    --(6.853,2.523)--(6.864,2.524)--(6.874,2.525)--(6.884,2.526)--(6.895,2.527)%
    --(6.905,2.527)--(6.915,2.527)--(6.925,2.526)--(6.936,2.525)--(6.946,2.524)%
    --(6.956,2.523)--(6.966,2.522)--(6.977,2.520)--(6.987,2.518)--(6.997,2.515)%
    --(7.007,2.513)--(7.018,2.510)--(7.028,2.507)--(7.038,2.504)--(7.049,2.500)%
    --(7.059,2.497)--(7.069,2.493)--(7.079,2.489)--(7.090,2.485)--(7.100,2.480)%
    --(7.110,2.476)--(7.120,2.471)--(7.131,2.467)--(7.141,2.462)--(7.151,2.457)%
    --(7.162,2.453)--(7.172,2.448)--(7.182,2.443)--(7.192,2.438)--(7.203,2.434)%
    --(7.213,2.429)--(7.223,2.424)--(7.233,2.420)--(7.244,2.415)--(7.254,2.411)%
    --(7.264,2.406)--(7.274,2.402)--(7.285,2.398)--(7.295,2.394)--(7.305,2.390)%
    --(7.316,2.387)--(7.326,2.383)--(7.336,2.380)--(7.346,2.377)--(7.357,2.375)%
    --(7.367,2.372)--(7.377,2.370)--(7.387,2.368)--(7.398,2.367)--(7.408,2.366)%
    --(7.418,2.365)--(7.429,2.364)--(7.439,2.364)--(7.449,2.365)--(7.459,2.365)%
    --(7.470,2.366)--(7.480,2.368)--(7.490,2.369)--(7.500,2.371)--(7.511,2.374)%
    --(7.521,2.376)--(7.531,2.379)--(7.541,2.382)--(7.552,2.386)--(7.562,2.390)%
    --(7.572,2.394)--(7.583,2.398)--(7.593,2.403)--(7.603,2.408)--(7.613,2.413)%
    --(7.624,2.419)--(7.634,2.424)--(7.644,2.430)--(7.654,2.436)--(7.665,2.443)%
    --(7.675,2.449)--(7.685,2.456)--(7.696,2.463)--(7.706,2.470)--(7.716,2.477)%
    --(7.726,2.485)--(7.737,2.492)--(7.747,2.500)--(7.757,2.508)--(7.767,2.516)%
    --(7.778,2.524)--(7.788,2.532)--(7.798,2.541)--(7.808,2.549)--(7.819,2.558)%
    --(7.829,2.567)--(7.839,2.575)--(7.850,2.584)--(7.860,2.593)--(7.870,2.602)%
    --(7.880,2.611)--(7.891,2.620)--(7.901,2.630)--(7.911,2.639)--(7.921,2.649)%
    --(7.932,2.659)--(7.942,2.669)--(7.952,2.680)--(7.963,2.691)--(7.973,2.702)%
    --(7.983,2.713)--(7.993,2.724)--(8.004,2.736)--(8.014,2.748)--(8.024,2.761)%
    --(8.034,2.774)--(8.045,2.787)--(8.055,2.801)--(8.065,2.815)--(8.075,2.830)%
    --(8.086,2.845)--(8.096,2.861)--(8.106,2.877)--(8.117,2.894)--(8.127,2.911)%
    --(8.137,2.929)--(8.147,2.947)--(8.158,2.966)--(8.168,2.985)--(8.178,3.005)%
    --(8.188,3.026)--(8.199,3.047)--(8.209,3.070)--(8.219,3.092)--(8.230,3.116)%
    --(8.240,3.140)--(8.250,3.165)--(8.260,3.191)--(8.271,3.217)--(8.281,3.244)%
    --(8.291,3.272)--(8.301,3.301)--(8.312,3.330)--(8.322,3.360)--(8.332,3.391)%
    --(8.342,3.422)--(8.353,3.453)--(8.363,3.486)--(8.373,3.518)--(8.384,3.551)%
    --(8.394,3.585)--(8.404,3.619)--(8.414,3.654)--(8.425,3.688)--(8.435,3.724)%
    --(8.445,3.759)--(8.455,3.795)--(8.466,3.831)--(8.476,3.868)--(8.486,3.905)%
    --(8.497,3.941)--(8.507,3.979)--(8.517,4.016)--(8.527,4.053)--(8.538,4.091)%
    --(8.548,4.128)--(8.558,4.166)--(8.568,4.204)--(8.579,4.242)--(8.589,4.279)%
    --(8.599,4.317)--(8.609,4.355)--(8.620,4.392)--(8.630,4.430)--(8.640,4.467)%
    --(8.651,4.505)--(8.661,4.542)--(8.671,4.579)--(8.681,4.615)--(8.692,4.652)%
    --(8.702,4.688)--(8.712,4.724)--(8.722,4.760)--(8.733,4.796)--(8.743,4.831)%
    --(8.753,4.866)--(8.764,4.901)--(8.774,4.936)--(8.784,4.971)--(8.794,5.005)%
    --(8.805,5.039)--(8.815,5.073)--(8.825,5.107)--(8.835,5.141)--(8.846,5.174)%
    --(8.856,5.207)--(8.866,5.240)--(8.876,5.273)--(8.887,5.306)--(8.897,5.338)%
    --(8.907,5.370)--(8.918,5.402)--(8.928,5.434)--(8.938,5.465)--(8.948,5.496)%
    --(8.959,5.527)--(8.969,5.558)--(8.979,5.589)--(8.989,5.619)--(9.000,5.649)%
    --(9.010,5.679)--(9.020,5.709)--(9.031,5.739)--(9.041,5.768)--(9.051,5.797)%
    --(9.061,5.826)--(9.072,5.855)--(9.082,5.883)--(9.092,5.911)--(9.102,5.939)%
    --(9.113,5.967)--(9.123,5.995)--(9.133,6.022)--(9.143,6.049)--(9.154,6.076)%
    --(9.164,6.103)--(9.174,6.129)--(9.185,6.156)--(9.195,6.182)--(9.205,6.207)%
    --(9.215,6.233)--(9.226,6.258)--(9.236,6.283)--(9.246,6.308)--(9.256,6.333)%
    --(9.267,6.357)--(9.277,6.381)--(9.287,6.405)--(9.298,6.429)--(9.308,6.452)%
    --(9.318,6.475)--(9.328,6.498)--(9.339,6.521)--(9.349,6.544)--(9.359,6.566)%
    --(9.369,6.588)--(9.380,6.609)--(9.390,6.631)--(9.400,6.652)--(9.410,6.673)%
    --(9.421,6.694)--(9.431,6.714)--(9.441,6.734)--(9.452,6.754)--(9.462,6.774)%
    --(9.472,6.793)--(9.482,6.812)--(9.493,6.831)--(9.503,6.850)--(9.513,6.868)%
    --(9.523,6.885)--(9.534,6.902)--(9.544,6.919)--(9.554,6.934)--(9.565,6.949)%
    --(9.575,6.962)--(9.585,6.975)--(9.595,6.987)--(9.606,6.997)--(9.616,7.006)%
    --(9.626,7.013)--(9.636,7.019)--(9.647,7.024)--(9.657,7.027)--(9.667,7.028)%
    --(9.677,7.027)--(9.688,7.024)--(9.698,7.019)--(9.708,7.012)--(9.719,7.003)%
    --(9.729,6.992)--(9.739,6.978)--(9.749,6.962)--(9.760,6.943)--(9.770,6.921)%
    --(9.780,6.897)--(9.790,6.870)--(9.801,6.840)--(9.811,6.807)--(9.821,6.771)%
    --(9.832,6.731)--(9.842,6.688)--(9.852,6.642)--(9.862,6.593)--(9.873,6.540)%
    --(9.883,6.483)--(9.893,6.423)--(9.903,6.358)--(9.914,6.290)--(9.924,6.219)%
    --(9.934,6.144)--(9.944,6.066)--(9.955,5.986)--(9.965,5.902)--(9.975,5.816)%
    --(9.986,5.728)--(9.996,5.638)--(10.006,5.546)--(10.016,5.452)--(10.027,5.357)%
    --(10.037,5.261)--(10.047,5.163)--(10.057,5.065)--(10.068,4.966)--(10.078,4.866)%
    --(10.088,4.767)--(10.099,4.667)--(10.109,4.567)--(10.119,4.468)--(10.129,4.369)%
    --(10.140,4.271)--(10.150,4.174)--(10.160,4.078)--(10.170,3.983)--(10.181,3.890)%
    --(10.191,3.799)--(10.201,3.710)--(10.211,3.623)--(10.222,3.538)--(10.232,3.456)%
    --(10.242,3.376)--(10.253,3.300)--(10.263,3.226)--(10.273,3.156)--(10.283,3.090)%
    --(10.294,3.027)--(10.304,2.969)--(10.314,2.914)--(10.324,2.864)--(10.335,2.817)%
    --(10.345,2.775)--(10.355,2.736)--(10.366,2.700)--(10.376,2.668)--(10.386,2.639)%
    --(10.396,2.613)--(10.407,2.590)--(10.417,2.569)--(10.427,2.552)--(10.437,2.537)%
    --(10.448,2.524)--(10.458,2.513)--(10.468,2.505)--(10.478,2.498)--(10.489,2.493)%
    --(10.499,2.490)--(10.509,2.488)--(10.520,2.487)--(10.530,2.488)--(10.540,2.489)%
    --(10.550,2.492)--(10.561,2.495)--(10.571,2.499)--(10.581,2.503)--(10.591,2.508)%
    --(10.602,2.512)--(10.612,2.517)--(10.622,2.521)--(10.633,2.526)--(10.643,2.529)%
    --(10.653,2.533)--(10.663,2.535)--(10.674,2.537)--(10.684,2.537)--(10.694,2.537)%
    --(10.704,2.535)--(10.715,2.531)--(10.725,2.526)--(10.735,2.520)--(10.745,2.512)%
    --(10.756,2.502)--(10.766,2.492)--(10.776,2.480)--(10.787,2.466)--(10.797,2.452)%
    --(10.807,2.436)--(10.817,2.420)--(10.828,2.403)--(10.838,2.384)--(10.848,2.365)%
    --(10.858,2.346)--(10.869,2.325)--(10.879,2.304)--(10.889,2.283)--(10.900,2.261)%
    --(10.910,2.239)--(10.920,2.217)--(10.930,2.194)--(10.941,2.171)--(10.951,2.148)%
    --(10.961,2.125)--(10.971,2.103)--(10.982,2.080)--(10.992,2.058)--(11.002,2.035)%
    --(11.012,2.014)--(11.023,1.992)--(11.033,1.971)--(11.043,1.951)--(11.054,1.931)%
    --(11.064,1.913)--(11.074,1.894)--(11.084,1.877)--(11.095,1.861)--(11.105,1.845)%
    --(11.115,1.831)--(11.125,1.818)--(11.136,1.806)--(11.146,1.795)--(11.156,1.786)%
    --(11.167,1.777)--(11.177,1.770)--(11.187,1.763)--(11.197,1.757)--(11.208,1.753)%
    --(11.218,1.749)--(11.228,1.746)--(11.238,1.743)--(11.249,1.741)--(11.259,1.740)%
    --(11.269,1.740)--(11.279,1.740)--(11.290,1.740)--(11.300,1.741)--(11.310,1.742)%
    --(11.321,1.744)--(11.331,1.746)--(11.341,1.748)--(11.351,1.751)--(11.362,1.753)%
    --(11.372,1.756)--(11.382,1.758)--(11.392,1.761)--(11.403,1.763)--(11.413,1.766)%
    --(11.423,1.768)--(11.434,1.770)--(11.444,1.772)--(11.454,1.774)--(11.464,1.775)%
    --(11.475,1.776)--(11.485,1.776)--(11.495,1.776)--(11.505,1.775)--(11.516,1.774)%
    --(11.526,1.772)--(11.536,1.769)--(11.546,1.766)--(11.557,1.762)--(11.567,1.757)%
    --(11.577,1.751)--(11.588,1.745)--(11.598,1.738)--(11.608,1.731)--(11.618,1.722)%
    --(11.629,1.714)--(11.639,1.704)--(11.649,1.694)--(11.659,1.684)--(11.670,1.673)%
    --(11.680,1.662)--(11.690,1.650)--(11.701,1.637)--(11.711,1.624)--(11.721,1.611)%
    --(11.731,1.597)--(11.742,1.583)--(11.752,1.569)--(11.762,1.554)--(11.772,1.539)%
    --(11.783,1.523)--(11.793,1.508)--(11.803,1.492)--(11.813,1.475)--(11.824,1.459)%
    --(11.834,1.442)--(11.844,1.425)--(11.855,1.408)--(11.865,1.390)--(11.875,1.373)%
    --(11.885,1.355)--(11.896,1.337)--(11.906,1.320)--(11.916,1.302)--(11.926,1.284)%
    --(11.937,1.266)--(11.947,1.248)--(11.947,0.985)--(1.688,0.985)--cycle;
\draw[gp path] (1.688,1.033)--(1.698,1.035)--(1.709,1.037)--(1.719,1.039)--(1.729,1.041)%
  --(1.739,1.043)--(1.750,1.045)--(1.760,1.047)--(1.770,1.049)--(1.780,1.050)--(1.791,1.052)%
  --(1.801,1.054)--(1.811,1.056)--(1.822,1.058)--(1.832,1.060)--(1.842,1.062)--(1.852,1.064)%
  --(1.863,1.066)--(1.873,1.067)--(1.883,1.069)--(1.893,1.071)--(1.904,1.073)--(1.914,1.075)%
  --(1.924,1.076)--(1.934,1.078)--(1.945,1.080)--(1.955,1.082)--(1.965,1.083)--(1.976,1.085)%
  --(1.986,1.087)--(1.996,1.088)--(2.006,1.090)--(2.017,1.092)--(2.027,1.093)--(2.037,1.095)%
  --(2.047,1.096)--(2.058,1.098)--(2.068,1.099)--(2.078,1.101)--(2.089,1.102)--(2.099,1.104)%
  --(2.109,1.105)--(2.119,1.107)--(2.130,1.108)--(2.140,1.109)--(2.150,1.111)--(2.160,1.112)%
  --(2.171,1.113)--(2.181,1.114)--(2.191,1.116)--(2.201,1.117)--(2.212,1.118)--(2.222,1.119)%
  --(2.232,1.120)--(2.243,1.121)--(2.253,1.122)--(2.263,1.123)--(2.273,1.124)--(2.284,1.125)%
  --(2.294,1.126)--(2.304,1.126)--(2.314,1.127)--(2.325,1.128)--(2.335,1.129)--(2.345,1.129)%
  --(2.356,1.130)--(2.366,1.130)--(2.376,1.131)--(2.386,1.131)--(2.397,1.132)--(2.407,1.132)%
  --(2.417,1.132)--(2.427,1.133)--(2.438,1.133)--(2.448,1.133)--(2.458,1.133)--(2.468,1.134)%
  --(2.479,1.134)--(2.489,1.134)--(2.499,1.134)--(2.510,1.133)--(2.520,1.133)--(2.530,1.133)%
  --(2.540,1.133)--(2.551,1.133)--(2.561,1.132)--(2.571,1.132)--(2.581,1.131)--(2.592,1.131)%
  --(2.602,1.130)--(2.612,1.130)--(2.623,1.129)--(2.633,1.128)--(2.643,1.127)--(2.653,1.126)%
  --(2.664,1.126)--(2.674,1.125)--(2.684,1.124)--(2.694,1.122)--(2.705,1.121)--(2.715,1.120)%
  --(2.725,1.119)--(2.735,1.117)--(2.746,1.116)--(2.756,1.114)--(2.766,1.113)--(2.777,1.111)%
  --(2.787,1.109)--(2.797,1.108)--(2.807,1.106)--(2.818,1.104)--(2.828,1.102)--(2.838,1.100)%
  --(2.848,1.098)--(2.859,1.096)--(2.869,1.093)--(2.879,1.091)--(2.890,1.089)--(2.900,1.086)%
  --(2.910,1.083)--(2.920,1.081)--(2.931,1.078)--(2.941,1.075)--(2.951,1.072)--(2.961,1.070)%
  --(2.972,1.067)--(2.982,1.064)--(2.992,1.061)--(3.002,1.058)--(3.013,1.055)--(3.023,1.052)%
  --(3.033,1.050)--(3.044,1.047)--(3.054,1.044)--(3.064,1.042)--(3.074,1.040)--(3.085,1.037)%
  --(3.095,1.035)--(3.105,1.033)--(3.115,1.031)--(3.126,1.030)--(3.136,1.028)--(3.146,1.027)%
  --(3.157,1.026)--(3.167,1.026)--(3.177,1.025)--(3.187,1.025)--(3.198,1.025)--(3.208,1.025)%
  --(3.218,1.026)--(3.228,1.027)--(3.239,1.028)--(3.249,1.029)--(3.259,1.031)--(3.269,1.034)%
  --(3.280,1.036)--(3.290,1.040)--(3.300,1.043)--(3.311,1.047)--(3.321,1.051)--(3.331,1.056)%
  --(3.341,1.061)--(3.352,1.067)--(3.362,1.073)--(3.372,1.080)--(3.382,1.086)--(3.393,1.094)%
  --(3.403,1.101)--(3.413,1.109)--(3.424,1.117)--(3.434,1.125)--(3.444,1.134)--(3.454,1.142)%
  --(3.465,1.151)--(3.475,1.160)--(3.485,1.169)--(3.495,1.178)--(3.506,1.188)--(3.516,1.197)%
  --(3.526,1.206)--(3.536,1.216)--(3.547,1.225)--(3.557,1.234)--(3.567,1.244)--(3.578,1.253)%
  --(3.588,1.262)--(3.598,1.271)--(3.608,1.279)--(3.619,1.288)--(3.629,1.296)--(3.639,1.304)%
  --(3.649,1.312)--(3.660,1.320)--(3.670,1.327)--(3.680,1.334)--(3.691,1.340)--(3.701,1.346)%
  --(3.711,1.352)--(3.721,1.357)--(3.732,1.362)--(3.742,1.367)--(3.752,1.371)--(3.762,1.374)%
  --(3.773,1.377)--(3.783,1.380)--(3.793,1.382)--(3.803,1.383)--(3.814,1.385)--(3.824,1.386)%
  --(3.834,1.386)--(3.845,1.386)--(3.855,1.386)--(3.865,1.386)--(3.875,1.385)--(3.886,1.384)%
  --(3.896,1.383)--(3.906,1.382)--(3.916,1.380)--(3.927,1.378)--(3.937,1.376)--(3.947,1.374)%
  --(3.958,1.371)--(3.968,1.369)--(3.978,1.366)--(3.988,1.363)--(3.999,1.360)--(4.009,1.357)%
  --(4.019,1.354)--(4.029,1.351)--(4.040,1.348)--(4.050,1.345)--(4.060,1.342)--(4.070,1.339)%
  --(4.081,1.336)--(4.091,1.333)--(4.101,1.330)--(4.112,1.327)--(4.122,1.325)--(4.132,1.322)%
  --(4.142,1.320)--(4.153,1.317)--(4.163,1.315)--(4.173,1.313)--(4.183,1.311)--(4.194,1.310)%
  --(4.204,1.308)--(4.214,1.307)--(4.225,1.306)--(4.235,1.304)--(4.245,1.303)--(4.255,1.303)%
  --(4.266,1.302)--(4.276,1.301)--(4.286,1.300)--(4.296,1.300)--(4.307,1.299)--(4.317,1.299)%
  --(4.327,1.299)--(4.337,1.299)--(4.348,1.298)--(4.358,1.298)--(4.368,1.298)--(4.379,1.298)%
  --(4.389,1.298)--(4.399,1.298)--(4.409,1.298)--(4.420,1.298)--(4.430,1.298)--(4.440,1.298)%
  --(4.450,1.298)--(4.461,1.298)--(4.471,1.298)--(4.481,1.298)--(4.492,1.298)--(4.502,1.298)%
  --(4.512,1.297)--(4.522,1.297)--(4.533,1.297)--(4.543,1.296)--(4.553,1.296)--(4.563,1.295)%
  --(4.574,1.295)--(4.584,1.294)--(4.594,1.293)--(4.604,1.293)--(4.615,1.292)--(4.625,1.291)%
  --(4.635,1.290)--(4.646,1.289)--(4.656,1.288)--(4.666,1.287)--(4.676,1.287)--(4.687,1.286)%
  --(4.697,1.285)--(4.707,1.284)--(4.717,1.284)--(4.728,1.283)--(4.738,1.283)--(4.748,1.282)%
  --(4.759,1.282)--(4.769,1.282)--(4.779,1.282)--(4.789,1.282)--(4.800,1.282)--(4.810,1.282)%
  --(4.820,1.283)--(4.830,1.284)--(4.841,1.285)--(4.851,1.286)--(4.861,1.287)--(4.871,1.289)%
  --(4.882,1.291)--(4.892,1.293)--(4.902,1.295)--(4.913,1.298)--(4.923,1.300)--(4.933,1.304)%
  --(4.943,1.307)--(4.954,1.311)--(4.964,1.315)--(4.974,1.319)--(4.984,1.324)--(4.995,1.329)%
  --(5.005,1.334)--(5.015,1.340)--(5.026,1.346)--(5.036,1.352)--(5.046,1.358)--(5.056,1.365)%
  --(5.067,1.372)--(5.077,1.379)--(5.087,1.386)--(5.097,1.393)--(5.108,1.401)--(5.118,1.408)%
  --(5.128,1.416)--(5.138,1.424)--(5.149,1.432)--(5.159,1.440)--(5.169,1.448)--(5.180,1.456)%
  --(5.190,1.464)--(5.200,1.472)--(5.210,1.481)--(5.221,1.489)--(5.231,1.497)--(5.241,1.505)%
  --(5.251,1.513)--(5.262,1.521)--(5.272,1.529)--(5.282,1.536)--(5.293,1.544)--(5.303,1.552)%
  --(5.313,1.559)--(5.323,1.566)--(5.334,1.573)--(5.344,1.580)--(5.354,1.586)--(5.364,1.593)%
  --(5.375,1.599)--(5.385,1.605)--(5.395,1.610)--(5.405,1.616)--(5.416,1.621)--(5.426,1.625)%
  --(5.436,1.630)--(5.447,1.634)--(5.457,1.639)--(5.467,1.642)--(5.477,1.646)--(5.488,1.650)%
  --(5.498,1.653)--(5.508,1.656)--(5.518,1.659)--(5.529,1.662)--(5.539,1.665)--(5.549,1.667)%
  --(5.560,1.669)--(5.570,1.672)--(5.580,1.674)--(5.590,1.676)--(5.601,1.678)--(5.611,1.679)%
  --(5.621,1.681)--(5.631,1.682)--(5.642,1.684)--(5.652,1.685)--(5.662,1.686)--(5.672,1.688)%
  --(5.683,1.689)--(5.693,1.690)--(5.703,1.691)--(5.714,1.692)--(5.724,1.693)--(5.734,1.694)%
  --(5.744,1.695)--(5.755,1.695)--(5.765,1.696)--(5.775,1.697)--(5.785,1.698)--(5.796,1.699)%
  --(5.806,1.700)--(5.816,1.701)--(5.827,1.702)--(5.837,1.703)--(5.847,1.704)--(5.857,1.706)%
  --(5.868,1.707)--(5.878,1.708)--(5.888,1.710)--(5.898,1.712)--(5.909,1.713)--(5.919,1.715)%
  --(5.929,1.717)--(5.939,1.720)--(5.950,1.722)--(5.960,1.725)--(5.970,1.728)--(5.981,1.731)%
  --(5.991,1.734)--(6.001,1.737)--(6.011,1.741)--(6.022,1.745)--(6.032,1.749)--(6.042,1.754)%
  --(6.052,1.758)--(6.063,1.763)--(6.073,1.769)--(6.083,1.774)--(6.094,1.780)--(6.104,1.786)%
  --(6.114,1.793)--(6.124,1.800)--(6.135,1.807)--(6.145,1.815)--(6.155,1.823)--(6.165,1.832)%
  --(6.176,1.840)--(6.186,1.850)--(6.196,1.859)--(6.206,1.870)--(6.217,1.880)--(6.227,1.891)%
  --(6.237,1.903)--(6.248,1.914)--(6.258,1.926)--(6.268,1.939)--(6.278,1.951)--(6.289,1.964)%
  --(6.299,1.978)--(6.309,1.991)--(6.319,2.005)--(6.330,2.019)--(6.340,2.033)--(6.350,2.047)%
  --(6.361,2.061)--(6.371,2.076)--(6.381,2.090)--(6.391,2.105)--(6.402,2.119)--(6.412,2.134)%
  --(6.422,2.148)--(6.432,2.163)--(6.443,2.177)--(6.453,2.192)--(6.463,2.206)--(6.473,2.220)%
  --(6.484,2.235)--(6.494,2.248)--(6.504,2.262)--(6.515,2.276)--(6.525,2.289)--(6.535,2.302)%
  --(6.545,2.315)--(6.556,2.327)--(6.566,2.340)--(6.576,2.351)--(6.586,2.363)--(6.597,2.374)%
  --(6.607,2.385)--(6.617,2.395)--(6.628,2.405)--(6.638,2.414)--(6.648,2.423)--(6.658,2.432)%
  --(6.669,2.440)--(6.679,2.448)--(6.689,2.455)--(6.699,2.462)--(6.710,2.469)--(6.720,2.475)%
  --(6.730,2.481)--(6.740,2.486)--(6.751,2.492)--(6.761,2.496)--(6.771,2.501)--(6.782,2.505)%
  --(6.792,2.508)--(6.802,2.512)--(6.812,2.515)--(6.823,2.517)--(6.833,2.519)--(6.843,2.521)%
  --(6.853,2.523)--(6.864,2.524)--(6.874,2.525)--(6.884,2.526)--(6.895,2.527)--(6.905,2.527)%
  --(6.915,2.527)--(6.925,2.526)--(6.936,2.525)--(6.946,2.524)--(6.956,2.523)--(6.966,2.522)%
  --(6.977,2.520)--(6.987,2.518)--(6.997,2.515)--(7.007,2.513)--(7.018,2.510)--(7.028,2.507)%
  --(7.038,2.504)--(7.049,2.500)--(7.059,2.497)--(7.069,2.493)--(7.079,2.489)--(7.090,2.485)%
  --(7.100,2.480)--(7.110,2.476)--(7.120,2.471)--(7.131,2.467)--(7.141,2.462)--(7.151,2.457)%
  --(7.162,2.453)--(7.172,2.448)--(7.182,2.443)--(7.192,2.438)--(7.203,2.434)--(7.213,2.429)%
  --(7.223,2.424)--(7.233,2.420)--(7.244,2.415)--(7.254,2.411)--(7.264,2.406)--(7.274,2.402)%
  --(7.285,2.398)--(7.295,2.394)--(7.305,2.390)--(7.316,2.387)--(7.326,2.383)--(7.336,2.380)%
  --(7.346,2.377)--(7.357,2.375)--(7.367,2.372)--(7.377,2.370)--(7.387,2.368)--(7.398,2.367)%
  --(7.408,2.366)--(7.418,2.365)--(7.429,2.364)--(7.439,2.364)--(7.449,2.365)--(7.459,2.365)%
  --(7.470,2.366)--(7.480,2.368)--(7.490,2.369)--(7.500,2.371)--(7.511,2.374)--(7.521,2.376)%
  --(7.531,2.379)--(7.541,2.382)--(7.552,2.386)--(7.562,2.390)--(7.572,2.394)--(7.583,2.398)%
  --(7.593,2.403)--(7.603,2.408)--(7.613,2.413)--(7.624,2.419)--(7.634,2.424)--(7.644,2.430)%
  --(7.654,2.436)--(7.665,2.443)--(7.675,2.449)--(7.685,2.456)--(7.696,2.463)--(7.706,2.470)%
  --(7.716,2.477)--(7.726,2.485)--(7.737,2.492)--(7.747,2.500)--(7.757,2.508)--(7.767,2.516)%
  --(7.778,2.524)--(7.788,2.532)--(7.798,2.541)--(7.808,2.549)--(7.819,2.558)--(7.829,2.567)%
  --(7.839,2.575)--(7.850,2.584)--(7.860,2.593)--(7.870,2.602)--(7.880,2.611)--(7.891,2.620)%
  --(7.901,2.630)--(7.911,2.639)--(7.921,2.649)--(7.932,2.659)--(7.942,2.669)--(7.952,2.680)%
  --(7.963,2.691)--(7.973,2.702)--(7.983,2.713)--(7.993,2.724)--(8.004,2.736)--(8.014,2.748)%
  --(8.024,2.761)--(8.034,2.774)--(8.045,2.787)--(8.055,2.801)--(8.065,2.815)--(8.075,2.830)%
  --(8.086,2.845)--(8.096,2.861)--(8.106,2.877)--(8.117,2.894)--(8.127,2.911)--(8.137,2.929)%
  --(8.147,2.947)--(8.158,2.966)--(8.168,2.985)--(8.178,3.005)--(8.188,3.026)--(8.199,3.047)%
  --(8.209,3.070)--(8.219,3.092)--(8.230,3.116)--(8.240,3.140)--(8.250,3.165)--(8.260,3.191)%
  --(8.271,3.217)--(8.281,3.244)--(8.291,3.272)--(8.301,3.301)--(8.312,3.330)--(8.322,3.360)%
  --(8.332,3.391)--(8.342,3.422)--(8.353,3.453)--(8.363,3.486)--(8.373,3.518)--(8.384,3.551)%
  --(8.394,3.585)--(8.404,3.619)--(8.414,3.654)--(8.425,3.688)--(8.435,3.724)--(8.445,3.759)%
  --(8.455,3.795)--(8.466,3.831)--(8.476,3.868)--(8.486,3.905)--(8.497,3.941)--(8.507,3.979)%
  --(8.517,4.016)--(8.527,4.053)--(8.538,4.091)--(8.548,4.128)--(8.558,4.166)--(8.568,4.204)%
  --(8.579,4.242)--(8.589,4.279)--(8.599,4.317)--(8.609,4.355)--(8.620,4.392)--(8.630,4.430)%
  --(8.640,4.467)--(8.651,4.505)--(8.661,4.542)--(8.671,4.579)--(8.681,4.615)--(8.692,4.652)%
  --(8.702,4.688)--(8.712,4.724)--(8.722,4.760)--(8.733,4.796)--(8.743,4.831)--(8.753,4.866)%
  --(8.764,4.901)--(8.774,4.936)--(8.784,4.971)--(8.794,5.005)--(8.805,5.039)--(8.815,5.073)%
  --(8.825,5.107)--(8.835,5.141)--(8.846,5.174)--(8.856,5.207)--(8.866,5.240)--(8.876,5.273)%
  --(8.887,5.306)--(8.897,5.338)--(8.907,5.370)--(8.918,5.402)--(8.928,5.434)--(8.938,5.465)%
  --(8.948,5.496)--(8.959,5.527)--(8.969,5.558)--(8.979,5.589)--(8.989,5.619)--(9.000,5.649)%
  --(9.010,5.679)--(9.020,5.709)--(9.031,5.739)--(9.041,5.768)--(9.051,5.797)--(9.061,5.826)%
  --(9.072,5.855)--(9.082,5.883)--(9.092,5.911)--(9.102,5.939)--(9.113,5.967)--(9.123,5.995)%
  --(9.133,6.022)--(9.143,6.049)--(9.154,6.076)--(9.164,6.103)--(9.174,6.129)--(9.185,6.156)%
  --(9.195,6.182)--(9.205,6.207)--(9.215,6.233)--(9.226,6.258)--(9.236,6.283)--(9.246,6.308)%
  --(9.256,6.333)--(9.267,6.357)--(9.277,6.381)--(9.287,6.405)--(9.298,6.429)--(9.308,6.452)%
  --(9.318,6.475)--(9.328,6.498)--(9.339,6.521)--(9.349,6.544)--(9.359,6.566)--(9.369,6.588)%
  --(9.380,6.609)--(9.390,6.631)--(9.400,6.652)--(9.410,6.673)--(9.421,6.694)--(9.431,6.714)%
  --(9.441,6.734)--(9.452,6.754)--(9.462,6.774)--(9.472,6.793)--(9.482,6.812)--(9.493,6.831)%
  --(9.503,6.850)--(9.513,6.868)--(9.523,6.885)--(9.534,6.902)--(9.544,6.919)--(9.554,6.934)%
  --(9.565,6.949)--(9.575,6.962)--(9.585,6.975)--(9.595,6.987)--(9.606,6.997)--(9.616,7.006)%
  --(9.626,7.013)--(9.636,7.019)--(9.647,7.024)--(9.657,7.027)--(9.667,7.028)--(9.677,7.027)%
  --(9.688,7.024)--(9.698,7.019)--(9.708,7.012)--(9.719,7.003)--(9.729,6.992)--(9.739,6.978)%
  --(9.749,6.962)--(9.760,6.943)--(9.770,6.921)--(9.780,6.897)--(9.790,6.870)--(9.801,6.840)%
  --(9.811,6.807)--(9.821,6.771)--(9.832,6.731)--(9.842,6.688)--(9.852,6.642)--(9.862,6.593)%
  --(9.873,6.540)--(9.883,6.483)--(9.893,6.423)--(9.903,6.358)--(9.914,6.290)--(9.924,6.219)%
  --(9.934,6.144)--(9.944,6.066)--(9.955,5.986)--(9.965,5.902)--(9.975,5.816)--(9.986,5.728)%
  --(9.996,5.638)--(10.006,5.546)--(10.016,5.452)--(10.027,5.357)--(10.037,5.261)--(10.047,5.163)%
  --(10.057,5.065)--(10.068,4.966)--(10.078,4.866)--(10.088,4.767)--(10.099,4.667)--(10.109,4.567)%
  --(10.119,4.468)--(10.129,4.369)--(10.140,4.271)--(10.150,4.174)--(10.160,4.078)--(10.170,3.983)%
  --(10.181,3.890)--(10.191,3.799)--(10.201,3.710)--(10.211,3.623)--(10.222,3.538)--(10.232,3.456)%
  --(10.242,3.376)--(10.253,3.300)--(10.263,3.226)--(10.273,3.156)--(10.283,3.090)--(10.294,3.027)%
  --(10.304,2.969)--(10.314,2.914)--(10.324,2.864)--(10.335,2.817)--(10.345,2.775)--(10.355,2.736)%
  --(10.366,2.700)--(10.376,2.668)--(10.386,2.639)--(10.396,2.613)--(10.407,2.590)--(10.417,2.569)%
  --(10.427,2.552)--(10.437,2.537)--(10.448,2.524)--(10.458,2.513)--(10.468,2.505)--(10.478,2.498)%
  --(10.489,2.493)--(10.499,2.490)--(10.509,2.488)--(10.520,2.487)--(10.530,2.488)--(10.540,2.489)%
  --(10.550,2.492)--(10.561,2.495)--(10.571,2.499)--(10.581,2.503)--(10.591,2.508)--(10.602,2.512)%
  --(10.612,2.517)--(10.622,2.521)--(10.633,2.526)--(10.643,2.529)--(10.653,2.533)--(10.663,2.535)%
  --(10.674,2.537)--(10.684,2.537)--(10.694,2.537)--(10.704,2.535)--(10.715,2.531)--(10.725,2.526)%
  --(10.735,2.520)--(10.745,2.512)--(10.756,2.502)--(10.766,2.492)--(10.776,2.480)--(10.787,2.466)%
  --(10.797,2.452)--(10.807,2.436)--(10.817,2.420)--(10.828,2.403)--(10.838,2.384)--(10.848,2.365)%
  --(10.858,2.346)--(10.869,2.325)--(10.879,2.304)--(10.889,2.283)--(10.900,2.261)--(10.910,2.239)%
  --(10.920,2.217)--(10.930,2.194)--(10.941,2.171)--(10.951,2.148)--(10.961,2.125)--(10.971,2.103)%
  --(10.982,2.080)--(10.992,2.058)--(11.002,2.035)--(11.012,2.014)--(11.023,1.992)--(11.033,1.971)%
  --(11.043,1.951)--(11.054,1.931)--(11.064,1.913)--(11.074,1.894)--(11.084,1.877)--(11.095,1.861)%
  --(11.105,1.845)--(11.115,1.831)--(11.125,1.818)--(11.136,1.806)--(11.146,1.795)--(11.156,1.786)%
  --(11.167,1.777)--(11.177,1.770)--(11.187,1.763)--(11.197,1.757)--(11.208,1.753)--(11.218,1.749)%
  --(11.228,1.746)--(11.238,1.743)--(11.249,1.741)--(11.259,1.740)--(11.269,1.740)--(11.279,1.740)%
  --(11.290,1.740)--(11.300,1.741)--(11.310,1.742)--(11.321,1.744)--(11.331,1.746)--(11.341,1.748)%
  --(11.351,1.751)--(11.362,1.753)--(11.372,1.756)--(11.382,1.758)--(11.392,1.761)--(11.403,1.763)%
  --(11.413,1.766)--(11.423,1.768)--(11.434,1.770)--(11.444,1.772)--(11.454,1.774)--(11.464,1.775)%
  --(11.475,1.776)--(11.485,1.776)--(11.495,1.776)--(11.505,1.775)--(11.516,1.774)--(11.526,1.772)%
  --(11.536,1.769)--(11.546,1.766)--(11.557,1.762)--(11.567,1.757)--(11.577,1.751)--(11.588,1.745)%
  --(11.598,1.738)--(11.608,1.731)--(11.618,1.722)--(11.629,1.714)--(11.639,1.704)--(11.649,1.694)%
  --(11.659,1.684)--(11.670,1.673)--(11.680,1.662)--(11.690,1.650)--(11.701,1.637)--(11.711,1.624)%
  --(11.721,1.611)--(11.731,1.597)--(11.742,1.583)--(11.752,1.569)--(11.762,1.554)--(11.772,1.539)%
  --(11.783,1.523)--(11.793,1.508)--(11.803,1.492)--(11.813,1.475)--(11.824,1.459)--(11.834,1.442)%
  --(11.844,1.425)--(11.855,1.408)--(11.865,1.390)--(11.875,1.373)--(11.885,1.355)--(11.896,1.337)%
  --(11.906,1.320)--(11.916,1.302)--(11.926,1.284)--(11.937,1.266)--(11.947,1.248);
\gpcolor{color=gp lt color border}
\node[gp node right] at (4.448,7.431) {w/o data flow};
\gpfill{color=gp lt color 6,opacity=0.10} (4.632,7.354)--(5.548,7.354)--(5.548,7.508)--(4.632,7.508)--cycle;
\gpcolor{color=gp lt color 6}
\draw[gp path] (4.632,7.354)--(5.548,7.354)--(5.548,7.508)--(4.632,7.508)--cycle;
\gpfill{color=gp lt color 6,opacity=0.10} (2.919,1.007)--(2.919,1.007)--(2.928,1.005)--(2.937,1.002)%
    --(2.946,1.000)--(2.955,0.998)--(2.964,0.996)--(2.973,0.993)--(2.982,0.991)%
    --(2.991,0.989)--(3.000,0.987)--(3.008,0.985)--(3.008,0.985)--(2.919,0.985)--cycle;
\gpfill{color=gp lt color 6,opacity=0.10} (3.278,0.985)--(3.281,0.986)--(3.290,0.989)--(3.299,0.993)%
    --(3.308,0.997)--(3.317,1.001)--(3.326,1.005)--(3.335,1.010)--(3.344,1.015)%
    --(3.353,1.020)--(3.362,1.026)--(3.371,1.032)--(3.380,1.038)--(3.389,1.044)%
    --(3.398,1.051)--(3.407,1.058)--(3.416,1.065)--(3.425,1.072)--(3.434,1.079)%
    --(3.443,1.086)--(3.452,1.094)--(3.461,1.101)--(3.470,1.109)--(3.479,1.116)%
    --(3.488,1.124)--(3.497,1.132)--(3.506,1.140)--(3.516,1.148)--(3.525,1.155)%
    --(3.534,1.163)--(3.543,1.171)--(3.552,1.178)--(3.561,1.186)--(3.570,1.193)%
    --(3.579,1.201)--(3.588,1.208)--(3.597,1.215)--(3.606,1.222)--(3.615,1.229)%
    --(3.624,1.235)--(3.633,1.242)--(3.642,1.248)--(3.651,1.254)--(3.660,1.259)%
    --(3.669,1.265)--(3.678,1.270)--(3.687,1.275)--(3.696,1.279)--(3.705,1.283)%
    --(3.714,1.287)--(3.723,1.290)--(3.732,1.293)--(3.741,1.296)--(3.750,1.298)%
    --(3.760,1.300)--(3.769,1.302)--(3.778,1.303)--(3.787,1.303)--(3.796,1.304)%
    --(3.805,1.304)--(3.814,1.304)--(3.823,1.303)--(3.832,1.302)--(3.841,1.301)%
    --(3.850,1.300)--(3.859,1.298)--(3.868,1.296)--(3.877,1.294)--(3.886,1.291)%
    --(3.895,1.289)--(3.904,1.286)--(3.913,1.283)--(3.922,1.280)--(3.931,1.276)%
    --(3.940,1.273)--(3.949,1.269)--(3.958,1.265)--(3.967,1.262)--(3.976,1.258)%
    --(3.985,1.253)--(3.994,1.249)--(4.004,1.245)--(4.013,1.241)--(4.022,1.236)%
    --(4.031,1.232)--(4.040,1.228)--(4.049,1.223)--(4.058,1.219)--(4.067,1.215)%
    --(4.076,1.210)--(4.085,1.206)--(4.094,1.202)--(4.103,1.198)--(4.112,1.193)%
    --(4.121,1.189)--(4.130,1.186)--(4.139,1.182)--(4.148,1.178)--(4.157,1.175)%
    --(4.166,1.171)--(4.175,1.168)--(4.184,1.165)--(4.193,1.162)--(4.202,1.159)%
    --(4.211,1.156)--(4.220,1.154)--(4.229,1.151)--(4.238,1.149)--(4.248,1.147)%
    --(4.257,1.145)--(4.266,1.143)--(4.275,1.141)--(4.284,1.139)--(4.293,1.137)%
    --(4.302,1.136)--(4.311,1.135)--(4.320,1.133)--(4.329,1.132)--(4.338,1.131)%
    --(4.347,1.130)--(4.356,1.129)--(4.365,1.128)--(4.374,1.127)--(4.383,1.127)%
    --(4.392,1.126)--(4.401,1.126)--(4.410,1.125)--(4.419,1.125)--(4.428,1.125)%
    --(4.437,1.125)--(4.446,1.125)--(4.455,1.125)--(4.464,1.125)--(4.473,1.125)%
    --(4.482,1.125)--(4.492,1.125)--(4.501,1.126)--(4.510,1.126)--(4.519,1.126)%
    --(4.528,1.127)--(4.537,1.127)--(4.546,1.128)--(4.555,1.129)--(4.564,1.129)%
    --(4.573,1.130)--(4.582,1.131)--(4.591,1.132)--(4.600,1.133)--(4.609,1.134)%
    --(4.618,1.135)--(4.627,1.136)--(4.636,1.137)--(4.645,1.138)--(4.654,1.140)%
    --(4.663,1.141)--(4.672,1.142)--(4.681,1.144)--(4.690,1.146)--(4.699,1.147)%
    --(4.708,1.149)--(4.717,1.151)--(4.726,1.153)--(4.736,1.155)--(4.745,1.157)%
    --(4.754,1.159)--(4.763,1.161)--(4.772,1.164)--(4.781,1.166)--(4.790,1.169)%
    --(4.799,1.172)--(4.808,1.175)--(4.817,1.177)--(4.826,1.181)--(4.835,1.184)%
    --(4.844,1.187)--(4.853,1.190)--(4.862,1.194)--(4.871,1.198)--(4.880,1.202)%
    --(4.889,1.205)--(4.898,1.210)--(4.907,1.214)--(4.916,1.218)--(4.925,1.223)%
    --(4.934,1.227)--(4.943,1.232)--(4.952,1.237)--(4.961,1.242)--(4.970,1.247)%
    --(4.980,1.253)--(4.989,1.258)--(4.998,1.264)--(5.007,1.270)--(5.016,1.276)%
    --(5.025,1.282)--(5.034,1.288)--(5.043,1.294)--(5.052,1.300)--(5.061,1.306)%
    --(5.070,1.313)--(5.079,1.319)--(5.088,1.325)--(5.097,1.332)--(5.106,1.338)%
    --(5.115,1.344)--(5.124,1.351)--(5.133,1.357)--(5.142,1.363)--(5.151,1.369)%
    --(5.160,1.375)--(5.169,1.381)--(5.178,1.387)--(5.187,1.392)--(5.196,1.398)%
    --(5.205,1.403)--(5.214,1.409)--(5.224,1.414)--(5.233,1.418)--(5.242,1.423)%
    --(5.251,1.428)--(5.260,1.432)--(5.269,1.436)--(5.278,1.440)--(5.287,1.443)%
    --(5.296,1.447)--(5.305,1.450)--(5.314,1.452)--(5.323,1.455)--(5.332,1.457)%
    --(5.341,1.458)--(5.350,1.460)--(5.359,1.461)--(5.368,1.462)--(5.377,1.462)%
    --(5.386,1.462)--(5.395,1.462)--(5.404,1.461)--(5.413,1.460)--(5.422,1.458)%
    --(5.431,1.456)--(5.440,1.454)--(5.449,1.452)--(5.458,1.450)--(5.468,1.447)%
    --(5.477,1.444)--(5.486,1.441)--(5.495,1.438)--(5.504,1.435)--(5.513,1.432)%
    --(5.522,1.428)--(5.531,1.425)--(5.540,1.422)--(5.549,1.418)--(5.558,1.415)%
    --(5.567,1.412)--(5.576,1.408)--(5.585,1.405)--(5.594,1.402)--(5.603,1.399)%
    --(5.612,1.397)--(5.621,1.394)--(5.630,1.392)--(5.639,1.390)--(5.648,1.388)%
    --(5.657,1.387)--(5.666,1.386)--(5.675,1.385)--(5.684,1.384)--(5.693,1.384)%
    --(5.702,1.385)--(5.711,1.385)--(5.721,1.386)--(5.730,1.388)--(5.739,1.390)%
    --(5.748,1.393)--(5.757,1.396)--(5.766,1.400)--(5.775,1.404)--(5.784,1.409)%
    --(5.793,1.415)--(5.802,1.421)--(5.811,1.428)--(5.820,1.435)--(5.829,1.443)%
    --(5.838,1.452)--(5.847,1.461)--(5.856,1.470)--(5.865,1.480)--(5.874,1.490)%
    --(5.883,1.501)--(5.892,1.511)--(5.901,1.523)--(5.910,1.534)--(5.919,1.545)%
    --(5.928,1.557)--(5.937,1.569)--(5.946,1.581)--(5.955,1.593)--(5.965,1.605)%
    --(5.974,1.617)--(5.983,1.629)--(5.992,1.641)--(6.001,1.653)--(6.010,1.665)%
    --(6.019,1.676)--(6.028,1.688)--(6.037,1.699)--(6.046,1.710)--(6.055,1.720)%
    --(6.064,1.731)--(6.073,1.740)--(6.082,1.750)--(6.091,1.759)--(6.100,1.767)%
    --(6.109,1.775)--(6.118,1.783)--(6.127,1.790)--(6.136,1.796)--(6.145,1.802)%
    --(6.154,1.807)--(6.163,1.811)--(6.172,1.815)--(6.181,1.818)--(6.190,1.819)%
    --(6.199,1.821)--(6.209,1.821)--(6.218,1.820)--(6.227,1.819)--(6.236,1.817)%
    --(6.245,1.814)--(6.254,1.810)--(6.263,1.806)--(6.272,1.801)--(6.281,1.796)%
    --(6.290,1.790)--(6.299,1.784)--(6.308,1.777)--(6.317,1.770)--(6.326,1.763)%
    --(6.335,1.755)--(6.344,1.747)--(6.353,1.739)--(6.362,1.731)--(6.371,1.723)%
    --(6.380,1.714)--(6.389,1.706)--(6.398,1.698)--(6.407,1.690)--(6.416,1.682)%
    --(6.425,1.674)--(6.434,1.666)--(6.443,1.659)--(6.453,1.652)--(6.462,1.645)%
    --(6.471,1.639)--(6.480,1.633)--(6.489,1.628)--(6.498,1.623)--(6.507,1.619)%
    --(6.516,1.615)--(6.525,1.612)--(6.534,1.610)--(6.543,1.609)--(6.552,1.608)%
    --(6.561,1.608)--(6.570,1.610)--(6.579,1.612)--(6.588,1.615)--(6.597,1.619)%
    --(6.606,1.624)--(6.615,1.631)--(6.624,1.638)--(6.633,1.647)--(6.642,1.656)%
    --(6.651,1.667)--(6.660,1.679)--(6.669,1.691)--(6.678,1.705)--(6.687,1.719)%
    --(6.697,1.734)--(6.706,1.749)--(6.715,1.766)--(6.724,1.782)--(6.733,1.800)%
    --(6.742,1.818)--(6.751,1.836)--(6.760,1.855)--(6.769,1.875)--(6.778,1.894)%
    --(6.787,1.914)--(6.796,1.934)--(6.805,1.954)--(6.814,1.975)--(6.823,1.995)%
    --(6.832,2.016)--(6.841,2.036)--(6.850,2.057)--(6.859,2.077)--(6.868,2.098)%
    --(6.877,2.118)--(6.886,2.137)--(6.895,2.157)--(6.904,2.176)--(6.913,2.195)%
    --(6.922,2.213)--(6.931,2.231)--(6.941,2.248)--(6.950,2.265)--(6.959,2.281)%
    --(6.968,2.296)--(6.977,2.311)--(6.986,2.324)--(6.995,2.337)--(7.004,2.349)%
    --(7.013,2.361)--(7.022,2.371)--(7.031,2.380)--(7.040,2.388)--(7.049,2.396)%
    --(7.058,2.402)--(7.067,2.407)--(7.076,2.412)--(7.085,2.415)--(7.094,2.418)%
    --(7.103,2.420)--(7.112,2.421)--(7.121,2.422)--(7.130,2.421)--(7.139,2.420)%
    --(7.148,2.419)--(7.157,2.417)--(7.166,2.414)--(7.175,2.410)--(7.185,2.406)%
    --(7.194,2.401)--(7.203,2.396)--(7.212,2.391)--(7.221,2.385)--(7.230,2.378)%
    --(7.239,2.371)--(7.248,2.364)--(7.257,2.357)--(7.266,2.349)--(7.275,2.341)%
    --(7.284,2.332)--(7.293,2.323)--(7.302,2.315)--(7.311,2.305)--(7.320,2.296)%
    --(7.329,2.287)--(7.338,2.277)--(7.347,2.268)--(7.356,2.258)--(7.365,2.249)%
    --(7.374,2.239)--(7.383,2.230)--(7.392,2.220)--(7.401,2.211)--(7.410,2.202)%
    --(7.419,2.193)--(7.429,2.184)--(7.438,2.175)--(7.447,2.167)--(7.456,2.158)%
    --(7.465,2.150)--(7.474,2.143)--(7.483,2.135)--(7.492,2.128)--(7.501,2.121)%
    --(7.510,2.115)--(7.519,2.109)--(7.528,2.103)--(7.537,2.097)--(7.546,2.092)%
    --(7.555,2.088)--(7.564,2.083)--(7.573,2.079)--(7.582,2.076)--(7.591,2.073)%
    --(7.600,2.071)--(7.609,2.069)--(7.618,2.067)--(7.627,2.066)--(7.636,2.065)%
    --(7.645,2.065)--(7.654,2.066)--(7.663,2.067)--(7.673,2.069)--(7.682,2.071)%
    --(7.691,2.074)--(7.700,2.078)--(7.709,2.082)--(7.718,2.086)--(7.727,2.092)%
    --(7.736,2.098)--(7.745,2.105)--(7.754,2.112)--(7.763,2.121)--(7.772,2.129)%
    --(7.781,2.139)--(7.790,2.150)--(7.799,2.161)--(7.808,2.173)--(7.817,2.186)%
    --(7.826,2.199)--(7.835,2.214)--(7.844,2.229)--(7.853,2.245)--(7.862,2.262)%
    --(7.871,2.280)--(7.880,2.298)--(7.889,2.317)--(7.898,2.337)--(7.907,2.357)%
    --(7.917,2.378)--(7.926,2.399)--(7.935,2.421)--(7.944,2.443)--(7.953,2.465)%
    --(7.962,2.488)--(7.971,2.511)--(7.980,2.534)--(7.989,2.557)--(7.998,2.580)%
    --(8.007,2.604)--(8.016,2.627)--(8.025,2.650)--(8.034,2.674)--(8.043,2.697)%
    --(8.052,2.720)--(8.061,2.742)--(8.070,2.765)--(8.079,2.787)--(8.088,2.809)%
    --(8.097,2.830)--(8.106,2.851)--(8.115,2.871)--(8.124,2.891)--(8.133,2.910)%
    --(8.142,2.929)--(8.151,2.947)--(8.161,2.964)--(8.170,2.980)--(8.179,2.996)%
    --(8.188,3.010)--(8.197,3.024)--(8.206,3.036)--(8.215,3.048)--(8.224,3.059)%
    --(8.233,3.068)--(8.242,3.076)--(8.251,3.083)--(8.260,3.089)--(8.269,3.094)%
    --(8.278,3.097)--(8.287,3.100)--(8.296,3.101)--(8.305,3.101)--(8.314,3.100)%
    --(8.323,3.099)--(8.332,3.096)--(8.341,3.093)--(8.350,3.089)--(8.359,3.084)%
    --(8.368,3.079)--(8.377,3.073)--(8.386,3.067)--(8.395,3.060)--(8.405,3.053)%
    --(8.414,3.046)--(8.423,3.038)--(8.432,3.031)--(8.441,3.023)--(8.450,3.015)%
    --(8.459,3.006)--(8.468,2.998)--(8.477,2.991)--(8.486,2.983)--(8.495,2.975)%
    --(8.504,2.968)--(8.513,2.961)--(8.522,2.954)--(8.531,2.948)--(8.540,2.942)%
    --(8.549,2.937)--(8.558,2.933)--(8.567,2.929)--(8.576,2.926)--(8.585,2.923)%
    --(8.594,2.922)--(8.603,2.921)--(8.612,2.922)--(8.621,2.923)--(8.630,2.925)%
    --(8.639,2.929)--(8.649,2.934)--(8.658,2.940)--(8.667,2.947)--(8.676,2.956)%
    --(8.685,2.966)--(8.694,2.977)--(8.703,2.989)--(8.712,3.002)--(8.721,3.017)%
    --(8.730,3.032)--(8.739,3.049)--(8.748,3.066)--(8.757,3.084)--(8.766,3.103)%
    --(8.775,3.123)--(8.784,3.143)--(8.793,3.164)--(8.802,3.186)--(8.811,3.208)%
    --(8.820,3.230)--(8.829,3.253)--(8.838,3.276)--(8.847,3.300)--(8.856,3.324)%
    --(8.865,3.348)--(8.874,3.372)--(8.883,3.396)--(8.893,3.420)--(8.902,3.445)%
    --(8.911,3.469)--(8.920,3.493)--(8.929,3.517)--(8.938,3.540)--(8.947,3.563)%
    --(8.956,3.586)--(8.965,3.609)--(8.974,3.631)--(8.983,3.652)--(8.992,3.673)%
    --(9.001,3.694)--(9.010,3.713)--(9.019,3.732)--(9.028,3.750)--(9.037,3.768)%
    --(9.046,3.784)--(9.055,3.799)--(9.064,3.814)--(9.073,3.827)--(9.082,3.839)%
    --(9.091,3.850)--(9.100,3.861)--(9.109,3.870)--(9.118,3.879)--(9.127,3.886)%
    --(9.137,3.894)--(9.146,3.900)--(9.155,3.907)--(9.164,3.913)--(9.173,3.918)%
    --(9.182,3.924)--(9.191,3.929)--(9.200,3.934)--(9.209,3.940)--(9.218,3.945)%
    --(9.227,3.951)--(9.236,3.957)--(9.245,3.964)--(9.254,3.971)--(9.263,3.979)%
    --(9.272,3.987)--(9.281,3.996)--(9.290,4.006)--(9.299,4.017)--(9.308,4.029)%
    --(9.317,4.042)--(9.326,4.056)--(9.335,4.071)--(9.344,4.088)--(9.353,4.107)%
    --(9.362,4.126)--(9.371,4.148)--(9.381,4.171)--(9.390,4.196)--(9.399,4.223)%
    --(9.408,4.252)--(9.417,4.282)--(9.426,4.315)--(9.435,4.351)--(9.444,4.388)%
    --(9.453,4.428)--(9.462,4.471)--(9.471,4.516)--(9.480,4.563)--(9.489,4.614)%
    --(9.498,4.667)--(9.507,4.723)--(9.516,4.780)--(9.525,4.841)--(9.534,4.903)%
    --(9.543,4.966)--(9.552,5.031)--(9.561,5.098)--(9.570,5.166)--(9.579,5.234)%
    --(9.588,5.304)--(9.597,5.374)--(9.606,5.444)--(9.615,5.515)--(9.625,5.585)%
    --(9.634,5.655)--(9.643,5.725)--(9.652,5.794)--(9.661,5.863)--(9.670,5.930)%
    --(9.679,5.996)--(9.688,6.061)--(9.697,6.124)--(9.706,6.186)--(9.715,6.245)%
    --(9.724,6.302)--(9.733,6.357)--(9.742,6.409)--(9.751,6.459)--(9.760,6.505)%
    --(9.769,6.549)--(9.778,6.589)--(9.787,6.625)--(9.796,6.658)--(9.805,6.687)%
    --(9.814,6.712)--(9.823,6.732)--(9.832,6.748)--(9.841,6.759)--(9.850,6.765)%
    --(9.859,6.766)--(9.868,6.762)--(9.878,6.752)--(9.887,6.737)--(9.896,6.716)%
    --(9.905,6.689)--(9.914,6.655)--(9.923,6.617)--(9.932,6.573)--(9.941,6.524)%
    --(9.950,6.471)--(9.959,6.413)--(9.968,6.352)--(9.977,6.286)--(9.986,6.217)%
    --(9.995,6.144)--(10.004,6.069)--(10.013,5.991)--(10.022,5.911)--(10.031,5.828)%
    --(10.040,5.744)--(10.049,5.658)--(10.058,5.571)--(10.067,5.483)--(10.076,5.395)%
    --(10.085,5.306)--(10.094,5.216)--(10.103,5.127)--(10.112,5.039)--(10.122,4.951)%
    --(10.131,4.864)--(10.140,4.779)--(10.149,4.695)--(10.158,4.612)--(10.167,4.533)%
    --(10.176,4.455)--(10.185,4.380)--(10.194,4.308)--(10.203,4.240)--(10.212,4.175)%
    --(10.221,4.114)--(10.230,4.057)--(10.239,4.004)--(10.248,3.957)--(10.257,3.914)%
    --(10.266,3.876)--(10.275,3.844)--(10.284,3.818)--(10.293,3.798)--(10.302,3.784)%
    --(10.311,3.777)--(10.320,3.777)--(10.329,3.782)--(10.338,3.794)--(10.347,3.812)%
    --(10.356,3.835)--(10.366,3.864)--(10.375,3.897)--(10.384,3.935)--(10.393,3.978)%
    --(10.402,4.025)--(10.411,4.075)--(10.420,4.129)--(10.429,4.187)--(10.438,4.247)%
    --(10.447,4.310)--(10.456,4.376)--(10.465,4.444)--(10.474,4.514)--(10.483,4.585)%
    --(10.492,4.658)--(10.501,4.732)--(10.510,4.807)--(10.519,4.882)--(10.528,4.958)%
    --(10.537,5.034)--(10.546,5.109)--(10.555,5.184)--(10.564,5.258)--(10.573,5.331)%
    --(10.582,5.402)--(10.591,5.472)--(10.600,5.540)--(10.610,5.606)--(10.619,5.670)%
    --(10.628,5.730)--(10.637,5.788)--(10.646,5.842)--(10.655,5.893)--(10.664,5.940)%
    --(10.673,5.983)--(10.682,6.021)--(10.691,6.055)--(10.700,6.083)--(10.709,6.107)%
    --(10.718,6.125)--(10.727,6.138)--(10.736,6.145)--(10.745,6.146)--(10.754,6.143)%
    --(10.763,6.135)--(10.772,6.121)--(10.781,6.104)--(10.790,6.082)--(10.799,6.056)%
    --(10.808,6.026)--(10.817,5.992)--(10.826,5.955)--(10.835,5.914)--(10.844,5.870)%
    --(10.854,5.823)--(10.863,5.773)--(10.872,5.720)--(10.881,5.665)--(10.890,5.608)%
    --(10.899,5.548)--(10.908,5.487)--(10.917,5.424)--(10.926,5.359)--(10.935,5.293)%
    --(10.944,5.226)--(10.953,5.158)--(10.962,5.090)--(10.971,5.020)--(10.980,4.951)%
    --(10.989,4.881)--(10.998,4.811)--(11.007,4.741)--(11.016,4.671)--(11.025,4.602)%
    --(11.034,4.534)--(11.043,4.467)--(11.052,4.401)--(11.061,4.336)--(11.070,4.272)%
    --(11.079,4.211)--(11.088,4.151)--(11.098,4.093)--(11.107,4.038)--(11.116,3.984)%
    --(11.125,3.934)--(11.134,3.886)--(11.143,3.841)--(11.152,3.799)--(11.161,3.759)%
    --(11.170,3.722)--(11.179,3.687)--(11.188,3.654)--(11.197,3.624)--(11.206,3.595)%
    --(11.215,3.569)--(11.224,3.545)--(11.233,3.522)--(11.242,3.501)--(11.251,3.482)%
    --(11.260,3.464)--(11.269,3.448)--(11.278,3.433)--(11.287,3.420)--(11.296,3.407)%
    --(11.305,3.396)--(11.314,3.386)--(11.323,3.376)--(11.332,3.368)--(11.342,3.360)%
    --(11.351,3.352)--(11.360,3.345)--(11.369,3.339)--(11.378,3.333)--(11.387,3.327)%
    --(11.396,3.322)--(11.405,3.316)--(11.414,3.310)--(11.423,3.305)--(11.432,3.299)%
    --(11.441,3.292)--(11.450,3.285)--(11.459,3.278)--(11.468,3.270)--(11.477,3.262)%
    --(11.486,3.253)--(11.495,3.242)--(11.504,3.231)--(11.513,3.219)--(11.522,3.206)%
    --(11.531,3.191)--(11.540,3.175)--(11.549,3.158)--(11.558,3.139)--(11.567,3.119)%
    --(11.576,3.097)--(11.586,3.075)--(11.595,3.051)--(11.604,3.026)--(11.613,3.000)%
    --(11.622,2.972)--(11.631,2.944)--(11.640,2.914)--(11.649,2.884)--(11.658,2.852)%
    --(11.667,2.820)--(11.676,2.786)--(11.685,2.751)--(11.694,2.716)--(11.703,2.680)%
    --(11.712,2.643)--(11.721,2.605)--(11.730,2.566)--(11.739,2.527)--(11.748,2.487)%
    --(11.757,2.446)--(11.766,2.405)--(11.775,2.363)--(11.784,2.320)--(11.793,2.277)%
    --(11.802,2.233)--(11.811,2.189)--(11.820,2.144)--(11.830,2.099)--(11.839,2.054)%
    --(11.848,2.008)--(11.857,1.962)--(11.866,1.915)--(11.875,1.868)--(11.884,1.821)%
    --(11.893,1.774)--(11.902,1.727)--(11.911,1.679)--(11.920,1.631)--(11.929,1.584)%
    --(11.938,1.536)--(11.947,1.488)--(11.947,0.985)--cycle;
\draw[gp path] (2.919,1.007)--(2.928,1.005)--(2.937,1.002)--(2.946,1.000)--(2.955,0.998)%
  --(2.964,0.996)--(2.973,0.993)--(2.982,0.991)--(2.991,0.989)--(3.000,0.987)--(3.008,0.985);
\draw[gp path] (3.278,0.985)--(3.281,0.986)--(3.290,0.989)--(3.299,0.993)--(3.308,0.997)%
  --(3.317,1.001)--(3.326,1.005)--(3.335,1.010)--(3.344,1.015)--(3.353,1.020)--(3.362,1.026)%
  --(3.371,1.032)--(3.380,1.038)--(3.389,1.044)--(3.398,1.051)--(3.407,1.058)--(3.416,1.065)%
  --(3.425,1.072)--(3.434,1.079)--(3.443,1.086)--(3.452,1.094)--(3.461,1.101)--(3.470,1.109)%
  --(3.479,1.116)--(3.488,1.124)--(3.497,1.132)--(3.506,1.140)--(3.516,1.148)--(3.525,1.155)%
  --(3.534,1.163)--(3.543,1.171)--(3.552,1.178)--(3.561,1.186)--(3.570,1.193)--(3.579,1.201)%
  --(3.588,1.208)--(3.597,1.215)--(3.606,1.222)--(3.615,1.229)--(3.624,1.235)--(3.633,1.242)%
  --(3.642,1.248)--(3.651,1.254)--(3.660,1.259)--(3.669,1.265)--(3.678,1.270)--(3.687,1.275)%
  --(3.696,1.279)--(3.705,1.283)--(3.714,1.287)--(3.723,1.290)--(3.732,1.293)--(3.741,1.296)%
  --(3.750,1.298)--(3.760,1.300)--(3.769,1.302)--(3.778,1.303)--(3.787,1.303)--(3.796,1.304)%
  --(3.805,1.304)--(3.814,1.304)--(3.823,1.303)--(3.832,1.302)--(3.841,1.301)--(3.850,1.300)%
  --(3.859,1.298)--(3.868,1.296)--(3.877,1.294)--(3.886,1.291)--(3.895,1.289)--(3.904,1.286)%
  --(3.913,1.283)--(3.922,1.280)--(3.931,1.276)--(3.940,1.273)--(3.949,1.269)--(3.958,1.265)%
  --(3.967,1.262)--(3.976,1.258)--(3.985,1.253)--(3.994,1.249)--(4.004,1.245)--(4.013,1.241)%
  --(4.022,1.236)--(4.031,1.232)--(4.040,1.228)--(4.049,1.223)--(4.058,1.219)--(4.067,1.215)%
  --(4.076,1.210)--(4.085,1.206)--(4.094,1.202)--(4.103,1.198)--(4.112,1.193)--(4.121,1.189)%
  --(4.130,1.186)--(4.139,1.182)--(4.148,1.178)--(4.157,1.175)--(4.166,1.171)--(4.175,1.168)%
  --(4.184,1.165)--(4.193,1.162)--(4.202,1.159)--(4.211,1.156)--(4.220,1.154)--(4.229,1.151)%
  --(4.238,1.149)--(4.248,1.147)--(4.257,1.145)--(4.266,1.143)--(4.275,1.141)--(4.284,1.139)%
  --(4.293,1.137)--(4.302,1.136)--(4.311,1.135)--(4.320,1.133)--(4.329,1.132)--(4.338,1.131)%
  --(4.347,1.130)--(4.356,1.129)--(4.365,1.128)--(4.374,1.127)--(4.383,1.127)--(4.392,1.126)%
  --(4.401,1.126)--(4.410,1.125)--(4.419,1.125)--(4.428,1.125)--(4.437,1.125)--(4.446,1.125)%
  --(4.455,1.125)--(4.464,1.125)--(4.473,1.125)--(4.482,1.125)--(4.492,1.125)--(4.501,1.126)%
  --(4.510,1.126)--(4.519,1.126)--(4.528,1.127)--(4.537,1.127)--(4.546,1.128)--(4.555,1.129)%
  --(4.564,1.129)--(4.573,1.130)--(4.582,1.131)--(4.591,1.132)--(4.600,1.133)--(4.609,1.134)%
  --(4.618,1.135)--(4.627,1.136)--(4.636,1.137)--(4.645,1.138)--(4.654,1.140)--(4.663,1.141)%
  --(4.672,1.142)--(4.681,1.144)--(4.690,1.146)--(4.699,1.147)--(4.708,1.149)--(4.717,1.151)%
  --(4.726,1.153)--(4.736,1.155)--(4.745,1.157)--(4.754,1.159)--(4.763,1.161)--(4.772,1.164)%
  --(4.781,1.166)--(4.790,1.169)--(4.799,1.172)--(4.808,1.175)--(4.817,1.177)--(4.826,1.181)%
  --(4.835,1.184)--(4.844,1.187)--(4.853,1.190)--(4.862,1.194)--(4.871,1.198)--(4.880,1.202)%
  --(4.889,1.205)--(4.898,1.210)--(4.907,1.214)--(4.916,1.218)--(4.925,1.223)--(4.934,1.227)%
  --(4.943,1.232)--(4.952,1.237)--(4.961,1.242)--(4.970,1.247)--(4.980,1.253)--(4.989,1.258)%
  --(4.998,1.264)--(5.007,1.270)--(5.016,1.276)--(5.025,1.282)--(5.034,1.288)--(5.043,1.294)%
  --(5.052,1.300)--(5.061,1.306)--(5.070,1.313)--(5.079,1.319)--(5.088,1.325)--(5.097,1.332)%
  --(5.106,1.338)--(5.115,1.344)--(5.124,1.351)--(5.133,1.357)--(5.142,1.363)--(5.151,1.369)%
  --(5.160,1.375)--(5.169,1.381)--(5.178,1.387)--(5.187,1.392)--(5.196,1.398)--(5.205,1.403)%
  --(5.214,1.409)--(5.224,1.414)--(5.233,1.418)--(5.242,1.423)--(5.251,1.428)--(5.260,1.432)%
  --(5.269,1.436)--(5.278,1.440)--(5.287,1.443)--(5.296,1.447)--(5.305,1.450)--(5.314,1.452)%
  --(5.323,1.455)--(5.332,1.457)--(5.341,1.458)--(5.350,1.460)--(5.359,1.461)--(5.368,1.462)%
  --(5.377,1.462)--(5.386,1.462)--(5.395,1.462)--(5.404,1.461)--(5.413,1.460)--(5.422,1.458)%
  --(5.431,1.456)--(5.440,1.454)--(5.449,1.452)--(5.458,1.450)--(5.468,1.447)--(5.477,1.444)%
  --(5.486,1.441)--(5.495,1.438)--(5.504,1.435)--(5.513,1.432)--(5.522,1.428)--(5.531,1.425)%
  --(5.540,1.422)--(5.549,1.418)--(5.558,1.415)--(5.567,1.412)--(5.576,1.408)--(5.585,1.405)%
  --(5.594,1.402)--(5.603,1.399)--(5.612,1.397)--(5.621,1.394)--(5.630,1.392)--(5.639,1.390)%
  --(5.648,1.388)--(5.657,1.387)--(5.666,1.386)--(5.675,1.385)--(5.684,1.384)--(5.693,1.384)%
  --(5.702,1.385)--(5.711,1.385)--(5.721,1.386)--(5.730,1.388)--(5.739,1.390)--(5.748,1.393)%
  --(5.757,1.396)--(5.766,1.400)--(5.775,1.404)--(5.784,1.409)--(5.793,1.415)--(5.802,1.421)%
  --(5.811,1.428)--(5.820,1.435)--(5.829,1.443)--(5.838,1.452)--(5.847,1.461)--(5.856,1.470)%
  --(5.865,1.480)--(5.874,1.490)--(5.883,1.501)--(5.892,1.511)--(5.901,1.523)--(5.910,1.534)%
  --(5.919,1.545)--(5.928,1.557)--(5.937,1.569)--(5.946,1.581)--(5.955,1.593)--(5.965,1.605)%
  --(5.974,1.617)--(5.983,1.629)--(5.992,1.641)--(6.001,1.653)--(6.010,1.665)--(6.019,1.676)%
  --(6.028,1.688)--(6.037,1.699)--(6.046,1.710)--(6.055,1.720)--(6.064,1.731)--(6.073,1.740)%
  --(6.082,1.750)--(6.091,1.759)--(6.100,1.767)--(6.109,1.775)--(6.118,1.783)--(6.127,1.790)%
  --(6.136,1.796)--(6.145,1.802)--(6.154,1.807)--(6.163,1.811)--(6.172,1.815)--(6.181,1.818)%
  --(6.190,1.819)--(6.199,1.821)--(6.209,1.821)--(6.218,1.820)--(6.227,1.819)--(6.236,1.817)%
  --(6.245,1.814)--(6.254,1.810)--(6.263,1.806)--(6.272,1.801)--(6.281,1.796)--(6.290,1.790)%
  --(6.299,1.784)--(6.308,1.777)--(6.317,1.770)--(6.326,1.763)--(6.335,1.755)--(6.344,1.747)%
  --(6.353,1.739)--(6.362,1.731)--(6.371,1.723)--(6.380,1.714)--(6.389,1.706)--(6.398,1.698)%
  --(6.407,1.690)--(6.416,1.682)--(6.425,1.674)--(6.434,1.666)--(6.443,1.659)--(6.453,1.652)%
  --(6.462,1.645)--(6.471,1.639)--(6.480,1.633)--(6.489,1.628)--(6.498,1.623)--(6.507,1.619)%
  --(6.516,1.615)--(6.525,1.612)--(6.534,1.610)--(6.543,1.609)--(6.552,1.608)--(6.561,1.608)%
  --(6.570,1.610)--(6.579,1.612)--(6.588,1.615)--(6.597,1.619)--(6.606,1.624)--(6.615,1.631)%
  --(6.624,1.638)--(6.633,1.647)--(6.642,1.656)--(6.651,1.667)--(6.660,1.679)--(6.669,1.691)%
  --(6.678,1.705)--(6.687,1.719)--(6.697,1.734)--(6.706,1.749)--(6.715,1.766)--(6.724,1.782)%
  --(6.733,1.800)--(6.742,1.818)--(6.751,1.836)--(6.760,1.855)--(6.769,1.875)--(6.778,1.894)%
  --(6.787,1.914)--(6.796,1.934)--(6.805,1.954)--(6.814,1.975)--(6.823,1.995)--(6.832,2.016)%
  --(6.841,2.036)--(6.850,2.057)--(6.859,2.077)--(6.868,2.098)--(6.877,2.118)--(6.886,2.137)%
  --(6.895,2.157)--(6.904,2.176)--(6.913,2.195)--(6.922,2.213)--(6.931,2.231)--(6.941,2.248)%
  --(6.950,2.265)--(6.959,2.281)--(6.968,2.296)--(6.977,2.311)--(6.986,2.324)--(6.995,2.337)%
  --(7.004,2.349)--(7.013,2.361)--(7.022,2.371)--(7.031,2.380)--(7.040,2.388)--(7.049,2.396)%
  --(7.058,2.402)--(7.067,2.407)--(7.076,2.412)--(7.085,2.415)--(7.094,2.418)--(7.103,2.420)%
  --(7.112,2.421)--(7.121,2.422)--(7.130,2.421)--(7.139,2.420)--(7.148,2.419)--(7.157,2.417)%
  --(7.166,2.414)--(7.175,2.410)--(7.185,2.406)--(7.194,2.401)--(7.203,2.396)--(7.212,2.391)%
  --(7.221,2.385)--(7.230,2.378)--(7.239,2.371)--(7.248,2.364)--(7.257,2.357)--(7.266,2.349)%
  --(7.275,2.341)--(7.284,2.332)--(7.293,2.323)--(7.302,2.315)--(7.311,2.305)--(7.320,2.296)%
  --(7.329,2.287)--(7.338,2.277)--(7.347,2.268)--(7.356,2.258)--(7.365,2.249)--(7.374,2.239)%
  --(7.383,2.230)--(7.392,2.220)--(7.401,2.211)--(7.410,2.202)--(7.419,2.193)--(7.429,2.184)%
  --(7.438,2.175)--(7.447,2.167)--(7.456,2.158)--(7.465,2.150)--(7.474,2.143)--(7.483,2.135)%
  --(7.492,2.128)--(7.501,2.121)--(7.510,2.115)--(7.519,2.109)--(7.528,2.103)--(7.537,2.097)%
  --(7.546,2.092)--(7.555,2.088)--(7.564,2.083)--(7.573,2.079)--(7.582,2.076)--(7.591,2.073)%
  --(7.600,2.071)--(7.609,2.069)--(7.618,2.067)--(7.627,2.066)--(7.636,2.065)--(7.645,2.065)%
  --(7.654,2.066)--(7.663,2.067)--(7.673,2.069)--(7.682,2.071)--(7.691,2.074)--(7.700,2.078)%
  --(7.709,2.082)--(7.718,2.086)--(7.727,2.092)--(7.736,2.098)--(7.745,2.105)--(7.754,2.112)%
  --(7.763,2.121)--(7.772,2.129)--(7.781,2.139)--(7.790,2.150)--(7.799,2.161)--(7.808,2.173)%
  --(7.817,2.186)--(7.826,2.199)--(7.835,2.214)--(7.844,2.229)--(7.853,2.245)--(7.862,2.262)%
  --(7.871,2.280)--(7.880,2.298)--(7.889,2.317)--(7.898,2.337)--(7.907,2.357)--(7.917,2.378)%
  --(7.926,2.399)--(7.935,2.421)--(7.944,2.443)--(7.953,2.465)--(7.962,2.488)--(7.971,2.511)%
  --(7.980,2.534)--(7.989,2.557)--(7.998,2.580)--(8.007,2.604)--(8.016,2.627)--(8.025,2.650)%
  --(8.034,2.674)--(8.043,2.697)--(8.052,2.720)--(8.061,2.742)--(8.070,2.765)--(8.079,2.787)%
  --(8.088,2.809)--(8.097,2.830)--(8.106,2.851)--(8.115,2.871)--(8.124,2.891)--(8.133,2.910)%
  --(8.142,2.929)--(8.151,2.947)--(8.161,2.964)--(8.170,2.980)--(8.179,2.996)--(8.188,3.010)%
  --(8.197,3.024)--(8.206,3.036)--(8.215,3.048)--(8.224,3.059)--(8.233,3.068)--(8.242,3.076)%
  --(8.251,3.083)--(8.260,3.089)--(8.269,3.094)--(8.278,3.097)--(8.287,3.100)--(8.296,3.101)%
  --(8.305,3.101)--(8.314,3.100)--(8.323,3.099)--(8.332,3.096)--(8.341,3.093)--(8.350,3.089)%
  --(8.359,3.084)--(8.368,3.079)--(8.377,3.073)--(8.386,3.067)--(8.395,3.060)--(8.405,3.053)%
  --(8.414,3.046)--(8.423,3.038)--(8.432,3.031)--(8.441,3.023)--(8.450,3.015)--(8.459,3.006)%
  --(8.468,2.998)--(8.477,2.991)--(8.486,2.983)--(8.495,2.975)--(8.504,2.968)--(8.513,2.961)%
  --(8.522,2.954)--(8.531,2.948)--(8.540,2.942)--(8.549,2.937)--(8.558,2.933)--(8.567,2.929)%
  --(8.576,2.926)--(8.585,2.923)--(8.594,2.922)--(8.603,2.921)--(8.612,2.922)--(8.621,2.923)%
  --(8.630,2.925)--(8.639,2.929)--(8.649,2.934)--(8.658,2.940)--(8.667,2.947)--(8.676,2.956)%
  --(8.685,2.966)--(8.694,2.977)--(8.703,2.989)--(8.712,3.002)--(8.721,3.017)--(8.730,3.032)%
  --(8.739,3.049)--(8.748,3.066)--(8.757,3.084)--(8.766,3.103)--(8.775,3.123)--(8.784,3.143)%
  --(8.793,3.164)--(8.802,3.186)--(8.811,3.208)--(8.820,3.230)--(8.829,3.253)--(8.838,3.276)%
  --(8.847,3.300)--(8.856,3.324)--(8.865,3.348)--(8.874,3.372)--(8.883,3.396)--(8.893,3.420)%
  --(8.902,3.445)--(8.911,3.469)--(8.920,3.493)--(8.929,3.517)--(8.938,3.540)--(8.947,3.563)%
  --(8.956,3.586)--(8.965,3.609)--(8.974,3.631)--(8.983,3.652)--(8.992,3.673)--(9.001,3.694)%
  --(9.010,3.713)--(9.019,3.732)--(9.028,3.750)--(9.037,3.768)--(9.046,3.784)--(9.055,3.799)%
  --(9.064,3.814)--(9.073,3.827)--(9.082,3.839)--(9.091,3.850)--(9.100,3.861)--(9.109,3.870)%
  --(9.118,3.879)--(9.127,3.886)--(9.137,3.894)--(9.146,3.900)--(9.155,3.907)--(9.164,3.913)%
  --(9.173,3.918)--(9.182,3.924)--(9.191,3.929)--(9.200,3.934)--(9.209,3.940)--(9.218,3.945)%
  --(9.227,3.951)--(9.236,3.957)--(9.245,3.964)--(9.254,3.971)--(9.263,3.979)--(9.272,3.987)%
  --(9.281,3.996)--(9.290,4.006)--(9.299,4.017)--(9.308,4.029)--(9.317,4.042)--(9.326,4.056)%
  --(9.335,4.071)--(9.344,4.088)--(9.353,4.107)--(9.362,4.126)--(9.371,4.148)--(9.381,4.171)%
  --(9.390,4.196)--(9.399,4.223)--(9.408,4.252)--(9.417,4.282)--(9.426,4.315)--(9.435,4.351)%
  --(9.444,4.388)--(9.453,4.428)--(9.462,4.471)--(9.471,4.516)--(9.480,4.563)--(9.489,4.614)%
  --(9.498,4.667)--(9.507,4.723)--(9.516,4.780)--(9.525,4.841)--(9.534,4.903)--(9.543,4.966)%
  --(9.552,5.031)--(9.561,5.098)--(9.570,5.166)--(9.579,5.234)--(9.588,5.304)--(9.597,5.374)%
  --(9.606,5.444)--(9.615,5.515)--(9.625,5.585)--(9.634,5.655)--(9.643,5.725)--(9.652,5.794)%
  --(9.661,5.863)--(9.670,5.930)--(9.679,5.996)--(9.688,6.061)--(9.697,6.124)--(9.706,6.186)%
  --(9.715,6.245)--(9.724,6.302)--(9.733,6.357)--(9.742,6.409)--(9.751,6.459)--(9.760,6.505)%
  --(9.769,6.549)--(9.778,6.589)--(9.787,6.625)--(9.796,6.658)--(9.805,6.687)--(9.814,6.712)%
  --(9.823,6.732)--(9.832,6.748)--(9.841,6.759)--(9.850,6.765)--(9.859,6.766)--(9.868,6.762)%
  --(9.878,6.752)--(9.887,6.737)--(9.896,6.716)--(9.905,6.689)--(9.914,6.655)--(9.923,6.617)%
  --(9.932,6.573)--(9.941,6.524)--(9.950,6.471)--(9.959,6.413)--(9.968,6.352)--(9.977,6.286)%
  --(9.986,6.217)--(9.995,6.144)--(10.004,6.069)--(10.013,5.991)--(10.022,5.911)--(10.031,5.828)%
  --(10.040,5.744)--(10.049,5.658)--(10.058,5.571)--(10.067,5.483)--(10.076,5.395)--(10.085,5.306)%
  --(10.094,5.216)--(10.103,5.127)--(10.112,5.039)--(10.122,4.951)--(10.131,4.864)--(10.140,4.779)%
  --(10.149,4.695)--(10.158,4.612)--(10.167,4.533)--(10.176,4.455)--(10.185,4.380)--(10.194,4.308)%
  --(10.203,4.240)--(10.212,4.175)--(10.221,4.114)--(10.230,4.057)--(10.239,4.004)--(10.248,3.957)%
  --(10.257,3.914)--(10.266,3.876)--(10.275,3.844)--(10.284,3.818)--(10.293,3.798)--(10.302,3.784)%
  --(10.311,3.777)--(10.320,3.777)--(10.329,3.782)--(10.338,3.794)--(10.347,3.812)--(10.356,3.835)%
  --(10.366,3.864)--(10.375,3.897)--(10.384,3.935)--(10.393,3.978)--(10.402,4.025)--(10.411,4.075)%
  --(10.420,4.129)--(10.429,4.187)--(10.438,4.247)--(10.447,4.310)--(10.456,4.376)--(10.465,4.444)%
  --(10.474,4.514)--(10.483,4.585)--(10.492,4.658)--(10.501,4.732)--(10.510,4.807)--(10.519,4.882)%
  --(10.528,4.958)--(10.537,5.034)--(10.546,5.109)--(10.555,5.184)--(10.564,5.258)--(10.573,5.331)%
  --(10.582,5.402)--(10.591,5.472)--(10.600,5.540)--(10.610,5.606)--(10.619,5.670)--(10.628,5.730)%
  --(10.637,5.788)--(10.646,5.842)--(10.655,5.893)--(10.664,5.940)--(10.673,5.983)--(10.682,6.021)%
  --(10.691,6.055)--(10.700,6.083)--(10.709,6.107)--(10.718,6.125)--(10.727,6.138)--(10.736,6.145)%
  --(10.745,6.146)--(10.754,6.143)--(10.763,6.135)--(10.772,6.121)--(10.781,6.104)--(10.790,6.082)%
  --(10.799,6.056)--(10.808,6.026)--(10.817,5.992)--(10.826,5.955)--(10.835,5.914)--(10.844,5.870)%
  --(10.854,5.823)--(10.863,5.773)--(10.872,5.720)--(10.881,5.665)--(10.890,5.608)--(10.899,5.548)%
  --(10.908,5.487)--(10.917,5.424)--(10.926,5.359)--(10.935,5.293)--(10.944,5.226)--(10.953,5.158)%
  --(10.962,5.090)--(10.971,5.020)--(10.980,4.951)--(10.989,4.881)--(10.998,4.811)--(11.007,4.741)%
  --(11.016,4.671)--(11.025,4.602)--(11.034,4.534)--(11.043,4.467)--(11.052,4.401)--(11.061,4.336)%
  --(11.070,4.272)--(11.079,4.211)--(11.088,4.151)--(11.098,4.093)--(11.107,4.038)--(11.116,3.984)%
  --(11.125,3.934)--(11.134,3.886)--(11.143,3.841)--(11.152,3.799)--(11.161,3.759)--(11.170,3.722)%
  --(11.179,3.687)--(11.188,3.654)--(11.197,3.624)--(11.206,3.595)--(11.215,3.569)--(11.224,3.545)%
  --(11.233,3.522)--(11.242,3.501)--(11.251,3.482)--(11.260,3.464)--(11.269,3.448)--(11.278,3.433)%
  --(11.287,3.420)--(11.296,3.407)--(11.305,3.396)--(11.314,3.386)--(11.323,3.376)--(11.332,3.368)%
  --(11.342,3.360)--(11.351,3.352)--(11.360,3.345)--(11.369,3.339)--(11.378,3.333)--(11.387,3.327)%
  --(11.396,3.322)--(11.405,3.316)--(11.414,3.310)--(11.423,3.305)--(11.432,3.299)--(11.441,3.292)%
  --(11.450,3.285)--(11.459,3.278)--(11.468,3.270)--(11.477,3.262)--(11.486,3.253)--(11.495,3.242)%
  --(11.504,3.231)--(11.513,3.219)--(11.522,3.206)--(11.531,3.191)--(11.540,3.175)--(11.549,3.158)%
  --(11.558,3.139)--(11.567,3.119)--(11.576,3.097)--(11.586,3.075)--(11.595,3.051)--(11.604,3.026)%
  --(11.613,3.000)--(11.622,2.972)--(11.631,2.944)--(11.640,2.914)--(11.649,2.884)--(11.658,2.852)%
  --(11.667,2.820)--(11.676,2.786)--(11.685,2.751)--(11.694,2.716)--(11.703,2.680)--(11.712,2.643)%
  --(11.721,2.605)--(11.730,2.566)--(11.739,2.527)--(11.748,2.487)--(11.757,2.446)--(11.766,2.405)%
  --(11.775,2.363)--(11.784,2.320)--(11.793,2.277)--(11.802,2.233)--(11.811,2.189)--(11.820,2.144)%
  --(11.830,2.099)--(11.839,2.054)--(11.848,2.008)--(11.857,1.962)--(11.866,1.915)--(11.875,1.868)%
  --(11.884,1.821)--(11.893,1.774)--(11.902,1.727)--(11.911,1.679)--(11.920,1.631)--(11.929,1.584)%
  --(11.938,1.536)--(11.947,1.488);
\gpcolor{color=gp lt color border}
\node[gp node right] at (4.448,7.123) {w/o types};
\gpfill{color=gp lt color 1,opacity=0.10} (4.632,7.046)--(5.548,7.046)--(5.548,7.200)--(4.632,7.200)--cycle;
\gpcolor{color=gp lt color 1}
\draw[gp path] (4.632,7.046)--(5.548,7.046)--(5.548,7.200)--(4.632,7.200)--cycle;
\gpfill{color=gp lt color 1,opacity=0.10} (1.688,1.063)--(1.688,1.063)--(1.698,1.065)--(1.709,1.067)%
    --(1.719,1.069)--(1.729,1.071)--(1.739,1.073)--(1.750,1.076)--(1.760,1.078)%
    --(1.770,1.080)--(1.780,1.082)--(1.791,1.084)--(1.801,1.086)--(1.811,1.088)%
    --(1.822,1.090)--(1.832,1.092)--(1.842,1.095)--(1.852,1.097)--(1.863,1.099)%
    --(1.873,1.101)--(1.883,1.103)--(1.893,1.105)--(1.904,1.107)--(1.914,1.109)%
    --(1.924,1.111)--(1.934,1.113)--(1.945,1.115)--(1.955,1.117)--(1.965,1.119)%
    --(1.976,1.121)--(1.986,1.122)--(1.996,1.124)--(2.006,1.126)--(2.017,1.128)%
    --(2.027,1.130)--(2.037,1.132)--(2.047,1.133)--(2.058,1.135)--(2.068,1.137)%
    --(2.078,1.139)--(2.089,1.140)--(2.099,1.142)--(2.109,1.144)--(2.119,1.145)%
    --(2.130,1.147)--(2.140,1.148)--(2.150,1.150)--(2.160,1.151)--(2.171,1.153)%
    --(2.181,1.154)--(2.191,1.156)--(2.201,1.157)--(2.212,1.158)--(2.222,1.160)%
    --(2.232,1.161)--(2.243,1.162)--(2.253,1.164)--(2.263,1.165)--(2.273,1.166)%
    --(2.284,1.167)--(2.294,1.168)--(2.304,1.169)--(2.314,1.170)--(2.325,1.171)%
    --(2.335,1.172)--(2.345,1.173)--(2.356,1.174)--(2.366,1.175)--(2.376,1.175)%
    --(2.386,1.176)--(2.397,1.177)--(2.407,1.177)--(2.417,1.178)--(2.427,1.178)%
    --(2.438,1.179)--(2.448,1.179)--(2.458,1.180)--(2.468,1.180)--(2.479,1.181)%
    --(2.489,1.181)--(2.499,1.181)--(2.510,1.181)--(2.520,1.181)--(2.530,1.181)%
    --(2.540,1.181)--(2.551,1.181)--(2.561,1.181)--(2.571,1.181)--(2.581,1.181)%
    --(2.592,1.181)--(2.602,1.180)--(2.612,1.180)--(2.623,1.180)--(2.633,1.179)%
    --(2.643,1.179)--(2.653,1.178)--(2.664,1.177)--(2.674,1.177)--(2.684,1.176)%
    --(2.694,1.175)--(2.705,1.174)--(2.715,1.173)--(2.725,1.172)--(2.735,1.171)%
    --(2.746,1.170)--(2.756,1.169)--(2.766,1.167)--(2.777,1.166)--(2.787,1.165)%
    --(2.797,1.163)--(2.807,1.162)--(2.818,1.160)--(2.828,1.158)--(2.838,1.157)%
    --(2.848,1.155)--(2.859,1.153)--(2.869,1.151)--(2.879,1.149)--(2.890,1.147)%
    --(2.900,1.145)--(2.910,1.142)--(2.920,1.140)--(2.931,1.138)--(2.941,1.135)%
    --(2.951,1.133)--(2.961,1.130)--(2.972,1.127)--(2.982,1.125)--(2.992,1.122)%
    --(3.002,1.119)--(3.013,1.116)--(3.023,1.113)--(3.033,1.110)--(3.044,1.107)%
    --(3.054,1.104)--(3.064,1.101)--(3.074,1.098)--(3.085,1.095)--(3.095,1.092)%
    --(3.105,1.089)--(3.115,1.086)--(3.126,1.083)--(3.136,1.080)--(3.146,1.076)%
    --(3.157,1.073)--(3.167,1.070)--(3.177,1.067)--(3.187,1.064)--(3.198,1.061)%
    --(3.208,1.058)--(3.218,1.055)--(3.228,1.052)--(3.239,1.049)--(3.249,1.046)%
    --(3.259,1.044)--(3.269,1.041)--(3.280,1.038)--(3.290,1.035)--(3.300,1.033)%
    --(3.311,1.030)--(3.321,1.028)--(3.331,1.025)--(3.341,1.023)--(3.352,1.021)%
    --(3.362,1.019)--(3.372,1.016)--(3.382,1.014)--(3.393,1.012)--(3.403,1.010)%
    --(3.413,1.009)--(3.424,1.007)--(3.434,1.005)--(3.444,1.003)--(3.454,1.002)%
    --(3.465,1.000)--(3.475,0.999)--(3.485,0.998)--(3.495,0.996)--(3.506,0.995)%
    --(3.516,0.994)--(3.526,0.993)--(3.536,0.991)--(3.547,0.990)--(3.557,0.989)%
    --(3.567,0.988)--(3.578,0.988)--(3.588,0.987)--(3.598,0.986)--(3.608,0.985)%
    --(3.613,0.985)--(3.613,0.985)--(1.688,0.985)--cycle;
\gpfill{color=gp lt color 1,opacity=0.10} (3.949,0.985)--(3.958,0.985)--(3.968,0.986)--(3.978,0.987)%
    --(3.988,0.987)--(3.999,0.988)--(4.009,0.988)--(4.019,0.989)--(4.029,0.990)%
    --(4.040,0.990)--(4.050,0.991)--(4.060,0.992)--(4.070,0.992)--(4.081,0.993)%
    --(4.091,0.994)--(4.101,0.995)--(4.112,0.995)--(4.122,0.996)--(4.132,0.997)%
    --(4.142,0.998)--(4.153,0.999)--(4.163,0.999)--(4.173,1.000)--(4.183,1.001)%
    --(4.194,1.002)--(4.204,1.003)--(4.214,1.003)--(4.225,1.004)--(4.235,1.005)%
    --(4.245,1.006)--(4.255,1.007)--(4.266,1.008)--(4.276,1.008)--(4.286,1.009)%
    --(4.296,1.010)--(4.307,1.011)--(4.317,1.011)--(4.327,1.012)--(4.337,1.013)%
    --(4.348,1.014)--(4.358,1.015)--(4.368,1.015)--(4.379,1.016)--(4.389,1.017)%
    --(4.399,1.017)--(4.409,1.018)--(4.420,1.019)--(4.430,1.019)--(4.440,1.020)%
    --(4.450,1.021)--(4.461,1.021)--(4.471,1.022)--(4.481,1.022)--(4.492,1.023)%
    --(4.502,1.023)--(4.512,1.024)--(4.522,1.024)--(4.533,1.025)--(4.543,1.025)%
    --(4.553,1.025)--(4.563,1.026)--(4.574,1.026)--(4.584,1.026)--(4.594,1.027)%
    --(4.604,1.027)--(4.615,1.027)--(4.625,1.027)--(4.635,1.027)--(4.646,1.028)%
    --(4.656,1.028)--(4.666,1.028)--(4.676,1.028)--(4.687,1.028)--(4.697,1.028)%
    --(4.707,1.028)--(4.717,1.028)--(4.728,1.028)--(4.738,1.028)--(4.748,1.028)%
    --(4.759,1.028)--(4.769,1.028)--(4.779,1.028)--(4.789,1.028)--(4.800,1.028)%
    --(4.810,1.028)--(4.820,1.028)--(4.830,1.027)--(4.841,1.027)--(4.851,1.027)%
    --(4.861,1.027)--(4.871,1.027)--(4.882,1.027)--(4.892,1.027)--(4.902,1.027)%
    --(4.913,1.026)--(4.923,1.026)--(4.933,1.026)--(4.943,1.026)--(4.954,1.026)%
    --(4.964,1.026)--(4.974,1.026)--(4.984,1.026)--(4.995,1.025)--(5.005,1.025)%
    --(5.015,1.025)--(5.026,1.025)--(5.036,1.025)--(5.046,1.025)--(5.056,1.025)%
    --(5.067,1.025)--(5.077,1.025)--(5.087,1.025)--(5.097,1.025)--(5.108,1.025)%
    --(5.118,1.025)--(5.128,1.025)--(5.138,1.025)--(5.149,1.025)--(5.159,1.025)%
    --(5.169,1.025)--(5.180,1.025)--(5.190,1.025)--(5.200,1.025)--(5.210,1.025)%
    --(5.221,1.025)--(5.231,1.025)--(5.241,1.025)--(5.251,1.025)--(5.262,1.025)%
    --(5.272,1.025)--(5.282,1.025)--(5.293,1.025)--(5.303,1.025)--(5.313,1.025)%
    --(5.323,1.025)--(5.334,1.025)--(5.344,1.025)--(5.354,1.025)--(5.364,1.025)%
    --(5.375,1.026)--(5.385,1.026)--(5.395,1.026)--(5.405,1.026)--(5.416,1.026)%
    --(5.426,1.026)--(5.436,1.026)--(5.447,1.027)--(5.457,1.027)--(5.467,1.027)%
    --(5.477,1.027)--(5.488,1.027)--(5.498,1.027)--(5.508,1.028)--(5.518,1.028)%
    --(5.529,1.028)--(5.539,1.028)--(5.549,1.029)--(5.560,1.029)--(5.570,1.029)%
    --(5.580,1.029)--(5.590,1.030)--(5.601,1.030)--(5.611,1.030)--(5.621,1.030)%
    --(5.631,1.031)--(5.642,1.031)--(5.652,1.031)--(5.662,1.032)--(5.672,1.032)%
    --(5.683,1.032)--(5.693,1.033)--(5.703,1.033)--(5.714,1.033)--(5.724,1.034)%
    --(5.734,1.034)--(5.744,1.034)--(5.755,1.035)--(5.765,1.035)--(5.775,1.036)%
    --(5.785,1.036)--(5.796,1.037)--(5.806,1.037)--(5.816,1.037)--(5.827,1.038)%
    --(5.837,1.038)--(5.847,1.039)--(5.857,1.039)--(5.868,1.040)--(5.878,1.040)%
    --(5.888,1.041)--(5.898,1.042)--(5.909,1.042)--(5.919,1.043)--(5.929,1.043)%
    --(5.939,1.044)--(5.950,1.044)--(5.960,1.045)--(5.970,1.046)--(5.981,1.046)%
    --(5.991,1.047)--(6.001,1.048)--(6.011,1.048)--(6.022,1.049)--(6.032,1.050)%
    --(6.042,1.050)--(6.052,1.051)--(6.063,1.052)--(6.073,1.052)--(6.083,1.053)%
    --(6.094,1.054)--(6.104,1.055)--(6.114,1.055)--(6.124,1.056)--(6.135,1.057)%
    --(6.145,1.058)--(6.155,1.059)--(6.165,1.060)--(6.176,1.060)--(6.186,1.061)%
    --(6.196,1.062)--(6.206,1.063)--(6.217,1.064)--(6.227,1.065)--(6.237,1.066)%
    --(6.248,1.067)--(6.258,1.068)--(6.268,1.069)--(6.278,1.070)--(6.289,1.071)%
    --(6.299,1.072)--(6.309,1.073)--(6.319,1.074)--(6.330,1.075)--(6.340,1.076)%
    --(6.350,1.077)--(6.361,1.078)--(6.371,1.079)--(6.381,1.080)--(6.391,1.081)%
    --(6.402,1.082)--(6.412,1.083)--(6.422,1.084)--(6.432,1.085)--(6.443,1.086)%
    --(6.453,1.087)--(6.463,1.088)--(6.473,1.089)--(6.484,1.090)--(6.494,1.091)%
    --(6.504,1.093)--(6.515,1.094)--(6.525,1.095)--(6.535,1.096)--(6.545,1.097)%
    --(6.556,1.098)--(6.566,1.099)--(6.576,1.100)--(6.586,1.101)--(6.597,1.102)%
    --(6.607,1.103)--(6.617,1.104)--(6.628,1.105)--(6.638,1.106)--(6.648,1.107)%
    --(6.658,1.108)--(6.669,1.109)--(6.679,1.109)--(6.689,1.110)--(6.699,1.111)%
    --(6.710,1.112)--(6.720,1.113)--(6.730,1.113)--(6.740,1.114)--(6.751,1.115)%
    --(6.761,1.115)--(6.771,1.116)--(6.782,1.116)--(6.792,1.117)--(6.802,1.117)%
    --(6.812,1.117)--(6.823,1.117)--(6.833,1.118)--(6.843,1.118)--(6.853,1.118)%
    --(6.864,1.118)--(6.874,1.117)--(6.884,1.117)--(6.895,1.117)--(6.905,1.116)%
    --(6.915,1.116)--(6.925,1.115)--(6.936,1.114)--(6.946,1.113)--(6.956,1.112)%
    --(6.966,1.111)--(6.977,1.110)--(6.987,1.109)--(6.997,1.107)--(7.007,1.106)%
    --(7.018,1.104)--(7.028,1.102)--(7.038,1.100)--(7.049,1.098)--(7.059,1.096)%
    --(7.069,1.094)--(7.079,1.092)--(7.090,1.089)--(7.100,1.087)--(7.110,1.084)%
    --(7.120,1.081)--(7.131,1.079)--(7.141,1.076)--(7.151,1.073)--(7.162,1.071)%
    --(7.172,1.068)--(7.182,1.065)--(7.192,1.062)--(7.203,1.060)--(7.213,1.057)%
    --(7.223,1.054)--(7.233,1.052)--(7.244,1.049)--(7.254,1.047)--(7.264,1.045)%
    --(7.274,1.042)--(7.285,1.040)--(7.295,1.038)--(7.305,1.036)--(7.316,1.034)%
    --(7.326,1.033)--(7.336,1.031)--(7.346,1.030)--(7.357,1.028)--(7.367,1.027)%
    --(7.377,1.027)--(7.387,1.026)--(7.398,1.026)--(7.408,1.025)--(7.418,1.025)%
    --(7.429,1.026)--(7.439,1.026)--(7.449,1.027)--(7.459,1.028)--(7.470,1.029)%
    --(7.480,1.030)--(7.490,1.032)--(7.500,1.033)--(7.511,1.035)--(7.521,1.038)%
    --(7.531,1.040)--(7.541,1.042)--(7.552,1.045)--(7.562,1.048)--(7.572,1.050)%
    --(7.583,1.053)--(7.593,1.057)--(7.603,1.060)--(7.613,1.063)--(7.624,1.066)%
    --(7.634,1.070)--(7.644,1.073)--(7.654,1.077)--(7.665,1.080)--(7.675,1.084)%
    --(7.685,1.088)--(7.696,1.091)--(7.706,1.095)--(7.716,1.099)--(7.726,1.102)%
    --(7.737,1.106)--(7.747,1.110)--(7.757,1.113)--(7.767,1.117)--(7.778,1.120)%
    --(7.788,1.124)--(7.798,1.127)--(7.808,1.130)--(7.819,1.133)--(7.829,1.136)%
    --(7.839,1.139)--(7.850,1.142)--(7.860,1.145)--(7.870,1.147)--(7.880,1.150)%
    --(7.891,1.152)--(7.901,1.154)--(7.911,1.156)--(7.921,1.159)--(7.932,1.161)%
    --(7.942,1.163)--(7.952,1.165)--(7.963,1.167)--(7.973,1.169)--(7.983,1.171)%
    --(7.993,1.173)--(8.004,1.175)--(8.014,1.177)--(8.024,1.179)--(8.034,1.181)%
    --(8.045,1.183)--(8.055,1.186)--(8.065,1.188)--(8.075,1.191)--(8.086,1.193)%
    --(8.096,1.196)--(8.106,1.199)--(8.117,1.202)--(8.127,1.205)--(8.137,1.208)%
    --(8.147,1.211)--(8.158,1.215)--(8.168,1.219)--(8.178,1.223)--(8.188,1.227)%
    --(8.199,1.231)--(8.209,1.236)--(8.219,1.241)--(8.230,1.246)--(8.240,1.251)%
    --(8.250,1.257)--(8.260,1.262)--(8.271,1.269)--(8.281,1.275)--(8.291,1.282)%
    --(8.301,1.289)--(8.312,1.296)--(8.322,1.304)--(8.332,1.312)--(8.342,1.320)%
    --(8.353,1.329)--(8.363,1.338)--(8.373,1.347)--(8.384,1.357)--(8.394,1.367)%
    --(8.404,1.377)--(8.414,1.388)--(8.425,1.399)--(8.435,1.410)--(8.445,1.422)%
    --(8.455,1.434)--(8.466,1.446)--(8.476,1.459)--(8.486,1.473)--(8.497,1.486)%
    --(8.507,1.500)--(8.517,1.515)--(8.527,1.530)--(8.538,1.545)--(8.548,1.561)%
    --(8.558,1.577)--(8.568,1.593)--(8.579,1.610)--(8.589,1.628)--(8.599,1.646)%
    --(8.609,1.664)--(8.620,1.683)--(8.630,1.702)--(8.640,1.722)--(8.651,1.742)%
    --(8.661,1.762)--(8.671,1.783)--(8.681,1.805)--(8.692,1.827)--(8.702,1.849)%
    --(8.712,1.872)--(8.722,1.895)--(8.733,1.918)--(8.743,1.941)--(8.753,1.965)%
    --(8.764,1.989)--(8.774,2.012)--(8.784,2.036)--(8.794,2.060)--(8.805,2.084)%
    --(8.815,2.107)--(8.825,2.131)--(8.835,2.154)--(8.846,2.177)--(8.856,2.200)%
    --(8.866,2.223)--(8.876,2.245)--(8.887,2.267)--(8.897,2.289)--(8.907,2.310)%
    --(8.918,2.330)--(8.928,2.350)--(8.938,2.370)--(8.948,2.388)--(8.959,2.407)%
    --(8.969,2.424)--(8.979,2.441)--(8.989,2.457)--(9.000,2.472)--(9.010,2.486)%
    --(9.020,2.499)--(9.031,2.512)--(9.041,2.523)--(9.051,2.533)--(9.061,2.543)%
    --(9.072,2.551)--(9.082,2.558)--(9.092,2.564)--(9.102,2.569)--(9.113,2.572)%
    --(9.123,2.575)--(9.133,2.577)--(9.143,2.578)--(9.154,2.578)--(9.164,2.577)%
    --(9.174,2.576)--(9.185,2.573)--(9.195,2.570)--(9.205,2.567)--(9.215,2.563)%
    --(9.226,2.558)--(9.236,2.553)--(9.246,2.547)--(9.256,2.541)--(9.267,2.534)%
    --(9.277,2.528)--(9.287,2.521)--(9.298,2.513)--(9.308,2.506)--(9.318,2.498)%
    --(9.328,2.491)--(9.339,2.483)--(9.349,2.475)--(9.359,2.467)--(9.369,2.460)%
    --(9.380,2.452)--(9.390,2.445)--(9.400,2.438)--(9.410,2.431)--(9.421,2.425)%
    --(9.431,2.418)--(9.441,2.413)--(9.452,2.407)--(9.462,2.403)--(9.472,2.398)%
    --(9.482,2.395)--(9.493,2.392)--(9.503,2.389)--(9.513,2.388)--(9.523,2.387)%
    --(9.534,2.387)--(9.544,2.388)--(9.554,2.389)--(9.565,2.392)--(9.575,2.396)%
    --(9.585,2.400)--(9.595,2.406)--(9.606,2.413)--(9.616,2.421)--(9.626,2.430)%
    --(9.636,2.441)--(9.647,2.452)--(9.657,2.466)--(9.667,2.480)--(9.677,2.496)%
    --(9.688,2.514)--(9.698,2.533)--(9.708,2.553)--(9.719,2.575)--(9.729,2.599)%
    --(9.739,2.625)--(9.749,2.652)--(9.760,2.681)--(9.770,2.712)--(9.780,2.745)%
    --(9.790,2.780)--(9.801,2.817)--(9.811,2.855)--(9.821,2.896)--(9.832,2.939)%
    --(9.842,2.984)--(9.852,3.032)--(9.862,3.081)--(9.873,3.133)--(9.883,3.187)%
    --(9.893,3.244)--(9.903,3.303)--(9.914,3.364)--(9.924,3.428)--(9.934,3.493)%
    --(9.944,3.561)--(9.955,3.630)--(9.965,3.701)--(9.975,3.773)--(9.986,3.846)%
    --(9.996,3.921)--(10.006,3.997)--(10.016,4.074)--(10.027,4.151)--(10.037,4.229)%
    --(10.047,4.307)--(10.057,4.386)--(10.068,4.465)--(10.078,4.544)--(10.088,4.623)%
    --(10.099,4.701)--(10.109,4.779)--(10.119,4.857)--(10.129,4.933)--(10.140,5.009)%
    --(10.150,5.084)--(10.160,5.158)--(10.170,5.231)--(10.181,5.302)--(10.191,5.372)%
    --(10.201,5.440)--(10.211,5.506)--(10.222,5.570)--(10.232,5.632)--(10.242,5.691)%
    --(10.253,5.748)--(10.263,5.803)--(10.273,5.855)--(10.283,5.904)--(10.294,5.950)%
    --(10.304,5.993)--(10.314,6.033)--(10.324,6.069)--(10.335,6.102)--(10.345,6.133)%
    --(10.355,6.160)--(10.366,6.185)--(10.376,6.207)--(10.386,6.226)--(10.396,6.243)%
    --(10.407,6.258)--(10.417,6.271)--(10.427,6.281)--(10.437,6.290)--(10.448,6.296)%
    --(10.458,6.301)--(10.468,6.305)--(10.478,6.307)--(10.489,6.307)--(10.499,6.306)%
    --(10.509,6.304)--(10.520,6.301)--(10.530,6.297)--(10.540,6.292)--(10.550,6.287)%
    --(10.561,6.281)--(10.571,6.274)--(10.581,6.267)--(10.591,6.260)--(10.602,6.253)%
    --(10.612,6.245)--(10.622,6.238)--(10.633,6.231)--(10.643,6.224)--(10.653,6.218)%
    --(10.663,6.212)--(10.674,6.207)--(10.684,6.203)--(10.694,6.199)--(10.704,6.197)%
    --(10.715,6.196)--(10.725,6.195)--(10.735,6.197)--(10.745,6.199)--(10.756,6.202)%
    --(10.766,6.206)--(10.776,6.211)--(10.787,6.217)--(10.797,6.223)--(10.807,6.230)%
    --(10.817,6.238)--(10.828,6.246)--(10.838,6.254)--(10.848,6.263)--(10.858,6.272)%
    --(10.869,6.281)--(10.879,6.290)--(10.889,6.299)--(10.900,6.308)--(10.910,6.317)%
    --(10.920,6.326)--(10.930,6.334)--(10.941,6.341)--(10.951,6.349)--(10.961,6.355)%
    --(10.971,6.361)--(10.982,6.366)--(10.992,6.371)--(11.002,6.374)--(11.012,6.377)%
    --(11.023,6.378)--(11.033,6.378)--(11.043,6.377)--(11.054,6.375)--(11.064,6.371)%
    --(11.074,6.366)--(11.084,6.359)--(11.095,6.351)--(11.105,6.341)--(11.115,6.329)%
    --(11.125,6.315)--(11.136,6.299)--(11.146,6.282)--(11.156,6.263)--(11.167,6.242)%
    --(11.177,6.220)--(11.187,6.197)--(11.197,6.173)--(11.208,6.148)--(11.218,6.123)%
    --(11.228,6.097)--(11.238,6.071)--(11.249,6.044)--(11.259,6.018)--(11.269,5.992)%
    --(11.279,5.967)--(11.290,5.942)--(11.300,5.917)--(11.310,5.894)--(11.321,5.872)%
    --(11.331,5.851)--(11.341,5.831)--(11.351,5.813)--(11.362,5.797)--(11.372,5.783)%
    --(11.382,5.771)--(11.392,5.761)--(11.403,5.754)--(11.413,5.749)--(11.423,5.747)%
    --(11.434,5.748)--(11.444,5.752)--(11.454,5.759)--(11.464,5.770)--(11.475,5.784)%
    --(11.485,5.803)--(11.495,5.825)--(11.505,5.851)--(11.516,5.882)--(11.526,5.917)%
    --(11.536,5.957)--(11.546,6.002)--(11.557,6.051)--(11.567,6.105)--(11.577,6.164)%
    --(11.588,6.226)--(11.598,6.293)--(11.608,6.365)--(11.618,6.440)--(11.629,6.519)%
    --(11.639,6.602)--(11.649,6.688)--(11.659,6.778)--(11.670,6.871)--(11.680,6.968)%
    --(11.690,7.068)--(11.701,7.170)--(11.711,7.276)--(11.721,7.385)--(11.731,7.496)%
    --(11.742,7.609)--(11.752,7.725)--(11.762,7.844)--(11.772,7.964)--(11.783,8.087)%
    --(11.793,8.212)--(11.803,8.338)--(11.807,8.381)--(11.807,0.985)--cycle;
\draw[gp path] (1.688,1.063)--(1.698,1.065)--(1.709,1.067)--(1.719,1.069)--(1.729,1.071)%
  --(1.739,1.073)--(1.750,1.076)--(1.760,1.078)--(1.770,1.080)--(1.780,1.082)--(1.791,1.084)%
  --(1.801,1.086)--(1.811,1.088)--(1.822,1.090)--(1.832,1.092)--(1.842,1.095)--(1.852,1.097)%
  --(1.863,1.099)--(1.873,1.101)--(1.883,1.103)--(1.893,1.105)--(1.904,1.107)--(1.914,1.109)%
  --(1.924,1.111)--(1.934,1.113)--(1.945,1.115)--(1.955,1.117)--(1.965,1.119)--(1.976,1.121)%
  --(1.986,1.122)--(1.996,1.124)--(2.006,1.126)--(2.017,1.128)--(2.027,1.130)--(2.037,1.132)%
  --(2.047,1.133)--(2.058,1.135)--(2.068,1.137)--(2.078,1.139)--(2.089,1.140)--(2.099,1.142)%
  --(2.109,1.144)--(2.119,1.145)--(2.130,1.147)--(2.140,1.148)--(2.150,1.150)--(2.160,1.151)%
  --(2.171,1.153)--(2.181,1.154)--(2.191,1.156)--(2.201,1.157)--(2.212,1.158)--(2.222,1.160)%
  --(2.232,1.161)--(2.243,1.162)--(2.253,1.164)--(2.263,1.165)--(2.273,1.166)--(2.284,1.167)%
  --(2.294,1.168)--(2.304,1.169)--(2.314,1.170)--(2.325,1.171)--(2.335,1.172)--(2.345,1.173)%
  --(2.356,1.174)--(2.366,1.175)--(2.376,1.175)--(2.386,1.176)--(2.397,1.177)--(2.407,1.177)%
  --(2.417,1.178)--(2.427,1.178)--(2.438,1.179)--(2.448,1.179)--(2.458,1.180)--(2.468,1.180)%
  --(2.479,1.181)--(2.489,1.181)--(2.499,1.181)--(2.510,1.181)--(2.520,1.181)--(2.530,1.181)%
  --(2.540,1.181)--(2.551,1.181)--(2.561,1.181)--(2.571,1.181)--(2.581,1.181)--(2.592,1.181)%
  --(2.602,1.180)--(2.612,1.180)--(2.623,1.180)--(2.633,1.179)--(2.643,1.179)--(2.653,1.178)%
  --(2.664,1.177)--(2.674,1.177)--(2.684,1.176)--(2.694,1.175)--(2.705,1.174)--(2.715,1.173)%
  --(2.725,1.172)--(2.735,1.171)--(2.746,1.170)--(2.756,1.169)--(2.766,1.167)--(2.777,1.166)%
  --(2.787,1.165)--(2.797,1.163)--(2.807,1.162)--(2.818,1.160)--(2.828,1.158)--(2.838,1.157)%
  --(2.848,1.155)--(2.859,1.153)--(2.869,1.151)--(2.879,1.149)--(2.890,1.147)--(2.900,1.145)%
  --(2.910,1.142)--(2.920,1.140)--(2.931,1.138)--(2.941,1.135)--(2.951,1.133)--(2.961,1.130)%
  --(2.972,1.127)--(2.982,1.125)--(2.992,1.122)--(3.002,1.119)--(3.013,1.116)--(3.023,1.113)%
  --(3.033,1.110)--(3.044,1.107)--(3.054,1.104)--(3.064,1.101)--(3.074,1.098)--(3.085,1.095)%
  --(3.095,1.092)--(3.105,1.089)--(3.115,1.086)--(3.126,1.083)--(3.136,1.080)--(3.146,1.076)%
  --(3.157,1.073)--(3.167,1.070)--(3.177,1.067)--(3.187,1.064)--(3.198,1.061)--(3.208,1.058)%
  --(3.218,1.055)--(3.228,1.052)--(3.239,1.049)--(3.249,1.046)--(3.259,1.044)--(3.269,1.041)%
  --(3.280,1.038)--(3.290,1.035)--(3.300,1.033)--(3.311,1.030)--(3.321,1.028)--(3.331,1.025)%
  --(3.341,1.023)--(3.352,1.021)--(3.362,1.019)--(3.372,1.016)--(3.382,1.014)--(3.393,1.012)%
  --(3.403,1.010)--(3.413,1.009)--(3.424,1.007)--(3.434,1.005)--(3.444,1.003)--(3.454,1.002)%
  --(3.465,1.000)--(3.475,0.999)--(3.485,0.998)--(3.495,0.996)--(3.506,0.995)--(3.516,0.994)%
  --(3.526,0.993)--(3.536,0.991)--(3.547,0.990)--(3.557,0.989)--(3.567,0.988)--(3.578,0.988)%
  --(3.588,0.987)--(3.598,0.986)--(3.608,0.985)--(3.613,0.985);
\draw[gp path] (3.949,0.985)--(3.958,0.985)--(3.968,0.986)--(3.978,0.987)--(3.988,0.987)%
  --(3.999,0.988)--(4.009,0.988)--(4.019,0.989)--(4.029,0.990)--(4.040,0.990)--(4.050,0.991)%
  --(4.060,0.992)--(4.070,0.992)--(4.081,0.993)--(4.091,0.994)--(4.101,0.995)--(4.112,0.995)%
  --(4.122,0.996)--(4.132,0.997)--(4.142,0.998)--(4.153,0.999)--(4.163,0.999)--(4.173,1.000)%
  --(4.183,1.001)--(4.194,1.002)--(4.204,1.003)--(4.214,1.003)--(4.225,1.004)--(4.235,1.005)%
  --(4.245,1.006)--(4.255,1.007)--(4.266,1.008)--(4.276,1.008)--(4.286,1.009)--(4.296,1.010)%
  --(4.307,1.011)--(4.317,1.011)--(4.327,1.012)--(4.337,1.013)--(4.348,1.014)--(4.358,1.015)%
  --(4.368,1.015)--(4.379,1.016)--(4.389,1.017)--(4.399,1.017)--(4.409,1.018)--(4.420,1.019)%
  --(4.430,1.019)--(4.440,1.020)--(4.450,1.021)--(4.461,1.021)--(4.471,1.022)--(4.481,1.022)%
  --(4.492,1.023)--(4.502,1.023)--(4.512,1.024)--(4.522,1.024)--(4.533,1.025)--(4.543,1.025)%
  --(4.553,1.025)--(4.563,1.026)--(4.574,1.026)--(4.584,1.026)--(4.594,1.027)--(4.604,1.027)%
  --(4.615,1.027)--(4.625,1.027)--(4.635,1.027)--(4.646,1.028)--(4.656,1.028)--(4.666,1.028)%
  --(4.676,1.028)--(4.687,1.028)--(4.697,1.028)--(4.707,1.028)--(4.717,1.028)--(4.728,1.028)%
  --(4.738,1.028)--(4.748,1.028)--(4.759,1.028)--(4.769,1.028)--(4.779,1.028)--(4.789,1.028)%
  --(4.800,1.028)--(4.810,1.028)--(4.820,1.028)--(4.830,1.027)--(4.841,1.027)--(4.851,1.027)%
  --(4.861,1.027)--(4.871,1.027)--(4.882,1.027)--(4.892,1.027)--(4.902,1.027)--(4.913,1.026)%
  --(4.923,1.026)--(4.933,1.026)--(4.943,1.026)--(4.954,1.026)--(4.964,1.026)--(4.974,1.026)%
  --(4.984,1.026)--(4.995,1.025)--(5.005,1.025)--(5.015,1.025)--(5.026,1.025)--(5.036,1.025)%
  --(5.046,1.025)--(5.056,1.025)--(5.067,1.025)--(5.077,1.025)--(5.087,1.025)--(5.097,1.025)%
  --(5.108,1.025)--(5.118,1.025)--(5.128,1.025)--(5.138,1.025)--(5.149,1.025)--(5.159,1.025)%
  --(5.169,1.025)--(5.180,1.025)--(5.190,1.025)--(5.200,1.025)--(5.210,1.025)--(5.221,1.025)%
  --(5.231,1.025)--(5.241,1.025)--(5.251,1.025)--(5.262,1.025)--(5.272,1.025)--(5.282,1.025)%
  --(5.293,1.025)--(5.303,1.025)--(5.313,1.025)--(5.323,1.025)--(5.334,1.025)--(5.344,1.025)%
  --(5.354,1.025)--(5.364,1.025)--(5.375,1.026)--(5.385,1.026)--(5.395,1.026)--(5.405,1.026)%
  --(5.416,1.026)--(5.426,1.026)--(5.436,1.026)--(5.447,1.027)--(5.457,1.027)--(5.467,1.027)%
  --(5.477,1.027)--(5.488,1.027)--(5.498,1.027)--(5.508,1.028)--(5.518,1.028)--(5.529,1.028)%
  --(5.539,1.028)--(5.549,1.029)--(5.560,1.029)--(5.570,1.029)--(5.580,1.029)--(5.590,1.030)%
  --(5.601,1.030)--(5.611,1.030)--(5.621,1.030)--(5.631,1.031)--(5.642,1.031)--(5.652,1.031)%
  --(5.662,1.032)--(5.672,1.032)--(5.683,1.032)--(5.693,1.033)--(5.703,1.033)--(5.714,1.033)%
  --(5.724,1.034)--(5.734,1.034)--(5.744,1.034)--(5.755,1.035)--(5.765,1.035)--(5.775,1.036)%
  --(5.785,1.036)--(5.796,1.037)--(5.806,1.037)--(5.816,1.037)--(5.827,1.038)--(5.837,1.038)%
  --(5.847,1.039)--(5.857,1.039)--(5.868,1.040)--(5.878,1.040)--(5.888,1.041)--(5.898,1.042)%
  --(5.909,1.042)--(5.919,1.043)--(5.929,1.043)--(5.939,1.044)--(5.950,1.044)--(5.960,1.045)%
  --(5.970,1.046)--(5.981,1.046)--(5.991,1.047)--(6.001,1.048)--(6.011,1.048)--(6.022,1.049)%
  --(6.032,1.050)--(6.042,1.050)--(6.052,1.051)--(6.063,1.052)--(6.073,1.052)--(6.083,1.053)%
  --(6.094,1.054)--(6.104,1.055)--(6.114,1.055)--(6.124,1.056)--(6.135,1.057)--(6.145,1.058)%
  --(6.155,1.059)--(6.165,1.060)--(6.176,1.060)--(6.186,1.061)--(6.196,1.062)--(6.206,1.063)%
  --(6.217,1.064)--(6.227,1.065)--(6.237,1.066)--(6.248,1.067)--(6.258,1.068)--(6.268,1.069)%
  --(6.278,1.070)--(6.289,1.071)--(6.299,1.072)--(6.309,1.073)--(6.319,1.074)--(6.330,1.075)%
  --(6.340,1.076)--(6.350,1.077)--(6.361,1.078)--(6.371,1.079)--(6.381,1.080)--(6.391,1.081)%
  --(6.402,1.082)--(6.412,1.083)--(6.422,1.084)--(6.432,1.085)--(6.443,1.086)--(6.453,1.087)%
  --(6.463,1.088)--(6.473,1.089)--(6.484,1.090)--(6.494,1.091)--(6.504,1.093)--(6.515,1.094)%
  --(6.525,1.095)--(6.535,1.096)--(6.545,1.097)--(6.556,1.098)--(6.566,1.099)--(6.576,1.100)%
  --(6.586,1.101)--(6.597,1.102)--(6.607,1.103)--(6.617,1.104)--(6.628,1.105)--(6.638,1.106)%
  --(6.648,1.107)--(6.658,1.108)--(6.669,1.109)--(6.679,1.109)--(6.689,1.110)--(6.699,1.111)%
  --(6.710,1.112)--(6.720,1.113)--(6.730,1.113)--(6.740,1.114)--(6.751,1.115)--(6.761,1.115)%
  --(6.771,1.116)--(6.782,1.116)--(6.792,1.117)--(6.802,1.117)--(6.812,1.117)--(6.823,1.117)%
  --(6.833,1.118)--(6.843,1.118)--(6.853,1.118)--(6.864,1.118)--(6.874,1.117)--(6.884,1.117)%
  --(6.895,1.117)--(6.905,1.116)--(6.915,1.116)--(6.925,1.115)--(6.936,1.114)--(6.946,1.113)%
  --(6.956,1.112)--(6.966,1.111)--(6.977,1.110)--(6.987,1.109)--(6.997,1.107)--(7.007,1.106)%
  --(7.018,1.104)--(7.028,1.102)--(7.038,1.100)--(7.049,1.098)--(7.059,1.096)--(7.069,1.094)%
  --(7.079,1.092)--(7.090,1.089)--(7.100,1.087)--(7.110,1.084)--(7.120,1.081)--(7.131,1.079)%
  --(7.141,1.076)--(7.151,1.073)--(7.162,1.071)--(7.172,1.068)--(7.182,1.065)--(7.192,1.062)%
  --(7.203,1.060)--(7.213,1.057)--(7.223,1.054)--(7.233,1.052)--(7.244,1.049)--(7.254,1.047)%
  --(7.264,1.045)--(7.274,1.042)--(7.285,1.040)--(7.295,1.038)--(7.305,1.036)--(7.316,1.034)%
  --(7.326,1.033)--(7.336,1.031)--(7.346,1.030)--(7.357,1.028)--(7.367,1.027)--(7.377,1.027)%
  --(7.387,1.026)--(7.398,1.026)--(7.408,1.025)--(7.418,1.025)--(7.429,1.026)--(7.439,1.026)%
  --(7.449,1.027)--(7.459,1.028)--(7.470,1.029)--(7.480,1.030)--(7.490,1.032)--(7.500,1.033)%
  --(7.511,1.035)--(7.521,1.038)--(7.531,1.040)--(7.541,1.042)--(7.552,1.045)--(7.562,1.048)%
  --(7.572,1.050)--(7.583,1.053)--(7.593,1.057)--(7.603,1.060)--(7.613,1.063)--(7.624,1.066)%
  --(7.634,1.070)--(7.644,1.073)--(7.654,1.077)--(7.665,1.080)--(7.675,1.084)--(7.685,1.088)%
  --(7.696,1.091)--(7.706,1.095)--(7.716,1.099)--(7.726,1.102)--(7.737,1.106)--(7.747,1.110)%
  --(7.757,1.113)--(7.767,1.117)--(7.778,1.120)--(7.788,1.124)--(7.798,1.127)--(7.808,1.130)%
  --(7.819,1.133)--(7.829,1.136)--(7.839,1.139)--(7.850,1.142)--(7.860,1.145)--(7.870,1.147)%
  --(7.880,1.150)--(7.891,1.152)--(7.901,1.154)--(7.911,1.156)--(7.921,1.159)--(7.932,1.161)%
  --(7.942,1.163)--(7.952,1.165)--(7.963,1.167)--(7.973,1.169)--(7.983,1.171)--(7.993,1.173)%
  --(8.004,1.175)--(8.014,1.177)--(8.024,1.179)--(8.034,1.181)--(8.045,1.183)--(8.055,1.186)%
  --(8.065,1.188)--(8.075,1.191)--(8.086,1.193)--(8.096,1.196)--(8.106,1.199)--(8.117,1.202)%
  --(8.127,1.205)--(8.137,1.208)--(8.147,1.211)--(8.158,1.215)--(8.168,1.219)--(8.178,1.223)%
  --(8.188,1.227)--(8.199,1.231)--(8.209,1.236)--(8.219,1.241)--(8.230,1.246)--(8.240,1.251)%
  --(8.250,1.257)--(8.260,1.262)--(8.271,1.269)--(8.281,1.275)--(8.291,1.282)--(8.301,1.289)%
  --(8.312,1.296)--(8.322,1.304)--(8.332,1.312)--(8.342,1.320)--(8.353,1.329)--(8.363,1.338)%
  --(8.373,1.347)--(8.384,1.357)--(8.394,1.367)--(8.404,1.377)--(8.414,1.388)--(8.425,1.399)%
  --(8.435,1.410)--(8.445,1.422)--(8.455,1.434)--(8.466,1.446)--(8.476,1.459)--(8.486,1.473)%
  --(8.497,1.486)--(8.507,1.500)--(8.517,1.515)--(8.527,1.530)--(8.538,1.545)--(8.548,1.561)%
  --(8.558,1.577)--(8.568,1.593)--(8.579,1.610)--(8.589,1.628)--(8.599,1.646)--(8.609,1.664)%
  --(8.620,1.683)--(8.630,1.702)--(8.640,1.722)--(8.651,1.742)--(8.661,1.762)--(8.671,1.783)%
  --(8.681,1.805)--(8.692,1.827)--(8.702,1.849)--(8.712,1.872)--(8.722,1.895)--(8.733,1.918)%
  --(8.743,1.941)--(8.753,1.965)--(8.764,1.989)--(8.774,2.012)--(8.784,2.036)--(8.794,2.060)%
  --(8.805,2.084)--(8.815,2.107)--(8.825,2.131)--(8.835,2.154)--(8.846,2.177)--(8.856,2.200)%
  --(8.866,2.223)--(8.876,2.245)--(8.887,2.267)--(8.897,2.289)--(8.907,2.310)--(8.918,2.330)%
  --(8.928,2.350)--(8.938,2.370)--(8.948,2.388)--(8.959,2.407)--(8.969,2.424)--(8.979,2.441)%
  --(8.989,2.457)--(9.000,2.472)--(9.010,2.486)--(9.020,2.499)--(9.031,2.512)--(9.041,2.523)%
  --(9.051,2.533)--(9.061,2.543)--(9.072,2.551)--(9.082,2.558)--(9.092,2.564)--(9.102,2.569)%
  --(9.113,2.572)--(9.123,2.575)--(9.133,2.577)--(9.143,2.578)--(9.154,2.578)--(9.164,2.577)%
  --(9.174,2.576)--(9.185,2.573)--(9.195,2.570)--(9.205,2.567)--(9.215,2.563)--(9.226,2.558)%
  --(9.236,2.553)--(9.246,2.547)--(9.256,2.541)--(9.267,2.534)--(9.277,2.528)--(9.287,2.521)%
  --(9.298,2.513)--(9.308,2.506)--(9.318,2.498)--(9.328,2.491)--(9.339,2.483)--(9.349,2.475)%
  --(9.359,2.467)--(9.369,2.460)--(9.380,2.452)--(9.390,2.445)--(9.400,2.438)--(9.410,2.431)%
  --(9.421,2.425)--(9.431,2.418)--(9.441,2.413)--(9.452,2.407)--(9.462,2.403)--(9.472,2.398)%
  --(9.482,2.395)--(9.493,2.392)--(9.503,2.389)--(9.513,2.388)--(9.523,2.387)--(9.534,2.387)%
  --(9.544,2.388)--(9.554,2.389)--(9.565,2.392)--(9.575,2.396)--(9.585,2.400)--(9.595,2.406)%
  --(9.606,2.413)--(9.616,2.421)--(9.626,2.430)--(9.636,2.441)--(9.647,2.452)--(9.657,2.466)%
  --(9.667,2.480)--(9.677,2.496)--(9.688,2.514)--(9.698,2.533)--(9.708,2.553)--(9.719,2.575)%
  --(9.729,2.599)--(9.739,2.625)--(9.749,2.652)--(9.760,2.681)--(9.770,2.712)--(9.780,2.745)%
  --(9.790,2.780)--(9.801,2.817)--(9.811,2.855)--(9.821,2.896)--(9.832,2.939)--(9.842,2.984)%
  --(9.852,3.032)--(9.862,3.081)--(9.873,3.133)--(9.883,3.187)--(9.893,3.244)--(9.903,3.303)%
  --(9.914,3.364)--(9.924,3.428)--(9.934,3.493)--(9.944,3.561)--(9.955,3.630)--(9.965,3.701)%
  --(9.975,3.773)--(9.986,3.846)--(9.996,3.921)--(10.006,3.997)--(10.016,4.074)--(10.027,4.151)%
  --(10.037,4.229)--(10.047,4.307)--(10.057,4.386)--(10.068,4.465)--(10.078,4.544)--(10.088,4.623)%
  --(10.099,4.701)--(10.109,4.779)--(10.119,4.857)--(10.129,4.933)--(10.140,5.009)--(10.150,5.084)%
  --(10.160,5.158)--(10.170,5.231)--(10.181,5.302)--(10.191,5.372)--(10.201,5.440)--(10.211,5.506)%
  --(10.222,5.570)--(10.232,5.632)--(10.242,5.691)--(10.253,5.748)--(10.263,5.803)--(10.273,5.855)%
  --(10.283,5.904)--(10.294,5.950)--(10.304,5.993)--(10.314,6.033)--(10.324,6.069)--(10.335,6.102)%
  --(10.345,6.133)--(10.355,6.160)--(10.366,6.185)--(10.376,6.207)--(10.386,6.226)--(10.396,6.243)%
  --(10.407,6.258)--(10.417,6.271)--(10.427,6.281)--(10.437,6.290)--(10.448,6.296)--(10.458,6.301)%
  --(10.468,6.305)--(10.478,6.307)--(10.489,6.307)--(10.499,6.306)--(10.509,6.304)--(10.520,6.301)%
  --(10.530,6.297)--(10.540,6.292)--(10.550,6.287)--(10.561,6.281)--(10.571,6.274)--(10.581,6.267)%
  --(10.591,6.260)--(10.602,6.253)--(10.612,6.245)--(10.622,6.238)--(10.633,6.231)--(10.643,6.224)%
  --(10.653,6.218)--(10.663,6.212)--(10.674,6.207)--(10.684,6.203)--(10.694,6.199)--(10.704,6.197)%
  --(10.715,6.196)--(10.725,6.195)--(10.735,6.197)--(10.745,6.199)--(10.756,6.202)--(10.766,6.206)%
  --(10.776,6.211)--(10.787,6.217)--(10.797,6.223)--(10.807,6.230)--(10.817,6.238)--(10.828,6.246)%
  --(10.838,6.254)--(10.848,6.263)--(10.858,6.272)--(10.869,6.281)--(10.879,6.290)--(10.889,6.299)%
  --(10.900,6.308)--(10.910,6.317)--(10.920,6.326)--(10.930,6.334)--(10.941,6.341)--(10.951,6.349)%
  --(10.961,6.355)--(10.971,6.361)--(10.982,6.366)--(10.992,6.371)--(11.002,6.374)--(11.012,6.377)%
  --(11.023,6.378)--(11.033,6.378)--(11.043,6.377)--(11.054,6.375)--(11.064,6.371)--(11.074,6.366)%
  --(11.084,6.359)--(11.095,6.351)--(11.105,6.341)--(11.115,6.329)--(11.125,6.315)--(11.136,6.299)%
  --(11.146,6.282)--(11.156,6.263)--(11.167,6.242)--(11.177,6.220)--(11.187,6.197)--(11.197,6.173)%
  --(11.208,6.148)--(11.218,6.123)--(11.228,6.097)--(11.238,6.071)--(11.249,6.044)--(11.259,6.018)%
  --(11.269,5.992)--(11.279,5.967)--(11.290,5.942)--(11.300,5.917)--(11.310,5.894)--(11.321,5.872)%
  --(11.331,5.851)--(11.341,5.831)--(11.351,5.813)--(11.362,5.797)--(11.372,5.783)--(11.382,5.771)%
  --(11.392,5.761)--(11.403,5.754)--(11.413,5.749)--(11.423,5.747)--(11.434,5.748)--(11.444,5.752)%
  --(11.454,5.759)--(11.464,5.770)--(11.475,5.784)--(11.485,5.803)--(11.495,5.825)--(11.505,5.851)%
  --(11.516,5.882)--(11.526,5.917)--(11.536,5.957)--(11.546,6.002)--(11.557,6.051)--(11.567,6.105)%
  --(11.577,6.164)--(11.588,6.226)--(11.598,6.293)--(11.608,6.365)--(11.618,6.440)--(11.629,6.519)%
  --(11.639,6.602)--(11.649,6.688)--(11.659,6.778)--(11.670,6.871)--(11.680,6.968)--(11.690,7.068)%
  --(11.701,7.170)--(11.711,7.276)--(11.721,7.385)--(11.731,7.496)--(11.742,7.609)--(11.752,7.725)%
  --(11.762,7.844)--(11.772,7.964)--(11.783,8.087)--(11.793,8.212)--(11.803,8.338)--(11.807,8.381);
\gpcolor{color=gp lt color border}
\gpsetlinetype{gp lt border}
\gpsetlinewidth{1.00}
\draw[gp path] (1.688,8.381)--(1.688,0.985)--(11.947,0.985)--(11.947,8.381)--cycle;
%% coordinates of the plot area
\gpdefrectangularnode{gp plot 1}{\pgfpoint{1.688cm}{0.985cm}}{\pgfpoint{11.947cm}{8.381cm}}
\end{tikzpicture}
%% gnuplot variables

      \end{myplot}

      \begin{myplot}%
        {Распределение методов от значения меры точности для \eng{SPEC~JVM98}}%
        {plot:specjvm_all_aliases_distribution_cumulative}
        \begin{tikzpicture}[gnuplot]
%% generated with GNUPLOT 4.5p0 (Lua 5.1; terminal rev. 99, script rev. 98)
%% 27.05.2011 11:34:39
\path (0.000,0.000) rectangle (12.500,8.750);
\gpcolor{color=gp lt color border}
\gpsetlinetype{gp lt border}
\gpsetlinewidth{1.00}
\draw[gp path] (1.504,0.985)--(1.684,0.985);
\draw[gp path] (11.947,0.985)--(11.767,0.985);
\node[gp node right] at (1.320,0.985) {\num{0}};
\draw[gp path] (1.504,2.330)--(1.684,2.330);
\draw[gp path] (11.947,2.330)--(11.767,2.330);
\node[gp node right] at (1.320,2.330) {\num{0.2}};
\draw[gp path] (1.504,3.674)--(1.684,3.674);
\draw[gp path] (11.947,3.674)--(11.767,3.674);
\node[gp node right] at (1.320,3.674) {\num{0.4}};
\draw[gp path] (1.504,5.019)--(1.684,5.019);
\draw[gp path] (11.947,5.019)--(11.767,5.019);
\node[gp node right] at (1.320,5.019) {\num{0.6}};
\draw[gp path] (1.504,6.364)--(1.684,6.364);
\draw[gp path] (11.947,6.364)--(11.767,6.364);
\node[gp node right] at (1.320,6.364) {\num{0.8}};
\draw[gp path] (1.504,7.709)--(1.684,7.709);
\draw[gp path] (11.947,7.709)--(11.767,7.709);
\node[gp node right] at (1.320,7.709) {\num{1}};
\gpcolor{color=gp lt color axes}
\gpsetlinetype{gp lt axes}
\draw[gp path] (1.504,0.985)--(1.504,8.381);
\gpcolor{color=gp lt color border}
\gpsetlinetype{gp lt border}
\draw[gp path] (1.504,0.985)--(1.504,1.165);
\draw[gp path] (1.504,8.381)--(1.504,8.201);
\node[gp node center] at (1.504,0.677) {\num{0}};
\gpcolor{color=gp lt color axes}
\gpsetlinetype{gp lt axes}
\draw[gp path] (3.593,0.985)--(3.593,6.969);
\draw[gp path] (3.593,8.201)--(3.593,8.381);
\gpcolor{color=gp lt color border}
\gpsetlinetype{gp lt border}
\draw[gp path] (3.593,0.985)--(3.593,1.165);
\draw[gp path] (3.593,8.381)--(3.593,8.201);
\node[gp node center] at (3.593,0.677) {\num{0.2}};
\gpcolor{color=gp lt color axes}
\gpsetlinetype{gp lt axes}
\draw[gp path] (5.681,0.985)--(5.681,8.381);
\gpcolor{color=gp lt color border}
\gpsetlinetype{gp lt border}
\draw[gp path] (5.681,0.985)--(5.681,1.165);
\draw[gp path] (5.681,8.381)--(5.681,8.201);
\node[gp node center] at (5.681,0.677) {\num{0.4}};
\gpcolor{color=gp lt color axes}
\gpsetlinetype{gp lt axes}
\draw[gp path] (7.770,0.985)--(7.770,8.381);
\gpcolor{color=gp lt color border}
\gpsetlinetype{gp lt border}
\draw[gp path] (7.770,0.985)--(7.770,1.165);
\draw[gp path] (7.770,8.381)--(7.770,8.201);
\node[gp node center] at (7.770,0.677) {\num{0.6}};
\gpcolor{color=gp lt color axes}
\gpsetlinetype{gp lt axes}
\draw[gp path] (9.858,0.985)--(9.858,8.381);
\gpcolor{color=gp lt color border}
\gpsetlinetype{gp lt border}
\draw[gp path] (9.858,0.985)--(9.858,1.165);
\draw[gp path] (9.858,8.381)--(9.858,8.201);
\node[gp node center] at (9.858,0.677) {\num{0.8}};
\gpcolor{color=gp lt color axes}
\gpsetlinetype{gp lt axes}
\draw[gp path] (11.947,0.985)--(11.947,8.381);
\gpcolor{color=gp lt color border}
\gpsetlinetype{gp lt border}
\draw[gp path] (11.947,0.985)--(11.947,1.165);
\draw[gp path] (11.947,8.381)--(11.947,8.201);
\node[gp node center] at (11.947,0.677) {\num{1}};
\draw[gp path] (1.504,8.381)--(1.504,0.985)--(11.947,0.985)--(11.947,8.381)--cycle;
\node[gp node center,rotate=-270] at (0.246,4.683) {Количество методов, \%};
\node[gp node center] at (6.725,0.215) {Отношение среднего количества синонимов к числу переменных};
\node[gp node right] at (4.264,8.047) {base};
\gpcolor{color=gp lt color 0}
\gpsetlinetype{gp lt plot 0}
\gpsetlinewidth{2.00}
\draw[gp path] (4.448,8.047)--(5.364,8.047);
\draw[gp path] (1.504,0.989)--(1.608,0.993)--(2.026,0.997)--(2.548,1.006)--(2.757,1.015)%
  --(2.966,1.019)--(3.175,1.023)--(3.384,1.057)--(3.488,1.061)--(3.593,1.070)--(3.697,1.082)%
  --(3.801,1.130)--(3.906,1.152)--(4.010,1.169)--(4.115,1.178)--(4.219,1.222)--(4.324,1.226)%
  --(4.428,1.238)--(4.532,1.251)--(4.637,1.263)--(4.741,1.294)--(4.846,1.328)--(4.950,1.345)%
  --(5.055,1.380)--(5.159,1.388)--(5.263,1.410)--(5.368,1.419)--(5.472,1.466)--(5.577,1.505)%
  --(5.681,1.557)--(5.786,1.613)--(5.890,1.639)--(5.994,1.733)--(6.099,1.776)--(6.203,1.802)%
  --(6.308,1.863)--(6.412,1.910)--(6.517,2.026)--(6.621,2.035)--(6.726,2.104)--(6.830,2.186)%
  --(6.934,2.285)--(7.039,2.337)--(7.143,2.428)--(7.248,2.492)--(7.352,2.574)--(7.457,2.679)%
  --(7.561,2.800)--(7.665,2.887)--(7.770,2.940)--(7.874,3.000)--(7.979,3.178)--(8.083,3.281)%
  --(8.188,3.389)--(8.292,3.541)--(8.396,3.744)--(8.501,3.904)--(8.605,4.090)--(8.710,4.245)%
  --(8.814,4.452)--(8.919,4.530)--(9.023,4.954)--(9.127,5.222)--(9.232,5.512)--(9.336,5.724)%
  --(9.441,5.880)--(9.545,5.992)--(9.650,6.157)--(9.754,6.265)--(9.858,6.429)--(9.963,6.810)%
  --(10.067,6.927)--(10.172,6.961)--(10.276,7.065)--(10.381,7.134)--(10.485,7.191)--(10.589,7.225)%
  --(10.694,7.316)--(10.798,7.363)--(10.903,7.393)--(11.007,7.444)--(11.112,7.487)--(11.216,7.504)%
  --(11.320,7.555)--(11.425,7.577)--(11.529,7.607)--(11.634,7.655)--(11.738,7.680)--(11.843,7.688)%
  --(11.947,7.697);
\gpcolor{color=gp lt color border}
\node[gp node right] at (4.264,7.739) {equality-based};
\gpcolor{color=gp lt color 2}
\draw[gp path] (4.448,7.739)--(5.364,7.739);
\draw[gp path] (1.504,0.989)--(1.608,0.993)--(2.548,1.001)--(2.757,1.010)--(2.966,1.019)%
  --(3.175,1.023)--(3.384,1.044)--(3.488,1.048)--(3.593,1.052)--(3.697,1.074)--(3.801,1.108)%
  --(3.906,1.116)--(4.010,1.133)--(4.115,1.137)--(4.219,1.167)--(4.324,1.171)--(4.428,1.179)%
  --(4.532,1.188)--(4.637,1.210)--(4.741,1.227)--(4.846,1.249)--(4.950,1.257)--(5.055,1.300)%
  --(5.159,1.313)--(5.263,1.347)--(5.472,1.386)--(5.577,1.417)--(5.681,1.447)--(5.786,1.499)%
  --(5.890,1.528)--(5.994,1.585)--(6.099,1.619)--(6.203,1.637)--(6.308,1.680)--(6.412,1.788)%
  --(6.517,1.896)--(6.621,1.913)--(6.726,1.965)--(6.830,2.031)--(6.934,2.100)--(7.039,2.161)%
  --(7.143,2.199)--(7.248,2.269)--(7.352,2.381)--(7.457,2.424)--(7.561,2.510)--(7.665,2.601)%
  --(7.770,2.652)--(7.874,2.691)--(7.979,2.826)--(8.083,2.878)--(8.188,3.003)--(8.292,3.064)%
  --(8.396,3.245)--(8.501,3.371)--(8.605,3.566)--(8.710,3.701)--(8.814,3.878)--(8.919,3.986)%
  --(9.023,4.251)--(9.127,4.519)--(9.232,4.908)--(9.336,5.107)--(9.441,5.311)--(9.545,5.614)%
  --(9.650,5.942)--(9.754,6.073)--(9.858,6.281)--(9.963,6.654)--(10.067,6.827)--(10.172,6.870)%
  --(10.276,6.991)--(10.381,7.065)--(10.485,7.138)--(10.589,7.194)--(10.694,7.311)--(10.798,7.362)%
  --(10.903,7.388)--(11.007,7.431)--(11.112,7.474)--(11.216,7.500)--(11.320,7.551)--(11.425,7.572)%
  --(11.529,7.606)--(11.634,7.654)--(11.738,7.684)--(11.843,7.692)--(11.947,7.701);
\gpcolor{color=gp lt color border}
\node[gp node right] at (4.264,7.431) {w/o data flow};
\gpcolor{color=gp lt color 6}
\draw[gp path] (4.448,7.431)--(5.364,7.431);
\draw[gp path] (2.548,0.989)--(3.175,0.993)--(3.384,1.011)--(3.488,1.015)--(3.593,1.019)%
  --(3.697,1.036)--(3.801,1.053)--(4.010,1.066)--(4.115,1.074)--(4.219,1.091)--(4.428,1.099)%
  --(4.532,1.103)--(4.637,1.111)--(4.741,1.115)--(4.846,1.140)--(4.950,1.148)--(5.055,1.183)%
  --(5.159,1.187)--(5.263,1.199)--(5.368,1.216)--(5.472,1.242)--(5.577,1.259)--(5.681,1.284)%
  --(5.786,1.310)--(5.890,1.327)--(5.994,1.384)--(6.099,1.413)--(6.203,1.426)--(6.308,1.482)%
  --(6.412,1.525)--(6.517,1.564)--(6.621,1.573)--(6.726,1.607)--(6.830,1.655)--(6.934,1.737)%
  --(7.039,1.799)--(7.143,1.842)--(7.248,1.907)--(7.352,2.033)--(7.457,2.037)--(7.561,2.084)%
  --(7.665,2.176)--(7.770,2.215)--(7.874,2.258)--(7.979,2.379)--(8.083,2.449)--(8.188,2.496)%
  --(8.292,2.531)--(8.396,2.700)--(8.501,2.757)--(8.605,2.904)--(8.710,2.969)--(8.814,3.069)%
  --(8.919,3.165)--(9.023,3.375)--(9.127,3.474)--(9.232,3.639)--(9.336,3.813)--(9.441,3.992)%
  --(9.545,4.083)--(9.650,4.356)--(9.754,4.452)--(9.858,4.643)--(9.963,4.973)--(10.067,5.285)%
  --(10.172,5.324)--(10.276,5.589)--(10.381,5.702)--(10.485,5.841)--(10.589,6.054)--(10.694,6.344)%
  --(10.798,6.618)--(10.903,6.800)--(11.007,6.952)--(11.112,7.095)--(11.216,7.156)--(11.320,7.273)%
  --(11.425,7.417)--(11.529,7.512)--(11.634,7.586)--(11.738,7.643)--(11.843,7.660)--(11.947,7.694);
\gpcolor{color=gp lt color border}
\node[gp node right] at (4.264,7.123) {w/o types};
\gpcolor{color=gp lt color 1}
\draw[gp path] (4.448,7.123)--(5.364,7.123);
\draw[gp path] (1.504,0.992)--(1.608,1.000)--(2.548,1.007)--(2.757,1.029)--(2.966,1.036)%
  --(4.219,1.043)--(4.846,1.051)--(5.368,1.058)--(5.994,1.066)--(6.099,1.073)--(6.621,1.080)%
  --(6.726,1.095)--(6.830,1.102)--(6.934,1.109)--(7.039,1.117)--(7.143,1.124)--(7.665,1.132)%
  --(7.770,1.139)--(7.874,1.146)--(7.979,1.161)--(8.083,1.169)--(8.188,1.197)--(8.292,1.204)%
  --(8.396,1.218)--(8.501,1.261)--(8.605,1.283)--(8.710,1.290)--(8.814,1.447)--(8.919,1.475)%
  --(9.023,1.575)--(9.127,1.596)--(9.232,1.660)--(9.336,1.682)--(9.441,1.731)--(9.545,1.824)%
  --(9.650,1.902)--(9.754,1.945)--(9.858,2.159)--(9.963,2.266)--(10.067,2.373)--(10.172,2.537)%
  --(10.276,2.708)--(10.381,3.085)--(10.485,3.506)--(10.589,3.684)--(10.694,3.891)--(10.798,3.955)%
  --(10.903,4.290)--(11.007,4.489)--(11.112,4.809)--(11.216,5.023)--(11.320,5.237)--(11.425,5.422)%
  --(11.529,5.814)--(11.634,5.878)--(11.738,6.269)--(11.843,6.718)--(11.947,7.715);
\gpcolor{color=gp lt color border}
\gpsetlinetype{gp lt border}
\gpsetlinewidth{1.00}
\draw[gp path] (1.504,8.381)--(1.504,0.985)--(11.947,0.985)--(11.947,8.381)--cycle;
%% coordinates of the plot area
\gpdefrectangularnode{gp plot 1}{\pgfpoint{1.504cm}{0.985cm}}{\pgfpoint{11.947cm}{8.381cm}}
\end{tikzpicture}
%% gnuplot variables

      \end{myplot}

      \begin{myplot}%
        {Зависимость количества методов от значения меры точности для \eng{SPECjvm2008}}%
        {plot:specjvm2008_all_aliases_distribution}
        \begin{tikzpicture}[gnuplot]
%% generated with GNUPLOT 4.5p0 (Lua 5.1; terminal rev. 99, script rev. 98)
%% 27.05.2011 12:45:12
\path (0.000,0.000) rectangle (12.500,8.750);
\gpcolor{color=gp lt color border}
\gpsetlinetype{gp lt border}
\gpsetlinewidth{1.00}
\draw[gp path] (1.688,0.985)--(1.868,0.985);
\draw[gp path] (11.947,0.985)--(11.767,0.985);
\node[gp node right] at (1.504,0.985) {\num{0}};
\draw[gp path] (1.688,2.834)--(1.868,2.834);
\draw[gp path] (11.947,2.834)--(11.767,2.834);
\node[gp node right] at (1.504,2.834) {\num{0.05}};
\draw[gp path] (1.688,4.683)--(1.868,4.683);
\draw[gp path] (11.947,4.683)--(11.767,4.683);
\node[gp node right] at (1.504,4.683) {\num{0.1}};
\draw[gp path] (1.688,6.532)--(1.868,6.532);
\draw[gp path] (11.947,6.532)--(11.767,6.532);
\node[gp node right] at (1.504,6.532) {\num{0.15}};
\draw[gp path] (1.688,8.381)--(1.868,8.381);
\draw[gp path] (11.947,8.381)--(11.767,8.381);
\node[gp node right] at (1.504,8.381) {\num{0.2}};
\draw[gp path] (1.688,0.985)--(1.688,1.165);
\draw[gp path] (1.688,8.381)--(1.688,8.201);
\node[gp node center] at (1.688,0.677) {\num{0}};
\draw[gp path] (3.740,0.985)--(3.740,1.165);
\draw[gp path] (3.740,8.381)--(3.740,8.201);
\node[gp node center] at (3.740,0.677) {\num{0.2}};
\draw[gp path] (5.792,0.985)--(5.792,1.165);
\draw[gp path] (5.792,8.381)--(5.792,8.201);
\node[gp node center] at (5.792,0.677) {\num{0.4}};
\draw[gp path] (7.843,0.985)--(7.843,1.165);
\draw[gp path] (7.843,8.381)--(7.843,8.201);
\node[gp node center] at (7.843,0.677) {\num{0.6}};
\draw[gp path] (9.895,0.985)--(9.895,1.165);
\draw[gp path] (9.895,8.381)--(9.895,8.201);
\node[gp node center] at (9.895,0.677) {\num{0.8}};
\draw[gp path] (11.947,0.985)--(11.947,1.165);
\draw[gp path] (11.947,8.381)--(11.947,8.201);
\node[gp node center] at (11.947,0.677) {\num{1}};
\draw[gp path] (1.688,8.381)--(1.688,0.985)--(11.947,0.985)--(11.947,8.381)--cycle;
\node[gp node center,rotate=-270] at (0.246,4.683) {Количество методов, \%};
\node[gp node center] at (6.817,0.215) {Отношение среднего количества синонимов к числу переменных};
\node[gp node right] at (4.448,8.047) {base};
\gpfill{color=gp lt color 0,opacity=0.10} (4.632,7.970)--(5.548,7.970)--(5.548,8.124)--(4.632,8.124)--cycle;
\gpcolor{color=gp lt color 0}
\gpsetlinetype{gp lt plot 0}
\gpsetlinewidth{2.00}
\draw[gp path] (4.632,7.970)--(5.548,7.970)--(5.548,8.124)--(4.632,8.124)--cycle;
\gpfill{color=gp lt color 0,opacity=0.10} (1.688,1.003)--(1.688,1.003)--(1.698,1.003)--(1.709,1.003)%
    --(1.719,1.002)--(1.729,1.002)--(1.739,1.001)--(1.750,1.001)--(1.760,1.001)%
    --(1.770,1.000)--(1.780,1.000)--(1.791,0.999)--(1.801,0.999)--(1.811,0.999)%
    --(1.822,0.998)--(1.832,0.998)--(1.842,0.998)--(1.852,0.997)--(1.863,0.997)%
    --(1.873,0.997)--(1.883,0.997)--(1.893,0.996)--(1.904,0.996)--(1.914,0.996)%
    --(1.924,0.996)--(1.934,0.995)--(1.945,0.995)--(1.955,0.995)--(1.965,0.995)%
    --(1.976,0.995)--(1.986,0.995)--(1.996,0.995)--(2.006,0.995)--(2.017,0.995)%
    --(2.027,0.995)--(2.037,0.995)--(2.047,0.995)--(2.058,0.995)--(2.068,0.995)%
    --(2.078,0.996)--(2.089,0.996)--(2.099,0.996)--(2.109,0.996)--(2.119,0.997)%
    --(2.130,0.997)--(2.140,0.998)--(2.150,0.998)--(2.160,0.998)--(2.171,0.999)%
    --(2.181,0.999)--(2.191,1.000)--(2.201,1.000)--(2.212,1.001)--(2.222,1.002)%
    --(2.232,1.002)--(2.243,1.003)--(2.253,1.004)--(2.263,1.004)--(2.273,1.005)%
    --(2.284,1.006)--(2.294,1.006)--(2.304,1.007)--(2.314,1.008)--(2.325,1.009)%
    --(2.335,1.009)--(2.345,1.010)--(2.356,1.011)--(2.366,1.012)--(2.376,1.012)%
    --(2.386,1.013)--(2.397,1.014)--(2.407,1.015)--(2.417,1.015)--(2.427,1.016)%
    --(2.438,1.017)--(2.448,1.018)--(2.458,1.018)--(2.468,1.019)--(2.479,1.020)%
    --(2.489,1.021)--(2.499,1.021)--(2.510,1.022)--(2.520,1.023)--(2.530,1.023)%
    --(2.540,1.024)--(2.551,1.025)--(2.561,1.025)--(2.571,1.026)--(2.581,1.026)%
    --(2.592,1.027)--(2.602,1.028)--(2.612,1.028)--(2.623,1.029)--(2.633,1.029)%
    --(2.643,1.030)--(2.653,1.030)--(2.664,1.031)--(2.674,1.032)--(2.684,1.032)%
    --(2.694,1.033)--(2.705,1.033)--(2.715,1.034)--(2.725,1.034)--(2.735,1.035)%
    --(2.746,1.036)--(2.756,1.036)--(2.766,1.037)--(2.777,1.037)--(2.787,1.038)%
    --(2.797,1.039)--(2.807,1.039)--(2.818,1.040)--(2.828,1.041)--(2.838,1.042)%
    --(2.848,1.042)--(2.859,1.043)--(2.869,1.044)--(2.879,1.045)--(2.890,1.045)%
    --(2.900,1.046)--(2.910,1.047)--(2.920,1.048)--(2.931,1.049)--(2.941,1.050)%
    --(2.951,1.051)--(2.961,1.052)--(2.972,1.053)--(2.982,1.054)--(2.992,1.055)%
    --(3.002,1.056)--(3.013,1.057)--(3.023,1.058)--(3.033,1.060)--(3.044,1.061)%
    --(3.054,1.062)--(3.064,1.063)--(3.074,1.065)--(3.085,1.066)--(3.095,1.068)%
    --(3.105,1.069)--(3.115,1.071)--(3.126,1.072)--(3.136,1.074)--(3.146,1.075)%
    --(3.157,1.077)--(3.167,1.079)--(3.177,1.080)--(3.187,1.082)--(3.198,1.084)%
    --(3.208,1.086)--(3.218,1.088)--(3.228,1.089)--(3.239,1.091)--(3.249,1.093)%
    --(3.259,1.095)--(3.269,1.098)--(3.280,1.100)--(3.290,1.102)--(3.300,1.104)%
    --(3.311,1.106)--(3.321,1.109)--(3.331,1.111)--(3.341,1.114)--(3.352,1.116)%
    --(3.362,1.119)--(3.372,1.121)--(3.382,1.124)--(3.393,1.126)--(3.403,1.129)%
    --(3.413,1.132)--(3.424,1.135)--(3.434,1.138)--(3.444,1.141)--(3.454,1.144)%
    --(3.465,1.147)--(3.475,1.151)--(3.485,1.154)--(3.495,1.158)--(3.506,1.161)%
    --(3.516,1.165)--(3.526,1.168)--(3.536,1.172)--(3.547,1.176)--(3.557,1.180)%
    --(3.567,1.185)--(3.578,1.189)--(3.588,1.193)--(3.598,1.198)--(3.608,1.202)%
    --(3.619,1.207)--(3.629,1.212)--(3.639,1.217)--(3.649,1.222)--(3.660,1.227)%
    --(3.670,1.233)--(3.680,1.238)--(3.691,1.244)--(3.701,1.250)--(3.711,1.256)%
    --(3.721,1.262)--(3.732,1.268)--(3.742,1.275)--(3.752,1.281)--(3.762,1.288)%
    --(3.773,1.295)--(3.783,1.302)--(3.793,1.309)--(3.803,1.316)--(3.814,1.323)%
    --(3.824,1.331)--(3.834,1.338)--(3.845,1.346)--(3.855,1.353)--(3.865,1.361)%
    --(3.875,1.368)--(3.886,1.376)--(3.896,1.383)--(3.906,1.391)--(3.916,1.398)%
    --(3.927,1.406)--(3.937,1.413)--(3.947,1.421)--(3.958,1.428)--(3.968,1.435)%
    --(3.978,1.442)--(3.988,1.449)--(3.999,1.456)--(4.009,1.463)--(4.019,1.469)%
    --(4.029,1.476)--(4.040,1.482)--(4.050,1.488)--(4.060,1.494)--(4.070,1.500)%
    --(4.081,1.505)--(4.091,1.511)--(4.101,1.516)--(4.112,1.520)--(4.122,1.525)%
    --(4.132,1.529)--(4.142,1.533)--(4.153,1.537)--(4.163,1.540)--(4.173,1.543)%
    --(4.183,1.546)--(4.194,1.549)--(4.204,1.551)--(4.214,1.554)--(4.225,1.556)%
    --(4.235,1.558)--(4.245,1.559)--(4.255,1.561)--(4.266,1.563)--(4.276,1.564)%
    --(4.286,1.566)--(4.296,1.568)--(4.307,1.569)--(4.317,1.571)--(4.327,1.573)%
    --(4.337,1.574)--(4.348,1.576)--(4.358,1.578)--(4.368,1.580)--(4.379,1.583)%
    --(4.389,1.585)--(4.399,1.588)--(4.409,1.591)--(4.420,1.594)--(4.430,1.598)%
    --(4.440,1.601)--(4.450,1.606)--(4.461,1.610)--(4.471,1.615)--(4.481,1.620)%
    --(4.492,1.626)--(4.502,1.632)--(4.512,1.639)--(4.522,1.646)--(4.533,1.653)%
    --(4.543,1.661)--(4.553,1.670)--(4.563,1.679)--(4.574,1.689)--(4.584,1.699)%
    --(4.594,1.710)--(4.604,1.722)--(4.615,1.733)--(4.625,1.746)--(4.635,1.758)%
    --(4.646,1.772)--(4.656,1.785)--(4.666,1.799)--(4.676,1.813)--(4.687,1.828)%
    --(4.697,1.842)--(4.707,1.857)--(4.717,1.873)--(4.728,1.888)--(4.738,1.904)%
    --(4.748,1.919)--(4.759,1.935)--(4.769,1.951)--(4.779,1.967)--(4.789,1.983)%
    --(4.800,1.999)--(4.810,2.015)--(4.820,2.031)--(4.830,2.047)--(4.841,2.063)%
    --(4.851,2.079)--(4.861,2.095)--(4.871,2.110)--(4.882,2.125)--(4.892,2.140)%
    --(4.902,2.155)--(4.913,2.170)--(4.923,2.184)--(4.933,2.198)--(4.943,2.211)%
    --(4.954,2.225)--(4.964,2.238)--(4.974,2.250)--(4.984,2.262)--(4.995,2.273)%
    --(5.005,2.285)--(5.015,2.295)--(5.026,2.306)--(5.036,2.316)--(5.046,2.325)%
    --(5.056,2.334)--(5.067,2.343)--(5.077,2.352)--(5.087,2.360)--(5.097,2.368)%
    --(5.108,2.375)--(5.118,2.382)--(5.128,2.389)--(5.138,2.395)--(5.149,2.402)%
    --(5.159,2.408)--(5.169,2.413)--(5.180,2.419)--(5.190,2.424)--(5.200,2.429)%
    --(5.210,2.433)--(5.221,2.438)--(5.231,2.442)--(5.241,2.446)--(5.251,2.449)%
    --(5.262,2.453)--(5.272,2.456)--(5.282,2.460)--(5.293,2.463)--(5.303,2.465)%
    --(5.313,2.468)--(5.323,2.471)--(5.334,2.473)--(5.344,2.475)--(5.354,2.477)%
    --(5.364,2.480)--(5.375,2.481)--(5.385,2.483)--(5.395,2.485)--(5.405,2.487)%
    --(5.416,2.489)--(5.426,2.490)--(5.436,2.492)--(5.447,2.494)--(5.457,2.496)%
    --(5.467,2.498)--(5.477,2.500)--(5.488,2.502)--(5.498,2.505)--(5.508,2.508)%
    --(5.518,2.511)--(5.529,2.515)--(5.539,2.518)--(5.549,2.522)--(5.560,2.527)%
    --(5.570,2.532)--(5.580,2.537)--(5.590,2.543)--(5.601,2.549)--(5.611,2.556)%
    --(5.621,2.564)--(5.631,2.572)--(5.642,2.581)--(5.652,2.590)--(5.662,2.600)%
    --(5.672,2.611)--(5.683,2.622)--(5.693,2.634)--(5.703,2.648)--(5.714,2.661)%
    --(5.724,2.676)--(5.734,2.692)--(5.744,2.708)--(5.755,2.726)--(5.765,2.744)%
    --(5.775,2.764)--(5.785,2.784)--(5.796,2.806)--(5.806,2.828)--(5.816,2.852)%
    --(5.827,2.876)--(5.837,2.902)--(5.847,2.928)--(5.857,2.954)--(5.868,2.981)%
    --(5.878,3.009)--(5.888,3.037)--(5.898,3.065)--(5.909,3.094)--(5.919,3.123)%
    --(5.929,3.151)--(5.939,3.180)--(5.950,3.209)--(5.960,3.237)--(5.970,3.265)%
    --(5.981,3.293)--(5.991,3.321)--(6.001,3.348)--(6.011,3.374)--(6.022,3.399)%
    --(6.032,3.424)--(6.042,3.448)--(6.052,3.471)--(6.063,3.493)--(6.073,3.514)%
    --(6.083,3.534)--(6.094,3.553)--(6.104,3.570)--(6.114,3.586)--(6.124,3.600)%
    --(6.135,3.613)--(6.145,3.624)--(6.155,3.633)--(6.165,3.641)--(6.176,3.646)%
    --(6.186,3.650)--(6.196,3.651)--(6.206,3.651)--(6.217,3.648)--(6.227,3.643)%
    --(6.237,3.636)--(6.248,3.628)--(6.258,3.617)--(6.268,3.605)--(6.278,3.592)%
    --(6.289,3.577)--(6.299,3.561)--(6.309,3.543)--(6.319,3.525)--(6.330,3.505)%
    --(6.340,3.485)--(6.350,3.464)--(6.361,3.442)--(6.371,3.420)--(6.381,3.397)%
    --(6.391,3.374)--(6.402,3.350)--(6.412,3.327)--(6.422,3.303)--(6.432,3.280)%
    --(6.443,3.257)--(6.453,3.234)--(6.463,3.211)--(6.473,3.189)--(6.484,3.168)%
    --(6.494,3.147)--(6.504,3.128)--(6.515,3.109)--(6.525,3.091)--(6.535,3.074)%
    --(6.545,3.059)--(6.556,3.045)--(6.566,3.032)--(6.576,3.021)--(6.586,3.012)%
    --(6.597,3.004)--(6.607,2.999)--(6.617,2.995)--(6.628,2.994)--(6.638,2.994)%
    --(6.648,2.996)--(6.658,3.000)--(6.669,3.006)--(6.679,3.013)--(6.689,3.022)%
    --(6.699,3.032)--(6.710,3.044)--(6.720,3.056)--(6.730,3.070)--(6.740,3.085)%
    --(6.751,3.101)--(6.761,3.118)--(6.771,3.136)--(6.782,3.155)--(6.792,3.174)%
    --(6.802,3.193)--(6.812,3.213)--(6.823,3.234)--(6.833,3.255)--(6.843,3.276)%
    --(6.853,3.297)--(6.864,3.318)--(6.874,3.339)--(6.884,3.360)--(6.895,3.381)%
    --(6.905,3.402)--(6.915,3.422)--(6.925,3.442)--(6.936,3.461)--(6.946,3.479)%
    --(6.956,3.497)--(6.966,3.514)--(6.977,3.530)--(6.987,3.545)--(6.997,3.559)%
    --(7.007,3.572)--(7.018,3.583)--(7.028,3.594)--(7.038,3.602)--(7.049,3.610)%
    --(7.059,3.616)--(7.069,3.622)--(7.079,3.626)--(7.090,3.629)--(7.100,3.631)%
    --(7.110,3.632)--(7.120,3.632)--(7.131,3.631)--(7.141,3.630)--(7.151,3.628)%
    --(7.162,3.625)--(7.172,3.621)--(7.182,3.617)--(7.192,3.613)--(7.203,3.608)%
    --(7.213,3.603)--(7.223,3.597)--(7.233,3.591)--(7.244,3.585)--(7.254,3.579)%
    --(7.264,3.573)--(7.274,3.566)--(7.285,3.560)--(7.295,3.554)--(7.305,3.548)%
    --(7.316,3.542)--(7.326,3.536)--(7.336,3.531)--(7.346,3.526)--(7.357,3.522)%
    --(7.367,3.518)--(7.377,3.514)--(7.387,3.511)--(7.398,3.509)--(7.408,3.507)%
    --(7.418,3.507)--(7.429,3.507)--(7.439,3.508)--(7.449,3.509)--(7.459,3.512)%
    --(7.470,3.516)--(7.480,3.520)--(7.490,3.525)--(7.500,3.531)--(7.511,3.538)%
    --(7.521,3.545)--(7.531,3.553)--(7.541,3.561)--(7.552,3.571)--(7.562,3.581)%
    --(7.572,3.591)--(7.583,3.602)--(7.593,3.613)--(7.603,3.625)--(7.613,3.638)%
    --(7.624,3.650)--(7.634,3.664)--(7.644,3.677)--(7.654,3.691)--(7.665,3.705)%
    --(7.675,3.720)--(7.685,3.735)--(7.696,3.750)--(7.706,3.765)--(7.716,3.781)%
    --(7.726,3.796)--(7.737,3.812)--(7.747,3.828)--(7.757,3.844)--(7.767,3.860)%
    --(7.778,3.876)--(7.788,3.892)--(7.798,3.908)--(7.808,3.923)--(7.819,3.939)%
    --(7.829,3.955)--(7.839,3.971)--(7.850,3.986)--(7.860,4.001)--(7.870,4.016)%
    --(7.880,4.031)--(7.891,4.045)--(7.901,4.059)--(7.911,4.073)--(7.921,4.087)%
    --(7.932,4.100)--(7.942,4.113)--(7.952,4.125)--(7.963,4.137)--(7.973,4.149)%
    --(7.983,4.160)--(7.993,4.171)--(8.004,4.181)--(8.014,4.191)--(8.024,4.200)%
    --(8.034,4.208)--(8.045,4.216)--(8.055,4.224)--(8.065,4.230)--(8.075,4.236)%
    --(8.086,4.242)--(8.096,4.246)--(8.106,4.250)--(8.117,4.253)--(8.127,4.256)%
    --(8.137,4.257)--(8.147,4.258)--(8.158,4.258)--(8.168,4.257)--(8.178,4.256)%
    --(8.188,4.253)--(8.199,4.249)--(8.209,4.245)--(8.219,4.239)--(8.230,4.233)%
    --(8.240,4.225)--(8.250,4.217)--(8.260,4.207)--(8.271,4.196)--(8.281,4.185)%
    --(8.291,4.172)--(8.301,4.159)--(8.312,4.145)--(8.322,4.130)--(8.332,4.115)%
    --(8.342,4.099)--(8.353,4.082)--(8.363,4.065)--(8.373,4.047)--(8.384,4.029)%
    --(8.394,4.011)--(8.404,3.993)--(8.414,3.974)--(8.425,3.956)--(8.435,3.937)%
    --(8.445,3.918)--(8.455,3.900)--(8.466,3.881)--(8.476,3.863)--(8.486,3.845)%
    --(8.497,3.827)--(8.507,3.810)--(8.517,3.793)--(8.527,3.777)--(8.538,3.761)%
    --(8.548,3.746)--(8.558,3.731)--(8.568,3.717)--(8.579,3.705)--(8.589,3.692)%
    --(8.599,3.681)--(8.609,3.671)--(8.620,3.662)--(8.630,3.654)--(8.640,3.647)%
    --(8.651,3.642)--(8.661,3.638)--(8.671,3.635)--(8.681,3.633)--(8.692,3.633)%
    --(8.702,3.634)--(8.712,3.636)--(8.722,3.639)--(8.733,3.643)--(8.743,3.648)%
    --(8.753,3.654)--(8.764,3.660)--(8.774,3.668)--(8.784,3.676)--(8.794,3.684)%
    --(8.805,3.694)--(8.815,3.703)--(8.825,3.713)--(8.835,3.723)--(8.846,3.734)%
    --(8.856,3.744)--(8.866,3.755)--(8.876,3.766)--(8.887,3.777)--(8.897,3.788)%
    --(8.907,3.798)--(8.918,3.808)--(8.928,3.818)--(8.938,3.828)--(8.948,3.837)%
    --(8.959,3.846)--(8.969,3.854)--(8.979,3.861)--(8.989,3.868)--(9.000,3.874)%
    --(9.010,3.879)--(9.020,3.884)--(9.031,3.887)--(9.041,3.889)--(9.051,3.890)%
    --(9.061,3.890)--(9.072,3.889)--(9.082,3.886)--(9.092,3.882)--(9.102,3.877)%
    --(9.113,3.870)--(9.123,3.863)--(9.133,3.854)--(9.143,3.845)--(9.154,3.834)%
    --(9.164,3.823)--(9.174,3.811)--(9.185,3.799)--(9.195,3.785)--(9.205,3.772)%
    --(9.215,3.757)--(9.226,3.743)--(9.236,3.727)--(9.246,3.712)--(9.256,3.696)%
    --(9.267,3.681)--(9.277,3.665)--(9.287,3.649)--(9.298,3.633)--(9.308,3.617)%
    --(9.318,3.602)--(9.328,3.586)--(9.339,3.571)--(9.349,3.556)--(9.359,3.542)%
    --(9.369,3.528)--(9.380,3.515)--(9.390,3.503)--(9.400,3.491)--(9.410,3.480)%
    --(9.421,3.470)--(9.431,3.461)--(9.441,3.452)--(9.452,3.445)--(9.462,3.439)%
    --(9.472,3.434)--(9.482,3.430)--(9.493,3.428)--(9.503,3.426)--(9.513,3.426)%
    --(9.523,3.428)--(9.534,3.430)--(9.544,3.433)--(9.554,3.437)--(9.565,3.442)%
    --(9.575,3.447)--(9.585,3.453)--(9.595,3.460)--(9.606,3.467)--(9.616,3.475)%
    --(9.626,3.483)--(9.636,3.491)--(9.647,3.499)--(9.657,3.507)--(9.667,3.516)%
    --(9.677,3.524)--(9.688,3.532)--(9.698,3.540)--(9.708,3.548)--(9.719,3.555)%
    --(9.729,3.562)--(9.739,3.569)--(9.749,3.574)--(9.760,3.579)--(9.770,3.584)%
    --(9.780,3.587)--(9.790,3.590)--(9.801,3.591)--(9.811,3.591)--(9.821,3.591)%
    --(9.832,3.589)--(9.842,3.585)--(9.852,3.581)--(9.862,3.575)--(9.873,3.567)%
    --(9.883,3.557)--(9.893,3.546)--(9.903,3.534)--(9.914,3.519)--(9.924,3.503)%
    --(9.934,3.485)--(9.944,3.466)--(9.955,3.445)--(9.965,3.423)--(9.975,3.400)%
    --(9.986,3.375)--(9.996,3.350)--(10.006,3.323)--(10.016,3.296)--(10.027,3.267)%
    --(10.037,3.238)--(10.047,3.208)--(10.057,3.178)--(10.068,3.147)--(10.078,3.115)%
    --(10.088,3.083)--(10.099,3.051)--(10.109,3.019)--(10.119,2.986)--(10.129,2.954)%
    --(10.140,2.921)--(10.150,2.889)--(10.160,2.856)--(10.170,2.824)--(10.181,2.793)%
    --(10.191,2.761)--(10.201,2.730)--(10.211,2.700)--(10.222,2.671)--(10.232,2.642)%
    --(10.242,2.614)--(10.253,2.587)--(10.263,2.561)--(10.273,2.536)--(10.283,2.512)%
    --(10.294,2.489)--(10.304,2.468)--(10.314,2.447)--(10.324,2.429)--(10.335,2.411)%
    --(10.345,2.395)--(10.355,2.380)--(10.366,2.366)--(10.376,2.354)--(10.386,2.342)%
    --(10.396,2.331)--(10.407,2.322)--(10.417,2.313)--(10.427,2.305)--(10.437,2.298)%
    --(10.448,2.292)--(10.458,2.286)--(10.468,2.281)--(10.478,2.277)--(10.489,2.273)%
    --(10.499,2.270)--(10.509,2.267)--(10.520,2.265)--(10.530,2.263)--(10.540,2.261)%
    --(10.550,2.260)--(10.561,2.259)--(10.571,2.258)--(10.581,2.257)--(10.591,2.257)%
    --(10.602,2.256)--(10.612,2.256)--(10.622,2.255)--(10.633,2.254)--(10.643,2.253)%
    --(10.653,2.252)--(10.663,2.251)--(10.674,2.249)--(10.684,2.247)--(10.694,2.245)%
    --(10.704,2.242)--(10.715,2.239)--(10.725,2.235)--(10.735,2.231)--(10.745,2.226)%
    --(10.756,2.221)--(10.766,2.215)--(10.776,2.209)--(10.787,2.202)--(10.797,2.195)%
    --(10.807,2.187)--(10.817,2.179)--(10.828,2.171)--(10.838,2.162)--(10.848,2.153)%
    --(10.858,2.144)--(10.869,2.134)--(10.879,2.124)--(10.889,2.113)--(10.900,2.102)%
    --(10.910,2.091)--(10.920,2.080)--(10.930,2.069)--(10.941,2.057)--(10.951,2.045)%
    --(10.961,2.033)--(10.971,2.020)--(10.982,2.008)--(10.992,1.995)--(11.002,1.982)%
    --(11.012,1.969)--(11.023,1.956)--(11.033,1.943)--(11.043,1.930)--(11.054,1.916)%
    --(11.064,1.903)--(11.074,1.889)--(11.084,1.876)--(11.095,1.862)--(11.105,1.849)%
    --(11.115,1.835)--(11.125,1.822)--(11.136,1.808)--(11.146,1.795)--(11.156,1.782)%
    --(11.167,1.768)--(11.177,1.755)--(11.187,1.742)--(11.197,1.729)--(11.208,1.716)%
    --(11.218,1.703)--(11.228,1.690)--(11.238,1.677)--(11.249,1.665)--(11.259,1.652)%
    --(11.269,1.639)--(11.279,1.627)--(11.290,1.615)--(11.300,1.602)--(11.310,1.590)%
    --(11.321,1.578)--(11.331,1.566)--(11.341,1.554)--(11.351,1.543)--(11.362,1.531)%
    --(11.372,1.519)--(11.382,1.508)--(11.392,1.497)--(11.403,1.486)--(11.413,1.475)%
    --(11.423,1.464)--(11.434,1.453)--(11.444,1.443)--(11.454,1.432)--(11.464,1.422)%
    --(11.475,1.412)--(11.485,1.402)--(11.495,1.392)--(11.505,1.383)--(11.516,1.373)%
    --(11.526,1.364)--(11.536,1.355)--(11.546,1.346)--(11.557,1.338)--(11.567,1.329)%
    --(11.577,1.321)--(11.588,1.312)--(11.598,1.304)--(11.608,1.296)--(11.618,1.289)%
    --(11.629,1.281)--(11.639,1.274)--(11.649,1.266)--(11.659,1.259)--(11.670,1.252)%
    --(11.680,1.245)--(11.690,1.238)--(11.701,1.232)--(11.711,1.225)--(11.721,1.218)%
    --(11.731,1.212)--(11.742,1.206)--(11.752,1.199)--(11.762,1.193)--(11.772,1.187)%
    --(11.783,1.181)--(11.793,1.175)--(11.803,1.170)--(11.813,1.164)--(11.824,1.158)%
    --(11.834,1.152)--(11.844,1.147)--(11.855,1.141)--(11.865,1.136)--(11.875,1.130)%
    --(11.885,1.125)--(11.896,1.119)--(11.906,1.114)--(11.916,1.108)--(11.926,1.103)%
    --(11.937,1.098)--(11.947,1.092)--(11.947,0.985)--(1.688,0.985)--cycle;
\draw[gp path] (1.688,1.003)--(1.698,1.003)--(1.709,1.003)--(1.719,1.002)--(1.729,1.002)%
  --(1.739,1.001)--(1.750,1.001)--(1.760,1.001)--(1.770,1.000)--(1.780,1.000)--(1.791,0.999)%
  --(1.801,0.999)--(1.811,0.999)--(1.822,0.998)--(1.832,0.998)--(1.842,0.998)--(1.852,0.997)%
  --(1.863,0.997)--(1.873,0.997)--(1.883,0.997)--(1.893,0.996)--(1.904,0.996)--(1.914,0.996)%
  --(1.924,0.996)--(1.934,0.995)--(1.945,0.995)--(1.955,0.995)--(1.965,0.995)--(1.976,0.995)%
  --(1.986,0.995)--(1.996,0.995)--(2.006,0.995)--(2.017,0.995)--(2.027,0.995)--(2.037,0.995)%
  --(2.047,0.995)--(2.058,0.995)--(2.068,0.995)--(2.078,0.996)--(2.089,0.996)--(2.099,0.996)%
  --(2.109,0.996)--(2.119,0.997)--(2.130,0.997)--(2.140,0.998)--(2.150,0.998)--(2.160,0.998)%
  --(2.171,0.999)--(2.181,0.999)--(2.191,1.000)--(2.201,1.000)--(2.212,1.001)--(2.222,1.002)%
  --(2.232,1.002)--(2.243,1.003)--(2.253,1.004)--(2.263,1.004)--(2.273,1.005)--(2.284,1.006)%
  --(2.294,1.006)--(2.304,1.007)--(2.314,1.008)--(2.325,1.009)--(2.335,1.009)--(2.345,1.010)%
  --(2.356,1.011)--(2.366,1.012)--(2.376,1.012)--(2.386,1.013)--(2.397,1.014)--(2.407,1.015)%
  --(2.417,1.015)--(2.427,1.016)--(2.438,1.017)--(2.448,1.018)--(2.458,1.018)--(2.468,1.019)%
  --(2.479,1.020)--(2.489,1.021)--(2.499,1.021)--(2.510,1.022)--(2.520,1.023)--(2.530,1.023)%
  --(2.540,1.024)--(2.551,1.025)--(2.561,1.025)--(2.571,1.026)--(2.581,1.026)--(2.592,1.027)%
  --(2.602,1.028)--(2.612,1.028)--(2.623,1.029)--(2.633,1.029)--(2.643,1.030)--(2.653,1.030)%
  --(2.664,1.031)--(2.674,1.032)--(2.684,1.032)--(2.694,1.033)--(2.705,1.033)--(2.715,1.034)%
  --(2.725,1.034)--(2.735,1.035)--(2.746,1.036)--(2.756,1.036)--(2.766,1.037)--(2.777,1.037)%
  --(2.787,1.038)--(2.797,1.039)--(2.807,1.039)--(2.818,1.040)--(2.828,1.041)--(2.838,1.042)%
  --(2.848,1.042)--(2.859,1.043)--(2.869,1.044)--(2.879,1.045)--(2.890,1.045)--(2.900,1.046)%
  --(2.910,1.047)--(2.920,1.048)--(2.931,1.049)--(2.941,1.050)--(2.951,1.051)--(2.961,1.052)%
  --(2.972,1.053)--(2.982,1.054)--(2.992,1.055)--(3.002,1.056)--(3.013,1.057)--(3.023,1.058)%
  --(3.033,1.060)--(3.044,1.061)--(3.054,1.062)--(3.064,1.063)--(3.074,1.065)--(3.085,1.066)%
  --(3.095,1.068)--(3.105,1.069)--(3.115,1.071)--(3.126,1.072)--(3.136,1.074)--(3.146,1.075)%
  --(3.157,1.077)--(3.167,1.079)--(3.177,1.080)--(3.187,1.082)--(3.198,1.084)--(3.208,1.086)%
  --(3.218,1.088)--(3.228,1.089)--(3.239,1.091)--(3.249,1.093)--(3.259,1.095)--(3.269,1.098)%
  --(3.280,1.100)--(3.290,1.102)--(3.300,1.104)--(3.311,1.106)--(3.321,1.109)--(3.331,1.111)%
  --(3.341,1.114)--(3.352,1.116)--(3.362,1.119)--(3.372,1.121)--(3.382,1.124)--(3.393,1.126)%
  --(3.403,1.129)--(3.413,1.132)--(3.424,1.135)--(3.434,1.138)--(3.444,1.141)--(3.454,1.144)%
  --(3.465,1.147)--(3.475,1.151)--(3.485,1.154)--(3.495,1.158)--(3.506,1.161)--(3.516,1.165)%
  --(3.526,1.168)--(3.536,1.172)--(3.547,1.176)--(3.557,1.180)--(3.567,1.185)--(3.578,1.189)%
  --(3.588,1.193)--(3.598,1.198)--(3.608,1.202)--(3.619,1.207)--(3.629,1.212)--(3.639,1.217)%
  --(3.649,1.222)--(3.660,1.227)--(3.670,1.233)--(3.680,1.238)--(3.691,1.244)--(3.701,1.250)%
  --(3.711,1.256)--(3.721,1.262)--(3.732,1.268)--(3.742,1.275)--(3.752,1.281)--(3.762,1.288)%
  --(3.773,1.295)--(3.783,1.302)--(3.793,1.309)--(3.803,1.316)--(3.814,1.323)--(3.824,1.331)%
  --(3.834,1.338)--(3.845,1.346)--(3.855,1.353)--(3.865,1.361)--(3.875,1.368)--(3.886,1.376)%
  --(3.896,1.383)--(3.906,1.391)--(3.916,1.398)--(3.927,1.406)--(3.937,1.413)--(3.947,1.421)%
  --(3.958,1.428)--(3.968,1.435)--(3.978,1.442)--(3.988,1.449)--(3.999,1.456)--(4.009,1.463)%
  --(4.019,1.469)--(4.029,1.476)--(4.040,1.482)--(4.050,1.488)--(4.060,1.494)--(4.070,1.500)%
  --(4.081,1.505)--(4.091,1.511)--(4.101,1.516)--(4.112,1.520)--(4.122,1.525)--(4.132,1.529)%
  --(4.142,1.533)--(4.153,1.537)--(4.163,1.540)--(4.173,1.543)--(4.183,1.546)--(4.194,1.549)%
  --(4.204,1.551)--(4.214,1.554)--(4.225,1.556)--(4.235,1.558)--(4.245,1.559)--(4.255,1.561)%
  --(4.266,1.563)--(4.276,1.564)--(4.286,1.566)--(4.296,1.568)--(4.307,1.569)--(4.317,1.571)%
  --(4.327,1.573)--(4.337,1.574)--(4.348,1.576)--(4.358,1.578)--(4.368,1.580)--(4.379,1.583)%
  --(4.389,1.585)--(4.399,1.588)--(4.409,1.591)--(4.420,1.594)--(4.430,1.598)--(4.440,1.601)%
  --(4.450,1.606)--(4.461,1.610)--(4.471,1.615)--(4.481,1.620)--(4.492,1.626)--(4.502,1.632)%
  --(4.512,1.639)--(4.522,1.646)--(4.533,1.653)--(4.543,1.661)--(4.553,1.670)--(4.563,1.679)%
  --(4.574,1.689)--(4.584,1.699)--(4.594,1.710)--(4.604,1.722)--(4.615,1.733)--(4.625,1.746)%
  --(4.635,1.758)--(4.646,1.772)--(4.656,1.785)--(4.666,1.799)--(4.676,1.813)--(4.687,1.828)%
  --(4.697,1.842)--(4.707,1.857)--(4.717,1.873)--(4.728,1.888)--(4.738,1.904)--(4.748,1.919)%
  --(4.759,1.935)--(4.769,1.951)--(4.779,1.967)--(4.789,1.983)--(4.800,1.999)--(4.810,2.015)%
  --(4.820,2.031)--(4.830,2.047)--(4.841,2.063)--(4.851,2.079)--(4.861,2.095)--(4.871,2.110)%
  --(4.882,2.125)--(4.892,2.140)--(4.902,2.155)--(4.913,2.170)--(4.923,2.184)--(4.933,2.198)%
  --(4.943,2.211)--(4.954,2.225)--(4.964,2.238)--(4.974,2.250)--(4.984,2.262)--(4.995,2.273)%
  --(5.005,2.285)--(5.015,2.295)--(5.026,2.306)--(5.036,2.316)--(5.046,2.325)--(5.056,2.334)%
  --(5.067,2.343)--(5.077,2.352)--(5.087,2.360)--(5.097,2.368)--(5.108,2.375)--(5.118,2.382)%
  --(5.128,2.389)--(5.138,2.395)--(5.149,2.402)--(5.159,2.408)--(5.169,2.413)--(5.180,2.419)%
  --(5.190,2.424)--(5.200,2.429)--(5.210,2.433)--(5.221,2.438)--(5.231,2.442)--(5.241,2.446)%
  --(5.251,2.449)--(5.262,2.453)--(5.272,2.456)--(5.282,2.460)--(5.293,2.463)--(5.303,2.465)%
  --(5.313,2.468)--(5.323,2.471)--(5.334,2.473)--(5.344,2.475)--(5.354,2.477)--(5.364,2.480)%
  --(5.375,2.481)--(5.385,2.483)--(5.395,2.485)--(5.405,2.487)--(5.416,2.489)--(5.426,2.490)%
  --(5.436,2.492)--(5.447,2.494)--(5.457,2.496)--(5.467,2.498)--(5.477,2.500)--(5.488,2.502)%
  --(5.498,2.505)--(5.508,2.508)--(5.518,2.511)--(5.529,2.515)--(5.539,2.518)--(5.549,2.522)%
  --(5.560,2.527)--(5.570,2.532)--(5.580,2.537)--(5.590,2.543)--(5.601,2.549)--(5.611,2.556)%
  --(5.621,2.564)--(5.631,2.572)--(5.642,2.581)--(5.652,2.590)--(5.662,2.600)--(5.672,2.611)%
  --(5.683,2.622)--(5.693,2.634)--(5.703,2.648)--(5.714,2.661)--(5.724,2.676)--(5.734,2.692)%
  --(5.744,2.708)--(5.755,2.726)--(5.765,2.744)--(5.775,2.764)--(5.785,2.784)--(5.796,2.806)%
  --(5.806,2.828)--(5.816,2.852)--(5.827,2.876)--(5.837,2.902)--(5.847,2.928)--(5.857,2.954)%
  --(5.868,2.981)--(5.878,3.009)--(5.888,3.037)--(5.898,3.065)--(5.909,3.094)--(5.919,3.123)%
  --(5.929,3.151)--(5.939,3.180)--(5.950,3.209)--(5.960,3.237)--(5.970,3.265)--(5.981,3.293)%
  --(5.991,3.321)--(6.001,3.348)--(6.011,3.374)--(6.022,3.399)--(6.032,3.424)--(6.042,3.448)%
  --(6.052,3.471)--(6.063,3.493)--(6.073,3.514)--(6.083,3.534)--(6.094,3.553)--(6.104,3.570)%
  --(6.114,3.586)--(6.124,3.600)--(6.135,3.613)--(6.145,3.624)--(6.155,3.633)--(6.165,3.641)%
  --(6.176,3.646)--(6.186,3.650)--(6.196,3.651)--(6.206,3.651)--(6.217,3.648)--(6.227,3.643)%
  --(6.237,3.636)--(6.248,3.628)--(6.258,3.617)--(6.268,3.605)--(6.278,3.592)--(6.289,3.577)%
  --(6.299,3.561)--(6.309,3.543)--(6.319,3.525)--(6.330,3.505)--(6.340,3.485)--(6.350,3.464)%
  --(6.361,3.442)--(6.371,3.420)--(6.381,3.397)--(6.391,3.374)--(6.402,3.350)--(6.412,3.327)%
  --(6.422,3.303)--(6.432,3.280)--(6.443,3.257)--(6.453,3.234)--(6.463,3.211)--(6.473,3.189)%
  --(6.484,3.168)--(6.494,3.147)--(6.504,3.128)--(6.515,3.109)--(6.525,3.091)--(6.535,3.074)%
  --(6.545,3.059)--(6.556,3.045)--(6.566,3.032)--(6.576,3.021)--(6.586,3.012)--(6.597,3.004)%
  --(6.607,2.999)--(6.617,2.995)--(6.628,2.994)--(6.638,2.994)--(6.648,2.996)--(6.658,3.000)%
  --(6.669,3.006)--(6.679,3.013)--(6.689,3.022)--(6.699,3.032)--(6.710,3.044)--(6.720,3.056)%
  --(6.730,3.070)--(6.740,3.085)--(6.751,3.101)--(6.761,3.118)--(6.771,3.136)--(6.782,3.155)%
  --(6.792,3.174)--(6.802,3.193)--(6.812,3.213)--(6.823,3.234)--(6.833,3.255)--(6.843,3.276)%
  --(6.853,3.297)--(6.864,3.318)--(6.874,3.339)--(6.884,3.360)--(6.895,3.381)--(6.905,3.402)%
  --(6.915,3.422)--(6.925,3.442)--(6.936,3.461)--(6.946,3.479)--(6.956,3.497)--(6.966,3.514)%
  --(6.977,3.530)--(6.987,3.545)--(6.997,3.559)--(7.007,3.572)--(7.018,3.583)--(7.028,3.594)%
  --(7.038,3.602)--(7.049,3.610)--(7.059,3.616)--(7.069,3.622)--(7.079,3.626)--(7.090,3.629)%
  --(7.100,3.631)--(7.110,3.632)--(7.120,3.632)--(7.131,3.631)--(7.141,3.630)--(7.151,3.628)%
  --(7.162,3.625)--(7.172,3.621)--(7.182,3.617)--(7.192,3.613)--(7.203,3.608)--(7.213,3.603)%
  --(7.223,3.597)--(7.233,3.591)--(7.244,3.585)--(7.254,3.579)--(7.264,3.573)--(7.274,3.566)%
  --(7.285,3.560)--(7.295,3.554)--(7.305,3.548)--(7.316,3.542)--(7.326,3.536)--(7.336,3.531)%
  --(7.346,3.526)--(7.357,3.522)--(7.367,3.518)--(7.377,3.514)--(7.387,3.511)--(7.398,3.509)%
  --(7.408,3.507)--(7.418,3.507)--(7.429,3.507)--(7.439,3.508)--(7.449,3.509)--(7.459,3.512)%
  --(7.470,3.516)--(7.480,3.520)--(7.490,3.525)--(7.500,3.531)--(7.511,3.538)--(7.521,3.545)%
  --(7.531,3.553)--(7.541,3.561)--(7.552,3.571)--(7.562,3.581)--(7.572,3.591)--(7.583,3.602)%
  --(7.593,3.613)--(7.603,3.625)--(7.613,3.638)--(7.624,3.650)--(7.634,3.664)--(7.644,3.677)%
  --(7.654,3.691)--(7.665,3.705)--(7.675,3.720)--(7.685,3.735)--(7.696,3.750)--(7.706,3.765)%
  --(7.716,3.781)--(7.726,3.796)--(7.737,3.812)--(7.747,3.828)--(7.757,3.844)--(7.767,3.860)%
  --(7.778,3.876)--(7.788,3.892)--(7.798,3.908)--(7.808,3.923)--(7.819,3.939)--(7.829,3.955)%
  --(7.839,3.971)--(7.850,3.986)--(7.860,4.001)--(7.870,4.016)--(7.880,4.031)--(7.891,4.045)%
  --(7.901,4.059)--(7.911,4.073)--(7.921,4.087)--(7.932,4.100)--(7.942,4.113)--(7.952,4.125)%
  --(7.963,4.137)--(7.973,4.149)--(7.983,4.160)--(7.993,4.171)--(8.004,4.181)--(8.014,4.191)%
  --(8.024,4.200)--(8.034,4.208)--(8.045,4.216)--(8.055,4.224)--(8.065,4.230)--(8.075,4.236)%
  --(8.086,4.242)--(8.096,4.246)--(8.106,4.250)--(8.117,4.253)--(8.127,4.256)--(8.137,4.257)%
  --(8.147,4.258)--(8.158,4.258)--(8.168,4.257)--(8.178,4.256)--(8.188,4.253)--(8.199,4.249)%
  --(8.209,4.245)--(8.219,4.239)--(8.230,4.233)--(8.240,4.225)--(8.250,4.217)--(8.260,4.207)%
  --(8.271,4.196)--(8.281,4.185)--(8.291,4.172)--(8.301,4.159)--(8.312,4.145)--(8.322,4.130)%
  --(8.332,4.115)--(8.342,4.099)--(8.353,4.082)--(8.363,4.065)--(8.373,4.047)--(8.384,4.029)%
  --(8.394,4.011)--(8.404,3.993)--(8.414,3.974)--(8.425,3.956)--(8.435,3.937)--(8.445,3.918)%
  --(8.455,3.900)--(8.466,3.881)--(8.476,3.863)--(8.486,3.845)--(8.497,3.827)--(8.507,3.810)%
  --(8.517,3.793)--(8.527,3.777)--(8.538,3.761)--(8.548,3.746)--(8.558,3.731)--(8.568,3.717)%
  --(8.579,3.705)--(8.589,3.692)--(8.599,3.681)--(8.609,3.671)--(8.620,3.662)--(8.630,3.654)%
  --(8.640,3.647)--(8.651,3.642)--(8.661,3.638)--(8.671,3.635)--(8.681,3.633)--(8.692,3.633)%
  --(8.702,3.634)--(8.712,3.636)--(8.722,3.639)--(8.733,3.643)--(8.743,3.648)--(8.753,3.654)%
  --(8.764,3.660)--(8.774,3.668)--(8.784,3.676)--(8.794,3.684)--(8.805,3.694)--(8.815,3.703)%
  --(8.825,3.713)--(8.835,3.723)--(8.846,3.734)--(8.856,3.744)--(8.866,3.755)--(8.876,3.766)%
  --(8.887,3.777)--(8.897,3.788)--(8.907,3.798)--(8.918,3.808)--(8.928,3.818)--(8.938,3.828)%
  --(8.948,3.837)--(8.959,3.846)--(8.969,3.854)--(8.979,3.861)--(8.989,3.868)--(9.000,3.874)%
  --(9.010,3.879)--(9.020,3.884)--(9.031,3.887)--(9.041,3.889)--(9.051,3.890)--(9.061,3.890)%
  --(9.072,3.889)--(9.082,3.886)--(9.092,3.882)--(9.102,3.877)--(9.113,3.870)--(9.123,3.863)%
  --(9.133,3.854)--(9.143,3.845)--(9.154,3.834)--(9.164,3.823)--(9.174,3.811)--(9.185,3.799)%
  --(9.195,3.785)--(9.205,3.772)--(9.215,3.757)--(9.226,3.743)--(9.236,3.727)--(9.246,3.712)%
  --(9.256,3.696)--(9.267,3.681)--(9.277,3.665)--(9.287,3.649)--(9.298,3.633)--(9.308,3.617)%
  --(9.318,3.602)--(9.328,3.586)--(9.339,3.571)--(9.349,3.556)--(9.359,3.542)--(9.369,3.528)%
  --(9.380,3.515)--(9.390,3.503)--(9.400,3.491)--(9.410,3.480)--(9.421,3.470)--(9.431,3.461)%
  --(9.441,3.452)--(9.452,3.445)--(9.462,3.439)--(9.472,3.434)--(9.482,3.430)--(9.493,3.428)%
  --(9.503,3.426)--(9.513,3.426)--(9.523,3.428)--(9.534,3.430)--(9.544,3.433)--(9.554,3.437)%
  --(9.565,3.442)--(9.575,3.447)--(9.585,3.453)--(9.595,3.460)--(9.606,3.467)--(9.616,3.475)%
  --(9.626,3.483)--(9.636,3.491)--(9.647,3.499)--(9.657,3.507)--(9.667,3.516)--(9.677,3.524)%
  --(9.688,3.532)--(9.698,3.540)--(9.708,3.548)--(9.719,3.555)--(9.729,3.562)--(9.739,3.569)%
  --(9.749,3.574)--(9.760,3.579)--(9.770,3.584)--(9.780,3.587)--(9.790,3.590)--(9.801,3.591)%
  --(9.811,3.591)--(9.821,3.591)--(9.832,3.589)--(9.842,3.585)--(9.852,3.581)--(9.862,3.575)%
  --(9.873,3.567)--(9.883,3.557)--(9.893,3.546)--(9.903,3.534)--(9.914,3.519)--(9.924,3.503)%
  --(9.934,3.485)--(9.944,3.466)--(9.955,3.445)--(9.965,3.423)--(9.975,3.400)--(9.986,3.375)%
  --(9.996,3.350)--(10.006,3.323)--(10.016,3.296)--(10.027,3.267)--(10.037,3.238)--(10.047,3.208)%
  --(10.057,3.178)--(10.068,3.147)--(10.078,3.115)--(10.088,3.083)--(10.099,3.051)--(10.109,3.019)%
  --(10.119,2.986)--(10.129,2.954)--(10.140,2.921)--(10.150,2.889)--(10.160,2.856)--(10.170,2.824)%
  --(10.181,2.793)--(10.191,2.761)--(10.201,2.730)--(10.211,2.700)--(10.222,2.671)--(10.232,2.642)%
  --(10.242,2.614)--(10.253,2.587)--(10.263,2.561)--(10.273,2.536)--(10.283,2.512)--(10.294,2.489)%
  --(10.304,2.468)--(10.314,2.447)--(10.324,2.429)--(10.335,2.411)--(10.345,2.395)--(10.355,2.380)%
  --(10.366,2.366)--(10.376,2.354)--(10.386,2.342)--(10.396,2.331)--(10.407,2.322)--(10.417,2.313)%
  --(10.427,2.305)--(10.437,2.298)--(10.448,2.292)--(10.458,2.286)--(10.468,2.281)--(10.478,2.277)%
  --(10.489,2.273)--(10.499,2.270)--(10.509,2.267)--(10.520,2.265)--(10.530,2.263)--(10.540,2.261)%
  --(10.550,2.260)--(10.561,2.259)--(10.571,2.258)--(10.581,2.257)--(10.591,2.257)--(10.602,2.256)%
  --(10.612,2.256)--(10.622,2.255)--(10.633,2.254)--(10.643,2.253)--(10.653,2.252)--(10.663,2.251)%
  --(10.674,2.249)--(10.684,2.247)--(10.694,2.245)--(10.704,2.242)--(10.715,2.239)--(10.725,2.235)%
  --(10.735,2.231)--(10.745,2.226)--(10.756,2.221)--(10.766,2.215)--(10.776,2.209)--(10.787,2.202)%
  --(10.797,2.195)--(10.807,2.187)--(10.817,2.179)--(10.828,2.171)--(10.838,2.162)--(10.848,2.153)%
  --(10.858,2.144)--(10.869,2.134)--(10.879,2.124)--(10.889,2.113)--(10.900,2.102)--(10.910,2.091)%
  --(10.920,2.080)--(10.930,2.069)--(10.941,2.057)--(10.951,2.045)--(10.961,2.033)--(10.971,2.020)%
  --(10.982,2.008)--(10.992,1.995)--(11.002,1.982)--(11.012,1.969)--(11.023,1.956)--(11.033,1.943)%
  --(11.043,1.930)--(11.054,1.916)--(11.064,1.903)--(11.074,1.889)--(11.084,1.876)--(11.095,1.862)%
  --(11.105,1.849)--(11.115,1.835)--(11.125,1.822)--(11.136,1.808)--(11.146,1.795)--(11.156,1.782)%
  --(11.167,1.768)--(11.177,1.755)--(11.187,1.742)--(11.197,1.729)--(11.208,1.716)--(11.218,1.703)%
  --(11.228,1.690)--(11.238,1.677)--(11.249,1.665)--(11.259,1.652)--(11.269,1.639)--(11.279,1.627)%
  --(11.290,1.615)--(11.300,1.602)--(11.310,1.590)--(11.321,1.578)--(11.331,1.566)--(11.341,1.554)%
  --(11.351,1.543)--(11.362,1.531)--(11.372,1.519)--(11.382,1.508)--(11.392,1.497)--(11.403,1.486)%
  --(11.413,1.475)--(11.423,1.464)--(11.434,1.453)--(11.444,1.443)--(11.454,1.432)--(11.464,1.422)%
  --(11.475,1.412)--(11.485,1.402)--(11.495,1.392)--(11.505,1.383)--(11.516,1.373)--(11.526,1.364)%
  --(11.536,1.355)--(11.546,1.346)--(11.557,1.338)--(11.567,1.329)--(11.577,1.321)--(11.588,1.312)%
  --(11.598,1.304)--(11.608,1.296)--(11.618,1.289)--(11.629,1.281)--(11.639,1.274)--(11.649,1.266)%
  --(11.659,1.259)--(11.670,1.252)--(11.680,1.245)--(11.690,1.238)--(11.701,1.232)--(11.711,1.225)%
  --(11.721,1.218)--(11.731,1.212)--(11.742,1.206)--(11.752,1.199)--(11.762,1.193)--(11.772,1.187)%
  --(11.783,1.181)--(11.793,1.175)--(11.803,1.170)--(11.813,1.164)--(11.824,1.158)--(11.834,1.152)%
  --(11.844,1.147)--(11.855,1.141)--(11.865,1.136)--(11.875,1.130)--(11.885,1.125)--(11.896,1.119)%
  --(11.906,1.114)--(11.916,1.108)--(11.926,1.103)--(11.937,1.098)--(11.947,1.092);
\gpcolor{color=gp lt color border}
\node[gp node right] at (4.448,7.739) {equality-based};
\gpfill{color=gp lt color 2,opacity=0.10} (4.632,7.662)--(5.548,7.662)--(5.548,7.816)--(4.632,7.816)--cycle;
\gpcolor{color=gp lt color 2}
\draw[gp path] (4.632,7.662)--(5.548,7.662)--(5.548,7.816)--(4.632,7.816)--cycle;
\gpfill{color=gp lt color 2,opacity=0.10} (1.688,1.003)--(1.688,1.003)--(1.698,1.003)--(1.709,1.003)%
    --(1.719,1.002)--(1.729,1.002)--(1.739,1.001)--(1.750,1.001)--(1.760,1.000)%
    --(1.770,1.000)--(1.780,1.000)--(1.791,0.999)--(1.801,0.999)--(1.811,0.998)%
    --(1.822,0.998)--(1.832,0.998)--(1.842,0.997)--(1.852,0.997)--(1.863,0.997)%
    --(1.873,0.997)--(1.883,0.996)--(1.893,0.996)--(1.904,0.996)--(1.914,0.996)%
    --(1.924,0.995)--(1.934,0.995)--(1.945,0.995)--(1.955,0.995)--(1.965,0.995)%
    --(1.976,0.995)--(1.986,0.995)--(1.996,0.995)--(2.006,0.995)--(2.017,0.995)%
    --(2.027,0.995)--(2.037,0.995)--(2.047,0.995)--(2.058,0.995)--(2.068,0.995)%
    --(2.078,0.996)--(2.089,0.996)--(2.099,0.996)--(2.109,0.996)--(2.119,0.997)%
    --(2.130,0.997)--(2.140,0.998)--(2.150,0.998)--(2.160,0.999)--(2.171,0.999)%
    --(2.181,1.000)--(2.191,1.000)--(2.201,1.001)--(2.212,1.002)--(2.222,1.002)%
    --(2.232,1.003)--(2.243,1.004)--(2.253,1.004)--(2.263,1.005)--(2.273,1.006)%
    --(2.284,1.007)--(2.294,1.007)--(2.304,1.008)--(2.314,1.009)--(2.325,1.010)%
    --(2.335,1.010)--(2.345,1.011)--(2.356,1.012)--(2.366,1.013)--(2.376,1.013)%
    --(2.386,1.014)--(2.397,1.015)--(2.407,1.016)--(2.417,1.016)--(2.427,1.017)%
    --(2.438,1.018)--(2.448,1.018)--(2.458,1.019)--(2.468,1.020)--(2.479,1.020)%
    --(2.489,1.021)--(2.499,1.021)--(2.510,1.022)--(2.520,1.023)--(2.530,1.023)%
    --(2.540,1.024)--(2.551,1.024)--(2.561,1.024)--(2.571,1.025)--(2.581,1.025)%
    --(2.592,1.026)--(2.602,1.026)--(2.612,1.026)--(2.623,1.027)--(2.633,1.027)%
    --(2.643,1.027)--(2.653,1.028)--(2.664,1.028)--(2.674,1.028)--(2.684,1.029)%
    --(2.694,1.029)--(2.705,1.029)--(2.715,1.030)--(2.725,1.030)--(2.735,1.030)%
    --(2.746,1.031)--(2.756,1.031)--(2.766,1.031)--(2.777,1.032)--(2.787,1.032)%
    --(2.797,1.033)--(2.807,1.033)--(2.818,1.034)--(2.828,1.034)--(2.838,1.035)%
    --(2.848,1.035)--(2.859,1.036)--(2.869,1.037)--(2.879,1.037)--(2.890,1.038)%
    --(2.900,1.039)--(2.910,1.040)--(2.920,1.041)--(2.931,1.041)--(2.941,1.042)%
    --(2.951,1.043)--(2.961,1.044)--(2.972,1.045)--(2.982,1.047)--(2.992,1.048)%
    --(3.002,1.049)--(3.013,1.050)--(3.023,1.051)--(3.033,1.053)--(3.044,1.054)%
    --(3.054,1.055)--(3.064,1.057)--(3.074,1.058)--(3.085,1.059)--(3.095,1.061)%
    --(3.105,1.062)--(3.115,1.064)--(3.126,1.065)--(3.136,1.067)--(3.146,1.069)%
    --(3.157,1.070)--(3.167,1.072)--(3.177,1.073)--(3.187,1.075)--(3.198,1.077)%
    --(3.208,1.078)--(3.218,1.080)--(3.228,1.082)--(3.239,1.084)--(3.249,1.085)%
    --(3.259,1.087)--(3.269,1.089)--(3.280,1.091)--(3.290,1.093)--(3.300,1.094)%
    --(3.311,1.096)--(3.321,1.098)--(3.331,1.100)--(3.341,1.102)--(3.352,1.104)%
    --(3.362,1.106)--(3.372,1.107)--(3.382,1.109)--(3.393,1.111)--(3.403,1.113)%
    --(3.413,1.115)--(3.424,1.117)--(3.434,1.119)--(3.444,1.121)--(3.454,1.124)%
    --(3.465,1.126)--(3.475,1.128)--(3.485,1.131)--(3.495,1.133)--(3.506,1.136)%
    --(3.516,1.138)--(3.526,1.141)--(3.536,1.144)--(3.547,1.146)--(3.557,1.149)%
    --(3.567,1.152)--(3.578,1.155)--(3.588,1.159)--(3.598,1.162)--(3.608,1.166)%
    --(3.619,1.169)--(3.629,1.173)--(3.639,1.177)--(3.649,1.181)--(3.660,1.185)%
    --(3.670,1.189)--(3.680,1.193)--(3.691,1.198)--(3.701,1.202)--(3.711,1.207)%
    --(3.721,1.212)--(3.732,1.217)--(3.742,1.223)--(3.752,1.228)--(3.762,1.234)%
    --(3.773,1.240)--(3.783,1.246)--(3.793,1.252)--(3.803,1.258)--(3.814,1.264)%
    --(3.824,1.270)--(3.834,1.277)--(3.845,1.283)--(3.855,1.290)--(3.865,1.296)%
    --(3.875,1.303)--(3.886,1.310)--(3.896,1.316)--(3.906,1.323)--(3.916,1.330)%
    --(3.927,1.336)--(3.937,1.343)--(3.947,1.349)--(3.958,1.356)--(3.968,1.362)%
    --(3.978,1.368)--(3.988,1.375)--(3.999,1.381)--(4.009,1.387)--(4.019,1.393)%
    --(4.029,1.398)--(4.040,1.404)--(4.050,1.409)--(4.060,1.414)--(4.070,1.419)%
    --(4.081,1.424)--(4.091,1.429)--(4.101,1.433)--(4.112,1.438)--(4.122,1.441)%
    --(4.132,1.445)--(4.142,1.449)--(4.153,1.452)--(4.163,1.455)--(4.173,1.457)%
    --(4.183,1.459)--(4.194,1.462)--(4.204,1.463)--(4.214,1.465)--(4.225,1.467)%
    --(4.235,1.468)--(4.245,1.470)--(4.255,1.471)--(4.266,1.472)--(4.276,1.473)%
    --(4.286,1.474)--(4.296,1.475)--(4.307,1.476)--(4.317,1.477)--(4.327,1.478)%
    --(4.337,1.479)--(4.348,1.480)--(4.358,1.481)--(4.368,1.483)--(4.379,1.484)%
    --(4.389,1.486)--(4.399,1.488)--(4.409,1.490)--(4.420,1.492)--(4.430,1.494)%
    --(4.440,1.497)--(4.450,1.500)--(4.461,1.503)--(4.471,1.507)--(4.481,1.511)%
    --(4.492,1.515)--(4.502,1.520)--(4.512,1.525)--(4.522,1.530)--(4.533,1.536)%
    --(4.543,1.542)--(4.553,1.549)--(4.563,1.557)--(4.574,1.564)--(4.584,1.573)%
    --(4.594,1.581)--(4.604,1.590)--(4.615,1.600)--(4.625,1.610)--(4.635,1.620)%
    --(4.646,1.631)--(4.656,1.642)--(4.666,1.653)--(4.676,1.664)--(4.687,1.676)%
    --(4.697,1.688)--(4.707,1.700)--(4.717,1.713)--(4.728,1.725)--(4.738,1.738)%
    --(4.748,1.750)--(4.759,1.763)--(4.769,1.776)--(4.779,1.789)--(4.789,1.802)%
    --(4.800,1.814)--(4.810,1.827)--(4.820,1.840)--(4.830,1.853)--(4.841,1.865)%
    --(4.851,1.878)--(4.861,1.890)--(4.871,1.902)--(4.882,1.914)--(4.892,1.926)%
    --(4.902,1.937)--(4.913,1.948)--(4.923,1.959)--(4.933,1.970)--(4.943,1.980)%
    --(4.954,1.990)--(4.964,1.999)--(4.974,2.009)--(4.984,2.017)--(4.995,2.025)%
    --(5.005,2.033)--(5.015,2.041)--(5.026,2.048)--(5.036,2.055)--(5.046,2.061)%
    --(5.056,2.067)--(5.067,2.073)--(5.077,2.078)--(5.087,2.083)--(5.097,2.088)%
    --(5.108,2.093)--(5.118,2.097)--(5.128,2.101)--(5.138,2.105)--(5.149,2.108)%
    --(5.159,2.111)--(5.169,2.114)--(5.180,2.117)--(5.190,2.120)--(5.200,2.122)%
    --(5.210,2.125)--(5.221,2.127)--(5.231,2.129)--(5.241,2.131)--(5.251,2.133)%
    --(5.262,2.134)--(5.272,2.136)--(5.282,2.137)--(5.293,2.139)--(5.303,2.140)%
    --(5.313,2.142)--(5.323,2.143)--(5.334,2.144)--(5.344,2.145)--(5.354,2.147)%
    --(5.364,2.148)--(5.375,2.149)--(5.385,2.150)--(5.395,2.152)--(5.405,2.153)%
    --(5.416,2.155)--(5.426,2.156)--(5.436,2.158)--(5.447,2.160)--(5.457,2.162)%
    --(5.467,2.164)--(5.477,2.166)--(5.488,2.169)--(5.498,2.172)--(5.508,2.175)%
    --(5.518,2.178)--(5.529,2.182)--(5.539,2.186)--(5.549,2.190)--(5.560,2.195)%
    --(5.570,2.200)--(5.580,2.205)--(5.590,2.211)--(5.601,2.218)--(5.611,2.224)%
    --(5.621,2.232)--(5.631,2.239)--(5.642,2.248)--(5.652,2.256)--(5.662,2.266)%
    --(5.672,2.276)--(5.683,2.286)--(5.693,2.297)--(5.703,2.309)--(5.714,2.321)%
    --(5.724,2.334)--(5.734,2.348)--(5.744,2.362)--(5.755,2.377)--(5.765,2.393)%
    --(5.775,2.410)--(5.785,2.427)--(5.796,2.446)--(5.806,2.465)--(5.816,2.485)%
    --(5.827,2.505)--(5.837,2.526)--(5.847,2.548)--(5.857,2.570)--(5.868,2.592)%
    --(5.878,2.615)--(5.888,2.639)--(5.898,2.663)--(5.909,2.686)--(5.919,2.711)%
    --(5.929,2.735)--(5.939,2.759)--(5.950,2.783)--(5.960,2.808)--(5.970,2.832)%
    --(5.981,2.856)--(5.991,2.879)--(6.001,2.903)--(6.011,2.926)--(6.022,2.949)%
    --(6.032,2.971)--(6.042,2.993)--(6.052,3.015)--(6.063,3.036)--(6.073,3.056)%
    --(6.083,3.075)--(6.094,3.094)--(6.104,3.112)--(6.114,3.128)--(6.124,3.144)%
    --(6.135,3.160)--(6.145,3.173)--(6.155,3.186)--(6.165,3.198)--(6.176,3.209)%
    --(6.186,3.218)--(6.196,3.226)--(6.206,3.232)--(6.217,3.238)--(6.227,3.241)%
    --(6.237,3.244)--(6.248,3.246)--(6.258,3.246)--(6.268,3.245)--(6.278,3.243)%
    --(6.289,3.241)--(6.299,3.237)--(6.309,3.233)--(6.319,3.228)--(6.330,3.222)%
    --(6.340,3.216)--(6.350,3.209)--(6.361,3.202)--(6.371,3.194)--(6.381,3.186)%
    --(6.391,3.178)--(6.402,3.169)--(6.412,3.161)--(6.422,3.152)--(6.432,3.143)%
    --(6.443,3.134)--(6.453,3.126)--(6.463,3.117)--(6.473,3.109)--(6.484,3.101)%
    --(6.494,3.094)--(6.504,3.087)--(6.515,3.080)--(6.525,3.074)--(6.535,3.069)%
    --(6.545,3.064)--(6.556,3.060)--(6.566,3.057)--(6.576,3.055)--(6.586,3.054)%
    --(6.597,3.054)--(6.607,3.055)--(6.617,3.057)--(6.628,3.060)--(6.638,3.065)%
    --(6.648,3.070)--(6.658,3.076)--(6.669,3.084)--(6.679,3.092)--(6.689,3.101)%
    --(6.699,3.110)--(6.710,3.121)--(6.720,3.132)--(6.730,3.143)--(6.740,3.155)%
    --(6.751,3.168)--(6.761,3.181)--(6.771,3.194)--(6.782,3.207)--(6.792,3.221)%
    --(6.802,3.234)--(6.812,3.248)--(6.823,3.262)--(6.833,3.275)--(6.843,3.289)%
    --(6.853,3.302)--(6.864,3.316)--(6.874,3.329)--(6.884,3.341)--(6.895,3.353)%
    --(6.905,3.365)--(6.915,3.376)--(6.925,3.386)--(6.936,3.396)--(6.946,3.405)%
    --(6.956,3.414)--(6.966,3.421)--(6.977,3.428)--(6.987,3.433)--(6.997,3.438)%
    --(7.007,3.441)--(7.018,3.443)--(7.028,3.445)--(7.038,3.445)--(7.049,3.443)%
    --(7.059,3.441)--(7.069,3.438)--(7.079,3.433)--(7.090,3.428)--(7.100,3.422)%
    --(7.110,3.416)--(7.120,3.408)--(7.131,3.400)--(7.141,3.392)--(7.151,3.383)%
    --(7.162,3.373)--(7.172,3.363)--(7.182,3.353)--(7.192,3.342)--(7.203,3.332)%
    --(7.213,3.321)--(7.223,3.310)--(7.233,3.300)--(7.244,3.289)--(7.254,3.278)%
    --(7.264,3.268)--(7.274,3.258)--(7.285,3.248)--(7.295,3.239)--(7.305,3.230)%
    --(7.316,3.222)--(7.326,3.214)--(7.336,3.207)--(7.346,3.201)--(7.357,3.195)%
    --(7.367,3.191)--(7.377,3.187)--(7.387,3.184)--(7.398,3.182)--(7.408,3.182)%
    --(7.418,3.182)--(7.429,3.184)--(7.439,3.187)--(7.449,3.192)--(7.459,3.197)%
    --(7.470,3.204)--(7.480,3.212)--(7.490,3.221)--(7.500,3.232)--(7.511,3.243)%
    --(7.521,3.255)--(7.531,3.268)--(7.541,3.282)--(7.552,3.297)--(7.562,3.313)%
    --(7.572,3.330)--(7.583,3.347)--(7.593,3.365)--(7.603,3.384)--(7.613,3.403)%
    --(7.624,3.422)--(7.634,3.443)--(7.644,3.463)--(7.654,3.484)--(7.665,3.506)%
    --(7.675,3.527)--(7.685,3.549)--(7.696,3.571)--(7.706,3.594)--(7.716,3.616)%
    --(7.726,3.639)--(7.737,3.661)--(7.747,3.684)--(7.757,3.707)--(7.767,3.729)%
    --(7.778,3.751)--(7.788,3.774)--(7.798,3.795)--(7.808,3.817)--(7.819,3.838)%
    --(7.829,3.859)--(7.839,3.880)--(7.850,3.900)--(7.860,3.920)--(7.870,3.939)%
    --(7.880,3.957)--(7.891,3.975)--(7.901,3.993)--(7.911,4.010)--(7.921,4.027)%
    --(7.932,4.043)--(7.942,4.058)--(7.952,4.073)--(7.963,4.088)--(7.973,4.102)%
    --(7.983,4.115)--(7.993,4.128)--(8.004,4.140)--(8.014,4.152)--(8.024,4.163)%
    --(8.034,4.173)--(8.045,4.183)--(8.055,4.192)--(8.065,4.201)--(8.075,4.209)%
    --(8.086,4.217)--(8.096,4.224)--(8.106,4.230)--(8.117,4.235)--(8.127,4.240)%
    --(8.137,4.245)--(8.147,4.248)--(8.158,4.251)--(8.168,4.254)--(8.178,4.255)%
    --(8.188,4.256)--(8.199,4.257)--(8.209,4.256)--(8.219,4.255)--(8.230,4.254)%
    --(8.240,4.251)--(8.250,4.248)--(8.260,4.244)--(8.271,4.239)--(8.281,4.234)%
    --(8.291,4.228)--(8.301,4.222)--(8.312,4.215)--(8.322,4.207)--(8.332,4.199)%
    --(8.342,4.191)--(8.353,4.182)--(8.363,4.173)--(8.373,4.163)--(8.384,4.153)%
    --(8.394,4.143)--(8.404,4.133)--(8.414,4.123)--(8.425,4.112)--(8.435,4.101)%
    --(8.445,4.091)--(8.455,4.080)--(8.466,4.069)--(8.476,4.058)--(8.486,4.048)%
    --(8.497,4.037)--(8.507,4.027)--(8.517,4.016)--(8.527,4.006)--(8.538,3.997)%
    --(8.548,3.987)--(8.558,3.978)--(8.568,3.970)--(8.579,3.961)--(8.589,3.954)%
    --(8.599,3.946)--(8.609,3.939)--(8.620,3.933)--(8.630,3.928)--(8.640,3.923)%
    --(8.651,3.918)--(8.661,3.915)--(8.671,3.912)--(8.681,3.910)--(8.692,3.908)%
    --(8.702,3.908)--(8.712,3.908)--(8.722,3.908)--(8.733,3.909)--(8.743,3.911)%
    --(8.753,3.913)--(8.764,3.916)--(8.774,3.919)--(8.784,3.922)--(8.794,3.926)%
    --(8.805,3.930)--(8.815,3.934)--(8.825,3.939)--(8.835,3.943)--(8.846,3.948)%
    --(8.856,3.954)--(8.866,3.959)--(8.876,3.964)--(8.887,3.969)--(8.897,3.975)%
    --(8.907,3.980)--(8.918,3.985)--(8.928,3.990)--(8.938,3.995)--(8.948,4.000)%
    --(8.959,4.005)--(8.969,4.009)--(8.979,4.013)--(8.989,4.017)--(9.000,4.021)%
    --(9.010,4.024)--(9.020,4.026)--(9.031,4.029)--(9.041,4.030)--(9.051,4.032)%
    --(9.061,4.032)--(9.072,4.032)--(9.082,4.032)--(9.092,4.030)--(9.102,4.029)%
    --(9.113,4.026)--(9.123,4.024)--(9.133,4.020)--(9.143,4.017)--(9.154,4.012)%
    --(9.164,4.008)--(9.174,4.003)--(9.185,3.998)--(9.195,3.992)--(9.205,3.986)%
    --(9.215,3.981)--(9.226,3.974)--(9.236,3.968)--(9.246,3.962)--(9.256,3.955)%
    --(9.267,3.949)--(9.277,3.942)--(9.287,3.936)--(9.298,3.929)--(9.308,3.923)%
    --(9.318,3.916)--(9.328,3.910)--(9.339,3.904)--(9.349,3.898)--(9.359,3.893)%
    --(9.369,3.888)--(9.380,3.883)--(9.390,3.878)--(9.400,3.874)--(9.410,3.870)%
    --(9.421,3.867)--(9.431,3.864)--(9.441,3.862)--(9.452,3.860)--(9.462,3.859)%
    --(9.472,3.858)--(9.482,3.858)--(9.493,3.859)--(9.503,3.860)--(9.513,3.862)%
    --(9.523,3.865)--(9.534,3.868)--(9.544,3.871)--(9.554,3.875)--(9.565,3.879)%
    --(9.575,3.884)--(9.585,3.889)--(9.595,3.894)--(9.606,3.899)--(9.616,3.904)%
    --(9.626,3.910)--(9.636,3.915)--(9.647,3.920)--(9.657,3.926)--(9.667,3.931)%
    --(9.677,3.936)--(9.688,3.941)--(9.698,3.945)--(9.708,3.949)--(9.719,3.953)%
    --(9.729,3.957)--(9.739,3.960)--(9.749,3.962)--(9.760,3.964)--(9.770,3.965)%
    --(9.780,3.966)--(9.790,3.966)--(9.801,3.965)--(9.811,3.963)--(9.821,3.961)%
    --(9.832,3.958)--(9.842,3.953)--(9.852,3.948)--(9.862,3.942)--(9.873,3.934)%
    --(9.883,3.926)--(9.893,3.916)--(9.903,3.905)--(9.914,3.893)--(9.924,3.879)%
    --(9.934,3.865)--(9.944,3.849)--(9.955,3.833)--(9.965,3.815)--(9.975,3.797)%
    --(9.986,3.778)--(9.996,3.758)--(10.006,3.737)--(10.016,3.715)--(10.027,3.693)%
    --(10.037,3.670)--(10.047,3.646)--(10.057,3.622)--(10.068,3.597)--(10.078,3.572)%
    --(10.088,3.547)--(10.099,3.521)--(10.109,3.495)--(10.119,3.468)--(10.129,3.442)%
    --(10.140,3.415)--(10.150,3.388)--(10.160,3.361)--(10.170,3.334)--(10.181,3.307)%
    --(10.191,3.280)--(10.201,3.253)--(10.211,3.226)--(10.222,3.200)--(10.232,3.173)%
    --(10.242,3.147)--(10.253,3.122)--(10.263,3.096)--(10.273,3.072)--(10.283,3.047)%
    --(10.294,3.024)--(10.304,3.000)--(10.314,2.978)--(10.324,2.956)--(10.335,2.934)%
    --(10.345,2.914)--(10.355,2.893)--(10.366,2.874)--(10.376,2.855)--(10.386,2.836)%
    --(10.396,2.818)--(10.407,2.800)--(10.417,2.783)--(10.427,2.766)--(10.437,2.750)%
    --(10.448,2.734)--(10.458,2.719)--(10.468,2.704)--(10.478,2.689)--(10.489,2.675)%
    --(10.499,2.660)--(10.509,2.647)--(10.520,2.633)--(10.530,2.620)--(10.540,2.607)%
    --(10.550,2.594)--(10.561,2.582)--(10.571,2.570)--(10.581,2.558)--(10.591,2.546)%
    --(10.602,2.534)--(10.612,2.522)--(10.622,2.511)--(10.633,2.499)--(10.643,2.488)%
    --(10.653,2.477)--(10.663,2.466)--(10.674,2.455)--(10.684,2.444)--(10.694,2.432)%
    --(10.704,2.421)--(10.715,2.410)--(10.725,2.399)--(10.735,2.388)--(10.745,2.376)%
    --(10.756,2.365)--(10.766,2.353)--(10.776,2.342)--(10.787,2.330)--(10.797,2.319)%
    --(10.807,2.307)--(10.817,2.295)--(10.828,2.284)--(10.838,2.272)--(10.848,2.260)%
    --(10.858,2.248)--(10.869,2.236)--(10.879,2.224)--(10.889,2.212)--(10.900,2.199)%
    --(10.910,2.187)--(10.920,2.175)--(10.930,2.162)--(10.941,2.150)--(10.951,2.137)%
    --(10.961,2.125)--(10.971,2.112)--(10.982,2.100)--(10.992,2.087)--(11.002,2.074)%
    --(11.012,2.061)--(11.023,2.048)--(11.033,2.035)--(11.043,2.022)--(11.054,2.009)%
    --(11.064,1.995)--(11.074,1.982)--(11.084,1.969)--(11.095,1.955)--(11.105,1.942)%
    --(11.115,1.928)--(11.125,1.914)--(11.136,1.901)--(11.146,1.887)--(11.156,1.873)%
    --(11.167,1.859)--(11.177,1.845)--(11.187,1.831)--(11.197,1.817)--(11.208,1.803)%
    --(11.218,1.789)--(11.228,1.775)--(11.238,1.761)--(11.249,1.746)--(11.259,1.732)%
    --(11.269,1.718)--(11.279,1.705)--(11.290,1.691)--(11.300,1.677)--(11.310,1.663)%
    --(11.321,1.649)--(11.331,1.636)--(11.341,1.622)--(11.351,1.609)--(11.362,1.595)%
    --(11.372,1.582)--(11.382,1.569)--(11.392,1.556)--(11.403,1.543)--(11.413,1.530)%
    --(11.423,1.518)--(11.434,1.505)--(11.444,1.493)--(11.454,1.481)--(11.464,1.469)%
    --(11.475,1.458)--(11.485,1.446)--(11.495,1.435)--(11.505,1.424)--(11.516,1.413)%
    --(11.526,1.403)--(11.536,1.392)--(11.546,1.382)--(11.557,1.372)--(11.567,1.363)%
    --(11.577,1.353)--(11.588,1.344)--(11.598,1.335)--(11.608,1.327)--(11.618,1.318)%
    --(11.629,1.310)--(11.639,1.302)--(11.649,1.294)--(11.659,1.286)--(11.670,1.278)%
    --(11.680,1.271)--(11.690,1.264)--(11.701,1.257)--(11.711,1.250)--(11.721,1.243)%
    --(11.731,1.237)--(11.742,1.230)--(11.752,1.224)--(11.762,1.217)--(11.772,1.211)%
    --(11.783,1.205)--(11.793,1.199)--(11.803,1.193)--(11.813,1.188)--(11.824,1.182)%
    --(11.834,1.176)--(11.844,1.171)--(11.855,1.165)--(11.865,1.160)--(11.875,1.155)%
    --(11.885,1.149)--(11.896,1.144)--(11.906,1.139)--(11.916,1.134)--(11.926,1.128)%
    --(11.937,1.123)--(11.947,1.118)--(11.947,0.985)--(1.688,0.985)--cycle;
\draw[gp path] (1.688,1.003)--(1.698,1.003)--(1.709,1.003)--(1.719,1.002)--(1.729,1.002)%
  --(1.739,1.001)--(1.750,1.001)--(1.760,1.000)--(1.770,1.000)--(1.780,1.000)--(1.791,0.999)%
  --(1.801,0.999)--(1.811,0.998)--(1.822,0.998)--(1.832,0.998)--(1.842,0.997)--(1.852,0.997)%
  --(1.863,0.997)--(1.873,0.997)--(1.883,0.996)--(1.893,0.996)--(1.904,0.996)--(1.914,0.996)%
  --(1.924,0.995)--(1.934,0.995)--(1.945,0.995)--(1.955,0.995)--(1.965,0.995)--(1.976,0.995)%
  --(1.986,0.995)--(1.996,0.995)--(2.006,0.995)--(2.017,0.995)--(2.027,0.995)--(2.037,0.995)%
  --(2.047,0.995)--(2.058,0.995)--(2.068,0.995)--(2.078,0.996)--(2.089,0.996)--(2.099,0.996)%
  --(2.109,0.996)--(2.119,0.997)--(2.130,0.997)--(2.140,0.998)--(2.150,0.998)--(2.160,0.999)%
  --(2.171,0.999)--(2.181,1.000)--(2.191,1.000)--(2.201,1.001)--(2.212,1.002)--(2.222,1.002)%
  --(2.232,1.003)--(2.243,1.004)--(2.253,1.004)--(2.263,1.005)--(2.273,1.006)--(2.284,1.007)%
  --(2.294,1.007)--(2.304,1.008)--(2.314,1.009)--(2.325,1.010)--(2.335,1.010)--(2.345,1.011)%
  --(2.356,1.012)--(2.366,1.013)--(2.376,1.013)--(2.386,1.014)--(2.397,1.015)--(2.407,1.016)%
  --(2.417,1.016)--(2.427,1.017)--(2.438,1.018)--(2.448,1.018)--(2.458,1.019)--(2.468,1.020)%
  --(2.479,1.020)--(2.489,1.021)--(2.499,1.021)--(2.510,1.022)--(2.520,1.023)--(2.530,1.023)%
  --(2.540,1.024)--(2.551,1.024)--(2.561,1.024)--(2.571,1.025)--(2.581,1.025)--(2.592,1.026)%
  --(2.602,1.026)--(2.612,1.026)--(2.623,1.027)--(2.633,1.027)--(2.643,1.027)--(2.653,1.028)%
  --(2.664,1.028)--(2.674,1.028)--(2.684,1.029)--(2.694,1.029)--(2.705,1.029)--(2.715,1.030)%
  --(2.725,1.030)--(2.735,1.030)--(2.746,1.031)--(2.756,1.031)--(2.766,1.031)--(2.777,1.032)%
  --(2.787,1.032)--(2.797,1.033)--(2.807,1.033)--(2.818,1.034)--(2.828,1.034)--(2.838,1.035)%
  --(2.848,1.035)--(2.859,1.036)--(2.869,1.037)--(2.879,1.037)--(2.890,1.038)--(2.900,1.039)%
  --(2.910,1.040)--(2.920,1.041)--(2.931,1.041)--(2.941,1.042)--(2.951,1.043)--(2.961,1.044)%
  --(2.972,1.045)--(2.982,1.047)--(2.992,1.048)--(3.002,1.049)--(3.013,1.050)--(3.023,1.051)%
  --(3.033,1.053)--(3.044,1.054)--(3.054,1.055)--(3.064,1.057)--(3.074,1.058)--(3.085,1.059)%
  --(3.095,1.061)--(3.105,1.062)--(3.115,1.064)--(3.126,1.065)--(3.136,1.067)--(3.146,1.069)%
  --(3.157,1.070)--(3.167,1.072)--(3.177,1.073)--(3.187,1.075)--(3.198,1.077)--(3.208,1.078)%
  --(3.218,1.080)--(3.228,1.082)--(3.239,1.084)--(3.249,1.085)--(3.259,1.087)--(3.269,1.089)%
  --(3.280,1.091)--(3.290,1.093)--(3.300,1.094)--(3.311,1.096)--(3.321,1.098)--(3.331,1.100)%
  --(3.341,1.102)--(3.352,1.104)--(3.362,1.106)--(3.372,1.107)--(3.382,1.109)--(3.393,1.111)%
  --(3.403,1.113)--(3.413,1.115)--(3.424,1.117)--(3.434,1.119)--(3.444,1.121)--(3.454,1.124)%
  --(3.465,1.126)--(3.475,1.128)--(3.485,1.131)--(3.495,1.133)--(3.506,1.136)--(3.516,1.138)%
  --(3.526,1.141)--(3.536,1.144)--(3.547,1.146)--(3.557,1.149)--(3.567,1.152)--(3.578,1.155)%
  --(3.588,1.159)--(3.598,1.162)--(3.608,1.166)--(3.619,1.169)--(3.629,1.173)--(3.639,1.177)%
  --(3.649,1.181)--(3.660,1.185)--(3.670,1.189)--(3.680,1.193)--(3.691,1.198)--(3.701,1.202)%
  --(3.711,1.207)--(3.721,1.212)--(3.732,1.217)--(3.742,1.223)--(3.752,1.228)--(3.762,1.234)%
  --(3.773,1.240)--(3.783,1.246)--(3.793,1.252)--(3.803,1.258)--(3.814,1.264)--(3.824,1.270)%
  --(3.834,1.277)--(3.845,1.283)--(3.855,1.290)--(3.865,1.296)--(3.875,1.303)--(3.886,1.310)%
  --(3.896,1.316)--(3.906,1.323)--(3.916,1.330)--(3.927,1.336)--(3.937,1.343)--(3.947,1.349)%
  --(3.958,1.356)--(3.968,1.362)--(3.978,1.368)--(3.988,1.375)--(3.999,1.381)--(4.009,1.387)%
  --(4.019,1.393)--(4.029,1.398)--(4.040,1.404)--(4.050,1.409)--(4.060,1.414)--(4.070,1.419)%
  --(4.081,1.424)--(4.091,1.429)--(4.101,1.433)--(4.112,1.438)--(4.122,1.441)--(4.132,1.445)%
  --(4.142,1.449)--(4.153,1.452)--(4.163,1.455)--(4.173,1.457)--(4.183,1.459)--(4.194,1.462)%
  --(4.204,1.463)--(4.214,1.465)--(4.225,1.467)--(4.235,1.468)--(4.245,1.470)--(4.255,1.471)%
  --(4.266,1.472)--(4.276,1.473)--(4.286,1.474)--(4.296,1.475)--(4.307,1.476)--(4.317,1.477)%
  --(4.327,1.478)--(4.337,1.479)--(4.348,1.480)--(4.358,1.481)--(4.368,1.483)--(4.379,1.484)%
  --(4.389,1.486)--(4.399,1.488)--(4.409,1.490)--(4.420,1.492)--(4.430,1.494)--(4.440,1.497)%
  --(4.450,1.500)--(4.461,1.503)--(4.471,1.507)--(4.481,1.511)--(4.492,1.515)--(4.502,1.520)%
  --(4.512,1.525)--(4.522,1.530)--(4.533,1.536)--(4.543,1.542)--(4.553,1.549)--(4.563,1.557)%
  --(4.574,1.564)--(4.584,1.573)--(4.594,1.581)--(4.604,1.590)--(4.615,1.600)--(4.625,1.610)%
  --(4.635,1.620)--(4.646,1.631)--(4.656,1.642)--(4.666,1.653)--(4.676,1.664)--(4.687,1.676)%
  --(4.697,1.688)--(4.707,1.700)--(4.717,1.713)--(4.728,1.725)--(4.738,1.738)--(4.748,1.750)%
  --(4.759,1.763)--(4.769,1.776)--(4.779,1.789)--(4.789,1.802)--(4.800,1.814)--(4.810,1.827)%
  --(4.820,1.840)--(4.830,1.853)--(4.841,1.865)--(4.851,1.878)--(4.861,1.890)--(4.871,1.902)%
  --(4.882,1.914)--(4.892,1.926)--(4.902,1.937)--(4.913,1.948)--(4.923,1.959)--(4.933,1.970)%
  --(4.943,1.980)--(4.954,1.990)--(4.964,1.999)--(4.974,2.009)--(4.984,2.017)--(4.995,2.025)%
  --(5.005,2.033)--(5.015,2.041)--(5.026,2.048)--(5.036,2.055)--(5.046,2.061)--(5.056,2.067)%
  --(5.067,2.073)--(5.077,2.078)--(5.087,2.083)--(5.097,2.088)--(5.108,2.093)--(5.118,2.097)%
  --(5.128,2.101)--(5.138,2.105)--(5.149,2.108)--(5.159,2.111)--(5.169,2.114)--(5.180,2.117)%
  --(5.190,2.120)--(5.200,2.122)--(5.210,2.125)--(5.221,2.127)--(5.231,2.129)--(5.241,2.131)%
  --(5.251,2.133)--(5.262,2.134)--(5.272,2.136)--(5.282,2.137)--(5.293,2.139)--(5.303,2.140)%
  --(5.313,2.142)--(5.323,2.143)--(5.334,2.144)--(5.344,2.145)--(5.354,2.147)--(5.364,2.148)%
  --(5.375,2.149)--(5.385,2.150)--(5.395,2.152)--(5.405,2.153)--(5.416,2.155)--(5.426,2.156)%
  --(5.436,2.158)--(5.447,2.160)--(5.457,2.162)--(5.467,2.164)--(5.477,2.166)--(5.488,2.169)%
  --(5.498,2.172)--(5.508,2.175)--(5.518,2.178)--(5.529,2.182)--(5.539,2.186)--(5.549,2.190)%
  --(5.560,2.195)--(5.570,2.200)--(5.580,2.205)--(5.590,2.211)--(5.601,2.218)--(5.611,2.224)%
  --(5.621,2.232)--(5.631,2.239)--(5.642,2.248)--(5.652,2.256)--(5.662,2.266)--(5.672,2.276)%
  --(5.683,2.286)--(5.693,2.297)--(5.703,2.309)--(5.714,2.321)--(5.724,2.334)--(5.734,2.348)%
  --(5.744,2.362)--(5.755,2.377)--(5.765,2.393)--(5.775,2.410)--(5.785,2.427)--(5.796,2.446)%
  --(5.806,2.465)--(5.816,2.485)--(5.827,2.505)--(5.837,2.526)--(5.847,2.548)--(5.857,2.570)%
  --(5.868,2.592)--(5.878,2.615)--(5.888,2.639)--(5.898,2.663)--(5.909,2.686)--(5.919,2.711)%
  --(5.929,2.735)--(5.939,2.759)--(5.950,2.783)--(5.960,2.808)--(5.970,2.832)--(5.981,2.856)%
  --(5.991,2.879)--(6.001,2.903)--(6.011,2.926)--(6.022,2.949)--(6.032,2.971)--(6.042,2.993)%
  --(6.052,3.015)--(6.063,3.036)--(6.073,3.056)--(6.083,3.075)--(6.094,3.094)--(6.104,3.112)%
  --(6.114,3.128)--(6.124,3.144)--(6.135,3.160)--(6.145,3.173)--(6.155,3.186)--(6.165,3.198)%
  --(6.176,3.209)--(6.186,3.218)--(6.196,3.226)--(6.206,3.232)--(6.217,3.238)--(6.227,3.241)%
  --(6.237,3.244)--(6.248,3.246)--(6.258,3.246)--(6.268,3.245)--(6.278,3.243)--(6.289,3.241)%
  --(6.299,3.237)--(6.309,3.233)--(6.319,3.228)--(6.330,3.222)--(6.340,3.216)--(6.350,3.209)%
  --(6.361,3.202)--(6.371,3.194)--(6.381,3.186)--(6.391,3.178)--(6.402,3.169)--(6.412,3.161)%
  --(6.422,3.152)--(6.432,3.143)--(6.443,3.134)--(6.453,3.126)--(6.463,3.117)--(6.473,3.109)%
  --(6.484,3.101)--(6.494,3.094)--(6.504,3.087)--(6.515,3.080)--(6.525,3.074)--(6.535,3.069)%
  --(6.545,3.064)--(6.556,3.060)--(6.566,3.057)--(6.576,3.055)--(6.586,3.054)--(6.597,3.054)%
  --(6.607,3.055)--(6.617,3.057)--(6.628,3.060)--(6.638,3.065)--(6.648,3.070)--(6.658,3.076)%
  --(6.669,3.084)--(6.679,3.092)--(6.689,3.101)--(6.699,3.110)--(6.710,3.121)--(6.720,3.132)%
  --(6.730,3.143)--(6.740,3.155)--(6.751,3.168)--(6.761,3.181)--(6.771,3.194)--(6.782,3.207)%
  --(6.792,3.221)--(6.802,3.234)--(6.812,3.248)--(6.823,3.262)--(6.833,3.275)--(6.843,3.289)%
  --(6.853,3.302)--(6.864,3.316)--(6.874,3.329)--(6.884,3.341)--(6.895,3.353)--(6.905,3.365)%
  --(6.915,3.376)--(6.925,3.386)--(6.936,3.396)--(6.946,3.405)--(6.956,3.414)--(6.966,3.421)%
  --(6.977,3.428)--(6.987,3.433)--(6.997,3.438)--(7.007,3.441)--(7.018,3.443)--(7.028,3.445)%
  --(7.038,3.445)--(7.049,3.443)--(7.059,3.441)--(7.069,3.438)--(7.079,3.433)--(7.090,3.428)%
  --(7.100,3.422)--(7.110,3.416)--(7.120,3.408)--(7.131,3.400)--(7.141,3.392)--(7.151,3.383)%
  --(7.162,3.373)--(7.172,3.363)--(7.182,3.353)--(7.192,3.342)--(7.203,3.332)--(7.213,3.321)%
  --(7.223,3.310)--(7.233,3.300)--(7.244,3.289)--(7.254,3.278)--(7.264,3.268)--(7.274,3.258)%
  --(7.285,3.248)--(7.295,3.239)--(7.305,3.230)--(7.316,3.222)--(7.326,3.214)--(7.336,3.207)%
  --(7.346,3.201)--(7.357,3.195)--(7.367,3.191)--(7.377,3.187)--(7.387,3.184)--(7.398,3.182)%
  --(7.408,3.182)--(7.418,3.182)--(7.429,3.184)--(7.439,3.187)--(7.449,3.192)--(7.459,3.197)%
  --(7.470,3.204)--(7.480,3.212)--(7.490,3.221)--(7.500,3.232)--(7.511,3.243)--(7.521,3.255)%
  --(7.531,3.268)--(7.541,3.282)--(7.552,3.297)--(7.562,3.313)--(7.572,3.330)--(7.583,3.347)%
  --(7.593,3.365)--(7.603,3.384)--(7.613,3.403)--(7.624,3.422)--(7.634,3.443)--(7.644,3.463)%
  --(7.654,3.484)--(7.665,3.506)--(7.675,3.527)--(7.685,3.549)--(7.696,3.571)--(7.706,3.594)%
  --(7.716,3.616)--(7.726,3.639)--(7.737,3.661)--(7.747,3.684)--(7.757,3.707)--(7.767,3.729)%
  --(7.778,3.751)--(7.788,3.774)--(7.798,3.795)--(7.808,3.817)--(7.819,3.838)--(7.829,3.859)%
  --(7.839,3.880)--(7.850,3.900)--(7.860,3.920)--(7.870,3.939)--(7.880,3.957)--(7.891,3.975)%
  --(7.901,3.993)--(7.911,4.010)--(7.921,4.027)--(7.932,4.043)--(7.942,4.058)--(7.952,4.073)%
  --(7.963,4.088)--(7.973,4.102)--(7.983,4.115)--(7.993,4.128)--(8.004,4.140)--(8.014,4.152)%
  --(8.024,4.163)--(8.034,4.173)--(8.045,4.183)--(8.055,4.192)--(8.065,4.201)--(8.075,4.209)%
  --(8.086,4.217)--(8.096,4.224)--(8.106,4.230)--(8.117,4.235)--(8.127,4.240)--(8.137,4.245)%
  --(8.147,4.248)--(8.158,4.251)--(8.168,4.254)--(8.178,4.255)--(8.188,4.256)--(8.199,4.257)%
  --(8.209,4.256)--(8.219,4.255)--(8.230,4.254)--(8.240,4.251)--(8.250,4.248)--(8.260,4.244)%
  --(8.271,4.239)--(8.281,4.234)--(8.291,4.228)--(8.301,4.222)--(8.312,4.215)--(8.322,4.207)%
  --(8.332,4.199)--(8.342,4.191)--(8.353,4.182)--(8.363,4.173)--(8.373,4.163)--(8.384,4.153)%
  --(8.394,4.143)--(8.404,4.133)--(8.414,4.123)--(8.425,4.112)--(8.435,4.101)--(8.445,4.091)%
  --(8.455,4.080)--(8.466,4.069)--(8.476,4.058)--(8.486,4.048)--(8.497,4.037)--(8.507,4.027)%
  --(8.517,4.016)--(8.527,4.006)--(8.538,3.997)--(8.548,3.987)--(8.558,3.978)--(8.568,3.970)%
  --(8.579,3.961)--(8.589,3.954)--(8.599,3.946)--(8.609,3.939)--(8.620,3.933)--(8.630,3.928)%
  --(8.640,3.923)--(8.651,3.918)--(8.661,3.915)--(8.671,3.912)--(8.681,3.910)--(8.692,3.908)%
  --(8.702,3.908)--(8.712,3.908)--(8.722,3.908)--(8.733,3.909)--(8.743,3.911)--(8.753,3.913)%
  --(8.764,3.916)--(8.774,3.919)--(8.784,3.922)--(8.794,3.926)--(8.805,3.930)--(8.815,3.934)%
  --(8.825,3.939)--(8.835,3.943)--(8.846,3.948)--(8.856,3.954)--(8.866,3.959)--(8.876,3.964)%
  --(8.887,3.969)--(8.897,3.975)--(8.907,3.980)--(8.918,3.985)--(8.928,3.990)--(8.938,3.995)%
  --(8.948,4.000)--(8.959,4.005)--(8.969,4.009)--(8.979,4.013)--(8.989,4.017)--(9.000,4.021)%
  --(9.010,4.024)--(9.020,4.026)--(9.031,4.029)--(9.041,4.030)--(9.051,4.032)--(9.061,4.032)%
  --(9.072,4.032)--(9.082,4.032)--(9.092,4.030)--(9.102,4.029)--(9.113,4.026)--(9.123,4.024)%
  --(9.133,4.020)--(9.143,4.017)--(9.154,4.012)--(9.164,4.008)--(9.174,4.003)--(9.185,3.998)%
  --(9.195,3.992)--(9.205,3.986)--(9.215,3.981)--(9.226,3.974)--(9.236,3.968)--(9.246,3.962)%
  --(9.256,3.955)--(9.267,3.949)--(9.277,3.942)--(9.287,3.936)--(9.298,3.929)--(9.308,3.923)%
  --(9.318,3.916)--(9.328,3.910)--(9.339,3.904)--(9.349,3.898)--(9.359,3.893)--(9.369,3.888)%
  --(9.380,3.883)--(9.390,3.878)--(9.400,3.874)--(9.410,3.870)--(9.421,3.867)--(9.431,3.864)%
  --(9.441,3.862)--(9.452,3.860)--(9.462,3.859)--(9.472,3.858)--(9.482,3.858)--(9.493,3.859)%
  --(9.503,3.860)--(9.513,3.862)--(9.523,3.865)--(9.534,3.868)--(9.544,3.871)--(9.554,3.875)%
  --(9.565,3.879)--(9.575,3.884)--(9.585,3.889)--(9.595,3.894)--(9.606,3.899)--(9.616,3.904)%
  --(9.626,3.910)--(9.636,3.915)--(9.647,3.920)--(9.657,3.926)--(9.667,3.931)--(9.677,3.936)%
  --(9.688,3.941)--(9.698,3.945)--(9.708,3.949)--(9.719,3.953)--(9.729,3.957)--(9.739,3.960)%
  --(9.749,3.962)--(9.760,3.964)--(9.770,3.965)--(9.780,3.966)--(9.790,3.966)--(9.801,3.965)%
  --(9.811,3.963)--(9.821,3.961)--(9.832,3.958)--(9.842,3.953)--(9.852,3.948)--(9.862,3.942)%
  --(9.873,3.934)--(9.883,3.926)--(9.893,3.916)--(9.903,3.905)--(9.914,3.893)--(9.924,3.879)%
  --(9.934,3.865)--(9.944,3.849)--(9.955,3.833)--(9.965,3.815)--(9.975,3.797)--(9.986,3.778)%
  --(9.996,3.758)--(10.006,3.737)--(10.016,3.715)--(10.027,3.693)--(10.037,3.670)--(10.047,3.646)%
  --(10.057,3.622)--(10.068,3.597)--(10.078,3.572)--(10.088,3.547)--(10.099,3.521)--(10.109,3.495)%
  --(10.119,3.468)--(10.129,3.442)--(10.140,3.415)--(10.150,3.388)--(10.160,3.361)--(10.170,3.334)%
  --(10.181,3.307)--(10.191,3.280)--(10.201,3.253)--(10.211,3.226)--(10.222,3.200)--(10.232,3.173)%
  --(10.242,3.147)--(10.253,3.122)--(10.263,3.096)--(10.273,3.072)--(10.283,3.047)--(10.294,3.024)%
  --(10.304,3.000)--(10.314,2.978)--(10.324,2.956)--(10.335,2.934)--(10.345,2.914)--(10.355,2.893)%
  --(10.366,2.874)--(10.376,2.855)--(10.386,2.836)--(10.396,2.818)--(10.407,2.800)--(10.417,2.783)%
  --(10.427,2.766)--(10.437,2.750)--(10.448,2.734)--(10.458,2.719)--(10.468,2.704)--(10.478,2.689)%
  --(10.489,2.675)--(10.499,2.660)--(10.509,2.647)--(10.520,2.633)--(10.530,2.620)--(10.540,2.607)%
  --(10.550,2.594)--(10.561,2.582)--(10.571,2.570)--(10.581,2.558)--(10.591,2.546)--(10.602,2.534)%
  --(10.612,2.522)--(10.622,2.511)--(10.633,2.499)--(10.643,2.488)--(10.653,2.477)--(10.663,2.466)%
  --(10.674,2.455)--(10.684,2.444)--(10.694,2.432)--(10.704,2.421)--(10.715,2.410)--(10.725,2.399)%
  --(10.735,2.388)--(10.745,2.376)--(10.756,2.365)--(10.766,2.353)--(10.776,2.342)--(10.787,2.330)%
  --(10.797,2.319)--(10.807,2.307)--(10.817,2.295)--(10.828,2.284)--(10.838,2.272)--(10.848,2.260)%
  --(10.858,2.248)--(10.869,2.236)--(10.879,2.224)--(10.889,2.212)--(10.900,2.199)--(10.910,2.187)%
  --(10.920,2.175)--(10.930,2.162)--(10.941,2.150)--(10.951,2.137)--(10.961,2.125)--(10.971,2.112)%
  --(10.982,2.100)--(10.992,2.087)--(11.002,2.074)--(11.012,2.061)--(11.023,2.048)--(11.033,2.035)%
  --(11.043,2.022)--(11.054,2.009)--(11.064,1.995)--(11.074,1.982)--(11.084,1.969)--(11.095,1.955)%
  --(11.105,1.942)--(11.115,1.928)--(11.125,1.914)--(11.136,1.901)--(11.146,1.887)--(11.156,1.873)%
  --(11.167,1.859)--(11.177,1.845)--(11.187,1.831)--(11.197,1.817)--(11.208,1.803)--(11.218,1.789)%
  --(11.228,1.775)--(11.238,1.761)--(11.249,1.746)--(11.259,1.732)--(11.269,1.718)--(11.279,1.705)%
  --(11.290,1.691)--(11.300,1.677)--(11.310,1.663)--(11.321,1.649)--(11.331,1.636)--(11.341,1.622)%
  --(11.351,1.609)--(11.362,1.595)--(11.372,1.582)--(11.382,1.569)--(11.392,1.556)--(11.403,1.543)%
  --(11.413,1.530)--(11.423,1.518)--(11.434,1.505)--(11.444,1.493)--(11.454,1.481)--(11.464,1.469)%
  --(11.475,1.458)--(11.485,1.446)--(11.495,1.435)--(11.505,1.424)--(11.516,1.413)--(11.526,1.403)%
  --(11.536,1.392)--(11.546,1.382)--(11.557,1.372)--(11.567,1.363)--(11.577,1.353)--(11.588,1.344)%
  --(11.598,1.335)--(11.608,1.327)--(11.618,1.318)--(11.629,1.310)--(11.639,1.302)--(11.649,1.294)%
  --(11.659,1.286)--(11.670,1.278)--(11.680,1.271)--(11.690,1.264)--(11.701,1.257)--(11.711,1.250)%
  --(11.721,1.243)--(11.731,1.237)--(11.742,1.230)--(11.752,1.224)--(11.762,1.217)--(11.772,1.211)%
  --(11.783,1.205)--(11.793,1.199)--(11.803,1.193)--(11.813,1.188)--(11.824,1.182)--(11.834,1.176)%
  --(11.844,1.171)--(11.855,1.165)--(11.865,1.160)--(11.875,1.155)--(11.885,1.149)--(11.896,1.144)%
  --(11.906,1.139)--(11.916,1.134)--(11.926,1.128)--(11.937,1.123)--(11.947,1.118);
\gpcolor{color=gp lt color border}
\node[gp node right] at (4.448,7.431) {w/o data flow};
\gpfill{color=gp lt color 6,opacity=0.10} (4.632,7.354)--(5.548,7.354)--(5.548,7.508)--(4.632,7.508)--cycle;
\gpcolor{color=gp lt color 6}
\draw[gp path] (4.632,7.354)--(5.548,7.354)--(5.548,7.508)--(4.632,7.508)--cycle;
\gpfill{color=gp lt color 6,opacity=0.10} (2.098,0.992)--(2.098,0.992)--(2.108,0.993)--(2.118,0.993)%
    --(2.128,0.993)--(2.138,0.994)--(2.148,0.994)--(2.158,0.994)--(2.167,0.995)%
    --(2.177,0.995)--(2.187,0.995)--(2.197,0.995)--(2.207,0.996)--(2.217,0.996)%
    --(2.227,0.996)--(2.236,0.997)--(2.246,0.997)--(2.256,0.997)--(2.266,0.997)%
    --(2.276,0.998)--(2.286,0.998)--(2.296,0.998)--(2.305,0.998)--(2.315,0.998)%
    --(2.325,0.999)--(2.335,0.999)--(2.345,0.999)--(2.355,0.999)--(2.365,0.999)%
    --(2.374,0.999)--(2.384,1.000)--(2.394,1.000)--(2.404,1.000)--(2.414,1.000)%
    --(2.424,1.000)--(2.434,1.000)--(2.443,1.000)--(2.453,1.000)--(2.463,1.000)%
    --(2.473,1.000)--(2.483,1.000)--(2.493,1.000)--(2.503,1.000)--(2.512,1.000)%
    --(2.522,1.000)--(2.532,1.000)--(2.542,0.999)--(2.552,0.999)--(2.562,0.999)%
    --(2.572,0.999)--(2.581,0.999)--(2.591,0.998)--(2.601,0.998)--(2.611,0.998)%
    --(2.621,0.998)--(2.631,0.998)--(2.641,0.997)--(2.650,0.997)--(2.660,0.997)%
    --(2.670,0.997)--(2.680,0.997)--(2.690,0.996)--(2.700,0.996)--(2.710,0.996)%
    --(2.719,0.996)--(2.729,0.996)--(2.739,0.996)--(2.749,0.996)--(2.759,0.996)%
    --(2.769,0.996)--(2.779,0.996)--(2.788,0.996)--(2.798,0.996)--(2.808,0.996)%
    --(2.818,0.996)--(2.828,0.996)--(2.838,0.996)--(2.848,0.997)--(2.857,0.997)%
    --(2.867,0.997)--(2.877,0.998)--(2.887,0.998)--(2.897,0.999)--(2.907,0.999)%
    --(2.917,1.000)--(2.926,1.000)--(2.936,1.001)--(2.946,1.002)--(2.956,1.002)%
    --(2.966,1.003)--(2.976,1.004)--(2.986,1.005)--(2.995,1.006)--(3.005,1.007)%
    --(3.015,1.008)--(3.025,1.009)--(3.035,1.010)--(3.045,1.011)--(3.055,1.013)%
    --(3.064,1.014)--(3.074,1.015)--(3.084,1.016)--(3.094,1.018)--(3.104,1.019)%
    --(3.114,1.021)--(3.124,1.022)--(3.134,1.023)--(3.143,1.025)--(3.153,1.026)%
    --(3.163,1.028)--(3.173,1.029)--(3.183,1.031)--(3.193,1.033)--(3.203,1.034)%
    --(3.212,1.036)--(3.222,1.037)--(3.232,1.039)--(3.242,1.041)--(3.252,1.042)%
    --(3.262,1.044)--(3.272,1.046)--(3.281,1.047)--(3.291,1.049)--(3.301,1.050)%
    --(3.311,1.052)--(3.321,1.054)--(3.331,1.055)--(3.341,1.057)--(3.350,1.059)%
    --(3.360,1.060)--(3.370,1.062)--(3.380,1.064)--(3.390,1.065)--(3.400,1.067)%
    --(3.410,1.069)--(3.419,1.070)--(3.429,1.072)--(3.439,1.073)--(3.449,1.075)%
    --(3.459,1.076)--(3.469,1.078)--(3.479,1.079)--(3.488,1.081)--(3.498,1.082)%
    --(3.508,1.084)--(3.518,1.085)--(3.528,1.087)--(3.538,1.088)--(3.548,1.089)%
    --(3.557,1.091)--(3.567,1.092)--(3.577,1.093)--(3.587,1.095)--(3.597,1.096)%
    --(3.607,1.097)--(3.617,1.098)--(3.626,1.100)--(3.636,1.101)--(3.646,1.102)%
    --(3.656,1.103)--(3.666,1.104)--(3.676,1.105)--(3.686,1.106)--(3.695,1.107)%
    --(3.705,1.108)--(3.715,1.109)--(3.725,1.110)--(3.735,1.110)--(3.745,1.111)%
    --(3.755,1.112)--(3.764,1.113)--(3.774,1.113)--(3.784,1.114)--(3.794,1.114)%
    --(3.804,1.115)--(3.814,1.115)--(3.824,1.116)--(3.833,1.116)--(3.843,1.117)%
    --(3.853,1.117)--(3.863,1.117)--(3.873,1.118)--(3.883,1.118)--(3.893,1.118)%
    --(3.902,1.118)--(3.912,1.118)--(3.922,1.118)--(3.932,1.118)--(3.942,1.118)%
    --(3.952,1.118)--(3.962,1.118)--(3.971,1.118)--(3.981,1.118)--(3.991,1.118)%
    --(4.001,1.118)--(4.011,1.118)--(4.021,1.117)--(4.031,1.117)--(4.040,1.117)%
    --(4.050,1.116)--(4.060,1.116)--(4.070,1.115)--(4.080,1.115)--(4.090,1.114)%
    --(4.100,1.114)--(4.109,1.113)--(4.119,1.113)--(4.129,1.112)--(4.139,1.112)%
    --(4.149,1.111)--(4.159,1.110)--(4.169,1.109)--(4.179,1.109)--(4.188,1.108)%
    --(4.198,1.107)--(4.208,1.106)--(4.218,1.105)--(4.228,1.104)--(4.238,1.103)%
    --(4.248,1.103)--(4.257,1.102)--(4.267,1.101)--(4.277,1.100)--(4.287,1.099)%
    --(4.297,1.098)--(4.307,1.097)--(4.317,1.096)--(4.326,1.095)--(4.336,1.094)%
    --(4.346,1.093)--(4.356,1.092)--(4.366,1.091)--(4.376,1.090)--(4.386,1.089)%
    --(4.395,1.088)--(4.405,1.087)--(4.415,1.087)--(4.425,1.086)--(4.435,1.085)%
    --(4.445,1.084)--(4.455,1.083)--(4.464,1.083)--(4.474,1.082)--(4.484,1.081)%
    --(4.494,1.081)--(4.504,1.080)--(4.514,1.080)--(4.524,1.079)--(4.533,1.079)%
    --(4.543,1.078)--(4.553,1.078)--(4.563,1.077)--(4.573,1.077)--(4.583,1.077)%
    --(4.593,1.077)--(4.602,1.077)--(4.612,1.077)--(4.622,1.077)--(4.632,1.077)%
    --(4.642,1.077)--(4.652,1.077)--(4.662,1.077)--(4.671,1.078)--(4.681,1.078)%
    --(4.691,1.079)--(4.701,1.079)--(4.711,1.080)--(4.721,1.081)--(4.731,1.082)%
    --(4.740,1.083)--(4.750,1.084)--(4.760,1.085)--(4.770,1.086)--(4.780,1.088)%
    --(4.790,1.089)--(4.800,1.091)--(4.809,1.093)--(4.819,1.094)--(4.829,1.096)%
    --(4.839,1.099)--(4.849,1.101)--(4.859,1.103)--(4.869,1.106)--(4.878,1.108)%
    --(4.888,1.111)--(4.898,1.114)--(4.908,1.117)--(4.918,1.120)--(4.928,1.124)%
    --(4.938,1.127)--(4.947,1.131)--(4.957,1.135)--(4.967,1.139)--(4.977,1.143)%
    --(4.987,1.147)--(4.997,1.152)--(5.007,1.156)--(5.016,1.161)--(5.026,1.166)%
    --(5.036,1.171)--(5.046,1.176)--(5.056,1.181)--(5.066,1.186)--(5.076,1.191)%
    --(5.085,1.197)--(5.095,1.202)--(5.105,1.207)--(5.115,1.213)--(5.125,1.218)%
    --(5.135,1.224)--(5.145,1.229)--(5.154,1.234)--(5.164,1.240)--(5.174,1.245)%
    --(5.184,1.250)--(5.194,1.255)--(5.204,1.260)--(5.214,1.265)--(5.224,1.270)%
    --(5.233,1.275)--(5.243,1.279)--(5.253,1.284)--(5.263,1.288)--(5.273,1.292)%
    --(5.283,1.296)--(5.293,1.300)--(5.302,1.304)--(5.312,1.307)--(5.322,1.311)%
    --(5.332,1.314)--(5.342,1.317)--(5.352,1.319)--(5.362,1.321)--(5.371,1.323)%
    --(5.381,1.325)--(5.391,1.327)--(5.401,1.328)--(5.411,1.329)--(5.421,1.330)%
    --(5.431,1.330)--(5.440,1.330)--(5.450,1.331)--(5.460,1.331)--(5.470,1.330)%
    --(5.480,1.330)--(5.490,1.329)--(5.500,1.329)--(5.509,1.328)--(5.519,1.327)%
    --(5.529,1.327)--(5.539,1.326)--(5.549,1.325)--(5.559,1.324)--(5.569,1.323)%
    --(5.578,1.322)--(5.588,1.321)--(5.598,1.320)--(5.608,1.319)--(5.618,1.318)%
    --(5.628,1.318)--(5.638,1.317)--(5.647,1.316)--(5.657,1.316)--(5.667,1.316)%
    --(5.677,1.316)--(5.687,1.316)--(5.697,1.316)--(5.707,1.317)--(5.716,1.318)%
    --(5.726,1.319)--(5.736,1.320)--(5.746,1.322)--(5.756,1.323)--(5.766,1.325)%
    --(5.776,1.328)--(5.785,1.331)--(5.795,1.334)--(5.805,1.337)--(5.815,1.341)%
    --(5.825,1.345)--(5.835,1.350)--(5.845,1.354)--(5.854,1.359)--(5.864,1.364)%
    --(5.874,1.370)--(5.884,1.375)--(5.894,1.381)--(5.904,1.387)--(5.914,1.393)%
    --(5.923,1.399)--(5.933,1.405)--(5.943,1.412)--(5.953,1.418)--(5.963,1.424)%
    --(5.973,1.431)--(5.983,1.437)--(5.992,1.443)--(6.002,1.450)--(6.012,1.456)%
    --(6.022,1.462)--(6.032,1.468)--(6.042,1.474)--(6.052,1.480)--(6.061,1.485)%
    --(6.071,1.491)--(6.081,1.496)--(6.091,1.501)--(6.101,1.506)--(6.111,1.510)%
    --(6.121,1.515)--(6.130,1.519)--(6.140,1.522)--(6.150,1.525)--(6.160,1.528)%
    --(6.170,1.531)--(6.180,1.533)--(6.190,1.535)--(6.199,1.536)--(6.209,1.537)%
    --(6.219,1.537)--(6.229,1.537)--(6.239,1.536)--(6.249,1.535)--(6.259,1.534)%
    --(6.269,1.533)--(6.278,1.531)--(6.288,1.528)--(6.298,1.526)--(6.308,1.523)%
    --(6.318,1.520)--(6.328,1.517)--(6.338,1.514)--(6.347,1.511)--(6.357,1.507)%
    --(6.367,1.503)--(6.377,1.500)--(6.387,1.496)--(6.397,1.492)--(6.407,1.488)%
    --(6.416,1.484)--(6.426,1.480)--(6.436,1.477)--(6.446,1.473)--(6.456,1.469)%
    --(6.466,1.466)--(6.476,1.462)--(6.485,1.459)--(6.495,1.456)--(6.505,1.453)%
    --(6.515,1.451)--(6.525,1.449)--(6.535,1.447)--(6.545,1.445)--(6.554,1.444)%
    --(6.564,1.443)--(6.574,1.442)--(6.584,1.442)--(6.594,1.442)--(6.604,1.443)%
    --(6.614,1.444)--(6.623,1.445)--(6.633,1.447)--(6.643,1.450)--(6.653,1.453)%
    --(6.663,1.456)--(6.673,1.460)--(6.683,1.464)--(6.692,1.468)--(6.702,1.473)%
    --(6.712,1.479)--(6.722,1.484)--(6.732,1.490)--(6.742,1.496)--(6.752,1.502)%
    --(6.761,1.509)--(6.771,1.516)--(6.781,1.523)--(6.791,1.530)--(6.801,1.537)%
    --(6.811,1.545)--(6.821,1.553)--(6.830,1.561)--(6.840,1.569)--(6.850,1.577)%
    --(6.860,1.585)--(6.870,1.593)--(6.880,1.601)--(6.890,1.610)--(6.899,1.618)%
    --(6.909,1.626)--(6.919,1.635)--(6.929,1.643)--(6.939,1.651)--(6.949,1.660)%
    --(6.959,1.668)--(6.968,1.676)--(6.978,1.684)--(6.988,1.691)--(6.998,1.699)%
    --(7.008,1.706)--(7.018,1.714)--(7.028,1.721)--(7.037,1.728)--(7.047,1.734)%
    --(7.057,1.741)--(7.067,1.747)--(7.077,1.753)--(7.087,1.759)--(7.097,1.765)%
    --(7.106,1.770)--(7.116,1.776)--(7.126,1.781)--(7.136,1.786)--(7.146,1.791)%
    --(7.156,1.796)--(7.166,1.801)--(7.175,1.806)--(7.185,1.810)--(7.195,1.815)%
    --(7.205,1.819)--(7.215,1.823)--(7.225,1.828)--(7.235,1.832)--(7.244,1.836)%
    --(7.254,1.840)--(7.264,1.844)--(7.274,1.848)--(7.284,1.852)--(7.294,1.855)%
    --(7.304,1.859)--(7.314,1.863)--(7.323,1.867)--(7.333,1.870)--(7.343,1.874)%
    --(7.353,1.878)--(7.363,1.882)--(7.373,1.886)--(7.383,1.889)--(7.392,1.893)%
    --(7.402,1.897)--(7.412,1.901)--(7.422,1.905)--(7.432,1.909)--(7.442,1.913)%
    --(7.452,1.917)--(7.461,1.922)--(7.471,1.926)--(7.481,1.930)--(7.491,1.935)%
    --(7.501,1.940)--(7.511,1.945)--(7.521,1.950)--(7.530,1.956)--(7.540,1.961)%
    --(7.550,1.967)--(7.560,1.974)--(7.570,1.980)--(7.580,1.987)--(7.590,1.994)%
    --(7.599,2.002)--(7.609,2.010)--(7.619,2.018)--(7.629,2.027)--(7.639,2.036)%
    --(7.649,2.045)--(7.659,2.056)--(7.668,2.066)--(7.678,2.077)--(7.688,2.089)%
    --(7.698,2.101)--(7.708,2.113)--(7.718,2.127)--(7.728,2.140)--(7.737,2.155)%
    --(7.747,2.170)--(7.757,2.186)--(7.767,2.202)--(7.777,2.219)--(7.787,2.237)%
    --(7.797,2.255)--(7.806,2.274)--(7.816,2.294)--(7.826,2.315)--(7.836,2.337)%
    --(7.846,2.359)--(7.856,2.382)--(7.866,2.406)--(7.875,2.430)--(7.885,2.456)%
    --(7.895,2.481)--(7.905,2.508)--(7.915,2.535)--(7.925,2.562)--(7.935,2.589)%
    --(7.944,2.617)--(7.954,2.645)--(7.964,2.673)--(7.974,2.701)--(7.984,2.730)%
    --(7.994,2.758)--(8.004,2.786)--(8.013,2.814)--(8.023,2.842)--(8.033,2.870)%
    --(8.043,2.898)--(8.053,2.925)--(8.063,2.951)--(8.073,2.977)--(8.082,3.003)%
    --(8.092,3.028)--(8.102,3.053)--(8.112,3.076)--(8.122,3.099)--(8.132,3.121)%
    --(8.142,3.143)--(8.151,3.163)--(8.161,3.182)--(8.171,3.200)--(8.181,3.217)%
    --(8.191,3.233)--(8.201,3.248)--(8.211,3.262)--(8.220,3.274)--(8.230,3.284)%
    --(8.240,3.293)--(8.250,3.301)--(8.260,3.307)--(8.270,3.312)--(8.280,3.315)%
    --(8.289,3.317)--(8.299,3.317)--(8.309,3.316)--(8.319,3.314)--(8.329,3.311)%
    --(8.339,3.307)--(8.349,3.302)--(8.359,3.297)--(8.368,3.290)--(8.378,3.283)%
    --(8.388,3.276)--(8.398,3.267)--(8.408,3.259)--(8.418,3.250)--(8.428,3.241)%
    --(8.437,3.232)--(8.447,3.222)--(8.457,3.213)--(8.467,3.204)--(8.477,3.195)%
    --(8.487,3.186)--(8.497,3.178)--(8.506,3.170)--(8.516,3.162)--(8.526,3.155)%
    --(8.536,3.149)--(8.546,3.144)--(8.556,3.139)--(8.566,3.135)--(8.575,3.133)%
    --(8.585,3.131)--(8.595,3.131)--(8.605,3.131)--(8.615,3.133)--(8.625,3.137)%
    --(8.635,3.142)--(8.644,3.149)--(8.654,3.157)--(8.664,3.167)--(8.674,3.179)%
    --(8.684,3.192)--(8.694,3.207)--(8.704,3.224)--(8.713,3.242)--(8.723,3.262)%
    --(8.733,3.283)--(8.743,3.305)--(8.753,3.329)--(8.763,3.354)--(8.773,3.379)%
    --(8.782,3.406)--(8.792,3.433)--(8.802,3.462)--(8.812,3.491)--(8.822,3.520)%
    --(8.832,3.550)--(8.842,3.581)--(8.851,3.612)--(8.861,3.643)--(8.871,3.675)%
    --(8.881,3.707)--(8.891,3.739)--(8.901,3.770)--(8.911,3.802)--(8.920,3.834)%
    --(8.930,3.865)--(8.940,3.896)--(8.950,3.927)--(8.960,3.957)--(8.970,3.987)%
    --(8.980,4.016)--(8.989,4.044)--(8.999,4.071)--(9.009,4.098)--(9.019,4.124)%
    --(9.029,4.148)--(9.039,4.172)--(9.049,4.194)--(9.058,4.215)--(9.068,4.235)%
    --(9.078,4.253)--(9.088,4.270)--(9.098,4.286)--(9.108,4.300)--(9.118,4.313)%
    --(9.127,4.325)--(9.137,4.335)--(9.147,4.345)--(9.157,4.353)--(9.167,4.361)%
    --(9.177,4.368)--(9.187,4.374)--(9.196,4.379)--(9.206,4.384)--(9.216,4.388)%
    --(9.226,4.392)--(9.236,4.395)--(9.246,4.398)--(9.256,4.400)--(9.265,4.402)%
    --(9.275,4.404)--(9.285,4.406)--(9.295,4.408)--(9.305,4.410)--(9.315,4.412)%
    --(9.325,4.414)--(9.334,4.417)--(9.344,4.419)--(9.354,4.422)--(9.364,4.426)%
    --(9.374,4.430)--(9.384,4.434)--(9.394,4.439)--(9.404,4.445)--(9.413,4.451)%
    --(9.423,4.459)--(9.433,4.467)--(9.443,4.476)--(9.453,4.487)--(9.463,4.498)%
    --(9.473,4.510)--(9.482,4.524)--(9.492,4.539)--(9.502,4.555)--(9.512,4.573)%
    --(9.522,4.591)--(9.532,4.611)--(9.542,4.631)--(9.551,4.653)--(9.561,4.675)%
    --(9.571,4.698)--(9.581,4.722)--(9.591,4.746)--(9.601,4.771)--(9.611,4.796)%
    --(9.620,4.821)--(9.630,4.847)--(9.640,4.874)--(9.650,4.900)--(9.660,4.926)%
    --(9.670,4.953)--(9.680,4.979)--(9.689,5.005)--(9.699,5.031)--(9.709,5.057)%
    --(9.719,5.082)--(9.729,5.107)--(9.739,5.132)--(9.749,5.156)--(9.758,5.179)%
    --(9.768,5.202)--(9.778,5.224)--(9.788,5.244)--(9.798,5.264)--(9.808,5.283)%
    --(9.818,5.301)--(9.827,5.318)--(9.837,5.334)--(9.847,5.348)--(9.857,5.361)%
    --(9.867,5.372)--(9.877,5.382)--(9.887,5.391)--(9.896,5.397)--(9.906,5.402)%
    --(9.916,5.406)--(9.926,5.408)--(9.936,5.408)--(9.946,5.407)--(9.956,5.404)%
    --(9.965,5.400)--(9.975,5.395)--(9.985,5.389)--(9.995,5.381)--(10.005,5.373)%
    --(10.015,5.363)--(10.025,5.353)--(10.034,5.341)--(10.044,5.329)--(10.054,5.316)%
    --(10.064,5.303)--(10.074,5.288)--(10.084,5.274)--(10.094,5.258)--(10.103,5.243)%
    --(10.113,5.227)--(10.123,5.210)--(10.133,5.194)--(10.143,5.177)--(10.153,5.160)%
    --(10.163,5.143)--(10.172,5.126)--(10.182,5.110)--(10.192,5.093)--(10.202,5.077)%
    --(10.212,5.061)--(10.222,5.045)--(10.232,5.030)--(10.241,5.015)--(10.251,5.000)%
    --(10.261,4.987)--(10.271,4.974)--(10.281,4.961)--(10.291,4.950)--(10.301,4.939)%
    --(10.310,4.930)--(10.320,4.921)--(10.330,4.913)--(10.340,4.906)--(10.350,4.900)%
    --(10.360,4.895)--(10.370,4.890)--(10.379,4.887)--(10.389,4.884)--(10.399,4.881)%
    --(10.409,4.879)--(10.419,4.878)--(10.429,4.878)--(10.439,4.878)--(10.449,4.878)%
    --(10.458,4.879)--(10.468,4.881)--(10.478,4.883)--(10.488,4.885)--(10.498,4.887)%
    --(10.508,4.890)--(10.518,4.893)--(10.527,4.897)--(10.537,4.900)--(10.547,4.904)%
    --(10.557,4.908)--(10.567,4.912)--(10.577,4.916)--(10.587,4.920)--(10.596,4.924)%
    --(10.606,4.928)--(10.616,4.932)--(10.626,4.936)--(10.636,4.939)--(10.646,4.943)%
    --(10.656,4.946)--(10.665,4.949)--(10.675,4.952)--(10.685,4.955)--(10.695,4.957)%
    --(10.705,4.959)--(10.715,4.960)--(10.725,4.961)--(10.734,4.962)--(10.744,4.962)%
    --(10.754,4.962)--(10.764,4.961)--(10.774,4.960)--(10.784,4.958)--(10.794,4.956)%
    --(10.803,4.954)--(10.813,4.952)--(10.823,4.949)--(10.833,4.945)--(10.843,4.942)%
    --(10.853,4.938)--(10.863,4.933)--(10.872,4.929)--(10.882,4.924)--(10.892,4.918)%
    --(10.902,4.913)--(10.912,4.907)--(10.922,4.901)--(10.932,4.894)--(10.941,4.887)%
    --(10.951,4.881)--(10.961,4.873)--(10.971,4.866)--(10.981,4.858)--(10.991,4.850)%
    --(11.001,4.842)--(11.010,4.834)--(11.020,4.825)--(11.030,4.817)--(11.040,4.808)%
    --(11.050,4.799)--(11.060,4.790)--(11.070,4.780)--(11.079,4.771)--(11.089,4.761)%
    --(11.099,4.751)--(11.109,4.741)--(11.119,4.731)--(11.129,4.721)--(11.139,4.711)%
    --(11.148,4.700)--(11.158,4.690)--(11.168,4.679)--(11.178,4.668)--(11.188,4.657)%
    --(11.198,4.646)--(11.208,4.635)--(11.217,4.623)--(11.227,4.611)--(11.237,4.598)%
    --(11.247,4.586)--(11.257,4.573)--(11.267,4.560)--(11.277,4.546)--(11.286,4.532)%
    --(11.296,4.518)--(11.306,4.503)--(11.316,4.487)--(11.326,4.472)--(11.336,4.456)%
    --(11.346,4.439)--(11.355,4.422)--(11.365,4.404)--(11.375,4.386)--(11.385,4.367)%
    --(11.395,4.348)--(11.405,4.328)--(11.415,4.307)--(11.424,4.286)--(11.434,4.264)%
    --(11.444,4.242)--(11.454,4.219)--(11.464,4.195)--(11.474,4.170)--(11.484,4.145)%
    --(11.494,4.119)--(11.503,4.092)--(11.513,4.064)--(11.523,4.036)--(11.533,4.006)%
    --(11.543,3.976)--(11.553,3.945)--(11.563,3.913)--(11.572,3.881)--(11.582,3.848)%
    --(11.592,3.813)--(11.602,3.779)--(11.612,3.743)--(11.622,3.707)--(11.632,3.670)%
    --(11.641,3.632)--(11.651,3.594)--(11.661,3.555)--(11.671,3.516)--(11.681,3.476)%
    --(11.691,3.435)--(11.701,3.394)--(11.710,3.352)--(11.720,3.310)--(11.730,3.268)%
    --(11.740,3.225)--(11.750,3.181)--(11.760,3.137)--(11.770,3.093)--(11.779,3.048)%
    --(11.789,3.003)--(11.799,2.958)--(11.809,2.913)--(11.819,2.867)--(11.829,2.821)%
    --(11.839,2.774)--(11.848,2.728)--(11.858,2.681)--(11.868,2.634)--(11.878,2.587)%
    --(11.888,2.539)--(11.898,2.492)--(11.908,2.444)--(11.917,2.397)--(11.927,2.349)%
    --(11.937,2.301)--(11.947,2.253)--(11.947,0.985)--(2.098,0.985)--cycle;
\draw[gp path] (2.098,0.992)--(2.108,0.993)--(2.118,0.993)--(2.128,0.993)--(2.138,0.994)%
  --(2.148,0.994)--(2.158,0.994)--(2.167,0.995)--(2.177,0.995)--(2.187,0.995)--(2.197,0.995)%
  --(2.207,0.996)--(2.217,0.996)--(2.227,0.996)--(2.236,0.997)--(2.246,0.997)--(2.256,0.997)%
  --(2.266,0.997)--(2.276,0.998)--(2.286,0.998)--(2.296,0.998)--(2.305,0.998)--(2.315,0.998)%
  --(2.325,0.999)--(2.335,0.999)--(2.345,0.999)--(2.355,0.999)--(2.365,0.999)--(2.374,0.999)%
  --(2.384,1.000)--(2.394,1.000)--(2.404,1.000)--(2.414,1.000)--(2.424,1.000)--(2.434,1.000)%
  --(2.443,1.000)--(2.453,1.000)--(2.463,1.000)--(2.473,1.000)--(2.483,1.000)--(2.493,1.000)%
  --(2.503,1.000)--(2.512,1.000)--(2.522,1.000)--(2.532,1.000)--(2.542,0.999)--(2.552,0.999)%
  --(2.562,0.999)--(2.572,0.999)--(2.581,0.999)--(2.591,0.998)--(2.601,0.998)--(2.611,0.998)%
  --(2.621,0.998)--(2.631,0.998)--(2.641,0.997)--(2.650,0.997)--(2.660,0.997)--(2.670,0.997)%
  --(2.680,0.997)--(2.690,0.996)--(2.700,0.996)--(2.710,0.996)--(2.719,0.996)--(2.729,0.996)%
  --(2.739,0.996)--(2.749,0.996)--(2.759,0.996)--(2.769,0.996)--(2.779,0.996)--(2.788,0.996)%
  --(2.798,0.996)--(2.808,0.996)--(2.818,0.996)--(2.828,0.996)--(2.838,0.996)--(2.848,0.997)%
  --(2.857,0.997)--(2.867,0.997)--(2.877,0.998)--(2.887,0.998)--(2.897,0.999)--(2.907,0.999)%
  --(2.917,1.000)--(2.926,1.000)--(2.936,1.001)--(2.946,1.002)--(2.956,1.002)--(2.966,1.003)%
  --(2.976,1.004)--(2.986,1.005)--(2.995,1.006)--(3.005,1.007)--(3.015,1.008)--(3.025,1.009)%
  --(3.035,1.010)--(3.045,1.011)--(3.055,1.013)--(3.064,1.014)--(3.074,1.015)--(3.084,1.016)%
  --(3.094,1.018)--(3.104,1.019)--(3.114,1.021)--(3.124,1.022)--(3.134,1.023)--(3.143,1.025)%
  --(3.153,1.026)--(3.163,1.028)--(3.173,1.029)--(3.183,1.031)--(3.193,1.033)--(3.203,1.034)%
  --(3.212,1.036)--(3.222,1.037)--(3.232,1.039)--(3.242,1.041)--(3.252,1.042)--(3.262,1.044)%
  --(3.272,1.046)--(3.281,1.047)--(3.291,1.049)--(3.301,1.050)--(3.311,1.052)--(3.321,1.054)%
  --(3.331,1.055)--(3.341,1.057)--(3.350,1.059)--(3.360,1.060)--(3.370,1.062)--(3.380,1.064)%
  --(3.390,1.065)--(3.400,1.067)--(3.410,1.069)--(3.419,1.070)--(3.429,1.072)--(3.439,1.073)%
  --(3.449,1.075)--(3.459,1.076)--(3.469,1.078)--(3.479,1.079)--(3.488,1.081)--(3.498,1.082)%
  --(3.508,1.084)--(3.518,1.085)--(3.528,1.087)--(3.538,1.088)--(3.548,1.089)--(3.557,1.091)%
  --(3.567,1.092)--(3.577,1.093)--(3.587,1.095)--(3.597,1.096)--(3.607,1.097)--(3.617,1.098)%
  --(3.626,1.100)--(3.636,1.101)--(3.646,1.102)--(3.656,1.103)--(3.666,1.104)--(3.676,1.105)%
  --(3.686,1.106)--(3.695,1.107)--(3.705,1.108)--(3.715,1.109)--(3.725,1.110)--(3.735,1.110)%
  --(3.745,1.111)--(3.755,1.112)--(3.764,1.113)--(3.774,1.113)--(3.784,1.114)--(3.794,1.114)%
  --(3.804,1.115)--(3.814,1.115)--(3.824,1.116)--(3.833,1.116)--(3.843,1.117)--(3.853,1.117)%
  --(3.863,1.117)--(3.873,1.118)--(3.883,1.118)--(3.893,1.118)--(3.902,1.118)--(3.912,1.118)%
  --(3.922,1.118)--(3.932,1.118)--(3.942,1.118)--(3.952,1.118)--(3.962,1.118)--(3.971,1.118)%
  --(3.981,1.118)--(3.991,1.118)--(4.001,1.118)--(4.011,1.118)--(4.021,1.117)--(4.031,1.117)%
  --(4.040,1.117)--(4.050,1.116)--(4.060,1.116)--(4.070,1.115)--(4.080,1.115)--(4.090,1.114)%
  --(4.100,1.114)--(4.109,1.113)--(4.119,1.113)--(4.129,1.112)--(4.139,1.112)--(4.149,1.111)%
  --(4.159,1.110)--(4.169,1.109)--(4.179,1.109)--(4.188,1.108)--(4.198,1.107)--(4.208,1.106)%
  --(4.218,1.105)--(4.228,1.104)--(4.238,1.103)--(4.248,1.103)--(4.257,1.102)--(4.267,1.101)%
  --(4.277,1.100)--(4.287,1.099)--(4.297,1.098)--(4.307,1.097)--(4.317,1.096)--(4.326,1.095)%
  --(4.336,1.094)--(4.346,1.093)--(4.356,1.092)--(4.366,1.091)--(4.376,1.090)--(4.386,1.089)%
  --(4.395,1.088)--(4.405,1.087)--(4.415,1.087)--(4.425,1.086)--(4.435,1.085)--(4.445,1.084)%
  --(4.455,1.083)--(4.464,1.083)--(4.474,1.082)--(4.484,1.081)--(4.494,1.081)--(4.504,1.080)%
  --(4.514,1.080)--(4.524,1.079)--(4.533,1.079)--(4.543,1.078)--(4.553,1.078)--(4.563,1.077)%
  --(4.573,1.077)--(4.583,1.077)--(4.593,1.077)--(4.602,1.077)--(4.612,1.077)--(4.622,1.077)%
  --(4.632,1.077)--(4.642,1.077)--(4.652,1.077)--(4.662,1.077)--(4.671,1.078)--(4.681,1.078)%
  --(4.691,1.079)--(4.701,1.079)--(4.711,1.080)--(4.721,1.081)--(4.731,1.082)--(4.740,1.083)%
  --(4.750,1.084)--(4.760,1.085)--(4.770,1.086)--(4.780,1.088)--(4.790,1.089)--(4.800,1.091)%
  --(4.809,1.093)--(4.819,1.094)--(4.829,1.096)--(4.839,1.099)--(4.849,1.101)--(4.859,1.103)%
  --(4.869,1.106)--(4.878,1.108)--(4.888,1.111)--(4.898,1.114)--(4.908,1.117)--(4.918,1.120)%
  --(4.928,1.124)--(4.938,1.127)--(4.947,1.131)--(4.957,1.135)--(4.967,1.139)--(4.977,1.143)%
  --(4.987,1.147)--(4.997,1.152)--(5.007,1.156)--(5.016,1.161)--(5.026,1.166)--(5.036,1.171)%
  --(5.046,1.176)--(5.056,1.181)--(5.066,1.186)--(5.076,1.191)--(5.085,1.197)--(5.095,1.202)%
  --(5.105,1.207)--(5.115,1.213)--(5.125,1.218)--(5.135,1.224)--(5.145,1.229)--(5.154,1.234)%
  --(5.164,1.240)--(5.174,1.245)--(5.184,1.250)--(5.194,1.255)--(5.204,1.260)--(5.214,1.265)%
  --(5.224,1.270)--(5.233,1.275)--(5.243,1.279)--(5.253,1.284)--(5.263,1.288)--(5.273,1.292)%
  --(5.283,1.296)--(5.293,1.300)--(5.302,1.304)--(5.312,1.307)--(5.322,1.311)--(5.332,1.314)%
  --(5.342,1.317)--(5.352,1.319)--(5.362,1.321)--(5.371,1.323)--(5.381,1.325)--(5.391,1.327)%
  --(5.401,1.328)--(5.411,1.329)--(5.421,1.330)--(5.431,1.330)--(5.440,1.330)--(5.450,1.331)%
  --(5.460,1.331)--(5.470,1.330)--(5.480,1.330)--(5.490,1.329)--(5.500,1.329)--(5.509,1.328)%
  --(5.519,1.327)--(5.529,1.327)--(5.539,1.326)--(5.549,1.325)--(5.559,1.324)--(5.569,1.323)%
  --(5.578,1.322)--(5.588,1.321)--(5.598,1.320)--(5.608,1.319)--(5.618,1.318)--(5.628,1.318)%
  --(5.638,1.317)--(5.647,1.316)--(5.657,1.316)--(5.667,1.316)--(5.677,1.316)--(5.687,1.316)%
  --(5.697,1.316)--(5.707,1.317)--(5.716,1.318)--(5.726,1.319)--(5.736,1.320)--(5.746,1.322)%
  --(5.756,1.323)--(5.766,1.325)--(5.776,1.328)--(5.785,1.331)--(5.795,1.334)--(5.805,1.337)%
  --(5.815,1.341)--(5.825,1.345)--(5.835,1.350)--(5.845,1.354)--(5.854,1.359)--(5.864,1.364)%
  --(5.874,1.370)--(5.884,1.375)--(5.894,1.381)--(5.904,1.387)--(5.914,1.393)--(5.923,1.399)%
  --(5.933,1.405)--(5.943,1.412)--(5.953,1.418)--(5.963,1.424)--(5.973,1.431)--(5.983,1.437)%
  --(5.992,1.443)--(6.002,1.450)--(6.012,1.456)--(6.022,1.462)--(6.032,1.468)--(6.042,1.474)%
  --(6.052,1.480)--(6.061,1.485)--(6.071,1.491)--(6.081,1.496)--(6.091,1.501)--(6.101,1.506)%
  --(6.111,1.510)--(6.121,1.515)--(6.130,1.519)--(6.140,1.522)--(6.150,1.525)--(6.160,1.528)%
  --(6.170,1.531)--(6.180,1.533)--(6.190,1.535)--(6.199,1.536)--(6.209,1.537)--(6.219,1.537)%
  --(6.229,1.537)--(6.239,1.536)--(6.249,1.535)--(6.259,1.534)--(6.269,1.533)--(6.278,1.531)%
  --(6.288,1.528)--(6.298,1.526)--(6.308,1.523)--(6.318,1.520)--(6.328,1.517)--(6.338,1.514)%
  --(6.347,1.511)--(6.357,1.507)--(6.367,1.503)--(6.377,1.500)--(6.387,1.496)--(6.397,1.492)%
  --(6.407,1.488)--(6.416,1.484)--(6.426,1.480)--(6.436,1.477)--(6.446,1.473)--(6.456,1.469)%
  --(6.466,1.466)--(6.476,1.462)--(6.485,1.459)--(6.495,1.456)--(6.505,1.453)--(6.515,1.451)%
  --(6.525,1.449)--(6.535,1.447)--(6.545,1.445)--(6.554,1.444)--(6.564,1.443)--(6.574,1.442)%
  --(6.584,1.442)--(6.594,1.442)--(6.604,1.443)--(6.614,1.444)--(6.623,1.445)--(6.633,1.447)%
  --(6.643,1.450)--(6.653,1.453)--(6.663,1.456)--(6.673,1.460)--(6.683,1.464)--(6.692,1.468)%
  --(6.702,1.473)--(6.712,1.479)--(6.722,1.484)--(6.732,1.490)--(6.742,1.496)--(6.752,1.502)%
  --(6.761,1.509)--(6.771,1.516)--(6.781,1.523)--(6.791,1.530)--(6.801,1.537)--(6.811,1.545)%
  --(6.821,1.553)--(6.830,1.561)--(6.840,1.569)--(6.850,1.577)--(6.860,1.585)--(6.870,1.593)%
  --(6.880,1.601)--(6.890,1.610)--(6.899,1.618)--(6.909,1.626)--(6.919,1.635)--(6.929,1.643)%
  --(6.939,1.651)--(6.949,1.660)--(6.959,1.668)--(6.968,1.676)--(6.978,1.684)--(6.988,1.691)%
  --(6.998,1.699)--(7.008,1.706)--(7.018,1.714)--(7.028,1.721)--(7.037,1.728)--(7.047,1.734)%
  --(7.057,1.741)--(7.067,1.747)--(7.077,1.753)--(7.087,1.759)--(7.097,1.765)--(7.106,1.770)%
  --(7.116,1.776)--(7.126,1.781)--(7.136,1.786)--(7.146,1.791)--(7.156,1.796)--(7.166,1.801)%
  --(7.175,1.806)--(7.185,1.810)--(7.195,1.815)--(7.205,1.819)--(7.215,1.823)--(7.225,1.828)%
  --(7.235,1.832)--(7.244,1.836)--(7.254,1.840)--(7.264,1.844)--(7.274,1.848)--(7.284,1.852)%
  --(7.294,1.855)--(7.304,1.859)--(7.314,1.863)--(7.323,1.867)--(7.333,1.870)--(7.343,1.874)%
  --(7.353,1.878)--(7.363,1.882)--(7.373,1.886)--(7.383,1.889)--(7.392,1.893)--(7.402,1.897)%
  --(7.412,1.901)--(7.422,1.905)--(7.432,1.909)--(7.442,1.913)--(7.452,1.917)--(7.461,1.922)%
  --(7.471,1.926)--(7.481,1.930)--(7.491,1.935)--(7.501,1.940)--(7.511,1.945)--(7.521,1.950)%
  --(7.530,1.956)--(7.540,1.961)--(7.550,1.967)--(7.560,1.974)--(7.570,1.980)--(7.580,1.987)%
  --(7.590,1.994)--(7.599,2.002)--(7.609,2.010)--(7.619,2.018)--(7.629,2.027)--(7.639,2.036)%
  --(7.649,2.045)--(7.659,2.056)--(7.668,2.066)--(7.678,2.077)--(7.688,2.089)--(7.698,2.101)%
  --(7.708,2.113)--(7.718,2.127)--(7.728,2.140)--(7.737,2.155)--(7.747,2.170)--(7.757,2.186)%
  --(7.767,2.202)--(7.777,2.219)--(7.787,2.237)--(7.797,2.255)--(7.806,2.274)--(7.816,2.294)%
  --(7.826,2.315)--(7.836,2.337)--(7.846,2.359)--(7.856,2.382)--(7.866,2.406)--(7.875,2.430)%
  --(7.885,2.456)--(7.895,2.481)--(7.905,2.508)--(7.915,2.535)--(7.925,2.562)--(7.935,2.589)%
  --(7.944,2.617)--(7.954,2.645)--(7.964,2.673)--(7.974,2.701)--(7.984,2.730)--(7.994,2.758)%
  --(8.004,2.786)--(8.013,2.814)--(8.023,2.842)--(8.033,2.870)--(8.043,2.898)--(8.053,2.925)%
  --(8.063,2.951)--(8.073,2.977)--(8.082,3.003)--(8.092,3.028)--(8.102,3.053)--(8.112,3.076)%
  --(8.122,3.099)--(8.132,3.121)--(8.142,3.143)--(8.151,3.163)--(8.161,3.182)--(8.171,3.200)%
  --(8.181,3.217)--(8.191,3.233)--(8.201,3.248)--(8.211,3.262)--(8.220,3.274)--(8.230,3.284)%
  --(8.240,3.293)--(8.250,3.301)--(8.260,3.307)--(8.270,3.312)--(8.280,3.315)--(8.289,3.317)%
  --(8.299,3.317)--(8.309,3.316)--(8.319,3.314)--(8.329,3.311)--(8.339,3.307)--(8.349,3.302)%
  --(8.359,3.297)--(8.368,3.290)--(8.378,3.283)--(8.388,3.276)--(8.398,3.267)--(8.408,3.259)%
  --(8.418,3.250)--(8.428,3.241)--(8.437,3.232)--(8.447,3.222)--(8.457,3.213)--(8.467,3.204)%
  --(8.477,3.195)--(8.487,3.186)--(8.497,3.178)--(8.506,3.170)--(8.516,3.162)--(8.526,3.155)%
  --(8.536,3.149)--(8.546,3.144)--(8.556,3.139)--(8.566,3.135)--(8.575,3.133)--(8.585,3.131)%
  --(8.595,3.131)--(8.605,3.131)--(8.615,3.133)--(8.625,3.137)--(8.635,3.142)--(8.644,3.149)%
  --(8.654,3.157)--(8.664,3.167)--(8.674,3.179)--(8.684,3.192)--(8.694,3.207)--(8.704,3.224)%
  --(8.713,3.242)--(8.723,3.262)--(8.733,3.283)--(8.743,3.305)--(8.753,3.329)--(8.763,3.354)%
  --(8.773,3.379)--(8.782,3.406)--(8.792,3.433)--(8.802,3.462)--(8.812,3.491)--(8.822,3.520)%
  --(8.832,3.550)--(8.842,3.581)--(8.851,3.612)--(8.861,3.643)--(8.871,3.675)--(8.881,3.707)%
  --(8.891,3.739)--(8.901,3.770)--(8.911,3.802)--(8.920,3.834)--(8.930,3.865)--(8.940,3.896)%
  --(8.950,3.927)--(8.960,3.957)--(8.970,3.987)--(8.980,4.016)--(8.989,4.044)--(8.999,4.071)%
  --(9.009,4.098)--(9.019,4.124)--(9.029,4.148)--(9.039,4.172)--(9.049,4.194)--(9.058,4.215)%
  --(9.068,4.235)--(9.078,4.253)--(9.088,4.270)--(9.098,4.286)--(9.108,4.300)--(9.118,4.313)%
  --(9.127,4.325)--(9.137,4.335)--(9.147,4.345)--(9.157,4.353)--(9.167,4.361)--(9.177,4.368)%
  --(9.187,4.374)--(9.196,4.379)--(9.206,4.384)--(9.216,4.388)--(9.226,4.392)--(9.236,4.395)%
  --(9.246,4.398)--(9.256,4.400)--(9.265,4.402)--(9.275,4.404)--(9.285,4.406)--(9.295,4.408)%
  --(9.305,4.410)--(9.315,4.412)--(9.325,4.414)--(9.334,4.417)--(9.344,4.419)--(9.354,4.422)%
  --(9.364,4.426)--(9.374,4.430)--(9.384,4.434)--(9.394,4.439)--(9.404,4.445)--(9.413,4.451)%
  --(9.423,4.459)--(9.433,4.467)--(9.443,4.476)--(9.453,4.487)--(9.463,4.498)--(9.473,4.510)%
  --(9.482,4.524)--(9.492,4.539)--(9.502,4.555)--(9.512,4.573)--(9.522,4.591)--(9.532,4.611)%
  --(9.542,4.631)--(9.551,4.653)--(9.561,4.675)--(9.571,4.698)--(9.581,4.722)--(9.591,4.746)%
  --(9.601,4.771)--(9.611,4.796)--(9.620,4.821)--(9.630,4.847)--(9.640,4.874)--(9.650,4.900)%
  --(9.660,4.926)--(9.670,4.953)--(9.680,4.979)--(9.689,5.005)--(9.699,5.031)--(9.709,5.057)%
  --(9.719,5.082)--(9.729,5.107)--(9.739,5.132)--(9.749,5.156)--(9.758,5.179)--(9.768,5.202)%
  --(9.778,5.224)--(9.788,5.244)--(9.798,5.264)--(9.808,5.283)--(9.818,5.301)--(9.827,5.318)%
  --(9.837,5.334)--(9.847,5.348)--(9.857,5.361)--(9.867,5.372)--(9.877,5.382)--(9.887,5.391)%
  --(9.896,5.397)--(9.906,5.402)--(9.916,5.406)--(9.926,5.408)--(9.936,5.408)--(9.946,5.407)%
  --(9.956,5.404)--(9.965,5.400)--(9.975,5.395)--(9.985,5.389)--(9.995,5.381)--(10.005,5.373)%
  --(10.015,5.363)--(10.025,5.353)--(10.034,5.341)--(10.044,5.329)--(10.054,5.316)--(10.064,5.303)%
  --(10.074,5.288)--(10.084,5.274)--(10.094,5.258)--(10.103,5.243)--(10.113,5.227)--(10.123,5.210)%
  --(10.133,5.194)--(10.143,5.177)--(10.153,5.160)--(10.163,5.143)--(10.172,5.126)--(10.182,5.110)%
  --(10.192,5.093)--(10.202,5.077)--(10.212,5.061)--(10.222,5.045)--(10.232,5.030)--(10.241,5.015)%
  --(10.251,5.000)--(10.261,4.987)--(10.271,4.974)--(10.281,4.961)--(10.291,4.950)--(10.301,4.939)%
  --(10.310,4.930)--(10.320,4.921)--(10.330,4.913)--(10.340,4.906)--(10.350,4.900)--(10.360,4.895)%
  --(10.370,4.890)--(10.379,4.887)--(10.389,4.884)--(10.399,4.881)--(10.409,4.879)--(10.419,4.878)%
  --(10.429,4.878)--(10.439,4.878)--(10.449,4.878)--(10.458,4.879)--(10.468,4.881)--(10.478,4.883)%
  --(10.488,4.885)--(10.498,4.887)--(10.508,4.890)--(10.518,4.893)--(10.527,4.897)--(10.537,4.900)%
  --(10.547,4.904)--(10.557,4.908)--(10.567,4.912)--(10.577,4.916)--(10.587,4.920)--(10.596,4.924)%
  --(10.606,4.928)--(10.616,4.932)--(10.626,4.936)--(10.636,4.939)--(10.646,4.943)--(10.656,4.946)%
  --(10.665,4.949)--(10.675,4.952)--(10.685,4.955)--(10.695,4.957)--(10.705,4.959)--(10.715,4.960)%
  --(10.725,4.961)--(10.734,4.962)--(10.744,4.962)--(10.754,4.962)--(10.764,4.961)--(10.774,4.960)%
  --(10.784,4.958)--(10.794,4.956)--(10.803,4.954)--(10.813,4.952)--(10.823,4.949)--(10.833,4.945)%
  --(10.843,4.942)--(10.853,4.938)--(10.863,4.933)--(10.872,4.929)--(10.882,4.924)--(10.892,4.918)%
  --(10.902,4.913)--(10.912,4.907)--(10.922,4.901)--(10.932,4.894)--(10.941,4.887)--(10.951,4.881)%
  --(10.961,4.873)--(10.971,4.866)--(10.981,4.858)--(10.991,4.850)--(11.001,4.842)--(11.010,4.834)%
  --(11.020,4.825)--(11.030,4.817)--(11.040,4.808)--(11.050,4.799)--(11.060,4.790)--(11.070,4.780)%
  --(11.079,4.771)--(11.089,4.761)--(11.099,4.751)--(11.109,4.741)--(11.119,4.731)--(11.129,4.721)%
  --(11.139,4.711)--(11.148,4.700)--(11.158,4.690)--(11.168,4.679)--(11.178,4.668)--(11.188,4.657)%
  --(11.198,4.646)--(11.208,4.635)--(11.217,4.623)--(11.227,4.611)--(11.237,4.598)--(11.247,4.586)%
  --(11.257,4.573)--(11.267,4.560)--(11.277,4.546)--(11.286,4.532)--(11.296,4.518)--(11.306,4.503)%
  --(11.316,4.487)--(11.326,4.472)--(11.336,4.456)--(11.346,4.439)--(11.355,4.422)--(11.365,4.404)%
  --(11.375,4.386)--(11.385,4.367)--(11.395,4.348)--(11.405,4.328)--(11.415,4.307)--(11.424,4.286)%
  --(11.434,4.264)--(11.444,4.242)--(11.454,4.219)--(11.464,4.195)--(11.474,4.170)--(11.484,4.145)%
  --(11.494,4.119)--(11.503,4.092)--(11.513,4.064)--(11.523,4.036)--(11.533,4.006)--(11.543,3.976)%
  --(11.553,3.945)--(11.563,3.913)--(11.572,3.881)--(11.582,3.848)--(11.592,3.813)--(11.602,3.779)%
  --(11.612,3.743)--(11.622,3.707)--(11.632,3.670)--(11.641,3.632)--(11.651,3.594)--(11.661,3.555)%
  --(11.671,3.516)--(11.681,3.476)--(11.691,3.435)--(11.701,3.394)--(11.710,3.352)--(11.720,3.310)%
  --(11.730,3.268)--(11.740,3.225)--(11.750,3.181)--(11.760,3.137)--(11.770,3.093)--(11.779,3.048)%
  --(11.789,3.003)--(11.799,2.958)--(11.809,2.913)--(11.819,2.867)--(11.829,2.821)--(11.839,2.774)%
  --(11.848,2.728)--(11.858,2.681)--(11.868,2.634)--(11.878,2.587)--(11.888,2.539)--(11.898,2.492)%
  --(11.908,2.444)--(11.917,2.397)--(11.927,2.349)--(11.937,2.301)--(11.947,2.253);
\gpcolor{color=gp lt color border}
\node[gp node right] at (4.448,7.123) {w/o types};
\gpfill{color=gp lt color 1,opacity=0.10} (4.632,7.046)--(5.548,7.046)--(5.548,7.200)--(4.632,7.200)--cycle;
\gpcolor{color=gp lt color 1}
\draw[gp path] (4.632,7.046)--(5.548,7.046)--(5.548,7.200)--(4.632,7.200)--cycle;
\gpfill{color=gp lt color 1,opacity=0.10} (1.688,1.011)--(1.688,1.011)--(1.698,1.011)--(1.709,1.010)%
    --(1.719,1.010)--(1.729,1.010)--(1.739,1.009)--(1.750,1.009)--(1.760,1.009)%
    --(1.770,1.008)--(1.780,1.008)--(1.791,1.008)--(1.801,1.007)--(1.811,1.007)%
    --(1.822,1.007)--(1.832,1.007)--(1.842,1.006)--(1.852,1.006)--(1.863,1.006)%
    --(1.873,1.006)--(1.883,1.005)--(1.893,1.005)--(1.904,1.005)--(1.914,1.005)%
    --(1.924,1.005)--(1.934,1.004)--(1.945,1.004)--(1.955,1.004)--(1.965,1.004)%
    --(1.976,1.004)--(1.986,1.004)--(1.996,1.004)--(2.006,1.004)--(2.017,1.003)%
    --(2.027,1.003)--(2.037,1.003)--(2.047,1.003)--(2.058,1.003)--(2.068,1.003)%
    --(2.078,1.003)--(2.089,1.003)--(2.099,1.003)--(2.109,1.004)--(2.119,1.004)%
    --(2.130,1.004)--(2.140,1.004)--(2.150,1.004)--(2.160,1.004)--(2.171,1.004)%
    --(2.181,1.005)--(2.191,1.005)--(2.201,1.005)--(2.212,1.005)--(2.222,1.006)%
    --(2.232,1.006)--(2.243,1.006)--(2.253,1.006)--(2.263,1.007)--(2.273,1.007)%
    --(2.284,1.007)--(2.294,1.007)--(2.304,1.008)--(2.314,1.008)--(2.325,1.008)%
    --(2.335,1.008)--(2.345,1.009)--(2.356,1.009)--(2.366,1.009)--(2.376,1.009)%
    --(2.386,1.010)--(2.397,1.010)--(2.407,1.010)--(2.417,1.010)--(2.427,1.010)%
    --(2.438,1.010)--(2.448,1.011)--(2.458,1.011)--(2.468,1.011)--(2.479,1.011)%
    --(2.489,1.011)--(2.499,1.011)--(2.510,1.011)--(2.520,1.011)--(2.530,1.011)%
    --(2.540,1.011)--(2.551,1.011)--(2.561,1.011)--(2.571,1.010)--(2.581,1.010)%
    --(2.592,1.010)--(2.602,1.010)--(2.612,1.010)--(2.623,1.010)--(2.633,1.010)%
    --(2.643,1.009)--(2.653,1.009)--(2.664,1.009)--(2.674,1.009)--(2.684,1.009)%
    --(2.694,1.009)--(2.705,1.008)--(2.715,1.008)--(2.725,1.008)--(2.735,1.008)%
    --(2.746,1.008)--(2.756,1.008)--(2.766,1.008)--(2.777,1.008)--(2.787,1.008)%
    --(2.797,1.008)--(2.807,1.008)--(2.818,1.008)--(2.828,1.008)--(2.838,1.008)%
    --(2.848,1.008)--(2.859,1.009)--(2.869,1.009)--(2.879,1.009)--(2.890,1.010)%
    --(2.900,1.010)--(2.910,1.010)--(2.920,1.011)--(2.931,1.011)--(2.941,1.012)%
    --(2.951,1.013)--(2.961,1.013)--(2.972,1.014)--(2.982,1.015)--(2.992,1.016)%
    --(3.002,1.016)--(3.013,1.017)--(3.023,1.018)--(3.033,1.019)--(3.044,1.020)%
    --(3.054,1.021)--(3.064,1.021)--(3.074,1.022)--(3.085,1.023)--(3.095,1.024)%
    --(3.105,1.025)--(3.115,1.026)--(3.126,1.027)--(3.136,1.028)--(3.146,1.028)%
    --(3.157,1.029)--(3.167,1.030)--(3.177,1.031)--(3.187,1.032)--(3.198,1.032)%
    --(3.208,1.033)--(3.218,1.034)--(3.228,1.034)--(3.239,1.035)--(3.249,1.035)%
    --(3.259,1.036)--(3.269,1.036)--(3.280,1.036)--(3.290,1.036)--(3.300,1.037)%
    --(3.311,1.037)--(3.321,1.037)--(3.331,1.037)--(3.341,1.037)--(3.352,1.036)%
    --(3.362,1.036)--(3.372,1.036)--(3.382,1.036)--(3.393,1.035)--(3.403,1.035)%
    --(3.413,1.034)--(3.424,1.034)--(3.434,1.033)--(3.444,1.033)--(3.454,1.032)%
    --(3.465,1.032)--(3.475,1.032)--(3.485,1.031)--(3.495,1.031)--(3.506,1.031)%
    --(3.516,1.030)--(3.526,1.030)--(3.536,1.030)--(3.547,1.030)--(3.557,1.030)%
    --(3.567,1.030)--(3.578,1.031)--(3.588,1.031)--(3.598,1.032)--(3.608,1.032)%
    --(3.619,1.033)--(3.629,1.034)--(3.639,1.035)--(3.649,1.036)--(3.660,1.038)%
    --(3.670,1.039)--(3.680,1.041)--(3.691,1.043)--(3.701,1.045)--(3.711,1.047)%
    --(3.721,1.050)--(3.732,1.053)--(3.742,1.056)--(3.752,1.059)--(3.762,1.063)%
    --(3.773,1.066)--(3.783,1.070)--(3.793,1.074)--(3.803,1.078)--(3.814,1.083)%
    --(3.824,1.087)--(3.834,1.092)--(3.845,1.097)--(3.855,1.102)--(3.865,1.107)%
    --(3.875,1.112)--(3.886,1.117)--(3.896,1.122)--(3.906,1.127)--(3.916,1.132)%
    --(3.927,1.137)--(3.937,1.143)--(3.947,1.148)--(3.958,1.153)--(3.968,1.158)%
    --(3.978,1.163)--(3.988,1.168)--(3.999,1.173)--(4.009,1.177)--(4.019,1.182)%
    --(4.029,1.187)--(4.040,1.191)--(4.050,1.195)--(4.060,1.199)--(4.070,1.203)%
    --(4.081,1.207)--(4.091,1.210)--(4.101,1.213)--(4.112,1.216)--(4.122,1.219)%
    --(4.132,1.222)--(4.142,1.224)--(4.153,1.226)--(4.163,1.227)--(4.173,1.229)%
    --(4.183,1.230)--(4.194,1.231)--(4.204,1.231)--(4.214,1.232)--(4.225,1.232)%
    --(4.235,1.233)--(4.245,1.233)--(4.255,1.233)--(4.266,1.233)--(4.276,1.233)%
    --(4.286,1.233)--(4.296,1.233)--(4.307,1.233)--(4.317,1.233)--(4.327,1.234)%
    --(4.337,1.234)--(4.348,1.235)--(4.358,1.235)--(4.368,1.236)--(4.379,1.237)%
    --(4.389,1.239)--(4.399,1.240)--(4.409,1.242)--(4.420,1.245)--(4.430,1.247)%
    --(4.440,1.250)--(4.450,1.254)--(4.461,1.257)--(4.471,1.262)--(4.481,1.266)%
    --(4.492,1.272)--(4.502,1.277)--(4.512,1.284)--(4.522,1.291)--(4.533,1.298)%
    --(4.543,1.306)--(4.553,1.315)--(4.563,1.324)--(4.574,1.334)--(4.584,1.345)%
    --(4.594,1.356)--(4.604,1.368)--(4.615,1.380)--(4.625,1.393)--(4.635,1.406)%
    --(4.646,1.420)--(4.656,1.433)--(4.666,1.447)--(4.676,1.462)--(4.687,1.476)%
    --(4.697,1.491)--(4.707,1.505)--(4.717,1.520)--(4.728,1.535)--(4.738,1.549)%
    --(4.748,1.564)--(4.759,1.578)--(4.769,1.592)--(4.779,1.606)--(4.789,1.620)%
    --(4.800,1.633)--(4.810,1.646)--(4.820,1.659)--(4.830,1.671)--(4.841,1.682)%
    --(4.851,1.693)--(4.861,1.704)--(4.871,1.713)--(4.882,1.723)--(4.892,1.731)%
    --(4.902,1.738)--(4.913,1.745)--(4.923,1.751)--(4.933,1.756)--(4.943,1.760)%
    --(4.954,1.763)--(4.964,1.765)--(4.974,1.765)--(4.984,1.765)--(4.995,1.764)%
    --(5.005,1.761)--(5.015,1.758)--(5.026,1.754)--(5.036,1.748)--(5.046,1.743)%
    --(5.056,1.736)--(5.067,1.728)--(5.077,1.720)--(5.087,1.712)--(5.097,1.702)%
    --(5.108,1.693)--(5.118,1.683)--(5.128,1.672)--(5.138,1.661)--(5.149,1.650)%
    --(5.159,1.638)--(5.169,1.627)--(5.180,1.615)--(5.190,1.603)--(5.200,1.591)%
    --(5.210,1.579)--(5.221,1.567)--(5.231,1.555)--(5.241,1.544)--(5.251,1.532)%
    --(5.262,1.521)--(5.272,1.511)--(5.282,1.500)--(5.293,1.490)--(5.303,1.481)%
    --(5.313,1.472)--(5.323,1.463)--(5.334,1.455)--(5.344,1.448)--(5.354,1.442)%
    --(5.364,1.436)--(5.375,1.431)--(5.385,1.427)--(5.395,1.424)--(5.405,1.422)%
    --(5.416,1.421)--(5.426,1.420)--(5.436,1.421)--(5.447,1.422)--(5.457,1.424)%
    --(5.467,1.426)--(5.477,1.429)--(5.488,1.433)--(5.498,1.437)--(5.508,1.442)%
    --(5.518,1.448)--(5.529,1.454)--(5.539,1.461)--(5.549,1.468)--(5.560,1.475)%
    --(5.570,1.483)--(5.580,1.491)--(5.590,1.500)--(5.601,1.509)--(5.611,1.518)%
    --(5.621,1.527)--(5.631,1.537)--(5.642,1.547)--(5.652,1.557)--(5.662,1.567)%
    --(5.672,1.577)--(5.683,1.588)--(5.693,1.598)--(5.703,1.608)--(5.714,1.619)%
    --(5.724,1.629)--(5.734,1.640)--(5.744,1.650)--(5.755,1.660)--(5.765,1.670)%
    --(5.775,1.680)--(5.785,1.689)--(5.796,1.699)--(5.806,1.708)--(5.816,1.717)%
    --(5.827,1.725)--(5.837,1.733)--(5.847,1.741)--(5.857,1.749)--(5.868,1.756)%
    --(5.878,1.763)--(5.888,1.770)--(5.898,1.777)--(5.909,1.783)--(5.919,1.788)%
    --(5.929,1.794)--(5.939,1.799)--(5.950,1.803)--(5.960,1.807)--(5.970,1.811)%
    --(5.981,1.815)--(5.991,1.818)--(6.001,1.820)--(6.011,1.822)--(6.022,1.824)%
    --(6.032,1.826)--(6.042,1.826)--(6.052,1.827)--(6.063,1.827)--(6.073,1.826)%
    --(6.083,1.825)--(6.094,1.824)--(6.104,1.822)--(6.114,1.819)--(6.124,1.816)%
    --(6.135,1.813)--(6.145,1.809)--(6.155,1.804)--(6.165,1.799)--(6.176,1.793)%
    --(6.186,1.787)--(6.196,1.780)--(6.206,1.773)--(6.217,1.765)--(6.227,1.757)%
    --(6.237,1.748)--(6.248,1.739)--(6.258,1.730)--(6.268,1.721)--(6.278,1.711)%
    --(6.289,1.702)--(6.299,1.693)--(6.309,1.684)--(6.319,1.676)--(6.330,1.668)%
    --(6.340,1.661)--(6.350,1.654)--(6.361,1.648)--(6.371,1.643)--(6.381,1.638)%
    --(6.391,1.635)--(6.402,1.633)--(6.412,1.632)--(6.422,1.632)--(6.432,1.633)%
    --(6.443,1.636)--(6.453,1.641)--(6.463,1.647)--(6.473,1.654)--(6.484,1.664)%
    --(6.494,1.675)--(6.504,1.689)--(6.515,1.704)--(6.525,1.722)--(6.535,1.742)%
    --(6.545,1.764)--(6.556,1.789)--(6.566,1.816)--(6.576,1.846)--(6.586,1.878)%
    --(6.597,1.913)--(6.607,1.951)--(6.617,1.992)--(6.628,2.036)--(6.638,2.083)%
    --(6.648,2.132)--(6.658,2.184)--(6.669,2.237)--(6.679,2.293)--(6.689,2.350)%
    --(6.699,2.408)--(6.710,2.468)--(6.720,2.528)--(6.730,2.590)--(6.740,2.652)%
    --(6.751,2.714)--(6.761,2.777)--(6.771,2.840)--(6.782,2.902)--(6.792,2.964)%
    --(6.802,3.025)--(6.812,3.086)--(6.823,3.145)--(6.833,3.203)--(6.843,3.260)%
    --(6.853,3.314)--(6.864,3.367)--(6.874,3.418)--(6.884,3.466)--(6.895,3.512)%
    --(6.905,3.555)--(6.915,3.595)--(6.925,3.632)--(6.936,3.666)--(6.946,3.696)%
    --(6.956,3.722)--(6.966,3.744)--(6.977,3.762)--(6.987,3.775)--(6.997,3.784)%
    --(7.007,3.788)--(7.018,3.787)--(7.028,3.780)--(7.038,3.769)--(7.049,3.752)%
    --(7.059,3.730)--(7.069,3.704)--(7.079,3.674)--(7.090,3.640)--(7.100,3.602)%
    --(7.110,3.560)--(7.120,3.515)--(7.131,3.466)--(7.141,3.415)--(7.151,3.362)%
    --(7.162,3.306)--(7.172,3.248)--(7.182,3.188)--(7.192,3.126)--(7.203,3.063)%
    --(7.213,2.999)--(7.223,2.933)--(7.233,2.868)--(7.244,2.801)--(7.254,2.735)%
    --(7.264,2.668)--(7.274,2.602)--(7.285,2.537)--(7.295,2.472)--(7.305,2.408)%
    --(7.316,2.345)--(7.326,2.284)--(7.336,2.225)--(7.346,2.167)--(7.357,2.112)%
    --(7.367,2.060)--(7.377,2.009)--(7.387,1.962)--(7.398,1.918)--(7.408,1.878)%
    --(7.418,1.841)--(7.429,1.808)--(7.439,1.779)--(7.449,1.755)--(7.459,1.734)%
    --(7.470,1.717)--(7.480,1.704)--(7.490,1.694)--(7.500,1.688)--(7.511,1.685)%
    --(7.521,1.685)--(7.531,1.689)--(7.541,1.694)--(7.552,1.703)--(7.562,1.714)%
    --(7.572,1.727)--(7.583,1.743)--(7.593,1.760)--(7.603,1.780)--(7.613,1.801)%
    --(7.624,1.824)--(7.634,1.848)--(7.644,1.873)--(7.654,1.899)--(7.665,1.927)%
    --(7.675,1.955)--(7.685,1.984)--(7.696,2.013)--(7.706,2.043)--(7.716,2.073)%
    --(7.726,2.103)--(7.737,2.132)--(7.747,2.162)--(7.757,2.191)--(7.767,2.220)%
    --(7.778,2.247)--(7.788,2.274)--(7.798,2.300)--(7.808,2.325)--(7.819,2.348)%
    --(7.829,2.370)--(7.839,2.390)--(7.850,2.408)--(7.860,2.425)--(7.870,2.440)%
    --(7.880,2.453)--(7.891,2.464)--(7.901,2.474)--(7.911,2.483)--(7.921,2.490)%
    --(7.932,2.495)--(7.942,2.499)--(7.952,2.502)--(7.963,2.504)--(7.973,2.505)%
    --(7.983,2.504)--(7.993,2.502)--(8.004,2.500)--(8.014,2.497)--(8.024,2.492)%
    --(8.034,2.487)--(8.045,2.482)--(8.055,2.475)--(8.065,2.468)--(8.075,2.461)%
    --(8.086,2.453)--(8.096,2.444)--(8.106,2.436)--(8.117,2.427)--(8.127,2.417)%
    --(8.137,2.408)--(8.147,2.398)--(8.158,2.389)--(8.168,2.379)--(8.178,2.370)%
    --(8.188,2.361)--(8.199,2.351)--(8.209,2.343)--(8.219,2.334)--(8.230,2.326)%
    --(8.240,2.318)--(8.250,2.311)--(8.260,2.305)--(8.271,2.299)--(8.281,2.293)%
    --(8.291,2.289)--(8.301,2.285)--(8.312,2.281)--(8.322,2.278)--(8.332,2.275)%
    --(8.342,2.274)--(8.353,2.272)--(8.363,2.272)--(8.373,2.271)--(8.384,2.272)%
    --(8.394,2.273)--(8.404,2.274)--(8.414,2.276)--(8.425,2.279)--(8.435,2.282)%
    --(8.445,2.286)--(8.455,2.290)--(8.466,2.295)--(8.476,2.300)--(8.486,2.306)%
    --(8.497,2.312)--(8.507,2.319)--(8.517,2.326)--(8.527,2.334)--(8.538,2.343)%
    --(8.548,2.351)--(8.558,2.361)--(8.568,2.371)--(8.579,2.381)--(8.589,2.392)%
    --(8.599,2.403)--(8.609,2.415)--(8.620,2.428)--(8.630,2.440)--(8.640,2.454)%
    --(8.651,2.468)--(8.661,2.482)--(8.671,2.496)--(8.681,2.512)--(8.692,2.527)%
    --(8.702,2.543)--(8.712,2.559)--(8.722,2.576)--(8.733,2.593)--(8.743,2.610)%
    --(8.753,2.627)--(8.764,2.645)--(8.774,2.663)--(8.784,2.681)--(8.794,2.699)%
    --(8.805,2.717)--(8.815,2.735)--(8.825,2.753)--(8.835,2.771)--(8.846,2.789)%
    --(8.856,2.807)--(8.866,2.825)--(8.876,2.843)--(8.887,2.861)--(8.897,2.878)%
    --(8.907,2.896)--(8.918,2.913)--(8.928,2.929)--(8.938,2.946)--(8.948,2.962)%
    --(8.959,2.977)--(8.969,2.993)--(8.979,3.007)--(8.989,3.022)--(9.000,3.036)%
    --(9.010,3.049)--(9.020,3.062)--(9.031,3.074)--(9.041,3.085)--(9.051,3.096)%
    --(9.061,3.107)--(9.072,3.116)--(9.082,3.125)--(9.092,3.133)--(9.102,3.141)%
    --(9.113,3.147)--(9.123,3.154)--(9.133,3.159)--(9.143,3.164)--(9.154,3.169)%
    --(9.164,3.173)--(9.174,3.177)--(9.185,3.181)--(9.195,3.184)--(9.205,3.187)%
    --(9.215,3.190)--(9.226,3.192)--(9.236,3.195)--(9.246,3.197)--(9.256,3.200)%
    --(9.267,3.202)--(9.277,3.204)--(9.287,3.207)--(9.298,3.209)--(9.308,3.212)%
    --(9.318,3.215)--(9.328,3.219)--(9.339,3.222)--(9.349,3.226)--(9.359,3.230)%
    --(9.369,3.235)--(9.380,3.240)--(9.390,3.246)--(9.400,3.252)--(9.410,3.259)%
    --(9.421,3.266)--(9.431,3.274)--(9.441,3.283)--(9.452,3.293)--(9.462,3.303)%
    --(9.472,3.314)--(9.482,3.326)--(9.493,3.340)--(9.503,3.353)--(9.513,3.368)%
    --(9.523,3.384)--(9.534,3.400)--(9.544,3.417)--(9.554,3.435)--(9.565,3.453)%
    --(9.575,3.472)--(9.585,3.491)--(9.595,3.511)--(9.606,3.531)--(9.616,3.552)%
    --(9.626,3.572)--(9.636,3.593)--(9.647,3.615)--(9.657,3.636)--(9.667,3.657)%
    --(9.677,3.679)--(9.688,3.700)--(9.698,3.722)--(9.708,3.743)--(9.719,3.764)%
    --(9.729,3.785)--(9.739,3.806)--(9.749,3.826)--(9.760,3.846)--(9.770,3.865)%
    --(9.780,3.885)--(9.790,3.903)--(9.801,3.921)--(9.811,3.938)--(9.821,3.955)%
    --(9.832,3.971)--(9.842,3.986)--(9.852,4.001)--(9.862,4.014)--(9.873,4.027)%
    --(9.883,4.038)--(9.893,4.049)--(9.903,4.058)--(9.914,4.067)--(9.924,4.074)%
    --(9.934,4.080)--(9.944,4.086)--(9.955,4.091)--(9.965,4.095)--(9.975,4.098)%
    --(9.986,4.100)--(9.996,4.102)--(10.006,4.103)--(10.016,4.104)--(10.027,4.104)%
    --(10.037,4.104)--(10.047,4.103)--(10.057,4.102)--(10.068,4.101)--(10.078,4.099)%
    --(10.088,4.098)--(10.099,4.096)--(10.109,4.094)--(10.119,4.091)--(10.129,4.089)%
    --(10.140,4.087)--(10.150,4.085)--(10.160,4.084)--(10.170,4.082)--(10.181,4.081)%
    --(10.191,4.079)--(10.201,4.079)--(10.211,4.078)--(10.222,4.079)--(10.232,4.079)%
    --(10.242,4.080)--(10.253,4.082)--(10.263,4.084)--(10.273,4.088)--(10.283,4.091)%
    --(10.294,4.096)--(10.304,4.101)--(10.314,4.108)--(10.324,4.115)--(10.335,4.123)%
    --(10.345,4.132)--(10.355,4.141)--(10.366,4.151)--(10.376,4.162)--(10.386,4.174)%
    --(10.396,4.185)--(10.407,4.198)--(10.417,4.210)--(10.427,4.224)--(10.437,4.237)%
    --(10.448,4.251)--(10.458,4.265)--(10.468,4.279)--(10.478,4.293)--(10.489,4.307)%
    --(10.499,4.321)--(10.509,4.336)--(10.520,4.350)--(10.530,4.364)--(10.540,4.378)%
    --(10.550,4.391)--(10.561,4.405)--(10.571,4.418)--(10.581,4.430)--(10.591,4.443)%
    --(10.602,4.454)--(10.612,4.466)--(10.622,4.476)--(10.633,4.486)--(10.643,4.496)%
    --(10.653,4.504)--(10.663,4.512)--(10.674,4.519)--(10.684,4.526)--(10.694,4.531)%
    --(10.704,4.535)--(10.715,4.538)--(10.725,4.541)--(10.735,4.542)--(10.745,4.542)%
    --(10.756,4.541)--(10.766,4.540)--(10.776,4.537)--(10.787,4.534)--(10.797,4.530)%
    --(10.807,4.526)--(10.817,4.520)--(10.828,4.514)--(10.838,4.508)--(10.848,4.501)%
    --(10.858,4.493)--(10.869,4.485)--(10.879,4.477)--(10.889,4.468)--(10.900,4.459)%
    --(10.910,4.450)--(10.920,4.440)--(10.930,4.431)--(10.941,4.421)--(10.951,4.411)%
    --(10.961,4.401)--(10.971,4.392)--(10.982,4.382)--(10.992,4.372)--(11.002,4.363)%
    --(11.012,4.354)--(11.023,4.345)--(11.033,4.336)--(11.043,4.328)--(11.054,4.320)%
    --(11.064,4.313)--(11.074,4.306)--(11.084,4.299)--(11.095,4.294)--(11.105,4.288)%
    --(11.115,4.284)--(11.125,4.280)--(11.136,4.277)--(11.146,4.275)--(11.156,4.273)%
    --(11.167,4.272)--(11.177,4.272)--(11.187,4.273)--(11.197,4.274)--(11.208,4.275)%
    --(11.218,4.278)--(11.228,4.280)--(11.238,4.284)--(11.249,4.287)--(11.259,4.291)%
    --(11.269,4.296)--(11.279,4.301)--(11.290,4.307)--(11.300,4.312)--(11.310,4.318)%
    --(11.321,4.325)--(11.331,4.332)--(11.341,4.338)--(11.351,4.346)--(11.362,4.353)%
    --(11.372,4.361)--(11.382,4.368)--(11.392,4.376)--(11.403,4.384)--(11.413,4.392)%
    --(11.423,4.400)--(11.434,4.408)--(11.444,4.416)--(11.454,4.424)--(11.464,4.432)%
    --(11.475,4.439)--(11.485,4.447)--(11.495,4.455)--(11.505,4.462)--(11.516,4.469)%
    --(11.526,4.476)--(11.536,4.483)--(11.546,4.490)--(11.557,4.496)--(11.567,4.502)%
    --(11.577,4.507)--(11.588,4.513)--(11.598,4.518)--(11.608,4.523)--(11.618,4.528)%
    --(11.629,4.532)--(11.639,4.537)--(11.649,4.541)--(11.659,4.545)--(11.670,4.548)%
    --(11.680,4.552)--(11.690,4.555)--(11.701,4.558)--(11.711,4.561)--(11.721,4.564)%
    --(11.731,4.567)--(11.742,4.570)--(11.752,4.572)--(11.762,4.574)--(11.772,4.577)%
    --(11.783,4.579)--(11.793,4.581)--(11.803,4.582)--(11.813,4.584)--(11.824,4.586)%
    --(11.834,4.588)--(11.844,4.589)--(11.855,4.590)--(11.865,4.592)--(11.875,4.593)%
    --(11.885,4.595)--(11.896,4.596)--(11.906,4.597)--(11.916,4.598)--(11.926,4.599)%
    --(11.937,4.600)--(11.947,4.602)--(11.947,0.985)--(1.688,0.985)--cycle;
\draw[gp path] (1.688,1.011)--(1.698,1.011)--(1.709,1.010)--(1.719,1.010)--(1.729,1.010)%
  --(1.739,1.009)--(1.750,1.009)--(1.760,1.009)--(1.770,1.008)--(1.780,1.008)--(1.791,1.008)%
  --(1.801,1.007)--(1.811,1.007)--(1.822,1.007)--(1.832,1.007)--(1.842,1.006)--(1.852,1.006)%
  --(1.863,1.006)--(1.873,1.006)--(1.883,1.005)--(1.893,1.005)--(1.904,1.005)--(1.914,1.005)%
  --(1.924,1.005)--(1.934,1.004)--(1.945,1.004)--(1.955,1.004)--(1.965,1.004)--(1.976,1.004)%
  --(1.986,1.004)--(1.996,1.004)--(2.006,1.004)--(2.017,1.003)--(2.027,1.003)--(2.037,1.003)%
  --(2.047,1.003)--(2.058,1.003)--(2.068,1.003)--(2.078,1.003)--(2.089,1.003)--(2.099,1.003)%
  --(2.109,1.004)--(2.119,1.004)--(2.130,1.004)--(2.140,1.004)--(2.150,1.004)--(2.160,1.004)%
  --(2.171,1.004)--(2.181,1.005)--(2.191,1.005)--(2.201,1.005)--(2.212,1.005)--(2.222,1.006)%
  --(2.232,1.006)--(2.243,1.006)--(2.253,1.006)--(2.263,1.007)--(2.273,1.007)--(2.284,1.007)%
  --(2.294,1.007)--(2.304,1.008)--(2.314,1.008)--(2.325,1.008)--(2.335,1.008)--(2.345,1.009)%
  --(2.356,1.009)--(2.366,1.009)--(2.376,1.009)--(2.386,1.010)--(2.397,1.010)--(2.407,1.010)%
  --(2.417,1.010)--(2.427,1.010)--(2.438,1.010)--(2.448,1.011)--(2.458,1.011)--(2.468,1.011)%
  --(2.479,1.011)--(2.489,1.011)--(2.499,1.011)--(2.510,1.011)--(2.520,1.011)--(2.530,1.011)%
  --(2.540,1.011)--(2.551,1.011)--(2.561,1.011)--(2.571,1.010)--(2.581,1.010)--(2.592,1.010)%
  --(2.602,1.010)--(2.612,1.010)--(2.623,1.010)--(2.633,1.010)--(2.643,1.009)--(2.653,1.009)%
  --(2.664,1.009)--(2.674,1.009)--(2.684,1.009)--(2.694,1.009)--(2.705,1.008)--(2.715,1.008)%
  --(2.725,1.008)--(2.735,1.008)--(2.746,1.008)--(2.756,1.008)--(2.766,1.008)--(2.777,1.008)%
  --(2.787,1.008)--(2.797,1.008)--(2.807,1.008)--(2.818,1.008)--(2.828,1.008)--(2.838,1.008)%
  --(2.848,1.008)--(2.859,1.009)--(2.869,1.009)--(2.879,1.009)--(2.890,1.010)--(2.900,1.010)%
  --(2.910,1.010)--(2.920,1.011)--(2.931,1.011)--(2.941,1.012)--(2.951,1.013)--(2.961,1.013)%
  --(2.972,1.014)--(2.982,1.015)--(2.992,1.016)--(3.002,1.016)--(3.013,1.017)--(3.023,1.018)%
  --(3.033,1.019)--(3.044,1.020)--(3.054,1.021)--(3.064,1.021)--(3.074,1.022)--(3.085,1.023)%
  --(3.095,1.024)--(3.105,1.025)--(3.115,1.026)--(3.126,1.027)--(3.136,1.028)--(3.146,1.028)%
  --(3.157,1.029)--(3.167,1.030)--(3.177,1.031)--(3.187,1.032)--(3.198,1.032)--(3.208,1.033)%
  --(3.218,1.034)--(3.228,1.034)--(3.239,1.035)--(3.249,1.035)--(3.259,1.036)--(3.269,1.036)%
  --(3.280,1.036)--(3.290,1.036)--(3.300,1.037)--(3.311,1.037)--(3.321,1.037)--(3.331,1.037)%
  --(3.341,1.037)--(3.352,1.036)--(3.362,1.036)--(3.372,1.036)--(3.382,1.036)--(3.393,1.035)%
  --(3.403,1.035)--(3.413,1.034)--(3.424,1.034)--(3.434,1.033)--(3.444,1.033)--(3.454,1.032)%
  --(3.465,1.032)--(3.475,1.032)--(3.485,1.031)--(3.495,1.031)--(3.506,1.031)--(3.516,1.030)%
  --(3.526,1.030)--(3.536,1.030)--(3.547,1.030)--(3.557,1.030)--(3.567,1.030)--(3.578,1.031)%
  --(3.588,1.031)--(3.598,1.032)--(3.608,1.032)--(3.619,1.033)--(3.629,1.034)--(3.639,1.035)%
  --(3.649,1.036)--(3.660,1.038)--(3.670,1.039)--(3.680,1.041)--(3.691,1.043)--(3.701,1.045)%
  --(3.711,1.047)--(3.721,1.050)--(3.732,1.053)--(3.742,1.056)--(3.752,1.059)--(3.762,1.063)%
  --(3.773,1.066)--(3.783,1.070)--(3.793,1.074)--(3.803,1.078)--(3.814,1.083)--(3.824,1.087)%
  --(3.834,1.092)--(3.845,1.097)--(3.855,1.102)--(3.865,1.107)--(3.875,1.112)--(3.886,1.117)%
  --(3.896,1.122)--(3.906,1.127)--(3.916,1.132)--(3.927,1.137)--(3.937,1.143)--(3.947,1.148)%
  --(3.958,1.153)--(3.968,1.158)--(3.978,1.163)--(3.988,1.168)--(3.999,1.173)--(4.009,1.177)%
  --(4.019,1.182)--(4.029,1.187)--(4.040,1.191)--(4.050,1.195)--(4.060,1.199)--(4.070,1.203)%
  --(4.081,1.207)--(4.091,1.210)--(4.101,1.213)--(4.112,1.216)--(4.122,1.219)--(4.132,1.222)%
  --(4.142,1.224)--(4.153,1.226)--(4.163,1.227)--(4.173,1.229)--(4.183,1.230)--(4.194,1.231)%
  --(4.204,1.231)--(4.214,1.232)--(4.225,1.232)--(4.235,1.233)--(4.245,1.233)--(4.255,1.233)%
  --(4.266,1.233)--(4.276,1.233)--(4.286,1.233)--(4.296,1.233)--(4.307,1.233)--(4.317,1.233)%
  --(4.327,1.234)--(4.337,1.234)--(4.348,1.235)--(4.358,1.235)--(4.368,1.236)--(4.379,1.237)%
  --(4.389,1.239)--(4.399,1.240)--(4.409,1.242)--(4.420,1.245)--(4.430,1.247)--(4.440,1.250)%
  --(4.450,1.254)--(4.461,1.257)--(4.471,1.262)--(4.481,1.266)--(4.492,1.272)--(4.502,1.277)%
  --(4.512,1.284)--(4.522,1.291)--(4.533,1.298)--(4.543,1.306)--(4.553,1.315)--(4.563,1.324)%
  --(4.574,1.334)--(4.584,1.345)--(4.594,1.356)--(4.604,1.368)--(4.615,1.380)--(4.625,1.393)%
  --(4.635,1.406)--(4.646,1.420)--(4.656,1.433)--(4.666,1.447)--(4.676,1.462)--(4.687,1.476)%
  --(4.697,1.491)--(4.707,1.505)--(4.717,1.520)--(4.728,1.535)--(4.738,1.549)--(4.748,1.564)%
  --(4.759,1.578)--(4.769,1.592)--(4.779,1.606)--(4.789,1.620)--(4.800,1.633)--(4.810,1.646)%
  --(4.820,1.659)--(4.830,1.671)--(4.841,1.682)--(4.851,1.693)--(4.861,1.704)--(4.871,1.713)%
  --(4.882,1.723)--(4.892,1.731)--(4.902,1.738)--(4.913,1.745)--(4.923,1.751)--(4.933,1.756)%
  --(4.943,1.760)--(4.954,1.763)--(4.964,1.765)--(4.974,1.765)--(4.984,1.765)--(4.995,1.764)%
  --(5.005,1.761)--(5.015,1.758)--(5.026,1.754)--(5.036,1.748)--(5.046,1.743)--(5.056,1.736)%
  --(5.067,1.728)--(5.077,1.720)--(5.087,1.712)--(5.097,1.702)--(5.108,1.693)--(5.118,1.683)%
  --(5.128,1.672)--(5.138,1.661)--(5.149,1.650)--(5.159,1.638)--(5.169,1.627)--(5.180,1.615)%
  --(5.190,1.603)--(5.200,1.591)--(5.210,1.579)--(5.221,1.567)--(5.231,1.555)--(5.241,1.544)%
  --(5.251,1.532)--(5.262,1.521)--(5.272,1.511)--(5.282,1.500)--(5.293,1.490)--(5.303,1.481)%
  --(5.313,1.472)--(5.323,1.463)--(5.334,1.455)--(5.344,1.448)--(5.354,1.442)--(5.364,1.436)%
  --(5.375,1.431)--(5.385,1.427)--(5.395,1.424)--(5.405,1.422)--(5.416,1.421)--(5.426,1.420)%
  --(5.436,1.421)--(5.447,1.422)--(5.457,1.424)--(5.467,1.426)--(5.477,1.429)--(5.488,1.433)%
  --(5.498,1.437)--(5.508,1.442)--(5.518,1.448)--(5.529,1.454)--(5.539,1.461)--(5.549,1.468)%
  --(5.560,1.475)--(5.570,1.483)--(5.580,1.491)--(5.590,1.500)--(5.601,1.509)--(5.611,1.518)%
  --(5.621,1.527)--(5.631,1.537)--(5.642,1.547)--(5.652,1.557)--(5.662,1.567)--(5.672,1.577)%
  --(5.683,1.588)--(5.693,1.598)--(5.703,1.608)--(5.714,1.619)--(5.724,1.629)--(5.734,1.640)%
  --(5.744,1.650)--(5.755,1.660)--(5.765,1.670)--(5.775,1.680)--(5.785,1.689)--(5.796,1.699)%
  --(5.806,1.708)--(5.816,1.717)--(5.827,1.725)--(5.837,1.733)--(5.847,1.741)--(5.857,1.749)%
  --(5.868,1.756)--(5.878,1.763)--(5.888,1.770)--(5.898,1.777)--(5.909,1.783)--(5.919,1.788)%
  --(5.929,1.794)--(5.939,1.799)--(5.950,1.803)--(5.960,1.807)--(5.970,1.811)--(5.981,1.815)%
  --(5.991,1.818)--(6.001,1.820)--(6.011,1.822)--(6.022,1.824)--(6.032,1.826)--(6.042,1.826)%
  --(6.052,1.827)--(6.063,1.827)--(6.073,1.826)--(6.083,1.825)--(6.094,1.824)--(6.104,1.822)%
  --(6.114,1.819)--(6.124,1.816)--(6.135,1.813)--(6.145,1.809)--(6.155,1.804)--(6.165,1.799)%
  --(6.176,1.793)--(6.186,1.787)--(6.196,1.780)--(6.206,1.773)--(6.217,1.765)--(6.227,1.757)%
  --(6.237,1.748)--(6.248,1.739)--(6.258,1.730)--(6.268,1.721)--(6.278,1.711)--(6.289,1.702)%
  --(6.299,1.693)--(6.309,1.684)--(6.319,1.676)--(6.330,1.668)--(6.340,1.661)--(6.350,1.654)%
  --(6.361,1.648)--(6.371,1.643)--(6.381,1.638)--(6.391,1.635)--(6.402,1.633)--(6.412,1.632)%
  --(6.422,1.632)--(6.432,1.633)--(6.443,1.636)--(6.453,1.641)--(6.463,1.647)--(6.473,1.654)%
  --(6.484,1.664)--(6.494,1.675)--(6.504,1.689)--(6.515,1.704)--(6.525,1.722)--(6.535,1.742)%
  --(6.545,1.764)--(6.556,1.789)--(6.566,1.816)--(6.576,1.846)--(6.586,1.878)--(6.597,1.913)%
  --(6.607,1.951)--(6.617,1.992)--(6.628,2.036)--(6.638,2.083)--(6.648,2.132)--(6.658,2.184)%
  --(6.669,2.237)--(6.679,2.293)--(6.689,2.350)--(6.699,2.408)--(6.710,2.468)--(6.720,2.528)%
  --(6.730,2.590)--(6.740,2.652)--(6.751,2.714)--(6.761,2.777)--(6.771,2.840)--(6.782,2.902)%
  --(6.792,2.964)--(6.802,3.025)--(6.812,3.086)--(6.823,3.145)--(6.833,3.203)--(6.843,3.260)%
  --(6.853,3.314)--(6.864,3.367)--(6.874,3.418)--(6.884,3.466)--(6.895,3.512)--(6.905,3.555)%
  --(6.915,3.595)--(6.925,3.632)--(6.936,3.666)--(6.946,3.696)--(6.956,3.722)--(6.966,3.744)%
  --(6.977,3.762)--(6.987,3.775)--(6.997,3.784)--(7.007,3.788)--(7.018,3.787)--(7.028,3.780)%
  --(7.038,3.769)--(7.049,3.752)--(7.059,3.730)--(7.069,3.704)--(7.079,3.674)--(7.090,3.640)%
  --(7.100,3.602)--(7.110,3.560)--(7.120,3.515)--(7.131,3.466)--(7.141,3.415)--(7.151,3.362)%
  --(7.162,3.306)--(7.172,3.248)--(7.182,3.188)--(7.192,3.126)--(7.203,3.063)--(7.213,2.999)%
  --(7.223,2.933)--(7.233,2.868)--(7.244,2.801)--(7.254,2.735)--(7.264,2.668)--(7.274,2.602)%
  --(7.285,2.537)--(7.295,2.472)--(7.305,2.408)--(7.316,2.345)--(7.326,2.284)--(7.336,2.225)%
  --(7.346,2.167)--(7.357,2.112)--(7.367,2.060)--(7.377,2.009)--(7.387,1.962)--(7.398,1.918)%
  --(7.408,1.878)--(7.418,1.841)--(7.429,1.808)--(7.439,1.779)--(7.449,1.755)--(7.459,1.734)%
  --(7.470,1.717)--(7.480,1.704)--(7.490,1.694)--(7.500,1.688)--(7.511,1.685)--(7.521,1.685)%
  --(7.531,1.689)--(7.541,1.694)--(7.552,1.703)--(7.562,1.714)--(7.572,1.727)--(7.583,1.743)%
  --(7.593,1.760)--(7.603,1.780)--(7.613,1.801)--(7.624,1.824)--(7.634,1.848)--(7.644,1.873)%
  --(7.654,1.899)--(7.665,1.927)--(7.675,1.955)--(7.685,1.984)--(7.696,2.013)--(7.706,2.043)%
  --(7.716,2.073)--(7.726,2.103)--(7.737,2.132)--(7.747,2.162)--(7.757,2.191)--(7.767,2.220)%
  --(7.778,2.247)--(7.788,2.274)--(7.798,2.300)--(7.808,2.325)--(7.819,2.348)--(7.829,2.370)%
  --(7.839,2.390)--(7.850,2.408)--(7.860,2.425)--(7.870,2.440)--(7.880,2.453)--(7.891,2.464)%
  --(7.901,2.474)--(7.911,2.483)--(7.921,2.490)--(7.932,2.495)--(7.942,2.499)--(7.952,2.502)%
  --(7.963,2.504)--(7.973,2.505)--(7.983,2.504)--(7.993,2.502)--(8.004,2.500)--(8.014,2.497)%
  --(8.024,2.492)--(8.034,2.487)--(8.045,2.482)--(8.055,2.475)--(8.065,2.468)--(8.075,2.461)%
  --(8.086,2.453)--(8.096,2.444)--(8.106,2.436)--(8.117,2.427)--(8.127,2.417)--(8.137,2.408)%
  --(8.147,2.398)--(8.158,2.389)--(8.168,2.379)--(8.178,2.370)--(8.188,2.361)--(8.199,2.351)%
  --(8.209,2.343)--(8.219,2.334)--(8.230,2.326)--(8.240,2.318)--(8.250,2.311)--(8.260,2.305)%
  --(8.271,2.299)--(8.281,2.293)--(8.291,2.289)--(8.301,2.285)--(8.312,2.281)--(8.322,2.278)%
  --(8.332,2.275)--(8.342,2.274)--(8.353,2.272)--(8.363,2.272)--(8.373,2.271)--(8.384,2.272)%
  --(8.394,2.273)--(8.404,2.274)--(8.414,2.276)--(8.425,2.279)--(8.435,2.282)--(8.445,2.286)%
  --(8.455,2.290)--(8.466,2.295)--(8.476,2.300)--(8.486,2.306)--(8.497,2.312)--(8.507,2.319)%
  --(8.517,2.326)--(8.527,2.334)--(8.538,2.343)--(8.548,2.351)--(8.558,2.361)--(8.568,2.371)%
  --(8.579,2.381)--(8.589,2.392)--(8.599,2.403)--(8.609,2.415)--(8.620,2.428)--(8.630,2.440)%
  --(8.640,2.454)--(8.651,2.468)--(8.661,2.482)--(8.671,2.496)--(8.681,2.512)--(8.692,2.527)%
  --(8.702,2.543)--(8.712,2.559)--(8.722,2.576)--(8.733,2.593)--(8.743,2.610)--(8.753,2.627)%
  --(8.764,2.645)--(8.774,2.663)--(8.784,2.681)--(8.794,2.699)--(8.805,2.717)--(8.815,2.735)%
  --(8.825,2.753)--(8.835,2.771)--(8.846,2.789)--(8.856,2.807)--(8.866,2.825)--(8.876,2.843)%
  --(8.887,2.861)--(8.897,2.878)--(8.907,2.896)--(8.918,2.913)--(8.928,2.929)--(8.938,2.946)%
  --(8.948,2.962)--(8.959,2.977)--(8.969,2.993)--(8.979,3.007)--(8.989,3.022)--(9.000,3.036)%
  --(9.010,3.049)--(9.020,3.062)--(9.031,3.074)--(9.041,3.085)--(9.051,3.096)--(9.061,3.107)%
  --(9.072,3.116)--(9.082,3.125)--(9.092,3.133)--(9.102,3.141)--(9.113,3.147)--(9.123,3.154)%
  --(9.133,3.159)--(9.143,3.164)--(9.154,3.169)--(9.164,3.173)--(9.174,3.177)--(9.185,3.181)%
  --(9.195,3.184)--(9.205,3.187)--(9.215,3.190)--(9.226,3.192)--(9.236,3.195)--(9.246,3.197)%
  --(9.256,3.200)--(9.267,3.202)--(9.277,3.204)--(9.287,3.207)--(9.298,3.209)--(9.308,3.212)%
  --(9.318,3.215)--(9.328,3.219)--(9.339,3.222)--(9.349,3.226)--(9.359,3.230)--(9.369,3.235)%
  --(9.380,3.240)--(9.390,3.246)--(9.400,3.252)--(9.410,3.259)--(9.421,3.266)--(9.431,3.274)%
  --(9.441,3.283)--(9.452,3.293)--(9.462,3.303)--(9.472,3.314)--(9.482,3.326)--(9.493,3.340)%
  --(9.503,3.353)--(9.513,3.368)--(9.523,3.384)--(9.534,3.400)--(9.544,3.417)--(9.554,3.435)%
  --(9.565,3.453)--(9.575,3.472)--(9.585,3.491)--(9.595,3.511)--(9.606,3.531)--(9.616,3.552)%
  --(9.626,3.572)--(9.636,3.593)--(9.647,3.615)--(9.657,3.636)--(9.667,3.657)--(9.677,3.679)%
  --(9.688,3.700)--(9.698,3.722)--(9.708,3.743)--(9.719,3.764)--(9.729,3.785)--(9.739,3.806)%
  --(9.749,3.826)--(9.760,3.846)--(9.770,3.865)--(9.780,3.885)--(9.790,3.903)--(9.801,3.921)%
  --(9.811,3.938)--(9.821,3.955)--(9.832,3.971)--(9.842,3.986)--(9.852,4.001)--(9.862,4.014)%
  --(9.873,4.027)--(9.883,4.038)--(9.893,4.049)--(9.903,4.058)--(9.914,4.067)--(9.924,4.074)%
  --(9.934,4.080)--(9.944,4.086)--(9.955,4.091)--(9.965,4.095)--(9.975,4.098)--(9.986,4.100)%
  --(9.996,4.102)--(10.006,4.103)--(10.016,4.104)--(10.027,4.104)--(10.037,4.104)--(10.047,4.103)%
  --(10.057,4.102)--(10.068,4.101)--(10.078,4.099)--(10.088,4.098)--(10.099,4.096)--(10.109,4.094)%
  --(10.119,4.091)--(10.129,4.089)--(10.140,4.087)--(10.150,4.085)--(10.160,4.084)--(10.170,4.082)%
  --(10.181,4.081)--(10.191,4.079)--(10.201,4.079)--(10.211,4.078)--(10.222,4.079)--(10.232,4.079)%
  --(10.242,4.080)--(10.253,4.082)--(10.263,4.084)--(10.273,4.088)--(10.283,4.091)--(10.294,4.096)%
  --(10.304,4.101)--(10.314,4.108)--(10.324,4.115)--(10.335,4.123)--(10.345,4.132)--(10.355,4.141)%
  --(10.366,4.151)--(10.376,4.162)--(10.386,4.174)--(10.396,4.185)--(10.407,4.198)--(10.417,4.210)%
  --(10.427,4.224)--(10.437,4.237)--(10.448,4.251)--(10.458,4.265)--(10.468,4.279)--(10.478,4.293)%
  --(10.489,4.307)--(10.499,4.321)--(10.509,4.336)--(10.520,4.350)--(10.530,4.364)--(10.540,4.378)%
  --(10.550,4.391)--(10.561,4.405)--(10.571,4.418)--(10.581,4.430)--(10.591,4.443)--(10.602,4.454)%
  --(10.612,4.466)--(10.622,4.476)--(10.633,4.486)--(10.643,4.496)--(10.653,4.504)--(10.663,4.512)%
  --(10.674,4.519)--(10.684,4.526)--(10.694,4.531)--(10.704,4.535)--(10.715,4.538)--(10.725,4.541)%
  --(10.735,4.542)--(10.745,4.542)--(10.756,4.541)--(10.766,4.540)--(10.776,4.537)--(10.787,4.534)%
  --(10.797,4.530)--(10.807,4.526)--(10.817,4.520)--(10.828,4.514)--(10.838,4.508)--(10.848,4.501)%
  --(10.858,4.493)--(10.869,4.485)--(10.879,4.477)--(10.889,4.468)--(10.900,4.459)--(10.910,4.450)%
  --(10.920,4.440)--(10.930,4.431)--(10.941,4.421)--(10.951,4.411)--(10.961,4.401)--(10.971,4.392)%
  --(10.982,4.382)--(10.992,4.372)--(11.002,4.363)--(11.012,4.354)--(11.023,4.345)--(11.033,4.336)%
  --(11.043,4.328)--(11.054,4.320)--(11.064,4.313)--(11.074,4.306)--(11.084,4.299)--(11.095,4.294)%
  --(11.105,4.288)--(11.115,4.284)--(11.125,4.280)--(11.136,4.277)--(11.146,4.275)--(11.156,4.273)%
  --(11.167,4.272)--(11.177,4.272)--(11.187,4.273)--(11.197,4.274)--(11.208,4.275)--(11.218,4.278)%
  --(11.228,4.280)--(11.238,4.284)--(11.249,4.287)--(11.259,4.291)--(11.269,4.296)--(11.279,4.301)%
  --(11.290,4.307)--(11.300,4.312)--(11.310,4.318)--(11.321,4.325)--(11.331,4.332)--(11.341,4.338)%
  --(11.351,4.346)--(11.362,4.353)--(11.372,4.361)--(11.382,4.368)--(11.392,4.376)--(11.403,4.384)%
  --(11.413,4.392)--(11.423,4.400)--(11.434,4.408)--(11.444,4.416)--(11.454,4.424)--(11.464,4.432)%
  --(11.475,4.439)--(11.485,4.447)--(11.495,4.455)--(11.505,4.462)--(11.516,4.469)--(11.526,4.476)%
  --(11.536,4.483)--(11.546,4.490)--(11.557,4.496)--(11.567,4.502)--(11.577,4.507)--(11.588,4.513)%
  --(11.598,4.518)--(11.608,4.523)--(11.618,4.528)--(11.629,4.532)--(11.639,4.537)--(11.649,4.541)%
  --(11.659,4.545)--(11.670,4.548)--(11.680,4.552)--(11.690,4.555)--(11.701,4.558)--(11.711,4.561)%
  --(11.721,4.564)--(11.731,4.567)--(11.742,4.570)--(11.752,4.572)--(11.762,4.574)--(11.772,4.577)%
  --(11.783,4.579)--(11.793,4.581)--(11.803,4.582)--(11.813,4.584)--(11.824,4.586)--(11.834,4.588)%
  --(11.844,4.589)--(11.855,4.590)--(11.865,4.592)--(11.875,4.593)--(11.885,4.595)--(11.896,4.596)%
  --(11.906,4.597)--(11.916,4.598)--(11.926,4.599)--(11.937,4.600)--(11.947,4.602);
\gpcolor{color=gp lt color border}
\gpsetlinetype{gp lt border}
\gpsetlinewidth{1.00}
\draw[gp path] (1.688,8.381)--(1.688,0.985)--(11.947,0.985)--(11.947,8.381)--cycle;
%% coordinates of the plot area
\gpdefrectangularnode{gp plot 1}{\pgfpoint{1.688cm}{0.985cm}}{\pgfpoint{11.947cm}{8.381cm}}
\end{tikzpicture}
%% gnuplot variables

      \end{myplot}

      \begin{myplot}%
        {Распределение методов от значения меры точности для \eng{SPECjvm2008}}%
        {plot:specjvm2008_all_aliases_distribution_cumulative}
        \begin{tikzpicture}[gnuplot]
%% generated with GNUPLOT 4.5p0 (Lua 5.1; terminal rev. 99, script rev. 98)
%% 27.05.2011 12:44:18
\path (0.000,0.000) rectangle (12.500,8.750);
\gpcolor{color=gp lt color border}
\gpsetlinetype{gp lt border}
\gpsetlinewidth{1.00}
\draw[gp path] (1.504,0.985)--(1.684,0.985);
\draw[gp path] (11.947,0.985)--(11.767,0.985);
\node[gp node right] at (1.320,0.985) {\num{0}};
\draw[gp path] (1.504,2.330)--(1.684,2.330);
\draw[gp path] (11.947,2.330)--(11.767,2.330);
\node[gp node right] at (1.320,2.330) {\num{0.2}};
\draw[gp path] (1.504,3.674)--(1.684,3.674);
\draw[gp path] (11.947,3.674)--(11.767,3.674);
\node[gp node right] at (1.320,3.674) {\num{0.4}};
\draw[gp path] (1.504,5.019)--(1.684,5.019);
\draw[gp path] (11.947,5.019)--(11.767,5.019);
\node[gp node right] at (1.320,5.019) {\num{0.6}};
\draw[gp path] (1.504,6.364)--(1.684,6.364);
\draw[gp path] (11.947,6.364)--(11.767,6.364);
\node[gp node right] at (1.320,6.364) {\num{0.8}};
\draw[gp path] (1.504,7.709)--(1.684,7.709);
\draw[gp path] (11.947,7.709)--(11.767,7.709);
\node[gp node right] at (1.320,7.709) {\num{1}};
\gpcolor{color=gp lt color axes}
\gpsetlinetype{gp lt axes}
\draw[gp path] (3.508,0.985)--(3.508,6.969);
\draw[gp path] (3.508,8.201)--(3.508,8.381);
\gpcolor{color=gp lt color border}
\gpsetlinetype{gp lt border}
\draw[gp path] (3.508,0.985)--(3.508,1.165);
\draw[gp path] (3.508,8.381)--(3.508,8.201);
\node[gp node center] at (3.508,0.677) {\num{0.2}};
\gpcolor{color=gp lt color axes}
\gpsetlinetype{gp lt axes}
\draw[gp path] (5.618,0.985)--(5.618,8.381);
\gpcolor{color=gp lt color border}
\gpsetlinetype{gp lt border}
\draw[gp path] (5.618,0.985)--(5.618,1.165);
\draw[gp path] (5.618,8.381)--(5.618,8.201);
\node[gp node center] at (5.618,0.677) {\num{0.4}};
\gpcolor{color=gp lt color axes}
\gpsetlinetype{gp lt axes}
\draw[gp path] (7.728,0.985)--(7.728,8.381);
\gpcolor{color=gp lt color border}
\gpsetlinetype{gp lt border}
\draw[gp path] (7.728,0.985)--(7.728,1.165);
\draw[gp path] (7.728,8.381)--(7.728,8.201);
\node[gp node center] at (7.728,0.677) {\num{0.6}};
\gpcolor{color=gp lt color axes}
\gpsetlinetype{gp lt axes}
\draw[gp path] (9.837,0.985)--(9.837,8.381);
\gpcolor{color=gp lt color border}
\gpsetlinetype{gp lt border}
\draw[gp path] (9.837,0.985)--(9.837,1.165);
\draw[gp path] (9.837,8.381)--(9.837,8.201);
\node[gp node center] at (9.837,0.677) {\num{0.8}};
\gpcolor{color=gp lt color axes}
\gpsetlinetype{gp lt axes}
\draw[gp path] (11.947,0.985)--(11.947,8.381);
\gpcolor{color=gp lt color border}
\gpsetlinetype{gp lt border}
\draw[gp path] (11.947,0.985)--(11.947,1.165);
\draw[gp path] (11.947,8.381)--(11.947,8.201);
\node[gp node center] at (11.947,0.677) {\num{1}};
\draw[gp path] (1.504,8.381)--(1.504,0.985)--(11.947,0.985)--(11.947,8.381)--cycle;
\node[gp node center,rotate=-270] at (0.246,4.683) {Количество методов, \%};
\node[gp node center] at (6.725,0.215) {Отношение среднего количества синонимов к числу переменных};
\node[gp node right] at (4.264,8.047) {base};
\gpcolor{color=gp lt color 0}
\gpsetlinetype{gp lt plot 0}
\gpsetlinewidth{2.00}
\draw[gp path] (4.448,8.047)--(5.364,8.047);
\draw[gp path] (1.504,0.987)--(1.609,0.988)--(1.715,0.990)--(1.926,0.991)--(2.242,0.996)%
  --(2.348,0.998)--(2.559,1.004)--(2.770,1.007)--(2.875,1.017)--(2.981,1.018)--(3.086,1.022)%
  --(3.192,1.031)--(3.297,1.046)--(3.403,1.055)--(3.508,1.064)--(3.614,1.074)--(3.719,1.099)%
  --(3.825,1.115)--(3.930,1.140)--(4.036,1.169)--(4.141,1.202)--(4.247,1.218)--(4.352,1.266)%
  --(4.458,1.292)--(4.563,1.339)--(4.669,1.376)--(4.774,1.421)--(4.880,1.514)--(4.985,1.588)%
  --(5.090,1.637)--(5.196,1.704)--(5.301,1.768)--(5.407,1.850)--(5.512,1.961)--(5.618,2.051)%
  --(5.723,2.108)--(5.829,2.221)--(5.934,2.383)--(6.040,2.496)--(6.145,2.576)--(6.251,2.665)%
  --(6.356,2.758)--(6.462,2.873)--(6.567,2.965)--(6.673,3.058)--(6.778,3.179)--(6.884,3.266)%
  --(6.989,3.405)--(7.095,3.509)--(7.200,3.614)--(7.306,3.737)--(7.411,3.830)--(7.517,3.986)%
  --(7.622,4.144)--(7.728,4.308)--(7.833,4.403)--(7.939,4.571)--(8.044,4.732)--(8.150,4.888)%
  --(8.255,4.959)--(8.361,5.105)--(8.466,5.215)--(8.571,5.355)--(8.677,5.449)--(8.782,5.651)%
  --(8.888,5.728)--(8.993,5.903)--(9.099,5.985)--(9.204,6.133)--(9.310,6.242)--(9.415,6.390)%
  --(9.521,6.459)--(9.626,6.599)--(9.732,6.682)--(9.837,6.876)--(9.943,6.929)--(10.048,7.015)%
  --(10.154,7.066)--(10.259,7.149)--(10.365,7.208)--(10.470,7.267)--(10.576,7.300)--(10.681,7.434)%
  --(10.787,7.454)--(10.892,7.501)--(10.998,7.528)--(11.103,7.584)--(11.209,7.602)--(11.314,7.635)%
  --(11.420,7.656)--(11.525,7.678)--(11.631,7.682)--(11.736,7.697)--(11.842,7.702)--(11.947,7.705);
\gpcolor{color=gp lt color border}
\node[gp node right] at (4.264,7.739) {equality-based};
\gpcolor{color=gp lt color 2}
\draw[gp path] (4.448,7.739)--(5.364,7.739);
\draw[gp path] (1.504,0.987)--(1.609,0.988)--(1.715,0.990)--(1.926,0.991)--(2.242,0.996)%
  --(2.348,0.998)--(2.559,1.004)--(2.770,1.007)--(2.875,1.016)--(2.981,1.017)--(3.086,1.022)%
  --(3.192,1.030)--(3.297,1.042)--(3.403,1.050)--(3.508,1.056)--(3.614,1.062)--(3.719,1.085)%
  --(3.825,1.100)--(3.930,1.119)--(4.036,1.144)--(4.141,1.174)--(4.247,1.185)--(4.352,1.227)%
  --(4.458,1.249)--(4.563,1.278)--(4.669,1.303)--(4.774,1.345)--(4.880,1.428)--(4.985,1.488)%
  --(5.090,1.513)--(5.196,1.577)--(5.301,1.628)--(5.407,1.686)--(5.512,1.761)--(5.618,1.828)%
  --(5.723,1.897)--(5.829,1.963)--(5.934,2.115)--(6.040,2.212)--(6.145,2.281)--(6.251,2.387)%
  --(6.356,2.508)--(6.462,2.602)--(6.567,2.683)--(6.673,2.780)--(6.778,2.903)--(6.884,2.993)%
  --(6.989,3.123)--(7.095,3.200)--(7.200,3.274)--(7.306,3.393)--(7.411,3.479)--(7.517,3.605)%
  --(7.622,3.739)--(7.728,3.892)--(7.833,4.009)--(7.939,4.187)--(8.044,4.333)--(8.150,4.497)%
  --(8.255,4.591)--(8.361,4.737)--(8.466,4.854)--(8.571,4.993)--(8.677,5.111)--(8.782,5.341)%
  --(8.888,5.421)--(8.993,5.601)--(9.099,5.680)--(9.204,5.846)--(9.310,5.961)--(9.415,6.127)%
  --(9.521,6.233)--(9.626,6.375)--(9.732,6.468)--(9.837,6.690)--(9.943,6.769)--(10.048,6.873)%
  --(10.154,6.942)--(10.259,7.069)--(10.365,7.146)--(10.470,7.216)--(10.576,7.259)--(10.681,7.391)%
  --(10.787,7.420)--(10.892,7.486)--(10.998,7.515)--(11.103,7.569)--(11.209,7.592)--(11.314,7.629)%
  --(11.420,7.648)--(11.525,7.676)--(11.631,7.682)--(11.736,7.696)--(11.842,7.701)--(11.947,7.708);
\gpcolor{color=gp lt color border}
\node[gp node right] at (4.264,7.431) {w/o data flow};
\gpcolor{color=gp lt color 6}
\draw[gp path] (4.448,7.431)--(5.364,7.431);
\draw[gp path] (1.926,0.986)--(2.242,0.989)--(2.875,0.996)--(2.981,0.998)--(3.086,1.003)%
  --(3.192,1.004)--(3.297,1.015)--(3.403,1.018)--(3.508,1.022)--(3.614,1.023)--(3.719,1.038)%
  --(3.825,1.039)--(3.930,1.045)--(4.036,1.051)--(4.141,1.056)--(4.352,1.065)--(4.458,1.068)%
  --(4.563,1.076)--(4.669,1.082)--(4.774,1.088)--(4.880,1.095)--(4.985,1.107)--(5.090,1.118)%
  --(5.196,1.141)--(5.301,1.146)--(5.407,1.171)--(5.512,1.191)--(5.618,1.209)--(5.723,1.222)%
  --(5.829,1.238)--(5.934,1.266)--(6.040,1.297)--(6.145,1.308)--(6.251,1.331)--(6.356,1.368)%
  --(6.462,1.392)--(6.567,1.407)--(6.673,1.441)--(6.778,1.464)--(6.884,1.497)--(6.989,1.532)%
  --(7.095,1.562)--(7.200,1.596)--(7.306,1.649)--(7.411,1.684)--(7.517,1.727)--(7.622,1.818)%
  --(7.728,1.901)--(7.833,1.931)--(7.939,2.010)--(8.044,2.107)--(8.150,2.219)--(8.255,2.285)%
  --(8.361,2.418)--(8.466,2.527)--(8.571,2.630)--(8.677,2.711)--(8.782,2.871)--(8.888,2.948)%
  --(8.993,3.180)--(9.099,3.280)--(9.204,3.451)--(9.310,3.596)--(9.415,3.819)--(9.521,3.907)%
  --(9.626,4.119)--(9.732,4.245)--(9.837,4.567)--(9.943,4.712)--(10.048,4.926)--(10.154,5.039)%
  --(10.259,5.291)--(10.365,5.461)--(10.470,5.612)--(10.576,5.730)--(10.681,6.063)--(10.787,6.184)%
  --(10.892,6.369)--(10.998,6.488)--(11.103,6.757)--(11.209,6.846)--(11.314,7.021)--(11.420,7.120)%
  --(11.525,7.320)--(11.631,7.415)--(11.736,7.576)--(11.842,7.680)--(11.947,7.715);
\gpcolor{color=gp lt color border}
\node[gp node right] at (4.264,7.123) {w/o types};
\gpcolor{color=gp lt color 1}
\draw[gp path] (4.448,7.123)--(5.364,7.123);
\draw[gp path] (1.504,0.988)--(1.609,0.990)--(1.715,0.991)--(1.926,0.992)--(2.242,0.995)%
  --(2.348,0.996)--(2.559,1.001)--(2.875,1.004)--(3.192,1.010)--(3.297,1.017)--(3.508,1.018)%
  --(3.614,1.023)--(3.825,1.041)--(3.930,1.046)--(4.036,1.058)--(4.141,1.077)--(4.247,1.090)%
  --(4.352,1.109)--(4.458,1.124)--(4.563,1.133)--(4.669,1.155)--(4.774,1.163)--(4.880,1.260)%
  --(4.985,1.286)--(5.090,1.300)--(5.196,1.321)--(5.301,1.331)--(5.407,1.362)--(5.512,1.397)%
  --(5.618,1.429)--(5.723,1.465)--(5.829,1.503)--(5.934,1.517)--(6.040,1.591)--(6.145,1.606)%
  --(6.251,1.632)--(6.356,1.672)--(6.462,1.726)--(6.567,1.781)--(6.673,1.844)--(6.778,2.074)%
  --(6.884,2.119)--(6.989,2.280)--(7.095,2.334)--(7.200,2.357)--(7.306,2.418)--(7.411,2.435)%
  --(7.517,2.511)--(7.622,2.543)--(7.728,2.640)--(7.833,2.667)--(7.939,2.743)--(8.044,2.807)%
  --(8.150,2.894)--(8.255,2.926)--(8.361,2.984)--(8.466,3.070)--(8.571,3.136)--(8.677,3.170)%
  --(8.782,3.287)--(8.888,3.330)--(8.993,3.442)--(9.099,3.493)--(9.204,3.653)--(9.310,3.719)%
  --(9.415,3.940)--(9.521,3.992)--(9.626,4.095)--(9.732,4.265)--(9.837,4.396)--(9.943,4.541)%
  --(10.048,4.659)--(10.154,4.788)--(10.259,4.975)--(10.365,5.089)--(10.470,5.246)--(10.576,5.374)%
  --(10.681,5.596)--(10.787,5.706)--(10.892,5.915)--(10.998,6.025)--(11.103,6.220)--(11.209,6.357)%
  --(11.314,6.488)--(11.420,6.627)--(11.525,6.796)--(11.631,6.956)--(11.736,7.178)--(11.842,7.379)%
  --(11.947,7.700);
\gpcolor{color=gp lt color border}
\gpsetlinetype{gp lt border}
\gpsetlinewidth{1.00}
\draw[gp path] (1.504,8.381)--(1.504,0.985)--(11.947,0.985)--(11.947,8.381)--cycle;
%% coordinates of the plot area
\gpdefrectangularnode{gp plot 1}{\pgfpoint{1.504cm}{0.985cm}}{\pgfpoint{11.947cm}{8.381cm}}
\end{tikzpicture}
%% gnuplot variables

      \end{myplot}

      \begin{myplot}%
        {Зависимость количества методов от значения меры точности для \eng{Eclipse~IDE}}%
        {plot:eclipse_all_aliases_distribution}
        \begin{tikzpicture}[gnuplot]
%% generated with GNUPLOT 4.5p0 (Lua 5.1; terminal rev. 99, script rev. 98)
%% 27.05.2011 12:48:41
\path (0.000,0.000) rectangle (12.500,8.750);
\gpcolor{color=gp lt color border}
\gpsetlinetype{gp lt border}
\gpsetlinewidth{1.00}
\draw[gp path] (1.688,0.985)--(1.868,0.985);
\draw[gp path] (11.947,0.985)--(11.767,0.985);
\node[gp node right] at (1.504,0.985) {\num{0}};
\draw[gp path] (1.688,2.834)--(1.868,2.834);
\draw[gp path] (11.947,2.834)--(11.767,2.834);
\node[gp node right] at (1.504,2.834) {\num{0.05}};
\draw[gp path] (1.688,4.683)--(1.868,4.683);
\draw[gp path] (11.947,4.683)--(11.767,4.683);
\node[gp node right] at (1.504,4.683) {\num{0.1}};
\draw[gp path] (1.688,6.532)--(1.868,6.532);
\draw[gp path] (11.947,6.532)--(11.767,6.532);
\node[gp node right] at (1.504,6.532) {\num{0.15}};
\draw[gp path] (1.688,8.381)--(1.868,8.381);
\draw[gp path] (11.947,8.381)--(11.767,8.381);
\node[gp node right] at (1.504,8.381) {\num{0.2}};
\draw[gp path] (1.688,0.985)--(1.688,1.165);
\draw[gp path] (1.688,8.381)--(1.688,8.201);
\node[gp node center] at (1.688,0.677) {\num{0}};
\draw[gp path] (3.740,0.985)--(3.740,1.165);
\draw[gp path] (3.740,8.381)--(3.740,8.201);
\node[gp node center] at (3.740,0.677) {\num{0.2}};
\draw[gp path] (5.792,0.985)--(5.792,1.165);
\draw[gp path] (5.792,8.381)--(5.792,8.201);
\node[gp node center] at (5.792,0.677) {\num{0.4}};
\draw[gp path] (7.843,0.985)--(7.843,1.165);
\draw[gp path] (7.843,8.381)--(7.843,8.201);
\node[gp node center] at (7.843,0.677) {\num{0.6}};
\draw[gp path] (9.895,0.985)--(9.895,1.165);
\draw[gp path] (9.895,8.381)--(9.895,8.201);
\node[gp node center] at (9.895,0.677) {\num{0.8}};
\draw[gp path] (11.947,0.985)--(11.947,1.165);
\draw[gp path] (11.947,8.381)--(11.947,8.201);
\node[gp node center] at (11.947,0.677) {\num{1}};
\draw[gp path] (1.688,8.381)--(1.688,0.985)--(11.947,0.985)--(11.947,8.381)--cycle;
\node[gp node center,rotate=-270] at (0.246,4.683) {Количество методов, \%};
\node[gp node center] at (6.817,0.215) {Отношение среднего количества синонимов к числу переменных};
\node[gp node right] at (4.448,8.047) {base};
\gpfill{color=gp lt color 0,opacity=0.10} (4.632,7.970)--(5.548,7.970)--(5.548,8.124)--(4.632,8.124)--cycle;
\gpcolor{color=gp lt color 0}
\gpsetlinetype{gp lt plot 0}
\gpsetlinewidth{2.00}
\draw[gp path] (4.632,7.970)--(5.548,7.970)--(5.548,8.124)--(4.632,8.124)--cycle;
\gpfill{color=gp lt color 0,opacity=0.10} (1.688,0.989)--(1.688,0.989)--(1.698,0.989)--(1.709,0.989)%
    --(1.719,0.989)--(1.729,0.989)--(1.739,0.989)--(1.750,0.990)--(1.760,0.990)%
    --(1.770,0.990)--(1.780,0.990)--(1.791,0.990)--(1.801,0.990)--(1.811,0.991)%
    --(1.822,0.991)--(1.832,0.991)--(1.842,0.991)--(1.852,0.991)--(1.863,0.992)%
    --(1.873,0.992)--(1.883,0.992)--(1.893,0.992)--(1.904,0.993)--(1.914,0.993)%
    --(1.924,0.993)--(1.934,0.993)--(1.945,0.994)--(1.955,0.994)--(1.965,0.994)%
    --(1.976,0.995)--(1.986,0.995)--(1.996,0.995)--(2.006,0.996)--(2.017,0.996)%
    --(2.027,0.997)--(2.037,0.997)--(2.047,0.997)--(2.058,0.998)--(2.068,0.998)%
    --(2.078,0.999)--(2.089,0.999)--(2.099,1.000)--(2.109,1.000)--(2.119,1.001)%
    --(2.130,1.001)--(2.140,1.002)--(2.150,1.003)--(2.160,1.003)--(2.171,1.004)%
    --(2.181,1.005)--(2.191,1.005)--(2.201,1.006)--(2.212,1.007)--(2.222,1.008)%
    --(2.232,1.009)--(2.243,1.009)--(2.253,1.010)--(2.263,1.011)--(2.273,1.012)%
    --(2.284,1.013)--(2.294,1.014)--(2.304,1.015)--(2.314,1.016)--(2.325,1.017)%
    --(2.335,1.018)--(2.345,1.020)--(2.356,1.021)--(2.366,1.022)--(2.376,1.023)%
    --(2.386,1.025)--(2.397,1.026)--(2.407,1.028)--(2.417,1.029)--(2.427,1.031)%
    --(2.438,1.032)--(2.448,1.034)--(2.458,1.035)--(2.468,1.037)--(2.479,1.039)%
    --(2.489,1.041)--(2.499,1.042)--(2.510,1.044)--(2.520,1.046)--(2.530,1.048)%
    --(2.540,1.050)--(2.551,1.052)--(2.561,1.054)--(2.571,1.057)--(2.581,1.059)%
    --(2.592,1.061)--(2.602,1.063)--(2.612,1.065)--(2.623,1.068)--(2.633,1.070)%
    --(2.643,1.072)--(2.653,1.074)--(2.664,1.077)--(2.674,1.079)--(2.684,1.081)%
    --(2.694,1.083)--(2.705,1.086)--(2.715,1.088)--(2.725,1.090)--(2.735,1.092)%
    --(2.746,1.094)--(2.756,1.096)--(2.766,1.098)--(2.777,1.100)--(2.787,1.102)%
    --(2.797,1.104)--(2.807,1.106)--(2.818,1.108)--(2.828,1.110)--(2.838,1.111)%
    --(2.848,1.113)--(2.859,1.114)--(2.869,1.116)--(2.879,1.117)--(2.890,1.119)%
    --(2.900,1.120)--(2.910,1.121)--(2.920,1.122)--(2.931,1.123)--(2.941,1.124)%
    --(2.951,1.125)--(2.961,1.125)--(2.972,1.126)--(2.982,1.127)--(2.992,1.127)%
    --(3.002,1.128)--(3.013,1.129)--(3.023,1.129)--(3.033,1.130)--(3.044,1.131)%
    --(3.054,1.131)--(3.064,1.132)--(3.074,1.133)--(3.085,1.134)--(3.095,1.136)%
    --(3.105,1.137)--(3.115,1.139)--(3.126,1.140)--(3.136,1.142)--(3.146,1.144)%
    --(3.157,1.146)--(3.167,1.149)--(3.177,1.151)--(3.187,1.154)--(3.198,1.157)%
    --(3.208,1.161)--(3.218,1.164)--(3.228,1.168)--(3.239,1.172)--(3.249,1.177)%
    --(3.259,1.182)--(3.269,1.187)--(3.280,1.193)--(3.290,1.199)--(3.300,1.205)%
    --(3.311,1.212)--(3.321,1.219)--(3.331,1.227)--(3.341,1.235)--(3.352,1.243)%
    --(3.362,1.252)--(3.372,1.261)--(3.382,1.270)--(3.393,1.280)--(3.403,1.290)%
    --(3.413,1.301)--(3.424,1.311)--(3.434,1.322)--(3.444,1.333)--(3.454,1.344)%
    --(3.465,1.355)--(3.475,1.366)--(3.485,1.378)--(3.495,1.389)--(3.506,1.401)%
    --(3.516,1.412)--(3.526,1.423)--(3.536,1.435)--(3.547,1.446)--(3.557,1.457)%
    --(3.567,1.468)--(3.578,1.479)--(3.588,1.490)--(3.598,1.500)--(3.608,1.510)%
    --(3.619,1.520)--(3.629,1.530)--(3.639,1.540)--(3.649,1.549)--(3.660,1.557)%
    --(3.670,1.566)--(3.680,1.574)--(3.691,1.581)--(3.701,1.588)--(3.711,1.595)%
    --(3.721,1.601)--(3.732,1.606)--(3.742,1.611)--(3.752,1.615)--(3.762,1.619)%
    --(3.773,1.622)--(3.783,1.625)--(3.793,1.627)--(3.803,1.629)--(3.814,1.631)%
    --(3.824,1.632)--(3.834,1.633)--(3.845,1.634)--(3.855,1.634)--(3.865,1.634)%
    --(3.875,1.634)--(3.886,1.634)--(3.896,1.633)--(3.906,1.633)--(3.916,1.632)%
    --(3.927,1.632)--(3.937,1.631)--(3.947,1.631)--(3.958,1.630)--(3.968,1.630)%
    --(3.978,1.630)--(3.988,1.630)--(3.999,1.630)--(4.009,1.630)--(4.019,1.631)%
    --(4.029,1.632)--(4.040,1.633)--(4.050,1.635)--(4.060,1.637)--(4.070,1.639)%
    --(4.081,1.642)--(4.091,1.646)--(4.101,1.650)--(4.112,1.654)--(4.122,1.659)%
    --(4.132,1.665)--(4.142,1.671)--(4.153,1.678)--(4.163,1.686)--(4.173,1.694)%
    --(4.183,1.703)--(4.194,1.713)--(4.204,1.723)--(4.214,1.734)--(4.225,1.745)%
    --(4.235,1.757)--(4.245,1.770)--(4.255,1.782)--(4.266,1.796)--(4.276,1.809)%
    --(4.286,1.823)--(4.296,1.838)--(4.307,1.852)--(4.317,1.867)--(4.327,1.882)%
    --(4.337,1.898)--(4.348,1.914)--(4.358,1.929)--(4.368,1.945)--(4.379,1.962)%
    --(4.389,1.978)--(4.399,1.994)--(4.409,2.010)--(4.420,2.027)--(4.430,2.043)%
    --(4.440,2.059)--(4.450,2.076)--(4.461,2.092)--(4.471,2.108)--(4.481,2.124)%
    --(4.492,2.140)--(4.502,2.155)--(4.512,2.170)--(4.522,2.185)--(4.533,2.200)%
    --(4.543,2.215)--(4.553,2.229)--(4.563,2.242)--(4.574,2.256)--(4.584,2.269)%
    --(4.594,2.281)--(4.604,2.294)--(4.615,2.306)--(4.625,2.317)--(4.635,2.328)%
    --(4.646,2.339)--(4.656,2.350)--(4.666,2.360)--(4.676,2.370)--(4.687,2.380)%
    --(4.697,2.390)--(4.707,2.399)--(4.717,2.408)--(4.728,2.416)--(4.738,2.425)%
    --(4.748,2.433)--(4.759,2.441)--(4.769,2.449)--(4.779,2.457)--(4.789,2.464)%
    --(4.800,2.471)--(4.810,2.478)--(4.820,2.485)--(4.830,2.492)--(4.841,2.498)%
    --(4.851,2.505)--(4.861,2.511)--(4.871,2.517)--(4.882,2.523)--(4.892,2.529)%
    --(4.902,2.535)--(4.913,2.540)--(4.923,2.546)--(4.933,2.552)--(4.943,2.557)%
    --(4.954,2.562)--(4.964,2.568)--(4.974,2.573)--(4.984,2.578)--(4.995,2.584)%
    --(5.005,2.589)--(5.015,2.594)--(5.026,2.600)--(5.036,2.605)--(5.046,2.610)%
    --(5.056,2.616)--(5.067,2.621)--(5.077,2.627)--(5.087,2.632)--(5.097,2.638)%
    --(5.108,2.644)--(5.118,2.650)--(5.128,2.656)--(5.138,2.662)--(5.149,2.669)%
    --(5.159,2.675)--(5.169,2.682)--(5.180,2.689)--(5.190,2.696)--(5.200,2.703)%
    --(5.210,2.711)--(5.221,2.719)--(5.231,2.727)--(5.241,2.735)--(5.251,2.743)%
    --(5.262,2.752)--(5.272,2.761)--(5.282,2.771)--(5.293,2.780)--(5.303,2.790)%
    --(5.313,2.801)--(5.323,2.811)--(5.334,2.822)--(5.344,2.834)--(5.354,2.845)%
    --(5.364,2.858)--(5.375,2.870)--(5.385,2.883)--(5.395,2.896)--(5.405,2.910)%
    --(5.416,2.924)--(5.426,2.939)--(5.436,2.953)--(5.447,2.968)--(5.457,2.983)%
    --(5.467,2.999)--(5.477,3.014)--(5.488,3.030)--(5.498,3.046)--(5.508,3.062)%
    --(5.518,3.077)--(5.529,3.093)--(5.539,3.109)--(5.549,3.125)--(5.560,3.141)%
    --(5.570,3.156)--(5.580,3.172)--(5.590,3.187)--(5.601,3.202)--(5.611,3.217)%
    --(5.621,3.231)--(5.631,3.246)--(5.642,3.260)--(5.652,3.273)--(5.662,3.287)%
    --(5.672,3.299)--(5.683,3.312)--(5.693,3.324)--(5.703,3.335)--(5.714,3.346)%
    --(5.724,3.356)--(5.734,3.366)--(5.744,3.375)--(5.755,3.383)--(5.765,3.391)%
    --(5.775,3.398)--(5.785,3.404)--(5.796,3.409)--(5.806,3.414)--(5.816,3.418)%
    --(5.827,3.421)--(5.837,3.423)--(5.847,3.425)--(5.857,3.426)--(5.868,3.427)%
    --(5.878,3.427)--(5.888,3.426)--(5.898,3.425)--(5.909,3.423)--(5.919,3.421)%
    --(5.929,3.418)--(5.939,3.415)--(5.950,3.412)--(5.960,3.408)--(5.970,3.404)%
    --(5.981,3.400)--(5.991,3.395)--(6.001,3.391)--(6.011,3.386)--(6.022,3.380)%
    --(6.032,3.375)--(6.042,3.370)--(6.052,3.364)--(6.063,3.359)--(6.073,3.353)%
    --(6.083,3.347)--(6.094,3.342)--(6.104,3.336)--(6.114,3.331)--(6.124,3.326)%
    --(6.135,3.320)--(6.145,3.316)--(6.155,3.311)--(6.165,3.306)--(6.176,3.302)%
    --(6.186,3.298)--(6.196,3.294)--(6.206,3.291)--(6.217,3.288)--(6.227,3.286)%
    --(6.237,3.284)--(6.248,3.282)--(6.258,3.280)--(6.268,3.279)--(6.278,3.278)%
    --(6.289,3.278)--(6.299,3.278)--(6.309,3.278)--(6.319,3.279)--(6.330,3.280)%
    --(6.340,3.281)--(6.350,3.283)--(6.361,3.285)--(6.371,3.287)--(6.381,3.289)%
    --(6.391,3.292)--(6.402,3.295)--(6.412,3.299)--(6.422,3.303)--(6.432,3.307)%
    --(6.443,3.311)--(6.453,3.316)--(6.463,3.321)--(6.473,3.326)--(6.484,3.331)%
    --(6.494,3.337)--(6.504,3.343)--(6.515,3.349)--(6.525,3.356)--(6.535,3.363)%
    --(6.545,3.370)--(6.556,3.377)--(6.566,3.385)--(6.576,3.393)--(6.586,3.401)%
    --(6.597,3.409)--(6.607,3.417)--(6.617,3.426)--(6.628,3.435)--(6.638,3.444)%
    --(6.648,3.454)--(6.658,3.463)--(6.669,3.473)--(6.679,3.483)--(6.689,3.493)%
    --(6.699,3.502)--(6.710,3.513)--(6.720,3.523)--(6.730,3.533)--(6.740,3.543)%
    --(6.751,3.553)--(6.761,3.563)--(6.771,3.573)--(6.782,3.583)--(6.792,3.592)%
    --(6.802,3.602)--(6.812,3.612)--(6.823,3.621)--(6.833,3.630)--(6.843,3.639)%
    --(6.853,3.648)--(6.864,3.656)--(6.874,3.665)--(6.884,3.673)--(6.895,3.680)%
    --(6.905,3.688)--(6.915,3.695)--(6.925,3.702)--(6.936,3.708)--(6.946,3.714)%
    --(6.956,3.719)--(6.966,3.724)--(6.977,3.729)--(6.987,3.733)--(6.997,3.737)%
    --(7.007,3.740)--(7.018,3.743)--(7.028,3.745)--(7.038,3.746)--(7.049,3.747)%
    --(7.059,3.748)--(7.069,3.748)--(7.079,3.747)--(7.090,3.746)--(7.100,3.745)%
    --(7.110,3.743)--(7.120,3.741)--(7.131,3.739)--(7.141,3.736)--(7.151,3.734)%
    --(7.162,3.730)--(7.172,3.727)--(7.182,3.724)--(7.192,3.720)--(7.203,3.716)%
    --(7.213,3.712)--(7.223,3.708)--(7.233,3.704)--(7.244,3.699)--(7.254,3.695)%
    --(7.264,3.691)--(7.274,3.687)--(7.285,3.683)--(7.295,3.678)--(7.305,3.674)%
    --(7.316,3.671)--(7.326,3.667)--(7.336,3.663)--(7.346,3.660)--(7.357,3.657)%
    --(7.367,3.654)--(7.377,3.652)--(7.387,3.649)--(7.398,3.648)--(7.408,3.646)%
    --(7.418,3.645)--(7.429,3.644)--(7.439,3.644)--(7.449,3.644)--(7.459,3.644)%
    --(7.470,3.645)--(7.480,3.647)--(7.490,3.648)--(7.500,3.650)--(7.511,3.652)%
    --(7.521,3.655)--(7.531,3.658)--(7.541,3.661)--(7.552,3.664)--(7.562,3.668)%
    --(7.572,3.672)--(7.583,3.676)--(7.593,3.680)--(7.603,3.684)--(7.613,3.688)%
    --(7.624,3.693)--(7.634,3.697)--(7.644,3.702)--(7.654,3.707)--(7.665,3.711)%
    --(7.675,3.716)--(7.685,3.721)--(7.696,3.725)--(7.706,3.730)--(7.716,3.735)%
    --(7.726,3.739)--(7.737,3.743)--(7.747,3.748)--(7.757,3.752)--(7.767,3.755)%
    --(7.778,3.759)--(7.788,3.763)--(7.798,3.766)--(7.808,3.769)--(7.819,3.772)%
    --(7.829,3.774)--(7.839,3.776)--(7.850,3.778)--(7.860,3.780)--(7.870,3.781)%
    --(7.880,3.782)--(7.891,3.782)--(7.901,3.782)--(7.911,3.782)--(7.921,3.782)%
    --(7.932,3.782)--(7.942,3.781)--(7.952,3.780)--(7.963,3.779)--(7.973,3.777)%
    --(7.983,3.775)--(7.993,3.773)--(8.004,3.771)--(8.014,3.769)--(8.024,3.767)%
    --(8.034,3.764)--(8.045,3.761)--(8.055,3.758)--(8.065,3.755)--(8.075,3.752)%
    --(8.086,3.749)--(8.096,3.745)--(8.106,3.742)--(8.117,3.738)--(8.127,3.734)%
    --(8.137,3.730)--(8.147,3.726)--(8.158,3.722)--(8.168,3.718)--(8.178,3.714)%
    --(8.188,3.710)--(8.199,3.706)--(8.209,3.702)--(8.219,3.698)--(8.230,3.694)%
    --(8.240,3.690)--(8.250,3.686)--(8.260,3.682)--(8.271,3.678)--(8.281,3.674)%
    --(8.291,3.670)--(8.301,3.666)--(8.312,3.663)--(8.322,3.659)--(8.332,3.655)%
    --(8.342,3.651)--(8.353,3.648)--(8.363,3.644)--(8.373,3.641)--(8.384,3.637)%
    --(8.394,3.634)--(8.404,3.630)--(8.414,3.627)--(8.425,3.624)--(8.435,3.620)%
    --(8.445,3.617)--(8.455,3.614)--(8.466,3.611)--(8.476,3.608)--(8.486,3.604)%
    --(8.497,3.601)--(8.507,3.598)--(8.517,3.596)--(8.527,3.593)--(8.538,3.590)%
    --(8.548,3.587)--(8.558,3.584)--(8.568,3.582)--(8.579,3.579)--(8.589,3.576)%
    --(8.599,3.574)--(8.609,3.571)--(8.620,3.569)--(8.630,3.566)--(8.640,3.564)%
    --(8.651,3.562)--(8.661,3.560)--(8.671,3.557)--(8.681,3.555)--(8.692,3.553)%
    --(8.702,3.551)--(8.712,3.549)--(8.722,3.547)--(8.733,3.545)--(8.743,3.542)%
    --(8.753,3.540)--(8.764,3.538)--(8.774,3.536)--(8.784,3.533)--(8.794,3.531)%
    --(8.805,3.528)--(8.815,3.525)--(8.825,3.523)--(8.835,3.520)--(8.846,3.516)%
    --(8.856,3.513)--(8.866,3.510)--(8.876,3.506)--(8.887,3.502)--(8.897,3.498)%
    --(8.907,3.494)--(8.918,3.489)--(8.928,3.484)--(8.938,3.479)--(8.948,3.474)%
    --(8.959,3.468)--(8.969,3.462)--(8.979,3.456)--(8.989,3.450)--(9.000,3.443)%
    --(9.010,3.435)--(9.020,3.428)--(9.031,3.420)--(9.041,3.412)--(9.051,3.403)%
    --(9.061,3.394)--(9.072,3.384)--(9.082,3.374)--(9.092,3.364)--(9.102,3.353)%
    --(9.113,3.342)--(9.123,3.330)--(9.133,3.318)--(9.143,3.306)--(9.154,3.293)%
    --(9.164,3.281)--(9.174,3.268)--(9.185,3.255)--(9.195,3.241)--(9.205,3.228)%
    --(9.215,3.214)--(9.226,3.200)--(9.236,3.186)--(9.246,3.172)--(9.256,3.158)%
    --(9.267,3.144)--(9.277,3.130)--(9.287,3.116)--(9.298,3.102)--(9.308,3.088)%
    --(9.318,3.074)--(9.328,3.060)--(9.339,3.046)--(9.349,3.033)--(9.359,3.019)%
    --(9.369,3.006)--(9.380,2.993)--(9.390,2.981)--(9.400,2.968)--(9.410,2.956)%
    --(9.421,2.944)--(9.431,2.933)--(9.441,2.921)--(9.452,2.911)--(9.462,2.900)%
    --(9.472,2.890)--(9.482,2.881)--(9.493,2.872)--(9.503,2.863)--(9.513,2.855)%
    --(9.523,2.847)--(9.534,2.840)--(9.544,2.833)--(9.554,2.826)--(9.565,2.820)%
    --(9.575,2.814)--(9.585,2.808)--(9.595,2.803)--(9.606,2.798)--(9.616,2.793)%
    --(9.626,2.789)--(9.636,2.784)--(9.647,2.780)--(9.657,2.776)--(9.667,2.773)%
    --(9.677,2.769)--(9.688,2.766)--(9.698,2.762)--(9.708,2.759)--(9.719,2.756)%
    --(9.729,2.753)--(9.739,2.750)--(9.749,2.747)--(9.760,2.744)--(9.770,2.741)%
    --(9.780,2.738)--(9.790,2.735)--(9.801,2.732)--(9.811,2.728)--(9.821,2.725)%
    --(9.832,2.722)--(9.842,2.718)--(9.852,2.715)--(9.862,2.711)--(9.873,2.707)%
    --(9.883,2.703)--(9.893,2.698)--(9.903,2.693)--(9.914,2.688)--(9.924,2.683)%
    --(9.934,2.678)--(9.944,2.672)--(9.955,2.667)--(9.965,2.661)--(9.975,2.654)%
    --(9.986,2.648)--(9.996,2.641)--(10.006,2.634)--(10.016,2.627)--(10.027,2.620)%
    --(10.037,2.613)--(10.047,2.605)--(10.057,2.598)--(10.068,2.590)--(10.078,2.582)%
    --(10.088,2.574)--(10.099,2.565)--(10.109,2.557)--(10.119,2.548)--(10.129,2.540)%
    --(10.140,2.531)--(10.150,2.522)--(10.160,2.513)--(10.170,2.504)--(10.181,2.494)%
    --(10.191,2.485)--(10.201,2.476)--(10.211,2.466)--(10.222,2.456)--(10.232,2.447)%
    --(10.242,2.437)--(10.253,2.427)--(10.263,2.417)--(10.273,2.407)--(10.283,2.397)%
    --(10.294,2.387)--(10.304,2.377)--(10.314,2.367)--(10.324,2.357)--(10.335,2.347)%
    --(10.345,2.336)--(10.355,2.326)--(10.366,2.316)--(10.376,2.306)--(10.386,2.296)%
    --(10.396,2.285)--(10.407,2.275)--(10.417,2.265)--(10.427,2.255)--(10.437,2.245)%
    --(10.448,2.235)--(10.458,2.226)--(10.468,2.216)--(10.478,2.206)--(10.489,2.197)%
    --(10.499,2.187)--(10.509,2.178)--(10.520,2.168)--(10.530,2.159)--(10.540,2.150)%
    --(10.550,2.141)--(10.561,2.133)--(10.571,2.124)--(10.581,2.116)--(10.591,2.108)%
    --(10.602,2.099)--(10.612,2.092)--(10.622,2.084)--(10.633,2.076)--(10.643,2.069)%
    --(10.653,2.062)--(10.663,2.055)--(10.674,2.049)--(10.684,2.042)--(10.694,2.036)%
    --(10.704,2.030)--(10.715,2.025)--(10.725,2.019)--(10.735,2.014)--(10.745,2.010)%
    --(10.756,2.005)--(10.766,2.001)--(10.776,1.996)--(10.787,1.992)--(10.797,1.988)%
    --(10.807,1.985)--(10.817,1.981)--(10.828,1.978)--(10.838,1.974)--(10.848,1.971)%
    --(10.858,1.968)--(10.869,1.965)--(10.879,1.962)--(10.889,1.959)--(10.900,1.956)%
    --(10.910,1.953)--(10.920,1.950)--(10.930,1.947)--(10.941,1.944)--(10.951,1.941)%
    --(10.961,1.938)--(10.971,1.934)--(10.982,1.931)--(10.992,1.928)--(11.002,1.924)%
    --(11.012,1.920)--(11.023,1.916)--(11.033,1.912)--(11.043,1.908)--(11.054,1.904)%
    --(11.064,1.899)--(11.074,1.894)--(11.084,1.889)--(11.095,1.884)--(11.105,1.878)%
    --(11.115,1.872)--(11.125,1.866)--(11.136,1.859)--(11.146,1.852)--(11.156,1.845)%
    --(11.167,1.837)--(11.177,1.830)--(11.187,1.822)--(11.197,1.813)--(11.208,1.805)%
    --(11.218,1.796)--(11.228,1.787)--(11.238,1.778)--(11.249,1.768)--(11.259,1.759)%
    --(11.269,1.749)--(11.279,1.739)--(11.290,1.728)--(11.300,1.718)--(11.310,1.707)%
    --(11.321,1.697)--(11.331,1.686)--(11.341,1.675)--(11.351,1.664)--(11.362,1.653)%
    --(11.372,1.641)--(11.382,1.630)--(11.392,1.618)--(11.403,1.607)--(11.413,1.595)%
    --(11.423,1.584)--(11.434,1.572)--(11.444,1.560)--(11.454,1.549)--(11.464,1.537)%
    --(11.475,1.525)--(11.485,1.513)--(11.495,1.502)--(11.505,1.490)--(11.516,1.478)%
    --(11.526,1.467)--(11.536,1.455)--(11.546,1.444)--(11.557,1.432)--(11.567,1.421)%
    --(11.577,1.410)--(11.588,1.398)--(11.598,1.387)--(11.608,1.376)--(11.618,1.365)%
    --(11.629,1.354)--(11.639,1.343)--(11.649,1.332)--(11.659,1.321)--(11.670,1.310)%
    --(11.680,1.299)--(11.690,1.289)--(11.701,1.278)--(11.711,1.267)--(11.721,1.257)%
    --(11.731,1.246)--(11.742,1.236)--(11.752,1.225)--(11.762,1.215)--(11.772,1.204)%
    --(11.783,1.194)--(11.793,1.183)--(11.803,1.173)--(11.813,1.163)--(11.824,1.152)%
    --(11.834,1.142)--(11.844,1.132)--(11.855,1.121)--(11.865,1.111)--(11.875,1.101)%
    --(11.885,1.091)--(11.896,1.080)--(11.906,1.070)--(11.916,1.060)--(11.926,1.050)%
    --(11.937,1.040)--(11.947,1.029)--(11.947,0.985)--(1.688,0.985)--cycle;
\draw[gp path] (1.688,0.989)--(1.698,0.989)--(1.709,0.989)--(1.719,0.989)--(1.729,0.989)%
  --(1.739,0.989)--(1.750,0.990)--(1.760,0.990)--(1.770,0.990)--(1.780,0.990)--(1.791,0.990)%
  --(1.801,0.990)--(1.811,0.991)--(1.822,0.991)--(1.832,0.991)--(1.842,0.991)--(1.852,0.991)%
  --(1.863,0.992)--(1.873,0.992)--(1.883,0.992)--(1.893,0.992)--(1.904,0.993)--(1.914,0.993)%
  --(1.924,0.993)--(1.934,0.993)--(1.945,0.994)--(1.955,0.994)--(1.965,0.994)--(1.976,0.995)%
  --(1.986,0.995)--(1.996,0.995)--(2.006,0.996)--(2.017,0.996)--(2.027,0.997)--(2.037,0.997)%
  --(2.047,0.997)--(2.058,0.998)--(2.068,0.998)--(2.078,0.999)--(2.089,0.999)--(2.099,1.000)%
  --(2.109,1.000)--(2.119,1.001)--(2.130,1.001)--(2.140,1.002)--(2.150,1.003)--(2.160,1.003)%
  --(2.171,1.004)--(2.181,1.005)--(2.191,1.005)--(2.201,1.006)--(2.212,1.007)--(2.222,1.008)%
  --(2.232,1.009)--(2.243,1.009)--(2.253,1.010)--(2.263,1.011)--(2.273,1.012)--(2.284,1.013)%
  --(2.294,1.014)--(2.304,1.015)--(2.314,1.016)--(2.325,1.017)--(2.335,1.018)--(2.345,1.020)%
  --(2.356,1.021)--(2.366,1.022)--(2.376,1.023)--(2.386,1.025)--(2.397,1.026)--(2.407,1.028)%
  --(2.417,1.029)--(2.427,1.031)--(2.438,1.032)--(2.448,1.034)--(2.458,1.035)--(2.468,1.037)%
  --(2.479,1.039)--(2.489,1.041)--(2.499,1.042)--(2.510,1.044)--(2.520,1.046)--(2.530,1.048)%
  --(2.540,1.050)--(2.551,1.052)--(2.561,1.054)--(2.571,1.057)--(2.581,1.059)--(2.592,1.061)%
  --(2.602,1.063)--(2.612,1.065)--(2.623,1.068)--(2.633,1.070)--(2.643,1.072)--(2.653,1.074)%
  --(2.664,1.077)--(2.674,1.079)--(2.684,1.081)--(2.694,1.083)--(2.705,1.086)--(2.715,1.088)%
  --(2.725,1.090)--(2.735,1.092)--(2.746,1.094)--(2.756,1.096)--(2.766,1.098)--(2.777,1.100)%
  --(2.787,1.102)--(2.797,1.104)--(2.807,1.106)--(2.818,1.108)--(2.828,1.110)--(2.838,1.111)%
  --(2.848,1.113)--(2.859,1.114)--(2.869,1.116)--(2.879,1.117)--(2.890,1.119)--(2.900,1.120)%
  --(2.910,1.121)--(2.920,1.122)--(2.931,1.123)--(2.941,1.124)--(2.951,1.125)--(2.961,1.125)%
  --(2.972,1.126)--(2.982,1.127)--(2.992,1.127)--(3.002,1.128)--(3.013,1.129)--(3.023,1.129)%
  --(3.033,1.130)--(3.044,1.131)--(3.054,1.131)--(3.064,1.132)--(3.074,1.133)--(3.085,1.134)%
  --(3.095,1.136)--(3.105,1.137)--(3.115,1.139)--(3.126,1.140)--(3.136,1.142)--(3.146,1.144)%
  --(3.157,1.146)--(3.167,1.149)--(3.177,1.151)--(3.187,1.154)--(3.198,1.157)--(3.208,1.161)%
  --(3.218,1.164)--(3.228,1.168)--(3.239,1.172)--(3.249,1.177)--(3.259,1.182)--(3.269,1.187)%
  --(3.280,1.193)--(3.290,1.199)--(3.300,1.205)--(3.311,1.212)--(3.321,1.219)--(3.331,1.227)%
  --(3.341,1.235)--(3.352,1.243)--(3.362,1.252)--(3.372,1.261)--(3.382,1.270)--(3.393,1.280)%
  --(3.403,1.290)--(3.413,1.301)--(3.424,1.311)--(3.434,1.322)--(3.444,1.333)--(3.454,1.344)%
  --(3.465,1.355)--(3.475,1.366)--(3.485,1.378)--(3.495,1.389)--(3.506,1.401)--(3.516,1.412)%
  --(3.526,1.423)--(3.536,1.435)--(3.547,1.446)--(3.557,1.457)--(3.567,1.468)--(3.578,1.479)%
  --(3.588,1.490)--(3.598,1.500)--(3.608,1.510)--(3.619,1.520)--(3.629,1.530)--(3.639,1.540)%
  --(3.649,1.549)--(3.660,1.557)--(3.670,1.566)--(3.680,1.574)--(3.691,1.581)--(3.701,1.588)%
  --(3.711,1.595)--(3.721,1.601)--(3.732,1.606)--(3.742,1.611)--(3.752,1.615)--(3.762,1.619)%
  --(3.773,1.622)--(3.783,1.625)--(3.793,1.627)--(3.803,1.629)--(3.814,1.631)--(3.824,1.632)%
  --(3.834,1.633)--(3.845,1.634)--(3.855,1.634)--(3.865,1.634)--(3.875,1.634)--(3.886,1.634)%
  --(3.896,1.633)--(3.906,1.633)--(3.916,1.632)--(3.927,1.632)--(3.937,1.631)--(3.947,1.631)%
  --(3.958,1.630)--(3.968,1.630)--(3.978,1.630)--(3.988,1.630)--(3.999,1.630)--(4.009,1.630)%
  --(4.019,1.631)--(4.029,1.632)--(4.040,1.633)--(4.050,1.635)--(4.060,1.637)--(4.070,1.639)%
  --(4.081,1.642)--(4.091,1.646)--(4.101,1.650)--(4.112,1.654)--(4.122,1.659)--(4.132,1.665)%
  --(4.142,1.671)--(4.153,1.678)--(4.163,1.686)--(4.173,1.694)--(4.183,1.703)--(4.194,1.713)%
  --(4.204,1.723)--(4.214,1.734)--(4.225,1.745)--(4.235,1.757)--(4.245,1.770)--(4.255,1.782)%
  --(4.266,1.796)--(4.276,1.809)--(4.286,1.823)--(4.296,1.838)--(4.307,1.852)--(4.317,1.867)%
  --(4.327,1.882)--(4.337,1.898)--(4.348,1.914)--(4.358,1.929)--(4.368,1.945)--(4.379,1.962)%
  --(4.389,1.978)--(4.399,1.994)--(4.409,2.010)--(4.420,2.027)--(4.430,2.043)--(4.440,2.059)%
  --(4.450,2.076)--(4.461,2.092)--(4.471,2.108)--(4.481,2.124)--(4.492,2.140)--(4.502,2.155)%
  --(4.512,2.170)--(4.522,2.185)--(4.533,2.200)--(4.543,2.215)--(4.553,2.229)--(4.563,2.242)%
  --(4.574,2.256)--(4.584,2.269)--(4.594,2.281)--(4.604,2.294)--(4.615,2.306)--(4.625,2.317)%
  --(4.635,2.328)--(4.646,2.339)--(4.656,2.350)--(4.666,2.360)--(4.676,2.370)--(4.687,2.380)%
  --(4.697,2.390)--(4.707,2.399)--(4.717,2.408)--(4.728,2.416)--(4.738,2.425)--(4.748,2.433)%
  --(4.759,2.441)--(4.769,2.449)--(4.779,2.457)--(4.789,2.464)--(4.800,2.471)--(4.810,2.478)%
  --(4.820,2.485)--(4.830,2.492)--(4.841,2.498)--(4.851,2.505)--(4.861,2.511)--(4.871,2.517)%
  --(4.882,2.523)--(4.892,2.529)--(4.902,2.535)--(4.913,2.540)--(4.923,2.546)--(4.933,2.552)%
  --(4.943,2.557)--(4.954,2.562)--(4.964,2.568)--(4.974,2.573)--(4.984,2.578)--(4.995,2.584)%
  --(5.005,2.589)--(5.015,2.594)--(5.026,2.600)--(5.036,2.605)--(5.046,2.610)--(5.056,2.616)%
  --(5.067,2.621)--(5.077,2.627)--(5.087,2.632)--(5.097,2.638)--(5.108,2.644)--(5.118,2.650)%
  --(5.128,2.656)--(5.138,2.662)--(5.149,2.669)--(5.159,2.675)--(5.169,2.682)--(5.180,2.689)%
  --(5.190,2.696)--(5.200,2.703)--(5.210,2.711)--(5.221,2.719)--(5.231,2.727)--(5.241,2.735)%
  --(5.251,2.743)--(5.262,2.752)--(5.272,2.761)--(5.282,2.771)--(5.293,2.780)--(5.303,2.790)%
  --(5.313,2.801)--(5.323,2.811)--(5.334,2.822)--(5.344,2.834)--(5.354,2.845)--(5.364,2.858)%
  --(5.375,2.870)--(5.385,2.883)--(5.395,2.896)--(5.405,2.910)--(5.416,2.924)--(5.426,2.939)%
  --(5.436,2.953)--(5.447,2.968)--(5.457,2.983)--(5.467,2.999)--(5.477,3.014)--(5.488,3.030)%
  --(5.498,3.046)--(5.508,3.062)--(5.518,3.077)--(5.529,3.093)--(5.539,3.109)--(5.549,3.125)%
  --(5.560,3.141)--(5.570,3.156)--(5.580,3.172)--(5.590,3.187)--(5.601,3.202)--(5.611,3.217)%
  --(5.621,3.231)--(5.631,3.246)--(5.642,3.260)--(5.652,3.273)--(5.662,3.287)--(5.672,3.299)%
  --(5.683,3.312)--(5.693,3.324)--(5.703,3.335)--(5.714,3.346)--(5.724,3.356)--(5.734,3.366)%
  --(5.744,3.375)--(5.755,3.383)--(5.765,3.391)--(5.775,3.398)--(5.785,3.404)--(5.796,3.409)%
  --(5.806,3.414)--(5.816,3.418)--(5.827,3.421)--(5.837,3.423)--(5.847,3.425)--(5.857,3.426)%
  --(5.868,3.427)--(5.878,3.427)--(5.888,3.426)--(5.898,3.425)--(5.909,3.423)--(5.919,3.421)%
  --(5.929,3.418)--(5.939,3.415)--(5.950,3.412)--(5.960,3.408)--(5.970,3.404)--(5.981,3.400)%
  --(5.991,3.395)--(6.001,3.391)--(6.011,3.386)--(6.022,3.380)--(6.032,3.375)--(6.042,3.370)%
  --(6.052,3.364)--(6.063,3.359)--(6.073,3.353)--(6.083,3.347)--(6.094,3.342)--(6.104,3.336)%
  --(6.114,3.331)--(6.124,3.326)--(6.135,3.320)--(6.145,3.316)--(6.155,3.311)--(6.165,3.306)%
  --(6.176,3.302)--(6.186,3.298)--(6.196,3.294)--(6.206,3.291)--(6.217,3.288)--(6.227,3.286)%
  --(6.237,3.284)--(6.248,3.282)--(6.258,3.280)--(6.268,3.279)--(6.278,3.278)--(6.289,3.278)%
  --(6.299,3.278)--(6.309,3.278)--(6.319,3.279)--(6.330,3.280)--(6.340,3.281)--(6.350,3.283)%
  --(6.361,3.285)--(6.371,3.287)--(6.381,3.289)--(6.391,3.292)--(6.402,3.295)--(6.412,3.299)%
  --(6.422,3.303)--(6.432,3.307)--(6.443,3.311)--(6.453,3.316)--(6.463,3.321)--(6.473,3.326)%
  --(6.484,3.331)--(6.494,3.337)--(6.504,3.343)--(6.515,3.349)--(6.525,3.356)--(6.535,3.363)%
  --(6.545,3.370)--(6.556,3.377)--(6.566,3.385)--(6.576,3.393)--(6.586,3.401)--(6.597,3.409)%
  --(6.607,3.417)--(6.617,3.426)--(6.628,3.435)--(6.638,3.444)--(6.648,3.454)--(6.658,3.463)%
  --(6.669,3.473)--(6.679,3.483)--(6.689,3.493)--(6.699,3.502)--(6.710,3.513)--(6.720,3.523)%
  --(6.730,3.533)--(6.740,3.543)--(6.751,3.553)--(6.761,3.563)--(6.771,3.573)--(6.782,3.583)%
  --(6.792,3.592)--(6.802,3.602)--(6.812,3.612)--(6.823,3.621)--(6.833,3.630)--(6.843,3.639)%
  --(6.853,3.648)--(6.864,3.656)--(6.874,3.665)--(6.884,3.673)--(6.895,3.680)--(6.905,3.688)%
  --(6.915,3.695)--(6.925,3.702)--(6.936,3.708)--(6.946,3.714)--(6.956,3.719)--(6.966,3.724)%
  --(6.977,3.729)--(6.987,3.733)--(6.997,3.737)--(7.007,3.740)--(7.018,3.743)--(7.028,3.745)%
  --(7.038,3.746)--(7.049,3.747)--(7.059,3.748)--(7.069,3.748)--(7.079,3.747)--(7.090,3.746)%
  --(7.100,3.745)--(7.110,3.743)--(7.120,3.741)--(7.131,3.739)--(7.141,3.736)--(7.151,3.734)%
  --(7.162,3.730)--(7.172,3.727)--(7.182,3.724)--(7.192,3.720)--(7.203,3.716)--(7.213,3.712)%
  --(7.223,3.708)--(7.233,3.704)--(7.244,3.699)--(7.254,3.695)--(7.264,3.691)--(7.274,3.687)%
  --(7.285,3.683)--(7.295,3.678)--(7.305,3.674)--(7.316,3.671)--(7.326,3.667)--(7.336,3.663)%
  --(7.346,3.660)--(7.357,3.657)--(7.367,3.654)--(7.377,3.652)--(7.387,3.649)--(7.398,3.648)%
  --(7.408,3.646)--(7.418,3.645)--(7.429,3.644)--(7.439,3.644)--(7.449,3.644)--(7.459,3.644)%
  --(7.470,3.645)--(7.480,3.647)--(7.490,3.648)--(7.500,3.650)--(7.511,3.652)--(7.521,3.655)%
  --(7.531,3.658)--(7.541,3.661)--(7.552,3.664)--(7.562,3.668)--(7.572,3.672)--(7.583,3.676)%
  --(7.593,3.680)--(7.603,3.684)--(7.613,3.688)--(7.624,3.693)--(7.634,3.697)--(7.644,3.702)%
  --(7.654,3.707)--(7.665,3.711)--(7.675,3.716)--(7.685,3.721)--(7.696,3.725)--(7.706,3.730)%
  --(7.716,3.735)--(7.726,3.739)--(7.737,3.743)--(7.747,3.748)--(7.757,3.752)--(7.767,3.755)%
  --(7.778,3.759)--(7.788,3.763)--(7.798,3.766)--(7.808,3.769)--(7.819,3.772)--(7.829,3.774)%
  --(7.839,3.776)--(7.850,3.778)--(7.860,3.780)--(7.870,3.781)--(7.880,3.782)--(7.891,3.782)%
  --(7.901,3.782)--(7.911,3.782)--(7.921,3.782)--(7.932,3.782)--(7.942,3.781)--(7.952,3.780)%
  --(7.963,3.779)--(7.973,3.777)--(7.983,3.775)--(7.993,3.773)--(8.004,3.771)--(8.014,3.769)%
  --(8.024,3.767)--(8.034,3.764)--(8.045,3.761)--(8.055,3.758)--(8.065,3.755)--(8.075,3.752)%
  --(8.086,3.749)--(8.096,3.745)--(8.106,3.742)--(8.117,3.738)--(8.127,3.734)--(8.137,3.730)%
  --(8.147,3.726)--(8.158,3.722)--(8.168,3.718)--(8.178,3.714)--(8.188,3.710)--(8.199,3.706)%
  --(8.209,3.702)--(8.219,3.698)--(8.230,3.694)--(8.240,3.690)--(8.250,3.686)--(8.260,3.682)%
  --(8.271,3.678)--(8.281,3.674)--(8.291,3.670)--(8.301,3.666)--(8.312,3.663)--(8.322,3.659)%
  --(8.332,3.655)--(8.342,3.651)--(8.353,3.648)--(8.363,3.644)--(8.373,3.641)--(8.384,3.637)%
  --(8.394,3.634)--(8.404,3.630)--(8.414,3.627)--(8.425,3.624)--(8.435,3.620)--(8.445,3.617)%
  --(8.455,3.614)--(8.466,3.611)--(8.476,3.608)--(8.486,3.604)--(8.497,3.601)--(8.507,3.598)%
  --(8.517,3.596)--(8.527,3.593)--(8.538,3.590)--(8.548,3.587)--(8.558,3.584)--(8.568,3.582)%
  --(8.579,3.579)--(8.589,3.576)--(8.599,3.574)--(8.609,3.571)--(8.620,3.569)--(8.630,3.566)%
  --(8.640,3.564)--(8.651,3.562)--(8.661,3.560)--(8.671,3.557)--(8.681,3.555)--(8.692,3.553)%
  --(8.702,3.551)--(8.712,3.549)--(8.722,3.547)--(8.733,3.545)--(8.743,3.542)--(8.753,3.540)%
  --(8.764,3.538)--(8.774,3.536)--(8.784,3.533)--(8.794,3.531)--(8.805,3.528)--(8.815,3.525)%
  --(8.825,3.523)--(8.835,3.520)--(8.846,3.516)--(8.856,3.513)--(8.866,3.510)--(8.876,3.506)%
  --(8.887,3.502)--(8.897,3.498)--(8.907,3.494)--(8.918,3.489)--(8.928,3.484)--(8.938,3.479)%
  --(8.948,3.474)--(8.959,3.468)--(8.969,3.462)--(8.979,3.456)--(8.989,3.450)--(9.000,3.443)%
  --(9.010,3.435)--(9.020,3.428)--(9.031,3.420)--(9.041,3.412)--(9.051,3.403)--(9.061,3.394)%
  --(9.072,3.384)--(9.082,3.374)--(9.092,3.364)--(9.102,3.353)--(9.113,3.342)--(9.123,3.330)%
  --(9.133,3.318)--(9.143,3.306)--(9.154,3.293)--(9.164,3.281)--(9.174,3.268)--(9.185,3.255)%
  --(9.195,3.241)--(9.205,3.228)--(9.215,3.214)--(9.226,3.200)--(9.236,3.186)--(9.246,3.172)%
  --(9.256,3.158)--(9.267,3.144)--(9.277,3.130)--(9.287,3.116)--(9.298,3.102)--(9.308,3.088)%
  --(9.318,3.074)--(9.328,3.060)--(9.339,3.046)--(9.349,3.033)--(9.359,3.019)--(9.369,3.006)%
  --(9.380,2.993)--(9.390,2.981)--(9.400,2.968)--(9.410,2.956)--(9.421,2.944)--(9.431,2.933)%
  --(9.441,2.921)--(9.452,2.911)--(9.462,2.900)--(9.472,2.890)--(9.482,2.881)--(9.493,2.872)%
  --(9.503,2.863)--(9.513,2.855)--(9.523,2.847)--(9.534,2.840)--(9.544,2.833)--(9.554,2.826)%
  --(9.565,2.820)--(9.575,2.814)--(9.585,2.808)--(9.595,2.803)--(9.606,2.798)--(9.616,2.793)%
  --(9.626,2.789)--(9.636,2.784)--(9.647,2.780)--(9.657,2.776)--(9.667,2.773)--(9.677,2.769)%
  --(9.688,2.766)--(9.698,2.762)--(9.708,2.759)--(9.719,2.756)--(9.729,2.753)--(9.739,2.750)%
  --(9.749,2.747)--(9.760,2.744)--(9.770,2.741)--(9.780,2.738)--(9.790,2.735)--(9.801,2.732)%
  --(9.811,2.728)--(9.821,2.725)--(9.832,2.722)--(9.842,2.718)--(9.852,2.715)--(9.862,2.711)%
  --(9.873,2.707)--(9.883,2.703)--(9.893,2.698)--(9.903,2.693)--(9.914,2.688)--(9.924,2.683)%
  --(9.934,2.678)--(9.944,2.672)--(9.955,2.667)--(9.965,2.661)--(9.975,2.654)--(9.986,2.648)%
  --(9.996,2.641)--(10.006,2.634)--(10.016,2.627)--(10.027,2.620)--(10.037,2.613)--(10.047,2.605)%
  --(10.057,2.598)--(10.068,2.590)--(10.078,2.582)--(10.088,2.574)--(10.099,2.565)--(10.109,2.557)%
  --(10.119,2.548)--(10.129,2.540)--(10.140,2.531)--(10.150,2.522)--(10.160,2.513)--(10.170,2.504)%
  --(10.181,2.494)--(10.191,2.485)--(10.201,2.476)--(10.211,2.466)--(10.222,2.456)--(10.232,2.447)%
  --(10.242,2.437)--(10.253,2.427)--(10.263,2.417)--(10.273,2.407)--(10.283,2.397)--(10.294,2.387)%
  --(10.304,2.377)--(10.314,2.367)--(10.324,2.357)--(10.335,2.347)--(10.345,2.336)--(10.355,2.326)%
  --(10.366,2.316)--(10.376,2.306)--(10.386,2.296)--(10.396,2.285)--(10.407,2.275)--(10.417,2.265)%
  --(10.427,2.255)--(10.437,2.245)--(10.448,2.235)--(10.458,2.226)--(10.468,2.216)--(10.478,2.206)%
  --(10.489,2.197)--(10.499,2.187)--(10.509,2.178)--(10.520,2.168)--(10.530,2.159)--(10.540,2.150)%
  --(10.550,2.141)--(10.561,2.133)--(10.571,2.124)--(10.581,2.116)--(10.591,2.108)--(10.602,2.099)%
  --(10.612,2.092)--(10.622,2.084)--(10.633,2.076)--(10.643,2.069)--(10.653,2.062)--(10.663,2.055)%
  --(10.674,2.049)--(10.684,2.042)--(10.694,2.036)--(10.704,2.030)--(10.715,2.025)--(10.725,2.019)%
  --(10.735,2.014)--(10.745,2.010)--(10.756,2.005)--(10.766,2.001)--(10.776,1.996)--(10.787,1.992)%
  --(10.797,1.988)--(10.807,1.985)--(10.817,1.981)--(10.828,1.978)--(10.838,1.974)--(10.848,1.971)%
  --(10.858,1.968)--(10.869,1.965)--(10.879,1.962)--(10.889,1.959)--(10.900,1.956)--(10.910,1.953)%
  --(10.920,1.950)--(10.930,1.947)--(10.941,1.944)--(10.951,1.941)--(10.961,1.938)--(10.971,1.934)%
  --(10.982,1.931)--(10.992,1.928)--(11.002,1.924)--(11.012,1.920)--(11.023,1.916)--(11.033,1.912)%
  --(11.043,1.908)--(11.054,1.904)--(11.064,1.899)--(11.074,1.894)--(11.084,1.889)--(11.095,1.884)%
  --(11.105,1.878)--(11.115,1.872)--(11.125,1.866)--(11.136,1.859)--(11.146,1.852)--(11.156,1.845)%
  --(11.167,1.837)--(11.177,1.830)--(11.187,1.822)--(11.197,1.813)--(11.208,1.805)--(11.218,1.796)%
  --(11.228,1.787)--(11.238,1.778)--(11.249,1.768)--(11.259,1.759)--(11.269,1.749)--(11.279,1.739)%
  --(11.290,1.728)--(11.300,1.718)--(11.310,1.707)--(11.321,1.697)--(11.331,1.686)--(11.341,1.675)%
  --(11.351,1.664)--(11.362,1.653)--(11.372,1.641)--(11.382,1.630)--(11.392,1.618)--(11.403,1.607)%
  --(11.413,1.595)--(11.423,1.584)--(11.434,1.572)--(11.444,1.560)--(11.454,1.549)--(11.464,1.537)%
  --(11.475,1.525)--(11.485,1.513)--(11.495,1.502)--(11.505,1.490)--(11.516,1.478)--(11.526,1.467)%
  --(11.536,1.455)--(11.546,1.444)--(11.557,1.432)--(11.567,1.421)--(11.577,1.410)--(11.588,1.398)%
  --(11.598,1.387)--(11.608,1.376)--(11.618,1.365)--(11.629,1.354)--(11.639,1.343)--(11.649,1.332)%
  --(11.659,1.321)--(11.670,1.310)--(11.680,1.299)--(11.690,1.289)--(11.701,1.278)--(11.711,1.267)%
  --(11.721,1.257)--(11.731,1.246)--(11.742,1.236)--(11.752,1.225)--(11.762,1.215)--(11.772,1.204)%
  --(11.783,1.194)--(11.793,1.183)--(11.803,1.173)--(11.813,1.163)--(11.824,1.152)--(11.834,1.142)%
  --(11.844,1.132)--(11.855,1.121)--(11.865,1.111)--(11.875,1.101)--(11.885,1.091)--(11.896,1.080)%
  --(11.906,1.070)--(11.916,1.060)--(11.926,1.050)--(11.937,1.040)--(11.947,1.029);
\gpcolor{color=gp lt color border}
\node[gp node right] at (4.448,7.739) {equality-based};
\gpfill{color=gp lt color 2,opacity=0.10} (4.632,7.662)--(5.548,7.662)--(5.548,7.816)--(4.632,7.816)--cycle;
\gpcolor{color=gp lt color 2}
\draw[gp path] (4.632,7.662)--(5.548,7.662)--(5.548,7.816)--(4.632,7.816)--cycle;
\gpfill{color=gp lt color 2,opacity=0.10} (1.688,0.989)--(1.688,0.989)--(1.698,0.989)--(1.709,0.989)%
    --(1.719,0.990)--(1.729,0.990)--(1.739,0.990)--(1.750,0.990)--(1.760,0.991)%
    --(1.770,0.991)--(1.780,0.991)--(1.791,0.991)--(1.801,0.992)--(1.811,0.992)%
    --(1.822,0.992)--(1.832,0.992)--(1.842,0.993)--(1.852,0.993)--(1.863,0.993)%
    --(1.873,0.994)--(1.883,0.994)--(1.893,0.994)--(1.904,0.994)--(1.914,0.995)%
    --(1.924,0.995)--(1.934,0.995)--(1.945,0.995)--(1.955,0.996)--(1.965,0.996)%
    --(1.976,0.996)--(1.986,0.997)--(1.996,0.997)--(2.006,0.997)--(2.017,0.997)%
    --(2.027,0.998)--(2.037,0.998)--(2.047,0.998)--(2.058,0.999)--(2.068,0.999)%
    --(2.078,0.999)--(2.089,1.000)--(2.099,1.000)--(2.109,1.000)--(2.119,1.000)%
    --(2.130,1.001)--(2.140,1.001)--(2.150,1.001)--(2.160,1.002)--(2.171,1.002)%
    --(2.181,1.002)--(2.191,1.003)--(2.201,1.003)--(2.212,1.004)--(2.222,1.004)%
    --(2.232,1.004)--(2.243,1.005)--(2.253,1.005)--(2.263,1.006)--(2.273,1.006)%
    --(2.284,1.007)--(2.294,1.007)--(2.304,1.008)--(2.314,1.009)--(2.325,1.009)%
    --(2.335,1.010)--(2.345,1.011)--(2.356,1.012)--(2.366,1.013)--(2.376,1.013)%
    --(2.386,1.014)--(2.397,1.015)--(2.407,1.016)--(2.417,1.017)--(2.427,1.019)%
    --(2.438,1.020)--(2.448,1.021)--(2.458,1.022)--(2.468,1.024)--(2.479,1.025)%
    --(2.489,1.026)--(2.499,1.028)--(2.510,1.030)--(2.520,1.031)--(2.530,1.033)%
    --(2.540,1.035)--(2.551,1.036)--(2.561,1.038)--(2.571,1.040)--(2.581,1.042)%
    --(2.592,1.044)--(2.602,1.046)--(2.612,1.048)--(2.623,1.050)--(2.633,1.052)%
    --(2.643,1.054)--(2.653,1.056)--(2.664,1.058)--(2.674,1.060)--(2.684,1.062)%
    --(2.694,1.064)--(2.705,1.066)--(2.715,1.068)--(2.725,1.070)--(2.735,1.072)%
    --(2.746,1.074)--(2.756,1.076)--(2.766,1.078)--(2.777,1.079)--(2.787,1.081)%
    --(2.797,1.083)--(2.807,1.084)--(2.818,1.086)--(2.828,1.087)--(2.838,1.089)%
    --(2.848,1.090)--(2.859,1.091)--(2.869,1.092)--(2.879,1.093)--(2.890,1.094)%
    --(2.900,1.095)--(2.910,1.095)--(2.920,1.096)--(2.931,1.096)--(2.941,1.097)%
    --(2.951,1.097)--(2.961,1.097)--(2.972,1.097)--(2.982,1.097)--(2.992,1.097)%
    --(3.002,1.097)--(3.013,1.097)--(3.023,1.097)--(3.033,1.097)--(3.044,1.097)%
    --(3.054,1.097)--(3.064,1.097)--(3.074,1.097)--(3.085,1.097)--(3.095,1.097)%
    --(3.105,1.097)--(3.115,1.098)--(3.126,1.098)--(3.136,1.099)--(3.146,1.099)%
    --(3.157,1.100)--(3.167,1.101)--(3.177,1.102)--(3.187,1.103)--(3.198,1.105)%
    --(3.208,1.106)--(3.218,1.108)--(3.228,1.110)--(3.239,1.113)--(3.249,1.115)%
    --(3.259,1.118)--(3.269,1.121)--(3.280,1.124)--(3.290,1.128)--(3.300,1.131)%
    --(3.311,1.136)--(3.321,1.140)--(3.331,1.145)--(3.341,1.150)--(3.352,1.155)%
    --(3.362,1.161)--(3.372,1.167)--(3.382,1.173)--(3.393,1.180)--(3.403,1.187)%
    --(3.413,1.194)--(3.424,1.201)--(3.434,1.208)--(3.444,1.216)--(3.454,1.223)%
    --(3.465,1.231)--(3.475,1.239)--(3.485,1.247)--(3.495,1.255)--(3.506,1.263)%
    --(3.516,1.272)--(3.526,1.280)--(3.536,1.288)--(3.547,1.296)--(3.557,1.305)%
    --(3.567,1.313)--(3.578,1.321)--(3.588,1.329)--(3.598,1.337)--(3.608,1.345)%
    --(3.619,1.353)--(3.629,1.360)--(3.639,1.368)--(3.649,1.375)--(3.660,1.382)%
    --(3.670,1.389)--(3.680,1.395)--(3.691,1.402)--(3.701,1.408)--(3.711,1.414)%
    --(3.721,1.420)--(3.732,1.425)--(3.742,1.430)--(3.752,1.434)--(3.762,1.439)%
    --(3.773,1.443)--(3.783,1.446)--(3.793,1.450)--(3.803,1.453)--(3.814,1.456)%
    --(3.824,1.459)--(3.834,1.461)--(3.845,1.463)--(3.855,1.466)--(3.865,1.468)%
    --(3.875,1.469)--(3.886,1.471)--(3.896,1.473)--(3.906,1.474)--(3.916,1.476)%
    --(3.927,1.477)--(3.937,1.478)--(3.947,1.480)--(3.958,1.481)--(3.968,1.482)%
    --(3.978,1.484)--(3.988,1.485)--(3.999,1.487)--(4.009,1.488)--(4.019,1.490)%
    --(4.029,1.491)--(4.040,1.493)--(4.050,1.495)--(4.060,1.497)--(4.070,1.500)%
    --(4.081,1.502)--(4.091,1.505)--(4.101,1.508)--(4.112,1.511)--(4.122,1.514)%
    --(4.132,1.518)--(4.142,1.522)--(4.153,1.526)--(4.163,1.530)--(4.173,1.535)%
    --(4.183,1.540)--(4.194,1.546)--(4.204,1.552)--(4.214,1.557)--(4.225,1.564)%
    --(4.235,1.570)--(4.245,1.577)--(4.255,1.583)--(4.266,1.590)--(4.276,1.598)%
    --(4.286,1.605)--(4.296,1.613)--(4.307,1.620)--(4.317,1.628)--(4.327,1.636)%
    --(4.337,1.645)--(4.348,1.653)--(4.358,1.661)--(4.368,1.670)--(4.379,1.678)%
    --(4.389,1.687)--(4.399,1.695)--(4.409,1.704)--(4.420,1.713)--(4.430,1.722)%
    --(4.440,1.731)--(4.450,1.739)--(4.461,1.748)--(4.471,1.757)--(4.481,1.766)%
    --(4.492,1.775)--(4.502,1.783)--(4.512,1.792)--(4.522,1.801)--(4.533,1.809)%
    --(4.543,1.818)--(4.553,1.826)--(4.563,1.834)--(4.574,1.842)--(4.584,1.850)%
    --(4.594,1.858)--(4.604,1.866)--(4.615,1.874)--(4.625,1.881)--(4.635,1.889)%
    --(4.646,1.897)--(4.656,1.904)--(4.666,1.912)--(4.676,1.920)--(4.687,1.927)%
    --(4.697,1.935)--(4.707,1.943)--(4.717,1.950)--(4.728,1.958)--(4.738,1.966)%
    --(4.748,1.974)--(4.759,1.982)--(4.769,1.991)--(4.779,1.999)--(4.789,2.007)%
    --(4.800,2.016)--(4.810,2.025)--(4.820,2.034)--(4.830,2.043)--(4.841,2.053)%
    --(4.851,2.062)--(4.861,2.072)--(4.871,2.082)--(4.882,2.092)--(4.892,2.103)%
    --(4.902,2.114)--(4.913,2.125)--(4.923,2.136)--(4.933,2.148)--(4.943,2.160)%
    --(4.954,2.173)--(4.964,2.185)--(4.974,2.198)--(4.984,2.212)--(4.995,2.226)%
    --(5.005,2.240)--(5.015,2.254)--(5.026,2.269)--(5.036,2.284)--(5.046,2.299)%
    --(5.056,2.314)--(5.067,2.329)--(5.077,2.345)--(5.087,2.360)--(5.097,2.376)%
    --(5.108,2.391)--(5.118,2.407)--(5.128,2.422)--(5.138,2.438)--(5.149,2.453)%
    --(5.159,2.468)--(5.169,2.483)--(5.180,2.498)--(5.190,2.513)--(5.200,2.527)%
    --(5.210,2.542)--(5.221,2.555)--(5.231,2.569)--(5.241,2.582)--(5.251,2.595)%
    --(5.262,2.607)--(5.272,2.619)--(5.282,2.631)--(5.293,2.642)--(5.303,2.652)%
    --(5.313,2.662)--(5.323,2.672)--(5.334,2.680)--(5.344,2.688)--(5.354,2.696)%
    --(5.364,2.703)--(5.375,2.709)--(5.385,2.714)--(5.395,2.718)--(5.405,2.722)%
    --(5.416,2.725)--(5.426,2.728)--(5.436,2.730)--(5.447,2.731)--(5.457,2.732)%
    --(5.467,2.732)--(5.477,2.732)--(5.488,2.732)--(5.498,2.731)--(5.508,2.731)%
    --(5.518,2.729)--(5.529,2.728)--(5.539,2.727)--(5.549,2.725)--(5.560,2.724)%
    --(5.570,2.722)--(5.580,2.720)--(5.590,2.719)--(5.601,2.718)--(5.611,2.716)%
    --(5.621,2.716)--(5.631,2.715)--(5.642,2.714)--(5.652,2.714)--(5.662,2.715)%
    --(5.672,2.716)--(5.683,2.717)--(5.693,2.719)--(5.703,2.721)--(5.714,2.724)%
    --(5.724,2.728)--(5.734,2.732)--(5.744,2.737)--(5.755,2.743)--(5.765,2.750)%
    --(5.775,2.757)--(5.785,2.766)--(5.796,2.775)--(5.806,2.785)--(5.816,2.797)%
    --(5.827,2.809)--(5.837,2.822)--(5.847,2.836)--(5.857,2.850)--(5.868,2.866)%
    --(5.878,2.882)--(5.888,2.898)--(5.898,2.915)--(5.909,2.933)--(5.919,2.951)%
    --(5.929,2.969)--(5.939,2.988)--(5.950,3.007)--(5.960,3.026)--(5.970,3.045)%
    --(5.981,3.065)--(5.991,3.085)--(6.001,3.104)--(6.011,3.124)--(6.022,3.144)%
    --(6.032,3.163)--(6.042,3.183)--(6.052,3.202)--(6.063,3.221)--(6.073,3.240)%
    --(6.083,3.259)--(6.094,3.277)--(6.104,3.295)--(6.114,3.312)--(6.124,3.328)%
    --(6.135,3.345)--(6.145,3.360)--(6.155,3.375)--(6.165,3.389)--(6.176,3.403)%
    --(6.186,3.415)--(6.196,3.427)--(6.206,3.438)--(6.217,3.448)--(6.227,3.457)%
    --(6.237,3.465)--(6.248,3.472)--(6.258,3.479)--(6.268,3.485)--(6.278,3.490)%
    --(6.289,3.494)--(6.299,3.498)--(6.309,3.501)--(6.319,3.504)--(6.330,3.506)%
    --(6.340,3.507)--(6.350,3.508)--(6.361,3.509)--(6.371,3.509)--(6.381,3.508)%
    --(6.391,3.508)--(6.402,3.507)--(6.412,3.505)--(6.422,3.504)--(6.432,3.502)%
    --(6.443,3.500)--(6.453,3.498)--(6.463,3.495)--(6.473,3.493)--(6.484,3.491)%
    --(6.494,3.488)--(6.504,3.485)--(6.515,3.483)--(6.525,3.480)--(6.535,3.478)%
    --(6.545,3.476)--(6.556,3.474)--(6.566,3.472)--(6.576,3.470)--(6.586,3.469)%
    --(6.597,3.467)--(6.607,3.467)--(6.617,3.466)--(6.628,3.466)--(6.638,3.466)%
    --(6.648,3.467)--(6.658,3.468)--(6.669,3.469)--(6.679,3.470)--(6.689,3.472)%
    --(6.699,3.474)--(6.710,3.476)--(6.720,3.479)--(6.730,3.482)--(6.740,3.485)%
    --(6.751,3.488)--(6.761,3.491)--(6.771,3.495)--(6.782,3.499)--(6.792,3.502)%
    --(6.802,3.506)--(6.812,3.511)--(6.823,3.515)--(6.833,3.519)--(6.843,3.524)%
    --(6.853,3.528)--(6.864,3.533)--(6.874,3.538)--(6.884,3.542)--(6.895,3.547)%
    --(6.905,3.552)--(6.915,3.556)--(6.925,3.561)--(6.936,3.566)--(6.946,3.571)%
    --(6.956,3.575)--(6.966,3.580)--(6.977,3.584)--(6.987,3.589)--(6.997,3.593)%
    --(7.007,3.597)--(7.018,3.601)--(7.028,3.605)--(7.038,3.609)--(7.049,3.613)%
    --(7.059,3.616)--(7.069,3.620)--(7.079,3.623)--(7.090,3.626)--(7.100,3.629)%
    --(7.110,3.632)--(7.120,3.635)--(7.131,3.637)--(7.141,3.640)--(7.151,3.642)%
    --(7.162,3.645)--(7.172,3.647)--(7.182,3.650)--(7.192,3.652)--(7.203,3.654)%
    --(7.213,3.656)--(7.223,3.658)--(7.233,3.660)--(7.244,3.662)--(7.254,3.663)%
    --(7.264,3.665)--(7.274,3.667)--(7.285,3.669)--(7.295,3.670)--(7.305,3.672)%
    --(7.316,3.674)--(7.326,3.675)--(7.336,3.677)--(7.346,3.678)--(7.357,3.680)%
    --(7.367,3.681)--(7.377,3.683)--(7.387,3.685)--(7.398,3.686)--(7.408,3.688)%
    --(7.418,3.690)--(7.429,3.691)--(7.439,3.693)--(7.449,3.695)--(7.459,3.696)%
    --(7.470,3.698)--(7.480,3.700)--(7.490,3.702)--(7.500,3.704)--(7.511,3.706)%
    --(7.521,3.708)--(7.531,3.710)--(7.541,3.712)--(7.552,3.714)--(7.562,3.716)%
    --(7.572,3.719)--(7.583,3.721)--(7.593,3.723)--(7.603,3.726)--(7.613,3.728)%
    --(7.624,3.731)--(7.634,3.733)--(7.644,3.736)--(7.654,3.739)--(7.665,3.742)%
    --(7.675,3.745)--(7.685,3.747)--(7.696,3.750)--(7.706,3.754)--(7.716,3.757)%
    --(7.726,3.760)--(7.737,3.763)--(7.747,3.767)--(7.757,3.770)--(7.767,3.774)%
    --(7.778,3.777)--(7.788,3.781)--(7.798,3.785)--(7.808,3.789)--(7.819,3.793)%
    --(7.829,3.797)--(7.839,3.801)--(7.850,3.805)--(7.860,3.810)--(7.870,3.814)%
    --(7.880,3.819)--(7.891,3.823)--(7.901,3.828)--(7.911,3.833)--(7.921,3.837)%
    --(7.932,3.842)--(7.942,3.846)--(7.952,3.851)--(7.963,3.856)--(7.973,3.860)%
    --(7.983,3.865)--(7.993,3.869)--(8.004,3.873)--(8.014,3.878)--(8.024,3.882)%
    --(8.034,3.886)--(8.045,3.889)--(8.055,3.893)--(8.065,3.897)--(8.075,3.900)%
    --(8.086,3.903)--(8.096,3.906)--(8.106,3.909)--(8.117,3.911)--(8.127,3.914)%
    --(8.137,3.916)--(8.147,3.917)--(8.158,3.919)--(8.168,3.920)--(8.178,3.921)%
    --(8.188,3.922)--(8.199,3.922)--(8.209,3.922)--(8.219,3.922)--(8.230,3.921)%
    --(8.240,3.920)--(8.250,3.918)--(8.260,3.916)--(8.271,3.914)--(8.281,3.911)%
    --(8.291,3.908)--(8.301,3.905)--(8.312,3.901)--(8.322,3.897)--(8.332,3.893)%
    --(8.342,3.888)--(8.353,3.883)--(8.363,3.878)--(8.373,3.873)--(8.384,3.868)%
    --(8.394,3.862)--(8.404,3.856)--(8.414,3.850)--(8.425,3.844)--(8.435,3.838)%
    --(8.445,3.832)--(8.455,3.826)--(8.466,3.819)--(8.476,3.813)--(8.486,3.806)%
    --(8.497,3.800)--(8.507,3.794)--(8.517,3.787)--(8.527,3.781)--(8.538,3.775)%
    --(8.548,3.769)--(8.558,3.762)--(8.568,3.757)--(8.579,3.751)--(8.589,3.745)%
    --(8.599,3.740)--(8.609,3.734)--(8.620,3.729)--(8.630,3.725)--(8.640,3.720)%
    --(8.651,3.716)--(8.661,3.712)--(8.671,3.708)--(8.681,3.705)--(8.692,3.701)%
    --(8.702,3.699)--(8.712,3.696)--(8.722,3.693)--(8.733,3.691)--(8.743,3.689)%
    --(8.753,3.688)--(8.764,3.686)--(8.774,3.684)--(8.784,3.683)--(8.794,3.682)%
    --(8.805,3.681)--(8.815,3.680)--(8.825,3.679)--(8.835,3.678)--(8.846,3.677)%
    --(8.856,3.676)--(8.866,3.676)--(8.876,3.675)--(8.887,3.674)--(8.897,3.673)%
    --(8.907,3.672)--(8.918,3.671)--(8.928,3.670)--(8.938,3.669)--(8.948,3.668)%
    --(8.959,3.666)--(8.969,3.665)--(8.979,3.663)--(8.989,3.661)--(9.000,3.659)%
    --(9.010,3.657)--(9.020,3.654)--(9.031,3.651)--(9.041,3.648)--(9.051,3.645)%
    --(9.061,3.642)--(9.072,3.638)--(9.082,3.633)--(9.092,3.629)--(9.102,3.624)%
    --(9.113,3.619)--(9.123,3.613)--(9.133,3.608)--(9.143,3.602)--(9.154,3.596)%
    --(9.164,3.589)--(9.174,3.582)--(9.185,3.576)--(9.195,3.568)--(9.205,3.561)%
    --(9.215,3.554)--(9.226,3.546)--(9.236,3.538)--(9.246,3.530)--(9.256,3.522)%
    --(9.267,3.514)--(9.277,3.506)--(9.287,3.497)--(9.298,3.489)--(9.308,3.480)%
    --(9.318,3.472)--(9.328,3.463)--(9.339,3.454)--(9.349,3.446)--(9.359,3.437)%
    --(9.369,3.428)--(9.380,3.420)--(9.390,3.411)--(9.400,3.402)--(9.410,3.393)%
    --(9.421,3.385)--(9.431,3.376)--(9.441,3.368)--(9.452,3.360)--(9.462,3.351)%
    --(9.472,3.343)--(9.482,3.335)--(9.493,3.327)--(9.503,3.320)--(9.513,3.312)%
    --(9.523,3.304)--(9.534,3.297)--(9.544,3.290)--(9.554,3.282)--(9.565,3.275)%
    --(9.575,3.268)--(9.585,3.261)--(9.595,3.255)--(9.606,3.248)--(9.616,3.241)%
    --(9.626,3.234)--(9.636,3.227)--(9.647,3.221)--(9.657,3.214)--(9.667,3.208)%
    --(9.677,3.201)--(9.688,3.194)--(9.698,3.188)--(9.708,3.181)--(9.719,3.174)%
    --(9.729,3.168)--(9.739,3.161)--(9.749,3.154)--(9.760,3.147)--(9.770,3.140)%
    --(9.780,3.133)--(9.790,3.126)--(9.801,3.119)--(9.811,3.112)--(9.821,3.105)%
    --(9.832,3.097)--(9.842,3.090)--(9.852,3.082)--(9.862,3.074)--(9.873,3.066)%
    --(9.883,3.058)--(9.893,3.050)--(9.903,3.042)--(9.914,3.033)--(9.924,3.025)%
    --(9.934,3.016)--(9.944,3.007)--(9.955,2.998)--(9.965,2.988)--(9.975,2.979)%
    --(9.986,2.969)--(9.996,2.960)--(10.006,2.950)--(10.016,2.940)--(10.027,2.930)%
    --(10.037,2.920)--(10.047,2.909)--(10.057,2.899)--(10.068,2.888)--(10.078,2.877)%
    --(10.088,2.866)--(10.099,2.855)--(10.109,2.844)--(10.119,2.833)--(10.129,2.821)%
    --(10.140,2.810)--(10.150,2.798)--(10.160,2.786)--(10.170,2.775)--(10.181,2.763)%
    --(10.191,2.750)--(10.201,2.738)--(10.211,2.726)--(10.222,2.713)--(10.232,2.701)%
    --(10.242,2.688)--(10.253,2.676)--(10.263,2.663)--(10.273,2.650)--(10.283,2.637)%
    --(10.294,2.624)--(10.304,2.611)--(10.314,2.597)--(10.324,2.584)--(10.335,2.570)%
    --(10.345,2.557)--(10.355,2.543)--(10.366,2.530)--(10.376,2.516)--(10.386,2.503)%
    --(10.396,2.489)--(10.407,2.476)--(10.417,2.462)--(10.427,2.449)--(10.437,2.435)%
    --(10.448,2.422)--(10.458,2.409)--(10.468,2.395)--(10.478,2.382)--(10.489,2.369)%
    --(10.499,2.357)--(10.509,2.344)--(10.520,2.331)--(10.530,2.319)--(10.540,2.307)%
    --(10.550,2.295)--(10.561,2.283)--(10.571,2.272)--(10.581,2.260)--(10.591,2.249)%
    --(10.602,2.238)--(10.612,2.228)--(10.622,2.217)--(10.633,2.207)--(10.643,2.198)%
    --(10.653,2.188)--(10.663,2.179)--(10.674,2.171)--(10.684,2.162)--(10.694,2.154)%
    --(10.704,2.147)--(10.715,2.140)--(10.725,2.133)--(10.735,2.126)--(10.745,2.120)%
    --(10.756,2.115)--(10.766,2.109)--(10.776,2.104)--(10.787,2.099)--(10.797,2.095)%
    --(10.807,2.091)--(10.817,2.087)--(10.828,2.083)--(10.838,2.079)--(10.848,2.075)%
    --(10.858,2.072)--(10.869,2.069)--(10.879,2.065)--(10.889,2.062)--(10.900,2.059)%
    --(10.910,2.056)--(10.920,2.053)--(10.930,2.050)--(10.941,2.047)--(10.951,2.044)%
    --(10.961,2.041)--(10.971,2.037)--(10.982,2.034)--(10.992,2.030)--(11.002,2.027)%
    --(11.012,2.023)--(11.023,2.019)--(11.033,2.014)--(11.043,2.010)--(11.054,2.005)%
    --(11.064,2.000)--(11.074,1.994)--(11.084,1.989)--(11.095,1.982)--(11.105,1.976)%
    --(11.115,1.969)--(11.125,1.962)--(11.136,1.954)--(11.146,1.946)--(11.156,1.938)%
    --(11.167,1.929)--(11.177,1.919)--(11.187,1.910)--(11.197,1.900)--(11.208,1.889)%
    --(11.218,1.879)--(11.228,1.868)--(11.238,1.856)--(11.249,1.845)--(11.259,1.833)%
    --(11.269,1.821)--(11.279,1.809)--(11.290,1.797)--(11.300,1.784)--(11.310,1.771)%
    --(11.321,1.758)--(11.331,1.745)--(11.341,1.732)--(11.351,1.718)--(11.362,1.705)%
    --(11.372,1.691)--(11.382,1.678)--(11.392,1.664)--(11.403,1.650)--(11.413,1.636)%
    --(11.423,1.622)--(11.434,1.609)--(11.444,1.595)--(11.454,1.581)--(11.464,1.567)%
    --(11.475,1.554)--(11.485,1.540)--(11.495,1.527)--(11.505,1.513)--(11.516,1.500)%
    --(11.526,1.487)--(11.536,1.474)--(11.546,1.461)--(11.557,1.448)--(11.567,1.436)%
    --(11.577,1.423)--(11.588,1.411)--(11.598,1.399)--(11.608,1.387)--(11.618,1.375)%
    --(11.629,1.363)--(11.639,1.352)--(11.649,1.340)--(11.659,1.329)--(11.670,1.318)%
    --(11.680,1.307)--(11.690,1.296)--(11.701,1.285)--(11.711,1.274)--(11.721,1.264)%
    --(11.731,1.253)--(11.742,1.242)--(11.752,1.232)--(11.762,1.222)--(11.772,1.211)%
    --(11.783,1.201)--(11.793,1.191)--(11.803,1.181)--(11.813,1.171)--(11.824,1.161)%
    --(11.834,1.151)--(11.844,1.141)--(11.855,1.131)--(11.865,1.122)--(11.875,1.112)%
    --(11.885,1.102)--(11.896,1.092)--(11.906,1.083)--(11.916,1.073)--(11.926,1.063)%
    --(11.937,1.054)--(11.947,1.044)--(11.947,0.985)--(1.688,0.985)--cycle;
\draw[gp path] (1.688,0.989)--(1.698,0.989)--(1.709,0.989)--(1.719,0.990)--(1.729,0.990)%
  --(1.739,0.990)--(1.750,0.990)--(1.760,0.991)--(1.770,0.991)--(1.780,0.991)--(1.791,0.991)%
  --(1.801,0.992)--(1.811,0.992)--(1.822,0.992)--(1.832,0.992)--(1.842,0.993)--(1.852,0.993)%
  --(1.863,0.993)--(1.873,0.994)--(1.883,0.994)--(1.893,0.994)--(1.904,0.994)--(1.914,0.995)%
  --(1.924,0.995)--(1.934,0.995)--(1.945,0.995)--(1.955,0.996)--(1.965,0.996)--(1.976,0.996)%
  --(1.986,0.997)--(1.996,0.997)--(2.006,0.997)--(2.017,0.997)--(2.027,0.998)--(2.037,0.998)%
  --(2.047,0.998)--(2.058,0.999)--(2.068,0.999)--(2.078,0.999)--(2.089,1.000)--(2.099,1.000)%
  --(2.109,1.000)--(2.119,1.000)--(2.130,1.001)--(2.140,1.001)--(2.150,1.001)--(2.160,1.002)%
  --(2.171,1.002)--(2.181,1.002)--(2.191,1.003)--(2.201,1.003)--(2.212,1.004)--(2.222,1.004)%
  --(2.232,1.004)--(2.243,1.005)--(2.253,1.005)--(2.263,1.006)--(2.273,1.006)--(2.284,1.007)%
  --(2.294,1.007)--(2.304,1.008)--(2.314,1.009)--(2.325,1.009)--(2.335,1.010)--(2.345,1.011)%
  --(2.356,1.012)--(2.366,1.013)--(2.376,1.013)--(2.386,1.014)--(2.397,1.015)--(2.407,1.016)%
  --(2.417,1.017)--(2.427,1.019)--(2.438,1.020)--(2.448,1.021)--(2.458,1.022)--(2.468,1.024)%
  --(2.479,1.025)--(2.489,1.026)--(2.499,1.028)--(2.510,1.030)--(2.520,1.031)--(2.530,1.033)%
  --(2.540,1.035)--(2.551,1.036)--(2.561,1.038)--(2.571,1.040)--(2.581,1.042)--(2.592,1.044)%
  --(2.602,1.046)--(2.612,1.048)--(2.623,1.050)--(2.633,1.052)--(2.643,1.054)--(2.653,1.056)%
  --(2.664,1.058)--(2.674,1.060)--(2.684,1.062)--(2.694,1.064)--(2.705,1.066)--(2.715,1.068)%
  --(2.725,1.070)--(2.735,1.072)--(2.746,1.074)--(2.756,1.076)--(2.766,1.078)--(2.777,1.079)%
  --(2.787,1.081)--(2.797,1.083)--(2.807,1.084)--(2.818,1.086)--(2.828,1.087)--(2.838,1.089)%
  --(2.848,1.090)--(2.859,1.091)--(2.869,1.092)--(2.879,1.093)--(2.890,1.094)--(2.900,1.095)%
  --(2.910,1.095)--(2.920,1.096)--(2.931,1.096)--(2.941,1.097)--(2.951,1.097)--(2.961,1.097)%
  --(2.972,1.097)--(2.982,1.097)--(2.992,1.097)--(3.002,1.097)--(3.013,1.097)--(3.023,1.097)%
  --(3.033,1.097)--(3.044,1.097)--(3.054,1.097)--(3.064,1.097)--(3.074,1.097)--(3.085,1.097)%
  --(3.095,1.097)--(3.105,1.097)--(3.115,1.098)--(3.126,1.098)--(3.136,1.099)--(3.146,1.099)%
  --(3.157,1.100)--(3.167,1.101)--(3.177,1.102)--(3.187,1.103)--(3.198,1.105)--(3.208,1.106)%
  --(3.218,1.108)--(3.228,1.110)--(3.239,1.113)--(3.249,1.115)--(3.259,1.118)--(3.269,1.121)%
  --(3.280,1.124)--(3.290,1.128)--(3.300,1.131)--(3.311,1.136)--(3.321,1.140)--(3.331,1.145)%
  --(3.341,1.150)--(3.352,1.155)--(3.362,1.161)--(3.372,1.167)--(3.382,1.173)--(3.393,1.180)%
  --(3.403,1.187)--(3.413,1.194)--(3.424,1.201)--(3.434,1.208)--(3.444,1.216)--(3.454,1.223)%
  --(3.465,1.231)--(3.475,1.239)--(3.485,1.247)--(3.495,1.255)--(3.506,1.263)--(3.516,1.272)%
  --(3.526,1.280)--(3.536,1.288)--(3.547,1.296)--(3.557,1.305)--(3.567,1.313)--(3.578,1.321)%
  --(3.588,1.329)--(3.598,1.337)--(3.608,1.345)--(3.619,1.353)--(3.629,1.360)--(3.639,1.368)%
  --(3.649,1.375)--(3.660,1.382)--(3.670,1.389)--(3.680,1.395)--(3.691,1.402)--(3.701,1.408)%
  --(3.711,1.414)--(3.721,1.420)--(3.732,1.425)--(3.742,1.430)--(3.752,1.434)--(3.762,1.439)%
  --(3.773,1.443)--(3.783,1.446)--(3.793,1.450)--(3.803,1.453)--(3.814,1.456)--(3.824,1.459)%
  --(3.834,1.461)--(3.845,1.463)--(3.855,1.466)--(3.865,1.468)--(3.875,1.469)--(3.886,1.471)%
  --(3.896,1.473)--(3.906,1.474)--(3.916,1.476)--(3.927,1.477)--(3.937,1.478)--(3.947,1.480)%
  --(3.958,1.481)--(3.968,1.482)--(3.978,1.484)--(3.988,1.485)--(3.999,1.487)--(4.009,1.488)%
  --(4.019,1.490)--(4.029,1.491)--(4.040,1.493)--(4.050,1.495)--(4.060,1.497)--(4.070,1.500)%
  --(4.081,1.502)--(4.091,1.505)--(4.101,1.508)--(4.112,1.511)--(4.122,1.514)--(4.132,1.518)%
  --(4.142,1.522)--(4.153,1.526)--(4.163,1.530)--(4.173,1.535)--(4.183,1.540)--(4.194,1.546)%
  --(4.204,1.552)--(4.214,1.557)--(4.225,1.564)--(4.235,1.570)--(4.245,1.577)--(4.255,1.583)%
  --(4.266,1.590)--(4.276,1.598)--(4.286,1.605)--(4.296,1.613)--(4.307,1.620)--(4.317,1.628)%
  --(4.327,1.636)--(4.337,1.645)--(4.348,1.653)--(4.358,1.661)--(4.368,1.670)--(4.379,1.678)%
  --(4.389,1.687)--(4.399,1.695)--(4.409,1.704)--(4.420,1.713)--(4.430,1.722)--(4.440,1.731)%
  --(4.450,1.739)--(4.461,1.748)--(4.471,1.757)--(4.481,1.766)--(4.492,1.775)--(4.502,1.783)%
  --(4.512,1.792)--(4.522,1.801)--(4.533,1.809)--(4.543,1.818)--(4.553,1.826)--(4.563,1.834)%
  --(4.574,1.842)--(4.584,1.850)--(4.594,1.858)--(4.604,1.866)--(4.615,1.874)--(4.625,1.881)%
  --(4.635,1.889)--(4.646,1.897)--(4.656,1.904)--(4.666,1.912)--(4.676,1.920)--(4.687,1.927)%
  --(4.697,1.935)--(4.707,1.943)--(4.717,1.950)--(4.728,1.958)--(4.738,1.966)--(4.748,1.974)%
  --(4.759,1.982)--(4.769,1.991)--(4.779,1.999)--(4.789,2.007)--(4.800,2.016)--(4.810,2.025)%
  --(4.820,2.034)--(4.830,2.043)--(4.841,2.053)--(4.851,2.062)--(4.861,2.072)--(4.871,2.082)%
  --(4.882,2.092)--(4.892,2.103)--(4.902,2.114)--(4.913,2.125)--(4.923,2.136)--(4.933,2.148)%
  --(4.943,2.160)--(4.954,2.173)--(4.964,2.185)--(4.974,2.198)--(4.984,2.212)--(4.995,2.226)%
  --(5.005,2.240)--(5.015,2.254)--(5.026,2.269)--(5.036,2.284)--(5.046,2.299)--(5.056,2.314)%
  --(5.067,2.329)--(5.077,2.345)--(5.087,2.360)--(5.097,2.376)--(5.108,2.391)--(5.118,2.407)%
  --(5.128,2.422)--(5.138,2.438)--(5.149,2.453)--(5.159,2.468)--(5.169,2.483)--(5.180,2.498)%
  --(5.190,2.513)--(5.200,2.527)--(5.210,2.542)--(5.221,2.555)--(5.231,2.569)--(5.241,2.582)%
  --(5.251,2.595)--(5.262,2.607)--(5.272,2.619)--(5.282,2.631)--(5.293,2.642)--(5.303,2.652)%
  --(5.313,2.662)--(5.323,2.672)--(5.334,2.680)--(5.344,2.688)--(5.354,2.696)--(5.364,2.703)%
  --(5.375,2.709)--(5.385,2.714)--(5.395,2.718)--(5.405,2.722)--(5.416,2.725)--(5.426,2.728)%
  --(5.436,2.730)--(5.447,2.731)--(5.457,2.732)--(5.467,2.732)--(5.477,2.732)--(5.488,2.732)%
  --(5.498,2.731)--(5.508,2.731)--(5.518,2.729)--(5.529,2.728)--(5.539,2.727)--(5.549,2.725)%
  --(5.560,2.724)--(5.570,2.722)--(5.580,2.720)--(5.590,2.719)--(5.601,2.718)--(5.611,2.716)%
  --(5.621,2.716)--(5.631,2.715)--(5.642,2.714)--(5.652,2.714)--(5.662,2.715)--(5.672,2.716)%
  --(5.683,2.717)--(5.693,2.719)--(5.703,2.721)--(5.714,2.724)--(5.724,2.728)--(5.734,2.732)%
  --(5.744,2.737)--(5.755,2.743)--(5.765,2.750)--(5.775,2.757)--(5.785,2.766)--(5.796,2.775)%
  --(5.806,2.785)--(5.816,2.797)--(5.827,2.809)--(5.837,2.822)--(5.847,2.836)--(5.857,2.850)%
  --(5.868,2.866)--(5.878,2.882)--(5.888,2.898)--(5.898,2.915)--(5.909,2.933)--(5.919,2.951)%
  --(5.929,2.969)--(5.939,2.988)--(5.950,3.007)--(5.960,3.026)--(5.970,3.045)--(5.981,3.065)%
  --(5.991,3.085)--(6.001,3.104)--(6.011,3.124)--(6.022,3.144)--(6.032,3.163)--(6.042,3.183)%
  --(6.052,3.202)--(6.063,3.221)--(6.073,3.240)--(6.083,3.259)--(6.094,3.277)--(6.104,3.295)%
  --(6.114,3.312)--(6.124,3.328)--(6.135,3.345)--(6.145,3.360)--(6.155,3.375)--(6.165,3.389)%
  --(6.176,3.403)--(6.186,3.415)--(6.196,3.427)--(6.206,3.438)--(6.217,3.448)--(6.227,3.457)%
  --(6.237,3.465)--(6.248,3.472)--(6.258,3.479)--(6.268,3.485)--(6.278,3.490)--(6.289,3.494)%
  --(6.299,3.498)--(6.309,3.501)--(6.319,3.504)--(6.330,3.506)--(6.340,3.507)--(6.350,3.508)%
  --(6.361,3.509)--(6.371,3.509)--(6.381,3.508)--(6.391,3.508)--(6.402,3.507)--(6.412,3.505)%
  --(6.422,3.504)--(6.432,3.502)--(6.443,3.500)--(6.453,3.498)--(6.463,3.495)--(6.473,3.493)%
  --(6.484,3.491)--(6.494,3.488)--(6.504,3.485)--(6.515,3.483)--(6.525,3.480)--(6.535,3.478)%
  --(6.545,3.476)--(6.556,3.474)--(6.566,3.472)--(6.576,3.470)--(6.586,3.469)--(6.597,3.467)%
  --(6.607,3.467)--(6.617,3.466)--(6.628,3.466)--(6.638,3.466)--(6.648,3.467)--(6.658,3.468)%
  --(6.669,3.469)--(6.679,3.470)--(6.689,3.472)--(6.699,3.474)--(6.710,3.476)--(6.720,3.479)%
  --(6.730,3.482)--(6.740,3.485)--(6.751,3.488)--(6.761,3.491)--(6.771,3.495)--(6.782,3.499)%
  --(6.792,3.502)--(6.802,3.506)--(6.812,3.511)--(6.823,3.515)--(6.833,3.519)--(6.843,3.524)%
  --(6.853,3.528)--(6.864,3.533)--(6.874,3.538)--(6.884,3.542)--(6.895,3.547)--(6.905,3.552)%
  --(6.915,3.556)--(6.925,3.561)--(6.936,3.566)--(6.946,3.571)--(6.956,3.575)--(6.966,3.580)%
  --(6.977,3.584)--(6.987,3.589)--(6.997,3.593)--(7.007,3.597)--(7.018,3.601)--(7.028,3.605)%
  --(7.038,3.609)--(7.049,3.613)--(7.059,3.616)--(7.069,3.620)--(7.079,3.623)--(7.090,3.626)%
  --(7.100,3.629)--(7.110,3.632)--(7.120,3.635)--(7.131,3.637)--(7.141,3.640)--(7.151,3.642)%
  --(7.162,3.645)--(7.172,3.647)--(7.182,3.650)--(7.192,3.652)--(7.203,3.654)--(7.213,3.656)%
  --(7.223,3.658)--(7.233,3.660)--(7.244,3.662)--(7.254,3.663)--(7.264,3.665)--(7.274,3.667)%
  --(7.285,3.669)--(7.295,3.670)--(7.305,3.672)--(7.316,3.674)--(7.326,3.675)--(7.336,3.677)%
  --(7.346,3.678)--(7.357,3.680)--(7.367,3.681)--(7.377,3.683)--(7.387,3.685)--(7.398,3.686)%
  --(7.408,3.688)--(7.418,3.690)--(7.429,3.691)--(7.439,3.693)--(7.449,3.695)--(7.459,3.696)%
  --(7.470,3.698)--(7.480,3.700)--(7.490,3.702)--(7.500,3.704)--(7.511,3.706)--(7.521,3.708)%
  --(7.531,3.710)--(7.541,3.712)--(7.552,3.714)--(7.562,3.716)--(7.572,3.719)--(7.583,3.721)%
  --(7.593,3.723)--(7.603,3.726)--(7.613,3.728)--(7.624,3.731)--(7.634,3.733)--(7.644,3.736)%
  --(7.654,3.739)--(7.665,3.742)--(7.675,3.745)--(7.685,3.747)--(7.696,3.750)--(7.706,3.754)%
  --(7.716,3.757)--(7.726,3.760)--(7.737,3.763)--(7.747,3.767)--(7.757,3.770)--(7.767,3.774)%
  --(7.778,3.777)--(7.788,3.781)--(7.798,3.785)--(7.808,3.789)--(7.819,3.793)--(7.829,3.797)%
  --(7.839,3.801)--(7.850,3.805)--(7.860,3.810)--(7.870,3.814)--(7.880,3.819)--(7.891,3.823)%
  --(7.901,3.828)--(7.911,3.833)--(7.921,3.837)--(7.932,3.842)--(7.942,3.846)--(7.952,3.851)%
  --(7.963,3.856)--(7.973,3.860)--(7.983,3.865)--(7.993,3.869)--(8.004,3.873)--(8.014,3.878)%
  --(8.024,3.882)--(8.034,3.886)--(8.045,3.889)--(8.055,3.893)--(8.065,3.897)--(8.075,3.900)%
  --(8.086,3.903)--(8.096,3.906)--(8.106,3.909)--(8.117,3.911)--(8.127,3.914)--(8.137,3.916)%
  --(8.147,3.917)--(8.158,3.919)--(8.168,3.920)--(8.178,3.921)--(8.188,3.922)--(8.199,3.922)%
  --(8.209,3.922)--(8.219,3.922)--(8.230,3.921)--(8.240,3.920)--(8.250,3.918)--(8.260,3.916)%
  --(8.271,3.914)--(8.281,3.911)--(8.291,3.908)--(8.301,3.905)--(8.312,3.901)--(8.322,3.897)%
  --(8.332,3.893)--(8.342,3.888)--(8.353,3.883)--(8.363,3.878)--(8.373,3.873)--(8.384,3.868)%
  --(8.394,3.862)--(8.404,3.856)--(8.414,3.850)--(8.425,3.844)--(8.435,3.838)--(8.445,3.832)%
  --(8.455,3.826)--(8.466,3.819)--(8.476,3.813)--(8.486,3.806)--(8.497,3.800)--(8.507,3.794)%
  --(8.517,3.787)--(8.527,3.781)--(8.538,3.775)--(8.548,3.769)--(8.558,3.762)--(8.568,3.757)%
  --(8.579,3.751)--(8.589,3.745)--(8.599,3.740)--(8.609,3.734)--(8.620,3.729)--(8.630,3.725)%
  --(8.640,3.720)--(8.651,3.716)--(8.661,3.712)--(8.671,3.708)--(8.681,3.705)--(8.692,3.701)%
  --(8.702,3.699)--(8.712,3.696)--(8.722,3.693)--(8.733,3.691)--(8.743,3.689)--(8.753,3.688)%
  --(8.764,3.686)--(8.774,3.684)--(8.784,3.683)--(8.794,3.682)--(8.805,3.681)--(8.815,3.680)%
  --(8.825,3.679)--(8.835,3.678)--(8.846,3.677)--(8.856,3.676)--(8.866,3.676)--(8.876,3.675)%
  --(8.887,3.674)--(8.897,3.673)--(8.907,3.672)--(8.918,3.671)--(8.928,3.670)--(8.938,3.669)%
  --(8.948,3.668)--(8.959,3.666)--(8.969,3.665)--(8.979,3.663)--(8.989,3.661)--(9.000,3.659)%
  --(9.010,3.657)--(9.020,3.654)--(9.031,3.651)--(9.041,3.648)--(9.051,3.645)--(9.061,3.642)%
  --(9.072,3.638)--(9.082,3.633)--(9.092,3.629)--(9.102,3.624)--(9.113,3.619)--(9.123,3.613)%
  --(9.133,3.608)--(9.143,3.602)--(9.154,3.596)--(9.164,3.589)--(9.174,3.582)--(9.185,3.576)%
  --(9.195,3.568)--(9.205,3.561)--(9.215,3.554)--(9.226,3.546)--(9.236,3.538)--(9.246,3.530)%
  --(9.256,3.522)--(9.267,3.514)--(9.277,3.506)--(9.287,3.497)--(9.298,3.489)--(9.308,3.480)%
  --(9.318,3.472)--(9.328,3.463)--(9.339,3.454)--(9.349,3.446)--(9.359,3.437)--(9.369,3.428)%
  --(9.380,3.420)--(9.390,3.411)--(9.400,3.402)--(9.410,3.393)--(9.421,3.385)--(9.431,3.376)%
  --(9.441,3.368)--(9.452,3.360)--(9.462,3.351)--(9.472,3.343)--(9.482,3.335)--(9.493,3.327)%
  --(9.503,3.320)--(9.513,3.312)--(9.523,3.304)--(9.534,3.297)--(9.544,3.290)--(9.554,3.282)%
  --(9.565,3.275)--(9.575,3.268)--(9.585,3.261)--(9.595,3.255)--(9.606,3.248)--(9.616,3.241)%
  --(9.626,3.234)--(9.636,3.227)--(9.647,3.221)--(9.657,3.214)--(9.667,3.208)--(9.677,3.201)%
  --(9.688,3.194)--(9.698,3.188)--(9.708,3.181)--(9.719,3.174)--(9.729,3.168)--(9.739,3.161)%
  --(9.749,3.154)--(9.760,3.147)--(9.770,3.140)--(9.780,3.133)--(9.790,3.126)--(9.801,3.119)%
  --(9.811,3.112)--(9.821,3.105)--(9.832,3.097)--(9.842,3.090)--(9.852,3.082)--(9.862,3.074)%
  --(9.873,3.066)--(9.883,3.058)--(9.893,3.050)--(9.903,3.042)--(9.914,3.033)--(9.924,3.025)%
  --(9.934,3.016)--(9.944,3.007)--(9.955,2.998)--(9.965,2.988)--(9.975,2.979)--(9.986,2.969)%
  --(9.996,2.960)--(10.006,2.950)--(10.016,2.940)--(10.027,2.930)--(10.037,2.920)--(10.047,2.909)%
  --(10.057,2.899)--(10.068,2.888)--(10.078,2.877)--(10.088,2.866)--(10.099,2.855)--(10.109,2.844)%
  --(10.119,2.833)--(10.129,2.821)--(10.140,2.810)--(10.150,2.798)--(10.160,2.786)--(10.170,2.775)%
  --(10.181,2.763)--(10.191,2.750)--(10.201,2.738)--(10.211,2.726)--(10.222,2.713)--(10.232,2.701)%
  --(10.242,2.688)--(10.253,2.676)--(10.263,2.663)--(10.273,2.650)--(10.283,2.637)--(10.294,2.624)%
  --(10.304,2.611)--(10.314,2.597)--(10.324,2.584)--(10.335,2.570)--(10.345,2.557)--(10.355,2.543)%
  --(10.366,2.530)--(10.376,2.516)--(10.386,2.503)--(10.396,2.489)--(10.407,2.476)--(10.417,2.462)%
  --(10.427,2.449)--(10.437,2.435)--(10.448,2.422)--(10.458,2.409)--(10.468,2.395)--(10.478,2.382)%
  --(10.489,2.369)--(10.499,2.357)--(10.509,2.344)--(10.520,2.331)--(10.530,2.319)--(10.540,2.307)%
  --(10.550,2.295)--(10.561,2.283)--(10.571,2.272)--(10.581,2.260)--(10.591,2.249)--(10.602,2.238)%
  --(10.612,2.228)--(10.622,2.217)--(10.633,2.207)--(10.643,2.198)--(10.653,2.188)--(10.663,2.179)%
  --(10.674,2.171)--(10.684,2.162)--(10.694,2.154)--(10.704,2.147)--(10.715,2.140)--(10.725,2.133)%
  --(10.735,2.126)--(10.745,2.120)--(10.756,2.115)--(10.766,2.109)--(10.776,2.104)--(10.787,2.099)%
  --(10.797,2.095)--(10.807,2.091)--(10.817,2.087)--(10.828,2.083)--(10.838,2.079)--(10.848,2.075)%
  --(10.858,2.072)--(10.869,2.069)--(10.879,2.065)--(10.889,2.062)--(10.900,2.059)--(10.910,2.056)%
  --(10.920,2.053)--(10.930,2.050)--(10.941,2.047)--(10.951,2.044)--(10.961,2.041)--(10.971,2.037)%
  --(10.982,2.034)--(10.992,2.030)--(11.002,2.027)--(11.012,2.023)--(11.023,2.019)--(11.033,2.014)%
  --(11.043,2.010)--(11.054,2.005)--(11.064,2.000)--(11.074,1.994)--(11.084,1.989)--(11.095,1.982)%
  --(11.105,1.976)--(11.115,1.969)--(11.125,1.962)--(11.136,1.954)--(11.146,1.946)--(11.156,1.938)%
  --(11.167,1.929)--(11.177,1.919)--(11.187,1.910)--(11.197,1.900)--(11.208,1.889)--(11.218,1.879)%
  --(11.228,1.868)--(11.238,1.856)--(11.249,1.845)--(11.259,1.833)--(11.269,1.821)--(11.279,1.809)%
  --(11.290,1.797)--(11.300,1.784)--(11.310,1.771)--(11.321,1.758)--(11.331,1.745)--(11.341,1.732)%
  --(11.351,1.718)--(11.362,1.705)--(11.372,1.691)--(11.382,1.678)--(11.392,1.664)--(11.403,1.650)%
  --(11.413,1.636)--(11.423,1.622)--(11.434,1.609)--(11.444,1.595)--(11.454,1.581)--(11.464,1.567)%
  --(11.475,1.554)--(11.485,1.540)--(11.495,1.527)--(11.505,1.513)--(11.516,1.500)--(11.526,1.487)%
  --(11.536,1.474)--(11.546,1.461)--(11.557,1.448)--(11.567,1.436)--(11.577,1.423)--(11.588,1.411)%
  --(11.598,1.399)--(11.608,1.387)--(11.618,1.375)--(11.629,1.363)--(11.639,1.352)--(11.649,1.340)%
  --(11.659,1.329)--(11.670,1.318)--(11.680,1.307)--(11.690,1.296)--(11.701,1.285)--(11.711,1.274)%
  --(11.721,1.264)--(11.731,1.253)--(11.742,1.242)--(11.752,1.232)--(11.762,1.222)--(11.772,1.211)%
  --(11.783,1.201)--(11.793,1.191)--(11.803,1.181)--(11.813,1.171)--(11.824,1.161)--(11.834,1.151)%
  --(11.844,1.141)--(11.855,1.131)--(11.865,1.122)--(11.875,1.112)--(11.885,1.102)--(11.896,1.092)%
  --(11.906,1.083)--(11.916,1.073)--(11.926,1.063)--(11.937,1.054)--(11.947,1.044);
\gpcolor{color=gp lt color border}
\node[gp node right] at (4.448,7.431) {w/o data flow};
\gpfill{color=gp lt color 6,opacity=0.10} (4.632,7.354)--(5.548,7.354)--(5.548,7.508)--(4.632,7.508)--cycle;
\gpcolor{color=gp lt color 6}
\draw[gp path] (4.632,7.354)--(5.548,7.354)--(5.548,7.508)--(4.632,7.508)--cycle;
\gpfill{color=gp lt color 6,opacity=0.10} (2.509,0.996)--(2.509,0.996)--(2.518,0.996)--(2.528,0.996)%
    --(2.537,0.996)--(2.547,0.996)--(2.556,0.996)--(2.565,0.996)--(2.575,0.996)%
    --(2.584,0.996)--(2.594,0.996)--(2.603,0.996)--(2.613,0.996)--(2.622,0.996)%
    --(2.632,0.996)--(2.641,0.996)--(2.650,0.996)--(2.660,0.996)--(2.669,0.996)%
    --(2.679,0.997)--(2.688,0.997)--(2.698,0.997)--(2.707,0.997)--(2.717,0.997)%
    --(2.726,0.998)--(2.735,0.998)--(2.745,0.998)--(2.754,0.999)--(2.764,0.999)%
    --(2.773,0.999)--(2.783,1.000)--(2.792,1.000)--(2.802,1.001)--(2.811,1.001)%
    --(2.820,1.002)--(2.830,1.003)--(2.839,1.003)--(2.849,1.004)--(2.858,1.005)%
    --(2.868,1.006)--(2.877,1.007)--(2.887,1.007)--(2.896,1.008)--(2.906,1.009)%
    --(2.915,1.010)--(2.924,1.012)--(2.934,1.013)--(2.943,1.014)--(2.953,1.015)%
    --(2.962,1.016)--(2.972,1.018)--(2.981,1.019)--(2.991,1.021)--(3.000,1.022)%
    --(3.009,1.024)--(3.019,1.025)--(3.028,1.027)--(3.038,1.028)--(3.047,1.030)%
    --(3.057,1.031)--(3.066,1.033)--(3.076,1.035)--(3.085,1.036)--(3.094,1.038)%
    --(3.104,1.040)--(3.113,1.042)--(3.123,1.043)--(3.132,1.045)--(3.142,1.047)%
    --(3.151,1.049)--(3.161,1.051)--(3.170,1.052)--(3.180,1.054)--(3.189,1.056)%
    --(3.198,1.058)--(3.208,1.060)--(3.217,1.061)--(3.227,1.063)--(3.236,1.065)%
    --(3.246,1.067)--(3.255,1.068)--(3.265,1.070)--(3.274,1.072)--(3.283,1.073)%
    --(3.293,1.075)--(3.302,1.077)--(3.312,1.078)--(3.321,1.080)--(3.331,1.081)%
    --(3.340,1.083)--(3.350,1.084)--(3.359,1.086)--(3.368,1.087)--(3.378,1.088)%
    --(3.387,1.090)--(3.397,1.091)--(3.406,1.092)--(3.416,1.094)--(3.425,1.095)%
    --(3.435,1.096)--(3.444,1.097)--(3.453,1.098)--(3.463,1.099)--(3.472,1.100)%
    --(3.482,1.101)--(3.491,1.102)--(3.501,1.103)--(3.510,1.104)--(3.520,1.105)%
    --(3.529,1.106)--(3.539,1.107)--(3.548,1.108)--(3.557,1.108)--(3.567,1.109)%
    --(3.576,1.110)--(3.586,1.111)--(3.595,1.111)--(3.605,1.112)--(3.614,1.113)%
    --(3.624,1.113)--(3.633,1.114)--(3.642,1.114)--(3.652,1.115)--(3.661,1.115)%
    --(3.671,1.116)--(3.680,1.116)--(3.690,1.116)--(3.699,1.117)--(3.709,1.117)%
    --(3.718,1.118)--(3.727,1.118)--(3.737,1.118)--(3.746,1.118)--(3.756,1.119)%
    --(3.765,1.119)--(3.775,1.119)--(3.784,1.119)--(3.794,1.119)--(3.803,1.119)%
    --(3.813,1.119)--(3.822,1.119)--(3.831,1.119)--(3.841,1.119)--(3.850,1.119)%
    --(3.860,1.119)--(3.869,1.119)--(3.879,1.119)--(3.888,1.119)--(3.898,1.119)%
    --(3.907,1.119)--(3.916,1.119)--(3.926,1.119)--(3.935,1.118)--(3.945,1.118)%
    --(3.954,1.118)--(3.964,1.118)--(3.973,1.117)--(3.983,1.117)--(3.992,1.117)%
    --(4.001,1.117)--(4.011,1.116)--(4.020,1.116)--(4.030,1.116)--(4.039,1.115)%
    --(4.049,1.115)--(4.058,1.115)--(4.068,1.114)--(4.077,1.114)--(4.086,1.114)%
    --(4.096,1.113)--(4.105,1.113)--(4.115,1.112)--(4.124,1.112)--(4.134,1.111)%
    --(4.143,1.111)--(4.153,1.111)--(4.162,1.110)--(4.172,1.110)--(4.181,1.109)%
    --(4.190,1.109)--(4.200,1.108)--(4.209,1.108)--(4.219,1.107)--(4.228,1.107)%
    --(4.238,1.106)--(4.247,1.106)--(4.257,1.106)--(4.266,1.105)--(4.275,1.105)%
    --(4.285,1.104)--(4.294,1.104)--(4.304,1.103)--(4.313,1.103)--(4.323,1.102)%
    --(4.332,1.102)--(4.342,1.101)--(4.351,1.101)--(4.360,1.101)--(4.370,1.100)%
    --(4.379,1.100)--(4.389,1.100)--(4.398,1.099)--(4.408,1.099)--(4.417,1.099)%
    --(4.427,1.098)--(4.436,1.098)--(4.446,1.098)--(4.455,1.097)--(4.464,1.097)%
    --(4.474,1.097)--(4.483,1.097)--(4.493,1.097)--(4.502,1.096)--(4.512,1.096)%
    --(4.521,1.096)--(4.531,1.096)--(4.540,1.096)--(4.549,1.096)--(4.559,1.096)%
    --(4.568,1.096)--(4.578,1.096)--(4.587,1.096)--(4.597,1.096)--(4.606,1.096)%
    --(4.616,1.096)--(4.625,1.097)--(4.634,1.097)--(4.644,1.097)--(4.653,1.097)%
    --(4.663,1.098)--(4.672,1.098)--(4.682,1.098)--(4.691,1.099)--(4.701,1.099)%
    --(4.710,1.100)--(4.719,1.100)--(4.729,1.101)--(4.738,1.101)--(4.748,1.102)%
    --(4.757,1.102)--(4.767,1.103)--(4.776,1.104)--(4.786,1.104)--(4.795,1.105)%
    --(4.805,1.106)--(4.814,1.107)--(4.823,1.108)--(4.833,1.108)--(4.842,1.109)%
    --(4.852,1.110)--(4.861,1.111)--(4.871,1.112)--(4.880,1.113)--(4.890,1.115)%
    --(4.899,1.116)--(4.908,1.117)--(4.918,1.118)--(4.927,1.119)--(4.937,1.121)%
    --(4.946,1.122)--(4.956,1.123)--(4.965,1.125)--(4.975,1.126)--(4.984,1.128)%
    --(4.993,1.129)--(5.003,1.131)--(5.012,1.132)--(5.022,1.134)--(5.031,1.135)%
    --(5.041,1.137)--(5.050,1.139)--(5.060,1.141)--(5.069,1.142)--(5.079,1.144)%
    --(5.088,1.146)--(5.097,1.148)--(5.107,1.149)--(5.116,1.151)--(5.126,1.153)%
    --(5.135,1.155)--(5.145,1.157)--(5.154,1.159)--(5.164,1.161)--(5.173,1.162)%
    --(5.182,1.164)--(5.192,1.166)--(5.201,1.168)--(5.211,1.170)--(5.220,1.172)%
    --(5.230,1.173)--(5.239,1.175)--(5.249,1.177)--(5.258,1.179)--(5.267,1.180)%
    --(5.277,1.182)--(5.286,1.184)--(5.296,1.186)--(5.305,1.187)--(5.315,1.189)%
    --(5.324,1.191)--(5.334,1.192)--(5.343,1.194)--(5.352,1.195)--(5.362,1.197)%
    --(5.371,1.198)--(5.381,1.199)--(5.390,1.201)--(5.400,1.202)--(5.409,1.203)%
    --(5.419,1.205)--(5.428,1.206)--(5.438,1.207)--(5.447,1.208)--(5.456,1.209)%
    --(5.466,1.210)--(5.475,1.211)--(5.485,1.212)--(5.494,1.214)--(5.504,1.215)%
    --(5.513,1.216)--(5.523,1.217)--(5.532,1.218)--(5.541,1.219)--(5.551,1.220)%
    --(5.560,1.221)--(5.570,1.222)--(5.579,1.223)--(5.589,1.224)--(5.598,1.225)%
    --(5.608,1.227)--(5.617,1.228)--(5.626,1.229)--(5.636,1.230)--(5.645,1.232)%
    --(5.655,1.233)--(5.664,1.234)--(5.674,1.236)--(5.683,1.237)--(5.693,1.239)%
    --(5.702,1.240)--(5.711,1.242)--(5.721,1.244)--(5.730,1.245)--(5.740,1.247)%
    --(5.749,1.249)--(5.759,1.251)--(5.768,1.253)--(5.778,1.255)--(5.787,1.258)%
    --(5.797,1.260)--(5.806,1.262)--(5.815,1.265)--(5.825,1.267)--(5.834,1.270)%
    --(5.844,1.272)--(5.853,1.275)--(5.863,1.278)--(5.872,1.281)--(5.882,1.284)%
    --(5.891,1.286)--(5.900,1.289)--(5.910,1.292)--(5.919,1.295)--(5.929,1.298)%
    --(5.938,1.301)--(5.948,1.304)--(5.957,1.307)--(5.967,1.310)--(5.976,1.313)%
    --(5.985,1.315)--(5.995,1.318)--(6.004,1.321)--(6.014,1.324)--(6.023,1.326)%
    --(6.033,1.329)--(6.042,1.331)--(6.052,1.334)--(6.061,1.336)--(6.071,1.339)%
    --(6.080,1.341)--(6.089,1.343)--(6.099,1.345)--(6.108,1.347)--(6.118,1.349)%
    --(6.127,1.350)--(6.137,1.352)--(6.146,1.353)--(6.156,1.354)--(6.165,1.356)%
    --(6.174,1.357)--(6.184,1.357)--(6.193,1.358)--(6.203,1.359)--(6.212,1.359)%
    --(6.222,1.359)--(6.231,1.359)--(6.241,1.359)--(6.250,1.359)--(6.259,1.358)%
    --(6.269,1.358)--(6.278,1.357)--(6.288,1.356)--(6.297,1.356)--(6.307,1.355)%
    --(6.316,1.354)--(6.326,1.353)--(6.335,1.352)--(6.344,1.351)--(6.354,1.349)%
    --(6.363,1.348)--(6.373,1.347)--(6.382,1.346)--(6.392,1.345)--(6.401,1.343)%
    --(6.411,1.342)--(6.420,1.341)--(6.430,1.340)--(6.439,1.339)--(6.448,1.338)%
    --(6.458,1.337)--(6.467,1.336)--(6.477,1.336)--(6.486,1.335)--(6.496,1.334)%
    --(6.505,1.334)--(6.515,1.334)--(6.524,1.334)--(6.533,1.333)--(6.543,1.334)%
    --(6.552,1.334)--(6.562,1.334)--(6.571,1.335)--(6.581,1.336)--(6.590,1.337)%
    --(6.600,1.338)--(6.609,1.339)--(6.618,1.341)--(6.628,1.343)--(6.637,1.345)%
    --(6.647,1.347)--(6.656,1.350)--(6.666,1.353)--(6.675,1.356)--(6.685,1.359)%
    --(6.694,1.362)--(6.704,1.366)--(6.713,1.369)--(6.722,1.373)--(6.732,1.377)%
    --(6.741,1.381)--(6.751,1.385)--(6.760,1.390)--(6.770,1.395)--(6.779,1.399)%
    --(6.789,1.404)--(6.798,1.409)--(6.807,1.414)--(6.817,1.419)--(6.826,1.425)%
    --(6.836,1.430)--(6.845,1.436)--(6.855,1.441)--(6.864,1.447)--(6.874,1.453)%
    --(6.883,1.459)--(6.892,1.465)--(6.902,1.471)--(6.911,1.477)--(6.921,1.483)%
    --(6.930,1.489)--(6.940,1.495)--(6.949,1.501)--(6.959,1.508)--(6.968,1.514)%
    --(6.977,1.520)--(6.987,1.527)--(6.996,1.533)--(7.006,1.539)--(7.015,1.546)%
    --(7.025,1.552)--(7.034,1.559)--(7.044,1.565)--(7.053,1.571)--(7.063,1.577)%
    --(7.072,1.584)--(7.081,1.590)--(7.091,1.596)--(7.100,1.602)--(7.110,1.608)%
    --(7.119,1.614)--(7.129,1.620)--(7.138,1.626)--(7.148,1.632)--(7.157,1.637)%
    --(7.166,1.643)--(7.176,1.649)--(7.185,1.654)--(7.195,1.659)--(7.204,1.664)%
    --(7.214,1.670)--(7.223,1.675)--(7.233,1.679)--(7.242,1.684)--(7.251,1.689)%
    --(7.261,1.693)--(7.270,1.697)--(7.280,1.702)--(7.289,1.706)--(7.299,1.709)%
    --(7.308,1.713)--(7.318,1.717)--(7.327,1.720)--(7.337,1.723)--(7.346,1.726)%
    --(7.355,1.729)--(7.365,1.731)--(7.374,1.734)--(7.384,1.736)--(7.393,1.738)%
    --(7.403,1.739)--(7.412,1.741)--(7.422,1.742)--(7.431,1.743)--(7.440,1.744)%
    --(7.450,1.744)--(7.459,1.744)--(7.469,1.745)--(7.478,1.745)--(7.488,1.744)%
    --(7.497,1.744)--(7.507,1.743)--(7.516,1.743)--(7.525,1.742)--(7.535,1.742)%
    --(7.544,1.741)--(7.554,1.740)--(7.563,1.739)--(7.573,1.738)--(7.582,1.738)%
    --(7.592,1.737)--(7.601,1.736)--(7.610,1.736)--(7.620,1.735)--(7.629,1.735)%
    --(7.639,1.735)--(7.648,1.734)--(7.658,1.735)--(7.667,1.735)--(7.677,1.735)%
    --(7.686,1.736)--(7.696,1.737)--(7.705,1.738)--(7.714,1.740)--(7.724,1.742)%
    --(7.733,1.744)--(7.743,1.746)--(7.752,1.749)--(7.762,1.752)--(7.771,1.756)%
    --(7.781,1.760)--(7.790,1.764)--(7.799,1.769)--(7.809,1.775)--(7.818,1.781)%
    --(7.828,1.787)--(7.837,1.794)--(7.847,1.801)--(7.856,1.809)--(7.866,1.818)%
    --(7.875,1.827)--(7.884,1.836)--(7.894,1.846)--(7.903,1.856)--(7.913,1.867)%
    --(7.922,1.878)--(7.932,1.889)--(7.941,1.901)--(7.951,1.912)--(7.960,1.925)%
    --(7.970,1.937)--(7.979,1.949)--(7.988,1.962)--(7.998,1.975)--(8.007,1.988)%
    --(8.017,2.001)--(8.026,2.014)--(8.036,2.027)--(8.045,2.040)--(8.055,2.053)%
    --(8.064,2.066)--(8.073,2.078)--(8.083,2.091)--(8.092,2.104)--(8.102,2.116)%
    --(8.111,2.128)--(8.121,2.140)--(8.130,2.152)--(8.140,2.163)--(8.149,2.174)%
    --(8.158,2.185)--(8.168,2.195)--(8.177,2.205)--(8.187,2.215)--(8.196,2.224)%
    --(8.206,2.233)--(8.215,2.241)--(8.225,2.248)--(8.234,2.255)--(8.243,2.262)%
    --(8.253,2.268)--(8.262,2.273)--(8.272,2.277)--(8.281,2.281)--(8.291,2.285)%
    --(8.300,2.288)--(8.310,2.290)--(8.319,2.292)--(8.329,2.294)--(8.338,2.295)%
    --(8.347,2.296)--(8.357,2.297)--(8.366,2.297)--(8.376,2.297)--(8.385,2.297)%
    --(8.395,2.296)--(8.404,2.296)--(8.414,2.295)--(8.423,2.295)--(8.432,2.294)%
    --(8.442,2.293)--(8.451,2.292)--(8.461,2.292)--(8.470,2.291)--(8.480,2.291)%
    --(8.489,2.291)--(8.499,2.290)--(8.508,2.291)--(8.517,2.291)--(8.527,2.292)%
    --(8.536,2.293)--(8.546,2.294)--(8.555,2.296)--(8.565,2.298)--(8.574,2.301)%
    --(8.584,2.304)--(8.593,2.308)--(8.603,2.312)--(8.612,2.317)--(8.621,2.323)%
    --(8.631,2.329)--(8.640,2.336)--(8.650,2.344)--(8.659,2.352)--(8.669,2.361)%
    --(8.678,2.372)--(8.688,2.382)--(8.697,2.394)--(8.706,2.406)--(8.716,2.419)%
    --(8.725,2.433)--(8.735,2.448)--(8.744,2.463)--(8.754,2.478)--(8.763,2.494)%
    --(8.773,2.511)--(8.782,2.529)--(8.791,2.546)--(8.801,2.565)--(8.810,2.584)%
    --(8.820,2.603)--(8.829,2.622)--(8.839,2.643)--(8.848,2.663)--(8.858,2.684)%
    --(8.867,2.705)--(8.876,2.726)--(8.886,2.748)--(8.895,2.770)--(8.905,2.792)%
    --(8.914,2.814)--(8.924,2.836)--(8.933,2.859)--(8.943,2.882)--(8.952,2.904)%
    --(8.962,2.927)--(8.971,2.950)--(8.980,2.973)--(8.990,2.996)--(8.999,3.019)%
    --(9.009,3.041)--(9.018,3.064)--(9.028,3.087)--(9.037,3.109)--(9.047,3.132)%
    --(9.056,3.154)--(9.065,3.176)--(9.075,3.197)--(9.084,3.219)--(9.094,3.240)%
    --(9.103,3.261)--(9.113,3.282)--(9.122,3.302)--(9.132,3.322)--(9.141,3.342)%
    --(9.150,3.362)--(9.160,3.381)--(9.169,3.401)--(9.179,3.419)--(9.188,3.438)%
    --(9.198,3.457)--(9.207,3.475)--(9.217,3.493)--(9.226,3.511)--(9.236,3.528)%
    --(9.245,3.545)--(9.254,3.562)--(9.264,3.579)--(9.273,3.596)--(9.283,3.612)%
    --(9.292,3.628)--(9.302,3.644)--(9.311,3.660)--(9.321,3.675)--(9.330,3.690)%
    --(9.339,3.705)--(9.349,3.720)--(9.358,3.734)--(9.368,3.748)--(9.377,3.762)%
    --(9.387,3.776)--(9.396,3.790)--(9.406,3.803)--(9.415,3.816)--(9.424,3.829)%
    --(9.434,3.842)--(9.443,3.854)--(9.453,3.867)--(9.462,3.879)--(9.472,3.890)%
    --(9.481,3.902)--(9.491,3.913)--(9.500,3.925)--(9.509,3.936)--(9.519,3.946)%
    --(9.528,3.957)--(9.538,3.968)--(9.547,3.978)--(9.557,3.988)--(9.566,3.999)%
    --(9.576,4.009)--(9.585,4.019)--(9.595,4.029)--(9.604,4.039)--(9.613,4.049)%
    --(9.623,4.059)--(9.632,4.070)--(9.642,4.080)--(9.651,4.090)--(9.661,4.100)%
    --(9.670,4.111)--(9.680,4.122)--(9.689,4.132)--(9.698,4.143)--(9.708,4.155)%
    --(9.717,4.166)--(9.727,4.177)--(9.736,4.189)--(9.746,4.201)--(9.755,4.214)%
    --(9.765,4.226)--(9.774,4.239)--(9.783,4.252)--(9.793,4.266)--(9.802,4.280)%
    --(9.812,4.294)--(9.821,4.309)--(9.831,4.324)--(9.840,4.339)--(9.850,4.355)%
    --(9.859,4.372)--(9.868,4.388)--(9.878,4.406)--(9.887,4.424)--(9.897,4.442)%
    --(9.906,4.461)--(9.916,4.481)--(9.925,4.501)--(9.935,4.521)--(9.944,4.542)%
    --(9.954,4.563)--(9.963,4.584)--(9.972,4.606)--(9.982,4.628)--(9.991,4.650)%
    --(10.001,4.673)--(10.010,4.695)--(10.020,4.718)--(10.029,4.741)--(10.039,4.764)%
    --(10.048,4.787)--(10.057,4.810)--(10.067,4.833)--(10.076,4.855)--(10.086,4.878)%
    --(10.095,4.900)--(10.105,4.923)--(10.114,4.945)--(10.124,4.967)--(10.133,4.988)%
    --(10.142,5.009)--(10.152,5.030)--(10.161,5.051)--(10.171,5.071)--(10.180,5.090)%
    --(10.190,5.109)--(10.199,5.128)--(10.209,5.146)--(10.218,5.163)--(10.228,5.180)%
    --(10.237,5.196)--(10.246,5.211)--(10.256,5.226)--(10.265,5.240)--(10.275,5.253)%
    --(10.284,5.265)--(10.294,5.276)--(10.303,5.287)--(10.313,5.296)--(10.322,5.305)%
    --(10.331,5.313)--(10.341,5.319)--(10.350,5.325)--(10.360,5.331)--(10.369,5.335)%
    --(10.379,5.339)--(10.388,5.343)--(10.398,5.345)--(10.407,5.348)--(10.416,5.349)%
    --(10.426,5.351)--(10.435,5.351)--(10.445,5.352)--(10.454,5.352)--(10.464,5.352)%
    --(10.473,5.352)--(10.483,5.351)--(10.492,5.351)--(10.501,5.350)--(10.511,5.350)%
    --(10.520,5.349)--(10.530,5.348)--(10.539,5.348)--(10.549,5.347)--(10.558,5.347)%
    --(10.568,5.347)--(10.577,5.347)--(10.587,5.347)--(10.596,5.348)--(10.605,5.350)%
    --(10.615,5.351)--(10.624,5.353)--(10.634,5.356)--(10.643,5.359)--(10.653,5.363)%
    --(10.662,5.368)--(10.672,5.373)--(10.681,5.379)--(10.690,5.385)--(10.700,5.393)%
    --(10.709,5.401)--(10.719,5.411)--(10.728,5.421)--(10.738,5.432)--(10.747,5.444)%
    --(10.757,5.457)--(10.766,5.471)--(10.775,5.486)--(10.785,5.501)--(10.794,5.517)%
    --(10.804,5.534)--(10.813,5.551)--(10.823,5.570)--(10.832,5.588)--(10.842,5.608)%
    --(10.851,5.628)--(10.861,5.648)--(10.870,5.670)--(10.879,5.691)--(10.889,5.713)%
    --(10.898,5.736)--(10.908,5.759)--(10.917,5.782)--(10.927,5.806)--(10.936,5.830)%
    --(10.946,5.855)--(10.955,5.879)--(10.964,5.904)--(10.974,5.930)--(10.983,5.955)%
    --(10.993,5.981)--(11.002,6.007)--(11.012,6.032)--(11.021,6.058)--(11.031,6.085)%
    --(11.040,6.111)--(11.049,6.137)--(11.059,6.163)--(11.068,6.189)--(11.078,6.215)%
    --(11.087,6.241)--(11.097,6.267)--(11.106,6.293)--(11.116,6.318)--(11.125,6.344)%
    --(11.134,6.369)--(11.144,6.394)--(11.153,6.418)--(11.163,6.443)--(11.172,6.466)%
    --(11.182,6.490)--(11.191,6.512)--(11.201,6.534)--(11.210,6.556)--(11.220,6.576)%
    --(11.229,6.596)--(11.238,6.615)--(11.248,6.634)--(11.257,6.651)--(11.267,6.667)%
    --(11.276,6.682)--(11.286,6.696)--(11.295,6.709)--(11.305,6.721)--(11.314,6.731)%
    --(11.323,6.740)--(11.333,6.748)--(11.342,6.754)--(11.352,6.759)--(11.361,6.762)%
    --(11.371,6.763)--(11.380,6.763)--(11.390,6.761)--(11.399,6.757)--(11.408,6.752)%
    --(11.418,6.744)--(11.427,6.735)--(11.437,6.724)--(11.446,6.710)--(11.456,6.694)%
    --(11.465,6.677)--(11.475,6.657)--(11.484,6.634)--(11.494,6.610)--(11.503,6.582)%
    --(11.512,6.553)--(11.522,6.521)--(11.531,6.486)--(11.541,6.449)--(11.550,6.409)%
    --(11.560,6.366)--(11.569,6.321)--(11.579,6.274)--(11.588,6.224)--(11.597,6.172)%
    --(11.607,6.117)--(11.616,6.061)--(11.626,6.002)--(11.635,5.941)--(11.645,5.878)%
    --(11.654,5.813)--(11.664,5.747)--(11.673,5.678)--(11.682,5.608)--(11.692,5.535)%
    --(11.701,5.462)--(11.711,5.386)--(11.720,5.309)--(11.730,5.231)--(11.739,5.151)%
    --(11.749,5.070)--(11.758,4.988)--(11.767,4.904)--(11.777,4.819)--(11.786,4.733)%
    --(11.796,4.646)--(11.805,4.558)--(11.815,4.469)--(11.824,4.380)--(11.834,4.289)%
    --(11.843,4.198)--(11.853,4.106)--(11.862,4.013)--(11.871,3.920)--(11.881,3.827)%
    --(11.890,3.733)--(11.900,3.638)--(11.909,3.544)--(11.919,3.449)--(11.928,3.354)%
    --(11.938,3.258)--(11.947,3.163)--(11.947,0.985)--(2.509,0.985)--cycle;
\draw[gp path] (2.509,0.996)--(2.518,0.996)--(2.528,0.996)--(2.537,0.996)--(2.547,0.996)%
  --(2.556,0.996)--(2.565,0.996)--(2.575,0.996)--(2.584,0.996)--(2.594,0.996)--(2.603,0.996)%
  --(2.613,0.996)--(2.622,0.996)--(2.632,0.996)--(2.641,0.996)--(2.650,0.996)--(2.660,0.996)%
  --(2.669,0.996)--(2.679,0.997)--(2.688,0.997)--(2.698,0.997)--(2.707,0.997)--(2.717,0.997)%
  --(2.726,0.998)--(2.735,0.998)--(2.745,0.998)--(2.754,0.999)--(2.764,0.999)--(2.773,0.999)%
  --(2.783,1.000)--(2.792,1.000)--(2.802,1.001)--(2.811,1.001)--(2.820,1.002)--(2.830,1.003)%
  --(2.839,1.003)--(2.849,1.004)--(2.858,1.005)--(2.868,1.006)--(2.877,1.007)--(2.887,1.007)%
  --(2.896,1.008)--(2.906,1.009)--(2.915,1.010)--(2.924,1.012)--(2.934,1.013)--(2.943,1.014)%
  --(2.953,1.015)--(2.962,1.016)--(2.972,1.018)--(2.981,1.019)--(2.991,1.021)--(3.000,1.022)%
  --(3.009,1.024)--(3.019,1.025)--(3.028,1.027)--(3.038,1.028)--(3.047,1.030)--(3.057,1.031)%
  --(3.066,1.033)--(3.076,1.035)--(3.085,1.036)--(3.094,1.038)--(3.104,1.040)--(3.113,1.042)%
  --(3.123,1.043)--(3.132,1.045)--(3.142,1.047)--(3.151,1.049)--(3.161,1.051)--(3.170,1.052)%
  --(3.180,1.054)--(3.189,1.056)--(3.198,1.058)--(3.208,1.060)--(3.217,1.061)--(3.227,1.063)%
  --(3.236,1.065)--(3.246,1.067)--(3.255,1.068)--(3.265,1.070)--(3.274,1.072)--(3.283,1.073)%
  --(3.293,1.075)--(3.302,1.077)--(3.312,1.078)--(3.321,1.080)--(3.331,1.081)--(3.340,1.083)%
  --(3.350,1.084)--(3.359,1.086)--(3.368,1.087)--(3.378,1.088)--(3.387,1.090)--(3.397,1.091)%
  --(3.406,1.092)--(3.416,1.094)--(3.425,1.095)--(3.435,1.096)--(3.444,1.097)--(3.453,1.098)%
  --(3.463,1.099)--(3.472,1.100)--(3.482,1.101)--(3.491,1.102)--(3.501,1.103)--(3.510,1.104)%
  --(3.520,1.105)--(3.529,1.106)--(3.539,1.107)--(3.548,1.108)--(3.557,1.108)--(3.567,1.109)%
  --(3.576,1.110)--(3.586,1.111)--(3.595,1.111)--(3.605,1.112)--(3.614,1.113)--(3.624,1.113)%
  --(3.633,1.114)--(3.642,1.114)--(3.652,1.115)--(3.661,1.115)--(3.671,1.116)--(3.680,1.116)%
  --(3.690,1.116)--(3.699,1.117)--(3.709,1.117)--(3.718,1.118)--(3.727,1.118)--(3.737,1.118)%
  --(3.746,1.118)--(3.756,1.119)--(3.765,1.119)--(3.775,1.119)--(3.784,1.119)--(3.794,1.119)%
  --(3.803,1.119)--(3.813,1.119)--(3.822,1.119)--(3.831,1.119)--(3.841,1.119)--(3.850,1.119)%
  --(3.860,1.119)--(3.869,1.119)--(3.879,1.119)--(3.888,1.119)--(3.898,1.119)--(3.907,1.119)%
  --(3.916,1.119)--(3.926,1.119)--(3.935,1.118)--(3.945,1.118)--(3.954,1.118)--(3.964,1.118)%
  --(3.973,1.117)--(3.983,1.117)--(3.992,1.117)--(4.001,1.117)--(4.011,1.116)--(4.020,1.116)%
  --(4.030,1.116)--(4.039,1.115)--(4.049,1.115)--(4.058,1.115)--(4.068,1.114)--(4.077,1.114)%
  --(4.086,1.114)--(4.096,1.113)--(4.105,1.113)--(4.115,1.112)--(4.124,1.112)--(4.134,1.111)%
  --(4.143,1.111)--(4.153,1.111)--(4.162,1.110)--(4.172,1.110)--(4.181,1.109)--(4.190,1.109)%
  --(4.200,1.108)--(4.209,1.108)--(4.219,1.107)--(4.228,1.107)--(4.238,1.106)--(4.247,1.106)%
  --(4.257,1.106)--(4.266,1.105)--(4.275,1.105)--(4.285,1.104)--(4.294,1.104)--(4.304,1.103)%
  --(4.313,1.103)--(4.323,1.102)--(4.332,1.102)--(4.342,1.101)--(4.351,1.101)--(4.360,1.101)%
  --(4.370,1.100)--(4.379,1.100)--(4.389,1.100)--(4.398,1.099)--(4.408,1.099)--(4.417,1.099)%
  --(4.427,1.098)--(4.436,1.098)--(4.446,1.098)--(4.455,1.097)--(4.464,1.097)--(4.474,1.097)%
  --(4.483,1.097)--(4.493,1.097)--(4.502,1.096)--(4.512,1.096)--(4.521,1.096)--(4.531,1.096)%
  --(4.540,1.096)--(4.549,1.096)--(4.559,1.096)--(4.568,1.096)--(4.578,1.096)--(4.587,1.096)%
  --(4.597,1.096)--(4.606,1.096)--(4.616,1.096)--(4.625,1.097)--(4.634,1.097)--(4.644,1.097)%
  --(4.653,1.097)--(4.663,1.098)--(4.672,1.098)--(4.682,1.098)--(4.691,1.099)--(4.701,1.099)%
  --(4.710,1.100)--(4.719,1.100)--(4.729,1.101)--(4.738,1.101)--(4.748,1.102)--(4.757,1.102)%
  --(4.767,1.103)--(4.776,1.104)--(4.786,1.104)--(4.795,1.105)--(4.805,1.106)--(4.814,1.107)%
  --(4.823,1.108)--(4.833,1.108)--(4.842,1.109)--(4.852,1.110)--(4.861,1.111)--(4.871,1.112)%
  --(4.880,1.113)--(4.890,1.115)--(4.899,1.116)--(4.908,1.117)--(4.918,1.118)--(4.927,1.119)%
  --(4.937,1.121)--(4.946,1.122)--(4.956,1.123)--(4.965,1.125)--(4.975,1.126)--(4.984,1.128)%
  --(4.993,1.129)--(5.003,1.131)--(5.012,1.132)--(5.022,1.134)--(5.031,1.135)--(5.041,1.137)%
  --(5.050,1.139)--(5.060,1.141)--(5.069,1.142)--(5.079,1.144)--(5.088,1.146)--(5.097,1.148)%
  --(5.107,1.149)--(5.116,1.151)--(5.126,1.153)--(5.135,1.155)--(5.145,1.157)--(5.154,1.159)%
  --(5.164,1.161)--(5.173,1.162)--(5.182,1.164)--(5.192,1.166)--(5.201,1.168)--(5.211,1.170)%
  --(5.220,1.172)--(5.230,1.173)--(5.239,1.175)--(5.249,1.177)--(5.258,1.179)--(5.267,1.180)%
  --(5.277,1.182)--(5.286,1.184)--(5.296,1.186)--(5.305,1.187)--(5.315,1.189)--(5.324,1.191)%
  --(5.334,1.192)--(5.343,1.194)--(5.352,1.195)--(5.362,1.197)--(5.371,1.198)--(5.381,1.199)%
  --(5.390,1.201)--(5.400,1.202)--(5.409,1.203)--(5.419,1.205)--(5.428,1.206)--(5.438,1.207)%
  --(5.447,1.208)--(5.456,1.209)--(5.466,1.210)--(5.475,1.211)--(5.485,1.212)--(5.494,1.214)%
  --(5.504,1.215)--(5.513,1.216)--(5.523,1.217)--(5.532,1.218)--(5.541,1.219)--(5.551,1.220)%
  --(5.560,1.221)--(5.570,1.222)--(5.579,1.223)--(5.589,1.224)--(5.598,1.225)--(5.608,1.227)%
  --(5.617,1.228)--(5.626,1.229)--(5.636,1.230)--(5.645,1.232)--(5.655,1.233)--(5.664,1.234)%
  --(5.674,1.236)--(5.683,1.237)--(5.693,1.239)--(5.702,1.240)--(5.711,1.242)--(5.721,1.244)%
  --(5.730,1.245)--(5.740,1.247)--(5.749,1.249)--(5.759,1.251)--(5.768,1.253)--(5.778,1.255)%
  --(5.787,1.258)--(5.797,1.260)--(5.806,1.262)--(5.815,1.265)--(5.825,1.267)--(5.834,1.270)%
  --(5.844,1.272)--(5.853,1.275)--(5.863,1.278)--(5.872,1.281)--(5.882,1.284)--(5.891,1.286)%
  --(5.900,1.289)--(5.910,1.292)--(5.919,1.295)--(5.929,1.298)--(5.938,1.301)--(5.948,1.304)%
  --(5.957,1.307)--(5.967,1.310)--(5.976,1.313)--(5.985,1.315)--(5.995,1.318)--(6.004,1.321)%
  --(6.014,1.324)--(6.023,1.326)--(6.033,1.329)--(6.042,1.331)--(6.052,1.334)--(6.061,1.336)%
  --(6.071,1.339)--(6.080,1.341)--(6.089,1.343)--(6.099,1.345)--(6.108,1.347)--(6.118,1.349)%
  --(6.127,1.350)--(6.137,1.352)--(6.146,1.353)--(6.156,1.354)--(6.165,1.356)--(6.174,1.357)%
  --(6.184,1.357)--(6.193,1.358)--(6.203,1.359)--(6.212,1.359)--(6.222,1.359)--(6.231,1.359)%
  --(6.241,1.359)--(6.250,1.359)--(6.259,1.358)--(6.269,1.358)--(6.278,1.357)--(6.288,1.356)%
  --(6.297,1.356)--(6.307,1.355)--(6.316,1.354)--(6.326,1.353)--(6.335,1.352)--(6.344,1.351)%
  --(6.354,1.349)--(6.363,1.348)--(6.373,1.347)--(6.382,1.346)--(6.392,1.345)--(6.401,1.343)%
  --(6.411,1.342)--(6.420,1.341)--(6.430,1.340)--(6.439,1.339)--(6.448,1.338)--(6.458,1.337)%
  --(6.467,1.336)--(6.477,1.336)--(6.486,1.335)--(6.496,1.334)--(6.505,1.334)--(6.515,1.334)%
  --(6.524,1.334)--(6.533,1.333)--(6.543,1.334)--(6.552,1.334)--(6.562,1.334)--(6.571,1.335)%
  --(6.581,1.336)--(6.590,1.337)--(6.600,1.338)--(6.609,1.339)--(6.618,1.341)--(6.628,1.343)%
  --(6.637,1.345)--(6.647,1.347)--(6.656,1.350)--(6.666,1.353)--(6.675,1.356)--(6.685,1.359)%
  --(6.694,1.362)--(6.704,1.366)--(6.713,1.369)--(6.722,1.373)--(6.732,1.377)--(6.741,1.381)%
  --(6.751,1.385)--(6.760,1.390)--(6.770,1.395)--(6.779,1.399)--(6.789,1.404)--(6.798,1.409)%
  --(6.807,1.414)--(6.817,1.419)--(6.826,1.425)--(6.836,1.430)--(6.845,1.436)--(6.855,1.441)%
  --(6.864,1.447)--(6.874,1.453)--(6.883,1.459)--(6.892,1.465)--(6.902,1.471)--(6.911,1.477)%
  --(6.921,1.483)--(6.930,1.489)--(6.940,1.495)--(6.949,1.501)--(6.959,1.508)--(6.968,1.514)%
  --(6.977,1.520)--(6.987,1.527)--(6.996,1.533)--(7.006,1.539)--(7.015,1.546)--(7.025,1.552)%
  --(7.034,1.559)--(7.044,1.565)--(7.053,1.571)--(7.063,1.577)--(7.072,1.584)--(7.081,1.590)%
  --(7.091,1.596)--(7.100,1.602)--(7.110,1.608)--(7.119,1.614)--(7.129,1.620)--(7.138,1.626)%
  --(7.148,1.632)--(7.157,1.637)--(7.166,1.643)--(7.176,1.649)--(7.185,1.654)--(7.195,1.659)%
  --(7.204,1.664)--(7.214,1.670)--(7.223,1.675)--(7.233,1.679)--(7.242,1.684)--(7.251,1.689)%
  --(7.261,1.693)--(7.270,1.697)--(7.280,1.702)--(7.289,1.706)--(7.299,1.709)--(7.308,1.713)%
  --(7.318,1.717)--(7.327,1.720)--(7.337,1.723)--(7.346,1.726)--(7.355,1.729)--(7.365,1.731)%
  --(7.374,1.734)--(7.384,1.736)--(7.393,1.738)--(7.403,1.739)--(7.412,1.741)--(7.422,1.742)%
  --(7.431,1.743)--(7.440,1.744)--(7.450,1.744)--(7.459,1.744)--(7.469,1.745)--(7.478,1.745)%
  --(7.488,1.744)--(7.497,1.744)--(7.507,1.743)--(7.516,1.743)--(7.525,1.742)--(7.535,1.742)%
  --(7.544,1.741)--(7.554,1.740)--(7.563,1.739)--(7.573,1.738)--(7.582,1.738)--(7.592,1.737)%
  --(7.601,1.736)--(7.610,1.736)--(7.620,1.735)--(7.629,1.735)--(7.639,1.735)--(7.648,1.734)%
  --(7.658,1.735)--(7.667,1.735)--(7.677,1.735)--(7.686,1.736)--(7.696,1.737)--(7.705,1.738)%
  --(7.714,1.740)--(7.724,1.742)--(7.733,1.744)--(7.743,1.746)--(7.752,1.749)--(7.762,1.752)%
  --(7.771,1.756)--(7.781,1.760)--(7.790,1.764)--(7.799,1.769)--(7.809,1.775)--(7.818,1.781)%
  --(7.828,1.787)--(7.837,1.794)--(7.847,1.801)--(7.856,1.809)--(7.866,1.818)--(7.875,1.827)%
  --(7.884,1.836)--(7.894,1.846)--(7.903,1.856)--(7.913,1.867)--(7.922,1.878)--(7.932,1.889)%
  --(7.941,1.901)--(7.951,1.912)--(7.960,1.925)--(7.970,1.937)--(7.979,1.949)--(7.988,1.962)%
  --(7.998,1.975)--(8.007,1.988)--(8.017,2.001)--(8.026,2.014)--(8.036,2.027)--(8.045,2.040)%
  --(8.055,2.053)--(8.064,2.066)--(8.073,2.078)--(8.083,2.091)--(8.092,2.104)--(8.102,2.116)%
  --(8.111,2.128)--(8.121,2.140)--(8.130,2.152)--(8.140,2.163)--(8.149,2.174)--(8.158,2.185)%
  --(8.168,2.195)--(8.177,2.205)--(8.187,2.215)--(8.196,2.224)--(8.206,2.233)--(8.215,2.241)%
  --(8.225,2.248)--(8.234,2.255)--(8.243,2.262)--(8.253,2.268)--(8.262,2.273)--(8.272,2.277)%
  --(8.281,2.281)--(8.291,2.285)--(8.300,2.288)--(8.310,2.290)--(8.319,2.292)--(8.329,2.294)%
  --(8.338,2.295)--(8.347,2.296)--(8.357,2.297)--(8.366,2.297)--(8.376,2.297)--(8.385,2.297)%
  --(8.395,2.296)--(8.404,2.296)--(8.414,2.295)--(8.423,2.295)--(8.432,2.294)--(8.442,2.293)%
  --(8.451,2.292)--(8.461,2.292)--(8.470,2.291)--(8.480,2.291)--(8.489,2.291)--(8.499,2.290)%
  --(8.508,2.291)--(8.517,2.291)--(8.527,2.292)--(8.536,2.293)--(8.546,2.294)--(8.555,2.296)%
  --(8.565,2.298)--(8.574,2.301)--(8.584,2.304)--(8.593,2.308)--(8.603,2.312)--(8.612,2.317)%
  --(8.621,2.323)--(8.631,2.329)--(8.640,2.336)--(8.650,2.344)--(8.659,2.352)--(8.669,2.361)%
  --(8.678,2.372)--(8.688,2.382)--(8.697,2.394)--(8.706,2.406)--(8.716,2.419)--(8.725,2.433)%
  --(8.735,2.448)--(8.744,2.463)--(8.754,2.478)--(8.763,2.494)--(8.773,2.511)--(8.782,2.529)%
  --(8.791,2.546)--(8.801,2.565)--(8.810,2.584)--(8.820,2.603)--(8.829,2.622)--(8.839,2.643)%
  --(8.848,2.663)--(8.858,2.684)--(8.867,2.705)--(8.876,2.726)--(8.886,2.748)--(8.895,2.770)%
  --(8.905,2.792)--(8.914,2.814)--(8.924,2.836)--(8.933,2.859)--(8.943,2.882)--(8.952,2.904)%
  --(8.962,2.927)--(8.971,2.950)--(8.980,2.973)--(8.990,2.996)--(8.999,3.019)--(9.009,3.041)%
  --(9.018,3.064)--(9.028,3.087)--(9.037,3.109)--(9.047,3.132)--(9.056,3.154)--(9.065,3.176)%
  --(9.075,3.197)--(9.084,3.219)--(9.094,3.240)--(9.103,3.261)--(9.113,3.282)--(9.122,3.302)%
  --(9.132,3.322)--(9.141,3.342)--(9.150,3.362)--(9.160,3.381)--(9.169,3.401)--(9.179,3.419)%
  --(9.188,3.438)--(9.198,3.457)--(9.207,3.475)--(9.217,3.493)--(9.226,3.511)--(9.236,3.528)%
  --(9.245,3.545)--(9.254,3.562)--(9.264,3.579)--(9.273,3.596)--(9.283,3.612)--(9.292,3.628)%
  --(9.302,3.644)--(9.311,3.660)--(9.321,3.675)--(9.330,3.690)--(9.339,3.705)--(9.349,3.720)%
  --(9.358,3.734)--(9.368,3.748)--(9.377,3.762)--(9.387,3.776)--(9.396,3.790)--(9.406,3.803)%
  --(9.415,3.816)--(9.424,3.829)--(9.434,3.842)--(9.443,3.854)--(9.453,3.867)--(9.462,3.879)%
  --(9.472,3.890)--(9.481,3.902)--(9.491,3.913)--(9.500,3.925)--(9.509,3.936)--(9.519,3.946)%
  --(9.528,3.957)--(9.538,3.968)--(9.547,3.978)--(9.557,3.988)--(9.566,3.999)--(9.576,4.009)%
  --(9.585,4.019)--(9.595,4.029)--(9.604,4.039)--(9.613,4.049)--(9.623,4.059)--(9.632,4.070)%
  --(9.642,4.080)--(9.651,4.090)--(9.661,4.100)--(9.670,4.111)--(9.680,4.122)--(9.689,4.132)%
  --(9.698,4.143)--(9.708,4.155)--(9.717,4.166)--(9.727,4.177)--(9.736,4.189)--(9.746,4.201)%
  --(9.755,4.214)--(9.765,4.226)--(9.774,4.239)--(9.783,4.252)--(9.793,4.266)--(9.802,4.280)%
  --(9.812,4.294)--(9.821,4.309)--(9.831,4.324)--(9.840,4.339)--(9.850,4.355)--(9.859,4.372)%
  --(9.868,4.388)--(9.878,4.406)--(9.887,4.424)--(9.897,4.442)--(9.906,4.461)--(9.916,4.481)%
  --(9.925,4.501)--(9.935,4.521)--(9.944,4.542)--(9.954,4.563)--(9.963,4.584)--(9.972,4.606)%
  --(9.982,4.628)--(9.991,4.650)--(10.001,4.673)--(10.010,4.695)--(10.020,4.718)--(10.029,4.741)%
  --(10.039,4.764)--(10.048,4.787)--(10.057,4.810)--(10.067,4.833)--(10.076,4.855)--(10.086,4.878)%
  --(10.095,4.900)--(10.105,4.923)--(10.114,4.945)--(10.124,4.967)--(10.133,4.988)--(10.142,5.009)%
  --(10.152,5.030)--(10.161,5.051)--(10.171,5.071)--(10.180,5.090)--(10.190,5.109)--(10.199,5.128)%
  --(10.209,5.146)--(10.218,5.163)--(10.228,5.180)--(10.237,5.196)--(10.246,5.211)--(10.256,5.226)%
  --(10.265,5.240)--(10.275,5.253)--(10.284,5.265)--(10.294,5.276)--(10.303,5.287)--(10.313,5.296)%
  --(10.322,5.305)--(10.331,5.313)--(10.341,5.319)--(10.350,5.325)--(10.360,5.331)--(10.369,5.335)%
  --(10.379,5.339)--(10.388,5.343)--(10.398,5.345)--(10.407,5.348)--(10.416,5.349)--(10.426,5.351)%
  --(10.435,5.351)--(10.445,5.352)--(10.454,5.352)--(10.464,5.352)--(10.473,5.352)--(10.483,5.351)%
  --(10.492,5.351)--(10.501,5.350)--(10.511,5.350)--(10.520,5.349)--(10.530,5.348)--(10.539,5.348)%
  --(10.549,5.347)--(10.558,5.347)--(10.568,5.347)--(10.577,5.347)--(10.587,5.347)--(10.596,5.348)%
  --(10.605,5.350)--(10.615,5.351)--(10.624,5.353)--(10.634,5.356)--(10.643,5.359)--(10.653,5.363)%
  --(10.662,5.368)--(10.672,5.373)--(10.681,5.379)--(10.690,5.385)--(10.700,5.393)--(10.709,5.401)%
  --(10.719,5.411)--(10.728,5.421)--(10.738,5.432)--(10.747,5.444)--(10.757,5.457)--(10.766,5.471)%
  --(10.775,5.486)--(10.785,5.501)--(10.794,5.517)--(10.804,5.534)--(10.813,5.551)--(10.823,5.570)%
  --(10.832,5.588)--(10.842,5.608)--(10.851,5.628)--(10.861,5.648)--(10.870,5.670)--(10.879,5.691)%
  --(10.889,5.713)--(10.898,5.736)--(10.908,5.759)--(10.917,5.782)--(10.927,5.806)--(10.936,5.830)%
  --(10.946,5.855)--(10.955,5.879)--(10.964,5.904)--(10.974,5.930)--(10.983,5.955)--(10.993,5.981)%
  --(11.002,6.007)--(11.012,6.032)--(11.021,6.058)--(11.031,6.085)--(11.040,6.111)--(11.049,6.137)%
  --(11.059,6.163)--(11.068,6.189)--(11.078,6.215)--(11.087,6.241)--(11.097,6.267)--(11.106,6.293)%
  --(11.116,6.318)--(11.125,6.344)--(11.134,6.369)--(11.144,6.394)--(11.153,6.418)--(11.163,6.443)%
  --(11.172,6.466)--(11.182,6.490)--(11.191,6.512)--(11.201,6.534)--(11.210,6.556)--(11.220,6.576)%
  --(11.229,6.596)--(11.238,6.615)--(11.248,6.634)--(11.257,6.651)--(11.267,6.667)--(11.276,6.682)%
  --(11.286,6.696)--(11.295,6.709)--(11.305,6.721)--(11.314,6.731)--(11.323,6.740)--(11.333,6.748)%
  --(11.342,6.754)--(11.352,6.759)--(11.361,6.762)--(11.371,6.763)--(11.380,6.763)--(11.390,6.761)%
  --(11.399,6.757)--(11.408,6.752)--(11.418,6.744)--(11.427,6.735)--(11.437,6.724)--(11.446,6.710)%
  --(11.456,6.694)--(11.465,6.677)--(11.475,6.657)--(11.484,6.634)--(11.494,6.610)--(11.503,6.582)%
  --(11.512,6.553)--(11.522,6.521)--(11.531,6.486)--(11.541,6.449)--(11.550,6.409)--(11.560,6.366)%
  --(11.569,6.321)--(11.579,6.274)--(11.588,6.224)--(11.597,6.172)--(11.607,6.117)--(11.616,6.061)%
  --(11.626,6.002)--(11.635,5.941)--(11.645,5.878)--(11.654,5.813)--(11.664,5.747)--(11.673,5.678)%
  --(11.682,5.608)--(11.692,5.535)--(11.701,5.462)--(11.711,5.386)--(11.720,5.309)--(11.730,5.231)%
  --(11.739,5.151)--(11.749,5.070)--(11.758,4.988)--(11.767,4.904)--(11.777,4.819)--(11.786,4.733)%
  --(11.796,4.646)--(11.805,4.558)--(11.815,4.469)--(11.824,4.380)--(11.834,4.289)--(11.843,4.198)%
  --(11.853,4.106)--(11.862,4.013)--(11.871,3.920)--(11.881,3.827)--(11.890,3.733)--(11.900,3.638)%
  --(11.909,3.544)--(11.919,3.449)--(11.928,3.354)--(11.938,3.258)--(11.947,3.163);
\gpcolor{color=gp lt color border}
\node[gp node right] at (4.448,7.123) {w/o types};
\gpfill{color=gp lt color 1,opacity=0.10} (4.632,7.046)--(5.548,7.046)--(5.548,7.200)--(4.632,7.200)--cycle;
\gpcolor{color=gp lt color 1}
\draw[gp path] (4.632,7.046)--(5.548,7.046)--(5.548,7.200)--(4.632,7.200)--cycle;
\gpfill{color=gp lt color 1,opacity=0.10} (1.688,0.992)--(1.688,0.992)--(1.698,0.992)--(1.709,0.992)%
    --(1.719,0.992)--(1.729,0.992)--(1.739,0.992)--(1.750,0.992)--(1.760,0.992)%
    --(1.770,0.992)--(1.780,0.992)--(1.791,0.992)--(1.801,0.992)--(1.811,0.992)%
    --(1.822,0.992)--(1.832,0.992)--(1.842,0.992)--(1.852,0.992)--(1.863,0.992)%
    --(1.873,0.992)--(1.883,0.992)--(1.893,0.992)--(1.904,0.992)--(1.914,0.992)%
    --(1.924,0.992)--(1.934,0.992)--(1.945,0.992)--(1.955,0.992)--(1.965,0.992)%
    --(1.976,0.993)--(1.986,0.993)--(1.996,0.993)--(2.006,0.993)--(2.017,0.993)%
    --(2.027,0.994)--(2.037,0.994)--(2.047,0.994)--(2.058,0.995)--(2.068,0.995)%
    --(2.078,0.995)--(2.089,0.996)--(2.099,0.996)--(2.109,0.997)--(2.119,0.997)%
    --(2.130,0.998)--(2.140,0.998)--(2.150,0.999)--(2.160,0.999)--(2.171,1.000)%
    --(2.181,1.000)--(2.191,1.001)--(2.201,1.001)--(2.212,1.002)--(2.222,1.003)%
    --(2.232,1.003)--(2.243,1.004)--(2.253,1.004)--(2.263,1.005)--(2.273,1.006)%
    --(2.284,1.006)--(2.294,1.007)--(2.304,1.007)--(2.314,1.008)--(2.325,1.009)%
    --(2.335,1.009)--(2.345,1.010)--(2.356,1.010)--(2.366,1.011)--(2.376,1.011)%
    --(2.386,1.012)--(2.397,1.012)--(2.407,1.012)--(2.417,1.013)--(2.427,1.013)%
    --(2.438,1.013)--(2.448,1.014)--(2.458,1.014)--(2.468,1.014)--(2.479,1.014)%
    --(2.489,1.014)--(2.499,1.015)--(2.510,1.015)--(2.520,1.015)--(2.530,1.015)%
    --(2.540,1.014)--(2.551,1.014)--(2.561,1.014)--(2.571,1.014)--(2.581,1.014)%
    --(2.592,1.013)--(2.602,1.013)--(2.612,1.013)--(2.623,1.012)--(2.633,1.012)%
    --(2.643,1.012)--(2.653,1.011)--(2.664,1.011)--(2.674,1.011)--(2.684,1.010)%
    --(2.694,1.010)--(2.705,1.010)--(2.715,1.009)--(2.725,1.009)--(2.735,1.008)%
    --(2.746,1.008)--(2.756,1.008)--(2.766,1.008)--(2.777,1.007)--(2.787,1.007)%
    --(2.797,1.007)--(2.807,1.007)--(2.818,1.006)--(2.828,1.006)--(2.838,1.006)%
    --(2.848,1.006)--(2.859,1.006)--(2.869,1.006)--(2.879,1.006)--(2.890,1.006)%
    --(2.900,1.007)--(2.910,1.007)--(2.920,1.007)--(2.931,1.008)--(2.941,1.008)%
    --(2.951,1.009)--(2.961,1.009)--(2.972,1.010)--(2.982,1.010)--(2.992,1.011)%
    --(3.002,1.012)--(3.013,1.013)--(3.023,1.014)--(3.033,1.015)--(3.044,1.016)%
    --(3.054,1.017)--(3.064,1.018)--(3.074,1.020)--(3.085,1.021)--(3.095,1.022)%
    --(3.105,1.024)--(3.115,1.025)--(3.126,1.027)--(3.136,1.028)--(3.146,1.030)%
    --(3.157,1.032)--(3.167,1.034)--(3.177,1.035)--(3.187,1.037)--(3.198,1.039)%
    --(3.208,1.041)--(3.218,1.044)--(3.228,1.046)--(3.239,1.048)--(3.249,1.050)%
    --(3.259,1.053)--(3.269,1.055)--(3.280,1.057)--(3.290,1.060)--(3.300,1.062)%
    --(3.311,1.065)--(3.321,1.068)--(3.331,1.070)--(3.341,1.073)--(3.352,1.076)%
    --(3.362,1.079)--(3.372,1.082)--(3.382,1.085)--(3.393,1.088)--(3.403,1.091)%
    --(3.413,1.094)--(3.424,1.098)--(3.434,1.101)--(3.444,1.104)--(3.454,1.107)%
    --(3.465,1.111)--(3.475,1.114)--(3.485,1.117)--(3.495,1.121)--(3.506,1.124)%
    --(3.516,1.128)--(3.526,1.131)--(3.536,1.135)--(3.547,1.138)--(3.557,1.142)%
    --(3.567,1.146)--(3.578,1.149)--(3.588,1.153)--(3.598,1.157)--(3.608,1.160)%
    --(3.619,1.164)--(3.629,1.167)--(3.639,1.171)--(3.649,1.175)--(3.660,1.178)%
    --(3.670,1.182)--(3.680,1.186)--(3.691,1.189)--(3.701,1.193)--(3.711,1.197)%
    --(3.721,1.200)--(3.732,1.204)--(3.742,1.208)--(3.752,1.211)--(3.762,1.215)%
    --(3.773,1.218)--(3.783,1.222)--(3.793,1.225)--(3.803,1.229)--(3.814,1.232)%
    --(3.824,1.236)--(3.834,1.240)--(3.845,1.243)--(3.855,1.247)--(3.865,1.250)%
    --(3.875,1.254)--(3.886,1.257)--(3.896,1.261)--(3.906,1.264)--(3.916,1.268)%
    --(3.927,1.271)--(3.937,1.275)--(3.947,1.278)--(3.958,1.282)--(3.968,1.285)%
    --(3.978,1.289)--(3.988,1.292)--(3.999,1.296)--(4.009,1.300)--(4.019,1.303)%
    --(4.029,1.307)--(4.040,1.310)--(4.050,1.314)--(4.060,1.318)--(4.070,1.321)%
    --(4.081,1.325)--(4.091,1.329)--(4.101,1.333)--(4.112,1.337)--(4.122,1.340)%
    --(4.132,1.344)--(4.142,1.348)--(4.153,1.352)--(4.163,1.356)--(4.173,1.360)%
    --(4.183,1.364)--(4.194,1.368)--(4.204,1.372)--(4.214,1.376)--(4.225,1.381)%
    --(4.235,1.385)--(4.245,1.389)--(4.255,1.393)--(4.266,1.398)--(4.276,1.402)%
    --(4.286,1.407)--(4.296,1.411)--(4.307,1.416)--(4.317,1.420)--(4.327,1.425)%
    --(4.337,1.430)--(4.348,1.435)--(4.358,1.439)--(4.368,1.444)--(4.379,1.449)%
    --(4.389,1.454)--(4.399,1.460)--(4.409,1.465)--(4.420,1.470)--(4.430,1.475)%
    --(4.440,1.481)--(4.450,1.486)--(4.461,1.492)--(4.471,1.498)--(4.481,1.503)%
    --(4.492,1.509)--(4.502,1.515)--(4.512,1.521)--(4.522,1.527)--(4.533,1.533)%
    --(4.543,1.540)--(4.553,1.546)--(4.563,1.553)--(4.574,1.559)--(4.584,1.566)%
    --(4.594,1.573)--(4.604,1.579)--(4.615,1.586)--(4.625,1.593)--(4.635,1.600)%
    --(4.646,1.607)--(4.656,1.614)--(4.666,1.621)--(4.676,1.629)--(4.687,1.636)%
    --(4.697,1.643)--(4.707,1.650)--(4.717,1.657)--(4.728,1.664)--(4.738,1.672)%
    --(4.748,1.679)--(4.759,1.686)--(4.769,1.693)--(4.779,1.700)--(4.789,1.707)%
    --(4.800,1.714)--(4.810,1.721)--(4.820,1.728)--(4.830,1.735)--(4.841,1.741)%
    --(4.851,1.748)--(4.861,1.755)--(4.871,1.761)--(4.882,1.767)--(4.892,1.774)%
    --(4.902,1.780)--(4.913,1.786)--(4.923,1.791)--(4.933,1.797)--(4.943,1.803)%
    --(4.954,1.808)--(4.964,1.814)--(4.974,1.819)--(4.984,1.824)--(4.995,1.828)%
    --(5.005,1.833)--(5.015,1.838)--(5.026,1.842)--(5.036,1.846)--(5.046,1.850)%
    --(5.056,1.854)--(5.067,1.858)--(5.077,1.862)--(5.087,1.866)--(5.097,1.870)%
    --(5.108,1.873)--(5.118,1.877)--(5.128,1.880)--(5.138,1.883)--(5.149,1.887)%
    --(5.159,1.890)--(5.169,1.893)--(5.180,1.897)--(5.190,1.900)--(5.200,1.903)%
    --(5.210,1.906)--(5.221,1.909)--(5.231,1.912)--(5.241,1.916)--(5.251,1.919)%
    --(5.262,1.922)--(5.272,1.926)--(5.282,1.929)--(5.293,1.932)--(5.303,1.936)%
    --(5.313,1.939)--(5.323,1.943)--(5.334,1.947)--(5.344,1.950)--(5.354,1.954)%
    --(5.364,1.958)--(5.375,1.962)--(5.385,1.966)--(5.395,1.971)--(5.405,1.975)%
    --(5.416,1.980)--(5.426,1.984)--(5.436,1.989)--(5.447,1.994)--(5.457,1.999)%
    --(5.467,2.004)--(5.477,2.009)--(5.488,2.014)--(5.498,2.020)--(5.508,2.025)%
    --(5.518,2.030)--(5.529,2.036)--(5.539,2.041)--(5.549,2.047)--(5.560,2.052)%
    --(5.570,2.058)--(5.580,2.064)--(5.590,2.069)--(5.601,2.075)--(5.611,2.081)%
    --(5.621,2.086)--(5.631,2.092)--(5.642,2.098)--(5.652,2.104)--(5.662,2.109)%
    --(5.672,2.115)--(5.683,2.121)--(5.693,2.127)--(5.703,2.132)--(5.714,2.138)%
    --(5.724,2.144)--(5.734,2.149)--(5.744,2.155)--(5.755,2.160)--(5.765,2.166)%
    --(5.775,2.171)--(5.785,2.176)--(5.796,2.182)--(5.806,2.187)--(5.816,2.192)%
    --(5.827,2.197)--(5.837,2.202)--(5.847,2.207)--(5.857,2.211)--(5.868,2.216)%
    --(5.878,2.221)--(5.888,2.225)--(5.898,2.230)--(5.909,2.234)--(5.919,2.238)%
    --(5.929,2.242)--(5.939,2.246)--(5.950,2.250)--(5.960,2.254)--(5.970,2.257)%
    --(5.981,2.261)--(5.991,2.264)--(6.001,2.267)--(6.011,2.270)--(6.022,2.273)%
    --(6.032,2.276)--(6.042,2.278)--(6.052,2.281)--(6.063,2.283)--(6.073,2.285)%
    --(6.083,2.287)--(6.094,2.289)--(6.104,2.291)--(6.114,2.292)--(6.124,2.294)%
    --(6.135,2.295)--(6.145,2.296)--(6.155,2.297)--(6.165,2.297)--(6.176,2.298)%
    --(6.186,2.298)--(6.196,2.298)--(6.206,2.298)--(6.217,2.297)--(6.227,2.297)%
    --(6.237,2.296)--(6.248,2.295)--(6.258,2.294)--(6.268,2.293)--(6.278,2.292)%
    --(6.289,2.291)--(6.299,2.290)--(6.309,2.289)--(6.319,2.288)--(6.330,2.287)%
    --(6.340,2.286)--(6.350,2.286)--(6.361,2.285)--(6.371,2.285)--(6.381,2.284)%
    --(6.391,2.284)--(6.402,2.284)--(6.412,2.285)--(6.422,2.285)--(6.432,2.286)%
    --(6.443,2.288)--(6.453,2.289)--(6.463,2.291)--(6.473,2.294)--(6.484,2.296)%
    --(6.494,2.300)--(6.504,2.303)--(6.515,2.307)--(6.525,2.312)--(6.535,2.317)%
    --(6.545,2.323)--(6.556,2.329)--(6.566,2.336)--(6.576,2.344)--(6.586,2.352)%
    --(6.597,2.361)--(6.607,2.370)--(6.617,2.380)--(6.628,2.391)--(6.638,2.403)%
    --(6.648,2.415)--(6.658,2.428)--(6.669,2.441)--(6.679,2.455)--(6.689,2.469)%
    --(6.699,2.484)--(6.710,2.499)--(6.720,2.514)--(6.730,2.530)--(6.740,2.546)%
    --(6.751,2.562)--(6.761,2.578)--(6.771,2.594)--(6.782,2.611)--(6.792,2.627)%
    --(6.802,2.643)--(6.812,2.659)--(6.823,2.675)--(6.833,2.691)--(6.843,2.707)%
    --(6.853,2.723)--(6.864,2.738)--(6.874,2.753)--(6.884,2.767)--(6.895,2.781)%
    --(6.905,2.795)--(6.915,2.808)--(6.925,2.821)--(6.936,2.833)--(6.946,2.844)%
    --(6.956,2.855)--(6.966,2.865)--(6.977,2.874)--(6.987,2.882)--(6.997,2.890)%
    --(7.007,2.896)--(7.018,2.902)--(7.028,2.907)--(7.038,2.910)--(7.049,2.913)%
    --(7.059,2.915)--(7.069,2.916)--(7.079,2.916)--(7.090,2.916)--(7.100,2.915)%
    --(7.110,2.913)--(7.120,2.911)--(7.131,2.908)--(7.141,2.904)--(7.151,2.901)%
    --(7.162,2.896)--(7.172,2.892)--(7.182,2.887)--(7.192,2.882)--(7.203,2.876)%
    --(7.213,2.871)--(7.223,2.865)--(7.233,2.859)--(7.244,2.853)--(7.254,2.848)%
    --(7.264,2.842)--(7.274,2.836)--(7.285,2.831)--(7.295,2.826)--(7.305,2.821)%
    --(7.316,2.816)--(7.326,2.812)--(7.336,2.808)--(7.346,2.805)--(7.357,2.802)%
    --(7.367,2.799)--(7.377,2.798)--(7.387,2.797)--(7.398,2.796)--(7.408,2.797)%
    --(7.418,2.798)--(7.429,2.800)--(7.439,2.802)--(7.449,2.806)--(7.459,2.811)%
    --(7.470,2.816)--(7.480,2.822)--(7.490,2.829)--(7.500,2.836)--(7.511,2.844)%
    --(7.521,2.853)--(7.531,2.862)--(7.541,2.871)--(7.552,2.881)--(7.562,2.892)%
    --(7.572,2.902)--(7.583,2.913)--(7.593,2.924)--(7.603,2.936)--(7.613,2.947)%
    --(7.624,2.959)--(7.634,2.971)--(7.644,2.983)--(7.654,2.994)--(7.665,3.006)%
    --(7.675,3.017)--(7.685,3.029)--(7.696,3.040)--(7.706,3.051)--(7.716,3.061)%
    --(7.726,3.071)--(7.737,3.081)--(7.747,3.090)--(7.757,3.099)--(7.767,3.108)%
    --(7.778,3.115)--(7.788,3.123)--(7.798,3.129)--(7.808,3.135)--(7.819,3.140)%
    --(7.829,3.144)--(7.839,3.147)--(7.850,3.150)--(7.860,3.151)--(7.870,3.152)%
    --(7.880,3.152)--(7.891,3.151)--(7.901,3.149)--(7.911,3.147)--(7.921,3.144)%
    --(7.932,3.140)--(7.942,3.136)--(7.952,3.132)--(7.963,3.127)--(7.973,3.121)%
    --(7.983,3.115)--(7.993,3.109)--(8.004,3.102)--(8.014,3.095)--(8.024,3.088)%
    --(8.034,3.080)--(8.045,3.072)--(8.055,3.065)--(8.065,3.057)--(8.075,3.049)%
    --(8.086,3.041)--(8.096,3.033)--(8.106,3.025)--(8.117,3.018)--(8.127,3.010)%
    --(8.137,3.003)--(8.147,2.996)--(8.158,2.989)--(8.168,2.982)--(8.178,2.976)%
    --(8.188,2.970)--(8.199,2.965)--(8.209,2.960)--(8.219,2.955)--(8.230,2.952)%
    --(8.240,2.948)--(8.250,2.946)--(8.260,2.944)--(8.271,2.942)--(8.281,2.942)%
    --(8.291,2.942)--(8.301,2.943)--(8.312,2.944)--(8.322,2.946)--(8.332,2.948)%
    --(8.342,2.951)--(8.353,2.955)--(8.363,2.959)--(8.373,2.964)--(8.384,2.969)%
    --(8.394,2.975)--(8.404,2.981)--(8.414,2.988)--(8.425,2.995)--(8.435,3.003)%
    --(8.445,3.011)--(8.455,3.019)--(8.466,3.028)--(8.476,3.038)--(8.486,3.047)%
    --(8.497,3.057)--(8.507,3.068)--(8.517,3.079)--(8.527,3.090)--(8.538,3.101)%
    --(8.548,3.113)--(8.558,3.125)--(8.568,3.137)--(8.579,3.150)--(8.589,3.163)%
    --(8.599,3.176)--(8.609,3.189)--(8.620,3.203)--(8.630,3.216)--(8.640,3.230)%
    --(8.651,3.244)--(8.661,3.258)--(8.671,3.273)--(8.681,3.287)--(8.692,3.302)%
    --(8.702,3.316)--(8.712,3.331)--(8.722,3.345)--(8.733,3.360)--(8.743,3.374)%
    --(8.753,3.389)--(8.764,3.403)--(8.774,3.417)--(8.784,3.431)--(8.794,3.444)%
    --(8.805,3.458)--(8.815,3.471)--(8.825,3.484)--(8.835,3.496)--(8.846,3.508)%
    --(8.856,3.520)--(8.866,3.531)--(8.876,3.542)--(8.887,3.552)--(8.897,3.562)%
    --(8.907,3.571)--(8.918,3.579)--(8.928,3.587)--(8.938,3.594)--(8.948,3.601)%
    --(8.959,3.607)--(8.969,3.612)--(8.979,3.616)--(8.989,3.620)--(9.000,3.623)%
    --(9.010,3.624)--(9.020,3.625)--(9.031,3.625)--(9.041,3.624)--(9.051,3.623)%
    --(9.061,3.620)--(9.072,3.616)--(9.082,3.611)--(9.092,3.604)--(9.102,3.597)%
    --(9.113,3.589)--(9.123,3.581)--(9.133,3.571)--(9.143,3.561)--(9.154,3.550)%
    --(9.164,3.539)--(9.174,3.527)--(9.185,3.515)--(9.195,3.503)--(9.205,3.490)%
    --(9.215,3.477)--(9.226,3.465)--(9.236,3.452)--(9.246,3.439)--(9.256,3.427)%
    --(9.267,3.414)--(9.277,3.402)--(9.287,3.391)--(9.298,3.380)--(9.308,3.369)%
    --(9.318,3.359)--(9.328,3.350)--(9.339,3.341)--(9.349,3.333)--(9.359,3.326)%
    --(9.369,3.321)--(9.380,3.316)--(9.390,3.312)--(9.400,3.309)--(9.410,3.308)%
    --(9.421,3.308)--(9.431,3.310)--(9.441,3.313)--(9.452,3.318)--(9.462,3.324)%
    --(9.472,3.332)--(9.482,3.342)--(9.493,3.353)--(9.503,3.367)--(9.513,3.382)%
    --(9.523,3.399)--(9.534,3.417)--(9.544,3.437)--(9.554,3.458)--(9.565,3.480)%
    --(9.575,3.504)--(9.585,3.528)--(9.595,3.554)--(9.606,3.580)--(9.616,3.607)%
    --(9.626,3.635)--(9.636,3.663)--(9.647,3.692)--(9.657,3.721)--(9.667,3.751)%
    --(9.677,3.780)--(9.688,3.810)--(9.698,3.840)--(9.708,3.869)--(9.719,3.898)%
    --(9.729,3.927)--(9.739,3.956)--(9.749,3.984)--(9.760,4.012)--(9.770,4.038)%
    --(9.780,4.064)--(9.790,4.090)--(9.801,4.114)--(9.811,4.137)--(9.821,4.159)%
    --(9.832,4.179)--(9.842,4.198)--(9.852,4.216)--(9.862,4.232)--(9.873,4.247)%
    --(9.883,4.260)--(9.893,4.271)--(9.903,4.280)--(9.914,4.287)--(9.924,4.292)%
    --(9.934,4.295)--(9.944,4.297)--(9.955,4.297)--(9.965,4.295)--(9.975,4.292)%
    --(9.986,4.287)--(9.996,4.281)--(10.006,4.274)--(10.016,4.265)--(10.027,4.255)%
    --(10.037,4.244)--(10.047,4.232)--(10.057,4.219)--(10.068,4.205)--(10.078,4.190)%
    --(10.088,4.174)--(10.099,4.157)--(10.109,4.139)--(10.119,4.121)--(10.129,4.103)%
    --(10.140,4.083)--(10.150,4.064)--(10.160,4.043)--(10.170,4.023)--(10.181,4.002)%
    --(10.191,3.981)--(10.201,3.960)--(10.211,3.939)--(10.222,3.918)--(10.232,3.896)%
    --(10.242,3.875)--(10.253,3.854)--(10.263,3.834)--(10.273,3.813)--(10.283,3.793)%
    --(10.294,3.773)--(10.304,3.754)--(10.314,3.735)--(10.324,3.717)--(10.335,3.700)%
    --(10.345,3.683)--(10.355,3.666)--(10.366,3.650)--(10.376,3.634)--(10.386,3.619)%
    --(10.396,3.604)--(10.407,3.590)--(10.417,3.576)--(10.427,3.563)--(10.437,3.549)%
    --(10.448,3.537)--(10.458,3.524)--(10.468,3.512)--(10.478,3.501)--(10.489,3.489)%
    --(10.499,3.478)--(10.509,3.467)--(10.520,3.457)--(10.530,3.447)--(10.540,3.437)%
    --(10.550,3.427)--(10.561,3.418)--(10.571,3.408)--(10.581,3.399)--(10.591,3.390)%
    --(10.602,3.382)--(10.612,3.373)--(10.622,3.365)--(10.633,3.356)--(10.643,3.348)%
    --(10.653,3.340)--(10.663,3.332)--(10.674,3.324)--(10.684,3.317)--(10.694,3.309)%
    --(10.704,3.301)--(10.715,3.293)--(10.725,3.286)--(10.735,3.278)--(10.745,3.271)%
    --(10.756,3.263)--(10.766,3.255)--(10.776,3.248)--(10.787,3.241)--(10.797,3.233)%
    --(10.807,3.226)--(10.817,3.219)--(10.828,3.212)--(10.838,3.206)--(10.848,3.199)%
    --(10.858,3.193)--(10.869,3.187)--(10.879,3.181)--(10.889,3.175)--(10.900,3.169)%
    --(10.910,3.164)--(10.920,3.159)--(10.930,3.154)--(10.941,3.149)--(10.951,3.145)%
    --(10.961,3.141)--(10.971,3.138)--(10.982,3.134)--(10.992,3.131)--(11.002,3.129)%
    --(11.012,3.127)--(11.023,3.125)--(11.033,3.123)--(11.043,3.122)--(11.054,3.121)%
    --(11.064,3.121)--(11.074,3.121)--(11.084,3.122)--(11.095,3.123)--(11.105,3.125)%
    --(11.115,3.127)--(11.125,3.130)--(11.136,3.133)--(11.146,3.136)--(11.156,3.140)%
    --(11.167,3.145)--(11.177,3.150)--(11.187,3.155)--(11.197,3.161)--(11.208,3.167)%
    --(11.218,3.173)--(11.228,3.179)--(11.238,3.186)--(11.249,3.193)--(11.259,3.200)%
    --(11.269,3.207)--(11.279,3.214)--(11.290,3.222)--(11.300,3.229)--(11.310,3.236)%
    --(11.321,3.244)--(11.331,3.251)--(11.341,3.258)--(11.351,3.265)--(11.362,3.272)%
    --(11.372,3.279)--(11.382,3.286)--(11.392,3.292)--(11.403,3.298)--(11.413,3.304)%
    --(11.423,3.310)--(11.434,3.315)--(11.444,3.320)--(11.454,3.324)--(11.464,3.328)%
    --(11.475,3.331)--(11.485,3.334)--(11.495,3.337)--(11.505,3.339)--(11.516,3.340)%
    --(11.526,3.341)--(11.536,3.341)--(11.546,3.340)--(11.557,3.339)--(11.567,3.337)%
    --(11.577,3.334)--(11.588,3.331)--(11.598,3.327)--(11.608,3.323)--(11.618,3.318)%
    --(11.629,3.312)--(11.639,3.306)--(11.649,3.300)--(11.659,3.292)--(11.670,3.285)%
    --(11.680,3.277)--(11.690,3.269)--(11.701,3.260)--(11.711,3.250)--(11.721,3.241)%
    --(11.731,3.231)--(11.742,3.220)--(11.752,3.210)--(11.762,3.199)--(11.772,3.187)%
    --(11.783,3.175)--(11.793,3.164)--(11.803,3.151)--(11.813,3.139)--(11.824,3.126)%
    --(11.834,3.113)--(11.844,3.100)--(11.855,3.087)--(11.865,3.074)--(11.875,3.060)%
    --(11.885,3.047)--(11.896,3.033)--(11.906,3.019)--(11.916,3.005)--(11.926,2.991)%
    --(11.937,2.977)--(11.947,2.963)--(11.947,0.985)--(1.688,0.985)--cycle;
\draw[gp path] (1.688,0.992)--(1.698,0.992)--(1.709,0.992)--(1.719,0.992)--(1.729,0.992)%
  --(1.739,0.992)--(1.750,0.992)--(1.760,0.992)--(1.770,0.992)--(1.780,0.992)--(1.791,0.992)%
  --(1.801,0.992)--(1.811,0.992)--(1.822,0.992)--(1.832,0.992)--(1.842,0.992)--(1.852,0.992)%
  --(1.863,0.992)--(1.873,0.992)--(1.883,0.992)--(1.893,0.992)--(1.904,0.992)--(1.914,0.992)%
  --(1.924,0.992)--(1.934,0.992)--(1.945,0.992)--(1.955,0.992)--(1.965,0.992)--(1.976,0.993)%
  --(1.986,0.993)--(1.996,0.993)--(2.006,0.993)--(2.017,0.993)--(2.027,0.994)--(2.037,0.994)%
  --(2.047,0.994)--(2.058,0.995)--(2.068,0.995)--(2.078,0.995)--(2.089,0.996)--(2.099,0.996)%
  --(2.109,0.997)--(2.119,0.997)--(2.130,0.998)--(2.140,0.998)--(2.150,0.999)--(2.160,0.999)%
  --(2.171,1.000)--(2.181,1.000)--(2.191,1.001)--(2.201,1.001)--(2.212,1.002)--(2.222,1.003)%
  --(2.232,1.003)--(2.243,1.004)--(2.253,1.004)--(2.263,1.005)--(2.273,1.006)--(2.284,1.006)%
  --(2.294,1.007)--(2.304,1.007)--(2.314,1.008)--(2.325,1.009)--(2.335,1.009)--(2.345,1.010)%
  --(2.356,1.010)--(2.366,1.011)--(2.376,1.011)--(2.386,1.012)--(2.397,1.012)--(2.407,1.012)%
  --(2.417,1.013)--(2.427,1.013)--(2.438,1.013)--(2.448,1.014)--(2.458,1.014)--(2.468,1.014)%
  --(2.479,1.014)--(2.489,1.014)--(2.499,1.015)--(2.510,1.015)--(2.520,1.015)--(2.530,1.015)%
  --(2.540,1.014)--(2.551,1.014)--(2.561,1.014)--(2.571,1.014)--(2.581,1.014)--(2.592,1.013)%
  --(2.602,1.013)--(2.612,1.013)--(2.623,1.012)--(2.633,1.012)--(2.643,1.012)--(2.653,1.011)%
  --(2.664,1.011)--(2.674,1.011)--(2.684,1.010)--(2.694,1.010)--(2.705,1.010)--(2.715,1.009)%
  --(2.725,1.009)--(2.735,1.008)--(2.746,1.008)--(2.756,1.008)--(2.766,1.008)--(2.777,1.007)%
  --(2.787,1.007)--(2.797,1.007)--(2.807,1.007)--(2.818,1.006)--(2.828,1.006)--(2.838,1.006)%
  --(2.848,1.006)--(2.859,1.006)--(2.869,1.006)--(2.879,1.006)--(2.890,1.006)--(2.900,1.007)%
  --(2.910,1.007)--(2.920,1.007)--(2.931,1.008)--(2.941,1.008)--(2.951,1.009)--(2.961,1.009)%
  --(2.972,1.010)--(2.982,1.010)--(2.992,1.011)--(3.002,1.012)--(3.013,1.013)--(3.023,1.014)%
  --(3.033,1.015)--(3.044,1.016)--(3.054,1.017)--(3.064,1.018)--(3.074,1.020)--(3.085,1.021)%
  --(3.095,1.022)--(3.105,1.024)--(3.115,1.025)--(3.126,1.027)--(3.136,1.028)--(3.146,1.030)%
  --(3.157,1.032)--(3.167,1.034)--(3.177,1.035)--(3.187,1.037)--(3.198,1.039)--(3.208,1.041)%
  --(3.218,1.044)--(3.228,1.046)--(3.239,1.048)--(3.249,1.050)--(3.259,1.053)--(3.269,1.055)%
  --(3.280,1.057)--(3.290,1.060)--(3.300,1.062)--(3.311,1.065)--(3.321,1.068)--(3.331,1.070)%
  --(3.341,1.073)--(3.352,1.076)--(3.362,1.079)--(3.372,1.082)--(3.382,1.085)--(3.393,1.088)%
  --(3.403,1.091)--(3.413,1.094)--(3.424,1.098)--(3.434,1.101)--(3.444,1.104)--(3.454,1.107)%
  --(3.465,1.111)--(3.475,1.114)--(3.485,1.117)--(3.495,1.121)--(3.506,1.124)--(3.516,1.128)%
  --(3.526,1.131)--(3.536,1.135)--(3.547,1.138)--(3.557,1.142)--(3.567,1.146)--(3.578,1.149)%
  --(3.588,1.153)--(3.598,1.157)--(3.608,1.160)--(3.619,1.164)--(3.629,1.167)--(3.639,1.171)%
  --(3.649,1.175)--(3.660,1.178)--(3.670,1.182)--(3.680,1.186)--(3.691,1.189)--(3.701,1.193)%
  --(3.711,1.197)--(3.721,1.200)--(3.732,1.204)--(3.742,1.208)--(3.752,1.211)--(3.762,1.215)%
  --(3.773,1.218)--(3.783,1.222)--(3.793,1.225)--(3.803,1.229)--(3.814,1.232)--(3.824,1.236)%
  --(3.834,1.240)--(3.845,1.243)--(3.855,1.247)--(3.865,1.250)--(3.875,1.254)--(3.886,1.257)%
  --(3.896,1.261)--(3.906,1.264)--(3.916,1.268)--(3.927,1.271)--(3.937,1.275)--(3.947,1.278)%
  --(3.958,1.282)--(3.968,1.285)--(3.978,1.289)--(3.988,1.292)--(3.999,1.296)--(4.009,1.300)%
  --(4.019,1.303)--(4.029,1.307)--(4.040,1.310)--(4.050,1.314)--(4.060,1.318)--(4.070,1.321)%
  --(4.081,1.325)--(4.091,1.329)--(4.101,1.333)--(4.112,1.337)--(4.122,1.340)--(4.132,1.344)%
  --(4.142,1.348)--(4.153,1.352)--(4.163,1.356)--(4.173,1.360)--(4.183,1.364)--(4.194,1.368)%
  --(4.204,1.372)--(4.214,1.376)--(4.225,1.381)--(4.235,1.385)--(4.245,1.389)--(4.255,1.393)%
  --(4.266,1.398)--(4.276,1.402)--(4.286,1.407)--(4.296,1.411)--(4.307,1.416)--(4.317,1.420)%
  --(4.327,1.425)--(4.337,1.430)--(4.348,1.435)--(4.358,1.439)--(4.368,1.444)--(4.379,1.449)%
  --(4.389,1.454)--(4.399,1.460)--(4.409,1.465)--(4.420,1.470)--(4.430,1.475)--(4.440,1.481)%
  --(4.450,1.486)--(4.461,1.492)--(4.471,1.498)--(4.481,1.503)--(4.492,1.509)--(4.502,1.515)%
  --(4.512,1.521)--(4.522,1.527)--(4.533,1.533)--(4.543,1.540)--(4.553,1.546)--(4.563,1.553)%
  --(4.574,1.559)--(4.584,1.566)--(4.594,1.573)--(4.604,1.579)--(4.615,1.586)--(4.625,1.593)%
  --(4.635,1.600)--(4.646,1.607)--(4.656,1.614)--(4.666,1.621)--(4.676,1.629)--(4.687,1.636)%
  --(4.697,1.643)--(4.707,1.650)--(4.717,1.657)--(4.728,1.664)--(4.738,1.672)--(4.748,1.679)%
  --(4.759,1.686)--(4.769,1.693)--(4.779,1.700)--(4.789,1.707)--(4.800,1.714)--(4.810,1.721)%
  --(4.820,1.728)--(4.830,1.735)--(4.841,1.741)--(4.851,1.748)--(4.861,1.755)--(4.871,1.761)%
  --(4.882,1.767)--(4.892,1.774)--(4.902,1.780)--(4.913,1.786)--(4.923,1.791)--(4.933,1.797)%
  --(4.943,1.803)--(4.954,1.808)--(4.964,1.814)--(4.974,1.819)--(4.984,1.824)--(4.995,1.828)%
  --(5.005,1.833)--(5.015,1.838)--(5.026,1.842)--(5.036,1.846)--(5.046,1.850)--(5.056,1.854)%
  --(5.067,1.858)--(5.077,1.862)--(5.087,1.866)--(5.097,1.870)--(5.108,1.873)--(5.118,1.877)%
  --(5.128,1.880)--(5.138,1.883)--(5.149,1.887)--(5.159,1.890)--(5.169,1.893)--(5.180,1.897)%
  --(5.190,1.900)--(5.200,1.903)--(5.210,1.906)--(5.221,1.909)--(5.231,1.912)--(5.241,1.916)%
  --(5.251,1.919)--(5.262,1.922)--(5.272,1.926)--(5.282,1.929)--(5.293,1.932)--(5.303,1.936)%
  --(5.313,1.939)--(5.323,1.943)--(5.334,1.947)--(5.344,1.950)--(5.354,1.954)--(5.364,1.958)%
  --(5.375,1.962)--(5.385,1.966)--(5.395,1.971)--(5.405,1.975)--(5.416,1.980)--(5.426,1.984)%
  --(5.436,1.989)--(5.447,1.994)--(5.457,1.999)--(5.467,2.004)--(5.477,2.009)--(5.488,2.014)%
  --(5.498,2.020)--(5.508,2.025)--(5.518,2.030)--(5.529,2.036)--(5.539,2.041)--(5.549,2.047)%
  --(5.560,2.052)--(5.570,2.058)--(5.580,2.064)--(5.590,2.069)--(5.601,2.075)--(5.611,2.081)%
  --(5.621,2.086)--(5.631,2.092)--(5.642,2.098)--(5.652,2.104)--(5.662,2.109)--(5.672,2.115)%
  --(5.683,2.121)--(5.693,2.127)--(5.703,2.132)--(5.714,2.138)--(5.724,2.144)--(5.734,2.149)%
  --(5.744,2.155)--(5.755,2.160)--(5.765,2.166)--(5.775,2.171)--(5.785,2.176)--(5.796,2.182)%
  --(5.806,2.187)--(5.816,2.192)--(5.827,2.197)--(5.837,2.202)--(5.847,2.207)--(5.857,2.211)%
  --(5.868,2.216)--(5.878,2.221)--(5.888,2.225)--(5.898,2.230)--(5.909,2.234)--(5.919,2.238)%
  --(5.929,2.242)--(5.939,2.246)--(5.950,2.250)--(5.960,2.254)--(5.970,2.257)--(5.981,2.261)%
  --(5.991,2.264)--(6.001,2.267)--(6.011,2.270)--(6.022,2.273)--(6.032,2.276)--(6.042,2.278)%
  --(6.052,2.281)--(6.063,2.283)--(6.073,2.285)--(6.083,2.287)--(6.094,2.289)--(6.104,2.291)%
  --(6.114,2.292)--(6.124,2.294)--(6.135,2.295)--(6.145,2.296)--(6.155,2.297)--(6.165,2.297)%
  --(6.176,2.298)--(6.186,2.298)--(6.196,2.298)--(6.206,2.298)--(6.217,2.297)--(6.227,2.297)%
  --(6.237,2.296)--(6.248,2.295)--(6.258,2.294)--(6.268,2.293)--(6.278,2.292)--(6.289,2.291)%
  --(6.299,2.290)--(6.309,2.289)--(6.319,2.288)--(6.330,2.287)--(6.340,2.286)--(6.350,2.286)%
  --(6.361,2.285)--(6.371,2.285)--(6.381,2.284)--(6.391,2.284)--(6.402,2.284)--(6.412,2.285)%
  --(6.422,2.285)--(6.432,2.286)--(6.443,2.288)--(6.453,2.289)--(6.463,2.291)--(6.473,2.294)%
  --(6.484,2.296)--(6.494,2.300)--(6.504,2.303)--(6.515,2.307)--(6.525,2.312)--(6.535,2.317)%
  --(6.545,2.323)--(6.556,2.329)--(6.566,2.336)--(6.576,2.344)--(6.586,2.352)--(6.597,2.361)%
  --(6.607,2.370)--(6.617,2.380)--(6.628,2.391)--(6.638,2.403)--(6.648,2.415)--(6.658,2.428)%
  --(6.669,2.441)--(6.679,2.455)--(6.689,2.469)--(6.699,2.484)--(6.710,2.499)--(6.720,2.514)%
  --(6.730,2.530)--(6.740,2.546)--(6.751,2.562)--(6.761,2.578)--(6.771,2.594)--(6.782,2.611)%
  --(6.792,2.627)--(6.802,2.643)--(6.812,2.659)--(6.823,2.675)--(6.833,2.691)--(6.843,2.707)%
  --(6.853,2.723)--(6.864,2.738)--(6.874,2.753)--(6.884,2.767)--(6.895,2.781)--(6.905,2.795)%
  --(6.915,2.808)--(6.925,2.821)--(6.936,2.833)--(6.946,2.844)--(6.956,2.855)--(6.966,2.865)%
  --(6.977,2.874)--(6.987,2.882)--(6.997,2.890)--(7.007,2.896)--(7.018,2.902)--(7.028,2.907)%
  --(7.038,2.910)--(7.049,2.913)--(7.059,2.915)--(7.069,2.916)--(7.079,2.916)--(7.090,2.916)%
  --(7.100,2.915)--(7.110,2.913)--(7.120,2.911)--(7.131,2.908)--(7.141,2.904)--(7.151,2.901)%
  --(7.162,2.896)--(7.172,2.892)--(7.182,2.887)--(7.192,2.882)--(7.203,2.876)--(7.213,2.871)%
  --(7.223,2.865)--(7.233,2.859)--(7.244,2.853)--(7.254,2.848)--(7.264,2.842)--(7.274,2.836)%
  --(7.285,2.831)--(7.295,2.826)--(7.305,2.821)--(7.316,2.816)--(7.326,2.812)--(7.336,2.808)%
  --(7.346,2.805)--(7.357,2.802)--(7.367,2.799)--(7.377,2.798)--(7.387,2.797)--(7.398,2.796)%
  --(7.408,2.797)--(7.418,2.798)--(7.429,2.800)--(7.439,2.802)--(7.449,2.806)--(7.459,2.811)%
  --(7.470,2.816)--(7.480,2.822)--(7.490,2.829)--(7.500,2.836)--(7.511,2.844)--(7.521,2.853)%
  --(7.531,2.862)--(7.541,2.871)--(7.552,2.881)--(7.562,2.892)--(7.572,2.902)--(7.583,2.913)%
  --(7.593,2.924)--(7.603,2.936)--(7.613,2.947)--(7.624,2.959)--(7.634,2.971)--(7.644,2.983)%
  --(7.654,2.994)--(7.665,3.006)--(7.675,3.017)--(7.685,3.029)--(7.696,3.040)--(7.706,3.051)%
  --(7.716,3.061)--(7.726,3.071)--(7.737,3.081)--(7.747,3.090)--(7.757,3.099)--(7.767,3.108)%
  --(7.778,3.115)--(7.788,3.123)--(7.798,3.129)--(7.808,3.135)--(7.819,3.140)--(7.829,3.144)%
  --(7.839,3.147)--(7.850,3.150)--(7.860,3.151)--(7.870,3.152)--(7.880,3.152)--(7.891,3.151)%
  --(7.901,3.149)--(7.911,3.147)--(7.921,3.144)--(7.932,3.140)--(7.942,3.136)--(7.952,3.132)%
  --(7.963,3.127)--(7.973,3.121)--(7.983,3.115)--(7.993,3.109)--(8.004,3.102)--(8.014,3.095)%
  --(8.024,3.088)--(8.034,3.080)--(8.045,3.072)--(8.055,3.065)--(8.065,3.057)--(8.075,3.049)%
  --(8.086,3.041)--(8.096,3.033)--(8.106,3.025)--(8.117,3.018)--(8.127,3.010)--(8.137,3.003)%
  --(8.147,2.996)--(8.158,2.989)--(8.168,2.982)--(8.178,2.976)--(8.188,2.970)--(8.199,2.965)%
  --(8.209,2.960)--(8.219,2.955)--(8.230,2.952)--(8.240,2.948)--(8.250,2.946)--(8.260,2.944)%
  --(8.271,2.942)--(8.281,2.942)--(8.291,2.942)--(8.301,2.943)--(8.312,2.944)--(8.322,2.946)%
  --(8.332,2.948)--(8.342,2.951)--(8.353,2.955)--(8.363,2.959)--(8.373,2.964)--(8.384,2.969)%
  --(8.394,2.975)--(8.404,2.981)--(8.414,2.988)--(8.425,2.995)--(8.435,3.003)--(8.445,3.011)%
  --(8.455,3.019)--(8.466,3.028)--(8.476,3.038)--(8.486,3.047)--(8.497,3.057)--(8.507,3.068)%
  --(8.517,3.079)--(8.527,3.090)--(8.538,3.101)--(8.548,3.113)--(8.558,3.125)--(8.568,3.137)%
  --(8.579,3.150)--(8.589,3.163)--(8.599,3.176)--(8.609,3.189)--(8.620,3.203)--(8.630,3.216)%
  --(8.640,3.230)--(8.651,3.244)--(8.661,3.258)--(8.671,3.273)--(8.681,3.287)--(8.692,3.302)%
  --(8.702,3.316)--(8.712,3.331)--(8.722,3.345)--(8.733,3.360)--(8.743,3.374)--(8.753,3.389)%
  --(8.764,3.403)--(8.774,3.417)--(8.784,3.431)--(8.794,3.444)--(8.805,3.458)--(8.815,3.471)%
  --(8.825,3.484)--(8.835,3.496)--(8.846,3.508)--(8.856,3.520)--(8.866,3.531)--(8.876,3.542)%
  --(8.887,3.552)--(8.897,3.562)--(8.907,3.571)--(8.918,3.579)--(8.928,3.587)--(8.938,3.594)%
  --(8.948,3.601)--(8.959,3.607)--(8.969,3.612)--(8.979,3.616)--(8.989,3.620)--(9.000,3.623)%
  --(9.010,3.624)--(9.020,3.625)--(9.031,3.625)--(9.041,3.624)--(9.051,3.623)--(9.061,3.620)%
  --(9.072,3.616)--(9.082,3.611)--(9.092,3.604)--(9.102,3.597)--(9.113,3.589)--(9.123,3.581)%
  --(9.133,3.571)--(9.143,3.561)--(9.154,3.550)--(9.164,3.539)--(9.174,3.527)--(9.185,3.515)%
  --(9.195,3.503)--(9.205,3.490)--(9.215,3.477)--(9.226,3.465)--(9.236,3.452)--(9.246,3.439)%
  --(9.256,3.427)--(9.267,3.414)--(9.277,3.402)--(9.287,3.391)--(9.298,3.380)--(9.308,3.369)%
  --(9.318,3.359)--(9.328,3.350)--(9.339,3.341)--(9.349,3.333)--(9.359,3.326)--(9.369,3.321)%
  --(9.380,3.316)--(9.390,3.312)--(9.400,3.309)--(9.410,3.308)--(9.421,3.308)--(9.431,3.310)%
  --(9.441,3.313)--(9.452,3.318)--(9.462,3.324)--(9.472,3.332)--(9.482,3.342)--(9.493,3.353)%
  --(9.503,3.367)--(9.513,3.382)--(9.523,3.399)--(9.534,3.417)--(9.544,3.437)--(9.554,3.458)%
  --(9.565,3.480)--(9.575,3.504)--(9.585,3.528)--(9.595,3.554)--(9.606,3.580)--(9.616,3.607)%
  --(9.626,3.635)--(9.636,3.663)--(9.647,3.692)--(9.657,3.721)--(9.667,3.751)--(9.677,3.780)%
  --(9.688,3.810)--(9.698,3.840)--(9.708,3.869)--(9.719,3.898)--(9.729,3.927)--(9.739,3.956)%
  --(9.749,3.984)--(9.760,4.012)--(9.770,4.038)--(9.780,4.064)--(9.790,4.090)--(9.801,4.114)%
  --(9.811,4.137)--(9.821,4.159)--(9.832,4.179)--(9.842,4.198)--(9.852,4.216)--(9.862,4.232)%
  --(9.873,4.247)--(9.883,4.260)--(9.893,4.271)--(9.903,4.280)--(9.914,4.287)--(9.924,4.292)%
  --(9.934,4.295)--(9.944,4.297)--(9.955,4.297)--(9.965,4.295)--(9.975,4.292)--(9.986,4.287)%
  --(9.996,4.281)--(10.006,4.274)--(10.016,4.265)--(10.027,4.255)--(10.037,4.244)--(10.047,4.232)%
  --(10.057,4.219)--(10.068,4.205)--(10.078,4.190)--(10.088,4.174)--(10.099,4.157)--(10.109,4.139)%
  --(10.119,4.121)--(10.129,4.103)--(10.140,4.083)--(10.150,4.064)--(10.160,4.043)--(10.170,4.023)%
  --(10.181,4.002)--(10.191,3.981)--(10.201,3.960)--(10.211,3.939)--(10.222,3.918)--(10.232,3.896)%
  --(10.242,3.875)--(10.253,3.854)--(10.263,3.834)--(10.273,3.813)--(10.283,3.793)--(10.294,3.773)%
  --(10.304,3.754)--(10.314,3.735)--(10.324,3.717)--(10.335,3.700)--(10.345,3.683)--(10.355,3.666)%
  --(10.366,3.650)--(10.376,3.634)--(10.386,3.619)--(10.396,3.604)--(10.407,3.590)--(10.417,3.576)%
  --(10.427,3.563)--(10.437,3.549)--(10.448,3.537)--(10.458,3.524)--(10.468,3.512)--(10.478,3.501)%
  --(10.489,3.489)--(10.499,3.478)--(10.509,3.467)--(10.520,3.457)--(10.530,3.447)--(10.540,3.437)%
  --(10.550,3.427)--(10.561,3.418)--(10.571,3.408)--(10.581,3.399)--(10.591,3.390)--(10.602,3.382)%
  --(10.612,3.373)--(10.622,3.365)--(10.633,3.356)--(10.643,3.348)--(10.653,3.340)--(10.663,3.332)%
  --(10.674,3.324)--(10.684,3.317)--(10.694,3.309)--(10.704,3.301)--(10.715,3.293)--(10.725,3.286)%
  --(10.735,3.278)--(10.745,3.271)--(10.756,3.263)--(10.766,3.255)--(10.776,3.248)--(10.787,3.241)%
  --(10.797,3.233)--(10.807,3.226)--(10.817,3.219)--(10.828,3.212)--(10.838,3.206)--(10.848,3.199)%
  --(10.858,3.193)--(10.869,3.187)--(10.879,3.181)--(10.889,3.175)--(10.900,3.169)--(10.910,3.164)%
  --(10.920,3.159)--(10.930,3.154)--(10.941,3.149)--(10.951,3.145)--(10.961,3.141)--(10.971,3.138)%
  --(10.982,3.134)--(10.992,3.131)--(11.002,3.129)--(11.012,3.127)--(11.023,3.125)--(11.033,3.123)%
  --(11.043,3.122)--(11.054,3.121)--(11.064,3.121)--(11.074,3.121)--(11.084,3.122)--(11.095,3.123)%
  --(11.105,3.125)--(11.115,3.127)--(11.125,3.130)--(11.136,3.133)--(11.146,3.136)--(11.156,3.140)%
  --(11.167,3.145)--(11.177,3.150)--(11.187,3.155)--(11.197,3.161)--(11.208,3.167)--(11.218,3.173)%
  --(11.228,3.179)--(11.238,3.186)--(11.249,3.193)--(11.259,3.200)--(11.269,3.207)--(11.279,3.214)%
  --(11.290,3.222)--(11.300,3.229)--(11.310,3.236)--(11.321,3.244)--(11.331,3.251)--(11.341,3.258)%
  --(11.351,3.265)--(11.362,3.272)--(11.372,3.279)--(11.382,3.286)--(11.392,3.292)--(11.403,3.298)%
  --(11.413,3.304)--(11.423,3.310)--(11.434,3.315)--(11.444,3.320)--(11.454,3.324)--(11.464,3.328)%
  --(11.475,3.331)--(11.485,3.334)--(11.495,3.337)--(11.505,3.339)--(11.516,3.340)--(11.526,3.341)%
  --(11.536,3.341)--(11.546,3.340)--(11.557,3.339)--(11.567,3.337)--(11.577,3.334)--(11.588,3.331)%
  --(11.598,3.327)--(11.608,3.323)--(11.618,3.318)--(11.629,3.312)--(11.639,3.306)--(11.649,3.300)%
  --(11.659,3.292)--(11.670,3.285)--(11.680,3.277)--(11.690,3.269)--(11.701,3.260)--(11.711,3.250)%
  --(11.721,3.241)--(11.731,3.231)--(11.742,3.220)--(11.752,3.210)--(11.762,3.199)--(11.772,3.187)%
  --(11.783,3.175)--(11.793,3.164)--(11.803,3.151)--(11.813,3.139)--(11.824,3.126)--(11.834,3.113)%
  --(11.844,3.100)--(11.855,3.087)--(11.865,3.074)--(11.875,3.060)--(11.885,3.047)--(11.896,3.033)%
  --(11.906,3.019)--(11.916,3.005)--(11.926,2.991)--(11.937,2.977)--(11.947,2.963);
\gpcolor{color=gp lt color border}
\gpsetlinetype{gp lt border}
\gpsetlinewidth{1.00}
\draw[gp path] (1.688,8.381)--(1.688,0.985)--(11.947,0.985)--(11.947,8.381)--cycle;
%% coordinates of the plot area
\gpdefrectangularnode{gp plot 1}{\pgfpoint{1.688cm}{0.985cm}}{\pgfpoint{11.947cm}{8.381cm}}
\end{tikzpicture}
%% gnuplot variables

      \end{myplot}

      \begin{myplot}%
        {Распределение методов от значения меры точности для \eng{Eclipse~IDE}}%
        {plot:eclipse_all_aliases_distribution_cumulative}
        \begin{tikzpicture}[gnuplot]
%% generated with GNUPLOT 4.5p0 (Lua 5.1; terminal rev. 99, script rev. 98)
%% 27.05.2011 12:50:11
\path (0.000,0.000) rectangle (12.500,8.750);
\gpcolor{color=gp lt color border}
\gpsetlinetype{gp lt border}
\gpsetlinewidth{1.00}
\draw[gp path] (1.504,0.985)--(1.684,0.985);
\draw[gp path] (11.947,0.985)--(11.767,0.985);
\node[gp node right] at (1.320,0.985) {\num{0}};
\draw[gp path] (1.504,2.330)--(1.684,2.330);
\draw[gp path] (11.947,2.330)--(11.767,2.330);
\node[gp node right] at (1.320,2.330) {\num{0.2}};
\draw[gp path] (1.504,3.674)--(1.684,3.674);
\draw[gp path] (11.947,3.674)--(11.767,3.674);
\node[gp node right] at (1.320,3.674) {\num{0.4}};
\draw[gp path] (1.504,5.019)--(1.684,5.019);
\draw[gp path] (11.947,5.019)--(11.767,5.019);
\node[gp node right] at (1.320,5.019) {\num{0.6}};
\draw[gp path] (1.504,6.364)--(1.684,6.364);
\draw[gp path] (11.947,6.364)--(11.767,6.364);
\node[gp node right] at (1.320,6.364) {\num{0.8}};
\draw[gp path] (1.504,7.709)--(1.684,7.709);
\draw[gp path] (11.947,7.709)--(11.767,7.709);
\node[gp node right] at (1.320,7.709) {\num{1}};
\gpcolor{color=gp lt color axes}
\gpsetlinetype{gp lt axes}
\draw[gp path] (3.508,0.985)--(3.508,6.969);
\draw[gp path] (3.508,8.201)--(3.508,8.381);
\gpcolor{color=gp lt color border}
\gpsetlinetype{gp lt border}
\draw[gp path] (3.508,0.985)--(3.508,1.165);
\draw[gp path] (3.508,8.381)--(3.508,8.201);
\node[gp node center] at (3.508,0.677) {\num{0.2}};
\gpcolor{color=gp lt color axes}
\gpsetlinetype{gp lt axes}
\draw[gp path] (5.618,0.985)--(5.618,8.381);
\gpcolor{color=gp lt color border}
\gpsetlinetype{gp lt border}
\draw[gp path] (5.618,0.985)--(5.618,1.165);
\draw[gp path] (5.618,8.381)--(5.618,8.201);
\node[gp node center] at (5.618,0.677) {\num{0.4}};
\gpcolor{color=gp lt color axes}
\gpsetlinetype{gp lt axes}
\draw[gp path] (7.728,0.985)--(7.728,8.381);
\gpcolor{color=gp lt color border}
\gpsetlinetype{gp lt border}
\draw[gp path] (7.728,0.985)--(7.728,1.165);
\draw[gp path] (7.728,8.381)--(7.728,8.201);
\node[gp node center] at (7.728,0.677) {\num{0.6}};
\gpcolor{color=gp lt color axes}
\gpsetlinetype{gp lt axes}
\draw[gp path] (9.837,0.985)--(9.837,8.381);
\gpcolor{color=gp lt color border}
\gpsetlinetype{gp lt border}
\draw[gp path] (9.837,0.985)--(9.837,1.165);
\draw[gp path] (9.837,8.381)--(9.837,8.201);
\node[gp node center] at (9.837,0.677) {\num{0.8}};
\gpcolor{color=gp lt color axes}
\gpsetlinetype{gp lt axes}
\draw[gp path] (11.947,0.985)--(11.947,8.381);
\gpcolor{color=gp lt color border}
\gpsetlinetype{gp lt border}
\draw[gp path] (11.947,0.985)--(11.947,1.165);
\draw[gp path] (11.947,8.381)--(11.947,8.201);
\node[gp node center] at (11.947,0.677) {\num{1}};
\draw[gp path] (1.504,8.381)--(1.504,0.985)--(11.947,0.985)--(11.947,8.381)--cycle;
\node[gp node center,rotate=-270] at (0.246,4.683) {Количество методов, \%};
\node[gp node center] at (6.725,0.215) {Отношение среднего количества синонимов к числу переменных};
\node[gp node right] at (4.264,8.047) {base};
\gpcolor{color=gp lt color 0}
\gpsetlinetype{gp lt plot 0}
\gpsetlinewidth{2.00}
\draw[gp path] (4.448,8.047)--(5.364,8.047);
\draw[gp path] (1.504,0.986)--(1.820,0.986)--(1.926,0.987)--(2.031,0.988)--(2.137,0.990)%
  --(2.242,0.993)--(2.348,0.996)--(2.453,1.000)--(2.559,1.005)--(2.664,1.017)--(2.875,1.035)%
  --(2.981,1.041)--(3.086,1.057)--(3.192,1.066)--(3.297,1.097)--(3.403,1.105)--(3.508,1.130)%
  --(3.614,1.150)--(3.719,1.196)--(3.825,1.210)--(3.930,1.247)--(4.036,1.292)--(4.141,1.337)%
  --(4.247,1.374)--(4.352,1.463)--(4.458,1.501)--(4.563,1.572)--(4.669,1.645)--(4.774,1.721)%
  --(4.880,1.788)--(4.985,1.887)--(5.090,1.943)--(5.196,2.035)--(5.301,2.107)--(5.407,2.211)%
  --(5.512,2.356)--(5.618,2.480)--(5.723,2.555)--(5.829,2.662)--(5.934,2.767)--(6.040,2.898)%
  --(6.145,2.968)--(6.251,3.072)--(6.356,3.183)--(6.462,3.312)--(6.567,3.406)--(6.673,3.547)%
  --(6.778,3.646)--(6.884,3.791)--(6.989,3.898)--(7.095,4.039)--(7.200,4.150)--(7.306,4.303)%
  --(7.411,4.384)--(7.517,4.501)--(7.622,4.627)--(7.728,4.797)--(7.833,4.895)--(7.939,5.058)%
  --(8.044,5.172)--(8.150,5.290)--(8.255,5.378)--(8.361,5.499)--(8.466,5.623)--(8.571,5.772)%
  --(8.677,5.852)--(8.782,5.993)--(8.888,6.093)--(8.993,6.225)--(9.099,6.294)--(9.204,6.410)%
  --(9.310,6.476)--(9.415,6.583)--(9.521,6.648)--(9.626,6.758)--(9.732,6.808)--(9.837,6.926)%
  --(9.943,6.980)--(10.048,7.040)--(10.154,7.089)--(10.259,7.182)--(10.365,7.233)--(10.470,7.284)%
  --(10.576,7.306)--(10.681,7.413)--(10.787,7.443)--(10.892,7.490)--(10.998,7.520)--(11.103,7.583)%
  --(11.209,7.599)--(11.314,7.629)--(11.420,7.651)--(11.525,7.682)--(11.631,7.695)--(11.736,7.705)%
  --(11.842,7.707);
\gpcolor{color=gp lt color border}
\node[gp node right] at (4.264,7.739) {equality-based};
\gpcolor{color=gp lt color 2}
\draw[gp path] (4.448,7.739)--(5.364,7.739);
\draw[gp path] (1.504,0.986)--(1.820,0.986)--(1.926,0.987)--(2.031,0.988)--(2.137,0.990)%
  --(2.242,0.991)--(2.348,0.994)--(2.453,0.998)--(2.559,1.004)--(2.664,1.014)--(2.875,1.025)%
  --(2.981,1.029)--(3.086,1.038)--(3.192,1.042)--(3.297,1.072)--(3.403,1.080)--(3.508,1.094)%
  --(3.614,1.103)--(3.719,1.138)--(3.825,1.149)--(3.930,1.170)--(4.036,1.202)--(4.141,1.245)%
  --(4.247,1.261)--(4.352,1.321)--(4.458,1.344)--(4.563,1.403)--(4.669,1.452)--(4.774,1.516)%
  --(4.880,1.572)--(4.985,1.649)--(5.090,1.693)--(5.196,1.795)--(5.301,1.856)--(5.407,1.946)%
  --(5.512,2.021)--(5.618,2.119)--(5.723,2.187)--(5.829,2.291)--(5.934,2.400)--(6.040,2.555)%
  --(6.145,2.620)--(6.251,2.749)--(6.356,2.858)--(6.462,2.990)--(6.567,3.065)--(6.673,3.205)%
  --(6.778,3.294)--(6.884,3.426)--(6.989,3.540)--(7.095,3.671)--(7.200,3.775)--(7.306,3.943)%
  --(7.411,4.027)--(7.517,4.151)--(7.622,4.265)--(7.728,4.446)--(7.833,4.542)--(7.939,4.720)%
  --(8.044,4.822)--(8.150,4.963)--(8.255,5.057)--(8.361,5.200)--(8.466,5.330)--(8.571,5.474)%
  --(8.677,5.566)--(8.782,5.711)--(8.888,5.815)--(8.993,5.982)--(9.099,6.060)--(9.204,6.211)%
  --(9.310,6.301)--(9.415,6.437)--(9.521,6.494)--(9.626,6.611)--(9.732,6.673)--(9.837,6.815)%
  --(9.943,6.888)--(10.048,6.968)--(10.154,7.034)--(10.259,7.135)--(10.365,7.188)--(10.470,7.245)%
  --(10.576,7.274)--(10.681,7.387)--(10.787,7.418)--(10.892,7.472)--(10.998,7.505)--(11.103,7.574)%
  --(11.209,7.591)--(11.314,7.623)--(11.420,7.644)--(11.525,7.676)--(11.631,7.691)--(11.736,7.703)%
  --(11.842,7.705)--(11.947,7.706);
\gpcolor{color=gp lt color border}
\node[gp node right] at (4.264,7.431) {w/o data flow};
\gpcolor{color=gp lt color 6}
\draw[gp path] (4.448,7.431)--(5.364,7.431);
\draw[gp path] (2.137,0.986)--(2.348,0.986)--(2.559,0.990)--(2.664,0.990)--(2.875,0.993)%
  --(2.981,0.994)--(3.086,0.999)--(3.192,1.007)--(3.297,1.022)--(3.403,1.029)--(3.614,1.029)%
  --(3.719,1.037)--(3.825,1.037)--(3.930,1.047)--(4.036,1.050)--(4.141,1.061)--(4.247,1.063)%
  --(4.352,1.068)--(4.458,1.071)--(4.563,1.078)--(4.669,1.085)--(4.774,1.095)--(4.985,1.105)%
  --(5.090,1.107)--(5.196,1.114)--(5.301,1.124)--(5.407,1.144)--(5.512,1.157)--(5.618,1.169)%
  --(5.723,1.183)--(5.829,1.193)--(5.934,1.212)--(6.040,1.235)--(6.145,1.242)--(6.251,1.260)%
  --(6.356,1.292)--(6.462,1.305)--(6.567,1.312)--(6.673,1.328)--(6.778,1.355)--(6.884,1.382)%
  --(6.989,1.401)--(7.095,1.433)--(7.200,1.458)--(7.306,1.505)--(7.411,1.533)--(7.517,1.566)%
  --(7.622,1.611)--(7.728,1.652)--(7.833,1.683)--(7.939,1.735)--(8.044,1.784)--(8.150,1.842)%
  --(8.255,1.881)--(8.361,1.961)--(8.466,2.025)--(8.571,2.092)--(8.677,2.158)--(8.782,2.249)%
  --(8.888,2.297)--(8.993,2.463)--(9.099,2.541)--(9.204,2.650)--(9.310,2.771)--(9.415,2.924)%
  --(9.521,3.040)--(9.626,3.202)--(9.732,3.299)--(9.837,3.550)--(9.943,3.675)--(10.048,3.853)%
  --(10.154,3.993)--(10.259,4.241)--(10.365,4.419)--(10.470,4.629)--(10.576,4.770)--(10.681,5.108)%
  --(10.787,5.246)--(10.892,5.530)--(10.998,5.700)--(11.103,6.010)--(11.209,6.174)--(11.314,6.453)%
  --(11.420,6.642)--(11.525,6.948)--(11.631,7.162)--(11.736,7.471)--(11.842,7.631)--(11.947,7.701);
\gpcolor{color=gp lt color border}
\node[gp node right] at (4.264,7.123) {w/o types};
\gpcolor{color=gp lt color 1}
\draw[gp path] (4.448,7.123)--(5.364,7.123);
\draw[gp path] (1.504,0.986)--(1.820,0.988)--(1.926,0.989)--(2.137,0.991)--(2.242,0.992)%
  --(2.348,0.994)--(2.664,0.997)--(2.770,0.998)--(2.875,1.009)--(2.981,1.011)--(3.086,1.014)%
  --(3.297,1.029)--(3.403,1.036)--(3.508,1.048)--(3.614,1.056)--(3.719,1.062)--(3.825,1.070)%
  --(3.930,1.075)--(4.036,1.121)--(4.141,1.143)--(4.247,1.150)--(4.352,1.190)--(4.458,1.208)%
  --(4.563,1.238)--(4.669,1.273)--(4.774,1.300)--(4.880,1.367)--(4.985,1.419)--(5.090,1.433)%
  --(5.196,1.488)--(5.301,1.520)--(5.407,1.581)--(5.512,1.649)--(5.618,1.706)--(5.723,1.745)%
  --(5.829,1.803)--(5.934,1.833)--(6.040,1.955)--(6.145,1.985)--(6.251,2.052)--(6.356,2.107)%
  --(6.462,2.166)--(6.567,2.240)--(6.673,2.297)--(6.778,2.404)--(6.884,2.471)--(6.989,2.561)%
  --(7.095,2.655)--(7.200,2.720)--(7.306,2.870)--(7.411,2.913)--(7.517,2.999)--(7.622,3.077)%
  --(7.728,3.223)--(7.833,3.266)--(7.939,3.391)--(8.044,3.447)--(8.150,3.590)--(8.255,3.649)%
  --(8.361,3.750)--(8.466,3.869)--(8.571,3.958)--(8.677,4.027)--(8.782,4.163)--(8.888,4.219)%
  --(8.993,4.370)--(9.099,4.442)--(9.204,4.654)--(9.310,4.727)--(9.415,4.925)--(9.521,4.966)%
  --(9.626,5.097)--(9.732,5.238)--(9.837,5.361)--(9.943,5.547)--(10.048,5.663)--(10.154,5.818)%
  --(10.259,5.959)--(10.365,6.062)--(10.470,6.190)--(10.576,6.252)--(10.681,6.408)--(10.787,6.485)%
  --(10.892,6.609)--(10.998,6.683)--(11.103,6.804)--(11.209,6.878)--(11.314,6.983)--(11.420,7.060)%
  --(11.525,7.203)--(11.631,7.291)--(11.736,7.406)--(11.842,7.553)--(11.947,7.711);
\gpcolor{color=gp lt color border}
\gpsetlinetype{gp lt border}
\gpsetlinewidth{1.00}
\draw[gp path] (1.504,8.381)--(1.504,0.985)--(11.947,0.985)--(11.947,8.381)--cycle;
%% coordinates of the plot area
\gpdefrectangularnode{gp plot 1}{\pgfpoint{1.504cm}{0.985cm}}{\pgfpoint{11.947cm}{8.381cm}}
\end{tikzpicture}
%% gnuplot variables

      \end{myplot}

      Интересно наблюдать за разностью в точности на разных тестовых
      приложениях. Видно, что в наборе \eng{SPEC~JVM98} алгоритм анализа, не
      использующий информацию о потоках данных, дает существенно худшие
      результаты по сравнению с основным алгоритмом. Однако алгоритм анализа,
      не использующий информацию о типах, обладает еще меньшей точностью. В
      наборе \eng{SPECjvm2008} ситуация отличается: алгоритмы, не использующие
      информацию о типах и о потоках данных, имеют близкую точность. Наконец,
      в приложении \eng{Eclipse~IDE} ситуация обратна набору \eng{SPEC~JVM98}:
      алгоритм анализа, не использующий информацию о типах, обладает заметно
      лучшей точностью по сравнению с алгоритмом, не использующим информацию о
      потоках данных.
      Эти измерения демонстрируют, что использование информации о типах
      и о потоках данных является необходимым для получения удовлетворительной
      точности, хотя использование такой информации вносит разный вклад в
      итоговую точность алгоритма на разных приложениях.

      Также было проведено сравнение основного алгоритма \eng{subset-based}
      типа и его модификации, являющейся алгоритмом \eng{equality-based}. На
      всех тестовых приложениях точность второго алгоритма оказалсь хуже, но не
      так сильно, как в случае алгоритмов, не использующих информацию о потоках
      данных или типах. Эти результаты подтверждают, что выбор в пользу
      алгоритма \eng{subset-based} типа был сделан правильно, и он
      действительно обладает большей точностью. Однако результаты не
      согласуется с предположением о том, что точность алгоритмов
      \eng{subset-based} типа значительно выше точности алгоритмов
      \eng{equality-based} типа. Это может быть объяснено тем, что при
      тестировании использовалась лишь имитация алгоритма \eng{equality-based}
      типа с помощью \eng{subset-based} алгоритма, что дает некоторый выйигрыш
      в точности из-за особенностей реализации.

      В целом можно видеть, что использование определенных свойств языка при
      анализе дает существенный прирост точности конечного алгоритма. Причем
      вклад от использования отдельных свойств может меняться в зависимости от
      анализируемого приложения, поэтому невозможно выделить какое-либо
      <<главное>> свойство языка, которое дает наибольший прирост точности.
      Важно использование всех доступных свойств анализируемого языка, которые,
      дополняя друг друга, дают наилучший результат.


  \sectionwithoutnumber{Заключение}

    Целью данной работы являлась разработка внутрипроцедурного алгоритма
    анализа указателей, адаптированного для анализа многопточных \java программ.
    Данная цель была успешно достигнута и в ходе работы было сделано следующее:
    \begin{itemize}
      \item Проведен анализ существующих алгоритмов анализа указателей и
            синонимов, выделены их основные отличительные характеристики.
      \item Реализован тестовый стенд для изучения свойств разных алгоритмов
            анализа.
      \item Изучена спецификация языка \java и выявлены свойства, влияющие на
            анализ указателей.
      \item Разработан внутрипроцедурный алгоритм анализа указателей,
            учитывающий особенности языка \java и адаптированный для анализа
            многопоточных программ в соответствии с моделью памяти языка \java.
      \item Разработано внутреннее представление, которое позволяет эффективно
            проводить анализ указателей и хранить его результаты.
      \item Реализована компонента, интегрируемая в статический \java компилятор
            \eng{Excelsior~RVM}, выполняющая анализ синонимов.
      \item Проведено измерение времени работы и потребления памяти
            разработанного алгоритма анализа. Также проведено сравнение
            точности результатов с другими алгоритмами.
    \end{itemize}

    В дальнейшем планируется реализация данного алгоритма анализа указателей и
    синонимов в промышленном статическом \java компиляторе \eng{Excelsior~JET}.
    Также планируется адаптация представленного алгоритма для межпроцедурного
    анализа.

  \newpage
  \bibliography{../../biblio}

\end{document}

