\begin{tikzpicture}[gnuplot]
%% generated with GNUPLOT 4.5p0 (Lua 5.1; terminal rev. 99, script rev. 98)
%% 27.05.2011 10:51:01
\path (0.000,0.000) rectangle (12.500,8.750);
\gpcolor{color=gp lt color border}
\gpsetlinetype{gp lt border}
\gpsetlinewidth{1.00}
\draw[gp path] (1.504,0.985)--(1.684,0.985);
\draw[gp path] (11.947,0.985)--(11.767,0.985);
\node[gp node right] at (1.320,0.985) {\num{0}};
\draw[gp path] (1.504,1.725)--(1.684,1.725);
\draw[gp path] (11.947,1.725)--(11.767,1.725);
\node[gp node right] at (1.320,1.725) {\num{0.1}};
\draw[gp path] (1.504,2.464)--(1.684,2.464);
\draw[gp path] (11.947,2.464)--(11.767,2.464);
\node[gp node right] at (1.320,2.464) {\num{0.2}};
\draw[gp path] (1.504,3.204)--(1.684,3.204);
\draw[gp path] (11.947,3.204)--(11.767,3.204);
\node[gp node right] at (1.320,3.204) {\num{0.3}};
\draw[gp path] (1.504,3.943)--(1.684,3.943);
\draw[gp path] (11.947,3.943)--(11.767,3.943);
\node[gp node right] at (1.320,3.943) {\num{0.4}};
\draw[gp path] (1.504,4.683)--(1.684,4.683);
\draw[gp path] (11.947,4.683)--(11.767,4.683);
\node[gp node right] at (1.320,4.683) {\num{0.5}};
\draw[gp path] (1.504,5.423)--(1.684,5.423);
\draw[gp path] (11.947,5.423)--(11.767,5.423);
\node[gp node right] at (1.320,5.423) {\num{0.6}};
\draw[gp path] (1.504,6.162)--(1.684,6.162);
\draw[gp path] (11.947,6.162)--(11.767,6.162);
\node[gp node right] at (1.320,6.162) {\num{0.7}};
\draw[gp path] (1.504,6.902)--(1.684,6.902);
\draw[gp path] (11.947,6.902)--(11.767,6.902);
\node[gp node right] at (1.320,6.902) {\num{0.8}};
\draw[gp path] (1.504,7.641)--(1.684,7.641);
\draw[gp path] (11.947,7.641)--(11.767,7.641);
\node[gp node right] at (1.320,7.641) {\num{0.9}};
\draw[gp path] (1.504,8.381)--(1.684,8.381);
\draw[gp path] (11.947,8.381)--(11.767,8.381);
\node[gp node right] at (1.320,8.381) {\num{1}};
\draw[gp path] (2.453,0.985)--(2.453,1.165);
\draw[gp path] (2.453,8.381)--(2.453,8.201);
\node[gp node center] at (2.453,0.677) {\num{10}};
\draw[gp path] (3.508,0.985)--(3.508,1.165);
\draw[gp path] (3.508,8.381)--(3.508,8.201);
\node[gp node center] at (3.508,0.677) {\num{20}};
\draw[gp path] (4.563,0.985)--(4.563,1.165);
\draw[gp path] (4.563,8.381)--(4.563,8.201);
\node[gp node center] at (4.563,0.677) {\num{30}};
\draw[gp path] (5.618,0.985)--(5.618,1.165);
\draw[gp path] (5.618,8.381)--(5.618,8.201);
\node[gp node center] at (5.618,0.677) {\num{40}};
\draw[gp path] (6.673,0.985)--(6.673,1.165);
\draw[gp path] (6.673,8.381)--(6.673,8.201);
\node[gp node center] at (6.673,0.677) {\num{50}};
\draw[gp path] (7.728,0.985)--(7.728,1.165);
\draw[gp path] (7.728,8.381)--(7.728,8.201);
\node[gp node center] at (7.728,0.677) {\num{60}};
\draw[gp path] (8.782,0.985)--(8.782,1.165);
\draw[gp path] (8.782,8.381)--(8.782,8.201);
\node[gp node center] at (8.782,0.677) {\num{70}};
\draw[gp path] (9.837,0.985)--(9.837,1.165);
\draw[gp path] (9.837,8.381)--(9.837,8.201);
\node[gp node center] at (9.837,0.677) {\num{80}};
\draw[gp path] (10.892,0.985)--(10.892,1.165);
\draw[gp path] (10.892,8.381)--(10.892,8.201);
\node[gp node center] at (10.892,0.677) {\num{90}};
\draw[gp path] (11.947,0.985)--(11.947,1.165);
\draw[gp path] (11.947,8.381)--(11.947,8.201);
\node[gp node center] at (11.947,0.677) {\num{100}};
\draw[gp path] (1.504,8.381)--(1.504,0.985)--(11.947,0.985)--(11.947,8.381)--cycle;
\node[gp node center,rotate=-270] at (0.246,4.683) {Количество методов, \%};
\node[gp node center] at (6.725,0.215) {Число присваиваний};
\gpsetlinetype{gp lt plot 0}
\draw[gp path] (1.504,5.076)--(2.453,5.076)--(2.453,2.097)--(3.508,2.097)--(3.508,1.476)%
  --(4.563,1.476)--(4.563,1.407)--(5.618,1.407)--(5.618,1.273)--(6.673,1.273)--(6.673,1.143)%
  --(7.728,1.143)--(7.728,1.125)--(8.782,1.125)--(8.782,1.097)--(9.837,1.097)--(9.837,1.073)%
  --(10.892,1.073)--(10.892,1.045)--(11.947,1.045)--(11.947,1.027);
\gpsetlinetype{gp lt border}
\draw[gp path] (1.504,8.381)--(1.504,0.985)--(11.947,0.985)--(11.947,8.381)--cycle;
%% coordinates of the plot area
\gpdefrectangularnode{gp plot 1}{\pgfpoint{1.504cm}{0.985cm}}{\pgfpoint{11.947cm}{8.381cm}}
\end{tikzpicture}
%% gnuplot variables
